% !TeX root = ../../book.tex
\section{逻辑}

\subsection{陈述和命题}

\begin{itemize}
    \item \verb|真| 和 \verb|假| 是仅有的两个真值。
    \item \textbf{数学陈述}(或\textbf{逻辑陈述})是一个语法正确的句子,且\textbf{只有}一个确定的真值。
    \item \textbf{变量命题}是一个语法正确的句子,它包含一个或多个变量;只要为这些变量赋值,该句子就会获得一个真值。
    \item 在定义陈述或命题时,我们为其分配一个字母名称,指明对变量的依赖关系(并为变量分配字母),然后将实际的陈述或命题放在引号内。以下是两个例子:
          \begin{quote}
              定义 $P$ 为``每个实数 $x$ 都满足 $x^2 \ge 0$''。\\
              定义 $Q(x, y)$ 为 ``对于任意 $x,y \in \mathbb{R}, xy \le (\frac{x+y}{2})^2$''。
          \end{quote}
    \item \textbf{排中律}是指我们假设每个陈述要么为\verb|真|,要么为\verb|假|。具体来说,对于任意陈述 $P$,我们可以保证 $P$ 要么为\verb|真|,要么为\verb|假|,并且\textbf{只有}其中一种情况成立。
\end{itemize}

\subsection{量词}

\begin{itemize}
    \item 当遇到``对于任意''或``对于所有''时,我们使用\textbf{全称量词} $\forall$。\\
          表达式 $\forall x \in S \centerdot P(x)$ 表示``对于任意元素 $x \in S$,性质 $P(x)$ 成立。''
    \item 当遇到``存在''或``至少有一个''时,我们使用\textbf{存在量词} $\exists$。\\
          表达式 $\exists x \in S \centerdot P(x)$ 表示``存在一个元素 $x \in S$,使得性质 $P(x)$ 成立。''
    \item 我们使用符号 ``$\centerdot$'' 来分隔量词陈述中的各个部分。
    \item 在朗读量词陈述时,\textbf{只}在 $\exists$ 量词之后使用``使得''一词。
    \item 我们使用 ``$\exists !$'' 表示\textbf{唯一存在};即 $\exists ! x \in S \centerdot P(x)$ 表示``唯一存在一个元素 $x \in S$,使得性质 $P(x)$ 成立''。
\end{itemize}

\subsection{连词}

假设 $P$ 和 $Q$ 为数学陈述,它们可能由带有量词的变量命题构成。

\begin{itemize}
    \item ``$P$ 与 $Q$'' 记作
        \[P \land Q\]
        当且仅当 $P$ 和 $Q$ 同时为\verb|真| 时,$P \land Q$ 才为\verb|真|。
    \item ``$P$ 或 $Q$'' 记作
        \[P \lor Q\]
        当且仅当 $P$ 和 $Q$ 中\textbf{至少一个}为\verb|真| 时,$P \lor Q$ 才为\verb|真|。(这里的``或''是\textbf{兼或},因此允许 $P$ 和 $Q$ 同时为\verb|真|。)
    \item ``如果 $P$ 则 $Q$'' 记作
        \[P \implies Q\]
        该陈述为\verb|真|当且仅当 $P$ 为\verb|假|或 $Q$ 为\verb|真|。换言之,当 $P$ 为\verb|真|时,$Q$ 必须为\verb|真|;但当 $P$ 为\verb|假|时,$Q$ 的真值可任意。\\
        注意 $P \implies Q$ 本身是一个逻辑陈述,具有真值(\verb|真| 或 \verb|假|),但它不对构成陈述的 $P$ 和 $Q$ 的真值作任何声明。\\
        我们称 $P \implies Q$ 为\textbf{条件陈述},其中 $P$ 称为\textbf{假设},$Q$ 称为\textbf{结论}。\\
        需要特别注意的是,当 $P$ 为\verb|假| 时,$P \implies Q$ 恒为\verb|真|。这是因为条件陈述仅在假设成立时对结论作出要求;当假设不成立时,不能声明条件陈述为\verb|假|,因此根据排中律,它必为\verb|真|。
    \item $P \implies Q$ 等价于
        \[\neg P \lor Q\]
    \item 条件陈述 $P \implies Q$ 的\textbf{逆否命题}为
        \[\neg Q \implies \neg P\]
        它始终与 $P \implies Q$ 具有相同的真值,即:
        \[(P \implies Q) \iff (\neg Q \implies \neg P)\]
    \item 条件陈述 $P \implies Q$ 的\textbf{逆命题}为
        \[Q \implies P\]
        它\textbf{不一定}与 $P \implies Q$ 具有相同的真值。存在某些陈述 $P$ 和 $Q$ 使得原命题和逆命题同时成立,也存在某些情形下原命题成立而逆命题不成立。
    \item ``$P$ 与 $Q$ \textbf{逻辑等价}''记作
        \[P \iff Q\]
        读作``$P$ 当且仅当 $Q$''。\\
        也可写做
        \[(P \implies Q) \land (Q \implies P)\]
        这意味着在任何情况下,$P$ 和 $Q$ 都\textbf{具有相同的真值}。
\end{itemize}

\subsection{逻辑否定}

\begin{itemize}
    \item 我们使用符号``$\neg$''表示一个陈述的\textbf{逻辑否定}。
    \item 陈述 $\neg P$ 与陈述 $P$ 具有\textbf{相反}的真值。
    \item 否定全称量词陈述 ($\forall$):
        \[\neg (\forall x \in S \centerdot P(x)) \iff \exists x \in S \centerdot \neg P(x)\]
    \item 否定存在量词陈述 ($\exists$):
        \[\neg (\exists x \in S \centerdot P(x)) \iff \forall x \in S \centerdot P(x)\]
    \item 否定析取陈述 ($\lor$):
        \[\neg (P \lor Q) \iff \neg P \land \neg Q\]
        这是\textbf{德摩根逻辑定律}之一。
    \item 否定合取陈述 ($\land$):
        \[\neg (P \land Q) \iff \neg P \lor \neg Q\]
        这是\textbf{德摩根逻辑定律}之一。
    \item 否定蕴含陈述 ($\implies$):
        \[\neg(P \implies Q) \iff \neg(\neg P \lor Q) \iff P \land \neg Q\]
    \item 否定等价陈述 ($\iff$):
        \[\neg (P \iff Q) \iff \neg [(P \implies Q) \land (Q \implies P)] \iff (P \land \neg Q) \lor (Q \land \neg P)\]
    \item 利用上述事实,我们可以否定\textbf{任何}数学陈述,因为数学陈述仅由量词、逻辑连接词和变量命题构成。通过从左到右阅读陈述并逐一否定,即可完成否定过程。
\end{itemize}

\subsection{证明策略}

我们使用缩写 AFSOC 表示``为了引出矛盾而假设 (assume for sake of contradiction)''。

\begin{itemize}
    \item 证明存在量词陈述:$\exists x \in S \centerdot P(x)$\\
            \emph{直接证明:}
            \begin{quote}
                构造一个具体例子,令 $y=\underline{\qquad}$。\\
                证明 $y \in S$。\\
                证明 $P(y)$ 为\verb|真|。
            \end{quote}
            \emph{间接证明:}
            \begin{quote}
                为了引出矛盾而假设,对于所有 $y \in S, \neg P(y)$ 成立。\\
                推导出矛盾。
            \end{quote}
    \item 证明全称量词陈述:$\forall x \in S \centerdot P(x)$\\
            \emph{直接证明:}
            \begin{quote}
                设 $y \in S$ 为任意固定元素。\\
                证明 $P(y)$ 为\verb|真|。
            \end{quote}
            \emph{间接证明:}
            \begin{quote}
                为了引出矛盾而假设,存在 $\exists y \in S$,使得 $\neg P(y)$ 成立。\\
                推导出矛盾。
            \end{quote}
    \item 证明析取陈述:$P \lor Q$\\
            \emph{直接证明:}
            \begin{quote}
                证明 $P$ 成立,或者证明 $Q$ 成立。
            \end{quote}
            \emph{间接证明 1:}
            \begin{quote}
                假设 $\neg P$ 成立,证明 $Q$ 成立。
            \end{quote}
            \emph{间接证明 2:}
            \begin{quote}
                为了引出矛盾而假设 $\neg P \land \neg Q$ 成立。\\
                推导出矛盾。
            \end{quote}
    \item 证明合取陈述:$P \land Q$\\
            \emph{直接证明:}
            \begin{quote}
                证明 $P$ 成立。\\
                证明 $Q$ 成立。
            \end{quote}
            \emph{间接证明:}
            \begin{quote}
                为了引出矛盾而假设 $\neg P \lor \neg Q$ 成立。\\
                考虑第一种情况,若 $\neg P$ 成立。推导出矛盾。\\
                考虑第二种情况,若 $\neg Q$ 成立。推导出矛盾。
            \end{quote}
    \item 证明蕴含陈述:$P \implies Q$\\
            \emph{直接证明:}
            \begin{quote}
                假设 $P$ 成立,证明 $Q$ 成立。
            \end{quote}
            \emph{反证法:}
            \begin{quote}
                假设 $\neg Q$ 成立,证明 $\neg P$ 成立。
            \end{quote}
            \emph{间接证明:}
            \begin{quote}
                为了引出矛盾而假设 $P$ 成立且 $Q$ 不成立。\\
                推导出矛盾。
            \end{quote}
    \item 证明等价陈述:$P \iff Q$\\
            \emph{直接证明:}
            \begin{quote}
                证明 $P \implies Q$(使用上述任意一种方法)。\\
                证明 $Q \implies P$(使用上述任意一种方法)。
            \end{quote}
            \emph{间接证明:}
            \begin{quote}
                为了引出矛盾而假设 $\neg (P \implies Q) \lor \neg (Q \implies P)$。\\
                考虑第一种情况,若 $P \land \neg Q$ 成立。推导出矛盾。\\
                考虑第二种情况,若 $Q \land \neg P$ 成立。推导出矛盾。
            \end{quote}
\end{itemize}
