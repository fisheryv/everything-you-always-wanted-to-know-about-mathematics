% !TeX root = ../../book.tex
\section{逻辑}

\subsection{陈述和命题}

\begin{itemize}
    \item \verb|真| 和 \verb|假| 是仅有的两个真值。
    \item \textbf{数学陈述}(或\textbf{逻辑陈述})是一个语法正确的句子且\textbf{只有}一个真值。
    \item \textbf{变量命题}是一个语法正确的句子,它包含一个或多个变量,只要为变量赋值,就能获得一个真值。
    \item 当我们定义一个陈述或命题时,我们为其分配一个字母名称,指出对变量的任何依赖(以及为它们分配字母),并将实际的陈述/命题括在引号内。这里有两个很好的例子:
          \begin{quote}
              将 $P$ 定义为``每个实数 $x$ 都满足 $x^2 \ge 0$''。\\
              将 $Q(x, y)$ 定义为 ``对于任意 $x,y \in \mathbb{R}, xy \le (\frac{x+y}{2})^2$''。
          \end{quote}
    \item \textbf{排中律}指我们假设每个陈述要么为\verb|真|要么为\verb|假|。它指出,当我们有一个陈述 $P$ 时,我们可以保证 $P$ 要么为\verb|真|要么为\verb|假|,并且\textbf{只有}其中一种情况成立。
\end{itemize}

\subsection{量词}

\begin{itemize}
    \item 遇到``对于任意''或``对于所有'',我们使用\textbf{全称量词} $\forall$。\\
        $\forall x \in S \centerdot P(x)$ 表示``对于任意元素 $x \in S$, 属性 $P(x)$ 成立。''
    \item 遇到``存在''或``至少有一个'',我们使用\textbf{存在量词} $\exists$。\\
        $\exists x \in S \centerdot P(x)$ 表示``存在一个元素 $x \in S$ 具有属性 $P(x)$。''
    \item 我们用 ``$\centerdot$'' 分隔量词陈述的各个部分。
    \item 当朗读量词陈述时,\textbf{只}在 $\exists$ 量词之后说``使得''。
    \item 我们用 ``$\exists !$'' 表示\textbf{唯一}存在;也就是说,$\exists ! x \in S \centerdot P(x)$ 表示``唯一存在一个元素 $x \in S$ 具有属性 $P(x)$''。
\end{itemize}

\subsection{连词}

假设 $P$ 和 $Q$ 为数学陈述。它们可能由带有量词的变量命题组成。

\begin{itemize}
    \item ``$P$ 与 $Q$'' 写做
        \[P \land Q\]
        当且仅当 $P$ 和 $Q$ 都为\verb|真| 时,才为\verb|真|。
    \item ``$P$ 或 $Q$'' 写做
        \[P \lor Q\]
        当且仅当 $P$ 和 $Q$ \textbf{至少一个}为\verb|真|时,它才为\verb|真|。(这里的或是\textbf{兼或},因此允许 $P$ 和 $Q$ 都为\verb|真|。)
    \item ``如果 $P$ 则 $Q$'' 写做
        \[P \implies Q\]
        当 $P$ 成立时,$Q$ 也成立,即 $P$ 和 $Q$ 具有相同的真值。\\
        请注意 $P \implies Q$ 本身就是一个逻辑陈述。它也具有真值,\verb|真|或\verb|假|。它不对构成陈述的 $P$ 和 $Q$ 的真值作任何声明。\\
        我们称之为\textbf{条件陈述};其中 $P$ 称为\textbf{假设},$Q$ 称为\textbf{结论}。\\
        这里需要注意的是,当 $P$ 为\verb|假|时,$P \implies Q$ 为\verb|真|。这是因为它是一个``如果……那么……''形式的陈述;\textbf{没有声明} $P$ 为\verb|假|的情况,因此我们不能声明条件陈述为\verb|假|,因此它只能为\verb|真|(根据排中律)。
    \item $P \implies Q$ 的等价写法为
        \[\neg P \lor Q\]
    \item 条件陈述 $P \implies Q$ 的\textbf{逆否命题}为
        \[\neg Q \implies \neg P\]
        它保证与 $P \implies Q$ 具有相同的真值。即,
        \[(P \implies Q) \iff (\neg Q \implies \neg P)\]
    \item 条件陈述 $P \implies Q$ 的\textbf{逆命题}为
        \[Q \implies P\]
        它\textbf{不}保证与 $P \implies Q$ 具有相同的真值。存在陈述 $P, Q$ 使得 $P \implies Q$ 成立并且逆命题也成立,也存在陈述 $P, Q$ 使得 $P \implies Q$ 成立但逆命题不成立。
    \item ``$P$ 与 $Q$ \textbf{逻辑等价}''写做
        \[P \iff Q\]
        并将其读做``$P$ 当且仅当 $Q$''。\\
        也可以写做
        \[(P \implies Q) \land (Q \implies P)\]
        这意味着无论哪种情况,$P$ 和 $Q$ 都\textbf{具有相同的真值}。
\end{itemize}

\subsection{逻辑否定}

\begin{itemize}
    \item 我们用 ``$\neg$'' 来表示一个陈述的\textbf{逻辑否定}。
    \item 陈述 $\neg P$ 与陈述 $P$ 具有\textbf{相反的}真值。
    \item 否定 $\forall$ 声明:
        \[\neg (\forall x \in S \centerdot P(x)) \iff \exists x \in S \centerdot \neg P(x)\]
    \item 否定 $\exists$ 声明:
        \[\neg (\exists x \in S \centerdot P(x)) \iff \forall x \in S \centerdot P(x)\]
    \item 否定 $\lor$ 声明:
        \[\neg (P \lor Q) \iff \neg P \land \neg Q\]
        这是\textbf{德摩根逻辑定律}之一。
    \item 否定 $\land$ 声明:
        \[\neg (P \land Q) \iff \neg P \lor \neg Q\]
        这是\textbf{德摩根逻辑定律}之一。
    \item 否定 $\implies$ 声明:
        \[\neg(P \implies Q) \iff \neg(\neg P \lor Q) \iff P \land \neg Q\]
    \item 否定 $\iff$ 声明:
        \[\neg (P \iff Q) \iff \neg [(P \implies Q) \land (Q \implies P)] \iff (P \land \neg Q) \lor (Q \land \neg P)\]
    \item 使用这些事实,我们可以否定\textbf{任何}数学陈述,因为陈述只是由量词、连词和变量命题组成。我们可以从左到右阅读陈述并否定每一部分。
\end{itemize}

\subsection{证明策略}

我们使用短语 AFSOC 表示``为了得到矛盾而假设 (assume for sake of contradiction)''。

\begin{itemize}
    \item 证明 $\exists$ 声明:$\exists x \in S \centerdot P(x)$\\
            \emph{直接证明:}
            \begin{quote}
                定义一个具体的例子,$y=\underline{\qquad}$。\\
                证明 $y \in S$。\\
                证明 $P(y)$ 为\verb|真|。
            \end{quote}
            \emph{间接证明:}
            \begin{quote}
                为了得到矛盾而假设,对于所有 $y \in S, \neg P(y)$ 成立。\\
                找到矛盾。
            \end{quote}
    \item 证明 $\forall$ 声明:$\forall x \in S \centerdot P(x)$\\
            \emph{直接证明:}
            \begin{quote}
                设 $y \in S$ 为任意固定元素。\\
                证明 $P(y)$ 为\verb|真|。
            \end{quote}
            \emph{间接证明:}
            \begin{quote}
                为了得到矛盾而假设,存在 $\exists y \in S$,使得 $\neg P(y)$ 成立。\\
                找到矛盾。
            \end{quote}
    \item 证明 $\lor$ 声明:$P \lor Q$\\
            \emph{直接证明:}
            \begin{quote}
                证明 $P$ 成立,或者证明 $Q$ 成立。
            \end{quote}
            \emph{间接证明 1:}
            \begin{quote}
                假设 $\neg P$ 成立。证明 $Q$ 成立。
            \end{quote}
            \emph{间接证明 2:}
            \begin{quote}
                为了得到矛盾而假设 $\neg P \land \neg Q$ 成立。\\
                找到矛盾。
            \end{quote}
    \item 证明 $\land$ 声明:$P \land Q$\\
            \emph{直接证明:}
            \begin{quote}
                证明 $P$ 成立。\\
                证明 $Q$ 成立。
            \end{quote}
            \emph{间接证明:}
            \begin{quote}
                为了得到矛盾而假设 $\neg P \lor \neg Q$ 成立。\\
                考虑第一种情况,其中 $\neg P$ 成立。找到矛盾。\\
                考虑第二种情况,其中 $\neg Q$ 成立。找到矛盾。
            \end{quote}
    \item 证明 $\implies$ 声明:$P \implies Q$\\
            \emph{直接证明:}
            \begin{quote}
                假设 $P$ 成立。 证明 $Q$ 成立。
            \end{quote}
            \emph{反证法:}
            \begin{quote}
                假设 $\neg Q$ 成立。 证明 $\neg P$ 成立。
            \end{quote}
            \emph{间接证明:}
            \begin{quote}
                为了得到矛盾而假设 $P$ 成立,假设 $Q$ 不成立。\\
                找到矛盾。
            \end{quote}
    \item 证明 $\iff$ 声明:$P \iff Q$\\
            \emph{直接证明:}
            \begin{quote}
                证明 $P \implies Q$(使用上述任意一种方法)。\\
                证明 $Q \implies P$(使用上述任意一种方法)。
            \end{quote}
            \emph{间接证明:}
            \begin{quote}
                为了得到矛盾而假设 $\neg (P \implies Q) \lor \neg (Q \implies P)$。\\
                考虑第一种情况,其中 $P \land \neg Q$ 成立。找到矛盾。\\
                考虑第二种情况,其中 $Q \land \neg P$ 成立。找到矛盾。
            \end{quote}
\end{itemize}
