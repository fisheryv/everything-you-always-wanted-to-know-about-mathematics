% !TeX root = ../../book.tex
\section{归纳法}

\subsection{数学归纳原理}

\begin{itemize}
    \item \textbf{定理}:设 $P(n)$ 是对所有 $n \in \mathbb{N}$ 定义的变量命题。\\
          若 $P(1)$ 成立,\\
          且 $\forall k \in \mathbb{N} \centerdot P(k) \implies P(k + 1)$。\\
          则 $\forall n \in \mathbb{N} \centerdot P(n)$。
    \item \textbf{归纳法证明}:设 $P(n)$ 是对所有 $n \in \mathbb{N}$ 定义的变量命题,欲证 $P(n)$ 对于每个 $n \in \mathbb{N}$ 均成立。
          \begin{quote}
              \textbf{基本情况}:证明 $P(1)$ 成立。\\
              \textbf{归纳假设}:设 $k$ 为任意固定自然数,假设 $P(k)$ 成立。\\
              \textbf{归纳步骤}:证明 $P(k + 1)$ 成立。\\
              \textbf{得出结论}:由数学归纳法可知,$\forall n \in \mathbb{N} \centerdot P(n)$。
          \end{quote}
\end{itemize}

\subsection{强数学归纳原理}

\begin{itemize}
    \item \textbf{定理}:设 $P(n)$ 是对所有 $n \in \mathbb{N}$ 定义的变量命题。\\
          若 $P(1)$ 成立,\\
          且 $\forall k \in \mathbb{N} \centerdot [P(1) \land P(2) \land \dots \land P(k)] \implies P(k + 1)$。\\
          则 $\forall n \in \mathbb{N} \centerdot P(n)$。
    \item \textbf{强归纳法证明}:设 $P(n)$ 是对所有 $n \in \mathbb{N}$ 定义的变量命题,欲证 $P(n)$ 对于每个 $n \in \mathbb{N}$ 均成立。
          \begin{quote}
              \textbf{基本情况}:证明 $P(1)$ 成立。(根据归纳步骤的需要,可能需要验证多个基本情况。)\\
              \textbf{归纳假设}:取满足条件(如 $k \ge \underline{\qquad}$,具体取决于归纳步骤)的任意固定自然数 $k$,假设 $P(1) \land P(2) \land \dots \land P(k)$ 成立。\\
              \textbf{归纳步骤}:证明 $P(k + 1)$ 成立。\\
              \textbf{得出结论}:由强归纳法可知,$\forall n \in \mathbb{N} \centerdot P(n)$。
          \end{quote}
\end{itemize}
