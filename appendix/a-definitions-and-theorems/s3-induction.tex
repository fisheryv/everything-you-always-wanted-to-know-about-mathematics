% !TeX root = ../../book.tex
\section{归纳法}

\subsection{特定数学归纳原理}

\begin{itemize}
    \item \textbf{定理}:假设 $P(n)$ 是对所有 $n \in \mathbb{N}$ 定义的变量命题。\\
          假设 $P(1)$ 成立。\\
          假设 $\forall k \in \mathbb{N} \centerdot P(k) \implies P(k + 1)$。\\
          则 $\forall n \in \mathbb{N} \centerdot P(n)$。
    \item \textbf{归纳法证明}:假设我们有一个对所有 $n \in \mathbb{N}$ 定义的变量命题 $P(n)$,我们想证明 $P(n)$ 对每个 $n \in \mathbb{N}$ 成立。
          \begin{quote}
              \textbf{基本情况}:证明 $P(1)$ 成立。\\
              \textbf{归纳假设}:假设 $k$ 为任意固定自然数,假设 $P(k)$ 成立。\\
              \textbf{归纳步骤}:证明 $P(k + 1)$ 成立。\\
              \textbf{得出结论}:通过归纳,$\forall n \in \mathbb{N} \centerdot P(n)$。
          \end{quote}
\end{itemize}

\subsection{强数学归纳原理}

\begin{itemize}
    \item \textbf{定理}:假设 $P(n)$ 是对所有 $n \in \mathbb{N}$ 定义的变量命题。\\
          假设 $P(1)$ 成立。\\
          假设 $\forall k \in \mathbb{N} \centerdot [P(1) \land P(2) \land \dots \land P(k)] \implies P(k + 1)$。\\
          则 $\forall n \in \mathbb{N} \centerdot P(n)$。
    \item \textbf{强归纳法证明}:假设我们有一个对所有 $n \in \mathbb{N}$ 定义的变量命题 $P(n)$,我们想证明 $P(n)$ 对每个 $n \in \mathbb{N}$ 成立。
          \begin{quote}
              \textbf{基本情况}:证明 $P(1)$ 成立。(根据归纳步骤中发生的情况,可能需要多个基本情况。)\\
              \textbf{归纳假设}:假设 $k$ 是满足某个不等式($k \ge \underline{\qquad}$,取决于归纳步骤中发生的情况)的任意固定自然数,并假设 $P(1) \land \dots \land P(k)$ 成立。\\
              \textbf{归纳步骤}:证明 $P(k + 1)$ 成立。\\
              \textbf{得出结论}:通过归纳,$\forall n \in \mathbb{N} \centerdot P(n)$。
          \end{quote}
\end{itemize}
