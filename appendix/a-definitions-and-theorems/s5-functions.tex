% !TeX root = ../../book.tex
\section{函数}

\begin{itemize}
    \item 设 $A$ 和 $B$ 为集合,$f$ 是 $A$ 与 $B$ 之间的\emph{关系},即 $f \subseteq A \times B$。若 $f$ 满足以下性质:
          \[\forall a \in A \centerdot \exists! b \in B \centerdot (a, b) \in f\]
          也就是说,定义域 $A$(即``输入''集)中的每一个元素都\emph{有且只有一个}值域 $B$(即``输出''集)中的元素与之对应。

          换言之,``每个输入都唯一对应一个输出。''

          满足这一性质的关系称为从 $A$ 到 $B$ 的\textbf{函数}。

          我们称 $A$ 为函数的定义域,$B$ 为函数的值域,并记作:
          \[f:A \to B\]
          表示 $f$ 是从 $A$ 到 $B$ 的函数。

          若 $a$ 对应 $b$,即 $(a, b) \in f$,则记作:
          \[f(a) = b\]
          这表明对于每个 $a \in A$,都存在\emph{唯一}的 $b \in B$ 与之对应。
    \item 给定定义域 $A$、值域 $B$ 以及一个描述输出对应关系的``规则'' $f$,若该规则对 $A$ 中所有元素都有定义,并且对于每个 $a \in A$ 都输出\emph{唯一}的 $B$ 中元素,则称 $f$ 是一个\textbf{明确定义}的函数。
    
    (注意:每个函数都是明确定义的;这一概念在判断给定``规则''是否构成函数时尤为有用。)
    \item 设 $f: A \to B$ 与 $g: A \to B$ 为两个函数。若 $\forall a \in A \centerdot f(a) = g(a)$,则称 $f$ 与 $g$ \textbf{相等}(在函数意义上),记作 $f = g$。换言之,若两个函数对所有输入都产生相同输出,则它们相等。
\end{itemize}

\subsection{像与原像}

\begin{itemize}
    \item 设 $f: A \to B$ 为函数,$X \subseteq A$。定义 $X$ 在 $f$ 下的\textbf{像}为:
          \[\im_f (X) = \{b \in B \mid \exists a \in X \centerdot f(a) = b\}\]
          该集合也可表示为:
          \[\im_f (X) = \{f(a) \mid a \in X\}\]
          (直观上,这是所有被 $X$ 中元素``命中''的值域元素的集合。)
    \item 设 $f: A \to B$ 为函数,$Z \subseteq B$。定义 $Z$ 在 $f$ 下的\textbf{原像}为:
          \[\pim_f (Z) = \{a \in A \mid f(a) \in Z\}\]
          (直观上,这是所有输出``落在'' $Z$ 上的``输入''的集合。)
    \item 注:$\im_f (\varnothing) = \varnothing$ 且 $\pim_f (\varnothing) = \varnothing$。
\end{itemize}

\subsection{映射}

\begin{itemize}
    \item 设 $f: A \to B$ 为函数。若 $\im_f (A) = B$,则称 $f$ 为\textbf{满射},或称 $f$ 是\textbf{满射的}。

          像的定义为我们提供了满射性的等价表述:
          \[f \text{\ 是满射的\ } \iff \forall b \in B \centerdot \exists a \in A \centerdot f(a) = b\]
          (直观上,当值域中\emph{所有}元素都被函数 $f$ ``命中''时,$f$ 是满射。)
    \item 设 $f: A \to B$ 为函数。若 $f$ 满足:
          \[\forall a_1, a_2 \in A \centerdot a_1 \ne a_2 \implies f(a_1) \ne f(a_2)\]
          则称 $f$ 为\textbf{单射},或称 $f$ 是textbf{单射的}。

          该条件陈述的逆否命题给出了单射的等价表述:
          \[\forall a_1, a_2 \in A \centerdot f(a_1) = f(a_2) \implies a_1 = a_2\]
          (直观上,若不同输入总对应不同输出,或相同输出必来自相同输入,则 $f$ 是单射。)
    \item 若函数 $f$ 既是单射又是满射,则称 $f$ 为\textbf{双射},或称 $f$ 是\textbf{双射的}。
\end{itemize}

\subsection{函数复合}

\begin{itemize}
    \item 设 $f: A \to B$,$g: B \to C$ 为函数。

          定义函数 $g \circ f : A \to C$ 为
          \[\forall a \in A \centerdot (g \circ f)(a) = g(f(a))\]
          称其为 $g$ 与 $f$ 的\textbf{复合},或 ``$g$ 复合 $f$''。

          注意:将 ``$\circ$'' 读作``之后''有助于理解操作顺序:$g \circ f$ 表示先应用 $f$,再应用 $g$。即先计算 $f(a)$,再计算 $g(f(a))$。
    \item 符号:我们写做 $(g \circ f)(x) = g(f(x))$。注意不要写成 $g \circ f(x)$,因为括号对明确复合顺序至关重要。
    \item 设 $f : A \to B$,$g : B \to C$,$h : C \to D$ 为函数,则 $(h \circ g) \circ f = h \circ (g \circ f)$。

          这称为\textbf{复合的结合律}。
    \item 若 $f : A \to B$ 和 $g : B \to C$ 都是单射(或满射,或双射),则 $g \circ f$ 也是单射(或满射,或双射)。

\end{itemize}

\subsection{反函数}

\begin{itemize}
    \item 设 $X$ 为任意集合。定义恒等函数 $\id_X : X \to X$ 为 $\forall z \in X \centerdot \id_X(z) = z$。
    \item 设 $f : A \to B$ 为函数。若存在函数 $F : B \to A$,使得 $f \circ F = \id_B$ 且 $F \circ f = \id_A$,则称 $F$ 是 $f$ 的\textbf{反函数},并写做 $F = f^{-1}$。

          注意,正式定义要求验证两种复合方式均得到恒等函数。存在一种复合成立而另一种不成立的情形,因此必须检查两种情况。

          (注意:在证明一个函数是另一个函数的反函数时,不应立即写成 $f^{-1}$,因为此时还在证明 $f$ 存在反函数。)

          若 $f$ 有反函数,则称 $f$ \textbf{可逆}。
    \item \textbf{定理}:函数 $f : A \to B$ 是双射当且仅当 $f$ 存在反函数 $f^{-1}: B \to A$。
    \item \textbf{定理}:设 $f : A \to B$ 和 $g : B \to C$ 均为双射,则 $g \circ f : A \to C$ 也是双射,因而存在反函数,且 $(g \circ f)^{-1} = f^{-1} \circ g^{-1}$。
\end{itemize}

\subsection{函数证明技巧}

\begin{itemize}
    \item 证明 $f$ 是\textbf{满射}:
          \begin{quote}
              设 $b \in B$ 为任意固定元素。

              定义 $a=\underline{\qquad}$ 。

              证明 $a \in A$。

              证明 $f(a) = b$ 。

              这表明 $b \in \im_f (A)$,因此 $B \subseteq \im_f (A)$。

              根据定义 $\im_f (A) \subseteq B$,这表明 $\im_f (A) = B$,所以 $f$ 是满射。
          \end{quote}
    \item 证明 $f$ \textbf{不是满射}:
          \begin{quote}
              定义 $b=\underline{\qquad}$。

              证明 $b \in B$。

              设 $a \in A$ 为任意固定元素。

              证明 $f(a) \ne b$。(或者假设 $f(a) = b$ 并推导出矛盾。)

              这表明 $\exists b \in B \centerdot b \notin \im_f (A)$,所以 $f$ 不是满射。
          \end{quote}
    \item 证明 $f$ 是\textbf{单射}:
          \begin{quote}
              设 $x,y \in A$ 为任意固定元素。

              假设 $f(x) = f(y)$。

              推导出 $x = y$。
          \end{quote}
          或者
          \begin{quote}
              设 $x,y \in A$ 为任意固定元素。

              假设 $x \ne y$。

              推导出 $f(x) \ne f(y)$。
          \end{quote}
    \item 证明 $f$ \textbf{不是单射}:
          \begin{quote}
              定义 $x=\underline{\qquad}$,定义 $y=\underline{\qquad}$。

              证明 $x \in A$ 且 $y \in A$。

              证明 $x \ne y$。

              证明 $f(x) = f(y)$。

              这表明 $\exists x, y \in A \centerdot x \ne y \land f(x) = f(y)$,所以 $f$ 不是单射。
          \end{quote}
    \item 证明 $f$ 是\textbf{双射}:
          \begin{quote}
              证明 $f$ 是满射。

              证明 $f$ 是单射。
          \end{quote}
          或者
          \begin{quote}
              定义函数 $F:B \to A$。

              证明 $F \circ f = \id_A$ 。

              证明 $f \circ F = \id_B$ 。

              这表明 $F = f^{-1}$,故 $f$ 可逆,因此为双射。
          \end{quote}
    \item 证明 $f$ \textbf{不是双射}:
          \begin{quote}
              证明 $f$ 不是单射,或证明 $f$ 不是满射。
          \end{quote}
          或者
          \begin{quote}
              为了引出矛盾而假设 $f$ 为双射,则必然存在反函数 $f^{-1}$。由此推导出矛盾。
          \end{quote}
    \item 对于 $X \subseteq A$,求像 $\im_f (X)$:
          \begin{itemize}
              \item 定义集合 $S$,并设 $S = \im_f(X)$。

                    (注意:提出该定义可能涉及大量计算,但无需在证明中展示过程,直接使用定义即可。)
              \item 证明 $\im_f (X) \subseteq S$。
                    \begin{itemize}
                        \item 设 $y \in \im_f(x)$ 为任意固定元素。
                        \item 这意味着 $\exists a \in X \centerdot f(a) = y$。
                        \item 利用 $f$ 的性质证明 $f(a) \in S$。
                        \item 这表明 $y \in S$。
                    \end{itemize}
              \item 证明 $S \subseteq \im_f (X)$。
                    \begin{itemize}
                        \item 设 $z \in S$ 为任意固定元素。
                        \item 定义 $x = \underline{\qquad}$。
                        \item 证明 $x \in X$。
                        \item 证明 $f(x) = z$。
                        \item 这表明 $z \in \im_f (X)$。
                    \end{itemize}
              \item 通过双向包含得到结论 $\im_f (X) = S$ 。
          \end{itemize}
    \item 对于 $Z \subseteq B$,求原像 $\pim_f (Z)$:
          \begin{itemize}
              \item 定义集合 $T$,并设 $T = \pim_f(Z)$。

                    (注意:提出该定义可能涉及大量计算,但无需在证明中展示过程,直接使用定义即可。)
              \item 证明 $\pim_f(Z) \subseteq T$。
                    \begin{itemize}
                        \item 设 $a \in \pim_f(Z)$ 为任意固定元素。
                        \item 这意味着 $f(a) \in Z$。
                        \item 利用 $f$ 的性质证明 $a \in T$。
                    \end{itemize}
              \item 证明 $T \subseteq \pim_f(Z)$。
                    \begin{itemize}
                        \item 设 $x \in T$ 为任意固定元素。
                        \item 利用 $f$ 的性质证明 $f(x) \in Z$。
                        \item 这表明 $x \in \pim_f(Z)$。
                    \end{itemize}
              \item 通过双向包含得到结论 $\pim_f (Z) = T$。
          \end{itemize}
    \item 求 $f$ 的\textbf{反函数}:
          \begin{itemize}
              \item 定义函数 $F: B \to A$。

                    (注意:提出该定义可能涉及大量计算,但无需在证明中展示过程,直接使用定义即可。)
              \item 证明 $F$ 是一个明确定义的函数:即每个来自 $B$ 的输入都只有一个 $A$ 上的输出。
              \item 证明 $F \circ f = \id_A$。
              \item 证明 $f \circ F = \id_B$。
              \item 由此可得 $F = f^{-1}$。(由于 $f$ 存在反函数,因此它必为双射。)
          \end{itemize}
\end{itemize}
