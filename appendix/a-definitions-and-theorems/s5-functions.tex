% !TeX root = ../../book.tex
\section{函数}

\begin{itemize}
    \item 设 $A,B$ 为集合。设 $f$ 是 $A, B$ 之间的\emph{关系},因此 $f \subseteq A \times B$。此外,假设 $f$ 具有以下性质
          \[\forall a \in A \centerdot \exists! b \in B \centerdot (a, b) \in f\]
          也就是说,假设定义域 $A$(``输入''集)的每个元素\emph{有且只有一个}对应的值域 $B$(``输出''集)的元素,使得这两个元素在 $f$ 下是相关的。

          换句话说,``每个输入都只有唯一一个对应的输出。''

          这种关系称为从 $A$ 到 $B$ 的函数。

          我们称 $A$ 为函数的定义域,$B$ 为函数的值域。写做
          \[f:A \to B\]
          表示 $f$ 是从 $A$ 到 $B$ 的函数。

          如果 $a$ 与 $b$ 相关,即 $(a, b) \in f$,那么写做
          \[f(a) = b\]
          可知对于每一个 $a$ 都有\emph{唯一一个} $b$ 与之相关。
    \item 给定定义域 $A$,值域 $B$,以及一个说明输出内容的``规则'' $f$,给定 $A$ 的一个元素,如果规则定义在 $A$ 的所有元素上,并且对于每个 $a \in A$,规则输出\emph{唯一一个}集合 $B$ 中的元素,我们说 $f$ 是一个\textbf{明确定义}的函数。

          (注意:每个函数都是一个明确定义的函数;当我们试图确定一个给定的``规则''是否真的是一个函数时,这条规则很有用。)
    \item 令 $f : A \to B, g : A \to B$ 为函数。当 $\forall a \in A \centerdot f(a) = g(a)$ 时,我们说 $f$ 和 $g$ \textbf{相等}(在函数意义上),写做 $f = g$。也就是说,当两个函数对每个输入都产生相同的输出时,$f = g$。
\end{itemize}

\subsection{像与原像}

\begin{itemize}
    \item 设 $f : A \to B$ 为函数。设 $X \subseteq A$。$X$ 在函数 $f$ 下的\textbf{像}为
          \[\im_f (X) = \{b \in B \mid \exists a \in X \centerdot f(a) = b\}\]
          这个集合的等效写法为
          \[\im_f (X) = \{f(a) \mid a \in X\}\]
          (直观地讲,这是被 $X$ 中的元素``击中''的所有值域元素的集合。)
    \item 设 $f : A \to B$ 为函数。设 $Z \subseteq B$。$Z$ 在函数 $f$ 下的\textbf{原像}为
          \[\pim_f (Z) = \{a \in A \mid f(a) \in Z\}\]
          (直观地讲,这是所有输出``落在'' $Z$ 上的``输入''的集合。)
    \item 注:$\\im_f (\varnothing) = \varnothing$ 且 $\pim_f (\varnothing) = \varnothing$。
\end{itemize}

\subsection{映射}

\begin{itemize}
    \item 设 $f : A \to B$ 为函数。如果 $\\im_f (A) = B$,那么我们说 $f$ 是\textbf{满射的},或者说它是一个\textbf{满射}。

          像的定义为我们提供了满射性的等效公式:
          \[f \text{是满射的} \iff \forall b \in B. \exists a \in A. f(a) = b\]
          (直观地讲,当\emph{所有}值域中的元素都被函数``命中''时,$f$ 是满射的。)
    \item 设 $f : A \to B$ 为函数。如果 $f$ 具有如下性质:
          \[\forall a_1, a_2 \in A \centerdot a_1 \ne a_2 \implies f(a_1) \ne f(a_2)\]
          那么我们说 $f$ 是\textbf{单射的},或者说它是一个\textbf{单射}。

          该条件陈述的逆否命题给出了单射性的等效公式:
          \[\forall a_1, a_2 \in A \centerdot f(a_1) = f(a_2) \implies a_1 = a_2\]
          (直观地讲,当两个不同的输入总是产生不同的输出时,或者当具有相同的输出意味着它们来自相同的输入时,$f$ 是单射的。)
    \item 如果函数 $f$ 既是单射又是满射,那么我们说 $f$ 是\textbf{双射的},或者说它是一个\textbf{双射}。
\end{itemize}

\subsection{函数复合}

\begin{itemize}
    \item 设 $f: A \to B , g: B \to C$ 为函数。

          函数 $g \circ f : A \to C$ 定义为
          \[\forall a \in A \centerdot (g \circ f)(a) = g(f(a))\]
          是 $g$ 与 $f$ 的\textbf{复合},或 ``$g$ 复合 $f$''。

          注意:将 ``$\circ$'' 读作``之后''有助于提醒你操作顺序:$g \circ f$ 表示 $g$ 在 $f$ 之后应用。我们先求出 $f(a)$ 然后再求出 $g(f(a))$。
    \item 符号:我们写做 $(g \circ f)(x) = g(f(x))$。不要写成 $g \circ f(x)$。括号很重要!
    \item 设 $f : A \to B , g : B \to C , h : C \to D$ 为函数。 那么 $(h \circ g) \circ f = h \circ (g \circ f)$。

          此为\textbf{复合的结合律}。
    \item 假设 $f : A \to B$ 和 $g : B \to C$ 都是单射/满射/双射。 那么 $g \circ f$ 也是单射/满射/双射。

\end{itemize}

\subsection{反函数}

\begin{itemize}
    \item 设 $X$ 为任意集合。单位函数 $\id_X : X \to X$ 定义为 $\forall z \in X \centerdot \id_X(z) = z$。
    \item 设 $f : A \to B$ 为函数。如果存在函数 $F : B \to A$ 使得 $f \circ F : B \to B$ 满足 $f \circ F = \id_B$ 且 $F \circ f : A \to A$ 满足 $F \circ f = \id_A$ ,那么我们说 $F$ 是 $f$ 的\textbf{反函数},并写做 $F = f^{-1}$。

          请注意,正式定义清楚地写明需要检查两个函数的\emph{两种}复合方式是否产生单位函数。存在一种复合方法有效而另一种复合方法无效的示例!

          (注意:当证明一个函数是另一个函数的反函数时,我们还不能写成 $f^{-1}$,因为事实上,我们还在证明 $f$ 有一个反函数。)

          如果 $f$ 有反函数,我们说 $f$ \textbf{可逆}。
    \item \textbf{定理}:$f : A \to B$ 是双射 $\iff f$ 存在反函数 $f^{-1}: B \to A$。
    \item \textbf{定理}:设 $f : A \to B$ 和 $g : B \to C$ 都是双射。那么 $g \circ f : A \to C$ 也是双射,因此它有反函数;反函数为 $(g \circ f)^{-1} = f^{-1} \circ g^{-1}$。
\end{itemize}

\subsection{函数证明技巧}

\begin{itemize}
    \item 证明 $f$ 是\textbf{满射}:
          \begin{quote}
              设 $b \in B$ 为任意固定元素。

              定义 $a=\underline{\qquad}$ 。

              证明 $a \in A$。

              证明 $f(a) = b$ 。

              这表明 $b \in \im_f (A)$,所以 $B \subseteq \im_f (A)$。

              根据定义 $\im_f (A) \subseteq B$,这表明 $\im_f (A) = B$,所以 $f$ 是满射。
          \end{quote}
    \item 证明 $f$ \textbf{不是满射}:
          \begin{quote}
              定义 $b=\underline{\qquad}$。

              证明 $b \in B$。

              设 $a \in A$ 为任意固定元素。

              证明 $f(a) \ne b$。(或者,假设 $f(a) = b$ 并找到矛盾。)

              这表明 $\exists b \in B \centerdot b \notin \im_f (A)$,所以 $f$ 不是满射。
          \end{quote}
    \item 证明 $f$ 是\textbf{单射}:
          \begin{quote}
              设 $x,y \in A$ 为任意固定元素。

              假设 $f(x) = f(y)$。

              推导出 $x = y$。
          \end{quote}
          或者
          \begin{quote}
              设 $x,y \in A$ 为任意固定元素。

              假设 $x \ne y$。

              推导出 $f(x) \ne f(y)$。
          \end{quote}
    \item 证明 $f$ \textbf{不是单射}:
          \begin{quote}
              定义 $x=\underline{\qquad}$,定义 $y=\underline{\qquad}$。

              证明 $x \in A$ 且 $y \in A$。

              证明 $x \ne y$。

              证明 $f(x) = f(y)$。

              这表明 $\exists x, y \in A \centerdot x \ne y \land f(x) = f(y)$,所以 $f$ 不是单射。
          \end{quote}
    \item 证明 $f$ 是\textbf{双射}:
          \begin{quote}
              证明 $f$ 是满射。

              证明 $f$ 是单射。
          \end{quote}
          或者
          \begin{quote}
              定义函数 $F:B \to A$。

              证明 $F \circ f = \id_A$ 。

              证明 $f \circ F = \id_B$ 。

              这表明 $F = f^{-1}$,所以 $f$ 可逆,因此它是双射。
          \end{quote}
    \item 证明 $f$ \textbf{不是双射}:
          \begin{quote}
              证明 $f$ 不是单射,或证明 $f$ 不是满射。
          \end{quote}
          或者
          \begin{quote}
              为了引出矛盾而假设 $f$ 是双射,所以必然存在反函数 $f^{-1}$。找到矛盾。
          \end{quote}
    \item 对于 $X \subseteq A$,求像 $\im_f (X)$:
          \begin{itemize}
              \item 定义集合 $S$,声明 $S = \im_f(X)$。

                    (注意:提出这个定义并非易事,并且会涉及大量演算工作。没有必要将其作为证明的一部分。只需从定义开始。)
              \item 证明 $\im_f (X) \subseteq S$。
                    \begin{itemize}
                        \item 设 $y \in \im_f(x)$ 为任意固定元素。
                        \item 这意味着 $\exists a \in X \centerdot f(a) = y$。
                        \item 利用 $f$ 的性质证明 $f(a) \in S$。
                        \item 这表明 $y \in S$。
                    \end{itemize}
              \item 证明 $S \subseteq \im_f (X)$。
                    \begin{itemize}
                        \item 设 $z \in S$ 为任意固定元素。
                        \item 定义 $x = \underline{\qquad}$。
                        \item 证明 $x \in X$。
                        \item 证明 $f(x) = z$。
                        \item 这表明 $z \in \im_f (X)$。
                    \end{itemize}
              \item 通过双向包含得到结论 $\im_f (X) = S$ 。
          \end{itemize}
    \item 对于 $Z \subseteq B$,求原像 $\pim_f (Z)$:
          \begin{itemize}
              \item 定义集合 $T$,声明 $T = \pim_f(Z)$。

                    (注意:提出这个定义并非易事,并且会涉及大量演算工作。没有必要将其作为证明的一部分。只需从定义开始。)
              \item 证明 $\pim_f(Z) \subseteq T$。
                    \begin{itemize}
                        \item 设 $a \in \pim_f(Z)$ 为任意固定元素。
                        \item 这意味着 $f(a) \in Z$。
                        \item 利用 $f$ 的性质证明 $a \in T$。
                    \end{itemize}
              \item 证明 $T \subseteq \pim_f(Z)$。
                    \begin{itemize}
                        \item 设 $x \in T$ 为任意固定元素。
                        \item 利用 $f$ 的性质证明 $f(x) \in Z$。
                        \item 这表明 $x \in \pim_f(Z)$。
                    \end{itemize}
              \item 通过双向包含得到结论 $\pim_f (Z) = T$。
          \end{itemize}
    \item 求 $f$ 的\textbf{反函数}:
          \begin{itemize}
              \item 定义函数 $F: B \to A$。

                    (注意:提出这个定义并非易事,并且会涉及大量演算工作。没有必要将其作为证明的一部分。只需从定义开始。)
              \item 证明 $F$ 是一个明确定义的函数:证明每个来自 $B$ 的输入都只有一个 $A$ 上的输出。
              \item 证明 $F \circ f = \id_A$。
              \item 证明 $f \circ F = \id_B$。
              \item 推导出 $F = f^{-1}$。(由于 $f$ 存在反函数,因此它也是双射。)
          \end{itemize}
\end{itemize}
