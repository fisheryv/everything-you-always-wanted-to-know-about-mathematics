% !TeX root = ../../book.tex
\section{集合}

\subsection{标准集}

\begin{itemize}
    \item \textbf{自然数}集定义为:
        \[\mathbb{N} = \{1,2,3,4,5,\dots\}\]
        注意:$0 \notin \mathbb{N}$
    \item 对于任意 $n \in \mathbb{N}$,集合 $[n]$ (``\textbf{中括号} $n$'')定义为:
        \[[n] = \{x \in \mathbb{N} \mid 1 \le x \le n\} = \{1,2,3,\dots,n\}\]
    \item \textbf{整数}集定义为:
        \[\mathbb{Z} = \{\dots,-3,-2,-1,0,1,2,3,\dots\}\]
    \item \textbf{有理数}集定义为:
        \[\mathbb{Q} = \{x \in \mathbb{R} \mid \exists a,b \in \mathbb{Z}. b \ne 0 \;\text{且}\; \frac{a}{b}=x\}\]
    \item \textbf{实数}集记作 $\mathbb{R}$。任何一个实数要么是\textbf{有理数},要么是\textbf{无理数}。
    \item \textbf{空集}是不含任何元素的集合,记作 $\varnothing$ 或 $\{\:\}$
\end{itemize}


\subsection{集合构建符}

\begin{itemize}
    \item 设 $U$ 是一个集合,$P(x)$ 是定义在 $U$ 上的一个\textbf{属性}(即对于每个 $x \in U$,$P(x)$ 要么成立,要么不成立)。则我们可以通过以下方式构造一个新集合:
        \[S = \{x \in U \mid P(x) \;\text{成立}\}\]
    \item 上述表示称为\textbf{集合构建符}。使用时必须明确指定\textbf{全集} $U$ 和\textbf{属性} $P(x)$。
\end{itemize}

\subsection{元素与子集}

\begin{itemize}
    \item ``$x$ 是集合 $S$ 的元素''记作:
        \[x \in S\]
        ``$x$ 不是集合 $S$ 的元素''记作:
        \[x \notin S\]
    \item ``$S$ 是 $T$ 的子集''记作:
        \[S \subseteq T\]
        其定义为条件陈述``$S$ 的每个元素都是 $T$ 的元素''。用逻辑符号表示为:
        \[\forall x \in U \centerdot x \in S \implies x \in T\]
        即,对于全集 $U$(假设 $S,T \subseteq U$)中的每个元素 $x$,只要 $x \in S$,必有 $x\in T$。
    \item 要\textbf{证明}一个集合是另一个集合的子集,例如 $S \subseteq T$,可按以下步骤进行:
        \begin{align*}
            &\text{设}\ x \in S \ \text{为任意固定元素}\\
            &\dots \ \text{此处是证明过程(略)}\ \dots\\
            &\text{所以}\ x \in T\\
            &\text{这就证明了}\ S \subseteq T
        \end{align*}
    \item ``$S$ 是 $T$ 的真子集''记作:
        \[S \subset T\]
        这意味着 $S \subseteq T$ 且 $S \ne T$。
    \item 空集是任意集合的子集,即 $\forall S,\varnothing \subseteq S$。
    \item 任意集合都是其本身的子集,即 $\forall S, S \subseteq S$。
\end{itemize}

\subsection{幂集}

\begin{itemize}
    \item 设 $S$ 为集合,$S$ 的幂集记作 $\mathcal{P}(S)$,定义为:
        \[\mathcal{P}(S) = \{A \mid A \subseteq S\}\]
        即 $\mathcal{P}(S)$ 是 $S$ 的所有子集构成的集合。
    \item 对于任意集合 $S$,空集 $\varnothing \in \mathcal{P}(S)$ 且  $S \in \mathcal{P}(S)$。
\end{itemize}

\subsection{集合相等}

\begin{itemize}
    \item ``集合 $S$ 与集合 $T$ 相等'',记作 $S=T$,定义为:
        \[S = T \ \text{当且仅当}\ S \subseteq T \ \text{且}\  T \subseteq S\]
    \item 要\textbf{证明}两个集合相等,比如 $S=T$,可按以下步骤进行:
        \begin{align*}
            &\text{首先证明}\ S \subseteq T\\
            &\text{设}\ x \in S \ \text{为任意固定元素}\\
            &\dots \text{此处是证明过程(略)}\dots\\
            &\text{所以}\ x \in T\\
            &\text{因此}\ S \subseteq T\\
            \\
            &\text{接着证明}\ T \subseteq S\\
            &\text{设}\ y \in T \ \text{为任意固定元素}\\
            &\dots \text{此处是证明过程(略)}\dots\\
            &\text{所以}\ y \in S\\
            &\text{因此}\ T \subseteq S\\
            \\
            &\text{综上,}\ S = T
        \end{align*}
        这被称为\textbf{双向包含论证}。
\end{itemize}

\subsection{集合运算}

设 $S,T,U$ 为集合,且 $S \subseteq U, T \subseteq U$。

\begin{itemize}
    \item 两个集合的\textbf{并集}定义为:
        \[ S \cup T = \{x \in U \mid x \in S \ \text{或}\ x \in T\}\]
        即属于 $S$ 或 $T$(或同时属于两者)的所有元素构成的集合。
    \item 两个集合的\textbf{交集}定义为:
        \[S \cap T = \{x \in U \mid x \in S \ \text{且}\ x \in T\}\]
        即同时属于 $S$ 和 $T$ 的所有元素构成的集合。
    \item 两个集合的\textbf{差集}定义为:
        \[S - T = \{x \in U \mid x \in S \ \text{且}\ x \notin T\}\]
        即属于 $S$ 但不属于 $T$ 的所有元素构成的集合。
    \item 集合的\textbf{补集}定义为:
        \[\overline{S} = \{x\in U \mid x \notin S\}=U-S\]
        即全集中不属于 $S$ 的所有元素构成的集合。
    \item 两个集合的\textbf{笛卡尔积}定义为:
        \[S \times T = \{(x,y) \mid x \in S \ \text{且}\  y \in T\}\]
        即所有形如 $(x,y)$ 的有序对构成的集合,其中 $x \in S$,$y \in T$。
\end{itemize}

\subsection{索引集运算}

设 $I$ 为索引集,$U$ 为全集,且对于每个 $i \in I$,有集合 $A_i \subseteq U$。

\begin{itemize}
    \item 所有 $A_i$ 集合的\textbf{索引并集}定义为:
        \[\bigcup_{i \in I}A_i = \{x \in U \mid \exists k \in I. x \in A_k\}\]
        即全集中\textbf{至少}属于某个 $A_k$($k \in I$)的所有元素 $x$ 构成的集合。
    \item 所有 $A_i$ 集合的\textbf{索引交集}定义为:
        \[\bigcap_{i \in I}A_i = \{x \in U \mid \forall i \in I.x \in A_i\}\]
        即全集中同时属于\textbf{所有} $A_i$($i \in I$)的所有元素 $x$ 构成的集合。
\end{itemize}

\subsection{划分}

\begin{itemize}
    \item 设 $S$ 为集合,$S$ 的\textbf{划分}是两两不相交且并集为 $S$ 的非空子集族。也就是说,划分由索引集 $I$ 和满足下列条件的\emph{非空集合} $S_i$($i \in I$)构成:
        \begin{align*}
            & \forall i \in I \centerdot S_i \ne \varnothing\\
            & \forall i \in I \centerdot S_i \subset S\\
            & \forall i,j \in I \centerdot i \ne j \implies S_i \cap S_j = \varnothing\\
            & \bigcup_{i \in I} S_i= S
        \end{align*}
\end{itemize}


