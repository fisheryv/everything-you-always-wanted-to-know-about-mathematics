% !TeX root = ../../book.tex
\section{关系}

设 $A, B$ 为集合。$A$ 与 $B$ 之间的关系是有序对的集合 $R \subseteq A \times B$。

给定元素 $a \in A$ 和 $b \in B$,称 $a$ 与 $b$ \textbf{相关},当且仅当 $ (a, b) \in R$。

集合 $A$ 称为关系的\textbf{定义域},集合 $B$ 称为关系的\textbf{值域}。

集合 $R$ 称为\textbf{关系集}。

我们称 $R$ 是 \textbf{$A$ 与 $B$ 之间}的关系。

当 $A = B$ 时,则称 $R$ 是\textbf{集合 $A$ 上}的关系。

\subsection{关系的性质}

设 $A$ 为集合,$R$ 是 $A$ 上的关系,即 $R \subseteq A \times A$。

(注意:以下性质\emph{仅}适用于 $A$ 上的关系,不适用于两个\emph{不同}集合 $A$ 和 $B$ 之间的关系。)

\begin{itemize}
    \item 称 $R$ 具有\textbf{自反性},如果
          \[\forall x \in A \centerdot (x, x) \in R\]
          (即每个元素都与其自身相关)。
    \item 称 $R$ 具有\textbf{对称性},如果
          \[\forall x, y \in A \centerdot (x, y) \in R \implies (y, x) \in R\]
          (即元素的顺序不影响相关性)。
    \item 称 $R$ 具有\textbf{传递性},如果
          \[\forall x, y, z \in A \centerdot [(x, y) \in R \land (y, z) \in R] \implies (x, z) \in R\]
          (即关系可以通过``中间元素''传递)。
    \item 称 $R$ 具有\textbf{反对称性},如果
          \[\forall x, y \in A \centerdot [(x, y) \in R \land (y, x) \in R] \implies x = y\]
          (即相互关联的两个元素必然相同)。
\end{itemize}

\subsection{等价关系}

设 $A$ 为集合,$R$ 是 $A$ 上的关系。

\begin{itemize}
    \item 称 $R$ 为\textbf{等价关系},当且仅当 $R$ 具有自反性、对称性和传递性。
    \item 若 $R$ 为等价关系且 $x \in A$,则 $x$ \textbf{在关系 $R$ 下的等价类} 定义为
          \[[x]_R = \{y \in A \mid (x, y) \in R\}\]
          即所有与 $x$ 相关的元素构成的集合。
    \item 若 $R$ 为等价关系,则 $A/R$ 表示 $A$ \textbf{模} $R$,即所有等价类组成的集合:
          \[A/R = \{[x]_R \mid x \in A\}\]
    \item \textbf{定理}:若 $R$ 是 $A$ 上的等价关系,则其等价类(即 $A/R$ 的元素)构成 $A$ 的一个\emph{划分}。
    \item \textbf{定理}:若 $I$ 为索引集,且 $\{S_i \mid i \in I\}$ 是 $A$ 的一个\emph{划分},则存在 $A$ 上唯一的\textbf{等价关系},使得两个元素相关当且仅当它们属于该划分的同一部分。
\end{itemize}

\subsection{模运算}

\subsubsection{模 $n$ 同余}

\begin{itemize}
    \item 给定 $n \in \mathbb{N}$,对于任意 $x, y \in \mathbb{Z}$,我们说 $x$ 和 $y$ \textbf{模 $n$ 同余},当且仅当 $n \mid x-y$。

    等价地,这意味着 $x$ 和 $y$ 除以 $n$ 后的余数相同。注意,这一等价关系并非定义的一部分,而是由下面的除法引理推导得出。

    我们将其写做 $x \equiv y \mod n$。

    (注意:$\mod n$ 不是运算符或函数;它是置于算术或代数式末尾的修饰符,表示所有运算都以 $n$ 为模进行。)
    \item 对于每个 $n \in \mathbb{N}$,关系 $\equiv$ 是一个等价关系。
    \item \textbf{除法引理}:给定 $n \in \mathbb{N}$ 和 $x \in \mathbb{Z}$,则
          \[\exists! k, r \in \mathbb{Z} \centerdot (x = kn + r) \land (0 \le r \le n - 1)\]
          注意,符号``$\exists!$''表示\emph{唯一存在}整数 $k$ 和 $r$,使得 $x$ 可以唯一地表示为 $n$ 的倍数加上余数。
    \item \textbf{模算术引理}:给定 $n \in \mathbb{N}$ 和 $a, b \in \mathbb{Z}$,假设 $a \equiv r \mod n$ 且 $b \equiv s \mod n$,则
          \[a + b \equiv r + s \mod n \quad \text{且} \quad a \cdot b \equiv r \cdot s \mod n\]
\end{itemize}

\subsubsection{$\mathbb{Z}$ mod $n$ 中的乘法逆元}

\begin{itemize}
    \item 给定 $x, y \in \mathbb{Z}$,称 $x$ 与 $y$ 互质,当且仅当它们除了 $1$ 以外没有其他公因数。
    \item \textbf{MIRP 引理}:(互质时的乘法逆元引理)假设 $n \in \mathbb{N}$,$a \in \mathbb{Z}$,且 $a$ 和 $n$ 互质,则同余式 $ax \equiv 1 \mod n$ 存在解 $x$。

    (事实上,该同余式有无穷多个解,且所有解模 $n$ 同余。)
    \item 若 $ax \equiv 1 \mod n$,则称 $x$ 是 $a$ 在 $\mathbb{Z}$ 模 $n$ 中的\textbf{乘法逆元},记作 $x \equiv a^{-1} \mod n$。

    实际上,任何与 $x$ 模 $n$ 同余的整数 $y$ 都满足 $ay \equiv 1 \mod n$。因此,严格来说,等价类 $[x]_{\mod n}$ 是等价类 $[a]_{\mod n}$ 的乘法逆元。
    \item 若 $a^{-1}$ 存在,则 $(a^{-1})^{-1} \equiv a \mod n$。
    \item 设 $p$ 为质数,则所有整数 $1, 2, 3, \dots, p-1$ 都与 $p$ 互质,因此在 $\mathbb{Z}$ 模 $p$ 中,它们都有乘法逆元。
\end{itemize}

\subsubsection{结论}

\begin{itemize}
    \item \textbf{中国剩余定理}:设 $r \in \mathbb{N}$,且有 $r$ 个两两互质的自然数 $n_1, n_2, \dots, n_r$(即除了 $1$ 以外,任意两个数没有公因数)。

    再设 $r$ 个整数 $a_1, a_2, \dots, a_r$。

    则由 $n_i$ 和 $a_i$ 定义的同余方程组存在解 $X$:
    \[\exists X \in \mathbb{Z} \centerdot \forall i \in [r] \centerdot X \equiv a_i \mod n_i\]
    此外,若定义 $N = \prod_{i \in [r]} n_i$,则方程组的任意解 $Y$ 都满足 $Y \equiv X \mod N$,即所有解模 $N$ 同余。
\end{itemize}

