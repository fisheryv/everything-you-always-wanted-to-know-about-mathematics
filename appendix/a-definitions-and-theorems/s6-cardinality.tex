% !TeX root = ../../book.tex
\section{基数}

\subsection{定义}

设 $S$ 为任意集合。

\begin{itemize}
    \item 若 $\exists n \in \mathbb{N} \cup \{0\}$ 及双射 $f : S \to [n]$,则称 $S$ 为\textbf{有限集}。
    \item 
          注意:空集 $S = \varnothing$ 是有限集,因为 $[0] = \varnothing$。
    \item 若 $S$ \emph{不是有限集},则称 $S$ 为\textbf{无限集};换言之,若 $\forall n \in \mathbb{N} \cup \{0\}$,则任何函数 $f : S \to [n]$ 都不是双射。
    \item 若存在双射 $f : S \to \mathbb{N}$,则称 $S$ 为\textbf{可数无限集}(或\textbf{可数集})。
    \item 若任何函数 $f : S \to \mathbb{N}$ 都不是双射,则称 $S$ 为\textbf{不可数无限集}(或\textbf{不可数集})。
    \item 我们用 $|S|$ 表示 $S$ 的\textbf{基数}。
          \begin{itemize}
            \item 若 $S$ 为有限集,即存在 $n \in \mathbb{N} \cup \{0\}$ 及双射 $f : S \to [n]$,则记 $|S| = n$,并称 $S$ 有 $n$ 个元素,$n$ 称为 $S$ 的大小。
            \item 若 $S$ 为无限集,则我们仅通过基数 $|S|$ 与其他集合的基数进行比较。具体而言,不写做 $|S| = \infty$,而写做 $|S| = |T|$ 表示 $S$ 与 $T$ \textbf{基数相同},或写做 $|S| < |T|$ 表示 $T$ 的基数严格大于 $S$ 的基数。
          \end{itemize}
    \item 记 $|S| = |T|$ 并称 $S$ 与 $T$ 基数相同,当且仅当存在双射 $f : S \to T$。
\end{itemize}

\subsection{结论}

一般而言,以下结论成立,其他结论可由这些一般性陈述推导得出。

\begin{itemize}
    \item 若 $|A| = |C|$ 且 $|B| = |D|$,则 $|A \times B| = |C \times D|$。
    \item 若 $|A| = |C|$ 且 $|B| = |D|$,且 $A \cap B = \varnothing$,$C \cap D = \varnothing$,则 $|A \cup B| = |C \cup D|$。
    \item 若存在单射 $f : A \to B$,则 $|A| \le |B|$。
    \item 若存在满射 $f : A \to B$,则 $|A| \ge |B|$。
\end{itemize}

\subsubsection*{有限集}

\begin{itemize}
    \item 若 $A$ 和 $B$ 是有限集,则 $A \cup B$ 也是有限集。
    \item 若 $A$ 和 $B$ 是有限集,且 $A \cap B = \varnothing$,则 $|A + B|= |A| + |B|$。
    \item 若 $A$ 和 $B$ 是有限集,则 $|A \times B|= |A| \cdot |B|$。
\end{itemize}


\subsubsection*{无限集}

\begin{itemize}
    \item 若 $A$ 是可数无限集,$B$ 是有限集或可数无限集,则 $A \cup B$ 是可数无限集。
    \item 若 $A$ 是可数无限集,$B$ 是有限集或可数无限集,则 $A \times B$ 是可数无限集。
    \item 若 $A$ 是不可数无限集,$B$ 是任意集合,则 $A \cup B$ 是不可数无限集。
    \item 若 $A$ 是不可数无限集,$B$ 是任意集合,则 $A \times B$ 是不可数无限集。
    \item 若 $A \subseteq B$,则 $|A| \le |B|$。(注意:这既适用于有限集也适用于无限集。)
    \item 对于任意集合 $A$,都有 $|A| < |\mathcal{P}(A)|$。(注意:这既适用于有限集也适用于无限集。)
    \item 若 $A$ 是无限集,则存在可数无限集 $C \subseteq A$。
    \item $A$ 是无限集 $\iff \exists C \subset A$ 使得存在双射 $f : A \to C$。(注意\emph{真子集}关系。)
    \item 可数无限个可数无限集的并集仍是可数无限集。
    \item 可数无限个有限集的笛卡尔积是不可数无限集。\\
          (请注意,这表明任何非空集合的可数无限笛卡尔积都是不可数无限集。)
    \item \textbf{康托尔-施罗德-伯恩斯坦定理}:若 $A$ 和 $B$ 为集合,且存在单射 $f : A \to B$ 和 $g : B \to A$。则存在双射 $h : A \to B$,且 $|A| = |B|$。
\end{itemize}

\subsection{标准基数类型}

\textbf{有限集}

\begin{multicols}{2}
    \begin{itemize}
        \item $\varnothing$
        \item 对于任意 $n \in \mathbb{N}, [n]$
    \end{itemize}
\end{multicols}

\textbf{可数无限集}

\begin{multicols}{2}
    \begin{itemize}
        \item $\mathbb{N}$
        \item $\mathbb{Z}$
        \item 奇自然数/奇整数,偶自然数/偶整数
        \item $\mathbb{Q}$
        \item $\mathbb{N} \times \mathbb{N}$
        \item 所有有限长二进制字符串的集合
    \end{itemize}
\end{multicols}


\textbf{不可数无限集}

\begin{itemize}
    \item $\mathbb{R}$
    \item $\mathbb{R}$ 的区间;即对于任意 $a, b \in \mathbb{R}, \{y \in \mathbb{R} \mid a \le y \le b\}$。\\
    (注意:区间中的 ``$\le$'' 也可以分别替换为 ``$<$''。)
    \item $\mathcal{P}(\mathbb{N})$
    \item $\mathcal{P}(\mathbb{Z})$
    \item 所有无限长二进制字符串的集合
\end{itemize}
