% !TeX root = ../../book.tex
\section{组合数学}

\subsection{定义}

\begin{itemize}
    \item 集合 $[n]$ 的\textbf{全排列}为双射 $f : [n] \to [n]$。
    \item 集合 $[n]$ 的 $k$-\textbf{选择}为子集 $S \subseteq [n]$,其中 $|S| = k$。
    \item 集合 $[n]$ 的\textbf{可重复 $k$-选择}为 $[n]$ 中 $k$ 个元素的无序列表,其中元素可以重复。
    \item 集合 $[n]$ 的\textbf{可重复 $k$-排列}为 $[n]$ 中 $k$ 个元素的有序列表,其中元素可以重复。
\end{itemize}

\subsection{计数原理}

\begin{itemize}
    \item \textbf{加法原理}:设 $A$ 为有限集。设 $n \in \mathbb{N}$。假设 $\{S_i \mid i \in [n]\}$ 是 $A$ 的一个划分。那么
          \[|A| = \sum_{i=1}^n |S_i| = |S_1| + |S_2| + \dots + |S_n|\]
    \item \textbf{乘法原理}:假设我们有一个过程,分 $n$ 步完成。假设第 $i$ 步(其中 $1 \le i \le n$)可以通过 $w_i$ 种方式完成,与上一步中的选择无关。那么这个过程的结果数为
          \[\prod_{i=1}^n w_i = w_1 \cdot w_2 \cdot\dots\cdot w_n\]
\end{itemize}


\subsection{公式}

\begin{itemize}
    \item $[n]$ 有 $n!$ 种全排列。
    \item $[n]$ 上有 ${n \choose k} = \frac{n!}{k!(n-k)!}$ 种 $k$-选择。
    \item $[n]$ 上有 ${n \choose k}k! = \frac{n!}{(n-k)!}$ 种 $k$-排列。
    \item $[n]$ 上有 ${k+n-1 \choose k}$ 种重复 $k$-选择。
    \item $[n]$ 上有 $n^k$ 种重复 $k$-排列。
\end{itemize}

\subsection{标准计数对象}

\begin{itemize}
    \item \textbf{扑克牌}:一副标准的扑克牌有 $52$ 张牌。每张牌都有花色($\heartsuit$ 或 $\diamondsuit$ 或 $\clubsuit$ 或 $\spadesuit$)和点数(\verb|2| 或 \verb|3| 或 \verb|4| 或 …… 或 \verb|10| 或 \verb|J| 或 \verb|Q| 或 \verb|K| 或 \verb|A|)。
    \item \textbf{元组}:设 $k, n \in \mathbb{N}$。集合 $T_{n,k}$ 是来自 $[k]$ 的所有 $n$ 元组的集合。也就是说,它是所有长度为 $n$ 的有序序列的集合,其坐标是 $[k]$ 的元素。
    \item \textbf{单词}:这相当于元组,其中 $[k]$ 代表字母表,$n$ 代表单词的长度。
    \item \textbf{格路径}:设 $x, y \in \mathbb{N}$。到 $(x, y)$ 的格路径是平面上自然数格点的序列,从 $(0, 0)$ 开始,到 $(x, y)$ 结束,其中每次移动要么向右要么向上。

          到 $(x, y)$ 的格路径有 ${x+y \choose x} = {x+y \choose y}$ 条。
\end{itemize}


\subsection{双法计数}

这是使用组合学论证证明恒等式的标准方法。

\textbf{方法大纲:}

\begin{enumerate}
    \item 陈述要证明的结果。注意:记得量化出现在表达式中的任何变量!
    \item 定义要计数的对象集合(我们称之为 $S$)。
    \item 遵循适当的组合论证,以一种方式计数 $S$ 的元素。推导出的表达式等于 $|S|$。
    \item 遵循适当的组合论据,以另一种方式计数 $S$ 的元素。推导出的表达式等于 $|S|$。
    \item 得出结论,因为两个表达式都等于 $|S|$,所以它们必然相等。
\end{enumerate}


\subsection{结论}

以下是本书中通过两法计数证明的结论。(你可以在没有证明的情况下引用这些结论,但记住计数论证的主要思想也很有帮助,这样你就可以重构公式而不必死记硬背。)

\begin{itemize}
    \item \textbf{帕斯卡恒等式}:${n \choose k}+{n \choose k+1}={n+1 \choose k+1}$
    \item \textbf{Chairperson 恒等式}:${n \choose k}\cdot k = n \cdot {n-1 \choose k-1}$
    \item \textbf{二项式定理}:$(x+y)^n = \sum_{k=0}^n {n \choose k}x^ky^{n-k}$
    \item \textbf{求和恒等式}:$\sum_{i=k}^n {i \choose k}={n+1 \choose k+1}$
\end{itemize}

\subsection{容斥原理}

假设我们有全集 $U$ 和其子集 $A_1,A_2,\dots, A_n \subseteq U$。我们想要计算 $U$ 中所有 $A_i$ 集合\emph{之外}的元素。
\begin{align*}
    |U-A_1|=                     & |U|-|A_1|                                     \\
    |U-(A_1 \cup A_2) |=         & |U| - |A_1| - |A_2| + |A_1 \cap A_2|          \\
    |U-(A_1 \cup A_2 \cup A_3)|= & |U| - |A_1| - |A_2| - |A_3|                   \\
                                 & +|A_1 \cap A_2|+|A_1 \cap A_3|+|A_2 \cap A_3| \\
                                 & -|A_1 \cap A_2 \cap A_3|
\end{align*}
依此类推。

一般来说,对于 $n$ 个集合,我们有
\[|U-(A_1 \cup A_2 \cup \dots \cup A_n) |= \sum_{S \subseteq [n]}(-1)^{|S|} |\bigcap_{i \in S} A_i|\]
其中 $\bigcap_{i \in \varnothing} A_i = U$

$k$ 个集合的交集\emph{大小}仅取决于值 $k$(而不是相交的集合),那么我们可以写成
\[|U-(A_1 \cup A_2 \cup \dots \cup A_n)|= \sum_{k=0}^n (-1)^k {n \choose k} |S_1 \cap S_2 \cap \dots \cap S_k|\]

\subsection{抽屉原理}

如果集合 $S$ 的基数 $|S|= n$,被划分为 $k$ 个不相交的子集,且并集为 $S$,如果 $k < n$,则划分中至少有一个子集存在一个以上元素。此外,该部分实际上至少具有 $\lceil \frac{n}{k} \rceil$ 个元素。

(也就是说,如果我们将 $n$ 个对象分成 $k$ 堆,则必然有一堆中至少有 $\frac{n}{k}$ 个对象。)
