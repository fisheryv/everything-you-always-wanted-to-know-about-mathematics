% !TeX root = ../../book.tex
\section{组合数学}

\subsection{定义}

\begin{itemize}
    \item 集合 $[n]$ 的\textbf{全排列}为双射 $f : [n] \to [n]$。
    \item 集合 $[n]$ 的 $k$-\textbf{选择}为子集 $S \subseteq [n]$,其中 $|S| = k$。
    \item 集合 $[n]$ 的\textbf{可重复 $k$-选择}为 $[n]$ 中 $k$ 个元素的无序序列,其中元素可以重复。
    \item 集合 $[n]$ 的\textbf{可重复 $k$-排列}为 $[n]$ 中 $k$ 个元素的有序序列,其中元素可以重复。
\end{itemize}

\subsection{计数原理}

\begin{itemize}
    \item \textbf{加法原理}:设 $A$ 为有限集,$n \in \mathbb{N}$。若 $\{S_i \mid i \in [n]\}$ 是 $A$ 的一个划分,则
          \[|A| = \sum_{i=1}^n |S_i| = |S_1| + |S_2| + \dots + |S_n|\]
    \item \textbf{乘法原理}:考虑一个分 $n$ 步完成的过程。若第 $i$ 步($1 \le i \le n$)有 $w_i$ 种完成方式,且各步的选择相互独立,则整个过程的总方式数为
          \[\prod_{i=1}^n w_i = w_1 \cdot w_2 \cdot \dots \cdot w_n\]
\end{itemize}


\subsection{公式}

\begin{itemize}
    \item $[n]$ 的全排列数为 $n!$。
    \item $[n]$ 的 $k$-选择数为 $\displaystyle {n \choose k} = \frac{n!}{k!(n-k)!}$。
    \item $[n]$ 的 $k$-排列数为 $\displaystyle {n \choose k}k! = \frac{n!}{(n-k)!}$。
    \item $[n]$ 的可重复 $k$-选择数为 $\displaystyle {k+n-1 \choose k}$。
    \item $[n]$ 的可重复 $k$-排列数为 $n^k$。
\end{itemize}

\subsection{标准计数对象}

\begin{itemize}
    \item \textbf{扑克牌}:一副标准扑克牌共有 $52$ 张牌。每张牌包含一个花色($\heartsuit$、$\diamondsuit$、$\clubsuit$ 或 $\spadesuit$)和一个点数(\verb|2|、\verb|3|、\verb|4|、……、\verb|10|、\verb|J|、\verb|Q|、\verb|K| 或 \verb|A|)。
    \item \textbf{元组}:设 $k, n \in \mathbb{N}$。集合 $T_{n,k}$ 是由 $[k]$ 中的元素组成的所有 $n$ 元组的集合。换言之,它是所有长度为 $n$ 的有序序列的集合,其中每个坐标取自 $[k]$。
    \item \textbf{单词}:这相当于元组,其中 $[k]$ 代表字母表,$n$ 代表单词的长度。
    \item \textbf{格路径}:设 $x, y \in \mathbb{N}$。从 $(0, 0)$ 到 $(x, y)$ 的格路径是平面上自然数格点的序列,该序列从 $(0, 0)$ 开始,到 $(x, y)$ 结束,且每次移动只能向右或向上。

        从 $(0, 0)$ 到 $(x, y)$ 的格路径总数为 $\displaystyle {x+y \choose x} = {x+y \choose y}$。
\end{itemize}


\subsection{双法计数}

这是使用组合学论证证明恒等式的标准方法。

\textbf{方法大纲:}

\begin{enumerate}
    \item 陈述要证明的结论。注意:务必量化表达式中出现的所有变量!
    \item 定义要计数的对象集合(记作 $S$)。
    \item 通过适当的组合论证,以第一种方式计数 $S$ 中的元素。所得表达式等于 $|S|$。
    \item 通过另一种组合论证,以第二种方式计数 $S$ 中的元素。所得表达式等于 $|S|$。
    \item 得出结论:由于两个表达式均等于 $|S|$,因此它们必然相等。
\end{enumerate}


\subsection{结论}

以下是本书中通过双法计数证明的结论。(你可以在不加证明的情况下引用这些结论,但建议理解计数论证的主要思想,以便重构公式而非死记硬背。)

\begin{itemize}
    \item \textbf{帕斯卡恒等式}:$\displaystyle {n \choose k}+{n \choose k+1}={n+1 \choose k+1}$
    \item \textbf{ 主席恒等式}:$\displaystyle {n \choose k}\cdot k = n \cdot {n-1 \choose k-1}$
    \item \textbf{ 二项式定理}:$\displaystyle (x+y)^n = \sum_{k=0}^n {n \choose k}x^ky^{n-k}$
    \item \textbf{ 求和恒等式}:$\displaystyle \sum_{i=k}^n {i \choose k}={n+1 \choose k+1}$
\end{itemize}

\subsection{抽屉原理}

设集合 $S$ 的基数 $|S|= n$,且被划分成 $k$ 个互不相交的子集,这些子集的并集为 $S$。若 $k < n$,则至少有一个子集包含多于一个元素。进一步,至少有一个子集包含至少 $\lceil \frac{n}{k} \rceil$ 个元素。

(换言之,若将 $n$ 个对象分配到 $k$ 个堆中,则至少有一个堆包含至少 $\lceil \frac{n}{k} \rceil$ 个对象。)


\subsection{容斥原理}

假设全集为 $U$,其子集为 $A_1, A_2, \dots, A_n \subseteq U$。我们希望计算 $U$ 中\emph{不属于}任何 $A_i$ 的元素数量。
\begin{align*}
    |U-A_1|=                     |U|& -|A_1|                                     \\
    |U-(A_1 \cup A_2) |=         |U|&  - |A_1| - |A_2| + |A_1 \cap A_2|          \\
    |U-(A_1 \cup A_2 \cup A_3)|= |U|&  - |A_1| - |A_2| - |A_3|                   \\
                                 & +|A_1 \cap A_2|+|A_1 \cap A_3|+|A_2 \cap A_3| \\
                                 & -|A_1 \cap A_2 \cap A_3|                      \\
    \text{依此类推……}             &
\end{align*}

一般地,对于 $n$ 个集合,我们有
\[\left|U-(A_1 \cup A_2 \cup \dots \cup A_n) \right|= \sum_{S \subseteq [n]}(-1)^{|S|} \left|\bigcap_{i \in S} A_i \right| \quad \text{其中} \enspace \bigcap_{i \in \varnothing} A_i = U\]

如果任意 $k$ 个集合的交集\emph{大小}仅取决于 $k$(而与具体哪些集合相交无关),则公式可简化为
\[\left|U-(A_1 \cup A_2 \cup \dots \cup A_n)\right|= \sum_{k=0}^n (-1)^k {n \choose k} \left|S_1 \cap S_2 \cap \dots \cap S_k \right|\]
