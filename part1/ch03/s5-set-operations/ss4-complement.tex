% !TeX root = ../../../book.tex
\subsection{补集}

此运算识别位于集合``外部''的所有元素。此操作取决于全集 $U$ 的上下文。你会注意到这在定义中很明显,我们也将通过示例来说明这一点。

\begin{definition}
    $A$ 的\dotuline{补集}是所有不是 $A$ 中元素的元素的集合,记为 $\overline{A}$。用数学符号表达如下:
    \[\overline{A} = \{x \in U \mid x \notin A\}\]
\end{definition}

请记住,我们假设 $A,B,U$ 是满足 $A \subseteq U$ 且 $B \subseteq U$ 的给定集合。在这种情况下,集合 $\overline{A}$ 是明确定义的,该集合必定取决于 $A$ 和 $U$!\\

\begin{example}
    例如,让我们回到上面例 \ref{ex:example3.5.1} 中定义的集合 $S_1, S_2, S_3$。在那里,我们使用了上下文 $U = \mathbb{Z}$。在这种情况下,
    \[\overline{S_1} = \{6, 7, 8, 9, \dots \}\]
    然而,如果我们另 $U = \{1, 2, 3, 4, 5, 6, 7\}$,在这种情况下,
    \[\overline{S_1} = \{6, 7\}\]
\end{example}

由于符号 $\overline{A}$ 没有指示它所依赖的全集 $U$,因此无论上下文如何,明确该全集都很重要。尝试给出集合 $A, U_1, U_2$,使得 $U_1$ 下的 $\overline{A}$ 与 $U_2$ 下的 $\overline{A}$ 不同,并尝试给出一些集合,使得两种情况下 $\overline{A}$ 相同。
