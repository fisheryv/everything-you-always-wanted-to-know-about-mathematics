% !TeX root = ../../../book.tex
\subsection{补集}

补集运算识别位于集合``外部''的所有元素,其具体结果依赖于全集 $U$ 的上下文,这一点在定义和后续示例中均有体现。

\begin{definition}
    $A$ 的\dotuline{补集}是所有不属于 $A$ 的元素的集合,记为 $\overline{A}$。用数学符号表示为:
    \[\overline{A} = \{x \in U \mid x \notin A\}\]
\end{definition}

此处假设 $A, B, U$ 均为给定集合,且满足 $A \subseteq U$ 与 $B \subseteq U$。此时 $\overline{A}$ 的定义是明确的,但该集合完全依赖于 $A$ 和 $U$ 的选择。

\begin{example}
    考虑例 \ref{ex:example3.5.1} 中定义的集合 $S_1, S_2, S_3$。当 $U = \mathbb{Z}$ 时,
    \[\overline{S_1} = \{6, 7, 8, 9, \dots \}\]
    若取 $U = \{1, 2, 3, 4, 5, 6, 7\}$,则
    \[\overline{S_1} = \{6, 7\}\]
\end{example}

由于符号 $\overline{A}$ 未显式指明其依赖的全集 $U$ \footnote{补集的另一个常见符号为 $\complement_U A$。该符号显式指明了全集 $U$。当上下文不易明确时,显式指明全集是更理想的做法。},明确上下文至关重要。试构造集合 $A, U_1, U_2$,使得 $\overline{A}$ 在 $U_1$ 和 $U_2$ 下不同;再构造一组集合,使得 $\overline{A}$ 在两种上下文下相同。
