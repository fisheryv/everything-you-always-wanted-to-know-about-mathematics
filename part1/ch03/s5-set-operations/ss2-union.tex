% !TeX root = ../../../book.tex
\subsection{并集}

此运算提取两个集合中的元素并将它们包含在一个新集合中,称为\textbf{并集}。

\begin{definition}
    设 $A, B$ 为任意集合。$A$ 和 $B$ 的\dotuline{并集}是属于 $A$ 或 $B$ 的元素的集合,用 $A \cup B$ 表示。用数学符号表达如下:
    \[A \cup B = \{x \in U \mid x \in A \;\text{或}\; x \in B\}\]
\end{definition}

请注意,定义中的``或''是\emph{包含}``或'',这意味着 $A \cup B$ 包括任何属于 $A$ 或 $B$ 或可能同时属于这两个集合的元素。\\

\begin{example}
    回到我们在例 \ref{ex:example3.5.1} 中定义的集合 $S_1, S_2, S_3$,我们可以说
    \begin{align*}
        S_1 \cup S_2 &= \{1, 2, 3, 4, 5, 7\} \\
        S_1 \cup S_3 &= \{1, 2, 3, 4, 5, 7\} \\
        S_2 \cup S_3 &= \{1, 2, 3, 4, 7\}
    \end{align*}
    此外,由于并集本身也是一个集合,因此可以与其他集合再次进行并集运算,例如
    \[(S_1 \cup S_2) \cup S_3 = \{1, 2, 3, 4, 5, 7\} \cup  \{2, 4, 7\} =  \{1, 2, 3, 4, 5, 7\}\]
\end{example}

\subsubsection*{并集与子集}

请注意,无论 $A$ 和 $B$ 是什么,都有 $A \subseteq (A \cup B)$ 和 $B \subseteq (A \cup B)$。让我们证明一下!

\begin{proposition}
    设 $A, B$ 为任意集合。则 $A \subseteq (A \cup B)$ 且 $B \subseteq (A \cup B)$。
\end{proposition}

\begin{proof}
    假设我们有两个集合 $A$ 和 $B$。为了证明 $A \subseteq (A \cup B)$,我们需要证明 $A$ 的每个元素也是 $A \cup B$ 的元素。

    设任意固定元素 $x \in A$。则必有 $x \in A$ 或 $x \in B$(因为已知 $x \in A$)。这表明 $x \in A \cup B$。由于 $x$ 是任意的,因此我们证明了 $A \subseteq A \cup B$。

    设任意固定元素 $y \in B$。则必有 $y \in A$ 或 $y \in B$(因为已知 $y \in B$)。这表明 $y \in A \cup B$。由于 $y$ 是任意的,因此我们证明了 $B \subseteq A \cup B$。
\end{proof}

你能说一下 $A \cap B$ 和 $A \cup B$ 之间的关系吗?如果 $A \subseteq B$,那么 $B$ 与 $A \cup B$ 之间有什么关系?尝试证明你的观察!

这里需要强调一点,像这样的主张 --- 对于任意集合 $A$ 和 $B$, $A \subseteq A \cup B$ --- 是需要证明的;\textbf{根据定义}它们不是显然成立的。上面给出了两个集合的并集的定义。请注意,它没有说明 $A$ 和 $A \cup B$ 之间的关系;定义只是告诉我们对象 $A \cup B$ 实际上是什么。当你调用或引用定义并使用它时,请务必这样做;而且,一定要解释任何不完全来自定义的主张。既然我们已经证明了这两个小引理,我们就可以在将来通过引用来使用它们;如果我们不这样做,我们每次尝试引用这些小事实时都必须重新解释它们!
