% !TeX root = ../../../book.tex
\subsection{交集}

交集运算提取两个集合共有的元素,并将它们组合成一个新的集合,称为\textbf{交集}。

\begin{definition}
    设 $A, B$ 为任意集合。$A$ 和 $B$ 的\dotuline{交集}是同时属于 $A$ 和 $B$ 的元素的集合,记作 $A \cap B$。用数学符号表示为:
    \[A \cap B = \{x \in U \mid x \in A \text{\ 且\ } x \in B\}\]
\end{definition}

\begin{example}\label{ex:example3.5.1}
    定义如下集合:
    \begin{align*}
        S_1 &= \{1, 2, 3, 4, 5\}\\
        S_2 &= \{1, 3, 7\}\\
        S_3 &= \{2, 4, 7\}\\
        U &= \mathbb{N}
    \end{align*}
    因此,我们有:
    \begin{align*}
        S_1 \cap S_2 &= \{1, 3\} \\
        S_1 \cap S_3 &= \{2, 4\} \\
        S_2 \cap S_3 &= \{7\}
    \end{align*}
    此外,由于交集本身也是一个集合,它可以与其他集合进一步进行交集运算。例如,$(S_1 \cap S_2) \cap S_3$ 是合理的。然而,$(S_1 \cap S_2)$ 和 $S_3$ 没有共同元素,因此:
    \[(S_1 \cap S_2) \cap S_3 = \varnothing\]
\end{example}

如上例所示,两个集合没有共同元素的情况很常见,因此引入特定术语描述此类集合:

\begin{definition}
    如果 $A \cap B = \varnothing$,则称 $A$ 与 $B$ \dotuline{不相交}。
\end{definition}

\subsubsection*{交集与子集}

你可能已经注意到,无论 $A$ 和 $B$ 是什么集合,总有 $A \cap B \subseteq A$ 且 $A \cap B \subseteq B$。现在我们来证明这一事实。

\begin{proposition}
    设 $A, B$ 为任意集合,则 $A \cap B \subseteq A$ 且 $A \cap B \subseteq B$。
\end{proposition}

顺带一提,\textbf{命题}是指``微小的结果''。它可能不像定理那样重要或复杂,但仍需简要证明。

\begin{proof}
    考虑两个集合 $A$ 和 $B$。为证明子集关系(例如 $A \cap B \subseteq A$),需证左边集合 $(A \cap B)$ 的每个\dotuline{元素}也是右边集合 $(A)$ 的元素。

    取任意元素 $x \in A \cap B$。由交集定义可知 $x \in A$ 且 $x \in B$。因此 $x \in A$,这正是所需结论,故 $A \cap B \subseteq A$ 成立。

    同理,由 $x \in B$ 可得 $A \cap B \subseteq B$。
\end{proof}

虽然这是直观的观察和简单的证明,但仍需严格遵循逻辑步骤以确立子集关系。此外,请注意此处的\textbf{证明结构}:为证子集关系,需考察集合的\textbf{任意元素}并推导其属于另一集合。这将作为所有子集命题的标准证明方法。

如果 $A \subseteq B$,那么 $A \cap B$ 与 $A$ 和 $B$ 有何关系?请尝试证明!
