% !TeX root = ../../../book.tex
\subsection{差集}

此运算提取一个集合的元素并删除也属于另一集合的元素。

\begin{definition}
    $A$ 和 $B$ 的差集用 $A - B$ 表示,是 $A$ 中所有不属于 $B$ 的元素的集合。用数学符号表达如下:
    \[A - B := \{x \in U \mid x \in A \;\text{且}\; x \notin B\}\]
\end{definition}

\begin{example}
    回到我们在例 \ref{ex:example3.5.1} 中定义的集合 $S_1, S_2, S_3$,我们可以说
    \begin{align*}
        S_1 - S_2 &= \{2, 4, 5\} \\
        S_1 - S_3 &= \{7\} \\
        S_2 - S_3 &= \{1,3\}
    \end{align*}
\end{example}

\subsubsection*{差集不对称}

请注意,上例中的 $S1 - S2 \ne S2 - S1$。一般来说,在集合的上下文中,运算``$-$''不是对称的,这里的例子表明了这一点。你能找到两个集合 $A, B$ 使得 $A - B = B - A$ 吗?你能找到两个集合 $A, B$ 使得 $A - B = B - A = ∅$ 吗?

到目前为止,我们定义的其他运算实际上都是对称的。也就是说,$A \cap B = B \cap A$ 且 $A \cup B = B \cup A$。回顾一下这些运算的定义,看看为什么这是合理的。定义中哪部分\emph{语言}使得对称性成立?

\subsubsection*{注释}

差集表示法还有一点需要注意。虽然我们使用标准减法符号``$-$'',但这里的减法符号与我们通常认为的``减法''(如数字)毫不相关。这可能是你第一次遇到这种歧义,也可能不是,但请记住这个与数学符号和术语相关的重要观点:许多符号根据\emph{上下文}不同而具有不同的含义。

当我们写 $7 - 5$ 时,我们显然指的是减法,即 $7 - 5 = 2$。然而,当我们写 $A - A$ 其中 $A$ 被识别为\emph{集合}时,我们指的是差集运算,即 $A - A = \varnothing$。请务必检查语句的上下文,以确保其中的符号确实具有你认为的含义!
