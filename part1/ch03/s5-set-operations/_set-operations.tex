% !TeX root = ../../../book.tex
\section{集合运算}\label{sec:section3.5}

当你初次学习数字时,很自然地就会学到如何\emph{组合}它们:乘法、加法等等。因此,我们接下来自然要研究如何将两个集合通过\emph{运算}生成其他集合。我们如何以有趣的方式组合集合?有几种这样的运算具有标准的符号,我们现在就介绍这些运算。

本节中,我们假设给定两个集合 $A$ 和 $B$,它们都是\emph{全集} $U$ 的子集。也就是说,我们假设 $A \subseteq U$ 且 $B \subseteq U$。我们做出这个假设的原因是,每个运算都涉及通过识别具有特定属性的较大集合的元素来定义另一个集合,因此我们必须有一个保证包含 $A$ 和 $B$ 所有元素的集合 $U$,以便我们可以使用这些元素。(再次强调,确保这一点可能看起来很苛刻,但这是为了避免出现像我们之前研究的令人讨厌的悖论。)假设这些集合 $A, B,U$ 存在,我们才可以继续我们的定义。

% !TeX root = ../../../book.tex
\subsection{交集}

此运算提取两个集合共有的元素并将它们包含在一个新集合中,称为\textbf{交集}。

\begin{definition}
    设 $A, B$ 为任意集合。$A$ 和 $B$ 的\dotuline{交集}是同时属于 $A$ 和 $B$ 的元素的集合,用 $A \cap B$ 表示。用数学符号表达如下:
    \[A \cap B = \{x \in U \mid x \in A \;\text{且}\; x \in B\}\]
\end{definition}

\begin{example}\label{ex:example3.5.1}
    定义如下集合:
    \begin{align*}
        S_1 &= \{1, 2, 3, 4, 5\}\\
        S_2 &= \{1, 3, 7\}\\
        S_3 &= \{2, 4, 7\}\\
        U &= \mathbb{N}
    \end{align*}
    那么,我们可得
    \begin{align*}
        S_1 \cap S_2 &= \{1, 3\} \\
        S_1 \cap S_3 &= \{2, 4\} \\
        S_2 \cap S_3 &= \{7\}
    \end{align*}
    此外,由于交集本身也是一个集合,因此可以与其他集合再次进行交集运算,比如 $(S_1 \cap S_2) \cap S_3$ 是有意义的。然而,这两个集合没有共同元素,所以我们可以写做
    \[(S_1 \cap S_2) \cap S_3 = \varnothing\]
\end{example}

如上例所示,两个集合没有共同元素的情况很常见,因此我们有一个特定的术语来描述此类集合:

\begin{definition}
    如果 $A \cap B = \varnothing$,则我们说 $A$ 和 $B$ \dotuline{不相交}。
\end{definition}

\subsubsection*{交集与子集}

你可能已经观察到,无论 $A$ 和 $B$ 是什么,我们都有 $A \cap B \subseteq A$ 且 $A \cap B \subseteq B$。让我们证明这一事实!

\begin{proposition}
    设 $A, B$ 为任意集合。则 $A \cap B \subseteq A$ 且 $A \cap B \subseteq B$。
\end{proposition}

顺带一提,\textbf{命题}只是``微小的结果''。它并不困难或重要到足以被称为定理,但它确实需要一点证明。

\begin{proof}
    假设我们有两个集合,$A$ 和 $B$。为了证明子集关系,例如 $A \cap B \subseteq A$,我们需要证明左边集合 $(A \cap B)$ 的每个\dotuline{元素}也是右边集合 $(A)$ 的元素。

    让我们考虑任意元素 $x \in A \cap B$。根据 $A \cap B$ 的定义,我们知道 $x \in A$ 和 $x \in B$。因此,我们知道 $x \in A$。这就是我们要证明的目标,所以我们证明了 $A \cap B \subseteq A$。

    同理,我们也知道 $x \in B$,因此我们也证明了 $A \cap B \subseteq B$。
\end{proof}

这看起来像是单纯的观察和简单的证明,但我们仍然需要通过这些逻辑步骤来严格解释为什么这些子集关系成立。另外,请注意我们此处使用的\textbf{证明结构}。为了证明子集关系成立,我们需要考虑集合的\textbf{任意元素}并推断它也是另一个集合的元素。这将是我们证明有关子集的任何命题的方法。

如果 $A \subseteq B$ 呢?$A \cap B$ 与 $A$ 和 $B$ 有什么关系?尝试证明这一点!


% !TeX root = ../../../book.tex
\subsection{并集}

并集运算将两个集合的元素组合到一个新集合中,称为\textbf{并集}。

\begin{definition}
    设 $A, B$ 为任意集合。$A$ 和 $B$ 的\dotuline{并集}是由所有属于 $A$ 或 $B$ 的元素组成的集合,记作 $A \cup B$。用数学符号表示为:
    \[A \cup B = \{x \in U \mid x \in A \text{\ 或\ } x \in B\}\]
\end{definition}

请注意,定义中的``或''是\emph{兼}``或'',即 $A \cup B$ 包含所有属于 $A$ 或 $B$ 或可能同时属于这两个集合的元素。

\begin{example}
    考虑例 \ref{ex:example3.5.1} 中定义的集合 $S_1, S_2, S_3$,我们有:
    \begin{align*}
        S_1 \cup S_2 &= \{1, 2, 3, 4, 5, 7\} \\
        S_1 \cup S_3 &= \{1, 2, 3, 4, 5, 7\} \\
        S_2 \cup S_3 &= \{1, 2, 3, 4, 7\}
    \end{align*}
    此外,由于并集本身也是一个集合,因此可以与其他集合再次进行并集运算,例如
    \[(S_1 \cup S_2) \cup S_3 = \{1, 2, 3, 4, 5, 7\} \cup  \{2, 4, 7\} =  \{1, 2, 3, 4, 5, 7\}\]
\end{example}

\subsubsection*{并集与子集}

请注意,对任意集合 $A$ 和 $B$,恒有 $A \subseteq (A \cup B)$ 和 $B \subseteq (A \cup B)$。我们来证明它!

\begin{proposition}
    设 $A, B$ 为任意集合,则 $A \subseteq (A \cup B)$ 且 $B \subseteq (A \cup B)$。
\end{proposition}

\begin{proof}
    设 $A$ 和 $B$ 为任意集合。为证 $A \subseteq(A \cup B)$,需证 $A$ 的每个元素也是 $A \cup B$ 的元素。

    任取 $x \in A$。由 $x \in A$ 可知 $x \in A$ 或 $x \in B$,故 $x \in A \cup B$。由于 $x$ 是任意的,故 $A \subseteq A \cup B$。

    同理,任取 $y \in B$,由 $y \in B$ 可知 $y \in A$ 或 $y \in B$,故 $y \in A \cup B$。由于 $y$ 是任意的,故 $B \subseteq A \cup B$。
\end{proof}

请思考:$A \cap B$ 与 $A \cup B$ 有何关系?若 $A \subseteq B$,则 $B$ 与 $A \cup B$ 有何关联?尝试证明你的结论!

需特别强调:诸如``对任意集合 $A, B$ 有 $A \subseteq A \cup B$''这样的命题——尽管直观——\textbf{仍需严格证明},因其并非\textbf{根据定义}直接可得。并集定义仅说明 $A \cup B$ 的构成,未直接揭示 $A$ 与 $A \cup B$ 的包含关系。引用定义时务必严谨推演,并对非显然结论给出解释。如今我们已证明此结论,后续可直接引用;否则每次使用均需重新论证!


% !TeX root = ../../../book.tex
\subsection{差集}

差集运算从一个集合中提取元素,并移除同时属于另一个集合的元素。

\begin{definition}
    $A$ 和 $B$ 的差集记为 $A - B$,即 $A$ 中所有不属于 $B$ 的元素构成的集合。用数学符号表示为:
    \[A - B := \{x \in U \mid x \in A \text{\ 且\ } x \notin B\}\]
\end{definition}

\begin{example}
    沿用例 \ref{ex:example3.5.1} 中定义的集合 $S_1, S_2, S_3$,可得:
    \begin{align*}
        S_1 - S_2 &= \{2, 4, 5\} \\
        S_1 - S_3 &= \{7\} \\
        S_2 - S_3 &= \{1,3\}
    \end{align*}
\end{example}

\subsubsection*{差集的不对称性}

请注意,上例中 $S_1 - S_2 \ne S_2 - S_1$。一般而言,差集运算不具有对称性。能否找到两个集合 $A$ 和 $B$ 满足 $A - B = B - A$?又能否找到 $A$ 和 $B$ 使得 $A - B = B - A = \varnothing$?

此前定义的其他运算均具有对称性,即 $A \cap B = B \cap A$ 且 $A \cup B = B \cup A$。请回顾其定义,思考对称性成立的原因,并分析定义中的\emph{表述}如何体现这一性质。

\subsubsection*{注释}

差集符号 ``$-$'' 需特别注意:尽管与算术减法符号相同,但二者含义毫不相关。这是数学符号的普遍特性——同一符号在不同\emph{上下文}中可能具有不同含义。

例如,$7 - 5$ 表示数字减法,结果为 $2$;而若 $A$ 为集合,$A - A$ 表示差集运算,结果为 $\varnothing$。理解符号时,务必结合上下文以确认其确切含义!


% !TeX root = ../../../book.tex
\subsection{补集}

补集运算识别位于集合``外部''的所有元素,其具体结果依赖于全集 $U$ 的上下文,这一点在定义和后续示例中均有体现。

\begin{definition}
    $A$ 的\dotuline{补集}是所有不属于 $A$ 的元素的集合,记为 $\overline{A}$。用数学符号表示为:
    \[\overline{A} = \{x \in U \mid x \notin A\}\]
\end{definition}

此处假设 $A, B, U$ 均为给定集合,且满足 $A \subseteq U$ 与 $B \subseteq U$。此时 $\overline{A}$ 的定义是明确的,但该集合完全依赖于 $A$ 和 $U$ 的选择。

\begin{example}
    考虑例 \ref{ex:example3.5.1} 中定义的集合 $S_1, S_2, S_3$。当 $U = \mathbb{Z}$ 时,
    \[\overline{S_1} = \{6, 7, 8, 9, \dots \}\]
    若取 $U = \{1, 2, 3, 4, 5, 6, 7\}$,则
    \[\overline{S_1} = \{6, 7\}\]
\end{example}

由于符号 $\overline{A}$ 未显式指明其依赖的全集 $U$ \footnote{补集的另一个常见符号是 $\complement_U A$。该符号显式指明了全集 $U$。当上下文不明确时,显式指明全集是更理想的符号。——译者注},明确上下文至关重要。试构造集合 $A, U_1, U_2$,使得 $\overline{A}$ 在 $U_1$ 和 $U_2$ 下不同;再构造一组集合,使得 $\overline{A}$ 在两种上下文下相同。


% !TeX root = ../../../book.tex
\subsection{习题}

\subsubsection*{温故知新}

以口头或书面的形式简要回答以下问题。这些问题全都基于你刚刚阅读的内容,如果忘记了具体定义、概念或示例,可以回顾相关内容。确保在继续学习之前能够自信地作答这些问题,这将有助于你的理解和记忆!

\begin{enumerate}[label=(\arabic*)]
    \item 两个集合的并集和交集有何区别?
    \item 两个集合不相交意味着什么?
    \item $\mathbb{Z} \cap \mathbb{N}$ 是什么? $\mathbb{Z} \cup \mathbb{N}$ 是什么? $\mathbb{Z}-\mathbb{N}$ 是什么?
    \item $A - B = B - A$ 可能成立吗?在什么情况下成立?
    \item $\mathbb{N}$ 下的 $\overline{[3]}$ 是什么?若上下文改为 $\mathbb{Z}$ 或 $\mathbb{R}$ 呢?尝试使用恰当的数学符号和集合构建符给出你的答案。
    \item $(A \cap B) \cap C = A \cap (B \cap C)$ 恒成立吗?请说明理由。若将 $\cap$ 代替为 $\cup$ 结果如何?
    \item ``$7-5$'' 与 ``$[7]-[5]$'' 有何区别?
    \item 假设 $x \in A$。表达式 $A - x$ 有意义吗?如何修改才能使其有意义?
    \item $(\mathbb{Z} - \mathbb{N}) \cup \mathbb{R}$ 是什么?
\end{enumerate}

\subsubsection*{小试牛刀}

尝试解答以下问题。这些题目需动笔书写或口头阐述答案,旨在帮助你熟练运用新概念、定义及符号。题目难度适中,确保掌握它们将大有裨益!

\begin{enumerate}[label=(\arabic*)]
    \item 列出下列集合的元素:
        \begin{enumerate}[label=(\alph*)]
            \item $[7] \cup [10]$
            \item $[10] \cap [7]$
            \item $[10] - [7]$
            \item $([12] - [3]) \cap [8]$
            \item $(\mathbb{N} - [3]) \cap [7]$
            \item $(\mathbb{Z}-\mathbb{N}) \cap N$
            \item $\mathbb{Z}$ 下 $\overline{\mathbb{N}} \cap \{0\}$
        \end{enumerate}
    \item 构造集合 $A,B,C$ 的示例,使得 $(A - B) - C = A - (B - C)$ 成立。再构造一个使该等式不成立的例子。
    \item 陈述并证明 $\overline{A}$ 与 $U - A$ 之间的关系。
    \item 设 $A = [12]$, $E$ 为偶数集,$P$ 为质数集。求 $A \cap E$, $A \cap P$ 以及 $(A \cap E) \cap P$。$(A \cap E) \cap P$ 是否等于 $A \cap (E \cap P)$?\\
    设全集 $U = \mathbb{N}$。求 $\overline{A \cap E}$ 和 $\overline{A} \cap \overline{E}$ 是什么?
    \item $ \{1\} \cap \mathcal{P}(\{1\})$ 是什么?
    \item 考虑集合 $\{1\}$ 和 $\{2, 3\}$。比较集合 $\mathcal{P}(\{1\} \cup \{2, 3\})$ 和 $\mathcal{P}(\{1\}) \cup \mathcal{P}(\{2, 3\})$。你注意到了什么?\\
    若将 $\cup$ 替换为 $\cap$,重复上述比较,结果有何不同?\label{exc:exercises3.5.6}
    \item 设 $A, U$ 为集合,并假设 $A \subseteq U$。在全集 $U$ 下,设 $B = \overline{A}$。你认为 $\overline{B}$ 是什么?请解释原因。
\end{enumerate}
