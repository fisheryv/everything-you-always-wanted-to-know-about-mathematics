% !TeX root = ../../../book.tex
\subsection{逻辑且严谨:使用定义}

这里要强调的一点是,当我们从描述性的、``冗长的''和直观的证明过渡到更严格的、数学上正确的和正式撰写的证明时,\textbf{正规定义非常重要}。从根本上讲,它们是必不可少的,因为当我们说``$A \cup B$''时,我们需要知道你确切地知道该符号的含义以及它如何在集合 $A$ 和 $B$ 上运行。

另一个例子,当我们说``证明 $A = B$''时,我们心中有一个非常具体的目标,并且你需要跟我们保持一致。对主要概念有直观理解总是有帮助的---``哦,陈述 $A = B$ 只是意味着 $A$ 和 $B$ 具有相同的元素''---但这\emph{不是}我们想要在严格证明中使用的语言/想法。为了证明 $A = B$ 这样的命题,我们需要\textbf{诉诸}集合上下文中``$=$''的\textbf{定义}:$A = B$ 当且仅当 $A \subseteq B$ 且 $B \subseteq A$。

这就是我们说的``满足定义"或``诉诸定义"的含义:要证明某个数学对象具有某种属性,你必须证明该对象满足该属性的正式定义。如果你不熟悉该定义,或者忘记了如何准确地表述它……无论如何,都应该去了解它!我们知道到需要吸收大量新信息,并且当你对某事还不熟悉时,忘记某些地方也在所难免。通过这样做,你将开始更快、更牢固地消化吸收这些想法。

在下面的示例中,你将看到我们如何使用 ``$\subseteq$''、``$=$'' 和 ``$\cap$'' 等定义。对于每个命题/引理,我们最终都会撰写一个正式的证明,但我们也会写出如何\emph{提出这样一个证明}的内容。通常这才是困难的部分!我们认为你会注意到,其中许多解释只是回忆相关定义并思考它的含义以及它在给定情况下应该如何应用。某种程度上,这就是数学。我们只是让我们使用的定义变得越来越复杂。
