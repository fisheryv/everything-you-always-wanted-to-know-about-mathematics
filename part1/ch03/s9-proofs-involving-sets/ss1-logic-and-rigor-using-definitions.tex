% !TeX root = ../../../book.tex
\subsection{逻辑且严谨:使用定义}

这里需要强调:当我们从描述性、直观的``非正式''证明转向严谨、数学正确的正式证明时,\textbf{正确定义至关重要}。根本原因在于,当我们使用符号如``$A \cup B$''时,必须明确知晓其精确含义及在集合 $A$ 和 $B$ 上的运算规则。

再以证明``$A = B$''为例,我们对此有明确目标,且需要你遵循相同的逻辑路径。对核心概念的直观理解固然有益——例如``$A = B$ 意味着 $A$ 和 $B$ 包含相同元素''——但这\emph{并非}严格证明应采用的语言或思路。要证明此类命题,必须\textbf{依据}集合论中``$=$''的\textbf{定义}:$A = B$ 当且仅当 $A \subseteq B$ 且 $B \subseteq A$。

这就是``满足定义''或``依据定义''的内涵:要证明某个数学对象具有特定性质,必须证明其符合该性质的正式定义。若不熟悉或遗忘了定义表述,务必及时回顾!我们理解新概念吸收的难度,初期遗忘在所难免。通过持续练习,你将逐步内化这些思想。

以下示例将展示如何运用``$\subseteq$''、``$=$''及``$\cap$''等定义。针对每个命题/引理,我们将提供完整证明,并阐述\emph{证明思路的构建过程}——这往往是最具挑战性的环节!你会发现,许多思路源于回溯相关定义,并思考其在具体情境中的应用。某种意义上,这正是数学的本质:我们通过日益复杂的定义不断拓展认知疆域。
