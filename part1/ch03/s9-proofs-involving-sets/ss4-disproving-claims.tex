% !TeX root = ../../../book.tex
\subsection{证伪}

\subsubsection*{举例说明}

考虑如下命题:
\begin{center}
    对于任意集合 $F, G, H$,若 $F \subseteq G \cup H$,则要么 $F \subseteq G$ 要么 $F \subseteq H$。
\end{center}

这个命题成立吗?如果成立,应该如何证明?取任意固定元素 $x \in F$。由于 $F \subseteq G \cup H$,可知 $x \in G \cup H$,进而 $x \in G$ 或 $x \in H$。以上推理看似正确,但证明完成了吗?

需要指出的是这是不成立的!关键在于尚未满足``$\subseteq$''的定义。若要证明``$F \subseteq G$ 或 $F \subseteq H$'',必须得出以下结论之一:$F$ 的\emph{所有}元素都属于 $G$,或 $F$ 的\emph{所有}元素都属于 $H$。

虽然 $F$ 的每个元素确实属于 $G$ 或 $H$,但无法保证所有元素都同时属于 $G$ 或同时属于 $H$。请重读最后两段以确保理解此逻辑。人们很容易为这个命题写出``证据'',却未意识到其中的关键错误!

\subsubsection*{定位错误}

识别此类错误是需要培养的重要能力,这将在多方面助益学习。你会注意到,许多练习(目前已有部分,后续会有更多)要求你指出某些命题``证明''中的缺陷。通过指正缺陷,可以帮助你构建正确的证明(或根据情况提供多个证明)。审阅存在逻辑错误、事实偏差或表述不清的证明是项基本技能。更重要的是,仔细研读他人的证明将有助于你成为更严谨的读者,助力发现类似前文的潜在错误。若你此前未能察觉,不必担心;现在理解后,未来便会警惕同类错误。这项能力需要持续训练,相信读完本书时,你将成为出色的数学证明读者与作者。

\subsubsection*{反例}

那么,现在该如何应对?我们刚刚发现上述``证明''无效。这是否意味着命题本身错误?其实,这表明当前的证明尝试未能成功。或许换一种逻辑路径能引导我们得出最终结论。

另一种可能是该命题确实不成立。如何证明这一点?审视其逻辑结构:它声称某个结论对所有集合 $F,G,H$ 成立。具体而言,它断言只要 $F \subseteq G \cup H$,则必有 $F \subseteq G$ 或 $F \subseteq H$。要证伪此命题,只需构造一个\textbf{反例}即可。

我们将在后续章节形式化这些逻辑概念,目前你只需理解:\textbf{反例}是一个具体详实的实例,用于证明``所有……都……''或``任意……皆……''形式的命题实际上并\emph{不}适用于所有情况。它通过展示其中存在\textbf{不具备}特定属性的对象,从而\textbf{推翻}关于整体的断言。

\subsubsection*{示例}

下面通过具体反例解决前述命题:

\begin{center}
    对于任意集合 $F, G, H$,若 $F \subseteq G \cup H$,则要么 $F \subseteq G$ 要么 $F \subseteq H$。
\end{center}

由于命题声称对所有集合成立,构造反例时必须\emph{明确定义}三个具体集合。不能仅描述可能存在满足条件的集合,而应精确定义它们。但直接给出定义并不现实——这正是关键所在。

我们需要探索集合应满足的性质:确保前提 $F \subseteq G \cup H$ 成立,而结论 $F \subseteq G$ 或 $F \subseteq H$ 不成立。这意味着需要同时满足 $F \nsubseteq G$ 和 $F \nsubseteq H$。(关于\textbf{逻辑否定}的形式化定义将在下一章展开;此处你可以依据日常逻辑进行理解。)

我们现在有一个具体的目标:找到满足以下三个条件的三个集合 $F,G,H$:
\begin{align*}
    F &\subseteq G \cup H \\
    F &\nsubseteq G \\
    F &\nsubseteq H
\end{align*}

接下来需明确``$\nsubseteq$''的含义。已知``$\subseteq$''的定义,其否定形式是什么?若 $F \subseteq G$ 成立,则 $F$ 中的每个元素必然也是 $G$ 的元素;若不成立,则 $F$ 中至少有一个元素不属于 $G$。类似的思路对于 $F \nsubseteq H$ 也适用。现在,可以通过定义重新表述目标:
\begin{align*}
    &F \text{\ 的每个元素都属于 } G \text{\ 或\ } H \\
    &F \text{\ 中至少有一个元素不属于\ } G  \\
    &F \text{\ 中至少有一个元素不属于\ } H
\end{align*}

这对构造反例至关重要!我们总结了所有关键条件并以更直观的方式重述了性质。剩余的工作就是在草稿纸上尝试各种不同的构造。一种方法是绘制 $F, G, H$ 的``空白''维恩图,标注潜在``重叠''区域,再填充元素以满足上述条件。

第一个条件要求 $F$ 完全``包含于'' $G$ 和 $H$ 的并集中;而第二、三条件要求 $F$ 中存在一个不属于 $G$ 的元素,以及另一个不属于 $H$ 的元素。这正是我们的思路!此例虽简单,但它是\emph{有效}的反例。现在正式给出证明:

\begin{proof}
    以下命题为假:
    \begin{center}
        对于任意集合 $F, G, H$,若 $F \subseteq G \cup H$,则要么 $F \subseteq G$ 要么 $F \subseteq H$。
    \end{center}

    我们用反例来证伪该命题。

    定义 $F = \{1, 2\}, G = \{1\}, H = \{2\}$。

    注意到 $G \cup H = \{1, 2\}$。由于 $F = G \cup H$,显然 $F \subseteq G \cup H$。因此,该命题的前提成立。

    然而,$2 \in F$ 但 $2 \notin G$。因此,$F \nsubseteq G$。

    同理,$1 \in F$ 但 $1 \notin H$。因此,$F \nsubseteq H$。

    综上,该命题不成立。
\end{proof}

该示例给了我们如下重要启示:

\begin{center}
    反例无需复杂或新奇,也不必涵盖所有可能情形。\\
    只需找到一个有效反例,并清晰展示其运作机制。
\end{center}

以上证明已达成此目标:我们明确定义了关键对象(集合 $F,G,H$),并详述了其相关性质。不要求读者自行验证反例,而是直接展示细节;不抽象论证存在性,而是构造具体实例。

这一方法至关重要,建议你构造的反例采用类似的证明结构。构造反例的核心工作常在证明前于``幕后''完成,然而一旦确定,便可如上清晰呈现。
