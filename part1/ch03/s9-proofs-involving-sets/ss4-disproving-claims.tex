% !TeX root = ../../../book.tex
\subsection{证伪}

\subsubsection*{举个例子}

考虑如下命题:
\begin{center}
    对于任意集合 $F, G, H$,如果 $F \subseteq G \cup H$,则要么 $F \subseteq G$ 要么 $F \subseteq H$。
\end{center}

这种说法成立吗?如果成立,我们该如何证明呢?我们取任意固定元素 $x \in F$。由于 $F \subseteq G \cup H$,这告诉我们 $x \in G \cup H$。 相应地,$x \in G$ 或 $x \in H$。这都没错吧?我们的证明完成了吗?

我们希望你能看出来这是行不通的!特别是,我们最后还没有满足 ``$\subseteq$'' 的定义。如果我们的目标是证明 ``$F \subseteq G$ 或 $F \subseteq H$'',那么我们应该得出结论,其中一个或另一个成立:即 $F$ 的\emph{每个}元素都是 $G$ 的元素,或者 $F$ 的\emph{每个}元素都是 $H$ 的元素。

我们发现 $F$ 的每个元素本身要么是 $G$ 的元素,要么是 $H$ 的元素,但我们不能确定 $F$ 的所有元素都是 $G$ 的元素或都是 $H$ 的元素。再次通读最后两段,以确保你跟上了逻辑。可能很容易为这个命题写出一个``证据'',却没有意识到你迈出了错误的一步!

\subsubsection*{定位错误}

这种对错误的识别是我们要发展的技能之一,它将在多个方面提供帮助。你会注意到,许多练习(到目前为止有一些,但随着我们继续深入,会有更多)要求你找出某些主张``证明''中的缺陷。通过指出存在缺陷,从而帮助你获得正确的证明(或多个证明,视情况而定)。阅读在逻辑、事实和清晰度上有误的证明是一项基本技能。更重要的是,仔细阅读别人的成果必然会让你成为一个对自己的成果更加挑剔的读者,并且会帮助你发现像前面段落中那样的潜在错误。如果你没有抓住它,请不要担心;既然你已经看到了它,你将来就会留意类似的错误!正如我们所说,这项技能是不断发展起来的,到读完本书时,你将成为数学证明的出色读者和作者。

\subsubsection*{反例}

那么,现在我们该怎么办?我们刚刚意识到我们上面的``证明''不起作用。这是否意味着该说法实际上是错误的?实际上,这一切意味着(到目前为止)我们的证明尝试失败了。也许其他一些逻辑路线会神奇地把我们带到难以捉摸的结论。

或者,也许这个说法确实是错误的。我们怎样才能证明这一点?考虑一下该命题的逻辑形式:它说某些陈述对于任意集合 $F,G,H$ 都成立。它说假设 $F \subseteq G \cup H$ 总是必然意味着 $F \subseteq G$ 或 $F \subseteq H$。要证明这并不总是成立,我们只需要找到所谓的\textbf{反例}即可。

我们将在下一章形式化逻辑时再次讨论所有这些想法,但现在你需要知道的是:\textbf{反例}是一个具体的、详细的、描述性的例子,它说明了关于``每个……''或``任意……''或``皆可能……''实际上并\emph{不}适用于所有情况。反例相当于通过展示该类中\emph{不具有}该属性的一个对象来\textbf{反驳}整个类对象具有某种属性的陈述。

\subsubsection*{示例}

让我们看一下寻找和陈述反例的过程如何解决我们上面的例子。\\

\begin{example}
    对于任意集合 $F, G, H$,如果 $F \subseteq G \cup H$,则要么 $F \subseteq G$ 要么 $F \subseteq H$。
\end{example}

这个命题应该适用于任意集合 $F,G,H$,因此当我们描述反例时,我们最好\emph{准确地}描述这三个集合是什么。我们不能只是解释解决这个问题的方法并讨论如何可能存在具有特定属性的三个集合。我们必须通过明确定义它们来告诉读者它们到底是什么。这就是我们反驳这一主张的第一行,但我们不能直接跳到这一点,因为我们还不知道如何定义它们!

这就是工作或乐趣所在:我们需要尝试这些集合所需的属性来帮助我们想出一个例子。回想一下,我们希望这些集合满足某些属性:我们应该确保假设 $F \subseteq G \cup H$ 成立,但我们希望结论 --- $F \subseteq G$ 或 $F \subseteq H$ --- 为假。

这是意味着什么呢?好吧,我们认为你会同意,从逻辑上讲,该陈述的``相反''或``否定''是 ``$F \nsubseteq G$'' 和 ``$F \nsubseteq H$''。(\textbf{逻辑否定}的概念将在下一章中再次出现;目前,我们认为你可以通过应用指导日常生活的逻辑原则来理解它。很快,我们将正式化这一想法。)

我们现在有一个具体的目标:找到满足以下所有三个条件的三个集合 $F,G,H$:

\begin{align*}
    F &\subseteq G \cup H \\
    F &\nsubseteq G \\
    F &\nsubseteq H \\
\end{align*}

接下来需要考虑的一件事是 ``$\nsubseteq$'' 的含义。我们有 ``$\subseteq$'' 的定义,那么它的``相反''或``否定''是什么?为了使 $F \subseteq G$ 成立,我们要求 $F$ 的每个元素也是 $G$ 的元素;因此,如果不成立,那么 $F$ 中至少有一个元素不是 $G$ 的元素。同理也适用于 $F \nsubseteq H$。现在,我们可以通过应用定义以一种有用的方式重申我们的目标:

\begin{align*}
    &F \text{的每个元素都是} G \text{的元素或} H \text{的元素} \\
    &F \text{中至少有一个元素不是} G \text{的元素} \\
    &F \text{中至少有一个元素不是} H \text{的元素} \\
\end{align*}
这对于最终找到我们的反例非常有帮助!我们总结了声明的所有基本部分,并以更直观的方式重述了这些属性。剩下的工作就是在草稿纸上写写画画,看看我们能想出什么。一种方法是为 $F, G$ 和 $H$ 及其潜在的``重叠''绘制一种``空''维恩图,然后填充足够的元素以满足上述三个属性。

第一个条件要求集合 $F$ 完全``位于'' $G$ 和 $H$ 之内;但是,第二个和第三个条件要求存在 $F$ 的两个元素,其中一个不是 $G$ 的元素,另一个不是 $H$ 的元素。这就是我们要做的!你可能会说这是一个简单的例子,但我们说这是一个\emph{有效的}例子。现在让我们开始写下我们的反驳:

\begin{proof}
    下面声明为假:
    \begin{center}
        对于任意集合 $F, G, H$,如果 $F \subseteq G \cup H$,则要么 $F \subseteq G$ 要么 $F \subseteq H$。
    \end{center}
    我们将用反例来反驳该观点。

    定义 $F = \{1, 2\}, G = \{1\}, H = \{2\}$。

    请注意 $G \cup H = \{1, 2\}$。由于 $F = G \cup H$,那么当然 $F \subseteq G \cup H$。因此,该主张的假设成立。

    然而,请注意 $2 \in F$ 但 $2 \notin G$。因此,$F \nsubseteq G$。

    同样,请注意 $1 \in F$ 但 $1 \notin H$。因此,$F \nsubseteq H$。

    因此,该主张为假。
\end{proof}

该示例的一个重要教训如下:

\begin{center}
    寻找反例不一定是最有趣或最复杂的,你也不需要以某种方式描述所有可能的反例。我们只需要找到一个反例,我们需要看看它是如何运作的。
\end{center}

就是这样!这正是我们在上面的证明中所做的:我们定义了所有重要的对象(三个集合 $F,G,H$),然后指出并描述了它们具有的所有相关属性。我们并没有让读者检查反例是否有效;我们向他们展示了细节。我们并没有争论宇宙中某个地方存在这样的集合;我们只是认为宇宙中存在这样的集合。我们明确地定义了它们。

这很重要,我们希望你的反例具有与我们上面类似的证明结构。当你尝试给出反例时,大部分工作将在证明开始之前``在幕后''进行。不过,一旦你找到了反例,就像我们一样把它写出来。
