% !TeX root = ../../../book.tex
\subsection{证明``$=$''}

\subsubsection*{双重包含证明}

我们需要再次回顾在集合上下文中``$=$''的定义,因为后续会频繁使用它。

\begin{definition}
    集合 $A$ 和 $B$ 相等,记作 $A = B$,当且仅当 $A \subseteq B$ 且 $B \subseteq A$。
\end{definition}

这一定义完全基于先前定义的``$\subseteq$''(因为``$\supseteq$''的定义等价)。因此,它并非新技术,而是先前技术的重复应用。具体而言,要证明 $A = B$,只需使用上一小节的技术证明 $A \subseteq B$,再证明 $B \subseteq A$。

事实上,这项技术非常常见,因此有专门的名称:\textbf{双重包含}。当证明两个集合彼此是子集并由此得出它们相等时,称为\textbf{双重包含证明}。

\subsubsection*{示例}

让我们看一下双重包含技术的实际应用案例。

\begin{lemma}
    设 $A$ 和 $B$ 为任意集合,则 $A - (A \cap B) = A - B$。
\end{lemma}

\emph{直观理解}:虽然维恩图可以帮助理解这一事实,但它不能替代证明。我们将采用双重包含证明。取元素 $x \in A - (A \cap B)$ 时,可依次应用``$-$''和``$\cap$''的定义,推导出 $x \in A - B$。类似地,取 $y \in A - B$ 时,应用定义可得出 $y \in A - (A \cap B)$。建议读者先尝试自行证明,再参考以下证明过程。

\begin{proof}
    通过双重包含证明 $A - (A \cap B) = A - B$。

    (``$\subseteq$'')设 $x \in A - (A \cap B)$ 为任意固定元素。根据差集定义,$x \in A$ 且 $x \notin A \cap B$。这意味着 $x$ 不可能同时属于 $A$ 和 $B$。已知 $x \in A$,故 $x \notin B$。因此,$x \in A$ 且 $x \notin B$,由差集定义得 $x \in A-B$。这表明 $A - (A \cap B) \subseteq A - B$。

    (``$\supseteq$'')设 $y \in A - B$ 为任意固定元素。根据差集定义,$y \in A$ 且 $y \notin B$。由于 $y \notin B$,$y$ 不可能同时属于 $A$ 和 $B$。由交集定义,$y \notin A \cap B$。结合 $y \in A$ 和 $y \notin A \cap B$,得 $y \in A - (A \cap B)$。这表明 $A - B \subseteq A - (A \cap B)$。

    综上,利用双重包含证明,$A - (A \cap B) = A - B$ 得证。
\end{proof}

纵观上述证明的整体结构,我们看到它分为两部分,因为它是一个双重包含证明。我们\emph{友好地提前}向读者指出了这一点,并将这两个部分明确分开。从技术上讲,忽略这一提示直接深入证明是可行的,但这可能会令读者感到困惑。证明的全部意义在于\emph{使他人信服}你所理解的事实,因此应尽可能让读者轻松理解你的思路。

现在看另一个证明集合相等的例子。这个例子有所不同,因为双重包含的一部分利用了补集操作。作为预习,请思考为何命题 $A \subseteq B$ 与 $\overline{B} \subseteq \overline{A}$ 是\emph{等价的}(假设存在全集 $U$ 满足 $A, B \subseteq U$)。尝试绘制维恩图、举例或直接证明。

\begin{proposition}
    \[\Big\{x \in \mathbb{N} \mid x + \frac{8}{x} \le 6\Big\} = \{2, 3, 4\}\]
\end{proposition}

\begin{proof}
    设 $A = \Big\{x \in \mathbb{N} \mid x + \frac{8}{x} \le 6\Big\}$, $B = \{2, 3, 4\}$。要证明 $A = B$,需要证明 $A \subseteq B$ 且 $B \subseteq A$。

    首先证明 $B \subseteq A$。逐一验证 $B$ 的元素是否满足 $A$ 的不等式:
    \begin{align*}
        2 + \frac{8}{2} &= 6 \le 6 \\
        3 + \frac{8}{3} &= \frac{17}{3} \le 6 \\
        4 + \frac{8}{4} &= 6 \le 6
    \end{align*}
    因为 $2,3,4 \in \mathbb{N}$,我们推导出 $2,3,4 \in A$,从而 $B \subseteq A$。

    接着证明 $A \subseteq B$。我们将证明 $\overline{B} \subseteq \overline{A}$,其中补集以 $\mathbb{N}$ 为全集。也就是说,我们要证明自然数 $1,5,6,7,\dots$ \dotuline{不}属于 $A$。

    为此,验证这些元素均\dotuline{不}满足 $A$ 的不等式定义:

    前两种情况很容易验证:
    \begin{align*}
        1 + \frac{8}{1} &= 9 \nleq 6 \\
        5 + \frac{8}{5} &= \frac{33}{5} \nleq 6
    \end{align*}

    对于其他情况,取任意固定元素 $x \in \mathbb{N}$ 且 $x \ge 6$,有 $x + \frac{8}{x} \ge 6 + \frac{8}{x}$。由于 $\frac{8}{x} > 0$,故 $x + \frac{8}{x} > 6$。

    这表明只有 $2,3,4$ 满足 $A$ 的不等式定义。

    综上,利用双重包含证明,$A = B$ 得证。
\end{proof}

思考为何证明后半部分的方法有效。(这是条件命题的\textbf{逆否}形式,后续章节讨论逻辑时将详细说明。)

让我们来看另一个证明集合相等的例子。这个例子略有不同,因为我们要证明某个集合是空集,为此需要证明它不包含任何元素。

\begin{proposition}
    对于每个 $n \in \mathbb{N}$,定义 $S_n = \mathbb{N}-[n]$。则
    \[\bigcap_{n \in \mathbb{N}}S_n = \varnothing\]
\end{proposition}

如果不理解上面等式的含义,建议尝试几个例子。例如,查看集合 $S_1$、$S_1 \cap S_2$、$S_1 \cap S_2 \cap S_3$ 的元素等等。可以先找出 $\bigcap_{n \in \mathbb{N}}S_n$ 的候选元素,再解释为什么它不属于该集合。之后尝试写出一个正式的证明;可以参考下面的证明过程!

\begin{proof}
    设 $T = \bigcap_{n \in \mathbb{N}}S_n$ 以便后续引用。

    要证明 $T = \varnothing$,需要证明 $T$ 不包含任何元素。注意 $T$ 是多个自然数集合的交集,因此其元素只能是自然数。

    考虑任意固定元素 $x \in \mathbb{N}$,我们需要证明 $x \notin T$。

    已知 $x \in [x] = \{1,2,\dots, X\}$,因此根据``$-$''的定义,$x \notin \mathbb{N}-[x]$。

    根据定义,$T$ 的元素必须属于每个 $\mathbb{N} - [n]$ 形式的集合。我们已经(至少)确定了交集中的一个集合 $\mathbb{N} - [x]$,使得 $x$ 不属于该集合。因此,$x$ 不可能是 $T$ 的元素,因为它不属于所有此类集合,所以 $x \notin T$。

    因为 $x \in \mathbb{N}$ 是任意的,我们证明了 $T$ 的元素中不包含自然数,故 $T$ 为空集。
\end{proof}

\emph{总结}:让我们再解释一下这个证明方法为何有效。我们证明了 $T$ 中没有元素,即 $T \subseteq \varnothing$。这就完成了论证,因为 $\varnothing \subseteq T$ 对任何集合恒成立。因此双重包含论证的条件已满足,可以得出 $T = \varnothing$ 的结论。

让我们再举一个例子。这个例子为我们提供了练习索引集操作的机会,本节习题中还有许多类似问题,我们鼓励你尽可能多尝试解答!

\begin{proposition}
    对于每个 $n \in \mathbb{N}$,定义 $A_n = \{x \in \mathbb{R} \mid 0 \le x < \frac{1}{n}\}$。则
    \[\bigcap_{n \in \mathbb{N}}A_n = \{0\}\]
\end{proposition}

思考上述命题的含义:在数轴上画出 $A_n$ 的图示,交集符号``$\cap$''表示什么?为什么 $0$ 属于该交集?为什么它是\emph{唯一}的元素?

证明的关键在于``$\cap$''的定义,请特别注意\emph{对于每个}这一条件:

\begin{definition}
    由集合 $I$ 索引的一系列集合 $A_i$ 的交集为
    \[\bigcap_{i \in I} A_i = \{x \in U \mid x \in A_i \text{\ 对于每个\ } i \in I\}\]
    其中我们假设存在集合 $U$ 满足对于每个 $i \in I, A_i \subseteq U$。
\end{definition}

请牢记:索引交集由属于所有成分集合的元素构成。因此在证明中,我们需要说明:
\begin{enumerate}[label=(\arabic*)]
    \item $0$ 确实是所有 $A_n$ 集合的元素。
    \item 不存在其他实数满足此性质,即对于任意非零实数,总存在某个 $A_n$ 不包含该数。
\end{enumerate} 

\begin{proof}
    首先,我们来证明
    \[\{0\} \subseteq \bigcap_{n \in \mathbb{N}}A_n\]

    这需要证明对于每个 $n \in \mathbb{N}, 0 \in A_n$。

    设 $n \in \mathbb{N}$ 为任意固定元素。注意,不等式 $0 \le 0 \le \frac{1}{n}$ 必然成立。

    (注:可能有人会担忧,因为``在极限内'' $0$ 是否``同时''小于每个分数 $\frac{1}{n}$,但这不是重点!正确的思路是:$0 \in A_1$ 吗?是的,因为 $0 \le 0 < 1$。$0 \in A_2$ 吗?是的,因为 $0 \le 0 < \frac{1}{2}$。$0 \in A_3$ 吗?是的,因为 $0 \le 0 < \frac{1}{3}$。依此类推。该不等式对于每个 $n \in N$ 均成立,所以 $0$ 是每个此类集合的元素。如果你不担心这一点,可跳过此备注,继续前进!)

    因此,对于每个 $n \in \mathbb{N}, 0 \in A_n$,所以根据 ``$\cap$'' 的定义,$\displaystyle{0 \in \bigcap_{n \in \mathbb{N}} A_n}$。这就证明了 $\displaystyle{\{0\} \subseteq \bigcap_{n \in \mathbb{N}} A_n}$。

    接着,我们来证明
    \[\bigcap_{n \in \mathbb{N}}A_n \subseteq \{0\}\]

    考虑在全集 $\mathbb{R}$ 下这些集合的\dotuline{补集}。具体来说,我们将证明
    \[\overline{\{0\}} \subseteq \overline{\bigcap_{n \in \mathbb{N}}A_n}\]

    这意味着我们要证明每个非零实数\dotuline{不属于}某个 $A_n$。

    设 $x \in \mathbb{R}$ 为任意固定元素,且 $x \ne 0$。也就是说,要么 $x > 0$ 要么 $x < 0$。接下来分两种情况讨论:

    \begin{itemize}
        \item 情况 1:假设 $x > 0$。考虑实数 $\frac{1}{x} \in \mathbb{R}$。由于 $\mathbb{R}$ 中的自然数集无上界,因此存在 $M \in \mathbb{N}$ 满足 $M > \frac{1}{x}$。
        
        (注意:思考为什么会这样。我们尚未\dotuline{证明} $\mathbb{N}$ 是无限的,即数字沿着实数轴``永远延续下去'',但我们希望这些想法对你来说直观且合理。)

        取 $M \in \mathbb{N}$ 且 $M > \frac{1}{x}$。由于 $x > 0$,不等式两边同时乘以 $x$;由于 $M > 0$(因此 $\frac{1}{M} > 0$),不等式两边再次乘以 $\frac{1}{M}$。由此可得 $x > \frac{1}{M}$。由于 $A_M$ 要求 $-\frac{1}{M} < x < \frac{1}{M}$,因此 $x \ne A_M$。

        由于 $x \notin A_M$,所以 $x$ 肯定不是所有此类集合的元素。因此
        \[x \notin \bigcap_{n \in \mathbb{N}} A_n\]

        \item 情况 2:假设 $x < 0$。考虑 $-x > 0$。采用与上面相同的逻辑,必然存在 $M \in \mathbb{N}$ 满足 $M > \frac{1}{-x} = -\frac{1}{x}$。整理不等式可得 $x < -\frac{1}{M}$。所以 $x \notin A_M$,因此 
        \[x \notin \bigcap_{n \in \mathbb{N}} A_n\]
    \end{itemize}

    综上,我们已经证明,任意 $x \in \mathbb{R}$ 且 $x \ne 0$ 不属于至少一个 $A_n$,因此任何这样的 $x$ 都不是它们交集的元素。因此,由双重包含论证可得:
    \[\{0\} = \bigcap_{n \in \mathbb{N}} A_n\]
\end{proof}

该证明具有一定难度,建议反复阅读确保理解每个步骤。特别留意选择 $M \in \mathbb{N}$ 且满足 $M > \frac{1}{x}$ 的动机:这并凭借直觉神奇偶得,而是从目标 $x > \frac{1}{M}$ 反推不等式得到的。
