% !TeX root = ../../../book.tex
\subsection{示例}

笛卡尔平面是 $\mathbb{R} \times \mathbb{R}$,因此我们有时也将其记为 $\mathbb{R}^2$。当 $A = B$ 且不会引起混淆(即 $A$ 作为集合的性质明确)时,我们有时也将笛卡尔积 $A \times A$ 简写为 $A^2$。

\begin{example}
    定义集合 $A = \{a, b, c\}$、$B = \{6, 7\}$ 和 $C = \{b, c, d\}$。我们可以列出以下笛卡尔积的元素:
    \begin{align*}
        A \times B &= \{(a, 6),(a, 7),(b, 6),(b, 7),(c, 6),(c, 7)\} \\
        B \times C &= \{(6, b),(6, c),(6, d),(7, b),(7, c),(7, d)\} \\
        A \times C &= \{(a, b),(a, c),(a, d),(b, b),(b, c),(b, d),(c, b),(c, c),(c, d)\} \\
        C \times B &= \{(b, 6),(b, 7),(c, 6),(c, 7),(d, 6),(d, 7)\}
    \end{align*}
    请注意,一般而言 $B \times C \ne C \times B$,本例即说明了这一点。(能否找到 $A \times B = B \times A$ 的情形?要使此等式成立,需对集合 $A$ 和 $B$ 施加什么条件?)
\end{example}

\subsubsection*{有序三元组及有序多元组}

类似地可以推广到三个或更多集合的笛卡尔积。对于三个集合的笛卡尔积,我们使用有序\emph{三元组};对于 $n$ 个集合的笛卡尔积,则使用有序 $n$ 元组。(需注意,存在定义有序 $n$ 元组的集合论方法,但本文不涉及这些细节。)

\begin{example}
    笛卡尔积 $\mathbb{N} \times \mathbb{N} \times \mathbb{N}$(常简写为 $\mathbb{N}^3$)是所有自然数三元组的集合。例如:$(1, 2, 3) \in \mathbb{N}^3$, $(7, 7, 100) \in \mathbb{N}^3$;但 $(0, 1, 2) \notin \mathbb{N}^3$, $(1, 2, 3, 4) \notin \mathbb{N}^3$。
\end{example}

请注意 $\mathbb{N}^3$ 与 $(\mathbb{N} \times \mathbb{N}) \times \mathbb{N}$ 的微妙区别:$\mathbb{N}^3$ 的元素是自然数构成的有序三元组,而 $(\mathbb{N} \times \mathbb{N}) \times \mathbb{N}$ 的元素是有序对,其首个分量为自然数的有序对,第二个分量为自然数。例如 $\big((1, 2), 3\big) \in (\mathbb{N} \times \mathbb{N}) \times \mathbb{N}$,但 $\big((1, 2), 3\big) \notin \mathbb{N}^3$。这表明它们是\emph{不同的集合}。

然而,存在自然的关联方式——通过``去除括号''将首个分量(有序对)映射为三元组的前两个分量。在研究函数与\emph{双射}时将进一步探讨此问题。现在只需注意二者的细微差异,并牢记:两个集合的笛卡尔积由\emph{有序对}组成,其中每个分量均来自对应的集合。

\begin{example}
    若 $B = \varnothing$ 会怎样?根据定义,$A \times B$ 的元素需以 $B$ 的元素作为第二个分量。由于 $B$ 无元素,故不存在满足条件的有序对。因此对任意集合 $A$:
    \[A \times \varnothing = \varnothing\]
    同理,对于任意集合 $B, \varnothing \times B = \varnothing$。
\end{example}
