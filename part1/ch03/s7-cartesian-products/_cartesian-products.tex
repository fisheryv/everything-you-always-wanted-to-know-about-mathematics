% !TeX root = ../../../book.tex
\section{笛卡尔积}

另一种``组合''集合以构建新集合的方法基于\textbf{顺序}的概念。在通过元素描述集合时,顺序无关紧要;例如,集合 $\{1, 2, 3\}$、$\{3, 1, 2\}$ 和 $\{2, 1, 3\}$ 是相等的,因为它们的元素完全相同(即彼此互为子集)。然而,考察顺序相关的数学对象,能让我们以新的方式组合集合并生成新的集合。

实平面 $\mathbb{R}^2$——亦称\textbf{笛卡尔平面},以法国数学家勒内·笛卡尔 (René Descartes) 命名——即为一例。平面上每个点由两个值描述:$x$ 坐标和 $y$ 坐标,且其顺序至关重要。通常 $x$ 坐标在前、$y$ 坐标在后,借此可区分不同点。例如,点 $(1, 0)$ 位于 $x$ 轴,而点 $(0, 1)$ 位于 $y$ 轴,二者并非同一点。

这背后蕴含着更普遍的数学思想:对任意集合 $A$ 和 $B$,可构造所有\textbf{有序元素对}的集合。其中\textbf{对}指表达式 $(a, b) \mid a \in A, b \in B$;\textbf{有序}强调 $a$ 在前、$b$ 在后的书写顺序不可互换。在实平面中,由于任何实数均可作为 $x$ 或 $y$ 坐标,点 $(x, y)$ 通常不同于 $(y, x)$(仅当 $x=y$ 时二者相等,请思考其原因)。

% !TeX root = ../../../book.tex
\subsection{定义}

在研究一些例子之前,我们先给出这个新集合的明确定义。

\begin{definition}
    给定两个集合 $A$ 和 $B$,则 $A$ 和 $B$ 的\dotuline{笛卡尔积}写做 $A \times B$,并定义为
    \[A \times B = \{(a, b) \mid a \in A \text{\ 且\ } b \in B\}\]
\end{definition}

该定义表明,笛卡尔积 $A \times B$ 是由所有有序对 $(a, b)$ 构成的集合,其中 $a$ 是 $A$ 中的任意元素,$b$ 是 $B$ 中的任意元素。

\subsubsection*{技术细节}

请注意,我们不再假设全集 $U$ 的存在。此前已讨论过不指定全集时可能出现的问题,但从现在起,我们将使用的集合能避免这些问题。因此,仅当不指定全集可能引发歧义时,我们才会明确指定全集。

在此定义中,可通过将有序对 $(a, b)$ 定义为一个集合来指定全集。具体来说,我们定义
\[(a, b) = \left\{ \{a\}, \{a, b\} \right\}\]
这个定义体现了有序对的性质,从某种意义上说
\[(a, b) = (c, d) \text{\ 当且仅当\ } a = c \text{\ 且\ } b = d\]
通过分析集合中的单个元素,可确定第一个坐标;通过分析包含两个元素的集合元素,可确定第二个坐标。对于有序对 $(a, a)$,该集合简化为 $\left\{\{a\}\right\}$,表明 $a$ 同时出现在两个坐标中。

基于此,全集可取为 $U = \mathcal{P}\left(\mathcal{P}(A \cup B)\right)$。我们不会深入探讨这些技术细节,但有必要指出这类定义的存在。本节的核心要点是:
\begin{center}
    两个有序对相等当且仅当它们的对应坐标相等。
\end{center}
这正是我们称之为\textbf{有序对}的原因。


% !TeX root = ../../../book.tex
\subsection{示例}

笛卡尔平面是 $\mathbb{R} \times \mathbb{R}$,因此我们有时也将其记为 $\mathbb{R}^2$。当 $A = B$ 且不会引起混淆(即 $A$ 作为集合的性质明确)时,我们有时也将笛卡尔积 $A \times A$ 简写为 $A^2$。

\begin{example}
    定义集合 $A = \{a, b, c\}$、$B = \{6, 7\}$ 和 $C = \{b, c, d\}$。我们可以列出以下笛卡尔积的元素:
    \begin{align*}
        A \times B &= \{(a, 6),(a, 7),(b, 6),(b, 7),(c, 6),(c, 7)\} \\
        B \times C &= \{(6, b),(6, c),(6, d),(7, b),(7, c),(7, d)\} \\
        A \times C &= \{(a, b),(a, c),(a, d),(b, b),(b, c),(b, d),(c, b),(c, c),(c, d)\} \\
        C \times B &= \{(b, 6),(b, 7),(c, 6),(c, 7),(d, 6),(d, 7)\}
    \end{align*}
    请注意,一般而言 $B \times C \ne C \times B$,本例即说明了这一点。(能否找到 $A \times B = B \times A$ 的情形?要使此等式成立,需对集合 $A$ 和 $B$ 施加什么条件?)
\end{example}

\subsubsection*{有序三元组及有序多元组}

类似地可以推广到三个或更多集合的笛卡尔积。对于三个集合的笛卡尔积,我们使用有序\emph{三元组};对于 $n$ 个集合的笛卡尔积,则使用有序 $n$ 元组。(需注意,存在定义有序 $n$ 元组的集合论方法,但本文不涉及这些细节。)

\begin{example}
    笛卡尔积 $\mathbb{N} \times \mathbb{N} \times \mathbb{N}$(常简写为 $\mathbb{N}^3$)是所有自然数三元组的集合。例如:$(1, 2, 3) \in \mathbb{N}^3$, $(7, 7, 100) \in \mathbb{N}^3$;但 $(0, 1, 2) \notin \mathbb{N}^3$, $(1, 2, 3, 4) \notin \mathbb{N}^3$。
\end{example}

请注意 $\mathbb{N}^3$ 与 $(\mathbb{N} \times \mathbb{N}) \times \mathbb{N}$ 的微妙区别:$\mathbb{N}^3$ 的元素是自然数构成的有序三元组,而 $(\mathbb{N} \times \mathbb{N}) \times \mathbb{N}$ 的元素是有序对,其首个分量为自然数的有序对,第二个分量为自然数。例如 $\left((1, 2), 3\right) \in (\mathbb{N} \times \mathbb{N}) \times \mathbb{N}$,但 $\left((1, 2), 3\right) \notin \mathbb{N}^3$。这表明它们是\emph{不同的集合}。

然而,存在自然的关联方式——通过``去除括号''将首个分量(有序对)映射为三元组的前两个分量。在研究函数与\emph{双射}时将进一步探讨此问题。现在只需注意二者的细微差异,并牢记:两个集合的笛卡尔积由\emph{有序对}组成,其中每个分量均来自对应的集合。

\begin{example}
    若 $B = \varnothing$ 会怎样?根据定义,$A \times B$ 的元素需以 $B$ 的元素作为第二个分量。由于 $B$ 无元素,故不存在满足条件的有序对。因此对任意集合 $A$:
    \[A \times \varnothing = \varnothing\]
    同理,对于任意集合 $B, \varnothing \times B = \varnothing$。
\end{example}


% !TeX root = ../../../book.tex
\subsection{习题}

\subsubsection*{温故知新}

以口头或书面的形式简要回答以下问题。这些问题全都基于你刚刚阅读的内容,所以如果忘记了具体的定义、概念或示例,可以回去重读相关部分。确保在继续学习之前能够自信地回答这些问题,这将有助于你的理解和记忆!

\begin{enumerate}[label=(\arabic*)]
    \item $\mathbb{R} \times \mathbb{N}$ 和 $\mathbb{N} \times \mathbb{R}$ 有什么区别?给出一个有序对的示例,该有序对是其中一个集合的元素,但不是另一个集合的元素。然后,再给出一个有序对的示例,该有序对\emph{同时}是两个集合的元素。
    \item $\varnothing \times \mathbb{Z}$ 是什么?
    \item 写出集合 $\{\heartsuit, \diamondsuit\} \times \{\smiley{}, \square, \heartsuit\}$ 的所有元素。
    \item $(\mathbb{N} \times \mathbb{N}) \times \mathbb{N}$ 和 $\mathbb{N} \times (\mathbb{N} \times \mathbb{N})$ 有什么区别? 为什么它们在技术上不是同一集合?你能解释一下为什么它们``本质上''是同一集合吗?
    \item 设 $A,B,C$ 为集合。假设 $A \subseteq B$。你认为 $A \times C \subseteq B \times C$ 成立吗?为什么成立或为什么不成立?
    \item 给出集合 $S$ 的一个例子,使得 $(\frac{1}{2}, -1) \in S$。
\end{enumerate}

\subsubsection*{小试牛刀}

尝试回答以下问题。这些题目要求你实际动笔写下答案,或(对朋友/同学)口头陈述答案。目的是帮助你练习使用新的概念、定义和符号。题目都比较简单,确保能够解决这些问题将对你大有帮助!

\begin{enumerate}[label=(\arabic*)]
    \item 写出 $[3] \times [3]$ 的元素  \\
    你能猜想一下,对于任意 $m, n \in \mathbb{N}, [m] \times [n]$ 有多少个元素吗?(你会如何证明你的猜想?)
    \item 给出集合 $\mathbb{N} \times \mathcal{P}(\mathbb{Z})$ 中的一个元素的示例。
    \item 给出集合 $\big((\mathbb{R} \times \mathbb{N}) \times \mathbb{Q}\big) \cup \big((\mathbb{Q} \times \mathbb{Z}) \times \mathbb{N}\big)$ 中的一个元素的示例。
    \item 给出集合 $C, D$ 的示例,使得 $C \times D = D \times C$。\\
    后续挑战:你能描述出像这样的所有可能情况吗?关于 $C$ 和 $D$ 的哪些事实一定是正确的?你能给出证明吗?
    \item 写出 $\mathcal{P}([1] \times [2])$ 的元素。
    \item 对于每个 $n \in \mathbb{N}$,令 $A_n = [n] \times [n]$。考虑集合
    \[B = \bigcup_{n \in \mathbb{N}} A_n\]
    $B = \mathbb{N} \times \mathbb{N}$ 吗?解释原因并举例说明。
    \item 如果你了解一些简单的计算机编程,请尝试编写代码(用你喜欢的语言)来输入 $m, n \in N$ 并打印出 $[m] \times [n]$ 的所有元素。(如果你不太熟悉编程,可以使用伪代码。)根据 $m$ 和 $n$,你认为程序需要运行多长时间?
\end{enumerate}