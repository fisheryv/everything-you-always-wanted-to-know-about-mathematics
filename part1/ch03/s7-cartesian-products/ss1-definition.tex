% !TeX root = ../../../book.tex
\subsection{定义}

在研究一些例子之前,我们先给出这个新集合的明确定义。

\begin{definition}
    给定两个集合 $A$ 和 $B$,则 $A$ 和 $B$ 的\dotuline{笛卡尔积}写做 $A \times B$,并定义为
    \[A \times B = \{(a, b) \mid a \in A \text{\ 且\ } b \in B\}\]
\end{definition}

该定义表明,笛卡尔积 $A \times B$ 是由所有有序对 $(a, b)$ 构成的集合,其中 $a$ 是 $A$ 中的任意元素,$b$ 是 $B$ 中的任意元素。

\subsubsection*{技术细节}

请注意,我们不再假设全集 $U$ 的存在。此前已讨论过不指定全集时可能出现的问题,但从现在起,我们将使用的集合能避免这些问题。因此,仅当不指定全集可能引发歧义时,我们才会明确指定全集。

在此定义中,可通过将有序对 $(a, b)$ 定义为一个集合来指定全集。具体来说,我们定义
\[(a, b) = \big\{ \{a\}, \{a, b\} \big\}\]
这个定义体现了有序对的性质,从某种意义上说
\[(a, b) = (c, d) \text{\ 当且仅当\ } a = c \text{\ 且\ } b = d\]
通过分析集合中的单个元素,可确定第一个坐标;通过分析包含两个元素的集合元素,可确定第二个坐标。对于有序对 $(a, a)$,该集合简化为 $\big\{\{a\}\big\}$,表明 $a$ 同时出现在两个坐标中。

基于此,全集可取为 $U = \mathcal{P}\big(\mathcal{P}(A \cup B)\big)$。我们不会深入探讨这些技术细节,但有必要指出这类定义的存在。本节的核心要点是:
\begin{center}
    两个有序对相等当且仅当它们的对应坐标相等。
\end{center}
这正是我们称之为\textbf{有序对}的原因。
