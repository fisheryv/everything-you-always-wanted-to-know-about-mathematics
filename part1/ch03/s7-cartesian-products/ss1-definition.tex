% !TeX root = ../../../book.tex
\subsection{定义}

在研究一些例子之前,让我们先给出这个新集合的明确定义。

\begin{definition}
    给定两个集合 $A$ 和 $B$,那么 $A$ 和 $B$ 的\dotuline{笛卡尔积}写做 $A \times B$ 并定义为
    \[A \times B = \{(a, b) \mid a \in A \;\text{且}\; b \in B\}\]
\end{definition}

这个定义告诉我们,笛卡尔积 $A \times B$ 将所有有序对 $(a, b)$ 汇总到一个新集合中,其中 $a$ 为 $A$ 中的任意元素,$b$ 允为 $B$ 中的任意元素。

\subsubsection*{技术细节}

请注意,我们已经放弃了全集 $U$ 的假设。我们已经讨论了当我们不指定全集时出现的一些问题,但从现在开始,我们使用的集合导致这些问题。因此,只有当不指定全集会导致歧义时,我们才会指定全集。

在该定义的情况下,我们可以通过将有序对 $(a, b)$ 定义为一个集合来指定一个全集。具体来说,我们可以定义
\[(a, b) = \{ \{a\}, \{a, b\} \}\]
这个定义也包含了这对的顺序,从某种意义上说
\[(a, b) = (c, d) \;\text{当且仅当}\; a = c \;\text{且}\; b = d\]
检查集合中的单个元素告诉我们第一个坐标,检查集合中具有两个元素的另一个元素告诉我们第二个坐标。如果我们有有序对 $(a, a)$,则该集合化简为 $\{\{a\}\}$,这告诉我们 $a$ 出现在两个坐标中。

通过这个定义,我们可以使用全集 $U = \mathcal{P}(\mathcal{P}(A \cup B))$。我们不会深入研究这些集合和定义的技术细节,但我们认为应该谨慎地指出这些定义的存在。上面给出了本节中需要记住的要点:
\[\text{两个有序对相等当且仅当它们的坐标相等。}\]
这就是为什么我们称其为\textbf{有序对}。
