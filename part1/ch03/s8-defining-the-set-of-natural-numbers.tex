% !TeX root = ../../book.tex
\section[定义自然数集]{[选学]定义自然数集}\label{sec:section3.8}

本节的目标是将自然数 $\mathbb{N}$ 置于严格的数学基础之上。具体来说,我们将通过集合论的公理和原理来定义和推导自然数,以此来证明自然数的存在。然后我们将讨论自然数的一些性质。在讨论数理逻辑的一些基本原理和结论之后,我们将在第 \ref{ch:chapter05} 章中使用其中一些属性来定义和证明数学归纳原理。

\subsection{定义}

我们如何用集合来\emph{定义}自然数?仅凭直觉我们就知道它们是什么。我们从 $1$ 开始,反复加 $1$,得到所有其他自然数。因此,我们必须从集合的角度来确定 ``$1$'' 的含义和 ``加 1'' 的含义。为此,我们首先考虑 $0$。我们之前说过,我们不会在集合 $\mathbb{N}$ 中包含 $0$,但有些作者会这样做,眼下这有助于我们用它来推导 $\mathbb{N}$。我们知道有一种不包含任何元素的集合,即空集。因此,将 $0$ 与空集\emph{关联起来}是合理的;事实上,我们\emph{定义} $0 = \varnothing$。接下来,我们希望定义 $1$,效仿 $0$ 的定义,我们用一个仅包含一个元素的集合来表示。(包含一个元素的集合也称为\textbf{单例}。)这里几个这样的集合:
\[\{\varnothing\}, \{\{\varnothing\}\} , \{\{\varnothing, \{\varnothing\}\}\}\]
我们如何选择一个单例来表示 $1$ 呢?请记住,我们希望这个过程持续下去并最终根据之前的数字定义 $2$(和 $3$ 等),现在根据我们可以使用的唯一对象 $0$ 来定义 $1$ 是合理的。因此,我们\emph{选择}这么定义
\[1 = \{0\} = \{\varnothing\}\]
这保证了 $0 \ne 1$。

接下来定义 $2$,我们考虑包含两个元素的集合,例如
\[\{\varnothing, \{\varnothing\}\}, \{\varnothing, \{\{\varnothing\}\}\}, \{\{\varnothing\}, \{\{\varnothing\}\}\}\]
等等等等。在这么多集合中,我们需要寻找一个自然的表示,我们注意到上面列出的第一个集合包含我们已经定义的两个对象,$0$ 和 $1$!因此,定义 $2 = \{0, 1\}$ 是一个自然的选择,并且我们再次可知 $2 \ne 0$ 且 $2 \ne 1$。

\subsubsection*{后继}

后继给了我们如何继续这一过程并从中定义任意自然数的直观想法:对于任意 $n \in \mathbb{N}$,我们定义
\[n = \{0, 1, 2, \dots , n - 2, n - 1\}\]
然而,给定一个集合,使用这个定义来检验该集合是否代表自然数将是相当困难的。我们希望对 $\mathbb{N}$ 的元素有\emph{更好的}定义;我们想知道,给定任意集合,它是否属于 $\mathbb{N}$。回顾上面的元素 $n$;我们也可以写
\[n = \{0, 1, 2, \dots , n - 2, n - 1\} = \{0, 1, 2, \dots , n - 2\} \cup \{n-1\} = (n-1) \cup \{n-1\}\]
瞧!我们得到一种根据前一个元素和集合运算来定义 $\mathbb{N}$ 中元素的自然方法。这引出了以下定义。

\begin{definition}
    给定任何集合 $X$, $X$ 的后继用 $S(X)$ 表示,定义为 $S(X) = X \cup \{X\}$。
\end{definition}

这个定义适用于所有集合,但在自然数的背景下,它意味着 $n$ 的后继正是我们直观``所知''的更大的自然数,即 $n + 1$。

\subsubsection*{归纳集}

这使我们更接近 $\mathbb{N}$ 的定义。我们当然想要 $1 \in \mathbb{N}$,并且对于任意元素 $n \in \mathbb{N}$,我们也想要 $S(n) \in \mathbb{N}$。为了符号化地对此进行编码,我们做出以下定义:

\begin{definition}
    $I$ 为\dotuline{归纳}
    \begin{enumerate}
        \item $1 \in I$
        \item 如果 $n \in I$,则 $S(n) \in I$。
    \end{enumerate}
\end{definition}

显然,$\mathbb{N}$ 本身是一个归纳集。还有其他归纳集吗?思考以下这个问题。它们会有什么属性呢?它们会包含非自然数的元素吗?我们不想深入讨论这些问题,但为了这里的讨论,我们将指出确实存在其他归纳集。我们不希望这些集合中的任何一个为 $\mathbb{N}$,因此我们做出以下定义:

\begin{definition}
    所有\dotuline{自然数}的集合是
    \[\mathbb{N} := \{x \mid \text{对于任意归纳集} I, x \in I\}\]
    换句话说,$\mathbb{N}$ 是最小归纳集,从集合包含的意义上:
    \[\mathbb{N} = \bigcap_{I \in \{S \mid S \text{为归纳集}\}} I\]
    这表明 $\mathbb{N}$ 是所有归纳集的子集。
\end{definition}

这为我们提供了我们想要的``检验属性''。任意集合 $x$ 是自然数(即 $x \in \mathbb{N}$)当且仅当它是\emph{每个}归纳集的元素(即对于每个归纳集合 $I$, $x \in I$)。这也告诉我们,对于每个归纳集 $I$, $\mathbb{N} \subseteq I$。

这里还可以进行一些其他集合论方面的讨论:我们如何知道这样一个无限集存在?(实际上,我们需要将此作为集合论的\emph{公理}!假设这些类型的集合存在,我们如何表征那些非 $\mathbb{N}$ 的其他归纳集?解决这些问题超出了本课程的范围和目标,因此我们不会在这里解决这些问题。但是,我们现在会提及 $\mathbb{N}$ 的一些属性,尤其是那些对严格构建数学归纳法有用的属性。(如果你想了解,请考虑整数集 $\mathbb{Z}$。尝试解释为什么这个集合是归纳性的。$\mathbb{R}$ 呢?$\mathbb{Z} - \mathbb{N}$ 呢?)

\subsubsection*{$\mathbb{N}$ 的性质}

在定义归纳原理之前,让我们先思考一下自然数的一些常见性质和用途:排序和算术。给定任意两个自然数,我们可以比较它们并确定哪一个更大,哪一个更小(或者它们是否相等)。我们通常用 $1 < 3, 1 \le 5, 4 \nless 2, 3 = 3$ 等符号来书写。

已知我们已经将 $\mathbb{N}$ 的元素本身定义为集合,我们是否可以用集合来表达这些比较?这是可以的!回顾一下后继的定义。该定义中内置的事实是 $X \in S(X)$!这一发现给了我们以下定义:

\begin{definition}
    给定两个自然数 $m, n \in \mathbb{N}$,当且仅当 $m \in n$ 时,我们写 $m < n$。
\end{definition}

这定义了集合 $\mathbb{N}$ 上的顺序关系。我们将在本书后面讨论关系和顺序的概念(第 \ref{sec:section6.3} 节)。

那算术呢?就集合 $m$ 和 $n$ 而言,$m + n$ 是多少?我们如何定义这个运算及其输出?我们怎么知道 $m + n$ 是另一个自然数? 我们能确定 $m + n = n + m$ 吗?这些问题在我们稍后讨论函数和关系之后就可以解决。

\subsection{数学归纳原理}\label{sec:section3.8.2}

现在,让我们提出一个更严格的归纳法版本:

\begin{theorem}{数学归纳原理}\label{theorem3.8}
    设 $P(n)$ 为某个依赖于自然数 $n$ 的``事实''或``观察''。假设
    \begin{itemize}
        \item $P(1)$ 为\verb|真|。
        \item 给定任意 $k \in \mathbb{N}$,如果 $P(k)$ 为\verb|真|,必然可以得出 $P(k+1)$ 为\verb|真|。
    \end{itemize}
    那么,陈述 $P(n)$ 对于每个自然数 $n \in \mathbb{N}$ 都必然成立。
\end{theorem}

在详细讨论其假设和结论之前,让我们先证明该定理。

\begin{proof}
    将集合 $S$ 定义为陈述 $P$ 为真的自然数。即定义 $S = \{n \in \mathbb{N} \mid P(n) \text{为真}\}$。根据定义,$S \subseteq N$。

    此外,该定理的假设保证 $1 \in S$,并且每当 $k \in S$ 时,我们也知道 $k + 1 \in S$。这意味着 $S$ 是\dotuline{归纳集}。通过定义 $\mathbb{N}$ 后的观察,我们知道 $\mathbb{N} \subseteq S$。

    因此 $S = \mathbb{N}$,因此命题 $P(n)$ 对于每个自然数 $n$ 都成立。
\end{proof}

这很顺滑,对吧?似乎所有想要的结论都“超出”了我们的定义!从这个意义上说,定义和公理是\emph{自然的}选择,因为它们完成了我们\emph{直觉}中已经``知道''的有关集合 $\mathbb{N}$ 及其属性的事情。

还有一些小问题我们没有讨论。具体来说,\emph{依赖于}自然数 $n$ 的``事实''或``观察''是什么意思?当 $P(k)$ 为真时\emph{必然得出} $P(k + 1)$ 为真意味着什么?我们所说的``\emph{为真}''是什么意思?这些都是高深的数学问题,涉及对逻辑的深入研究,我们将在下一章讨论这些问题!继续向前!加油!

\subsection{问题与练习}

口头或书面简要回答以下问题。这些题目全都基于你刚刚阅读的部分,因此如果你无法想起特定的定义、概念或示例,请返回重新阅读相应部分。确保自己在继续之前可以自信地回答这些问题,这将有助于你的理解和记忆!

\begin{enumerate}[label=(\arabic*)]
    \item 什么是归纳集?举一个非 $\mathbb{N}$ 非 $\mathbb{Z}$ 的例子
    \item 在数学归纳原理的证明中,我们定义 $S = \{n \in \mathbb{N} \mid P(n) \text{为真}\}$ 是什么意思?用文字描述一下这个集合。
    \item 对于归纳法的工作原理,想出你自己的类比。
\end{enumerate}

\subsubsection*{试一试}

尝试回答以下简答题。这些题目要求你实际动笔写一写,或(对朋友/同学)口头描述一些东西。目的是让你练习使用新概念、定义和符号。别担心,这些题本来就很简单。确保能够解决这些问题将对你有所帮助!

\begin{enumerate}[label=(\arabic*)]
    \item 如果我们将后继的定义更改为 $S(X) = {X}$ 会怎样?用 $0 = \varnothing$,集合中的 $1,2,3,4$ 分别代表什么?它们还满足等式 $n = \{0, 1, \dots, n - 1\}$吗?如果不满足,他们是否满足其他关系?探索一下!
    \item 与朋友讨论一下无限集是否存在。为什么我们需要\emph{假设}存在归纳集来定义 $\mathbb{N}$?这对你来说有效吗?从物理上讲,这有意义吗?从数学上讲呢?
    \item 考虑一个简单的算术陈述,例如 $1 + 2 = 3$。以集合的形式写出数字 $1,2,3$,并看看这个等式有何意义。在这种情况下,``$+$'' 是什么意思?
    \item 研究如何使用 $\mathbb{N}$ 来定义 $\mathbb{Z}$。在网上或书中进行一些探索,或者自己提出一个想法。
\end{enumerate}