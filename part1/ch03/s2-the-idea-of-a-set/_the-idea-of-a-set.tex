% !TeX root = ../../book.tex
\section{``集合''思想}

\subsubsection*{``物以类聚''}

集合的直观概念你可能并不陌生。如果你是奥特曼卡收藏者\footnote{原作这里用的是``棒球明星卡'',考虑到中国读者对棒球运动的陌生,译者将其改成风靡中国(青少年界)的奥特曼卡。—— 译者注},拥有``全套''卡片意味着集齐了发行商某个系列的所有卡牌。如果你和朋友玩桌游,你们会在开始前商定一套``规则'',以避免后续争议。如果你在生物、化学或物理课上做实验,你会将数据汇总成``数据集''并据此分析结果、检验假设。

这三种情境都涉及``集''或``套'' (\textbf{set}) 这个概念,它如何在不同的语境中被赋予准确含义?本质上,集合是指基于共同属性组织在一起的对象的全体。第一个例子中,所有稀有度为 UR 的卡片构成特定集合;第二个例子中,所有商定的规则组成规则集合;第三个例子中,实验中收集的数据构成数据集。每种情况都存在某种共同属性,让我们能将特定对象相互关联,并将其视为一个整体。

\subsubsection*{数学中的集合}

集合在数学中极其常见且基础,它既实用又至关重要。数学家研究抽象对象及其相互关系,若无法指代一组对象,便难以精确描述思考内容。事实上,我们早已不知不觉地使用着集合!

例如,讨论二次多项式求根公式时,我们提到:当判别式满足 $\frac{b^2}{4a} - c < 0$ 时,二次多项式 $p(x) = ax^2 + bx + c$ \emph{在实数范围内}无根。这意味着什么?我们试图说明:无论从全体实数中选取哪个 $x$,都保证 $p(x) \ne 0$。但实数的集合究竟是什么?如何定义?如何确信其存在?这些问题相当深刻,若深入探讨将使我们偏离集合论的主线。

数学语言追求表述的\emph{精确性},致力于基于基本假设建立真理体系。这些假设如同桌游开始前约定的``规则集合'',是数学推理的起点,被称为\textbf{公理}。

若你接触过几何或欧几里得 (Euclid) 的名著《\emph{几何原本}》,便不会对``公理''一词陌生。欧几里得\emph{证明}的所有几何定理都建立在几条基本假设之上:任意两点可连成线段,给定圆心和半径可作圆,非平行直线必相交等。这些命题被默认为真。

作为数学重要分支的\textbf{集合论}同样构建于公理之上。其公理体系为所有涉及集合的结论奠定基础,借助这些公理及其衍生定理,我们得以探索数学宇宙的新真理。不过,公理及其推演的深入研究更适合专门的集合论课程。本书将直接运用集合论公理的诸多推论,不作严格证明——这并非不可证,只是因为其证明过程会占用过多篇幅。

本书旨在提供符合使用场景的``集合''定义,阐述集合的基本性质,通过示例加以说明,并介绍基于集合构建新集合的运算方法。
