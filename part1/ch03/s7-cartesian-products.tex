% !TeX root = ../../book.tex
\section{笛卡尔积}

还有另一种方法可以``组合''集合来生成我们想要研究的其他集合。该方法基于\textbf{顺序}的思想。当我们通过列出元素来定义集合时,顺序是无关紧要的;也就是说,集合 $\{1, 2, 3\}$ 和 $\{3, 1, 2\}$ 还有 $\{2, 1, 3\}$ 是相等的,因为它们包含相同的元素。(更具体地说,它们在两个方向上都是彼此的子集)。然而,观察与顺序相关的数学对象,可以让我们以新的方式组合集合并产生新的集合。

你可能已经熟悉实平面 $\mathbb{R}^2$ 的概念(也称为\textbf{笛卡尔平面},以法国数学家勒内·笛卡尔命名)。平面上的每个``点''都由两个值描述,一个 $x$ 坐标和一个 $y$ 坐标,并且这些坐标的顺序很重要。我们通常认为 $x$ 坐标在前,$y$ 坐标在后,这有助于根据此顺序区分两个点。例如,点 $(1, 0)$ 位于 $x$ 轴上,而点 $(0, 1)$ 位于 $y$ 轴上。它们不是同一点。

笛卡尔平面背后有着更深邃的数学思想。给定任意两个集合 $A$ 和 $B$,我们可以查看 $A$ 和 $B$ 中所有\textbf{有序元素对}的集合。所谓\textbf{对},指的是表达式 $(a, b)$,其中 $a$ 和 $b$ 分别是 $A$ 和 $B$ 的元素。所谓\textbf{有序},指的是先写 $a$,再写 $b$,这很重要。在实平面中,这一点尤其重要,因为任何实数都可以显示为点的 $x$ 坐标或 $y$ 坐标,但点 $(x, y)$ 通常不同于点 $(y, x)$。(什么时候它们相等?仔细思考一下。)

\subsection{定义}

在研究一些例子之前,让我们先给出这个新集合的明确定义。

\begin{definition}
    给定两个集合 $A$ 和 $B$,那么 $A$ 和 $B$ 的\dotuline{笛卡尔积}写做 $A \times B$ 并定义为
    \[A \times B = \{(a, b) \mid a \in A \;\text{且}\; b \in B\}\]
\end{definition}

这个定义告诉我们,笛卡尔积 $A \times B$ 将所有有序对 $(a, b)$ 汇总到一个新集合中,其中 $a$ 为 $A$ 中的任意元素,$b$ 允为 $B$ 中的任意元素。

\subsubsection*{技术细节}

请注意,我们已经放弃了全集 $U$ 的假设。我们已经讨论了当我们不指定全集时出现的一些问题,但从现在开始,我们使用的集合导致这些问题。因此,只有当不指定全集会导致歧义时,我们才会指定全集。

在该定义的情况下,我们可以通过将有序对 $(a, b)$ 定义为一个集合来指定一个全集。具体来说,我们可以定义
\[(a, b) = \{ \{a\}, \{a, b\} \}\]
这个定义也包含了这对的顺序,从某种意义上说
\[(a, b) = (c, d) \;\text{当且仅当}\; a = c \;\text{且}\; b = d\]
检查集合中的单个元素告诉我们第一个坐标,检查集合中具有两个元素的另一个元素告诉我们第二个坐标。如果我们有有序对 $(a, a)$,则该集合化简为 $\{\{a\}\}$,这告诉我们 $a$ 出现在两个坐标中。

通过这个定义,我们可以使用全集 $U = \mathcal{P}(\mathcal{P}(A \cup B))$。我们不会深入研究这些集合和定义的技术细节,但我们认为应该谨慎地指出这些定义的存在。上面给出了本节中需要记住的要点:
\[\text{两个有序对相等当且仅当它们的坐标相等。}\]
这就是为什么我们称其为\textbf{有序对}。

\subsection{示例}

笛卡尔平面是 $\mathbb{R} \times \mathbb{R}$,这就是为什么我们有时将其写为 $\mathbb{R}^2$。如果 $A = B$,那么如果不会混淆 $A$ 是一个集合(而不是一个数字)这一事实,我们有时会将笛卡尔积写为 $A \times A = A^2$。\\

\begin{example}
    定义集合 $A = \{a, b, c\}$ 和 $B = \{6, 7\}$ 以及 $C = \{b, c, d\}$。那么我们可以列出以下笛卡尔积的元素:

    \begin{align*}
        A \times B &= \{(a, 6),(a, 7),(b, 6),(b, 7),(c, 6),(c, 7)\} \\
        B \times C &= \{(6, b),(6, c),(6, d),(7, b),(7, c),(7, d)\} \\
        A \times C &= \{(a, b),(a, c),(a, d),(b, b),(b, c),(b, d),(c, b),(c, c),(c, d)\} \\
        C \times B &= \{(b, 6),(b, 7),(c, 6),(c, 7),(d, 6),(d, 7)\}\\
    \end{align*}
    请注意,一般来说,$B \times C \ne C \times B$,如本例所示。(你能找到 $A \times B = B \times A$ 的情况吗?我们必须对集合 $A$ 和 $B$ 施加什么条件才能使这个等式成立?)
\end{example}

\subsubsection*{有序三元组及有序多元组}

该思想也适用于三组或多元组的笛卡尔积。我们只需为三个集合的笛卡尔积编写有序\emph{三元组},并且一般来说,对于 $n$ 个集合的笛卡尔积,我们编写有序 $n$ 元组。(我们再次指出,存在定义这些有序 $n$ 元组的集合论方法,但我们这里不会研究这些细节。)\\

\begin{example}
    笛卡尔积 $\mathbb{N} \times \mathbb{N} \times \mathbb{N}$(有时也写做 $\mathbb{N}^3$)是所有有序自然数三元组的集合。例如,$(1, 2, 3) \in \mathbb{N}^3$, $(7, 7, 100) \in \mathbb{N}^3$,但是 $(0, 1, 2) \notin \mathbb{N}^3$, $(1, 2, 3, 4) \notin \mathbb{N}^3$。
\end{example}

请注意 $\mathbb{N}^3$ 和 $(\mathbb{N} \times \mathbb{N}) \times \mathbb{N}$ 之间的微妙区别。$\mathbb{N}^3$ 的典型元素是一个有序\emph{三元组},其坐标均为自然数。$(\mathbb{N} \times \mathbb{N}) \times \mathbb{N}$ 的典型元素是有序对,其第一个坐标也是(自然数)有序对,第二个坐标是自然数。即,$((1, 2), 3) \in (\mathbb{N} \times \mathbb{N}) \times \mathbb{N}$ 而 $((1, 2), 3) \notin \mathbb{N}^3$。这表明两者是\emph{不同的集合}。

然而,有一种自然的方式来关联这两个集合,它本质上是在第一个坐标(有序对)周围``去掉括号''。我们稍后在研究函数和\emph{双射}时会讨论这个问题。但现在,我们只是希望你注意到两个集合之间的细微差别,并记住两个集合的笛卡尔积是一组\emph{有序对},其中每个坐标都是从相应的组成集合中提取的。\\

\begin{example}
    如果 $B = \varnothing$ 会发生什么?回顾一下 $A \times B$ 的定义。实际上没有 $B$ 的元素可以写为有序对的第二个``坐标'',因此我们实际上没有 $A \times B$ 的元素可以包含!所以,对于任意集合 $A$
    \[A \times \varnothing = \varnothing\]
    同理,对于任意集合 $B$,$\varnothing \times B = \varnothing$。
\end{example}

\subsection{习题}

\subsubsection*{温故知新}

以口头或书面的形式简要回答以下问题。这些问题全都基于你刚刚阅读的内容,所以如果忘记了具体的定义、概念或示例,可以回去重读相关部分。确保在继续学习之前能够自信地回答这些问题,这将有助于你的理解和记忆!

\begin{enumerate}[label=(\arabic*)]
    \item $\mathbb{R} \times \mathbb{N}$ 和 $\mathbb{N} \times \mathbb{R}$ 有什么区别?给出一个有序对的示例,该有序对是其中一个集合的元素,但不是另一个集合的元素。然后,再给出一个有序对的示例,该有序对\emph{同时}是两个集合的元素。
    \item $\varnothing \times \mathbb{Z}$ 是什么?
    \item 写出集合 $\{\heartsuit, \diamondsuit\} \times \{\smiley{}, \square, \heartsuit\}$ 的所有元素。
    \item $(\mathbb{N} \times \mathbb{N}) \times \mathbb{N}$ 和 $\mathbb{N} \times (\mathbb{N} \times \mathbb{N})$ 有什么区别? 为什么它们在技术上不是同一集合?你能解释一下为什么它们``本质上''是同一集合吗?
    \item 设 $A,B,C$ 为集合。假设 $A \subseteq B$。你认为 $A \times C \subseteq B \times C$ 成立吗?为什么成立或为什么不成立?
    \item 给出集合 $S$ 的一个例子,使得 $(\frac{1}{2}, -1) \in S$。
\end{enumerate}

\subsubsection*{小试牛刀}

尝试回答以下问题。这些题目要求你实际动笔写下答案,或(对朋友/同学)口头陈述答案。目的是帮助你练习使用新的概念、定义和符号。题目都比较简单,确保能够解决这些问题将对你大有帮助!

\begin{enumerate}[label=(\arabic*)]
    \item 写出 $[3] \times [3]$ 的元素  \\
    你能猜想一下,对于任意 $m, n \in \mathbb{N}, [m] \times [n]$ 有多少个元素吗?(你会如何证明你的猜想?)
    \item 给出集合 $\mathbb{N} \times \mathcal{P}(\mathbb{Z})$ 中的一个元素的示例。
    \item 给出集合 $\big((\mathbb{R} \times \mathbb{N}) \times \mathbb{Q}\big) \cup \big((\mathbb{Q} \times \mathbb{Z}) \times \mathbb{N}\big)$ 中的一个元素的示例。
    \item 给出集合 $C, D$ 的示例,使得 $C \times D = D \times C$。\\
    后续挑战:你能描述出像这样的所有可能情况吗?关于 $C$ 和 $D$ 的哪些事实一定是正确的?你能给出证明吗?
    \item 写出 $\mathcal{P}([1] \times [2])$ 的元素。
    \item 对于每个 $n \in \mathbb{N}$,令 $A_n = [n] \times [n]$。考虑集合
    \[B = \bigcup_{n \in \mathbb{N}} A_n\]
    $B = \mathbb{N} \times \mathbb{N}$ 吗?解释原因并举例说明。
    \item 如果你了解一些简单的计算机编程,请尝试编写代码(用你喜欢的语言)来输入 $m, n \in N$ 并打印出 $[m] \times [n]$ 的所有元素。(如果你不太熟悉编程,可以使用伪代码。)根据 $m$ 和 $n$,你认为程序需要运行多长时间?
\end{enumerate}