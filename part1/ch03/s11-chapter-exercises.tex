% !TeX root = ../../book.tex
\section{本章习题}

这些问题涵盖了本章的所有内容以及之前学过的知识,甚至可能涉及一些假设的数学知识。我们并不期望你解决所有问题,但你做得越多,学到的也会越多!记住,只有亲自动手\emph{做}数学,才能真正掌握它。试着亲自动手解决一个问题,阅读并思考其中的陈述。尝试写出证明并展示给朋友,看看他们是否能被说服。不断练习将你的思考清晰、准确、合乎逻辑地\emph{写}出来的能力。写完证明后再进行修改,使之更加完美。最重要的是,坚持不断\emph{做}数学!

标有 $\blacktriangleright$ 号的简答题只需解释或陈述答案,无需严格证明。

特别具有挑战性的问题带有 $\bigstar$。

\begin{exercise}
    $\blacktriangleright$ 以下关于元素和子集的陈述,请说明它们为真还是为假。准备好向持怀疑态度的朋友捍卫你的选择!
    在整个问题中,我们将使用以下定义:
    \begin{align*}
        A &= \{x \in \mathbb{Z} \mid -3 \le x \le 3\} \\
        B &= \{y \in \mathbb{Z} \mid -5 < y < 6\} \\
        C &= \{x \in \mathbb{R} \mid x^2 \ge 9\} \\
	    D &= \{x \in \mathbb{R} \mid x < -3\} \\
        E &= \{n \in \mathbb{N} \mid n \text{为偶数} \}
    \end{align*}
    \begin{enumerate}[label=(\alph*)]
        \item $A \subseteq B$
        \item $C \cap D = \varnothing$
        \item $4 \in E \cap B$
        \item $\{4\} \subseteq A \cap E$
        \item $10 \in C - D$
        \item $A \cup B \supseteq C$
        \item $3 \in A \cap C$
        \item $0 \in (A - B) \cup D$
        \item $E \cap C \subseteq \mathbb{Z}$
        \item $0 \notin B - C$
    \end{enumerate}
\end{exercise}

\begin{exercise}
    $\blacktriangleright$ 设 $m, n \in \mathbb{N}$。假设 $m \le n$。解释为什么 $\mathcal{P}([m]) \subseteq \mathcal{P}([n])$。
\end{exercise}

\begin{exercise}
    回顾 \ref{sec:section3.9} 节中的问题 \ref{exc:exercises3.9.7}。我们证明,每当两个集合满足 $A \subseteq B$ 时,它们也必然满足 $\mathcal{P}(A) \subseteq \mathcal{P}(B)$。仔细阅读该证明,提醒自己注意细节。

    这种说法``反过来''是否成立?也就是说,假设 $\mathcal{P}(A) \subseteq \mathcal{P}(B)$。你能证明 $A \subseteq B$ 也成立吗?或者你能找到一个反例吗?
\end{exercise}

\begin{exercise}
    使用``集合构建符''重写以下陈述来定义集合。然后,如果可能的话,使用大括号写出集合的所有元素;如果不能,请解释原因并写出该集合的三个示例元素。
    \begin{enumerate}[label=(\alph*)]
        \item 设 $A$ 为平方小于 $39$ 的所有自然数的集合。
        \item 设 $B$ 为方程 $x^2 - 3x - 10 = 0$ 所有实根的集合。
        \item 设 $C$ 为和为非负数的整数对的集合。
        \item 设 $D$ 为实数对的集合,其第一个坐标为正,第二个坐标为负,且两个坐标和为正。
    \end{enumerate}
\end{exercise}

\begin{exercise}
    定义如下集合:
    \begin{align*}
        A &= \{x \in \mathbb{R} \mid x^2 - x - 12 < 0 \} \\
        B &= \{y \in \mathbb{R} \mid -3 < y < 4\}
    \end{align*}
    证明 $A = B$。
\end{exercise}

\begin{exercise}
    设 $X$ 为你学校学生的集合。

    定义属性 $P(x)$ 使得 $A := \{x \in X \mid P(x)\}$ 是 $X$ 的真子集且 $A \ne \varnothing$。

    接着,定义属性 $Q(x)$ 使得 $B := \{x \in X \mid Q(x)\}$ 是 $A$ 的真子集(即 $B \subset A$)且 $B \ne \varnothing$。
\end{exercise}

\begin{exercise}
    设 $A, B, C$ 为集合且 $A \subseteq C, B \subseteq C$。
    \begin{enumerate}[label=(\alph*)]
        \item 绘制集合 $\overline{A} \cap \overline{B}$ 和 $\overline{(A \cap B)}$ 的维恩图。
        \item 证明 $\overline{A} \cap \overline{B} \subseteq \overline{(A \cap B)}$。
        \item 定义特定集合 $A,B,C$,使其严格包含,即 $\overline{A} \cap \overline{B} \subset \overline{(A \cap B)}$
        \item 定义特定集合 $A,B,C$,使得 $\overline{A} \cap \overline{B} = \overline{(A \cap B)}$。
    \end{enumerate}
\end{exercise}

\begin{exercise}
    令 $S = \{(m, n) \in \mathbb{Z} \times \mathbb{Z} \mid m = n^2\}$。$S$ 与集合 $T = \{(m, n) \in \mathbb{Z} \times \mathbb{Z} \mid n = m^2\}$ 的关系如何?如果一个是另一个的子集,请证明它。如果不是,请提供示例来证明这一点。
\end{exercise}

\begin{exercise}
    令 $(a,b)$ 为笛卡尔平面上的一点,即 $(a,b) \in \mathbb{R} \times \mathbb{R}$。设 $\varepsilon$(希腊字母 \emph{epsilon})为非负实数,即 $\varepsilon \in \mathbb{R}$ 且 $\varepsilon \ge 0$。

    设 $C_{(a,b),\varepsilon}$ 为``接近'' $(a, b)$ 的实数集合,定义如下:
    \[C_{(a,b),\varepsilon} = \Big\{(x, y) \in \mathbb{R} \times \mathbb{R} \mid \sqrt{(x - a)^2 + (y - b)^2} < \varepsilon\Big\}\]
    \begin{enumerate}
        \item 给出集合 $C_{(a,b),\varepsilon}$ 的几何描述。\\
        当我们改变 $a$ 和 $b$ 时,集合会发生什么?\\
        当我们改变 $\varepsilon$ 时会发生什么?
        \item $C_{(0,0),1} \cap C_{(0,0),2}$ 是什么?
        \item $C_{(0,0),1} \cup C_{(0,0),2}$ 是什么?
        \item $C_{(0,0),1} \cap C_{(2,2),1}$ 是什么?
    \end{enumerate}
\end{exercise}

\begin{exercise}
    考虑如下(错误)声明:
    \[\bigcup_{n \in \mathbb{N}}\mathcal{P}([n]) = \mathcal{P}(\mathbb{N})\]
    \begin{enumerate}[label=(\alph*)]
        \item 下面``证明''有什么问题?指出错误并解释为什么它/它们破坏了``证明''。
            \begin{spoof}
                首先证明 $\displaystyle{\bigcup_{n \in \mathbb{N}}\mathcal{P}([n]) \subseteq \mathcal{P}(\mathbb{N})}$

                考虑左侧并集的任意元素 $X$。

                根据索引并集的定义,我们知道存在 $k \in \mathbb{N}$ 使得 $X \subseteq [k]$。

                由于 $[k] \subseteq \mathbb{N}$,并且 $X \subseteq [k]$,我们推断出 $X \subseteq \mathbb{N}$。

                因此 $X \in \mathcal{P}(\mathbb{N})$。

                接着我们证明 ``$\subseteq$'' 关系在另一个方向上也成立。

                考虑任意元素 $Y \subseteq \mathbb{N}$。

                根据子集的定义,以及 $Y$ 是自然数的集合,我们知道存在 $\mathscr{l} \in \mathbb{N}$ 使得 $Y \subseteq [\mathscr{l}]$。

                根据索引并集的定义,我们知道 $\displaystyle{Y \in \bigcup_{n \in \mathbb{N}}\mathcal{P}([n])}$。

                由于我们已经证明了 $\subseteq$ 和 $\supseteq$,所以我们知道这两个集合是相等的。
            \end{spoof}
            $\quad$
        \item 通过定义集合 $S$ 的\textbf{明确}示例来反驳该主张,使得
            \[S \in \mathcal{P}(\mathbb{N}) \qquad \text{且} \qquad S \notin \bigcup_{n \in \mathbb{N}}\mathcal{P}([n])\]
    \end{enumerate}
\end{exercise}

\begin{exercise}
    设 $A = [3] \times [4]$。(记住 $[n] = \{1, 2, 3, \dots , n\}$。)
    设 $B = \{(x, y) \in \mathbb{Z} \times \mathbb{Z} \mid 0 \le 3x - y + 1 \le 9\}$。
    \begin{enumerate}[label=(\alph*)]
        \item \textbf{证明} $A \subseteq B$。
        \item $A = B$ 吗?为什么相等或者为什么不等?请\textbf{证明}你的主张。
    \end{enumerate}
\end{exercise}

\clearpage

\begin{exercise}
    令 $n \in \mathbb{N}$ 为固定自然数。设 $S = [n] \times [n]$。 设 $T$ 为集合
    \[T =\Big\{(x, y) \in \mathbb{Z} \times \mathbb{Z} \mid 0 \le nx + y - (n + 1) \le n^2 - 1\Big\}\]
    证明 $S \subseteq T$ 但 $S \ne T$。
\end{exercise}

\begin{exercise}
    假设 $A$ 和 $B$ 为集合。
    \begin{enumerate}[label=(\alph*)]
        \item \textbf{证明} 
        \[\mathcal{P}(A) \cup \mathcal{P}(B) \subseteq \mathcal{P}(A \cup B)\]
        \item 给出 $A$ 和 $B$ 的\textbf{明确}示例,其中 (a) 中的包含是\textbf{严格包含}。
    \end{enumerate}
\end{exercise}

\begin{exercise}
    设 $S$ 和 $T$ 为集合,其元素本身也是集合。假设 $S \subseteq T$,\textbf{证明} 
    \[\bigcup_{X \in S} X \subseteq \bigcup_{Y \in T} Y\]
\end{exercise}

\begin{exercise}
    设 $A, B, C, D$ 为集合。
    \begin{enumerate}[label=(\alph*)]
        \item \textbf{证明} 
        \[(A \times B) \cup (C \times D) \subseteq (A \cup C) \times (B \cup D)\]
        \item 给出 $A,B,C,D$ 的\textbf{明确}示例,其中 (a) 中的包含是\textbf{严格包含}。
    \end{enumerate}
\end{exercise}

\begin{exercise}
    设 $A, B, C$ 为集合。证明
        \[A \times (B \cap C) = (A \times B) \cap (A \times C)\]
    和
        \[A \times (B - C) = (A \times B) - (A \times C)\]
\end{exercise}

\begin{exercise}\label{exc:exercises3.11.17}
    设 $X,Y,Z$ 为集合。证明 $(X \cup Y ) - Z \subseteq X \cup (Y - Z)$ 但\emph{不一定}相等。
\end{exercise}

\begin{exercise}
    找出集合 $S$ 的示例,使得 $S \in \mathcal{P}(\mathbb{N})$ 且 $S$ 恰好拥有 $4$ 个元素。
    接着找出集合 $T$ 的示例,使得 $T \subseteq \mathcal{P}(\mathbb{N})$ 且 $T$ 恰好拥有 $4$ 个元素。
\end{exercise}

\begin{exercise}
    找出集合 $R,S,T$ 的示例,使得 $R \in S$ 且 $S \in T$ 且 $R \subseteq T$ 但 $R \notin T$。
\end{exercise}

\begin{exercise}
    确定以下每集合是什么,并证明你的主张。
    \[\bigcap_{n \in \mathbb{N}}[n] \qquad \text{和} \qquad \bigcup_{n \in \mathbb{N}}[n]\]
\end{exercise}

\begin{exercise}
    设 $I = \{-1, 0, 1\}$。对于每个 $i \in I$,定义 $A_i = \{i - 2, i - 1, i, i + 1, i + 2\}$ 和 $B_i = \{-2i, -i, i, 2i\}$。
    \begin{enumerate}[label=(\alph*)]
        \item 写出 $\displaystyle{\bigcup_{i \in I}A_i}$ 的元素。
        \item 写出 $\displaystyle{\bigcap_{i \in I}A_i}$ 的元素。
        \item 写出 $\displaystyle{\bigcup_{i \in I}B_i}$ 的元素。
        \item 写出 $\displaystyle{\bigcap_{i \in I}B_i}$ 的元素。
        \item 利用上面的答案,写出 $\displaystyle{\Big(\bigcup_{i \in I}A_i\Big) - \Big(\bigcup_{i \in I}B_i\Big)}$ 的元素。
        \item 利用上面的答案,写出 $\displaystyle{\Big(\bigcap_{i \in I}A_i\Big) - \Big(\bigcap_{i \in I}B_i\Big)}$ 的元素。
        \item 写出 $\displaystyle{\bigcup_{i \in I}(A_i - B_i)}$ 的元素。与 (e) 的答案有何区别?
        \item 写出 $\displaystyle{\bigcap_{i \in I}(A_i - B_i)}$ 的元素。与 (f) 的答案有何区别?
    \end{enumerate}
\end{exercise}

\begin{exercise}
    在这道题中,我们要``证明''负整数的存在性!我们说``证明''是因为直到后来我们才会真正理解我们所做的事情,但是,相信我们,这就是我们正在做的事情。

    由于这个目标,你不能\textbf{假设}存在任何严格小于 $0$ 的整数,因此你的代数步骤,尤其是 (d) 部分,不应涉及任何可能为负的项。

    也就是说,如果考虑这样的方程
    \[x + y = x + z\]
    我们\textbf{可以}通过两边同时减 $x$ 推导出 $y = z$,因为 $x - x = 0$。但是,如果我们考虑的是这样的方程
    \[x + y = z + w\]
    我们\textbf{不能}推导出 $x - z = w - y$。也许 $y > w$,所以 $w - y$ 在我们的上下文中不存在……

    设 $P = \mathbb{N} \times \mathbb{N}$。将集合 $R$ 定义为
    \[R = \{((a, b),(c, d)) \in P \times P \mid a + d = b + c\}\]
    \begin{enumerate}[label=(\alph*)]
        \item 找出三个不同的 $(c, d)$ 对,使得 $((1, 4),(c, d)) \in R$。
        \item 设 $(a, b) \in P$。证明 $((a, b),(a, b)) \in R$。
        \item 设 $((a, b),(c, d)) \in R$。证明 $((c, d),(a, b)) \in R$。
        \item 假设 $((a, b),(c, d)) \in R$ 且 $((c, d),(e, f)) \in R$。证明 $((a, b),(e, f)) \in R$。
    \end{enumerate}
\end{exercise}