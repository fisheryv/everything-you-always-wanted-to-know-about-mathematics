% !TeX root = ../../../book.tex
\subsection{定义与示例}

让我们讨论一个我们已经使用过其基本思想的主题。具体来说,让我们研究一下\emph{子集}的概念。

\begin{definition}
    给定两个集合 $A$ 和 $B$,如果 $A$ 的每个元素也是 $B$ 的元素,那么我们说 $A$ 是 $B$ 的\dotuline{子集}。

    子集的数学符号是 $\subset$,所以我们可以写成 $A \subseteq B$。

    如果我们想表明 $A$ 是 $B$ 的子集但又不等于 $B$,我们可以写作 $A \subset B$ 并说 $A$ 是 $B$ 的\dotuline{真子集}。

    我们还可以将这些关系分别写为 $B \supseteq A$ 或 $B \supset A$。在这些情况下,我们会分别说 $B$ 是 $A$ 的\dotuline{超集}或 $B$ 是 $A$ 的\dotuline{真超集}。
\end{definition}

请注意这些符号与我们用来比较实数的不等式符号之间的相似之处。我们写出 $x \le 2$ 或 $5 > z > 0$ 等不等式,并根据符号的``方向''以及是否在其下方放置横线来理解这些不等式的含义。符号 $\subseteq, \subset,\supseteq, \supset$ 的工作方式完全相同,只不过它们指的是``元素的包含''而不是``数字的大小''。

\subsubsection*{标准数集}

我们上一节中提到的标准数集可以通过子集关系很好地关联。具体来说,我们可以说
\[\mathbb{N} \subset \mathbb{Z} \subset \mathbb{Q} \subset \mathbb{R} \subset \mathbb{C}\]
同样,我们理所当然地认为我们对这些数集的知识让我们能够做出这些主张。然而,在准确描述为什么集合 $\mathbb{R}$ 存在并且是 $\mathbb{Q}$ 的真超集时,会涉及到一些深刻而复杂的数学概念。不过,现在我们使用这些集合来说明\textbf{子集}关系。

由于我们知道上面的子集关系是\textbf{正确的},因此我们使用相应的符号 ``$\subset$''。一般来说,在数学写作中简单地使用 ``$\subseteq$'' 符号很常见,即使知道 ``$\subset$'' 更适用。我们可能只会在上下文中重要的时候才使用 ``$\subset$'' 符号来表明两个集合不相等。如果该信息对于当前上下文并不重要,那么我们可能只使用 ``$\subseteq$'' 符号。

\subsubsection*{集合构建符创建子集}

我们已经在集合构建符中``使用''过子集的概念。用于将集合定义为``更大''集合中满足特定属性的所有元素。我们定义一个属性 $P(x)$,从一个更大的集合 $X$ 中提取一个变量对象 $x$,并包含满足属性 $P(x)$ 的任意元素 $x$。请注意,这个新集合的任何元素都必须是 $X$ 的元素,这仅基于我们定义它的方式。因此,以下关系成立
\[\{x \in X \mid P(x)\} \subseteq X\]
不管集合 $X$ 和属性 $P(x)$ 是什么。根据集合 $X$ 和属性 $P(x)$,真子集符号 $\subset$ 可能适用,但一般来说,我们可以肯定地说 $\subseteq$ 一定适用。

尝试提出一些集合 $X$ 和属性 $P(x)$ 的示例,使得 $\subseteq$ 适用,然后尝试提出一些 $\subset$ 适用的示例。尝试找到一个集合 $X$ 和两个不同属性 $P_1(x)$ 和 $P_2(x)$,使得 $\subset$ 适用于 $P_1(x), \subseteq$ 适用于 $P_2(x)$。尝试找到两个不同集合 $X_1$ 和 $X_2$ 以及两个不同属性 $P_1(x)$ 和 $P_2(x)$,使得
\[\{x \in X_1 \mid P_1(x)\} = \{x \in X_2 \mid P_2(x)\}\]
你能做到吗?

\subsubsection*{举例}

当且仅当第一个集合的每一个元素都是第二个集合的元素时,该集合才是另一个集合的子集。例如,这意味着以下关系均成立:

\begin{align*}
    \{142, 857\} &\subseteq \mathbb{N} \\
    \{\sqrt{3}, -\pi, 8.2\} &\subseteq \mathbb{R} \\
    \{x \in \mathbb{R} \mid x^2 = 1\} &\subseteq \mathbb{Z}
\end{align*}
你明白为什么这些都成立吗?

那么,为了使子集关系失败,我们必须找到在第一个集合中而\emph{不在}第二个集合中的元素。例如,这意味着以下关系均成立:

\begin{align*}
    \{142, -857\} &\nsubseteq \mathbb{N} \\
    \{\sqrt{3}, -\pi, 8.2\} &\nsubseteq \mathbb{Q} \\
    \{x \in \mathbb{R} \mid x^2 = 5\} &\nsubseteq \mathbb{Z}
\end{align*}

\subsubsection*{集合的所有子集}

让我们看一个特定的集合。定义 $A = \{1, 2, 3\}$。 我们可以找出 $A$ 的\emph{所有}子集吗?当然可以,为什么不能呢?

\begin{align*}
    \{1\} &\subseteq A  & \{2\} &\subseteq A \\
    \{3\} &\subseteq A & \{1,2\} &\subseteq A \\
    \{1,3\} &\subseteq A & \{2,3\} &\subseteq A \\
    A = \{1, 2,3\} &\subseteq A & \varnothing &\subseteq A \\
\end{align*}
找到前 6 个子集相当简单,但重要的是要记住 $A$ 和 $\varnothing$ 也是子集。(注意:一般来说,对于任何集合 $S, S \subseteq S$ 和 $∅ \subseteq S$ 都是正确的。思考一下!)

考虑集合 $B$,其元素是我们上面列出的所有集合:
\[B = \{\{1\}, \{2\}, \{3\}, \{1, 2\}, \{1, 3\}, \{2, 3\}, A, \varnothing\}\]
确实,任何元素 $X \in B$ 都满足 $X \subseteq A$。你明白为什么吗?
