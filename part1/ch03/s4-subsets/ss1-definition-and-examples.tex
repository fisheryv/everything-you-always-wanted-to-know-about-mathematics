% !TeX root = ../../../book.tex
\subsection{定义与示例}

现在让我们探讨一个已接触过其基本思想的主题——\emph{子集}。

\begin{definition}
    给定集合 $A$ 和 $B$,若 $A$ 中所有元素也是 $B$ 的元素,则称 $A$ 是 $B$ 的\dotuline{子集}。

    子集的数学符号为 $\subset$,记作 $A \subseteq B$。

    若 $A$ 是 $B$ 的子集且 $A \neq B$,则记作 $A \subset B$,并称 $A$ 是 $B$ 的\dotuline{真子集}。

    相应的关系也可表示为 $B \supseteq A$ 或 $B \supset A$,并称 $B$ 是 $A$ 的\dotuline{超集} 或 $B$ 是 $A$ 的\dotuline{真超集}。
\end{definition}

这些符号与实数比较中的不等式符号具有相似性。正如我们通过符号指向和等号存在性理解 $x \le 2$ 或 $5 > z > 0$ 的含义,符号 $\subseteq, \subset, \supseteq, \supset$ 同样依据指向和横线表示``元素包含''关系,而非``数值大小''关系。

\subsubsection*{标准数集}

上一节提及的标准数集存在明确的子集关系链:
\[\mathbb{N} \subset \mathbb{Z} \subset \mathbb{Q} \subset \mathbb{R} \subset \mathbb{C}\]
我们通常默认对这些数集的理解支持上述论断。然而,严格证明 $\mathbb{R}$ 存在且是 $\mathbb{Q}$ 的真超集涉及深刻的数学理论。此处我们主要用此关系链说明\textbf{子集}关系。

由于已知上述均为真子集关系,故使用``$\subset$''符号。数学写作中,即使明确是真子集关系,也常统一使用``$\subseteq$''符号。通常仅当强调集合不等时,才会使用``$\subset$''符号;若该信息不影响上下文,则优先采用``$\subseteq$''。

\subsubsection*{集合构建符创建子集}

我们已在集合构建式符号中提及子集的概念,即通过定义``更大''集合中满足特定属性的元素来构造集合。给定属性 $P(x)$,我们从集合 $X$ 中选取变量 $x$,并收集所有满足 $P(x)$ 的元素 $x$。根据定义,新集合的每个元素必然属于 $X$,因此恒有:
\[\{x \in X \mid P(x)\} \subseteq X\]
无论集合 $X$ 和属性 $P(x)$ 如何选择。依据 $X$ 和 $P(x)$ 的具体情况,真子集关系 $\subset$ 可能成立,但一般情形下 $\subseteq$ 必然成立。

请尝试为 $\subseteq$ 关系构造示例,再为 $\subset$ 关系构造示例。进一步,尝试寻找集合 $X$ 及两个不同属性 $P_1(x)$ 和 $P_2(x)$,使得 $\{x \in X \mid P_1(x)\} \subset X$ 而 $\{x \in X \mid P_2(x)\} \subseteq X$。最后,尝试寻找不同集合 $X_1, X_2$ 及不同属性 $P_1(x), P_2(x)$,使得 $\{x \in X_1 \mid P_1(x)\} = \{x \in X_2 \mid P_2(x)\}$。你能做到吗?

\subsubsection*{举例}

集合 $A$ 是 $B$ 的子集当且仅当 $A$ 的每个元素都是 $B$ 的元素。例如下列关系均成立:
\begin{align*}
    \{142, 857\} &\subseteq \mathbb{N} \\
    \{\sqrt{3}, -\pi, 8.2\} &\subseteq \mathbb{R} \\
    \{x \in \mathbb{R} \mid x^2 = 1\} &\subseteq \mathbb{Z}
\end{align*}
你明白为什么这些关系都成立吗?

反之,若存在属于 $A$ 但不属于 $B$ 的元素,则子集关系不成立。例如下列关系均成立:
\begin{align*}
    \{142, -857\} &\nsubseteq \mathbb{N} \\
    \{\sqrt{3}, -\pi, 8.2\} &\nsubseteq \mathbb{Q} \\
    \{x \in \mathbb{R} \mid x^2 = 5\} &\nsubseteq \mathbb{Z}
\end{align*}

\subsubsection*{集合的所有子集}

取特定集合 $A = \{1, 2, 3\}$,能否列出其\emph{所有}子集?答案是肯定的:
\begin{align*}
    \{1\} &\subseteq A  & \{2\} &\subseteq A & \{3\} &\subseteq A\\
    \{1,2\} &\subseteq A & \{1,3\} &\subseteq A & \{2,3\} &\subseteq A \\
    A = \{1, 2,3\} &\subseteq A & \varnothing &\subseteq A 
\end{align*}
前 6 个子集较易得出,但需牢记 $A$ 自身与空集 $\varnothing$ 也是子集。(注:对任意集合 $S$,恒有 $S \subseteq S$ 和 $\varnothing \subseteq S$,请思考原因。)

考虑集合 $B$,其元素为上述所有子集:
\[B = \big\{\{1\}, \{2\}, \{3\}, \{1, 2\}, \{1, 3\}, \{2, 3\}, A, \varnothing \big\}\]
对任意 $X \in B$,均有 $X \subseteq A$。是否理解其内在逻辑?
