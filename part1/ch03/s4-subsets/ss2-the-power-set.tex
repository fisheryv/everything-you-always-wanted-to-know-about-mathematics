% !TeX root = ../../../book.tex
\subsection{幂集}

这种找出给定集合的所有子集的过程是常见且有用的,因此我们赋予这个结果集一个特殊名字。

\begin{definition}
    给定一个集合 $A, A$ 的\dotuline{幂集}定义为元素为 $A$ 的所有子集的集合,记为 $\mathcal{P}(A)$。
\end{definition}

我们在上一小节最后观察到,对于任何集合 $S, S \in \mathcal{P}(S)$ 且 $\varnothing \in \mathcal{P}(S)$。

回顾一下上面的示例集合 $A = \{1, 2, 3\}$。关于 $\mathcal{P}(A)$ 中的元质数量,你注意到什么了?它与 $A$ 中元质数量有何关系?对于任意集合 $S$,你认为 $S$ 和 $\mathcal{P}(A)$ 中的元质数量之间存在一般关系吗?\\

\begin{example}
    我们来求 $\mathcal{P}(\varnothing)$。空集的子集是什么?只有一个,就是空集本身!(即,$\varnothing \subseteq \varnothing$,但没有其他集合满足这一点。)因此,幂集 $\mathcal{P}(\varnothing)$ 只有一个元素,即空集本身:
    \[\mathcal{P}(\varnothing) = \{ \varnothing \}\]
    请注意,这与空集本身不同:
    \[\varnothing \ne \{ \varnothing \}\]
    为什么这是真的?比较元素就能知道!空集没有元素,但右边的集合有一个元素。(一般来说,这可能是比较两个集合的有效方法。)为了给你一些练习,请大声读出上面一行:
    \begin{center}
        ``空集与包含空集的集合是两个不同的集合。''
    \end{center}
\end{example}

\begin{example}
    让我们用另一个集合尝试这个过程,比如 $A = \{\varnothing, \{1, \varnothing\}\}$。我们可以将 $\mathcal{P}(A)$ 的元素列出为
    \[\mathcal{P}(A) = \{\{\varnothing\}, \{\{1, \varnothing\}\}, \{\varnothing, \{1, \varnothing\}\}, \varnothing \}\]
    这可能看起来很奇怪,因为所有的都是空集和花括号,但保持子集关系的正确性很重要。确实,在这个例子中,
    \[\varnothing \in A, \quad \{\varnothing\} \subseteq A, \quad \{\varnothing\} \in \mathcal{P}(A), \quad \{\varnothing\} \subseteq \mathcal{P}(A)\]
    为什么这些关系是正确的?仔细思考一下,然后尝试自己多写一些。``$\in$'' 和 ``$\subseteq$'' 的区别非常重要!
\end{example}
