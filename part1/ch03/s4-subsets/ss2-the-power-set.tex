% !TeX root = ../../../book.tex
\subsection{幂集}

寻找给定集合所有子集的过程既常见又实用,因此我们赋予结果集一个特定名称。

\begin{definition}
    给定集合 $A, A$ 的\dotuline{幂集}定义为 $A$ 的所有子集构成的集合,记作 $\mathcal{P}(A)$。
\end{definition}

根据上一小节末尾的观察,对任意集合 $S$,均有 $S \in \mathcal{P}(S)$ 且 $\varnothing \in \mathcal{P}(S)$。

回顾示例集合 $A = \{1, 2, 3\}$。$\mathcal{P}(A)$ 的元素个数有何特点?它与 $A$ 的元素数量存在何种关联?对任意集合 $S$,你认为 $S$ 与 $\mathcal{P}(S)$ 的元素个数之间是否存在普遍关系?

\begin{example}
    试求 $\mathcal{P}(\varnothing)$。空集的唯一子集是其自身(即 $\varnothing \subseteq \varnothing$ 成立,且无其他子集)。因此其幂集为仅含空集的集合:
    \[\mathcal{P}(\varnothing) = \{ \varnothing \}\]
    注意该幂集与空集不同:
    \[\varnothing \ne \{ \varnothing \}\]
    原因在于元素差异:空集无元素,而右侧集合含一个元素。(此法常用于集合比较。)请朗读下列表述以加深理解:
    \begin{center}
        ``空集与包含空集的集合是两个不同的集合。''
    \end{center}
\end{example}

\begin{example}
    再以集合 $A = \{\varnothing, \{1, \varnothing\}\}$ 为例。其幂集 $\mathcal{P}(A)$ 的元素可列举如下:
    \[\mathcal{P}(A) = \Big\{\{\varnothing\}, \big\{\{1, \varnothing\}\big\}, \big\{\varnothing, \{1, \varnothing\}\big\}, \varnothing \Big\}\]
    此形式虽因嵌套结构显得复杂,但必须严格保持子集关系。在此例中,
    \[\varnothing \in A, \quad \{\varnothing\} \subseteq A, \quad \{\varnothing\} \in \mathcal{P}(A), \quad \{\varnothing\} \subseteq \mathcal{P}(A)\]
    请思考这些关系的合理性,并尝试自行推导更多结论。请务必厘清``$\in$''与``$\subseteq$''的本质区别!
\end{example}
