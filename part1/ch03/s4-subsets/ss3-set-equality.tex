% !TeX root = ../../../book.tex
\subsection{集合相等}

什么情况下两个集合相等?一般思路是,如果两个集合包含``相同的元素'',则它们相等,但这并不是相等的精确定义。我们如何才能更明确、更严格地描述该属性?说两个集合 $A$ 和 $B$ 具有``相同的元素''意味着 $A$ 的每个元素也是 $B$ 的元素,$B$ 的每个元素也是 $A$ 的元素。如果这两个属性同时成立,那么我们可以保证这两个集合包含完全相同的元素,所以相等。如果你仔细一想,就会发现我们可以用\textbf{子集}来表达它。多么方便啊!

\begin{definition}
    我们说两个集合 $A$ 和 $B$ \dotuline{相等},当且仅当 $A \subseteq B$ 且 $B \subseteq A$,并写为 $A = B$。
\end{definition}
(如果我们在定义中使用 $\subset$ 符号而不是 $\subseteq$ 会发生什么?这与集合相等的概念相同吗?为什么相同或为什么不同?)

当我们需要如何证明两个集合相等,但又不能简单地列出每个集合的元素并比较时,这个定义将非常有用。通过构造两个论证证明``两个方向''的子集关系,我们可以证明两个集合是相等的。现在,让我们看一个该定义的简单应用。\\

\begin{example}
    如何使用集合相等的定义得到以下等式成立?
    \[\{x \in \mathbb{Z} \mid x \ge 1\} = \mathbb{N}\]
    我们只需得到 $\subseteq$ 和 $\supseteq$ 关系适用于等式两端即可。首先,每个至少为 $1$ 的整数都是自然数吗?当然是的!这解释了为什么
    \[\{x \in \mathbb{Z} \mid x \ge 1\} \subseteq \mathbb{N}\]
    其次,是否每个自然数都是至少为 $1$ 的正整数?当然是的!这解释了为什么
    \[\{x \in \mathbb{Z} \mid x \ge 1\} \supseteq \mathbb{N}\]
    综上,这表明题目等式是成立的。
\end{example}
