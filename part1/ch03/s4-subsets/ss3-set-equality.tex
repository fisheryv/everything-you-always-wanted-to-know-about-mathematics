% !TeX root = ../../../book.tex
\subsection{集合相等}

什么情况下两个集合相等?通常认为,若两个集合包含``相同的元素'',则它们相等,但这并非精确定义。如何更明确、更严格地描述该属性?称集合 $A$ 和 $B$ 具有``相同的元素'',意味着 $A$ 的每个元素都属于 $B$,且 $B$ 的每个元素都属于 $A$。若同时满足这两点,则两个集合必然包含完全相同的元素,所以相等。值得注意的是,这一性质可用\textbf{子集}关系精确表述。

\begin{definition}
    集合 $A$ 和 $B$ \dotuline{相等}当且仅当 $A \subseteq B$ 且 $B \subseteq A$,记作 $A = B$。
\end{definition}

(若定义中使用 $\subset$ 而非 $\subseteq$,会产生什么影响?此时定义的相等概念是否相同?试说明原因。)

当无法直接列举元素比较集合时,此定义尤为重要。通过分别证明``两个方向''的子集关系,即可完成集合相等的论证。下面通过简单示例展示其应用。

\begin{example}
    如何用集合相等的定义证明下述等式?
    \[\{x \in \mathbb{Z} \mid x \ge 1\} = \mathbb{N}\]
    只需证明 $\subseteq$ 与 $\supseteq$ 关系同时成立即可。首先,每个不小于 $1$ 的整数均为自然数,故有
    \[\{x \in \mathbb{Z} \mid x \ge 1\} \subseteq \mathbb{N}\]
    其次,每个自然数均为不小于 $1$ 的正整数,故有
    \[\{x \in \mathbb{Z} \mid x \ge 1\} \supseteq \mathbb{N}\]
    综上,这表明题目等式成立。
\end{example}
