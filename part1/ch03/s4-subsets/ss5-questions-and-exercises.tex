% !TeX root = ../../../book.tex
\subsection{习题}

\subsubsection*{温故知新}

以口头或书面的形式简要回答以下问题。这些问题全都基于你刚刚阅读的内容,所以如果忘记了具体的定义、概念或示例,可以回去重读相关部分。确保在继续学习之前能够自信地回答这些问题,这将有助于你的理解和记忆!

\begin{enumerate}[label=(\arabic*)]
    \item $\mathbb{N} \subseteq \mathbb{R}$ 吗? $\mathbb{R} \subseteq \mathbb{N}$ 吗? $\mathbb{Q} \subseteq \mathbb{Z}$ 吗?为什么是或者为什么不是?
    \item $\subset$ 和 $\subseteq$ 有什么不同?给出集合 $A, B$ 的示例,使得 $A \subseteq B$ 为真,但 $A \subset B$ 为假。
    \item $\in$ 和 $\subseteq$ 有什么区别?给出集合 $C, D$ 的示例,使得 $C \subseteq D$ 但 $C \notin D$。
    \item 设 $S$ 为任意集合。$S$ 的幂集是什么?它是什么类型的数学对象?它应该如何定义?
    \item 假设 $S \subseteq T$。这是否意味着 $S = T$?为什么相等或者为什么不等?
    \item 解释为什么对于任意集合 $S$ 都有 $\varnothing \subseteq S$ 且 $\varnothing \in \mathcal{P}(S)$。
    \item 假设 $X \in \mathcal{P}(A)$。那么 $X$ 和 $A$ 有什么关系?
    \item $A = P(A)$ 可能为真吗?(这个问题比较棘手,请好好思考一下!)
\end{enumerate}

\subsubsection*{小试牛刀}

尝试回答以下问题。这些题目要求你实际动笔写下答案,或(对朋友/同学)口头陈述答案。目的是帮助你练习使用新的概念、定义和符号。题目都比较简单,确保能够解决这些问题将对你大有帮助!

\begin{enumerate}[label=(\arabic*)]
    \item 写出集合 $\mathcal{P}(\mathcal{P}(\varnothing))$ 的元素。
    \item 写出集合 $\mathcal{P}([1]), \mathcal{P}([2]), \mathcal{P}([3])$ 的元素。你能猜想 $\mathcal{P}([n])$ 有多少个元素吗?(你能证明这一点吗?我们不指望你现在就能证明出来,但很快就能了;好好想一想!)
    \item 设 $A = \{x, \heartsuit, \{4\} , \varnothing\}$。对于以下陈述,判断它是对是错,并简要解释原因。
        \begin{enumerate}[label=(\alph*)]
            \item $x \in A$
            \item $x \subseteq A$
            \item $\{x, \heartsuit\} \subseteq A$
            \item $\{x, \varnothing\} \subset A$
            \item $\{x, \heartsuit, z, 7\} \supseteq A$
            \item $\{x\} \in \mathcal{P}(A)$
            \item $\{x\} \subseteq \mathcal{P}(A)$
            \item $\{\heartsuit, x\} \in \mathcal{P}(A)$
            \item $\{4\} \in \mathcal{P}(A)$
            \item $\{\varnothing\} \in \mathcal{P}(A)$
            \item $\{\varnothing\} \subseteq \mathcal{P}(A)$
        \end{enumerate}
        \textbf{提示:}$7$ 个为真,$4$ 个为假。
    \item 举一个集合 $A, B$ 的例子,使得 $A \in B$ 且 $A \subseteq B$ 都为真。
    \item $\{1, 2, 12\} \subseteq \mathbb{R}$ 吗?
    \item $\{-5, 8, 12\} \subseteq \mathbb{N}$ 吗?
    \item $\{1, 3, 7\} \in \mathcal{P}(\mathbb{N})$ 吗?
    \item $\mathbb{N} \in \mathcal{P}(\mathbb{Z})$ 吗?
    \item $\mathcal{P}(\mathbb{N}) \subseteq \mathcal{P}(\mathbb{Z})$ 吗?它们是相等的集合吗?为什么是或者为什么不是?
    \item 给出一个无限集合 $T$ 的例子,使得 $T \in \mathcal{P}(\mathbb{Z})$ 但 $T \notin \mathcal{P}(\mathbb{N})$。
    \item 假设 $G, H$ 是集合并且它们满足 $\mathcal{P}(G) = \mathcal{P}(H)$。我们能得出 $G = H$ 的结论吗?为什么能或者为什么不能?(不要试图正式证明这一点;只需思考并尝试说出来。)
    \item 给出一个集合 $W$ 的例子,使得 $W \subseteq \mathcal{P}(\mathbb{N})$ 但 $W \notin \mathcal{P}(\mathbb{N})$。
\end{enumerate}
