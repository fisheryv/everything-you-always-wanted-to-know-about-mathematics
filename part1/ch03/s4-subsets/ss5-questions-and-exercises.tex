% !TeX root = ../../../book.tex
\subsection{习题}

\subsubsection*{温故知新}

以口头或书面的形式简要回答以下问题。这些问题全都基于你刚刚阅读的内容,如果忘记了具体定义、概念或示例,可以回顾相关内容。确保在继续学习之前能够自信地作答这些问题,这将有助于你的理解和记忆!

\begin{enumerate}[label=(\arabic*)]
    \item $\mathbb{N} \subseteq \mathbb{R}$ 吗? $\mathbb{R} \subseteq \mathbb{N}$ 吗? $\mathbb{Q} \subseteq \mathbb{Z}$ 吗?为什么?
    \item $\subset$ 和 $\subseteq$ 有何区别?给出集合 $A, B$ 的示例,使得 $A \subseteq B$ 为真,但 $A \subset B$ 为假。
    \item $\in$ 和 $\subseteq$ 有何区别?给出集合 $C, D$ 的示例,使得 $C \subseteq D$ 但 $C \notin D$。
    \item 设 $S$ 为任意集合。$S$ 的幂集是什么?它是何种数学对象?如何定义?
    \item 假设 $S \subseteq T$。这是否意味着 $S = T$?为什么?
    \item 解释为什么对于任意集合 $S$ 都有 $\varnothing \subseteq S$ 且 $\varnothing \in \mathcal{P}(S)$。
    \item 假设 $X \in \mathcal{P}(A)$。那么 $X$ 和 $A$ 有什么关系?
    \item $A = P(A)$ 可能为真吗?(此问题较棘手,请仔细思考!)
\end{enumerate}

\subsubsection*{小试牛刀}

尝试解答以下问题。这些题目需动笔书写或口头阐述答案,旨在帮助你熟练运用新概念、定义及符号。题目难度适中,确保掌握它们将大有裨益!

\begin{enumerate}[label=(\arabic*)]
    \item 写出集合 $\mathcal{P}(\mathcal{P}(\varnothing))$ 的元素。
    \item 写出集合 $\mathcal{P}([1]), \mathcal{P}([2]), \mathcal{P}([3])$ 的元素。你能推测 $\mathcal{P}([n])$ 的元素个数吗?\\(暂不要求证明,但请深入思考。)
    \item 设 $A = \big\{x, \heartsuit, \{4\} , \varnothing\big\}$。判断以下陈述的真假并简要说明理由。
        \begin{enumerate}[label=(\alph*)]
            \item $x \in A$
            \item $x \subseteq A$
            \item $\{x, \heartsuit\} \subseteq A$
            \item $\{x, \varnothing\} \subset A$
            \item $\{x, \heartsuit, z, 7\} \supseteq A$
            \item $\{x\} \in \mathcal{P}(A)$
            \item $\{x\} \subseteq \mathcal{P}(A)$
            \item $\{\heartsuit, x\} \in \mathcal{P}(A)$
            \item $\{4\} \in \mathcal{P}(A)$
            \item $\{\varnothing\} \in \mathcal{P}(A)$
            \item $\{\varnothing\} \subseteq \mathcal{P}(A)$
        \end{enumerate}
        \textbf{提示:}$7$ 个为真,$4$ 个为假。
    \item 举出集合 $A, B$ 的例子,使得 $A \in B$ 且 $A \subseteq B$ 同时成立。
    \item $\{1, 2, 12\} \subseteq \mathbb{R}$ 吗?
    \item $\{-5, 8, 12\} \subseteq \mathbb{N}$ 吗?
    \item $\{1, 3, 7\} \in \mathcal{P}(\mathbb{N})$ 吗?
    \item $\mathbb{N} \in \mathcal{P}(\mathbb{Z})$ 吗?
    \item $\mathcal{P}(\mathbb{N}) \subseteq \mathcal{P}(\mathbb{Z})$ 吗?二者是否相等?为什么?
    \item 给出无限集合 $T$ 的例子,使得 $T \in \mathcal{P}(\mathbb{Z})$ 但 $T \notin \mathcal{P}(\mathbb{N})$。
    \item 假设 $G, H$ 为集合且 $\mathcal{P}(G) = \mathcal{P}(H)$。能否推出 $G = H$?为什么?\\(无需严格证明,思考说明即可。)
    \item 给出集合 $W$ 的例子,使得 $W \subseteq \mathcal{P}(\mathbb{N})$ 但 $W \notin \mathcal{P}(\mathbb{N})$。
\end{enumerate}
