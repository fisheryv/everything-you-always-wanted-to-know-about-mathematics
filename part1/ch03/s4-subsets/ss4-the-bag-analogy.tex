% !TeX root = ../../../book.tex
\subsection{``口袋''类比}\label{sec:section3.4.4}

根据我们的经验,集合在引入时是一个很难理解的概念。具体来说,与集合相关的\textbf{符号}会让学生陷入困境,他们最终会写下毫无意义的东西!因此必须区分符号 $\in$ 和 $\subseteq$ 之间的差异。

请记住下面这个有用的类比:集合就像一个里面装着东西的\emph{口袋}。口袋本身无关紧要;我们只关心里面有什么\emph{样}的东西(即元素是什么)。甚至可以把这个口袋想象成你在杂货店买到的一个不起眼的塑料袋。所有这些口袋都是一样的;为了区分任意两个口袋,我们需要知道\emph{里面}装的是什么东西。

如果我将一个苹果和一个橙子放入口袋中,放置它们的顺序并不重要。你只需要知道我有苹果和橙子即可。我袋子里有多少苹果或橙子并不重要,因为我们只关心里面装着什么样的东西。将其视为回答``口袋里有 $\underline{\qquad}$ 吗?有还是没有?''形式的问题。无论口袋里是有两个苹果、七个苹果还是一个苹果,都没关系;如果你问我有没有苹果,我都会说``有''。这与集合中元素的顺序和重复无关紧要这个概念有关。集合完全由其元素来表征。

当我们将集合视为其他集合的元素时,这个类比也很有帮助。我们当然可以将整个袋子放入另一个袋子里。看看我们在上面的例子中定义的集合 $A$:
\[A = \{\varnothing, \{1, \varnothing\}\}\]
集合 $A$ 是一个口袋。口袋里有什么?口袋里有两个物体(即 $A$ 有两个元素)。它们本身恰好也都是口袋!其中一个是一个普通的空口袋,里面什么也没有。(那就是空集。) 好吧,那很酷。另一个里面有两个物体。其中一个对象是数字 $1$。酷。另一个物体又是一个空口袋。

\subsubsection*{区分 ``$\in$'' 和 ``$\subseteq$''}

口袋类比也有助于理解 ``$\in$'' 和 ``$\subseteq$'' 之间的区别。继续使用集合 $A$ 来做示例。当我们写 $x \in A$ 时,我们的意思是 $x$ 是口袋 $A$ 内的一个物体。如果我们打开 $A$ 去查看,我们会看到一个 $x$ 位于口袋 $A$ 的底部。让我们用这个思路来比较两个例子。

\begin{itemize}
    \item 我们看到 $\varnothing \in A$ 在这里是正确的。如果我们看一下口袋 $A$ 的内部,我们会在里面的东西(元素)中看到一个空袋子。
    \item 我们还看到 $\{\varnothing\} \notin A$ 在这里也是正确的。如果我们看一下口袋 $A$ 的内部,我们不会看到只装着另一个空袋子的袋子。(请注意,这就是 $\{\varnothing\}$:一个空袋子装在另一个袋子里。)\\
    你看到了这样的物体吗?在哪里?我不敢让你给我看,在口袋 $A$ 里面的东西中,有一个袋子只装着一个空袋子。\\
    我在口袋 $A$ 里看到了什么?好吧,我看到两样东西:一个空袋子,和一个里面有两个物体的袋子(一个空袋子和数字 $1$)。这些物体都不是我们要找的!
\end{itemize}

当我们写 $X \subseteq A$ 时,我们的意思是 $X$ 和 $A$ 这两个口袋在某种程度上是可以比较的。具体来讲,我们是说 $X$ 内部的所有内容也是 $A$ 内部的内容。我们实际上是在遍历 $X$ 内部的所有对象,将它们一一取出,并确保我们也能在 $A$ 内部找到该对象。让我们用这个思路来比较两个例子。

\begin{itemize}
    \item 我们看到 ${\varnothing} \subseteq A$ 是正确的。我们\emph{比较}左边的口袋和右边的口袋。左边的口袋里装着是什么?里面只有一个物体,这个物体本身就是一个空袋子。现在,我们看一下 $A$ 内部。看是否从里面能找到一个空袋子?没错,可以找到!因此,``$\nsubseteq$'' 符号适用于此。
    \item 我们还看到 $\{1\} \nsubseteq A$ 也是正确的。为了比较这两个口袋,我们从左边的口袋里拿出一个物体,看看它是否也在口袋 $A$ 中。这里,我们只有一个物体要拿出来:数字 $1$。现在,让我们看看口袋 $A$ 内部。我们看到里面有一个 $1$ 吗? 不,我们没有找到!\\
    我们必须进到口袋 $A$ 内部的口袋才能找到数字 $1$;这个数字不在我们直接视线内。因此 $\{1\} \nsubseteq A$。
\end{itemize}

回顾一下我们已经讨论过的一些例子,记住这个新的类比。它有助于你理解定义和示例吗?它是否有助于你理解 ``$\in$'',$\subseteq$''和 ``$\supseteq$'' 之间的区别?如果没有,你能想出其他对你有帮助的类比吗?
