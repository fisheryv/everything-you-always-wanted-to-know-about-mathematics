% !TeX root = ../../book.tex
\section{引言}

现在是时候学习集合了!这部分内容出现在上一章之后,似乎是一个奇怪的跳跃。请详细我们,这是自然且必要的。我们在数学中所做的一切都建立在集合的基础上,所以我们最好现在就开始学习集合并习惯使用集合。

\subsection{目标}

以下简短内容将向你展示本章如何融入本书的体系。这部分内容会描述我们之前的工作将如何发挥作用,还会激发我们为什么要研究本章出现的主题,并告诉你我们的目标,以及你在阅读时应该记住什么来实现这些目标。现在,我们将通过一系列陈述为你总结本章的主要目标,以及本章结束时你应该获得的技能和知识。以下各节将更详细地重申这些想法,但这里将为你提供一个简短的列表以供将来参考。当学完本章后,请返回此列表,看看你是否理解所有这些目标。你明白为什么我们在这里概述它们很重要吗?你能定义我们使用的所有术语吗?你能应用我们描述的技术吗?

\textbf{学完本章后,你应该能够……}

\begin{itemize}
    \item 定义什么是集合,并给出几个常见的例子。
    \item 使用正确的符号来定义集合并引用其元素。
    \item 定义并描述常见集合操作;即用两个或多个集合创建新集合的方法。
    \item 描述如何比较两组两个集合,并应用恰当的技术来证明此类观点。
    \item 解释自然数与集合的关系,并将其与数学归纳法联系起来。
\end{itemize}

\subsection{承上}

我们正在构建数学归纳法的正式表述,并将其证明为\emph{定理}。为了实现这一目标,我们需要一些基本对象以便于逻辑严谨地处理和讨论。集合就是那些对象!从历史上看,数学是在二十世纪初才建立在\emph{集合论}的基础之上。在那之前,数学家们倾向于对他们工作背后真正发生的事情``撒手不管''。他们做出了很多``直觉上的''假设,但从未尝试严格且\emph{公理化地}描述他们所做的一切。数学家\textbf{乔治·康托尔(Georg Cantor)}的工作向大家展示了一些令人惊讶且反直觉的结果,这些结果完全正确且与我们的假设一致……于是,我们意识到我们有必要确定我们一直谈论的内容。当然,这并不是要抹黑 1900 年之前的数学家的工作!我们只是说他们一直在玩一个游戏,但并没有真正就一套规则达成一致。这就是集合论\textbf{公理体系}。

\subsection{启下}

当然,我们的动机是不断学习了解\textbf{证明},发现它们是什么以及它们是如何工作的,尤其是严格的数学归纳法。不过,更一般地说,我们对数学家真正的工作充满兴趣,并且我们确信世界上任何一位数学家都会告诉你\textbf{集合}在他们的工作中有多重要。他们可能内心不情愿,而嘴上说他们自己永远无法在纯\emph{集合论}中工作,但我们相信你找不出任何人否认集合的重要性。

我们稍后所做的一切都将涉及对一组对象进行一些声明;也就是说,我们将尝试说(并随后证明)关于某些特定对象的某些事实为真。我们指定这些对象的方式就涉及集合。我们表达这些事实的方式将涉及数学逻辑,我们很快就会学到这一点。就目前而言,我们首先需要学习如何表达多种类型的数学对象,然后才能对它们做出声明。

\subsection{忠告}

本章可能会涉及一些新的数学思想,不像前面章节中,我们专注的都是仅依赖于数字、代数、算术和批判性思维的谜题。这些新思想需要仔细阅读和思考。当我们介绍这些概念和结果时,我们希望你仔细阅读并进行一些思考。与报纸文章相比,数学阐述对读者的要求更高;它期望读者能够\emph{全神贯注},仔细思考每一句话,有时必须暂停几分钟,以确保充分理解到目前为止所讲的内容。当你继续阅读时,请牢记这一点:阅读数学可能很困难,但这是意料之中的事!不必为此沮丧;只要把每一句话都想象成需要完成的大拼图中的一块即可。

需要特别指出的是,如果本章的阅读时间(连同上课时间)与前两章的总和一样长(可能更长),请不要对此感到惊讶!正如我们多年来观察到的,其中最令人困惑的部分是集合的\textbf{表示法}。这可能是你数学生涯中第一次被要求写得尽可能\textbf{精确}和\textbf{严谨}。在你的书面作品中仅仅``有正确的想法''已经不够了;我们真的很在乎你所说即所想,不会言不达意。当你写完问题或作业的答案后,请再读一遍并问自己:``这真的合理吗?它是否说出了我想表达的、我脑子里的真实想法是什么?别人能保证以我写的方式阅读吗?''

此外,本章将涉及一些比典型数学课程更\textbf{抽象的}思维。这可能会让你感到震惊,也可能不会。不管怎样,这肯定不是你可以快速浏览并指望第一眼就能看明白的内容。现在,你比以往任何时候都更应该花时间和精力来消化这些内容。先读上几页,然后在吃饭、洗澡或打球时思考一下这些内容。尝试在现实生活中寻找身边的例子。和你的朋友讨论集合。现在这听上去可能很愚蠢,但最终,它会让你受益匪浅。请相信我们。