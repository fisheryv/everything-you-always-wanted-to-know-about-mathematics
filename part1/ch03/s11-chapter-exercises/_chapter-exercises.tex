% !TeX root = ../../../book.tex
\section{本章习题}

本节习题涵盖本章全部内容,并涉及先前知识点及部分数学假设。我们不要求你解答\textbf{所有}题目,但解决得越多,收获越大!请牢记:真正\emph{掌握}数学必须亲自\emph{实践}。尝试动手解题,仔细阅读并思考题意。撰写证明并与朋友讨论,检验其说服力。持续练习如何清晰、准确、有条理地\emph{书写}思路。完成证明后要反复修改以臻完善。最重要的是,坚持\emph{钻研}数学!

标有 $\blacktriangleright$ 的简答题只需解释或陈述答案,无需严格证明。

特别具有挑战性的问题标记为 $\bigstar$。

\begin{exercise}
    $\blacktriangleright$ 请判断以下关于元素和子集的陈述是是否成立,并准备好向持怀疑态度的朋友解释你的结论!
    本题中统一使用如下定义:
    \begin{align*}
        A &= \{x \in \mathbb{Z} \mid -3 \le x \le 3\} \\
        B &= \{y \in \mathbb{Z} \mid -5 < y < 6\} \\
        C &= \{x \in \mathbb{R} \mid x^2 \ge 9\} \\
	    D &= \{x \in \mathbb{R} \mid x < -3\} \\
        E &= \{n \in \mathbb{N} \mid n \text{\ 为偶数} \}
    \end{align*}
    \begin{tasks}(2)
        \task $A \subseteq B$
        \task $C \cap D = \varnothing$
        \task $4 \in E \cap B$
        \task $\{4\} \subseteq A \cap E$
        \task $10 \in C - D$
        \task $A \cup B \supseteq C$
        \task $3 \in A \cap C$
        \task $0 \in (A - B) \cup D$
        \task $E \cap C \subseteq \mathbb{Z}$
        \task $0 \notin B - C$
    \end{tasks}
\end{exercise}

\begin{exercise}
    $\blacktriangleright$ 设 $m, n \in \mathbb{N}$,且 $m \le n$。请解释为什么 $\mathcal{P}([m]) \subseteq \mathcal{P}([n])$。
\end{exercise}

\begin{exercise}
    回顾 \ref{sec:section3.9} 节的问题 \ref{exc:exercises3.9.7}。我们已证明,若 $A \subseteq B$,则必有 $\mathcal{P}(A) \subseteq \mathcal{P}(B)$。请仔细阅读该证明并理解其细节。

    这种说法``反过来''是否成立?即假设 $\mathcal{P}(A) \subseteq \mathcal{P}(B)$,能否证明 $A \subseteq B$ 也成立?若不能,请构造反例。
\end{exercise}

\begin{exercise}
    用``集合构建符''重写以下定义。如果可能的话,直接列举集合元素;否则说明原因并给出三个示例元素。
    \begin{enumerate}[label=(\alph*)]
        \item 设 $A$ 为平方小于 $39$ 的所有自然数的集合。
        \item 设 $B$ 为方程 $x^2 - 3x - 10 = 0$ 所有实根的集合。
        \item 设 $C$ 为和为非负数的整数对的集合。
        \item 设 $D$ 为实数对的集合,其中第一个坐标为正,第二个坐标为负,且两个坐标和为正。
    \end{enumerate}
\end{exercise}

\begin{exercise}
    定义如下集合:
    \begin{align*}
        A &= \{x \in \mathbb{R} \mid x^2 - x - 12 < 0 \} \\
        B &= \{y \in \mathbb{R} \mid -3 < y < 4\}
    \end{align*}
    证明 $A = B$。
\end{exercise}

\begin{exercise}
    设 $X$ 为某学校全体学生的集合。
    \begin{itemize}
        \item 定义属性 $P(x)$,使得 $A := \{x \in X \mid P(x)\}$ 是 $X$ 的非空真子集。
        \item 定义属性 $Q(x)$,使得 $B := \{x \in X \mid Q(x)\}$ 是 $A$ 的非空真子集。
    \end{itemize}
\end{exercise}

\begin{exercise}
    设 $A, B, C$ 为集合,且 $A \subseteq C$, $B \subseteq C$。
    \begin{enumerate}[label=(\alph*)]
        \item 绘制集合 $\overline{A} \cap \overline{B}$ 和 $\overline{(A \cap B)}$ 的维恩图。
        \item 证明 $\overline{A} \cap \overline{B} \subseteq \overline{(A \cap B)}$。
        \item 构造集合 $A,B,C$,使得 $\overline{A} \cap \overline{B} \subset \overline{(A \cap B)}$。
        \item 构造集合 $A,B,C$,使得 $\overline{A} \cap \overline{B} = \overline{(A \cap B)}$。
    \end{enumerate}
\end{exercise}

\begin{exercise}
    令 $S = \{(m, n) \in \mathbb{Z} \times \mathbb{Z} \mid m = n^2\}$。$S$ 与集合 $T = \{(m, n) \in \mathbb{Z} \times \mathbb{Z} \mid n = m^2\}$ 的关系如何?如果一个是另一个的子集,请给出证明。如果不是,请给出反例。
\end{exercise}

\begin{exercise}
    令 $(a,b)$ 为笛卡尔平面上的一点,即 $(a,b) \in \mathbb{R} \times \mathbb{R}$。设 $\varepsilon$(希腊字母 \emph{epsilon})为非负实数,即 $\varepsilon \in \mathbb{R}$ 且 $\varepsilon \ge 0$。\\
    设 $C_{(a,b),\varepsilon}$ 为``接近'' $(a, b)$ 的点集,定义如下:
    \[C_{(a,b),\varepsilon} = \Big\{(x, y) \in \mathbb{R} \times \mathbb{R} \mid \sqrt{(x - a)^2 + (y - b)^2} < \varepsilon\Big\}\]
    \begin{enumerate}
        \item 给出集合 $C_{(a,b),\varepsilon}$ 的几何描述。
        \begin{itemize}
            \item 当我们改变 $a$ 和 $b$ 时,集合会如何变化?
            \item 当我们改变 $\varepsilon$ 时,集合会如何变化?
        \end{itemize}
        \item $C_{(0,0),1} \cap C_{(0,0),2}$ 是什么?
        \item $C_{(0,0),1} \cup C_{(0,0),2}$ 是什么?
        \item $C_{(0,0),1} \cap C_{(2,2),1}$ 是什么?
    \end{enumerate}
\end{exercise}

\begin{exercise}
    考虑如下(错误)命题:
    \[\bigcup_{n \in \mathbb{N}}\mathcal{P}([n]) = \mathcal{P}(\mathbb{N})\]
    \begin{enumerate}[label=(\alph*)]
        \item 下面的``证明''有什么问题?指出错误并解释其如何破坏了``证明''。
            \begin{spoof}
                首先证明 $\displaystyle{\bigcup_{n \in \mathbb{N}}\mathcal{P}([n]) \subseteq \mathcal{P}(\mathbb{N})}$。

                考虑左侧并集的任意元素 $X$。

                根据索引并集的定义,我们知道存在 $k \in \mathbb{N}$ 使得 $X \subseteq [k]$。

                由于 $[k] \subseteq \mathbb{N}$ 且 $X \subseteq [k]$,我们可以推断出 $X \subseteq \mathbb{N}$。

                因此 $X \in \mathcal{P}(\mathbb{N})$。

                接着证明 $\displaystyle{\bigcup_{n \in \mathbb{N}}\mathcal{P}([n]) \supseteq \mathcal{P}(\mathbb{N})}$。

                考虑任意元素 $Y \subseteq \mathcal{P}(\mathbb{N})$。

                由于 $Y$ 是自然数的子集,我们知道存在 $\ell \in \mathbb{N}$ 使得 $Y \subseteq [\ell]$。

                根据索引并集的定义,可得 $\displaystyle{Y \in \bigcup_{n \in \mathbb{N}}\mathcal{P}([n])}$。

                由于我们已经证明了 $\subseteq$ 和 $\supseteq$,因此两个集合相等。
            \end{spoof}
            $\quad$
        \item 通过构造集合 $S$ 的\textbf{具体}示例反驳该结论,使得
            \[S \in \mathcal{P}(\mathbb{N}) \qquad \text{且} \qquad S \notin \bigcup_{n \in \mathbb{N}}\mathcal{P}([n])\]
    \end{enumerate}
\end{exercise}

\begin{exercise}
    设 $A = [3] \times [4]$。(注意 $[n] = \{1, 2, 3, \dots , n\}$。)\\
    设 $B = \{(x, y) \in \mathbb{Z} \times \mathbb{Z} \mid 0 \le 3x - y + 1 \le 9\}$。
    \begin{enumerate}[label=(\alph*)]
        \item \textbf{证明} $A \subseteq B$。
        \item $A = B$ 吗?\textbf{证明}你的结论并说明原因。
    \end{enumerate}
\end{exercise}

\begin{exercise}
    令 $n \in \mathbb{N}$ 为固定自然数。设 $S = [n] \times [n]$。设 $T$ 为集合
    \[T =\Big\{(x, y) \in \mathbb{Z} \times \mathbb{Z} \mid 0 \le nx + y - (n + 1) \le n^2 - 1\Big\}\]
    证明 $S \subseteq T$ 但 $S \ne T$。
\end{exercise}

\begin{exercise}
    设 $A$ 和 $B$ 为集合。
    \begin{enumerate}[label=(\alph*)]
        \item \textbf{证明} 
        \[\mathcal{P}(A) \cup \mathcal{P}(B) \subseteq \mathcal{P}(A \cup B)\]
        \item 给出 $A$ 和 $B$ 的\textbf{明确}示例,使得 (a) 中的包含是\textbf{严格包含}。
    \end{enumerate}
\end{exercise}

\begin{exercise}
    设 $S$ 和 $T$ 为集合,且它们的本身也是集合。假设 $S \subseteq T$,\textbf{证明} 
    \[\bigcup_{X \in S} X \subseteq \bigcup_{Y \in T} Y\]
\end{exercise}

\begin{exercise}
    设 $A, B, C, D$ 为集合。
    \begin{enumerate}[label=(\alph*)]
        \item \textbf{证明} 
        \[(A \times B) \cup (C \times D) \subseteq (A \cup C) \times (B \cup D)\]
        \item 给出 $A,B,C,D$ 的\textbf{明确}示例,使得 (a) 中的包含是\textbf{严格包含}。
    \end{enumerate}
\end{exercise}

\begin{exercise}
    设 $A, B, C$ 为集合。证明
    \begin{align*}
        A \times (B \cap C) &= (A \times B) \cap (A \times C) \\
        A \times (B - C) &= (A \times B) - (A \times C)
    \end{align*}
\end{exercise}

\begin{exercise}\label{exc:exercises3.11.17} 
    设 $X,Y,Z$ 为集合。证明 $(X \cup Y ) - Z \subseteq X \cup (Y - Z)$,但两者\emph{不一定}相等。
\end{exercise}

\begin{exercise}
    找出集合 $S$ 的示例,使得 $S \in \mathcal{P}(\mathbb{N})$ 且 $S$ 恰好拥有 $4$ 个元素。
    接着找出集合 $T$ 的示例,使得 $T \subseteq \mathcal{P}(\mathbb{N})$ 且 $T$ 恰好拥有 $4$ 个元素。
\end{exercise}

\begin{exercise}
    给出集合 $R,S,T$ 的示例,使得 $R \in S$,$S \in T$,$R \subseteq T$,$R \notin T$。
\end{exercise}

\begin{exercise}
    确定下列集合是什么,并证明你的结论。
    \[\bigcap_{n \in \mathbb{N}}[n] \qquad \text{和} \qquad \bigcup_{n \in \mathbb{N}}[n]\]
\end{exercise}

\begin{exercise}
    设 $I = \{-1, 0, 1\}$。对于每个 $i \in I$,定义 $A_i = \{i - 2, i - 1, i, i + 1, i + 2\}$ 和 $B_i = \{-2i, -i, i, 2i\}$。
    \begin{enumerate}[label=(\alph*)]
        \item 写出 $\displaystyle{\bigcup_{i \in I}A_i}$ 的元素。
        \item 写出 $\displaystyle{\bigcap_{i \in I}A_i}$ 的元素。
        \item 写出 $\displaystyle{\bigcup_{i \in I}B_i}$ 的元素。
        \item 写出 $\displaystyle{\bigcap_{i \in I}B_i}$ 的元素。
        \item 利用上述答案,写出 $\displaystyle{\Big(\bigcup_{i \in I}A_i\Big) - \Big(\bigcup_{i \in I}B_i\Big)}$ 的元素。
        \item 利用上述答案,写出 $\displaystyle{\Big(\bigcap_{i \in I}A_i\Big) - \Big(\bigcap_{i \in I}B_i\Big)}$ 的元素。
        \item 写出 $\displaystyle{\bigcup_{i \in I}(A_i - B_i)}$ 的元素。与 (e) 的答案有何不同?
        \item 写出 $\displaystyle{\bigcap_{i \in I}(A_i - B_i)}$ 的元素。与 (f) 的答案有何不同?
    \end{enumerate}
\end{exercise}

\begin{exercise}\label{exc:exercises3.11.22} 
    在这道题中,我们要``证明''负整数的存在性!我们说``证明''是因为需要完成后续学习才会完全理解这一点,但请相信,这就是我们正在做的事。因此,不能\textbf{假设}存在任何严格小于 $0$ 的整数,故代数步骤(尤其是 (d) 部分)不应涉及可能为负的项。也就是说,若考虑方程
    \[x + y = x + z\]
    我们\textbf{可以}通过两边同时减去 $x$ 推导出 $y = z$,因为 $x - x = 0$。但若考虑方程
    \[x + y = z + w\]
    我们\textbf{不能}推导出 $x - z = w - y$。因为若 $y > w$,则 $w - y$ 在上下文中可能不存在……\\
    \\
    设 $P = \mathbb{N} \times \mathbb{N}$。定义集合 $R$ 为
    \[R = \big\{\big((a, b),(c, d)\big) \in P \times P \mid a + d = b + c \big\}\]
    \begin{enumerate}[label=(\alph*)]
        \item 找出三个不同的 $(c, d)$ 对,使得 $\big((1, 4),(c, d)\big) \in R$。
        \item 设 $(a, b) \in P$。证明 $\big((a, b),(a, b)\big) \in R$。
        \item 设 $\big((a, b),(c, d)\big) \in R$。证明 $\big((c, d),(a, b)\big) \in R$。
        \item 假设 $\big((a, b),(c, d)\big) \in R$ 且 $\big((c, d),(e, f)\big) \in R$。证明 $\big((a, b),(e, f)\big) \in R$。
    \end{enumerate}
\end{exercise}