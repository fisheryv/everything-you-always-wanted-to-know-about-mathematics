% !TeX root = ../../../book.tex
\subsection{目标}

以下内容简要说明本章在本书中的定位。我们将解释前期工作如何为本章研究奠定基础,阐明探讨本章主题的动机,并概述学习目标及注意事项。我们会先总结本章的主要目标,概括你在学完本章后应掌握的技能与知识。后续章节将详细展开这些思想,此处仅提供一个简要列表作为学习指引。完成本章后,请你返回此列表,确认自己是否达成了所有目标。你是否能理解这些目标的重要性?能否清晰地解释相关术语并熟练地应用相关技术?

\textbf{学完本章后,你应该能够……}

\begin{itemize}
    \item 定义什么是集合,并列举常见实例。
    \item 运用规范符号定义集合并引用其元素。
    \item 定义并描述常见集合运算,即由多个集合构造新集合的方法。
    \item 阐述集合比较方法,并运用恰当技术证明相关结论。
    \item 阐释自然数与集合的关联,并建立其与数学归纳法的联系。
\end{itemize}
