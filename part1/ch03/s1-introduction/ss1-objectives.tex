% !TeX root = ../../../book.tex
\subsection{目标}

以下简要说明本章在本书中的定位:它将阐释先前内容如何发挥作用,阐述研究本章主题的动机,明确学习目标,并提示阅读时的关注重点。我们先列出本章的核心目标及学成后应掌握的技能,后续章节将详细展开。学完本章后,请返回此处核验:你是否理解所有目标?能否阐述其重要性?能否定义相关术语?能否运用相关技术?

\textbf{学完本章后,你应该能够……}

\begin{itemize}
    \item 定义什么是集合,并列举常见实例。
    \item 运用规范符号定义集合并引用其元素。
    \item 定义并描述常见集合运算,即由多个集合构造新集合的方法。
    \item 阐述集合比较方法,并运用恰当技术证明相关结论。
    \item 阐释自然数与集合的关联,并建立其与数学归纳法的联系。
\end{itemize}
