% !TeX root = ../../../book.tex
\subsection{目标}

以下简短内容将向你展示本章如何融入本书的体系。这部分内容会描述我们之前的工作将如何发挥作用,还会激发我们为什么要研究本章出现的主题,并告诉你我们的目标,以及你在阅读时应该记住什么来实现这些目标。现在,我们将通过一系列陈述为你总结本章的主要目标,以及本章结束时你应该获得的技能和知识。以下各节将更详细地重申这些想法,但这里将为你提供一个简短的列表以供将来参考。当学完本章后,请返回此列表,看看你是否理解所有这些目标。你明白为什么我们在这里概述它们很重要吗?你能定义我们使用的所有术语吗?你能应用我们描述的技术吗?

\textbf{学完本章后,你应该能够……}

\begin{itemize}
    \item 定义什么是集合,并给出几个常见的例子。
    \item 使用正确的符号来定义集合并引用其元素。
    \item 定义并描述常见集合操作;即用两个或多个集合创建新集合的方法。
    \item 描述如何比较两组两个集合,并应用恰当的技术来证明此类观点。
    \item 解释自然数与集合的关系,并将其与数学归纳法联系起来。
\end{itemize}
