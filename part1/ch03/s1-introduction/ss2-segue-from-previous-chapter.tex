% !TeX root = ../../../book.tex
\subsection{承上}

我们致力于构建数学归纳法的形式化表述,并将其证明为\emph{定理}。为此,我们需要一些基础对象以便进行严谨的逻辑处理与讨论——集合正是这样的对象!历史上,数学于二十世纪初才建立在\emph{集合论}的基础之上。此前,数学家往往对其工作的底层逻辑``不予深究'',他们大量依赖``直觉性''假设,但未尝试严格、\emph{公理化}地描述理论体系。\textbf{乔治·康托尔 (Georg Cantor)} 的研究揭示了若干反直觉却逻辑自洽的结论,促使数学界认识到明确理论根基的必要性。这绝非否定 1900 年前数学家的成就——他们如同进行一场游戏,却从未预先统一规则。集合论的\emph{公理体系}正是为此而生。
