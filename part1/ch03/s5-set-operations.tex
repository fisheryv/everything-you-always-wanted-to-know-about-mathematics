% !TeX root = ../../book.tex
\section{集合运算}\label{sec:section3.5}

当你初次学习数字时,很自然地就会学到如何\emph{组合}它们:乘法、加法等等。因此,我们接下来自然要研究如何将两个集合通过\emph{运算}生成其他集合。我们如何以有趣的方式组合集合?有几种这样的运算具有标准的符号,我们现在就介绍这些运算。

本节中,我们假设给定两个集合 $A$ 和 $B$,它们都是\emph{全集} $U$ 的子集。也就是说,我们假设 $A \subseteq U$ 且 $B \subseteq U$。我们做出这个假设的原因是,每个运算都涉及通过识别具有特定属性的较大集合的元素来定义另一个集合,因此我们必须有一个保证包含 $A$ 和 $B$ 所有元素的集合 $U$,以便我们可以使用这些元素。(再次强调,确保这一点可能看起来很苛刻,但这是为了避免出现像我们之前研究的令人讨厌的悖论。)假设这些集合 $A, B,U$ 存在,我们才可以继续我们的定义。

\subsection{交集}

此运算提取两个集合共有的元素并将它们包含在一个新集合中,称为\textbf{交集}。

\begin{definition}
    设 $A, B$ 为任意集合。$A$ 和 $B$ 的\dotuline{交集}是同时属于 $A$ 和 $B$ 的元素的集合,用 $A \cap B$ 表示。用数学符号表达如下:
    \[A \cap B = \{x \in U \mid x \in A \;\text{且}\; x \in B\}\]
\end{definition}

\begin{example}\label{ex:example3.5.1}
    定义如下集合:
    \begin{align*}
        S_1 &= \{1, 2, 3, 4, 5\}\\
        S_2 &= \{1, 3, 7\}\\
        S_3 &= \{2, 4, 7\}\\
        U &= \mathbb{N}
    \end{align*}
    那么,我们可得
    \begin{align*}
        S_1 \cap S_2 &= \{1, 3\} \\
        S_1 \cap S_3 &= \{2, 4\} \\
        S_2 \cap S_3 &= \{7\}
    \end{align*}
    此外,由于交集本身也是一个集合,因此可以与其他集合再次进行交集运算,比如 $(S_1 \cap S_2) \cap S_3$ 是有意义的。然而,这两个集合没有共同元素,所以我们可以写做
    \[(S_1 \cap S_2) \cap S_3 = \varnothing\]
\end{example}

如上例所示,两个集合没有共同元素的情况很常见,因此我们有一个特定的术语来描述此类集合:

\begin{definition}
    如果 $A \cap B = \varnothing$,则我们说 $A$ 和 $B$ \dotuline{不相交}。
\end{definition}

\subsubsection*{交集与子集}

你可能已经观察到,无论 $A$ 和 $B$ 是什么,我们都有 $A \cap B \subseteq A$ 且 $A \cap B \subseteq B$。让我们证明这一事实!

\begin{proposition}
    设 $A, B$ 为任意集合。则 $A \cap B \subseteq A$ 且 $A \cap B \subseteq B$。
\end{proposition}

顺带一提,\textbf{命题}只是``微小的结果''。它并不困难或重要到足以被称为定理,但它确实需要一点证明。

\begin{proof}
    假设我们有两个集合,$A$ 和 $B$。为了证明子集关系,例如 $A \cap B \subseteq A$,我们需要证明左边集合 $(A \cap B)$ 的每个\dotuline{元素}也是右边集合 $(A)$ 的元素。

    让我们考虑任意元素 $x \in A \cap B$。根据 $A \cap B$ 的定义,我们知道 $x \in A$ 和 $x \in B$。因此,我们知道 $x \in A$。这就是我们要证明的目标,所以我们证明了 $A \cap B \subseteq A$。

    同理,我们也知道 $x \in B$,因此我们也证明了 $A \cap B \subseteq B$。
\end{proof}

这看起来像是单纯的观察和简单的证明,但我们仍然需要通过这些逻辑步骤来严格解释为什么这些子集关系成立。另外,请注意我们此处使用的\textbf{证明结构}。为了证明子集关系成立,我们需要考虑集合的\textbf{任意元素}并推断它也是另一个集合的元素。这将是我们证明有关子集的任何命题的方法。

如果 $A \subseteq B$ 呢?$A \cap B$ 与 $A$ 和 $B$ 有什么关系?尝试证明这一点!

\subsection{并集}

此运算提取两个集合中的元素并将它们包含在一个新集合中,称为\textbf{并集}。

\begin{definition}
    设 $A, B$ 为任意集合。$A$ 和 $B$ 的\dotuline{并集}是属于 $A$ 或 $B$ 的元素的集合,用 $A \cup B$ 表示。用数学符号表达如下:
    \[A \cup B = \{x \in U \mid x \in A \;\text{或}\; x \in B\}\]
\end{definition}

请注意,定义中的``或''是\emph{包含}``或'',这意味着 $A \cup B$ 包括任何属于 $A$ 或 $B$ 或可能同时属于这两个集合的元素。\\

\begin{example}
    回到我们在例 \ref{ex:example3.5.1} 中定义的集合 $S_1, S_2, S_3$,我们可以说
    \begin{align*}
        S_1 \cup S_2 &= \{1, 2, 3, 4, 5, 7\} \\
        S_1 \cup S_3 &= \{1, 2, 3, 4, 5, 7\} \\
        S_2 \cup S_3 &= \{1, 2, 3, 4, 7\}
    \end{align*}
    此外,由于并集本身也是一个集合,因此可以与其他集合再次进行并集运算,例如
    \[(S_1 \cup S_2) \cup S_3 = \{1, 2, 3, 4, 5, 7\} \cup  \{2, 4, 7\} =  \{1, 2, 3, 4, 5, 7\}\]
\end{example}

\subsubsection*{并集与子集}

请注意,无论 $A$ 和 $B$ 是什么,都有 $A \subseteq (A \cup B)$ 和 $B \subseteq (A \cup B)$。让我们证明一下!

\begin{proposition}
    设 $A, B$ 为任意集合。则 $A \subseteq (A \cup B)$ 且 $B \subseteq (A \cup B)$。
\end{proposition}

\begin{proof}
    假设我们有两个集合 $A$ 和 $B$。为了证明 $A \subseteq (A \cup B)$,我们需要证明 $A$ 的每个元素也是 $A \cup B$ 的元素。

    设任意固定元素 $x \in A$。则必有 $x \in A$ 或 $x \in B$(因为已知 $x \in A$)。这表明 $x \in A \cup B$。由于 $x$ 是任意的,因此我们证明了 $A \subseteq A \cup B$。

    设任意固定元素 $y \in B$。则必有 $y \in A$ 或 $y \in B$(因为已知 $y \in B$)。这表明 $y \in A \cup B$。由于 $y$ 是任意的,因此我们证明了 $B \subseteq A \cup B$。
\end{proof}

你能说一下 $A \cap B$ 和 $A \cup B$ 之间的关系吗?如果 $A \subseteq B$,那么 $B$ 与 $A \cup B$ 之间有什么关系?尝试证明你的观察!

这里需要强调一点,像这样的主张 --- 对于任意集合 $A$ 和 $B$, $A \subseteq A \cup B$ --- 是需要证明的;\textbf{根据定义}它们不是显然成立的。上面给出了两个集合的并集的定义。请注意,它没有说明 $A$ 和 $A \cup B$ 之间的关系;定义只是告诉我们对象 $A \cup B$ 实际上是什么。当你调用或引用定义并使用它时,请务必这样做;而且,一定要解释任何不完全来自定义的主张。既然我们已经证明了这两个小引理,我们就可以在将来通过引用来使用它们;如果我们不这样做,我们每次尝试引用这些小事实时都必须重新解释它们!

\subsection{差集}

此运算提取一个集合的元素并删除也属于另一集合的元素。

\begin{definition}
    $A$ 和 $B$ 的差集用 $A - B$ 表示,是 $A$ 中所有不属于 $B$ 的元素的集合。用数学符号表达如下:
    \[A - B := \{x \in U \mid x \in A \;\text{且}\; x \notin B\}\]
\end{definition}

\begin{example}
    回到我们在例 \ref{ex:example3.5.1} 中定义的集合 $S_1, S_2, S_3$,我们可以说
    \begin{align*}
        S_1 - S_2 &= \{2, 4, 5\} \\
        S_1 - S_3 &= \{7\} \\
        S_2 - S_3 &= \{1,3\}
    \end{align*}
\end{example}

\subsubsection*{差集不对称}

请注意,上例中的 $S1 - S2 \ne S2 - S1$。一般来说,在集合的上下文中,运算``$-$''不是对称的,这里的例子表明了这一点。你能找到两个集合 $A, B$ 使得 $A - B = B - A$ 吗?你能找到两个集合 $A, B$ 使得 $A - B = B - A = ∅$ 吗?

到目前为止,我们定义的其他运算实际上都是对称的。也就是说,$A \cap B = B \cap A$ 且 $A \cup B = B \cup A$。回顾一下这些运算的定义,看看为什么这是合理的。定义中哪部分\emph{语言}使得对称性成立?

\subsubsection*{注释}

差集表示法还有一点需要注意。虽然我们使用标准减法符号``$-$'',但这里的减法符号与我们通常认为的``减法''(如数字)毫不相关。这可能是你第一次遇到这种歧义,也可能不是,但请记住这个与数学符号和术语相关的重要观点:许多符号根据\emph{上下文}不同而具有不同的含义。

当我们写 $7 - 5$ 时,我们显然指的是减法,即 $7 - 5 = 2$。然而,当我们写 $A - A$ 其中 $A$ 被识别为\emph{集合}时,我们指的是差集运算,即 $A - A = \varnothing$。请务必检查语句的上下文,以确保其中的符号确实具有你认为的含义!

\subsection{补集}

此运算识别位于集合``外部''的所有元素。此操作取决于全集 $U$ 的上下文。你会注意到这在定义中很明显,我们也将通过示例来说明这一点。

\begin{definition}
    $A$ 的\dotuline{补集}是所有不是 $A$ 中元素的元素的集合,记为 $\overline{A}$。用数学符号表达如下:
    \[\overline{A} = \{x \in U \mid x \notin A\}\]
\end{definition}

请记住,我们假设 $A,B,U$ 是满足 $A \subseteq U$ 且 $B \subseteq U$ 的给定集合。在这种情况下,集合 $\overline{A}$ 是明确定义的,该集合必定取决于 $A$ 和 $U$!\\

\begin{example}
    例如,让我们回到上面例 \ref{ex:example3.5.1} 中定义的集合 $S_1, S_2, S_3$。在那里,我们使用了上下文 $U = \mathbb{Z}$。在这种情况下,
    \[\overline{S_1} = \{6, 7, 8, 9, \dots \}\]
    然而,如果我们另 $U = \{1, 2, 3, 4, 5, 6, 7\}$,在这种情况下,
    \[\overline{S_1} = \{6, 7\}\]
\end{example}

由于符号 $\overline{A}$ 没有指示它所依赖的全集 $U$,因此无论上下文如何,明确该全集都很重要。尝试给出集合 $A, U_1, U_2$,使得 $U_1$ 下的 $\overline{A}$ 与 $U_2$ 下的 $\overline{A}$ 不同,并尝试给出一些集合,使得两种情况下 $\overline{A}$ 相同。

\subsection{习题}

\subsubsection*{温故知新}

以口头或书面的形式简要回答以下问题。这些问题全都基于你刚刚阅读的内容,所以如果忘记了具体的定义、概念或示例,可以回去重读相关部分。确保在继续学习之前能够自信地回答这些问题,这将有助于你的理解和记忆!

\begin{enumerate}[label=(\arabic*)]
    \item 两个集合的并集和交集有什么区别?
    \item 两个集合不相交意味着什么?
    \item $\mathbb{Z} \cap \mathbb{N}$ 是什么? $\mathbb{Z} \cup \mathbb{N}$ 是什么? $\mathbb{Z}-\mathbb{N}$ 是什么?
    \item $A - B = B - A$ 可能成立吗?什么情况下成立?
    \item $\mathbb{N}$ 下的 $\overline{[3]}$ 是什么?上下文换成 $\mathbb{Z}$ 或 $\mathbb{R}$ 呢?尝试使用恰当的数学符号和集合构建符来写下你的答案。
    \item $(A \cap B) \cap C = A \cap (B \cap C)$ 永远成立吗?为什么成立或者为什么不成立? 用 $\cup$ 代替 $\cap$ 会怎样?
    \item ``$7-5$'' 与 ``$[7]-[5]$'' 有什么区别?
    \item 假设 $x \in A$。$A - x$ 有意义吗?如何改变才有意义?
    \item $(\mathbb{Z} - \mathbb{N}) \cup \mathbb{R}$ 是什么?
\end{enumerate}

\subsubsection*{小试牛刀}

尝试回答以下问题。这些题目要求你实际动笔写下答案,或(对朋友/同学)口头陈述答案。目的是帮助你练习使用新的概念、定义和符号。题目都比较简单,确保能够解决这些问题将对你大有帮助!

\begin{enumerate}[label=(\arabic*)]
    \item 列出下列集合的元素:
        \begin{enumerate}[label=(\alph*)]
            \item $[7] \cup [10]$
            \item $[10] \cap [7]$
            \item $[10] - [7]$
            \item $([12] - [3]) \cap [8]$
            \item $(\mathbb{N} - [3]) \cap [7]$
            \item $(\mathbb{Z}-\mathbb{N}) \cap N$
            \item $\mathbb{Z}$ 下 $\overline{\mathbb{N}} \cap \{0\}$
        \end{enumerate}
    \item 找到集合 $A,B,C$ 的示例,使得 $(A - B) - C = A - (B - C)$。然后,找一个它们不相等的例子。
    \item 陈述并证明 $\overline{A}$ 和 $U - A$ 之间的关系。
    \item 设 $A = [12]$,设 $E$ 为偶数集合,并设 $P$ 为质数集合。$A \cap E$ 是什么?$A \cap P$ 是什么?$(A \cap E) \cap P$ 是什么?它和 $A \cap (E \cap P)$ 一样吗?\\
    假设上下文为 $U = \mathbb{N}$。$\overline{A \cap E}$ 和 $\overline{A} \cap \overline{E}$ 是什么?
    \item $ \{1\} \cap \mathcal{P}(\{1\})$ 是什么?
    \item 考虑集合 $\{1\}$ 和 $\{2, 3\}$。比较集合 $\mathcal{P}(\{1\} \cup \{2, 3\})$ 和 $\mathcal{P}(\{1\}) \cup \mathcal{P}(\{2, 3\})$。你注意到了什么?\\
    用 $\cap$ 替换 $\cup$ 重复上面的问题,你又发现了什么?\label{exc:exercises3.5.6}
    \item 设 $A, U$ 为集合,并假设 $A \subseteq U$。在上下文 $U$ 下,设 $B = \overline{A}$。你认为 $\overline{B}$ 是什么?为什么?
\end{enumerate}
