% !TeX root = ../../../book.tex
\subsection{习题}

\subsubsection*{温故知新}

以口头或书面的形式简要回答以下问题。这些问题全都基于你刚刚阅读的内容,所以如果忘记了具体的定义、概念或示例,可以回去重读相关部分。确保在继续学习之前能够自信地回答这些问题,这将有助于你的理解和记忆!

\begin{enumerate}[label=(\arabic*)]
    \item 什么是归纳集?举一个非 $\mathbb{N}$ 非 $\mathbb{Z}$ 的例子
    \item 在数学归纳原理的证明中,我们定义 $S = \{n \in \mathbb{N} \mid P(n) \text{为真}\}$ 是什么意思?用文字描述一下这个集合。
    \item 对于归纳法的工作原理,想出你自己的类比。
\end{enumerate}

\subsubsection*{小试牛刀}

尝试回答以下问题。这些题目要求你实际动笔写下答案,或(对朋友/同学)口头陈述答案。目的是帮助你练习使用新的概念、定义和符号。题目都比较简单,确保能够解决这些问题将对你大有帮助!

\begin{enumerate}[label=(\arabic*)]
    \item 如果我们将后继的定义更改为 $S(X) = {X}$ 会怎样?用 $0 = \varnothing$,集合中的 $1,2,3,4$ 分别代表什么?它们还满足等式 $n = \{0, 1, \dots, n - 1\}$吗?如果不满足,他们是否满足其他关系?探索一下!
    \item 与朋友讨论一下无限集是否存在。为什么我们需要\emph{假设}存在归纳集来定义 $\mathbb{N}$?这对你来说有效吗?从物理上讲,这有意义吗?从数学上讲呢?
    \item 考虑一个简单的算术陈述,例如 $1 + 2 = 3$。以集合的形式写出数字 $1,2,3$,并看看这个等式有何意义。在这种情况下,``$+$'' 是什么意思?
    \item 研究如何使用 $\mathbb{N}$ 来定义 $\mathbb{Z}$。在网上或书中进行一些探索,或者自己提出一个想法。
\end{enumerate}