% !TeX root = ../../../book.tex
\subsection{定义}

如何用集合\emph{定义}自然数?直观上我们理解其概念:从 $1$ 起始,不断加 $1$ 即可得到所有自然数。因此,必须从集合角度明确``$1$''与``加 $1$''的含义。为此,先引入 $0$。尽管我们约定 $\mathbb{N}$ 不含 $0$(部分学者包含),但眼下用它来推导 $\mathbb{N}$ 更为方便。已知空集是不含元素的唯一集合,因此将 $0$ 与空集\emph{关联}是合理的;我们直接\emph{定义} $0 = \varnothing$。

接下来定义 $1$。仿照 $0$ 的定义,可用仅含一个元素的集合(称为\textbf{单例集})表示。这里有几个候选:
\[\{\varnothing\}, \{\{\varnothing\}\} , \{\{\varnothing, \{\varnothing\}\}\}\]
如何选择表示 $1$ 的单例集?考虑到后续需基于已有数字定义 $2$、$3$ 等,而当前唯一可用对象是 $0$,故合理定义为:
\[1 = \{0\} = \{\varnothing\}\]
这确保了 $0 \ne 1$。

对于 $2$,需选用包含两个元素的集合,例如:
\[\{\varnothing, \{\varnothing\}\}, \{\varnothing, \{\{\varnothing\}\}\}, \{\{\varnothing\}, \{\{\varnothing\}\}\}\]
等等。在众多可能中,我们注意到第一个集合恰好包含已定义对象 $0$ 与 $1$!因此,定义 $2 = \{0, 1\}$ 是自然的选择,且满足 $2 \ne 0$ 与 $2 \ne 1$。

\subsubsection*{后继}

后继概念提供了如何继续这一过程并定义任意自然数的直观想法:对于任意 $n \in \mathbb{N}$,我们定义
\[n = \{0, 1, 2, \dots , n - 2, n - 1\}\]
然而,给定一个集合,用这个定义检验它是否代表自然数会相当困难。我们希望为 $\mathbb{N}$ 的元素提供\emph{更好的}定义;给定任意集合,我们想知道它是否属于 $\mathbb{N}$。回顾元素 $n$ 的定义,我们可以写
\[n = \{0, 1, 2, \dots , n - 2, n - 1\} = \{0, 1, 2, \dots , n - 2\} \cup \{n-1\} = (n-1) \cup \{n-1\}\]
这样,我们就得到一种通过前一个元素和集合运算来定义 $\mathbb{N}$ 中任意元素的自然方法。这引出了以下定义。

\begin{definition}
    给定任意集合 $X$, $X$ 的后继记为 $S(X)$,定义为 $S(X) = X \cup \{X\}$。
\end{definition}

这个定义适用于所有集合,但在自然数的背景下,它意味着 $n$ 的后继正是我们直观``所知''的更大自然数,即 $n + 1$。

\subsubsection*{归纳集}

这使我们更接近 $\mathbb{N}$ 的定义。我们自然要求 $1 \in \mathbb{N}$,并且对于任意 $n \in \mathbb{N}$,有 $S(n) \in \mathbb{N}$。为了形式化这一点,我们给出以下定义:

\begin{definition}
    $I$ 为\dotuline{归纳集},当且仅当:
    \begin{enumerate}
        \item $1 \in I$
        \item 若 $n \in I$,则 $S(n) \in I$
    \end{enumerate}
\end{definition}

显然,$\mathbb{N}$ 本身是一个归纳集。还有其他归纳集吗?考虑这个问题:它们会有什么性质?会包含非自然数的元素吗?我们不想深入讨论这些问题,但这里我们指出,确实存在其他归纳集。我们不希望这些集合中的任何一个等于 $\mathbb{N}$,因此我们给出以下定义:

\begin{definition}
    所有\dotuline{自然数}的集合定义为
    \[\mathbb{N} := \{x \mid \text{对于任意归纳集\ } I, x \in I\}\]
    换句话说,$\mathbb{N}$ 是最小归纳集,在集合包含的意义下:
    \[\mathbb{N} = \bigcap_{I \in \{S \mid S \text{为归纳集}\}} I\]
    这表明 $\mathbb{N}$ 是所有归纳集的子集。
\end{definition}

这提供了我们想要的``检验属性''。任意集合 $x$ 是自然数(即 $x \in \mathbb{N}$)当且仅当它属于每个归纳集,也就是说,对于每个归纳集 $I$,有 $x \in I$。这也表明,对于每个归纳集 $I$, $\mathbb{N} \subseteq I$。

这里可以进一步讨论其他集合论问题:我们如何知道这样一个无限集存在?(实际上,我们需要将其作为集合论的\emph{公理}!)假设这类集合存在,我们如何表征那些不同于 $\mathbb{N}$ 的归纳集?解决这些问题超出了本课程的范围和目标,因此不作讨论。不过,我们现在会介绍 $\mathbb{N}$ 的一些性质,特别是那些对严格构建数学归纳法有用的性质。(若想探索,可考虑整数集 $\mathbb{Z}$,尝试解释其归纳性;$\mathbb{R}$ 呢?$\mathbb{Z} \setminus \mathbb{N}$ 呢?)

\subsubsection*{$\mathbb{N}$ 的性质}

在定义归纳原理之前,先考虑自然数的常见性质与用途:排序与算术。给定任意两个自然数,可比较并确定其大小关系(或相等性),通常用 $1 < 3$、$1 \le 5$、$4 \nless 2$、$3 = 3$ 等符号表示。

既然已将 $\mathbb{N}$ 的元素定义为集合,能否用集合表达这些比较?可以!回顾后继的定义,其本身包含 $X \in S(X)$ 这一事实。由此得到以下定义:

\begin{definition}
    给定两个自然数 $m, n \in \mathbb{N}$,当且仅当 $m \in n$ 时,记 $m < n$。
\end{definition}

这定义了 $\mathbb{N}$ 上的序关系,我们将在第 \ref{sec:section6.3} 节详细讨论关系和序的概念。

至于算术:对于集合 $m$ 和 $n$,$m + n$ 是什么?如何定义该运算及其输出?如何确保 $m + n$ 仍是自然数?能否确定 $m + n = n + m$?这些问题将在后续讨论函数和关系时解决。
