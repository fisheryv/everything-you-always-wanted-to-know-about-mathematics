% !TeX root = ../../../book.tex
\subsection{定义}

我们如何用集合来\emph{定义}自然数?仅凭直觉我们就知道它们是什么。我们从 $1$ 开始,反复加 $1$,得到所有其他自然数。因此,我们必须从集合的角度来确定 ``$1$'' 的含义和 ``加 1'' 的含义。为此,我们首先考虑 $0$。我们之前说过,我们不会在集合 $\mathbb{N}$ 中包含 $0$,但有些作者会这样做,眼下这有助于我们用它来推导 $\mathbb{N}$。我们知道有一种不包含任何元素的集合,即空集。因此,将 $0$ 与空集\emph{关联起来}是合理的;事实上,我们\emph{定义} $0 = \varnothing$。接下来,我们希望定义 $1$,效仿 $0$ 的定义,我们用一个仅包含一个元素的集合来表示。(包含一个元素的集合也称为\textbf{单例}。)这里几个这样的集合:
\[\{\varnothing\}, \{\{\varnothing\}\} , \{\{\varnothing, \{\varnothing\}\}\}\]
我们如何选择一个单例来表示 $1$ 呢?请记住,我们希望这个过程持续下去并最终根据之前的数字定义 $2$(和 $3$ 等),现在根据我们可以使用的唯一对象 $0$ 来定义 $1$ 是合理的。因此,我们\emph{选择}这么定义
\[1 = \{0\} = \{\varnothing\}\]
这保证了 $0 \ne 1$。

接下来定义 $2$,我们考虑包含两个元素的集合,例如
\[\{\varnothing, \{\varnothing\}\}, \{\varnothing, \{\{\varnothing\}\}\}, \{\{\varnothing\}, \{\{\varnothing\}\}\}\]
等等等等。在这么多集合中,我们需要寻找一个自然的表示,我们注意到上面列出的第一个集合包含我们已经定义的两个对象,$0$ 和 $1$!因此,定义 $2 = \{0, 1\}$ 是一个自然的选择,并且我们再次可知 $2 \ne 0$ 且 $2 \ne 1$。

\subsubsection*{后继}

后继给了我们如何继续这一过程并从中定义任意自然数的直观想法:对于任意 $n \in \mathbb{N}$,我们定义
\[n = \{0, 1, 2, \dots , n - 2, n - 1\}\]
然而,给定一个集合,使用这个定义来检验该集合是否代表自然数将是相当困难的。我们希望对 $\mathbb{N}$ 的元素有\emph{更好的}定义;我们想知道,给定任意集合,它是否属于 $\mathbb{N}$。回顾上面的元素 $n$;我们也可以写
\[n = \{0, 1, 2, \dots , n - 2, n - 1\} = \{0, 1, 2, \dots , n - 2\} \cup \{n-1\} = (n-1) \cup \{n-1\}\]
瞧!我们得到一种根据前一个元素和集合运算来定义 $\mathbb{N}$ 中元素的自然方法。这引出了以下定义。

\begin{definition}
    给定任何集合 $X$, $X$ 的后继用 $S(X)$ 表示,定义为 $S(X) = X \cup \{X\}$。
\end{definition}

这个定义适用于所有集合,但在自然数的背景下,它意味着 $n$ 的后继正是我们直观``所知''的更大的自然数,即 $n + 1$。

\subsubsection*{归纳集}

这使我们更接近 $\mathbb{N}$ 的定义。我们当然想要 $1 \in \mathbb{N}$,并且对于任意元素 $n \in \mathbb{N}$,我们也想要 $S(n) \in \mathbb{N}$。为了符号化地对此进行编码,我们做出以下定义:

\begin{definition}
    $I$ 为\dotuline{归纳}
    \begin{enumerate}
        \item $1 \in I$
        \item 如果 $n \in I$,则 $S(n) \in I$。
    \end{enumerate}
\end{definition}

显然,$\mathbb{N}$ 本身是一个归纳集。还有其他归纳集吗?思考以下这个问题。它们会有什么属性呢?它们会包含非自然数的元素吗?我们不想深入讨论这些问题,但为了这里的讨论,我们将指出确实存在其他归纳集。我们不希望这些集合中的任何一个为 $\mathbb{N}$,因此我们做出以下定义:

\begin{definition}
    所有\dotuline{自然数}的集合是
    \[\mathbb{N} := \{x \mid \text{对于任意归纳集} I, x \in I\}\]
    换句话说,$\mathbb{N}$ 是最小归纳集,从集合包含的意义上:
    \[\mathbb{N} = \bigcap_{I \in \{S \mid S \text{为归纳集}\}} I\]
    这表明 $\mathbb{N}$ 是所有归纳集的子集。
\end{definition}

这为我们提供了我们想要的``检验属性''。任意集合 $x$ 是自然数(即 $x \in \mathbb{N}$)当且仅当它是\emph{每个}归纳集的元素(即对于每个归纳集合 $I$, $x \in I$)。这也告诉我们,对于每个归纳集 $I$, $\mathbb{N} \subseteq I$。

这里还可以进行一些其他集合论方面的讨论:我们如何知道这样一个无限集存在?(实际上,我们需要将此作为集合论的\emph{公理}!假设这些类型的集合存在,我们如何表征那些非 $\mathbb{N}$ 的其他归纳集?解决这些问题超出了本课程的范围和目标,因此我们不会在这里解决这些问题。但是,我们现在会提及 $\mathbb{N}$ 的一些属性,尤其是那些对严格构建数学归纳法有用的属性。(如果你想了解,请考虑整数集 $\mathbb{Z}$。尝试解释为什么这个集合是归纳性的。$\mathbb{R}$ 呢?$\mathbb{Z} - \mathbb{N}$ 呢?)

\subsubsection*{$\mathbb{N}$ 的性质}

在定义归纳原理之前,让我们先思考一下自然数的一些常见性质和用途:排序和算术。给定任意两个自然数,我们可以比较它们并确定哪一个更大,哪一个更小(或者它们是否相等)。我们通常用 $1 < 3, 1 \le 5, 4 \nless 2, 3 = 3$ 等符号来书写。

已知我们已经将 $\mathbb{N}$ 的元素本身定义为集合,我们是否可以用集合来表达这些比较?这是可以的!回顾一下后继的定义。该定义中内置的事实是 $X \in S(X)$!这一发现给了我们以下定义:

\begin{definition}
    给定两个自然数 $m, n \in \mathbb{N}$,当且仅当 $m \in n$ 时,我们写 $m < n$。
\end{definition}

这定义了集合 $\mathbb{N}$ 上的顺序关系。我们将在本书后面讨论关系和顺序的概念(第 \ref{sec:section6.3} 节)。

那算术呢?就集合 $m$ 和 $n$ 而言,$m + n$ 是多少?我们如何定义这个运算及其输出?我们怎么知道 $m + n$ 是另一个自然数? 我们能确定 $m + n = n + m$ 吗?这些问题在我们稍后讨论函数和关系之后就可以解决。
