% !TeX root = ../../../book.tex
\subsection{数学归纳原理}\label{sec:section3.8.2}

现在,我们提出一个更严格的归纳法版本:

\begin{theorem}[数学归纳原理]\label{theorem3.8}
    设 $P(n)$ 为某个依赖自然数 $n$ 的命题。假设
    \begin{itemize}
        \item $P(1)$ 为\verb|真|。
        \item 给定任意 $k \in \mathbb{N}$,若 $P(k)$ 为\verb|真|,则 $P(k+1)$ 必然为\verb|真|。
    \end{itemize}
    那么 $P(n)$ 对所有自然数 $n \in \mathbb{N}$ 均成立。
\end{theorem}

在详细讨论其假设和结论之前,我们首先证明该定理。

\begin{proof}
    定义集合 $S = \{n \in \mathbb{N} \mid P(n) \text{\ 为真}\}$。根据定义,$S \subseteq N$。

    由定理假设可知:$1 \in S$,且若 $k \in S$,则 $k + 1 \in S$。因此 $S$ 是\dotuline{归纳集}。根据 $\mathbb{N}$ 的定义,有 $\mathbb{N} \subseteq S$。因此 $S = \mathbb{N}$,即 $P(n)$ 对所有自然数 $n$ 均成立。
\end{proof}

这一证明过程十分流畅,对吧?所有结论都直接从定义中自然得出!在这个意义上,定义和公理是\emph{自然}的选择——它们精确实现了我们\emph{直觉}中关于 $\mathbb{N}$ 及其性质的认知。

然而,尚有细节未及讨论:何谓``依赖 $n$ 的命题''?``$P(k)$ 为真\emph{必然得出} $P(k + 1)$ 为真''的确切含义是什么?``\emph{为真}''本身又指什么?这些都是深刻的数学问题,涉及逻辑学的深入研究,我们将在后续章节探讨!请继续前进!
