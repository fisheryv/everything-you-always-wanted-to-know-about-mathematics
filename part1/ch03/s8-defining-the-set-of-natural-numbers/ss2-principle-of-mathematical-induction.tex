% !TeX root = ../../../book.tex
\subsection{数学归纳原理}\label{sec:section3.8.2}

现在,让我们提出一个更严格的归纳法版本:

\begin{theorem}[数学归纳原理]\label{theorem3.8}
    设 $P(n)$ 为某个依赖于自然数 $n$ 的``事实''或``观察''。假设
    \begin{itemize}
        \item $P(1)$ 为\verb|真|。
        \item 给定任意 $k \in \mathbb{N}$,如果 $P(k)$ 为\verb|真|,必然可以得出 $P(k+1)$ 为\verb|真|。
    \end{itemize}
    那么,陈述 $P(n)$ 对于每个自然数 $n \in \mathbb{N}$ 都必然成立。
\end{theorem}

在详细讨论其假设和结论之前,让我们先证明该定理。

\begin{proof}
    将集合 $S$ 定义为陈述 $P$ 为真的自然数。即定义 $S = \{n \in \mathbb{N} \mid P(n) \text{为真}\}$。根据定义,$S \subseteq N$。

    此外,该定理的假设保证 $1 \in S$,并且每当 $k \in S$ 时,我们也知道 $k + 1 \in S$。这意味着 $S$ 是\dotuline{归纳集}。通过定义 $\mathbb{N}$ 后的观察,我们知道 $\mathbb{N} \subseteq S$。

    因此 $S = \mathbb{N}$,因此命题 $P(n)$ 对于每个自然数 $n$ 都成立。
\end{proof}

这很顺滑,对吧?似乎所有想要的结论都“超出”了我们的定义!从这个意义上说,定义和公理是\emph{自然的}选择,因为它们完成了我们\emph{直觉}中已经``知道''的有关集合 $\mathbb{N}$ 及其属性的事情。

还有一些小问题我们没有讨论。具体来说,\emph{依赖于}自然数 $n$ 的``事实''或``观察''是什么意思?当 $P(k)$ 为真时\emph{必然得出} $P(k + 1)$ 为真意味着什么?我们所说的``\emph{为真}''是什么意思?这些都是高深的数学问题,涉及对逻辑的深入研究,我们将在下一章讨论这些问题!继续向前!加油!
