% !TeX root = ../../book.tex
\section{子集}\label{sec:section3.4}

\subsection{定义与示例}

让我们讨论一个我们已经使用过其基本思想的主题。具体来说,让我们研究一下\emph{子集}的概念。

\begin{definition}
    给定两个集合 $A$ 和 $B$,如果 $A$ 的每个元素也是 $B$ 的元素,那么我们说 $A$ 是 $B$ 的\dotuline{子集}。

    子集的数学符号是 $\subset$,所以我们可以写成 $A \subseteq B$。

    如果我们想表明 $A$ 是 $B$ 的子集但又不等于 $B$,我们可以写作 $A \subset B$ 并说 $A$ 是 $B$ 的\dotuline{真子集}。

    我们还可以将这些关系分别写为 $B \supseteq A$ 或 $B \supset A$。在这些情况下,我们会分别说 $B$ 是 $A$ 的\dotuline{超集}或 $B$ 是 $A$ 的\dotuline{真超集}。
\end{definition}

请注意这些符号与我们用来比较实数的不等式符号之间的相似之处。我们写出 $x \le 2$ 或 $5 > z > 0$ 等不等式,并根据符号的``方向''以及是否在其下方放置横线来理解这些不等式的含义。符号 $\subseteq, \subset,\supseteq, \supset$ 的工作方式完全相同,只不过它们指的是``元素的包含''而不是``数字的大小''。

\subsubsection*{标准数集}

我们上一节中提到的标准数集可以通过子集关系很好地关联。具体来说,我们可以说
\[\mathbb{N} \subset \mathbb{Z} \subset \mathbb{Q} \subset \mathbb{R} \subset \mathbb{C}\]
同样,我们理所当然地认为我们对这些数集的知识让我们能够做出这些主张。然而,在准确描述为什么集合 $\mathbb{R}$ 存在并且是 $\mathbb{Q}$ 的真超集时,会涉及到一些深刻而复杂的数学概念。不过,现在我们使用这些集合来说明\textbf{子集}关系。

由于我们知道上面的子集关系是\textbf{正确的},因此我们使用相应的符号 ``$\subset$''。一般来说,在数学写作中简单地使用 ``$\subseteq$'' 符号很常见,即使知道 ``$\subset$'' 更适用。我们可能只会在上下文中重要的时候才使用 ``$\subset$'' 符号来表明两个集合不相等。如果该信息对于当前上下文并不重要,那么我们可能只使用 ``$\subseteq$'' 符号。

\subsubsection*{集合构建符创建子集}

我们已经在集合构建符中``使用''过子集的概念。用于将集合定义为``更大''集合中满足特定属性的所有元素。我们定义一个属性 $P(x)$,从一个更大的集合 $X$ 中提取一个变量对象 $x$,并包含满足属性 $P(x)$ 的任意元素 $x$。请注意,这个新集合的任何元素都必须是 $X$ 的元素,这仅基于我们定义它的方式。因此,以下关系成立
\[\{x \in X \mid P(x)\} \subseteq X\]
不管集合 $X$ 和属性 $P(x)$ 是什么。根据集合 $X$ 和属性 $P(x)$,真子集符号 $\subset$ 可能适用,但一般来说,我们可以肯定地说 $\subseteq$ 一定适用。

尝试提出一些集合 $X$ 和属性 $P(x)$ 的示例,使得 $\subseteq$ 适用,然后尝试提出一些 $\subset$ 适用的示例。尝试找到一个集合 $X$ 和两个不同属性 $P_1(x)$ 和 $P_2(x)$,使得 $\subset$ 适用于 $P_1(x), \subseteq$ 适用于 $P_2(x)$。尝试找到两个不同集合 $X_1$ 和 $X_2$ 以及两个不同属性 $P_1(x)$ 和 $P_2(x)$,使得
\[\{x \in X_1 \mid P_1(x)\} = \{x \in X_2 \mid P_2(x)\}\]
你能做到吗?

\subsubsection*{举例}

当且仅当第一个集合的每一个元素都是第二个集合的元素时,该集合才是另一个集合的子集。例如,这意味着以下关系均成立:

\begin{align*}
    \{142, 857\} &\subseteq \mathbb{N} \\
    \{\sqrt{3}, -\pi, 8.2\} &\subseteq \mathbb{R} \\
    \{x \in \mathbb{R} \mid x^2 = 1\} &\subseteq \mathbb{Z}
\end{align*}
你明白为什么这些都成立吗?

那么,为了使子集关系失败,我们必须找到在第一个集合中而\emph{不在}第二个集合中的元素。例如,这意味着以下关系均成立:

\begin{align*}
    \{142, -857\} &\nsubseteq \mathbb{N} \\
    \{\sqrt{3}, -\pi, 8.2\} &\nsubseteq \mathbb{Q} \\
    \{x \in \mathbb{R} \mid x^2 = 5\} &\nsubseteq \mathbb{Z}
\end{align*}

\subsubsection*{集合的所有子集}

让我们看一个特定的集合。定义 $A = \{1, 2, 3\}$。 我们可以找出 $A$ 的\emph{所有}子集吗?当然可以,为什么不能呢?

\begin{align*}
    \{1\} &\subseteq A  & \{2\} &\subseteq A \\
    \{3\} &\subseteq A & \{1,2\} &\subseteq A \\
    \{1,3\} &\subseteq A & \{2,3\} &\subseteq A \\
    A = \{1, 2,3\} &\subseteq A & \varnothing &\subseteq A \\
\end{align*}
找到前 6 个子集相当简单,但重要的是要记住 $A$ 和 $\varnothing$ 也是子集。(注意:一般来说,对于任何集合 $S, S \subseteq S$ 和 $∅ \subseteq S$ 都是正确的。思考一下!)

考虑集合 $B$,其元素是我们上面列出的所有集合:
\[B = \{\{1\}, \{2\}, \{3\}, \{1, 2\}, \{1, 3\}, \{2, 3\}, A, \varnothing\}\]
确实,任何元素 $X \in B$ 都满足 $X \subseteq A$。你明白为什么吗?

\subsection{幂集}

这种找出给定集合的所有子集的过程是常见且有用的,因此我们赋予这个结果集一个特殊名字。

\begin{definition}
    给定一个集合 $A, A$ 的\dotuline{幂集}定义为元素为 $A$ 的所有子集的集合,记为 $\mathcal{P}(A)$。
\end{definition}

我们在上一小节最后观察到,对于任何集合 $S, S \in \mathcal{P}(S)$ 且 $\varnothing \in \mathcal{P}(S)$。

回顾一下上面的示例集合 $A = \{1, 2, 3\}$。关于 $\mathcal{P}(A)$ 中的元质数量,你注意到什么了?它与 $A$ 中元质数量有何关系?对于任意集合 $S$,你认为 $S$ 和 $\mathcal{P}(A)$ 中的元质数量之间存在一般关系吗?\\

\begin{example}
    我们来求 $\mathcal{P}(\varnothing)$。空集的子集是什么?只有一个,就是空集本身!(即,$\varnothing \subseteq \varnothing$,但没有其他集合满足这一点。)因此,幂集 $\mathcal{P}(\varnothing)$ 只有一个元素,即空集本身:
    \[\mathcal{P}(\varnothing) = \{ \varnothing \}\]
    请注意,这与空集本身不同:
    \[\varnothing \ne \{ \varnothing \}\]
    为什么这是真的?比较元素就能知道!空集没有元素,但右边的集合有一个元素。(一般来说,这可能是比较两个集合的有效方法。)为了给你一些练习,请大声读出上面一行:
    \begin{center}
        ``空集与包含空集的集合是两个不同的集合。''
    \end{center}
\end{example}

\begin{example}
    让我们用另一个集合尝试这个过程,比如 $A = \{\varnothing, \{1, \varnothing\}\}$。我们可以将 $\mathcal{P}(A)$ 的元素列出为
    \[\mathcal{P}(A) = \{\{\varnothing\}, \{\{1, \varnothing\}\}, \{\varnothing, \{1, \varnothing\}\}, \varnothing \}\]
    这可能看起来很奇怪,因为所有的都是空集和花括号,但保持子集关系的正确性很重要。确实,在这个例子中,
    \[\varnothing \in A, \quad \{\varnothing\} \subseteq A, \quad \{\varnothing\} \in \mathcal{P}(A), \quad \{\varnothing\} \subseteq \mathcal{P}(A)\]
    为什么这些关系是正确的?仔细思考一下,然后尝试自己多写一些。``$\in$'' 和 ``$\subseteq$'' 的区别非常重要!
\end{example}

\subsection{集合相等}

什么情况下两个集合相等?一般思路是,如果两个集合包含``相同的元素'',则它们相等,但这并不是相等的精确定义。我们如何才能更明确、更严格地描述该属性?说两个集合 $A$ 和 $B$ 具有``相同的元素''意味着 $A$ 的每个元素也是 $B$ 的元素,$B$ 的每个元素也是 $A$ 的元素。如果这两个属性同时成立,那么我们可以保证这两个集合包含完全相同的元素,所以相等。如果你仔细一想,就会发现我们可以用\textbf{子集}来表达它。多么方便啊!

\begin{definition}
    我们说两个集合 $A$ 和 $B$ \dotuline{相等},当且仅当 $A \subseteq B$ 且 $B \subseteq A$,并写为 $A = B$。
\end{definition}
(如果我们在定义中使用 $\subset$ 符号而不是 $\subseteq$ 会发生什么?这与集合相等的概念相同吗?为什么相同或为什么不同?)

当我们需要如何证明两个集合相等,但又不能简单地列出每个集合的元素并比较时,这个定义将非常有用。通过构造两个论证证明``两个方向''的子集关系,我们可以证明两个集合是相等的。现在,让我们看一个该定义的简单应用。\\

\begin{example}
    如何使用集合相等的定义得到以下等式成立?
    \[\{x \in \mathbb{Z} \mid x \ge 1\} = \mathbb{N}\]
    我们只需得到 $\subseteq$ 和 $\supseteq$ 关系适用于等式两端即可。首先,每个至少为 $1$ 的整数都是自然数吗?当然是的!这解释了为什么
    \[\{x \in \mathbb{Z} \mid x \ge 1\} \subseteq \mathbb{N}\]
    其次,是否每个自然数都是至少为 $1$ 的正整数?当然是的!这解释了为什么
    \[\{x \in \mathbb{Z} \mid x \ge 1\} \supseteq \mathbb{N}\]
    综上,这表明题目等式是成立的。
\end{example}

\subsection{``口袋''类比}\label{sec:section3.4.4}

根据我们的经验,集合在引入时是一个很难理解的概念。具体来说,与集合相关的\textbf{符号}会让学生陷入困境,他们最终会写下毫无意义的东西!因此必须区分符号 $\in$ 和 $\subseteq$ 之间的差异。

请记住下面这个有用的类比:集合就像一个里面装着东西的\emph{口袋}。口袋本身无关紧要;我们只关心里面有什么\emph{样}的东西(即元素是什么)。甚至可以把这个口袋想象成你在杂货店买到的一个不起眼的塑料袋。所有这些口袋都是一样的;为了区分任意两个口袋,我们需要知道\emph{里面}装的是什么东西。

如果我将一个苹果和一个橙子放入口袋中,放置它们的顺序并不重要。你只需要知道我有苹果和橙子即可。我袋子里有多少苹果或橙子并不重要,因为我们只关心里面装着什么样的东西。将其视为回答``口袋里有 $\underline{\qquad}$ 吗?有还是没有?''形式的问题。无论口袋里是有两个苹果、七个苹果还是一个苹果,都没关系;如果你问我有没有苹果,我都会说``有''。这与集合中元素的顺序和重复无关紧要这个概念有关。集合完全由其元素来表征。

当我们将集合视为其他集合的元素时,这个类比也很有帮助。我们当然可以将整个袋子放入另一个袋子里。看看我们在上面的例子中定义的集合 $A$:
\[A = \{\varnothing, \{1, \varnothing\}\}\]
集合 $A$ 是一个口袋。口袋里有什么?口袋里有两个物体(即 $A$ 有两个元素)。它们本身恰好也都是口袋!其中一个是一个普通的空口袋,里面什么也没有。(那就是空集。) 好吧,那很酷。另一个里面有两个物体。其中一个对象是数字 $1$。酷。另一个物体又是一个空口袋。

\subsubsection*{区分 ``$\in$'' 和 ``$\subseteq$''}

口袋类比也有助于理解 ``$\in$'' 和 ``$\subseteq$'' 之间的区别。继续使用集合 $A$ 来做示例。当我们写 $x \in A$ 时,我们的意思是 $x$ 是口袋 $A$ 内的一个物体。如果我们打开 $A$ 去查看,我们会看到一个 $x$ 位于口袋 $A$ 的底部。让我们用这个思路来比较两个例子。

\begin{itemize}
    \item 我们看到 $\varnothing \in A$ 在这里是正确的。如果我们看一下口袋 $A$ 的内部,我们会在里面的东西(元素)中看到一个空袋子。
    \item 我们还看到 $\{\varnothing\} \notin A$ 在这里也是正确的。如果我们看一下口袋 $A$ 的内部,我们不会看到只装着另一个空袋子的袋子。(请注意,这就是 $\{\varnothing\}$:一个空袋子装在另一个袋子里。)\\
    你看到了这样的物体吗?在哪里?我不敢让你给我看,在口袋 $A$ 里面的东西中,有一个袋子只装着一个空袋子。\\
    我在口袋 $A$ 里看到了什么?好吧,我看到两样东西:一个空袋子,和一个里面有两个物体的袋子(一个空袋子和数字 $1$)。这些物体都不是我们要找的!
\end{itemize}

当我们写 $X \subseteq A$ 时,我们的意思是 $X$ 和 $A$ 这两个口袋在某种程度上是可以比较的。具体来讲,我们是说 $X$ 内部的所有内容也是 $A$ 内部的内容。我们实际上是在遍历 $X$ 内部的所有对象,将它们一一取出,并确保我们也能在 $A$ 内部找到该对象。让我们用这个思路来比较两个例子。

\begin{itemize}
    \item 我们看到 ${\varnothing} \subseteq A$ 是正确的。我们\emph{比较}左边的口袋和右边的口袋。左边的口袋里装着是什么?里面只有一个物体,这个物体本身就是一个空袋子。现在,我们看一下 $A$ 内部。看是否从里面能找到一个空袋子?没错,可以找到!因此,``$\nsubseteq$'' 符号适用于此。
    \item 我们还看到 $\{1\} \nsubseteq A$ 也是正确的。为了比较这两个口袋,我们从左边的口袋里拿出一个物体,看看它是否也在口袋 $A$ 中。这里,我们只有一个物体要拿出来:数字 $1$。现在,让我们看看口袋 $A$ 内部。我们看到里面有一个 $1$ 吗? 不,我们没有找到!\\
    我们必须进到口袋 $A$ 内部的口袋才能找到数字 $1$;这个数字不在我们直接视线内。因此 $\{1\} \nsubseteq A$。
\end{itemize}

回顾一下我们已经讨论过的一些例子,记住这个新的类比。它有助于你理解定义和示例吗?它是否有助于你理解 ``$\in$'',$\subseteq$''和 ``$\supseteq$'' 之间的区别?如果没有,你能想出其他对你有帮助的类比吗?

\subsection{习题}

\subsubsection*{温故知新}

以口头或书面的形式简要回答以下问题。这些问题全都基于你刚刚阅读的内容,所以如果忘记了具体的定义、概念或示例,可以回去重读相关部分。确保在继续学习之前能够自信地回答这些问题,这将有助于你的理解和记忆!

\begin{enumerate}[label=(\arabic*)]
    \item $\mathbb{N} \subseteq \mathbb{R}$ 吗? $\mathbb{R} \subseteq \mathbb{N}$ 吗? $\mathbb{Q} \subseteq \mathbb{Z}$ 吗?为什么是或者为什么不是?
    \item $\subset$ 和 $\subseteq$ 有什么不同?给出集合 $A, B$ 的示例,使得 $A \subseteq B$ 为真,但 $A \subset B$ 为假。
    \item $\in$ 和 $\subseteq$ 有什么区别?给出集合 $C, D$ 的示例,使得 $C \subseteq D$ 但 $C \notin D$。
    \item 设 $S$ 为任意集合。$S$ 的幂集是什么?它是什么类型的数学对象?它应该如何定义?
    \item 假设 $S \subseteq T$。这是否意味着 $S = T$?为什么相等或者为什么不等?
    \item 解释为什么对于任意集合 $S$ 都有 $\varnothing \subseteq S$ 且 $\varnothing \in \mathcal{P}(S)$。
    \item 假设 $X \in \mathcal{P}(A)$。那么 $X$ 和 $A$ 有什么关系?
    \item $A = P(A)$ 可能为真吗?(这个问题比较棘手,请好好思考一下!)
\end{enumerate}

\subsubsection*{小试牛刀}

尝试回答以下问题。这些题目要求你实际动笔写下答案,或(对朋友/同学)口头陈述答案。目的是帮助你练习使用新的概念、定义和符号。题目都比较简单,确保能够解决这些问题将对你大有帮助!

\begin{enumerate}[label=(\arabic*)]
    \item 写出集合 $\mathcal{P}(\mathcal{P}(\varnothing))$ 的元素。
    \item 写出集合 $\mathcal{P}([1]), \mathcal{P}([2]), \mathcal{P}([3])$ 的元素。你能猜想 $\mathcal{P}([n])$ 有多少个元素吗?(你能证明这一点吗?我们不指望你现在就能证明出来,但很快就能了;好好想一想!)
    \item 设 $A = \{x, \heartsuit, \{4\} , \varnothing\}$。对于以下陈述,判断它是对是错,并简要解释原因。
        \begin{enumerate}[label=(\alph*)]
            \item $x \in A$
            \item $x \subseteq A$
            \item $\{x, \heartsuit\} \subseteq A$
            \item $\{x, \varnothing\} \subset A$
            \item $\{x, \heartsuit, z, 7\} \supseteq A$
            \item $\{x\} \in \mathcal{P}(A)$
            \item $\{x\} \subseteq \mathcal{P}(A)$
            \item $\{\heartsuit, x\} \in \mathcal{P}(A)$
            \item $\{4\} \in \mathcal{P}(A)$
            \item $\{\varnothing\} \in \mathcal{P}(A)$
            \item $\{\varnothing\} \subseteq \mathcal{P}(A)$
        \end{enumerate}
        \textbf{提示:}$7$ 个为真,$4$ 个为假。
    \item 举一个集合 $A, B$ 的例子,使得 $A \in B$ 且 $A \subseteq B$ 都为真。
    \item $\{1, 2, 12\} \subseteq \mathbb{R}$ 吗?
    \item $\{-5, 8, 12\} \subseteq \mathbb{N}$ 吗?
    \item $\{1, 3, 7\} \in \mathcal{P}(\mathbb{N})$ 吗?
    \item $\mathbb{N} \in \mathcal{P}(\mathbb{Z})$ 吗?
    \item $\mathcal{P}(\mathbb{N}) \subseteq \mathcal{P}(\mathbb{Z})$ 吗?它们是相等的集合吗?为什么是或者为什么不是?
    \item 给出一个无限集合 $T$ 的例子,使得 $T \in \mathcal{P}(\mathbb{Z})$ 但 $T \notin \mathcal{P}(\mathbb{N})$。
    \item 假设 $G, H$ 是集合并且它们满足 $\mathcal{P}(G) = \mathcal{P}(H)$。我们能得出 $G = H$ 的结论吗?为什么能或者为什么不能?(不要试图正式证明这一点;只需思考并尝试说出来。)
    \item 给出一个集合 $W$ 的例子,使得 $W \subseteq \mathcal{P}(\mathbb{N})$ 但 $W \notin \mathcal{P}(\mathbb{N})$。
\end{enumerate}
