% !TeX root = ../../../book.tex
\subsection{引言}

本节讨论一个先前提及并已使用的概念:集合的\textbf{索引}表示法。当需要定义或引用大量集合而不显式列举时,这种表示法十分便捷。借鉴已知的集合运算符号,我们将能``一次性''组合和操作多个集合。虽然本节不涉及新的数学内容,但其符号体系初期可能令人困惑,因此我们将逐步引导理解其核心思想。

\subsubsection*{与求和符号的类比}

首先回顾一个相似概念。第一章讨论自然数求和时,曾引入 $\sum$ 符号将冗长的加法表达式简化为紧凑形式。例如,非正式求和(``非正式''意味着``不严格'',因为使用了省略号)可表示为:
\[1 + 2 + 3 + 4 + \dots + (n - 1) + n = \sum_{i=1}^{n} i\]

该符号的有效性源于\textbf{索引变量} $i$。$\sum$ 符号下方的``$i = 1$''表示 $i$ 从 $1$ 开始逐次递增,直至达到上方的终值 $n$。对于该范围内每个 $i$ 值,将 $\sum$ 右侧的表达式 $i$ 作为求和项,从而得到 $1, 2, 3,\dots, n$ 的连加式。

这里需要指出,索引 $i = 1$ 和上界 $n$ 共同限定了 $i$ 的取值范围——$1$ 到 $n$ 之间的全体自然数。

\subsubsection*{示例}

下面通过实例演示索引集的定义过程,并展示如何将集合运算应用于索引集合族。

\begin{example}\label{ex:example3.6.8}
    集合运算符可以进行类似简化。定义集合族 $A_1, A_2, A_3, \dots, A_{10}$
    
    \begin{align*}
        A_1 &= \{1, 2\} \\
        A_2 &= \{2, 4\} \\
        A_3 &= \{3, 6\} \\
        &\vdots \\
        A_i &= \{i, 2i\} \\
        &\vdots \\
        A_{10} &= \{10, 20\}
    \end{align*}

    此处明确定义 $A_i$ 对\emph{任意} $i$ 的取值,为集合族提供严谨定义。若未给出通项定义,读者需自行推测 $A_1,A_2,\dots,A_{10}$ 的模式,可能导致歧义。通过精确定义 $A_i$,这 $10$ 个集合的含义得以明确。

    进一步可简洁表达所有集合的并集。回顾定义:两个集合的并集包含二者所有元素(元素属于第一或第二集合,或同时属于二者)。多个集合的并集遵循相同原则:元素只要属于被并的\emph{任意}集合,即被包含。

    如何简洁准确地书写此并集?参照 $\sum$ 表示法:索引 $i$ 从 $1$ 至 $10$,故在``$\cup$''下方写 $i=1$,上方写 $10$。因被并项为$\{i, 2i\}$,需使用更大的``$\bigcup$''符号表示索引并集:
    \[A_1 \cup A_2 \cup A_3 \cup \dots \cup A_{10} = \bigcup_{i=1}^{10}A_i=\bigcup_{i=1}^{10}\{i, 2i\}\]
    相较于显式列出 $10$ 个集合,该表示法极为简洁,凸显其\emph{实用性}。需注意左侧使用省略号的表达式存在不精确性,而右侧才是严谨的数学表述——左侧仅为直观描述并集的启发式表达。
\end{example}

\subsubsection*{当索引集不是数字范围时}

让我们看一个更复杂的例子,进一步拓展这种表示方法。如何用求和符号表示所有质数的平方倒数之和?注意我们的目标是用符号表达求和项而非计算结果(精确求和本身是另一项挑战)。

此时不能沿用之前的表示法,因为求和对象并非自然数序列,而是特定的质数项。解决方法是通过定义\textbf{索引集} $I$ 来描述允许的索引值,并将其``代入''求和表达式。

本例中,我们希望为每个质数 $i$ 包含 $\frac{1}{i^2}$。因此索引集 $I$ 应包含所有质数,表达式可写作:
\[\sum_{i \in i}\frac{1}{i^2}, \text{其中\ } I = \{i \in \mathbb{N} \mid i \text{\ 为质数}\}\]

这种表示法不仅将无穷多项浓缩为简洁表达式,还精确限定了索引范围——不同于 $\sum_{i=1}^{n}i$ 这类连续自然数索引。

\begin{example}
    \emph{索引集}的概念具有广泛适用性,可拓展至任意集合甚至非数学对象。例如前文讨论集合时提到的 NBA 球队集合 $B$,如何用其表示所有球员集合 $P$?每支球队本质是球员集合,因此 $B$ 中所有球队集合的并集恰好构成全体球员集合:
    \[P = \bigcup_{b \in B} b\]

    参与并集的元素不依赖自然数索引,此类表达必须通过索引集实现。需注意:并集运算要求集合的元素本身也是集合,因此 $B$ 的元素(球队)必须是集合才能进行并集操作。这种``集合的元素仍是集合''的概念需仔细体会。
\end{example}

\subsubsection*{索引表达式的朗读方法}

为帮助理解,以下提供索引表达式的朗读方法。质数求和可读作:
\begin{quotation}
    ``对 $\frac{1}{i^2}$ 求和,其中 $i$ 为全体质数。'' 

    或

    ``对所有质数 $i$,求 $\frac{1}{i^2}$ 之和。'' 
\end{quotation}

同样地,NBA 球队的并集可读作:
\begin{quotation}
    ``对所有集合 $b$ 取并集,其中 $b$ 为 2023 赛季 NBA 球队。'' 

    或
    
    ``对所有 2023 赛季 NBA 球队取并集。'' 
\end{quotation}

\clearpage
