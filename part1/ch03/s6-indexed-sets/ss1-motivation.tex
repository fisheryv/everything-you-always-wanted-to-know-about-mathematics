% !TeX root = ../../../book.tex
\subsection{引言}

让我们讨论一个我们之前简单提过并且已经使用过的概念:集合\textbf{索引}。当我们希望定义或引用大量集合而不显式写出所有集合时,这种表示法会很方便。使用与我们已经定义的集合运算类似的符号,我们将能够``一次性''``组合''和``操作''大量集合。本节确实没有新的数学内容,但其思想中涉及的符号一开始可能会令人困惑且难以使用,因此我们希望仔细引导你了解这些思想。

\subsubsection*{与求和符号的关系}

我们先从之前见过的相关概念入手。还记得我们在第一章中研究自然数之和吗?我们提到了一些特殊符号,让我们能够使用 $\sum$ 符号将一长串求和项压缩成一种简洁的形式。例如,我们可以用 $\sum$ 表示法写出一个非正式的求和(``非正式''意味着``不严格'',因为使用了省略号),如下所示:
\[1 + 2 + 3 + 4 + \dots + (n - 1) + n = \sum_{i=1}^{n} i\]

为什么这个符号有效且合理?关键在于\textbf{索引变量} $i$。在 $\sum$ 符号下方写下 ``$i = 1$'' 意味着变量 $i$ 的值应从 $1$ 开始每次增加 $1$,直到达到写在 $\sum$ 符号上方的终值 $n$。对于该范围(从 $1$ 到 $n$)内的每个允许的 $i$ 值,我们在 $\sum$ 符号右侧的包含一个求和项;在上面案例中,该求和就是 $i$。因此,我们应该有项 $1, 2, 3,\dots, n$ 并用 $+$ 号将它们相连。

这里需要指出,将 $i = 1$ 和 $n$ 写为索引变量 $i$ 的\textbf{限制}意味着 $i$ 假设所有值都是 $1$ 到 $n$ 之间的自然数。

\subsubsection*{示例}

我们先通过一个例子来看看定义索引集的过程。我们还将了解如何使用索引变量将集合运算应用于多个集合。\\

\begin{example}\label{ex:example3.6.8}
    我们可以类似地压缩某些集合运算符。例如,让我们定义集合 $A_1, A_2, A_3, \dots, A_{10}$
    \begin{align*}
        A_1 &= \{1, 2\} \\
        A_2 &= \{2, 4\} \\
        A_3 &= \{3, 6\} \\
        &\vdots \\
        A_i &= \{i, 2i\} \\
        &\vdots \\
        A_{10} &= \{10, 20\}
    \end{align*}

    我们包含了 $A_i$ 对\emph{任意}值 $i$ 的定义,以便为这些集合提供严格的定义。如果不定义该集合(将 $A_i$ 定义为 $i$ 的任何相关值),读者需要自行解释集合 $A_1,A_2,A_3,A_{10}$ 之间的模式,并且可能产生多种解释方式。通过像这样明确定义 $A_i$ 项,这 $10$ 个集合是什么就不会产生混淆了。

    此外,我们可以更轻松地表达所有这些集合的并集。请记住,两个集合的并集是包含两个集合所有元素的集合(即,如果某个元素位于第一个集合或第二个集合中,或者可能同时位于两个集合中,则该元素包含在并集中)。两个以上集合的并集是什么?它遵循与两个集合的定义相同的思想;如果某个元素位于我们通过并集运算组合的\emph{任何}成分集合中,我们希望将其包含在并集中。

    怎样才能把这个并集写得简洁、准确呢?我们遵循与 $\sum$ 表示法的相同方式。这些集合的索引从 $1$ 到 $10$,因此我们应该在 ``$\cup$'' 符号下方写 $i = 1$,在其上方写 $10$。并集中的每一项的形式为 $\{i, 2i\}$,因此我们应该将其写在 ``$\cup$'' 符号的右侧。不过,对于像这样的索引联合,我们使用稍大的 ``$\bigcup$'' 符号,如下所示:
    \[A_1 \cup A_2 \cup A_3 \cup \dots \cup A_{10} = \bigcup_{i=1}^{10}A_i=\bigcup_{i=1}^{10}\{i, 2i\}\]
    这比写出所有 $10$ 个集合的元素要简洁得多,因此你可以看到这种表示法是多么的\emph{有用}。我们会不断提醒你左侧并集中省略号的不精确性,并告诉你实际上,像右侧这样的表达式是关于该并集的真正严格的数学陈述。左边的表达式更多的是一种直观的、启发式的方式来描述这 $10$ 个集合的并集。
\end{example}

\subsubsection*{当索引集不是数字范围时}

让我们看一个更难的例子,来进一步发展这种表示技术。如果我们要求你用求和符号写出以下求和:所有质数的平方倒数之和,该怎么办?我们怎样才能做到这一点?(注意:我们只是想表达求和的所有项而不计算出总和。这是一项艰巨的任务,留待下次吧!)

不幸的是,我们不能使用与上面完全相同的表示法,因为我们不想对两个自然数之间的一系列索引值求和;相反,我们只想在求和中包含与质数相关的项。解决这个问题的方法是定义一个\textbf{索引集} $I$,它将描述索引的允许值,然后我们将其``插入''求和右侧的任意项。

在这种情况下,如果我们有一个质数 $i$,我们希望在求和中包含 $\frac{1}{i^2}$ 项,因此该表达式将写在 $\sum$ 符号的右侧。 我们想用符号来表达值 $i$ 应该是质数并且应该包括所有可能的质数。因此,允许值的索引集合 $I$ 应该是所有质数的集合。也就是说,我们可以将这个求和写为:
\[\sum_{i \in i}\frac{1}{i^2}, \text{其中}\; I = \{i \in \mathbb{N} \mid i \text{为质数}\}\]

看看这个符号的作用!我们不仅将无穷多项压缩为一个表达式,而且还指出任意索引 $i$ 的值应限制为质数,而质数不像 $\sum_{i=1}^{n}i$ 那样以``通常''且简便的方式描述。\\

\begin{example}
    \emph{索引集}的概念非常有用,可以扩展到任意集合甚至非数学对象。例如,在前面对集合的讨论中,我们使用了所有 NBA 球队的集合 $B$。我们如何用这个集合来表达 NBA 所有球员的集合 $P$?每个球队本身就是一个集合,其元素是该球队中的球员,因此所有球队的并集(即 $B$ 中所有集合的并集)应该准确地生成所有球员的集合!在这种情况下,我们的索引集是 $B$,对于每个元素 $b \in B$,我们希望将 $b$ 作为集合包含在我们的并集中。因此,我们会写
    \[P = \bigcup_{b \in B} b\]

    该并集中的各项甚至不依赖于自然数,因此如果不使用索引集,就无法表达像上面这样的并集。此外,该并集依赖于这样一个事实:并集的项是索引集 $B$ 的元素,但它们本身也是集合;因此,对它们运用并集运算具有数学意义。这可能看起来仍然是一个奇怪的思想,所以一定要仔细思考集合的元素本身也是集合这一思想。
\end{example}

\subsubsection*{朗读索引表达式}

为了用语言表达这些类型的表达方式,并帮助你在头脑中思考它们,让我们举一个例子。我们可以将上面的表达式读作

\begin{quote}
    ``对 $\frac{1}{i^2}$ 求和,其中 $i$ 为全体质数。'' \\
    或 \\
    ``对所有质数 $i$,求和 $\frac{1}{i^2}$。'' 
\end{quote}
同样,我们可以将上面的另一个表达式读作
\begin{quote}
    ``对所有集合 $b$ 取并集,其中 $b$ 为 2023 赛季 NBA 球队。'' \\
    或 \\
    ``对所有 2023 赛季 NBA 球队,取并集。'' 
\end{quote}
