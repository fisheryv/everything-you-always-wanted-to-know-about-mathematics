% !TeX root = ../../../book.tex
\subsection{习题}

\subsubsection*{温故知新}

以口头或书面的形式简要回答以下问题。这些问题全都基于你刚刚阅读的内容,如果忘记了具体定义、概念或示例,可以回顾相关内容。确保在继续学习之前能够自信地作答这些问题,这将有助于你的理解和记忆!

\begin{enumerate}[label=(\arabic*)]
    \item 什么是索引集?
    \item 令 $I = \mathbb{N}$,对于每个 $i \in I$,令 $A_i = \{i, -i\}$。为什么下列集合是同一个集合?
    \[\bigcup_{i \in I} A_i \qquad\qquad \bigcup_{x \in \mathbb{N}} A_x \qquad\qquad \bigcup_{j \in I} A_j\]
    请问这个集合的元素是什么?
    \item 列出下列集合的元素:
    \begin{tasks}(3)
        \task $\displaystyle{\bigcup_{x \in \mathbb{N}}\{x\}}$
        \task $\displaystyle{\bigcap_{x \in \mathbb{N}}\{x\}}$
        \task $\displaystyle{\bigcup_{x \in \mathbb{N}}\{x,0,-x\}}$
    \end{tasks}
    \item 为什么我们通常只讨论索引并集和索引交集,而不讨论``索引差集''或``索引补集''?
    \item 什么是划分?一个集合族需满足哪些条件才能构成某个集合的划分?
\end{enumerate}

\subsubsection*{小试牛刀}

尝试解答以下问题。这些题目需动笔书写或口头阐述答案,旨在帮助你熟练运用新概念、定义及符号。题目难度适中,确保掌握它们将大有裨益!

\begin{enumerate}[label=(\arabic*)]
    \item 设集合 $A = \{-2, -1, 0, 1, 2\}$,集合 $B = \{1, 3, 5\}$。对于所有 $i \in \mathbb{Z}$,令 $S_i = \{i - 2, i, i + 2, i + 4\}$。求:
    \[\bigcup_{i \in A}S_i \quad \text{和} \quad \bigcap_{x \in B}S_x\]
    \item 对于所有 $n \in \mathbb{N}$,令 $A_n = [n]$。求:
    \[\bigcap_{x \in \mathbb{N}}A_n \quad \text{和} \quad \bigcup_{x \in \mathbb{N}}A_n\]
    \item 用集合构建符写出闭区间 $[-10, 10]$ 内所有整数构成的集合。然后,使用索引并集定义相同的集合。能否使并集中的集合两两不相交(即任意两个集合无共同元素)?\\(提示:可以。)
    \item 对于每个 $n \in \mathbb{N}$,令 $M_n$ 表示 $n$ 的所有倍数集(例如,$M_3 = \{3, 6, 9,\dots\}$)。使用集合构建符表示 $M_n$,并用这些集合的并集表示 $\mathbb{N}$。\\
    (\textbf{挑战:}能否用这些集合构造 $\mathbb{N}$ 的一个划分?)
    \item 设 $X$ 为任意集合。求:
    \[\bigcup_{S \in \mathcal{P}(X)}S \quad \text{和} \quad \bigcap_{S \in \mathcal{P}(X)}S\] 
    (建议先取具体集合,例如 $X = \{1, 2\}$,尝试计算以辅助理解。)
    \item 将 $\mathbb{Q}$ 表示为索引并集。\\
    能否使用无限索引集实现?\\
    (\textbf{挑战:}能否使该并集构成 $\mathbb{Q}$ 的一个划分?)
\end{enumerate}