% !TeX root = ../../../book.tex
\subsection{划分}

现在我们有了一种写出许多集合的并集的方法,由此我们可以定义一个有用的概念:\textbf{划分}。从语言上来说,划分是一种``将物体分解成部分''的方式,从数学上来说,这正是这个词的意思。

也就是说,划分只是一个集合中不重叠的子集的集合,其并集是整个集合。让我们在这里写下该定义,然后查看一些示例和伪例。将来我们会多次使用这个定义,所以现在让我们在讨论集合和索引并集时先定义好它。

\begin{definition}\label{def:definition3.6.9}
    设 $A$ 为集合。$A$ 的\dotuline{划分}为互不相交且并集为 $A$ 的集合所组成的集合。

    也就是说,划分由满足以下条件的索引集 $I$ 和非空集 $S_i$(定义在每个 $i \in I$ 上)构成:
    \begin{enumerate}[label=(\arabic*)]
        \item 对于所有 $i \in I, S_i \subseteq A$。
        \item 对于所有 $i, j \in I \;\text{且}\; i \ne j$,我们有 $S_i \cap S_j = \varnothing$。
        \item $\displaystyle \bigcup_{i \in I} S_i = A$
    \end{enumerate}
    这里的集合 $S_i$ 称为划分的\dotuline{部分}。
\end{definition}

这里的思想是集合 $S_i$ 将集合 $A$ ``分割''成不重不漏的部分。\\

\begin{example}
    让我们看几个例子
    \begin{enumerate}[label=(\arabic*)]
        \item 考虑集合 $\mathbb{N}$。设 $O$ 为奇数集合,设 $E$ 为偶数集合。 那么 $\{O, E\}$ 就是 $\mathbb{N}$ 的划分。这是因为
        \begin{itemize}
            \item $E, O \ne \varnothing$,
            \item $E, O \subseteq N$,
            \item $E \cap O = \varnothing$,
            \item $E \cup O = \mathbb{N}$ 
        \end{itemize}
        \item 考虑集合 $\mathbb{R}$。对于每个 $z \in \mathbb{Z}$,将集合 $S_z$ 定义为
        \[S_z = \{r \in \mathbb{R} \mid z \le r \le z + 1\}\]
        我们说 $\{\dots, S_{-2}, S_{-1}, S_0, S_1, S_2, \dots \}$ 是 $\mathbb{R}$ 的划分。你知道为什么吗?尝试写出这些集合成为划分所需的条件,看看你是否能理解为什么它们成立。
        具体来说,请记住,我们需要这些集合是成对不相交的。这意味着任意两个集合都必须是不相交的。特别注意,这与要求所有集合的交集为空完全不同。
        例如,考虑下面一组集合
        \[\big\{\{1, 2\}, \{2, 3\}, \{3, 4\}\big\}\]
        该集合就不是成对不相交的,因为
        \[\{1, 2\} \cap \{2, 3\} = \{2\} \ne \varnothing\]
        然而,所有三个集合的交集却是空集,因为这三个集合没有共同元素。
    \end{enumerate}
\end{example}

\begin{example}
    接着让我们看几个伪例。
    \begin{enumerate}[label=(\arabic*)]
        \item 考虑集合 $\mathbb{R}$。令 $P$ 为正实数集合,令 $N$ 为负实数集合。那么 $\{N, P\}$ 不是一个划分,因为 $0 \notin N \cup P$。
        你能否换一种方式,将 $\mathbb{R}$ 划分为两部分?
        \item 考虑集合 $Z$。设 $A_2$ 为 $2$ 的倍数的整数集合,设 $A_3$ 为 $3$ 的倍数的整数集合,设 $A_5$ 为 $5$ 的倍数的整数集合。集合 $\{A_2, A_3, A_5\}$ 不是划分有两个原因。
        \begin{itemize}
            \item 首先,这些集合不是成对不相交的。比如 $6=2 \cdot 3$,所以 $6 \in A_2$ 且 $6 \in A_3$。
            \item 其次,这些集合并未``覆盖''所有 $\mathbb{Z}$。比如 $7 \in \mathbb{Z}$ 但 $7 \notin A_2 \cup A_3 \cup A_5$。
        \end{itemize}
    \end{enumerate}
\end{example}

正如我们前面提到的,我们将来会经常使用这个定义,所以请牢记它。
