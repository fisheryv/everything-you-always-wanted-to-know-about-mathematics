% !TeX root = ../../../book.tex
\subsection{划分}

通过引入表示并集的方法,我们可以定义一个重要的概念:\textbf{划分}。直观上,划分是一种``将整体分解为若干部分''的方式,其数学定义精确地体现了这一思想。

具体而言,集合的划分是其子集构成的集合,这些子集互不相交且覆盖整个集合。下面给出正式定义,并辅以示例与反例说明。该定义在后续讨论集合及索引并集时将被频繁使用。

\begin{definition}\label{def:definition3.6.9}
    设 $A$ 为集合。$A$ 的\dotuline{划分}为互不相交且并集为 $A$ 的集合所组成的集合。

    也就是说,划分由满足以下条件的索引集 $I$ 和非空集 $S_i$(定义在每个 $i \in I$ 上)构成:
    \begin{enumerate}[label=(\arabic*)]
        \item 对于所有 $i \in I, S_i \subseteq A$。
        \item 对于所有 $i, j \in I \text{\ 且\ } i \ne j, S_i \cap S_j = \varnothing$。
        \item $\displaystyle \bigcup_{i \in I} S_i = A$
    \end{enumerate}
    集合 $S_i$ 称为划分的\dotuline{部分}。
\end{definition}

此定义的核心在于:子集族 $\{S_i\}$ 将 $A$ 不重不漏地``分割''为若干部分。

\begin{example}
    考察以下示例:
    \begin{enumerate}[label=(\arabic*)]
        \item 考虑自然数集 $\mathbb{N}$。设 $O$ 为奇数集,$E$ 为偶数集。则 $\{O, E\}$ 构成 $\mathbb{N}$ 的划分。因为:
        \begin{itemize}
            \item $E, O \ne \varnothing$,
            \item $E, O \subseteq N$,
            \item $E \cap O = \varnothing$,
            \item $E \cup O = \mathbb{N}$ 
        \end{itemize}
        \item 考虑实数集 $\mathbb{R}$。对每个整数 $z \in \mathbb{Z}$,定义集合
        \[S_z = \{r \in \mathbb{R} \mid z \le r \le z + 1\}\]
        则 $\{\dots, S_{-2}, S_{-1}, S_0, S_1, S_2, \dots \}$ 构成 $\mathbb{R}$ 的划分。请思考其成立的条件:首先需满足集合两两不相交(即任意两个集合交集为空),这与整体交集为空不同。例如集合族
        \[\big\{\{1, 2\}, \{2, 3\}, \{3, 4\}\big\}\]
        该集合虽满足 $\{1, 2\} \cap \{2, 3\} \cap \{3, 4\} = \varnothing$,但因为
        \[\{1, 2\} \cap \{2, 3\} = \{2\} \ne \varnothing\]
        所以不满足两两不相交。
    \end{enumerate}
\end{example}

\begin{example}
    接着让我们看几个反例:
    \begin{enumerate}[label=(\arabic*)]
        \item 考虑实数集 $\mathbb{R}$。设 $P$ 为正实数集,$N$ 为负实数集。则 $\{N, P\}$ 不构成划分,因为 $0 \notin N \cup P$。
        你能否换一种方式,将 $\mathbb{R}$ 划分成两个部分?
        \item 考虑整数集 $\mathbb{Z}$。设 $A_2$ 为 $2$ 的倍数集,$A_3$ 为 $3$ 的倍数集,$A_5$ 为 $5$ 的倍数集。集合 $\{A_2, A_3, A_5\}$ 不构成划分,原因有二:
        \begin{itemize}
            \item 首先,集合非两两不相交:比如 $6=2 \cdot 3$,所以 $6 \in A_2$ 且 $6 \in A_3$。
            \item 其次,这些集合并未``覆盖''所有 $\mathbb{Z}$。比如 $7 \in \mathbb{Z}$ 但 $7 \notin A_2 \cup A_3 \cup A_5$。
        \end{itemize}
    \end{enumerate}
\end{example}

划分的定义后续将频繁使用,请务必掌握。
