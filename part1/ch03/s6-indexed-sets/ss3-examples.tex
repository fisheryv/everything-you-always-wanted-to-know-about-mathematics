% !TeX root = ../../../book.tex
\subsection{示例}

让我们回到前面的例子,使这些概念更加清晰。

\begin{example}
    在例 \ref{ex:example3.6.8} 中,我们曾对 $1$ 到 $10$ 之间的每个自然数 $i$ 定义:
    \[A_i = \{i, 2i\}\]
    另一种定义方式是使用索引集 $I = [10]$(回顾符号 $[n] = \{i \in \mathbb{N} \mid 1 \le i \le n\}$),并将 $A$ 定义为集合族:
    \[A = \{A_i \mid i \in I\}, \text{其中对于所有\ } i \in I, A_i = \{i, 2i\}\]
    这定义了依赖于索引 $i \in I$ 的集合 $A_i$,并将它们汇集为集合 $A$。基于 $I$ 和 $A_i$ 的定义,我们可用另一种形式表示并集:
    \[\bigcup_{i \in I} A_i\]
    (思考:此并集与集合 $A$ 有何区别?此并集的具体形式是什么?如何简洁描述其元素?需逐一列举吗?若将索引集 $I$ 改为 $\mathbb{N}$,此并集会如何变化?)
\end{example}

\begin{example}
    设 $I = \{1, 2, 3\}$,对于所有 $i \in I$ 定义:
    \[A_i = \{i - 2, i - 1, i, i + 1, i + 2\}\]
    请写出以下集合的元素:
    \[\bigcup_{i \in I} A_i \qquad\text{和}\qquad \bigcap_{i \in I} A_i\]
    请注意,各 $A_i$ 的元素为:
    \begin{align*}
        A_1 &= \{-1, 0, 1, 2, 3\} \\
        A_2 &= \{0, 1, 2, 3, 4\} \\
        A_3 &= \{1, 2, 3, 4, 5\}
    \end{align*}
    因此
    \begin{align*}
        \bigcup_{i \in I} A_i &= A_1 \cup A_2 \cup A_3 = \{-1, 0, 1, 2, 3, 4, 5\} \\
        \bigcap_{i \in I} A_i &= A_1 \cap A_2 \cap A_3 = \{1, 2, 3\}
    \end{align*}
    现取 $J = \{-1, 0, 1\}$,按相同方式定义 $A_j$。请写出以下集合的元素:
    \[\bigcup_{j \in J} A_j \qquad\text{和}\qquad \bigcap_{j \in J} A_j\]
    通过列出元素可得:
    \begin{align*}
        \bigcup_{j \in J} A_j &= A_{-1} \cup A_0 \cup A_1 = \{-3, -2, -1, 0, 1, 2, 3\} \\
        \bigcap_{j \in J} A_j &= A_{-1} \cap A_0 \cap A_1 = \{-1, 0, 1\}
    \end{align*}
    尝试对不同的索引集求解相同的问题。
    例如,$K = \{1, 2, 3, 4, 5\}$ 或 $L = \{-3, -2, -1, 0, 1, 2, 3\}$。
\end{example}

\begin{example}
    定义索引集 $I = \mathbb{N}$。对于所有 $i \in I$ 定义集合:
    \[C_i = \left\{x \in \mathbb{R} \mid 1 \le x \le \frac{i + 1}{i}\right\}\]
    则有:
    \begin{align*}
        \bigcup_{i \in I} C_i &= \{y \in \mathbb{R} \mid 1 \le y \le 2\} \\
        \bigcap_{i \in I} C_i &= \{1\}
    \end{align*}
    你理解上述结论为何成立吗?后续我们将讨论证明此类等式的方法。现在请思考其正确性:能否向他人解释?你会采用何种技术证明这些结论?
\end{example}

\begin{example}
    设 $S$ 为本课程全体学生的集合。对于每个 $s \in S$,令 $C_s$ 表示学生 $s$ 本学期选修的课程集合。下列表达式分别代表什么?
    \[\bigcup_{s \in S} C_s \qquad\text{和}\qquad \bigcap_{s \in S} C_s\]
    相信你至少能找出右侧集合中的一个元素!
\end{example}
