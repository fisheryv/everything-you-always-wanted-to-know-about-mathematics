% !TeX root = ../../../book.tex
\subsection{示例}

让我们回到前面的例子,让这些思想更加清晰。\\

\begin{example}
    之前,在例 \ref{ex:example3.6.8} 中,我们定义了对于 $1$ 到 $10$ 之间的每个自然数 $i$。
    \[A_i = \{i, 2i\}\]
    定义该集合的另一种方法是使用索引集 $I = [10]$ (回想一下符号 $[n] = {i \in \mathbb{N} \mid 1 \le i \le n}$)并将 $A$ 定义为集合
    \[A = \{A_i \mid i \in I\}, \text{其中对于所有} i \in I, A_i = \{i, 2i\}\]
    这定义了每个集合 $A_i$,取决于从索引集 $I$ 中选择的索引值 $i$,并将所有这些集合``收集''到集合 $A$ 中。然后,我们可以基于 $I$ 和 $A_i$ 的定义用另一种方式编写并集
    \[\bigcup_{i \in I} A_i\]
    (仔细思考一下这个并集和集合 $A$ 有什么不同。还有,这个并集到底是什么?我们如何方便的表达它的元素?我们需要列出每个元素吗?如果我们把索引集合 $I$ 改为 $\mathbb{N}$ 呢?上面的并集会是什么?)
\end{example}

\begin{example}
    设 $I = \{1, 2, 3\}$,对于所有 $i \in I$,定义
    \[A_i = \{i - 2, i - 1, i, i + 1, i + 2\}\]
    让我们找出并写下以下集合的元素:
    \[\bigcup_{i \in I} A_i \qquad\text{和}\qquad \bigcap_{i \in I} A_i\]
    请注意,我们可以写出每个 $A_i$ 集合的元素,如下所示:
    \begin{align*}
        A_1 &= \{-1, 0, 1, 2, 3\} \\
        A_2 &= \{0, 1, 2, 3, 4\} \\
        A_3 &= \{1, 2, 3, 4, 5\}
    \end{align*}
    因此
    \[\bigcup_{i \in I} A_i = A_1 \cup A_2 \cup A_3 = \{-1, 0, 1, 2, 3, 4, 5\}\]
    且
    \[\bigcap_{i \in I} A_i = A_1 \cap A_2 \cap A_3 = \{1, 2, 3\}\]
    现在,考虑 $J = {-1, 0, 1}$, $A_j$ 的定义方式与之前相同。让我们来找出下面集合的元素
    \[\bigcup_{j \in J} A_j \qquad\text{和}\qquad \bigcap_{j \in J} A_j\]
    写出每个集合的元素,我们可以确定
    \[\bigcup_{j \in J} A_j = A_{-1} \cup A_0 \cup A_1 = \{-3, -2, -1, 0, 1, 2, 3\}\]
    且
    \[\bigcap_{j \in J} A_j = A_{-1} \cap A_0 \cap A_1 = \{-1, 0, 1\}\]
    尝试使用不同的索引集回答相同的问题。
    例如,考虑 $K = \{1, 2, 3, 4, 5\}$ 或 $L = \{-3, -2, -1, 0, 1, 2, 3\}$。
\end{example}

\begin{example}
    定义索引集 $I = \mathbb{N}$。对于所有 $i \in I$,定义集合
    \[C_i = \Bigg\{x \in \mathbb{R} \mid 1 \le x \le \frac{i + 1}{i}\Bigg\}\]
    则我们声称
    \[\bigcup_{i \in I} C_i = \{y \in \mathbb{R} \mid 1 \le y \le 2\} \qquad\text{和}\qquad \bigcap_{i \in I} C_i = \{1\}\]
    你知道为什么上述陈述为真吗?稍后我们将讨论证明这种等式所需的技术。现在,我们请你思考一下为什么这些都是正确的。你能向同学或朋友解释一下吗?你会用什么技术来证明这些主张? 
\end{example}

\begin{example}
    令 $S$ 为参加这门课程的学生的集合。对于每个 $s \in S$,令 $C_s$ 表示学生 $s$ 本学期所修课程的集合。下列表达式分别代表什么?
    \[\bigcup_{s \in S} C_s \qquad\text{和}\qquad \bigcap_{s \in S} C_s\]
    我们打赌你至少可以找到右侧集合中的一个元素!
\end{example}
