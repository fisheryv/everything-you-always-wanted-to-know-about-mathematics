% !TeX root = ../../../book.tex
\subsection{空集}

如果没有元素满足属性 $P(x)$ 怎么办?会发生什么呢?例如,考虑定义
\[S = \{x \in \mathbb{N} \mid x^2 - 2 = 0\}\]
我们知道,我们正在``寻找''的具有该属性的数字 $x$ 是 $\sqrt{2}$ (和 $-\sqrt{2}$),但是 $\sqrt{2} \notin \mathbb{N}$。因此,无论我们让 $x$ 代表 $\mathbb{N}$ 中的哪个元素,属性 $P(x)$——由 ``$x^2 - 2 = 0$'' 定义——实际上都不满足。因此,该集合中没有元素。这真的是一个集合吗?

请记住,集合完全由其元素来表征,而没有元素的集合(例如上面这个集合)则由该事实来表征。如果我们试图列出它的元素,我们最终会写成 $\{\}$。事实上,这个集合非常特别,我们给它起了一个名字和符号:

\begin{definition}
    空集是没有元素的集合。用符号 $\varnothing$ 表示。
\end{definition}

使用集合构建符定义空集的方法有很多。(是的,我们确实指的是空集;只有一个没有元素的集合!)我们在上面看到了一个例子,我们相信你可以想出许多其他例子。例如,考虑以下集合:
\begin{align*}
    &\{a \in \mathbb{N} \mid a < 0\} \\
    &\{r \in \mathbb{R} \mid r^2 < 0\} \\
    &\{q \in \mathbb{Q} \mid q^2 \notin \mathbb{Q}\}
\end{align*}
你理解为什么这些都定义了一个相同的集合,即没有元素的集合吗?

\subsubsection*{上下文相关}

我们还应再次注意到,在上面的集合构建符定义中,指定较大集合 $X$ 的重要性,我们会从该集合提取变量元素 $X$。例如,考虑以下两个集合:
\begin{align*}
    S_1 &= \{x \in \mathbb{N} \mid |x| = 5\} = \{5\}\\
    S_2 &= \{x \in \mathbb{R} \mid |x| = 5\} = \{-5, 5\}
\end{align*}
(注意:使用下标来索引集合也很常见,这允许我们反复使用相同的字母。)

在这种情况下,规范显然很重要,因为它产生了两个完全不同的集合!因此,我们定义集合时必须精确、清晰。像 $S = \{x \mid |x| = 5\}$ 这样的定义是含糊不清且不受欢迎的,因为它会导致类似上面的问题。
