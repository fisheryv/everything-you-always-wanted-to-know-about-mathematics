% !TeX root = ../../../book.tex
\subsection{空集}

如果没有元素满足性质 $P(x)$,会发生什么?例如,考虑定义
\[S = \{x \in \mathbb{N} \mid x^2 - 2 = 0\}\]
我们知道满足该性质的数 $x$ 是 $\sqrt{2}$ 和 $-\sqrt{2}$,但 $\sqrt{2} \notin \mathbb{N}$。因此,无论取 $\mathbb{N}$ 中的哪个元素 $x$,性质 $P(x)$(即 $x^2 - 2 = 0$)都不成立。这表明该集合不包含任何元素。这样的对象是否构成集合?

集合完全由其元素决定,没有元素的集合则通过``无元素''这一特性定义。若尝试列出其元素,可写作 $\{\}$。这个特殊的集合被称为空集:

\begin{definition}
    空集是没有元素的集合,记作 $\varnothing$。
\end{definition}

尽管用集合建构式符号定义空集的方式多种多样,但它们本质相同(注意:空集唯一存在!)。上文已展示一例,以下是其他例子:

\begin{align*}
    &\{a \in \mathbb{N} \mid a < 0\} \\
    &\{r \in \mathbb{R} \mid r^2 < 0\} \\
    &\{q \in \mathbb{Q} \mid q^2 \notin \mathbb{Q}\}
\end{align*}
你能理解为何这些定义均对应同一个空集吗?

\subsubsection*{上下文相关}

在集合建构式符号定义中,明确限定元素来源的集合 $X$ 至关重要。例如:
\begin{align*}
    S_1 &= \{x \in \mathbb{N} \mid |x| = 5\} = \{5\}\\
    S_2 &= \{x \in \mathbb{R} \mid |x| = 5\} = \{-5, 5\}
\end{align*}

(注意:使用下标区分同名集合是常见做法。)

此处限定范围直接影响结果,导致两个完全不同的集合。因此定义集合时必须精确清晰。类似 $S = \{x \mid |x| = 5\}$ 的表述因缺乏上下文而不严谨。
