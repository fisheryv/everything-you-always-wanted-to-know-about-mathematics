% !TeX root = ../../../book.tex
\subsection{罗素悖论}\label{sec:section3.3.5}

也许本节的内容看上去有点鸡蛋里挑骨头,但我们这样做背后的原因植根于集合论的一些基本思想。我们希望避免在没有这项规则的情况下可能出现的一些复杂问题和悖论。有一个十分著名的集合论悖论说明了为什么我们会有此需求,问题出在当我们使用集合构建符时,我们必须指定一个更大的集合。这个悖论被称为\emph{罗素悖论}(以英国数学家伯特兰·罗素(Bertrand Russell)命名),我们将在本节中介绍和讨论它。

\subsubsection*{集合的集合}

首先,我们需要指出,本节讨论将引入集合也可以是其他集合的元素的概念。这貌似是一个怪异且牵强的抽象想法,但它是数学中的一个基本概念。
举个具体的例子,回想一下所有 NBA 球队的集合 $B$。我们也可以将每个球队视为一个集合,其中的元素是球队中的球员。因此,可以说
\[\text{勒布朗·詹姆斯} \in \text{洛杉矶湖人队} \in B\]
因为 $\text{勒布朗·詹姆斯}$ 是集合 $\text{洛杉矶湖人队}$ 的元素,而 $\text{洛杉矶湖人队}$ 本身又是集合 $B$ 的元素。(但是请注意,$\text{勒布朗·詹姆斯} \notin B$。``$\in$'' 所表示的关系不具有\textbf{传递性}。我们将在后面定义这些术语。现在,我们说 ``$\le$'' 在实数集上表示的关系具有传递性。如果我们知道 $x \le y \le z$,那么我们可以推导出 $x \le z$。 但 ``$\in$'' 关系并非如此。)

另一个例子是 $S = \{1, 2, 3, \{10\}, \varnothing \}$。是的,空集本身可以是另一个集合的元素,集合 $\{10\}$ 也可以。为什么他们可以呢?作为思维训练,我们建议你思考一下 $\varnothing, \{\varnothing\}, \{\{\varnothing\}\}$ 之间的区别。为什么它们是不同的集合?

最后一个例子涉及自然数 $\mathbb{N}$。我们用 $\mathbb{O}$ 和 $\mathbb{E}$ 分别表示\emph{奇数}和\emph{偶数}。那么,集合 $S = \{\mathbb{O}, \mathbb{E}\}$ 是什么?它与 $\mathbb{N}$ 有何不同(如果有的话)?这是一个微妙的问题,所以要仔细思考哦。

\subsubsection*{矛盾的``集合''}

集合的集合这一概念值得花点时间仔细思考。不过,现在让我们继续讨论罗素悖论。考虑以下``集合''的定义。这里的``集合''加了引号是因为它实际上不是一个正确定义的集合,至于为什么会这样还有待考察。当我们理解它为什么不是集合后,在我们使用集合构建符时,这将成为需要指定更大集合的论据;这是因为下面的定义没有指定更大的集合。
\[\mathcal{R} = \{x \mid x \notin x\}\]
这是一个集合吗?$\mathcal{R}$ 的元素是什么?想想上面的定义所说的:$\mathcal{R}$ 的元素是恰巧不以自身为元素的集合。你能找出 $\mathcal{R}$ 的任何元素吗?你能找出不是 $\mathcal{R}$ 中元素的对象吗?

第一个问题更容易回答:到目前为止我们讨论的任何集合都是 $\mathcal{R}$ 的元素。例如,空集 $\varnothing$ 不包含任何元素,因此它本身肯定不具有元素。所以,$\varnothing \in \mathcal{R}$。另外,请注意 $\mathbb{N} \notin \mathbb{N}$(因为自然数集合本身不是自然数),所以 $\mathbb{N} \in \mathcal{R}$。

找出不是 $\mathcal{R}$ 中元素的对象是一件非常棘手的事情,我们通过提出以下问题来帮助你思考:$\mathcal{R}$ 本身是一个元素吗? $\mathcal{R} \in \mathcal{R}$ 是真是假?在继续阅读之前请先仔细思考这一点。我们将引领你如何正确的思考。

\begin{itemize}
    \item 假设 $\mathcal{R} \in \mathcal{R}$ 为真 \\
    $\mathcal{R}$ 的定义属性告诉我们,它的任何元素都是一个不以自身为元素的集合。由此,我们可以推导出 $\mathcal{R} \notin \mathcal{R}$。\\
    等一下!知道 $\mathcal{R} \in \mathcal{R}$ 使我们推导出 $\mathcal{R} \notin \mathcal{R}$。当然,这两个矛盾的事实不能同时成立。因此,一定是我们原来的假设有问题,所以一定是 $\mathcal{R} \notin \mathcal{R}$。
    \item 假设 $\mathcal{R} \notin \mathcal{R}$ 为真 \\
    $\mathcal{R}$ 的定义属性告诉我们,任何不是 $\mathcal{R}$ 元素的对象都必须是其自身的元素。(否则,它会被包含为 $\mathcal{R}$ 的元素。)因此,我们可以推导出 $\mathcal{R} \in \mathcal{R}$。\\
    等一下!知道 $\mathcal{R} \notin \mathcal{R}$ 使我们推导出 $\mathcal{R} \in \mathcal{R}$。这也是矛盾的。
\end{itemize}
无论我们选择哪一个—— $\mathcal{R} \in \mathcal{R}$ 还是 $\mathcal{R} \notin \mathcal{R}$——我们都会发现另一个也一定为真,然而这些相互矛盾的事实不可能同时为真。

这就是\textbf{悖论}。$\mathcal{R}$ 不是一个正确定义的集合。如果 $\mathcal{R}$ 是集合,我们就会发现自己陷入刚刚看到的两难境地,而这两种情况都不为真。而 $\mathcal{R}$ 也不只是空集 $\varnothing$;所以唯一的可能是 $\mathcal{R}$ 不是集合。

\subsubsection*{``所有集合的集合''\emph{不是}集合}

我们能否以某种方式修改 $\mathcal{R}$ 的定义,以产生该定义试图描述的``集合''?我们应该从哪个``更大的集合''中提取对象 $x$ ,以确保定义有意义并正确定义集合?

回顾一下我们对 $\mathcal{R}$ 定义的中文解释:``$\mathcal{R}$ 的元素是恰巧不以自身作为元素的集合。''我们需要检验所需属性 ($x \notin x$) 的对象 $x$ 实际上都是集合。那么,也许我们应该将 $X$ 定义为所有集合的集合,并使用短语 ``$x \in X$'' 作为 $\mathcal{R}$ 定义的一部分。这样不就解决了吗?
\[\mathcal{R} = \{x \in X \mid x \notin x\}\]

不,完全不是这样!\textbf{``所有集合的集合''本身并不是一个集合}。如果是的话,这将导致我们陷入与之前完全相同的悖论!唯一的区别在于我们会明确指出``更大的集合'',从中我们可以得到之前隐式指定的对象 $x$。

主要问题是,不指定从中提取对象的``更大的集合'',或者隐式引用``所有集合的集合'',会导致这种令人讨厌的悖论。因此,我们决不能允许这样的定义。任何试图从``所有集合的集合''中提取对象 $x$ 来定义一个集合,无论是隐式的还是显式的,都不是集合的正确定义。

\subsubsection*{进一步探讨}

不过,``$x \notin x$'' 给出的属性 $P(x)$ 并没有本质上的错误。问题在于我们使用的``更大的集合''。例如,拿下面这个集合来说,
\[S = \Bigg\{x \in \bigg\{\frac{1}{2}, \frac{3}{4}, \frac{5}{2}\bigg\} \mid x \notin x \Bigg\}\]
它的元素是什么?唯一的可能性是从更大的集合 $\{\frac{1}{2}, \frac{3}{4}, \frac{5}{2}\}$ 中提取的元素。请注意,这些数字都不是包含自身作为元素的集合。因此,这是集合 $\{\frac{1}{2}, \frac{3}{4}, \frac{5}{2}\}$ 本身的正确定义!根据前面 $\mathcal{R}$ 的定义,我们试图定义的对象在其自己的定义中被允许作为变量对象 $x$ 之一,这就是问题产生的地方。

可能我们稍稍偏离了最初讨论的主题,但我们认为重要的是要指出,有可能构建不明确定义的``集合'',而这些集合不是数学意义上的集合。在大多数情况下,我们在本书中使用的集合不会遇到此类问题,但掩盖这些问题或根本置之不理对作为学生的你来说不公平。如果你发现自己对这些问题感兴趣,可以找一本关于集合论的入门书来阅读。

``集合''的定义也有其他形式的错误,但接下来的例子来源于语言问题,而非数学基础出了问题,如罗素悖论。例如,我们可以说``设 $N$ 为 20 世纪所有经典小说的集合''。``经典小说''并不是一个明确定义的属性,无法用来确定集合中的元素。``经典''的概念是主观的,并不是严格精确的。此外,我们还可以说``设 $B$ 为明天出生的人的集合'',但定义中的这种时间依赖性使我们永远无法真正知道 $B$ 的元素是什么。当明天到来时,明天指的将是第二天,依此类推。你能举出其他形式不正确的元素``集合''的例子吗?你能想出像上面那样的悖论吗?

总的来说,以下陈述是从罗素悖论的讨论中得出的最重要的思想:

\begin{center}根据约定的集合规则(集合论公理),\textbf{不存在}所有集合的集合。\end{center}
