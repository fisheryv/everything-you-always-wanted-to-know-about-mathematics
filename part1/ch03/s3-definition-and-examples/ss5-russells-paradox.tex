% !TeX root = ../../../book.tex
\subsection{罗素悖论}\label{sec:section3.3.5}

本节内容看似有些吹毛求疵,但我们的考量源于集合论的基本思想。为了避免缺少相关规则时可能出现的复杂问题与悖论,我们必须建立严谨的框架。著名的\emph{罗素悖论}(以英国数学家伯特兰·罗素 (Bertrand Russell) 命名)正说明了这种必要性——其问题在于使用集合建构式符号时,必须指定一个更大集合以避免矛盾。本节将深入探讨这一悖论。

\subsubsection*{集合的集合}

首先需要明确,集合本身也可作为其他集合的元素。这虽是抽象概念,却是数学的基础。以 NBA 球队集合 $B$ 为例:每支球队可视为球员的集合,因此有
\[\text{勒布朗·詹姆斯} \in \text{洛杉矶湖人队} \in B\]
其中\verb|勒布朗·詹姆斯|是集合\verb|洛杉矶湖人队|的元素,而\verb|洛杉矶湖人队|本身又是集合 $B$ 的元素。(但请注意,$\text{勒布朗·詹姆斯} \notin B$。``$\in$'' 所表示的关系不具有\textbf{传递性}。我们将在后面定义这些术语。作为对比,实数集上的 ``$\le$'' 关系具有传递性。如果我们已知 $x \le y \le z$,则可推导出 $x \le z$。 但 ``$\in$'' 关系不满足此性质。)

再考察集合 $S = \{1, 2, 3, \{10\}, \varnothing \}$。没错,空集本身可以作为另一个集合的元素,集合 $\{10\}$ 也可以。作为思维训练,建议你思考一下 $\varnothing, \{\varnothing\}, \{\{\varnothing\}\}$ 之间的区别。为什么它们是不同的集合?

最后考虑自然数集 $\mathbb{N}$。设 $\mathbb{O}$ 和 $\mathbb{E}$ 分别表示\emph{奇数集}和\emph{偶数集}。那么,集合 $S = \{\mathbb{O}, \mathbb{E}\}$ 表示什么?它与 $\mathbb{N}$ 有何本质区别?这是一个微妙的问题,需要仔细辨析。

\subsubsection*{矛盾的``集合''}

集合的集合这一概念值得仔细思考。不过,现在让我们继续讨论罗素悖论。考虑以下``集合''的定义——这里的引号表示它可能不是一个正确定义的集合,其原因有待分析。理解这一点后,我们在使用集合建构式符号时,将认识到需要指定更大集合的必要性,因为下述定义未能满足这一要求:
\[\mathcal{R} = \{x \mid x \notin x\}\]
$\mathcal{R}$ 真的是集合吗?它的元素是什么?根据定义,$\mathcal{R}$ 的元素是那些不以自身为元素的集合。你能找到属于 $\mathcal{R}$ 的集合吗?又能找到不属于 $\mathcal{R}$ 的对象吗?

第一个问题较易回答:目前讨论的所有集合都属于 $\mathcal{R}$。例如空集 $\varnothing$ 不含任何元素,故 $\varnothing \notin \varnothing$,因此 $\varnothing \in \mathcal{R}$。同理,$\mathbb{N} \notin \mathbb{N}$(自然数集本身不是自然数),故 $\mathbb{N} \in \mathcal{R}$。

寻找不属于 $\mathcal{R}$ 的对象则较为困难。思考这个关键问题:$\mathcal{R}$ 自身是否属于 $\mathcal{R}$?$\mathcal{R} \in \mathcal{R}$ 是否成立?请先自行思考,我们将引导你进行严谨推导。

\begin{itemize}
    \item 假设 $\mathcal{R} \in \mathcal{R}$ 为真 \\
    根据 $\mathcal{R}$ 的定义,其元素必须满足 $x \notin x$。因此可导出 $\mathcal{R} \notin \mathcal{R}$。\\
    等一下!知道 $\mathcal{R} \in \mathcal{R}$ 使我们推导出 $\mathcal{R} \notin \mathcal{R}$。显然这两个相互矛盾的事实不能同时成立。因此,原假设有误,所以必有 $\mathcal{R} \notin \mathcal{R}$。
    \item 假设 $\mathcal{R} \notin \mathcal{R}$ 为真 \\
    $\mathcal{R}$ 的定义表明:任何对象若满足 $x \notin x$,则必属于 $\mathcal{R}$。因此可得出 $\mathcal{R} \in \mathcal{R}$。\\
    等一下!知道 $\mathcal{R} \notin \mathcal{R}$ 使我们推导出 $\mathcal{R} \in \mathcal{R}$。这显然也是矛盾的。
\end{itemize}
无论假设 $\mathcal{R} \in \mathcal{R}$ 或 $\mathcal{R} \notin \mathcal{R}$,都会导出矛盾的结论。

这就是\textbf{悖论}。$\mathcal{R}$ 不是一个正确定义的集合——若它是集合,必然导致上述两难困境。$\mathcal{R}$ 也非空集 $\varnothing$,唯一合理的结论是:$\mathcal{R}$ 根本不能作为集合存在。

\subsubsection*{``所有集合的集合''\emph{不是}集合}

我们能否修改 $\mathcal{R}$ 的定义,使其能够产生它试图描述的``集合''?我们应该从哪个``更大集合''中提取对象 $x$,以确保定义有意义并正确定义集合?

回顾 $\mathcal{R}$ 定义的中文解释:``$\mathcal{R}$ 的元素是恰好不以自身作为元素的集合。''我们需要检验具有属性 $x \notin x$ 的对象 $x$ 是否都是集合。那么,或许应该将 $X$ 定义为所有集合的集合,并在 $\mathcal{R}$ 的定义中加入``$x \in X$''。这样不就解决了吗?
\[\mathcal{R} = \{x \in X \mid x \notin x\}\]

不,完全不是这样!\textbf{``所有集合的集合''本身并不是一个集合}。如果它是集合,将导致与之前完全相同的悖论!唯一的区别在于我们明确指定了从中提取对象 $x$ 的``更大集合''。

核心问题在于,如果不指定提取对象的``更大集合'',或者隐式引用``所有集合的集合'',就会导致这种悖论。因此,我们决不能允许这样的定义。任何试图从``所有集合的集合''中提取对象 $x$ 来定义集合的方式,无论是隐式的还是显式的,都不是正确定义。

\subsubsection*{进一步探讨}

不过,属性 $P(x)$ 由``$x \notin x$''给出本身并无本质错误。问题在于所使用的``更大集合''。例如,考虑以下集合:
\[S = \Bigg\{x \in \bigg\{\frac{1}{2}, \frac{3}{4}, \frac{5}{2}\bigg\} \mid x \notin x \Bigg\}\]
它的元素是什么?只能从更大的集合 $\{\frac{1}{2}, \frac{3}{4}, \frac{5}{2}\}$ 中提取。这些数字都不是包含自身作为元素的集合,因此该集合的正确定义为 $\{\frac{1}{2}, \frac{3}{4}, \frac{5}{2}\}$ 本身!在 $\mathcal{R}$ 的定义中,我们试图定义的对象本身被允许作为变量 $x$ 之一,这就是问题所在。

我们可能稍微偏离了最初的主题,但指出有可能构建非数学意义上的``集合''至关重要。本书中大多数集合不会遇到此类问题,但掩盖或完全忽视它们对作为学生的你是不公平的。如果你对这些内容感兴趣,可以阅读集合论的入门书籍。

``集合''的定义还有其他形式的错误,但接下来的例子源于语言问题,而非数学基础(如罗素悖论)。例如,``设 $N$ 为 20 世纪所有经典小说的集合''中,``经典小说''不是明确定义的属性。``经典''是主观概念,并非严格精确。类似地,``设 $B$ 为明天出生的人的集合''对时间的依赖性使我们永远无法确定 $B$ 的元素:当明天到来时,它指向下一个日期,依此类推。你还能举出其他形式不正确的``集合''例子吗?你能构造类似悖论吗?

总之,对罗素悖论的讨论得出以下核心结论:

\begin{center}根据集合论公理(约定的集合规则),\textbf{不存在}所有集合的集合。\end{center}
