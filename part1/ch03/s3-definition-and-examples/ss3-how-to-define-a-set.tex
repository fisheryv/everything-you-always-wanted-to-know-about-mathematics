% !TeX root = ../../../book.tex
\subsection{如何定义集合}

定义或描述集合的另一种方法是直接列出其所有元素。当集合元素数量较少时,这种方法十分简便。例如,集合 $V$ 的以下定义都是等价的:
\begin{align*}
    V &= \{A,E,I,O,U\}\\
    V &= \{\text{英语中的元音}\}\\
    V &= \{U,E,I,A,O\}
\end{align*}
``等价''意味着虽然上述定义使用了不同表述,但都确定了\emph{相同的}集合 $V$。(注意:我们约定 $y$ 为辅音,故 $y \notin V$。)元素 $A, E, I, O, U$ 的共同属性在于它们都是元音(如第二个定义所示)。由于仅有五个元素,完整列举既简单又便捷(如第一个定义所示)。

\subsubsection*{顺序和重复无关紧要}

第三个定义为何与其他定义等价?因为它指代相同的元素整体——集合完全由其元素决定,因此元素的\emph{顺序无关紧要}。$U \in V$ 是否成立?答案是``肯定的'',无论 $U$ 位于列表的位置。

不仅元素顺序无关,元素的\emph{重复}也不影响集合的性质!集合 $A=\{a,a,a\}$ 与 $A=\{a\}$ 完全相同。集合的本质仅取决于其元素内容(我们将在 \ref{sec:section3.4.4} 节``口袋类比''中再次讨论)。$A = \{a, a, a\}$ 仅表示 $a \in A$ 重复三次,而 $A$ 实际仅含唯一元素 $a$。因此 $A = \{a\}$ 是最简洁的表述方式。

\subsubsection*{元素自身可构成共同属性}

延续通过列举元素定义集合的思路,考虑集合 $A$:
\[A=\{2, 7, 12, 888\}\]
这显然是一个集合。但元素间的共同属性是什么?元音集合 $V$ 可辅以语言描述,而 $A$ 似乎仅能罗列元素。数学上,$2,7,12,888$ 的共同属性恰恰是它们都属于集合 $A$!在抽象领域中,仅通过定义集合 $A$ 本身便赋予其元素共同属性。这合理吗?你能否找到\emph{另一个}明确定义的共同属性来精确生成 $A$ 的元素?(提示:构造多项式 $p(x)$ 使其根恰好为 $2,7,12,888$。)若元素具有多重关联属性,你认为选择哪一属性定义集合重要吗?如何理解集合 $S := \{2, 7, \text{M}, \text{波士顿凯尔特人队}\}$?除``同属此集合''外,它们是否存在其他共同属性?

\subsubsection*{省略号可用但非正式方法}

当集合定义无歧义或已通过其他方式明确定义时,可以列举示例元素并用省略号压缩元素列表。例如:
\[E = \{\text{所有偶数}\} = \{2, 4, 6, 8, 10, \dots\}\]
事实上,此集合为\emph{无限集},无法穷举所有元素,但前几项已清晰表明指代偶数集,且主定义 $E$ 为``所有偶数的集合''已明确其含义。需要强调的是,此方法并非精确的数学定义,仅适用于非正式场合。下一小节讨论集合的正规定义方法时,我们将进一步阐明这一点。

\subsubsection*{集合建构式符号}

定义或描述集合的最佳方法是将其元素限定为具有特定属性的另一个集合的元素。例如,若希望表示 $1$ 到 $100$(含)之间所有自然数的集合 $S$,虽可列出所有元素,但这过于冗长。也可使用省略号法 $S = \{1, 2, 3, \dots , 100\}$,但此方式缺乏正式定义,仍不够精确(不同读者可能对省略号有不同理解)。更精确且简洁的写法是:
\[S = \{x \in \mathbb{N} \mid 1 \le x \le 100\}\]
我们将此式理解为``$S$ 是自然数集 $\mathbb{N}$ 中所有满足 $1 \le x \le 100$ 的元素 $x$ 构成的集合''。

竖线符号 $\mid$ 表示``\textbf{满足}'',其左侧指定对象来自哪个``更大集合'',右侧描述对象应满足的属性。

(\textcolor{red}{注意}:\emph{请勿}在其他场景下用 $\mid$ 表示``满足''。该符号仅在定义集合时作为分隔符使用,用以区分左侧的元素来源集与右侧的属性描述。)

这是广泛使用的\textbf{集合建构式符号}示例。其核心在于从``更大''集合中\emph{筛选}具有特定属性的元素来\emph{构建}新集合,为此需要明确:
\begin{enumerate}[label=(\arabic*)]
    \item 更大集合是什么;
    \item 共同属性是什么。
\end{enumerate}

让我们用几个例子进一步说明:
\begin{align*}
    S &= \{x \in \mathbb{N} \mid 1 \le x \le 100\} = \{1, 2, 3, \dots , 100\} \\
    T &= \{z \in \mathbb{Z} \mid \text{存在某\ } k \in \mathbb{Z} \text{\ 使得\ } z = 2k\} = \{\dots , -4, -2, 0, 2, 4, \dots\} \\
    U &= \{x \in \mathbb{R} \mid x^2 - 2 = 0\} = \{-\sqrt{2}, \sqrt{2}\}\\
    V &= \{x \in \mathbb{N} \mid x^2 - 2 = 0\}= \{ \}
\end{align*}

最后两个例子凸显出上下文的重要性:当改变元素来源的\emph{更大集合}时,相同的属性条件($x^2 -2 = 0$)会产生不同的集合。该条件在实数范围内有两个解,但在自然数范围内无解!是否存在满足该条件的有理数?请思考。

这就解释了为何必须明确指定更大集合。类似``$U = \{x \mid x^2 - 2 = 0\}$''的定义\emph{没有意义},因其存在歧义,可能导致完全不同的解释。

\subsubsection*{朗读建构式符号}

我们正在学习一门新的\textbf{语言},前述内容涵盖基本词汇与语法规则。需要通过练习将这些数学表达式转化为汉语(在脑海中大声朗读),反之亦然。例如,可将集合 $S$ 合理定义为以下任一形式:
\begin{itemize}
    \item $S$ 是所有满足 $1 \leq x \leq 100$ 的自然数 $x$ 的集合。
    \item $S$ 是介于 $1$ 到 $100$(含端点)之间的全体自然数的集合。
    \item $S$ 是满足不等式 $1 \leq x \leq 100$ 的自然数 $x$ 的集合。
    \item $S$ 是满足 $1 \le x \le 100$ 属性的自然数 $x$ 的集合。
\end{itemize}
请注意,这些定义均通过指定更大集合与共同性质来限定元素;其差异仅在语言表述层面,数学本质完全一致。

请尝试为其他集合撰写类似定义。可收集他人对集合的口头描述,并将其转化为数学符号。

回顾有理数集 $\mathbb{Q}$ 的定义,可将其重写为:
\begin{align*}
    \mathbb{Q} &= \Big\{\frac{a}{b}, \text{其中\ } a,b \in \mathbb{Z} \text{\ 且\ } b \ne 0\Big\}\\
               &= \Big\{x \in \mathbb{R} \mid \text{存在\ } a, b \in \mathbb{Z} \text{\ 使得\ } \frac{a}{b}= x \text{\ 且\ } b \ne 0\Big\}
\end{align*}
请注意这两个定义的细微差别:前者强调有理数呈现 $\frac{a}{b}$ 的\textbf{形式}并限定参数条件;后者则表明有理数是满足特定性质的实数,可以表示为整数之比。后者更受青睐,因为它提供了更丰富的信息。

一般来说,若 $P(x)$ 表示明确定义的性质(自然语言或数学语言),$X$ 为给定集合,则符号
\[S = \{x \in X \mid P(x)\}\]
读作
\begin{center}``$S$ 是集合 $X$ 中所有满足性质 $P(x)$ 的元素 $x$ 的集合''。\end{center}
在符号 $P(x)$ 中,字母 $x$ 表示变量对象,根据我们输入 $x$ 的特定对象,属性 $P(x)$ 可能成立(即 $P(x)$ 为真)也可能不成立(即 $P (x)$ 为假)。若性质成立,则我们将 $x$ 包含在 $S$ 中(因此 $x \in S$);若不成立,则我们将 $x$ 排除在 $S$ 外(因此 $x \notin S$)。

以偶数集 $E$ 为例,其精确定义为:
\begin{align*}
    E &= \{\text{偶数}\} \\
      &= \{x \in \mathbb{N} \mid \text{存在自然数\ } n \text{\ 使得\ } x = 2n\}
\end{align*}
请注意,此处存在两层性质判定:自然数 $x$ 属于 $E$ 当且仅当存在另一自然数 $n$ 满足 $x = 2n$。请尝试为奇数集、平方数集、质数集、回文数集、完美数集等构造类似定义。你能否运用集合构造器为这些集合建立精确的数学表达?
