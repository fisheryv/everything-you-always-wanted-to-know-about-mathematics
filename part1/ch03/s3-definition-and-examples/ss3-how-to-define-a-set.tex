% !TeX root = ../../../book.tex
\subsection{如何定义集合}

定义或描述集合的另一种方法是简单地列出其所有元素。当集合中的元质数量较少时,这种方法很方便。例如,集合 $V$ 的以下定义都是等价的:
\begin{align*}
    V &= \{A,E,I,O,U\}\\
    V &= \{\text{英语中的元音}\}\\
    V &= \{U,E,I,A,O\}
\end{align*}
``等价''的意思是,虽然上面每一行使用了不同的术语,但定义了\emph{相同的}集合 $V$。(注意:我们采用了 $y$ 是辅音的约定,因此 $y \notin V$。)与对象 $A, E, I, O, U$ 相关联的共同明确定义的属性是它们都是元音这一事实(如第二个定义所示)并且由于只有五个这样的对象,因此将它们全部列出来既简单又方便(如第一个定义所示)。

\subsubsection*{顺序和重复无关紧要}

为什么你认为第三个定义与其他定义相同?它指的是同一个对象整体,任何集合都完全由其元素来表征,因此我们编写元素的\emph{顺序无关紧要}。$U \in V$ 吗?这个问题的答案是``肯定的'',无论 $U$ 是写在元素列表的第一个还是最后一个。

不仅集合中元素的顺序无关紧要,元素的\emph{重复}也无关紧要!即,集合 $A=\{a,a,a\}$ 和集合 $A=\{a\}$ 完全相同。再次强调,集合的特征完全由它的元素决定。我们只关心集合``中''的内容。(当我们讨论集合的``口袋类比''时,我们将在第 \ref{sec:section3.4.4} 节中再次提到这一点。)写 $A = \{a, a, a\}$ 只是 $a \in A$ 重复了三次,$A$ 中只有唯一一个元素 $a$。因此,$A = \{a\}$ 是陈述相同信息的最简洁方式。

\subsubsection*{是集合的元素本身可能就是共同属性}

现在,仍然遵循可以通过编写所有元素来定义集合的思想,考虑集合 $A$ 的以下定义:
\[A=\{2, 7, 12, 888\}\]
这是当然一个集合,因为我们只是通过列出其元素来定义它。但是,关联其元素的共同明确定义的属性是什么呢?对于元音集合 $V$,我们可以列出元素并提供语言上的定义,但对于集合 $A$,我们似乎只能列出元素而不知道如何\emph{描述}其共同属性。不过,从数学上来讲,$2,7,12,888$ 的一个共同属性是它们都是集合 $A$ 的元素!在数学宇宙中,我们拥有抽象思维的自由,并且仅仅通过讨论这个集合 $A$ 及其元素,我们就赋予了它们共同的属性。这让你满意吗?你能想出\emph{另一个}常见的、明确定义的属性来精确地产生集合 $A$ 的元素吗?(提示:确定一个多项式 $p(x)$,其根恰好为 $2,7, 12, 888$。)如果集合中的元素具有多个将它们关联在一起的属性,那么你认为在引用该集合时选择哪个属性重要吗?你如何看待集合 $S := \{2, 7, \text{M}, \text{波士顿凯尔特人队}\}$?除了我们在这里列出了它们之外,是否可能存在共同属性?

\subsubsection*{省略号有时没问题,但不够正式}

有时,当所讨论的集合没有歧义,或者已经以另一种方式定义,并且我们希望列出一些元素作为说明性示例时,那么使用省略号来压缩集合元素的列表会很方便。例如,我们可以这样写
\[E = \{\text{所有偶数}\} = \{2, 4, 6, 8, 10, \dots\}\]
事实上,这个集合是个\emph{无限集},所以我们无法列出它的所有元素,但是从列出的前几个元素可以清楚地看出我们指的是偶数,更主要的是我们已经指出 $E$ 为``偶数的集合''。然而,这里必须强调,这并不是所讨论集合的精确定义。它在非正式环境中可以,但在数学上并不严格,下一小节我们讨论定义集合的正确方法时,这一点就会变得清晰起来。

\subsubsection*{集合建构符}

定义或描述集合的最佳方法是将其元素标识为具有特定属性的另一个集合的特定对象。例如,如果我们希望引用 $1$ 到 $100$(含)之间所有自然数的集合 $S$,我们可以列出所有这些元素,但这需要大量不必要的书写。我们还可以使用省略号法 $S = {1, 2, 3, \dots , 100}$,但同样,在没有 $S$ 的正式定义的情况下,这是不精确的。(有人可能会以不同的方式误解省略号。)这样写会更加精确和简洁
\[S = \{x \in \mathbb{N} \mid 1 \le x \le 100\}\]
我们将其理解为``$S$ 是自然数集 $\mathbb{N}$ 中所有 $x$ 对象的集合,满足 $1 \le x \le 100$''。

竖线符号 $\mid$ 理解为``\textbf{满足}'',表示其左边的信息告诉我们对象来自哪个``更大的集合'',右边的信息告诉我们这些对象应该具有的特定属性。

(\textcolor{red}{注意}:\emph{请勿}在其他场景下用 $\mid$ 表示满足。仅在定义集合的情况下才这么用。它只是用作占位符,将左侧(用来获取元素的集合)与右侧(这些元素应具有属性的描述)分开。)

这是非常流行且有用的\textbf{集合建构符}的示例。我们之所以这样称呼它,是因为我们正在通过从``更大''的集合中提取元素来\emph{构建}一个集合,并且只包括那些具有特定属性的元素。为此,我们需要告知读者
\begin{enumerate}[label=(\arabic*)]
    \item 更大的集合是什么;
    \item 共同属性是什么。
\end{enumerate}
让我们用几个例子来说明这个问题:
\begin{align*}
    S &= \{x \in \mathbb{N} \mid 1 \le x \le 100\} = \{1, 2, 3, \dots , 100\} \\
    T &= \{z \in \mathbb{Z} \mid \text{存在某}\; k \in \mathbb{Z} \;\text{使得}\; z = 2k\} \\
      &= \{\dots , -4, -2, 0, 2, 4, \dots\} \\
    U &= \{x \in \mathbb{R} \mid x^2 - 2 = 0\} = \{-\sqrt{2}, \sqrt{2}\}\\
    V &= \{x \in \mathbb{N} \mid x^2 - 2 = 0\}= \{ \}
\end{align*}

最后两个例子表明上下文是多么的重要。当我们更改从中提取元素的\emph{较大集合}时,相同的共同属性(满足 $x^2 -2 = 0$)可以得到不同元素的集合。两个实数满足该性质,但没有自然数满足该性质!是否有任何有理数满足该性质?你怎么认为?

这就解释了为什么指定更大的集合是绝对必要的。类似 ``$U = \{x \mid x^2 - 2 = 0\}$'' 这样的定义是\emph{没有意义的},因为它是有歧义的,可能产生完全不同的解释。

\subsubsection*{朗读建构符}

我们真的是在学习一门新的\textbf{语言},上面这些都是一些基本的词汇和语法规则。我们需要一些练习将这些句子翻译成汉语(在我们的脑海中大声读出来),反之亦然。例如,我们可以合理地将上面的 $S$ 定义为以下任何一个:

\begin{itemize}
    \item $S$ 是所有自然数 $x$ 的集合,其中 $x$ 介于 $1$ 到 $100$ 之间(含 $1$ 和 $100$)。
    \item $S$ 是 $1$ 到 $100$(含 $1$ 和 $100$)之间所有自然数的集合。
    \item $S$ 是满足不等式 $1 \le x \le 100$ 的所有自然数 $x$ 的集合。
    \item $S$ 是满足 $1 \le x \le 100$ 属性的自然数 $x$ 的集合。
\end{itemize}
请注意,它们都确定了更大的集合和共同属性;它们之间唯一的区别是语言/语法上的区别,但不会改变其数学含义。

试着为其他定义写出类似的语句。试着从朋友那里获取集合的口头定义,并用数学符号写下他们所说的内容。

考虑我们之前看过的有理数 $\mathbb{Q}$ 的定义,并注意我们可以将其重写为:
\begin{align*}
    \mathbb{Q} &= \Big\{\frac{a}{b}, \text{其中}\; a,b \in \mathbb{Z} \;\text{且}\; b \ne 0\Big\}\\
               &= \Big\{x \in \mathbb{R} \mid \text{存在}\; a, b \in \mathbb{Z} \;\text{使得}\; \frac{a}{b}= x \;\text{且}\; b \ne 0\Big\}
\end{align*}
请注意这两个定义之间的细微差别。上面一个告诉我们所有有理数都是 $\frac{a}{b}$ \textbf{的形式},然后告诉我们 $a$ 和 $b$ 必须满足的特定条件。下一个告诉我们所有有理数都是具有特定属性的实数,我们可以将该实数表示为整数之比的形式。相比之下我们更喜欢后者,因为它向我们提供了更多的信息。

一般来说,如果 $P(x)$ 表示一个描述特定明确定义的属性的句子(自然语言和/或数学语言),并且 $X$ 是给定的集合,那么符号
\[S = \{x \in X \mid P(x)\}\]
读作
\begin{center}``$S$ 是集合 $X$ 中所有元素 $x$ 的集合,使得属性 $P(x)$ 为真''。\end{center}
在符号 $P(x)$ 中,字母 $x$ 表示变量对象,根据我们输入 $x$ 的特定对象,属性 $P(x)$ 可能成立(即 $P(x)$ 为真)也可能不成立(即 $P (x)$ 为假)。如果该性质成立,则我们将 $x$ 包含在 $S$ 中(因此 $x \in S$),如果不成立,则我们不将 $x$ 包含在 $S$ 中(因此 $x \notin S$)。

回到偶数集合 $E$ 的例子,更精确的写法是
\begin{align*}
    E &= \{\text{偶数}\} \\
      &= \{x \in \mathbb{N} \mid \text{存在自然数}\; n \;\text{使得}\; x = 2n\}
\end{align*}
请注意,这里有两个属性``层''。如果我们可以找到\emph{另一个}具有 $x = 2n$ 属性的自然数 $n$,则自然数 $x$ 包含在我们的集合 $E$ 中。尝试写出奇数集或平方数集的类似定义。那么质数集呢?回文数集呢?完美数集呢?你可以使用集合构建符为这些集合编写定义吗?
