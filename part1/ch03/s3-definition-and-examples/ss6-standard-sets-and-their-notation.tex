% !TeX root = ../../../book.tex
\subsection{标准集及其符号}

我们已经引用并使用了一些常见的数集,现在我们列出这些数集及其标准符号:

\begin{center}
\begin{itemize}
    \item \emph{自然数}:$\mathbb{N} = \{1, 2, 3, 4, \dots \}$
    \item \emph{前 $n$ 个自然数}:$[n] := \{1, 2, 3, \dots , n-1, n \}$
    \item \emph{整数}:$\mathbb{Z} = \{\dots, -3, -2, -1, 0, 1, 2, 3, \dots \}$
    \item \emph{有理数}:$\mathbb{Q} = \{\frac{m}{n} \mid m,n \in \mathbb{Z} \;\text{且}\; b \ne 0 \}$
    \item \emph{实数}:$\mathbb{R}$
    \item \emph{复数}:$\mathbb{C}$
\end{itemize}
\end{center}

我们已经使用过 $\mathbb{N}$ 和 $\mathbb{Z}$ 好几次了。有理数 $\mathbb{Q}$(我们使用 $\mathbb{Q}$ 是因为 $\mathbb{R}$ 已被占用,并且有理数就是商(Quotient),因此取其首字母 $\mathbb{Q}$)是所有分数或整数比,包括正数和负数。实数就更难描述了。为什么我们不能像列出 $\mathbb{N}$ 和 $\mathbb{Z}$ 那样列出其元素?为什么 $\mathbb{R} \ne \mathbb{Q}$?目前,我们基本上认为我们对这些数集的知识是理所当然的,但还是请思考一下。(我们提到复数 $\mathbb{C}$ 是因为你可能熟悉它们,但我们不会在本书中使用复数。)

我们怎么知道像 $\mathbb{N}$ 这样的集合存在?为什么我们将 $\mathbb{R}$ 视为数轴?与 $\mathbb{N}$ 相比,$\mathbb{Z}$ ``多出''多少元素?与 $\mathbb{Q}$ 相比,$\mathbb{R}$ ``多出''多少元素?我们能回答这些问题吗?在不久的将来,我们将严格推导集合 $\mathbb{N}$ 并证明它是唯一具有特定属性的集合。当我们回到数学归纳法的研究时,这一点至关重要。(还记得那一章我们的目标吗?)
