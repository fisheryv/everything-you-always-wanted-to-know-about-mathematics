% !TeX root = ../../../book.tex
\subsection{标准集及其符号}

我们已经介绍并使用了一些常见的数集,现在列出这些数集及其标准符号:

\begin{center}
\begin{itemize}
    \item \emph{自然数}:$\mathbb{N} = \{1, 2, 3, 4, \dots \}$
    \item \emph{前 $n$ 个自然数}:$[n] := \{1, 2, 3, \dots , n-1, n \}$
    \item \emph{整数}:$\mathbb{Z} = \{\dots, -3, -2, -1, 0, 1, 2, 3, \dots \}$
    \item \emph{有理数}:$\mathbb{Q} = \{\frac{m}{n} \mid m,n \in \mathbb{Z} \;\text{且}\; b \ne 0 \}$
    \item \emph{实数}:$\mathbb{R}$
    \item \emph{复数}:$\mathbb{C}$
\end{itemize}
\end{center}

我们已经多次使用过 $\mathbb{N}$ 和 $\mathbb{Z}$。有理数 $\mathbb{Q}$(选用 $\mathbb{Q}$ 是因为 $\mathbb{R}$ 已被实数占用,而有理数对应商 (Quotient) 的概念,因此取其首字母)是所有分数形式,即两个整数之比,包括正数和负数。实数则更难描述:为什么不能像列举 $\mathbb{N}$ 和 $\mathbb{Z}$ 的元素那样列出所有实数?为什么 $\mathbb{R} \ne \mathbb{Q}$?目前我们暂且假设读者对这些数集已有基本了解,但仍请思考这些问题。(我们提及复数 $\mathbb{C}$ 是因为你可能熟悉它们,但本书不会涉及复数。)

如何证明像 $\mathbb{N}$ 这样的集合存在?为什么将 $\mathbb{R}$ 视为数轴?与 $\mathbb{N}$ 相比,$\mathbb{Z}$ ``多出''多少元素?与 $\mathbb{Q}$ 相比,$\mathbb{R}$ ``多出''多少元素?我们能否回答这些问题?后续章节将严格构造自然数集 $\mathbb{N}$,并证明它是唯一满足特定性质的集合。这在讨论数学归纳法时至关重要。(还记得那一章的目标吗?)
