% !TeX root = ../../../book.tex
\subsection{示例}

让我们直接用一些集合(甚至非集合)的具体例子来说明这个定义。在数学中,通常使用大写字母表示集合,使用小写字母表示集合的元素,我们通常会遵循这个约定(但并非总是如此)。为了定义或描述一个集合,我们需要识别将集合中的元素相互关联起来的共同明确定义的属性。例如,我们可以将 $B$ 定义为 NBA (美国职业篮球联赛)\footnote{原作这里用的是``美国职业棒球大联盟'',考虑到中国读者对棒球运动的陌生,译者将其改成了 NBA。--- 译者注}中所有球队的集合。这是一个明确定义的属性吗?如果向你展示一个对象,对于``这个对象是否具有此定义属性?''这个问题,是否有明确的``是''或``否''的答案?没错,这里就是这种情况,所以这是一个表征集合的属性。(为了避免将来读者混淆,更具体地说,$B$ 指的是 2023 赛季的 NBA 球队。)用数学语言来说,我们会写成
\[B = {\text{2023 赛季所有 NBA 球队}}\]
``大括号''—— \{ 和 \} ——表示它们之间的描述构成一个集合,其中的文本是对对象及其共同定义明确的属性的描述。现在可以说 $\text{洛杉矶湖人} \in B$ 而 $\text{多伦多哈士奇} \notin B$。

汉语中数学符号 $\in$ 的常见读法是``是……的元素''或``是……的成员''或``属于……''或``在……中''。我们主要使用``是……的元素'',因为它是其中最明确的,并且适当地使用了数学术语``\textbf{元素}''。根据上下文的不同,可以适当使用其他等效说法,但不推荐。(尤其是,``在……中''可能会与其他集合关系混淆,因此我们将完全避免使用它,并鼓励你也这样做。)

我们也已经见过一些常用的数集。过往使用这些数字的过程中你知道了它们是什么,但你可能通常不会把它们看作集合。这些集合如下:
\begin{align*}
    \mathbb{N} &= \{ \text{自然数} \} = {1, 2, 3, \dots}\\
    \mathbb{Z} &= \{ \text{整数} \} = {\dots , -2, -1, 0, 1, 2, \dots}\\
    \mathbb{Q} &= \{ \text{有理数} \}\\
      &= \{ \text{能够写成}\; \frac{a}{b} \;\text{形式的数,其中}\; a, b \in Z \;\text{且}\; b \ne 0 \}\\
      \mathbb{R} &= \{ \text{实数}\}
\end{align*}
想想上面 $\mathbb{Q}$ 的第二个定义为什么合理。很快我们就会看到一种更简洁的方式来书写类似``能够写成某某形式的数,附加额外限制条件''这种形式的句子。此外,请注意,我们未能真正定义 $\mathbb{R}$,只能说它们是实数。你要如何定义什么是实数?你尝试过吗?
