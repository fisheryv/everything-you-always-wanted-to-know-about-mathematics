% !TeX root = ../../../book.tex
\subsection{示例}

我们通过具体示例(包括集合与非集合)来阐释集合的定义。数学中通常用大写字母表示集合,小写字母表示元素,我们遵循这一惯例(但非绝对)。定义集合需要明确其元素的共同属性。例如,定义 $B$ 为 NBA(美国职业篮球联赛)\footnote{原作这里用的是``MLB 美国职业棒球大联盟'',考虑到中国读者对棒球运动的陌生,译者将其改成了 NBA。—— 译者注}所有球队的集合。这是明确定义的属性吗?给定任意对象,能否明确判断其是否具有该属性?答案是肯定的,因此这构成一个合法集合。(为避免混淆,特指 2023 赛季的 NBA 球队。)数学表达为:
\[B = \{\text{2023\ 赛季所有\ NBA\ 球队}\}\]
``花括号'' \{ 和 \} 表示它们之间的描述构成一个集合。此时可表述为 $\text{洛杉矶湖人队} \in B$ 而 $\text{多伦多哈士奇} \notin B$。

符号 $\in$ 在汉语中常读作``是……的元素''、``是……的成员''、``属于……''或``在……中''。我们首选``是……的元素'',因为它足够明确,并且恰当地使用了数学术语``\textbf{元素}''。根据上下文的不同,可以适当使用其他等效说法,但不推荐。(尤其避免使用``在……中'',以防与其他集合关系混淆)。

以下为常见数集示例(你或许曾经使用过这些数字但从未以集合视角看待它们):
\begin{align*}
    \mathbb{N} &= \{ \text{自然数集} \} = \{1, 2, 3, \dots\}\\
    \mathbb{Z} &= \{ \text{整数集} \} = \{\dots , -2, -1, 0, 1, 2, \dots\}\\
    \mathbb{Q} &= \{ \text{有理数集} \} = \{ \text{能够写成\ } \frac{a}{b} \text{\ 形式的数,其中\ } a, b \in Z \text{\ 且\ } b \ne 0 \}\\
    \mathbb{R} &= \{ \text{实数集}\}
\end{align*}
思考 $\mathbb{Q}$ 的第二种定义为何合理。后文将展示更简洁的集合表示法。值得注意的是,$\mathbb{R}$ 的定义在此未具体展开——你能尝试定义实数吗?
