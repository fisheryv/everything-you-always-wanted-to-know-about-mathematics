% !TeX root = ../../../book.tex
\section{总结}

这是我们初次尝试抽象概念和结果。我们引入了\textbf{集合}的概念,并通过几个例子来激发大家对它的认识。我们讨论了\textbf{元素}和\textbf{子集}这两个关键关系,并指出区分两者是多么重要!(牢记``口袋类比''可能会对你有所帮助。)我们还讨论了一些符号,包括集合构建符。随着我们继续深入更抽象的数学领域,使用正确的形式符号将比以往任何时候都更加重要,以确保我们正确表达我们的想法。一个关键思想是\emph{幂集}的概念,它将\emph{元素}和\emph{子集}关系联系在一起。

对集合运算的讨论向我们展示了如何组合集合并创建新集合。所有这些操作都将在本书的其余部分中使用。我们还展示了如何对这些操作建立\emph{索引}。这让我们能够仅使用一些定义和符号来编写多个集合的并集。同样,这些思想将在我们的工作中频繁出现,因此我们将提供许多与这些思想相关的练习;我们鼓励你尽可能多地尝试和解决!

我们看到了与集合相关的证明技术:即\textbf{双重包含论证}。这是数学中的基本证明技术。你会看到我们频繁使用它,并且你也会发现它出现在其他课程和研究中。

还有一些讨论使我们能够触及抽象集合论中的一些深刻的思想,尽管我们无法完全深入研究它们。首先,\emph{罗素悖论}向我们表明不存在``所有集合的集合''。另一方面,我们讨论了如何用集合来正式定义自然数。在实践中,我们不会使用这个定义,并将继续依赖我们对 $\mathbb{N}$ 的直觉。但是,我们希望阅读这样的讨论是有趣的并且在某种程度上能够提供更多信息。