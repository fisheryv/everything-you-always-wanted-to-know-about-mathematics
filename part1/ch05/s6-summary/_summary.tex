% !TeX root = ../../../book.tex
\section{总结}

现在,我们终于为\textbf{归纳法}奠定了坚实的数学基础!经过长期的努力,在达成这一目标时,我们力求完整呈现其全貌。我们严谨地阐述并证明了数学归纳原理,并通过若干实际应用示例加以阐释。随后,基于数学归纳法,我们证明了更通用的\textbf{强}归纳法原理。在此过程中,我们指出任何归纳证明在某种意义上都\emph{可以视为}强归纳证明,因为后者在逻辑上蕴含前者。此外,我们在讨论 $\mathbb{N}$ 的良序原理时,进一步证明了这两种归纳原理与良序原理是\emph{逻辑等价}的。

我们还探讨了多种归纳法的变体,并为每种变体提供了一至两个示例。其中特别介绍了一种实用技术——``最小罪犯''论证,这是一种通过证明条件陈述的\emph{逆否命题}来完成归纳步骤的证明技巧。

针对上述所有归纳法变体,我们提供了相应的证明模板。未来请参考这些模板来组织你的证明,令证明结构清晰、逻辑严谨且易于理解。这不仅有助于读者理解你的论证,更能凸显证明技术背后的核心理念。这些模板绝非凭空而来,而是根植于基本原理!

接下来的练习将提供大量实践机会,帮助你熟练运用各类归纳论证。我们设计的问题比第 \ref{ch:chapter02} 章更具挑战性,因为现在你已深入理解归纳原理并具备足够能力运用它解决问题。此外,部分练习所证明的结论本身具有重要价值,我们可能在后续章节中引用这些结论!