% !TeX root = ../../book.tex
\section{导论}

我们已经讨论过数学归纳法,此时再加入本章似乎略显多余。然而,之所以这样安排有着多重目的,在之后的内容中,你将明白为什么我们需要稍微回顾一下这部分内容。

首先,我们对之前处理归纳法时使用的的非正式方式(从数学角度讲)心存芥蒂。其次,我们在第 \ref{ch:chapter02} 章中留下了一些挥之不去的问题。回想一下第 \ref{ch:chapter02} 章后面的一些例子(比如拿走游戏和汉诺塔问题),与其他例子相比(例如证明 $\sum_{k \in [n]} k=\frac{n(n+1)}{2}$),它们似乎在归纳论证中使用了``更多的假设''。事实确实如此,我们将在这里讨论这些差异。第三,还有许多有趣且有用的例子有待考察,它们本身就是重要的事实,通过研究这些例子将有助于我们发展对数学语言的理解。第四,本章最后所陈述并证明的定理(在你的帮助下!)是等价性的一个绝佳案例;具体来说,我们将展示三个定理是如何通过双条件语句相互连接的!(这将是我们在 \ref{sec:section4.9.6} 节指出的双条件语句证明策略中,``以下等价……''风格定理的首个绝佳示例。)

\subsection{目标}

鉴于我们之前已经探讨过归纳法,现在再次回到这一主题上,本章将不再赘述通常的引言部分。相反,我们将通过几个要点来概述本章的核心目标。

\textbf{学完本章后,你应该能够……}

\begin{itemize}
    \item 阐述数学归纳法的原理,并解释其证明过程是如何与自然数集联系起来的。
    \item 阐述强数学归纳法的原理,并与常规数学归纳法进行比较和对照,同时说明如何进行证明。
    \item 应用归纳法来证明各种数学命题,并特别指出在什么情况下需要使用强归纳法。
    \item 理解并阐释数学归纳法的几种变体,并指出在哪些问题中这些变体尤为有效。
    \item 阐述良序原理,并说明它与数学归纳法之间的联系。
\end{itemize}