% !TeX root = ../../../book.tex
\subsection{奇偶归纳}

让我们从一个观察开始引出这一节的内容,这将带领我们进入这种方法的第一个示例应用。考虑这样一个完全平方数序列:
\[1, 4, 9, 16, 25, 36, 49, 64, 81, 100, 121, 144, \dots\]
看看当我们将这些数除以 $8$ 时会发生什么;特别是观察余数(每种情况下分数的分子表示余数):
\[0+\frac{1}{8}, 0+\frac{4}{8}, 1+\frac{1}{8}, 2+\frac{0}{8}, 3+\frac{1}{8}, 4+\frac{2}{8}, 6+\frac{1}{8},  \dots\]
注意我们保留了像 $\frac{4}{8}$ 和 $\frac{2}{8}$ 这样未简化的分数,保持分母为 $8$,以表示余数。这些余数遵循以下模式:
\[1, 4, 1, 0, 1, 2, 1, \dots\]
看起来每隔一个余数就是 $1$。实际上,当我们将一个奇数的平方除以 $8$ 时,余数似乎总是 $1$。这很有趣!你可能会好奇这种模式是否会持续下去。探索这个想法的一个合理方式是直接尝试通过归纳法证明这个命题,并看一下结果如何。如果证明成功,那么我们就成功地发现并证明了一个事实。如果证明失败,我们可能能找出失败的原因。这是进行数学发现的一个很好的通用建议:如果你想验证某事是否为\verb|真|,不妨尝试去证明它,看看会发生什么!

\subsubsection*{示例}

在继续阅读之前,试着自己先仔细研究一下这个问题的细节。在这个过程中,你需要弄清楚如何仅对奇数进行归纳,而不是像我们之前那样对所有自然数进行归纳。我们将会展示这个命题的证明,并在之后讨论这种方法的工作原理,但你绝对应该先自己尝试解决这个问题!…… \\

\begin{example}[奇数平方和除以 $8$ 的余数]

   \textbf{声明}:设 $O$ 为奇数集,即
   \[\O = \{n \in \mathbb{N} \mid \exists m \in \mathbb{N} \cup \{0\} \centerdot n = 2m+1\}\]
   设 $P(n)$ 为命题``$n^2$ 比 $8$ 的倍数大 $1$'',则
   \[\forall n \in O \centerdot P(n)\]
\end{example}

\begin{proof}
    设 $P(n)$ 如题目定义,我们通过对 $n$ 应用归纳法证明 $\forall n \in O \centerdot P(n)$。

    \textbf{基本情况}:不难发现 $1^2=1$ 且 $1=0 \cdot 8 + 1$ (即 $1$ 比 $8$ 的倍数大 $1$) ,因此 $P(1)$ 成立。

    \textbf{归纳假设}:设 $k \in O$ 是任意固定的,假设 $P(k)$ 成立。

    \textbf{归纳步骤}:我们的目标是推导出 $P(k+2)$ 成立(这是因为 $k+2$ 是 $k$ 之后的下一个奇数)。

    因为 $k+2$ 为奇数,根据假设,我们知道 $\exists m \in \mathbb{N} \cup \{0\} \centerdot k = 2m+1$。给定此 $m$。

    根据归纳假设,我们知道 $\exists \ell \in N \centerdot l^2=8\ell+1$。给定此 $\ell$。

    利用上面的条件可得
    \begin{align*}
        (k + 2)^2 &= k^2 + 4k + 4 \\
        &= (8\ell + 1) + 4(2m + 1) + 4 \\
        &= 8\ell + 8m + 8 + 1 \\
        &= 8(\ell + m + 1) + 1 \\
    \end{align*}
    因为 $\ell, m \in \mathbb{Z}$,我们知道 $\ell+m \in \mathbb{Z}$。因此 $(k+2)^2$ 比 $8$ 的倍数大 $1$。所以 $P(k + 2)$ 成立。

    根据归纳法,$P(n)$ 对于所有 $n \in O$ 成立。
\end{proof}

\emph{后续问题}:你能否证明当偶数的平方除以 $8$ 时,其余数不为 $1$?(这会使该命题成为一个\emph{当且仅当}陈述。)你能发现这些偶数平方的余数有什么规律吗?你能证明你的观察吗?

(提示:你可能不需要使用归纳法来证明这些命题!)

\subsubsection*{方法讨论}

让我们探讨一下为什么这种方法有效。其基本原理与我们之前看到的其他归纳法完全一样。唯一的不同在于归纳步骤。由于奇数``间隔为2'',我们的目标是证明:
\[\forall k \in O \centerdot P(k) \implies P(k + 2)\]
这体现了与标准归纳法相同的思想:取命题的一个实例,并用它来推导``下一个''实例的成立。这里的不同之处在于``下一个''的定义。为了完整性,我们将给出一个描述这种方法的定理。再次强调,我们将把证明的具体细节留给你来完成。

\begin{theorem}[奇数上的归纳]\label{theorem5.3.5}
    设 $O$ 为奇数集,

    设 $P(n)$ 为变量命题。假设

    \begin{enumerate}[label=(\arabic*)]
        \item $P(1)$ 成立
        \item $\forall k \in O \centerdot P(k) \implies P(k + 2)$ 成立
    \end{enumerate}

    则 $\forall n \in O \centerdot P(n)$ 成立。
\end{theorem}

\begin{proof}
    留给读者作为 \ref{sec:section5.3.4} 节的习题 \ref{exc:exercises5.3.2}。
\end{proof}

同理,我们可以发现对偶数进行归纳同样有效。这里定理阐述了对偶数进行归纳。再次,我们把具体的证明过程留给你。

\begin{theorem}[偶数上的归纳]\label{theorem5.3.6}
    设 $E$ 为偶数集,

    设 $P(n)$ 为变量命题。假设

    \begin{enumerate}[label=(\arabic*)]
        \item $P(2)$ 成立
        \item $\forall k \in O \centerdot P(k) \implies P(k + 2)$ 成立
    \end{enumerate}

    则 $\forall n \in E \centerdot P(n)$ 成立。
\end{theorem}

\begin{proof}
    留给读者作为 \ref{sec:section5.3.4} 节的习题 \ref{exc:exercises5.3.2}。
\end{proof}


\subsubsection*{组合和修改这些方法}

假设我们有一个命题 $P(n)$,我们想要证明这个命题对于所有的自然数 $n$ 都成立。这个命题及其背后的理论可能相当复杂,使得我们无法用传统的归纳法来证明它。这可能是由于某种代数技巧的需要,或者我们找不到一种高效的证明方法,又或者是有一些深层的原因使得我们无法这样做。不管是什么原因,我们可以采用一些新型的归纳法,将证明分成几个部分,从而证明对所有 $n \in \mathbb{N}$,命题 $P(n)$ 都成立。

这些新型的方法可以被看作是``跳跃式''归纳法。例如,证明命题对每一个奇数成立的方法本质上和传统的归纳法相同,只是在归纳步骤中我们跳过了偶数。同样的方法也适用于偶数的证明(虽然我们会稍微调整一下基本情况,因为 $2$ 是第一个偶数,不是 $1$)。如果我们先用``奇数''方法证明,然后再用``偶数''方法证明,我们就可以证明这个命题对所有自然数都成立。

下面的例子正是采用了这种方法,但你会注意到它实际上是以 $3$ 为步长进行``跳跃''(而不是像奇偶归纳那样以 $2$ 为步长)。我们这里不会具体陈述和证明这些方法的定理(也不会要求你这么做)。此时,我们更依赖于对归纳法运作方式的直觉,这些定理和证明与我们之前见过的非常相似。如果你想要练习,或者想要为你的笔记和记录保留这些内容,尽管去陈述和证明我们即将使用的方法的定理吧!\\

\begin{example}[$2$ 的幂与 $7$ 的倍数]
    
    \textbf{声明}:对于所有自然数 $n \in \mathbb{N}, 2^n+1$ \emph{不是} $7$ 的倍数。
\end{example}

(这里,我们建议做一些探索性的计算,来找出当表达式 $2^n + 1$ 除以 $7$ 的余数的规律。你会发现这些余数形成了一个长度为 $3$ 的循环。真是太棒了!这实际上就是我们这里要证明的内容;只不过最初这个命题并没有以这种方式提出,因此我们需要做一些额外的工作,重新整理这个命题并设计出一个证明方法。)

\begin{proof}
    定义集合 $A_1, A_2, A_3$ 为:
    \begin{align*}
        A_1 &= \{n \in \mathbb{N} \mid \exists m \in \mathbb{N} \cup \{0\} \centerdot n = 3m + 1\} = \{1, 4, 7, 10, \dots \} \\
        A_2 &= \{n \in \mathbb{N} \mid \exists m \in \mathbb{N} \cup \{0\} \centerdot n = 3m + 2\} = \{2, 5, 8, 11, \dots \} \\
        A_3 &= \{n \in \mathbb{N} \mid \exists m \in \mathbb{N} \cup \{0\} \centerdot n = 3m \quad\:\:\:\} = \{3, 6, 9, 12, \dots \} \\
    \end{align*}
    (也就是说,这三个集合根据除以 $3$ 时的余数对 $\mathbb{N}$ 进行划分。)

    设 $P(n)$ 为命题``$2^n+1$ 不能被 $3$ 整除''。我们要通过归纳法证明 $\forall n \in \mathbb{N} \centerdot P(n)$ 成立。

    定义命题 $Q(n), R(n), S(n)$ 如下:
    \begin{align*}
        Q(n) \text{ 为 } \exists \ell \in \mathbb{N} \cup \{0\} \centerdot 2^n + 1 = 7\ell + 3 \\
        R(n) \text{ 为 } \exists \ell \in \mathbb{N} \cup \{0\} \centerdot 2^n + 1 = 7\ell + 5 \\
        S(n) \text{ 为 } \exists \ell \in \mathbb{N} \cup \{0\} \centerdot 2^n + 1 = 7\ell + 2 \\
    \end{align*}
    不难发现
    \[\forall n \in \mathbb{N} \centerdot \big(Q(n) \lor R(n) \lor S(n)\big) \implies P(n)\]
    这是因为 $7$ 的倍数加 $3$ 不是 $7$ 的倍数,$7$ 的倍数加 $5$ 和加 $2$ 也不是。\\
    
    首先,我们通过对 $n$ 应用归纳法来证明 $\forall n \in A_1 \centerdot Q(n)$ 成立。

    \textbf{基本情况}:不难发现 $2^1+1=3 = 0 \cdot 7 + 3$,因此 $Q(1)$ 成立。

    \textbf{归纳假设}:设 $k \in A_1$ 是任意固定的,假设 $Q(k)$ 成立。

    \textbf{归纳步骤}:我们的目标是推导出 $Q(k+3)$ 成立。

    因为 $k \in A_1$,我们知道 $\exists m \in \mathbb{N} \centerdot k = 3m + 1$。给定这样的 $m$。

    根据归纳假设,我们有 $\exists \ell \in \mathbb{N} \centerdot 2^k + 1 = 7\ell + 3$。给定这样的 $\ell$,这意味着 $2^k = 7\ell + 2$。

    我们能够推导出
    \[2^{k+3} = 2^3 \cdot 2^k = 8 \cdot (7\ell + 2) = 56\ell + 16\]
    因此
    \[2^{k+3} + 1 = 56\ell + 17 = 7(8\ell) + 14 + 3 = 7(8\ell + 2) + 3\]
    所以 $Q(k + 3)$ 也成立。因此 $\forall n \in A_1 \centerdot Q(n)$。\\

    接着,我们通过对 $n$ 应用归纳法来证明 $\forall n \in A_2 \centerdot R(n)$ 成立。

    \textbf{基本情况}:不难发现 $2^2+1=5 = 0 \cdot 7 + 5$,因此 $R(2)$ 成立。

    \textbf{归纳假设}:设 $k \in A_2$ 是任意固定的,假设 $R(k)$ 成立。

    \textbf{归纳步骤}:我们的目标是推导出 $R(k+3)$ 成立。

    根据归纳假设,我们有 $\exists \ell \in \mathbb{N} \centerdot 2^k + 1 = 7\ell + 5$。给定这样的 $\ell$,这意味着 $2^k = 7\ell + 4$。

    我们能够推导出
    \[2^{k+3} = 2^3 \cdot 2^k = 8 \cdot (7\ell + 4) = 56\ell + 32\]
    因此
    \[2^{k+3} + 1 = 56\ell + 33 = 7(8\ell) + 28 + 5 = 7(8\ell + 4) + 5\]
    所以 $R(k + 3)$ 也成立。因此 $\forall n \in A_2 \centerdot R(n)$。\\

    最后,我们通过对 $n$ 应用归纳法来证明 $\forall n \in A_3 \centerdot S(n)$ 成立。

    \textbf{基本情况}:不难发现 $2^3+1=9 = 1 \cdot 7 + 2$,因此 $S(3)$ 成立。

    \textbf{归纳假设}:设 $k \in A_3$ 是任意固定的,假设 $S(k)$ 成立。

    \textbf{归纳步骤}:我们的目标是推导出 $S(k+3)$ 成立。

    根据归纳假设,我们有 $\exists \ell \in \mathbb{N} \centerdot 2^k + 1 = 7\ell + 2$。给定这样的 $\ell$,这意味着 $2^k = 7\ell + 1$。

    我们能够推导出
    \[2^{k+3} = 2^3 \cdot 2^k = 8 \cdot (7\ell + 1) = 56\ell + 8\]
    因此
    \[2^{k+3} + 1 = 56\ell + 9 = 7(8\ell) + 7 + 2 = 7(8\ell + 1) + 2\]
    所以 $S(k + 3)$ 也成立。因此 $\forall n \in A_3 \centerdot S(n)$。\\

    综上,我们证明了对于每个自然数,$Q(n)$ 或 $R(n)$ 或 $S(n)$ 三者必有一个成立(取决于数字除以 $3$ 的余数)。因此,每个自然数都具有 $2^n + 1$ 不是 $7$ 的倍数的性质。
\end{proof}

实际上,我们证明中得出的结论比声明中提出的\emph{更强}。我们不仅证明了没有一个形如 $2^n + 1$ 的数是 $7$ 的倍数,还准确解释了这些数为何不是 $7$ 的倍数。

在本节的练习中,我们设计了一些练习,通过识别``跳跃''和声明来引导你进行类似的证明。在 \ref{sec:section5.7} 本章的练习中,我们还包含了一些可能需要这种论证的问题(但我们不会像这里那样告诉你论证的整体结构)。

值得注意的是,只要你想进行的``跳跃''遵循某种易于识别的模式,你可以很容易地将这些方法应用到任何情况中。在前面的例子中,我们进行了步长为 $3$ 的跳跃,因此将所有自然数分成三个集合,并在这些集合中跳跃。基本上,这依赖于我们有一个``公式''来获取命题的``下一个''实例:从 $P(k)$ 开始,并尝试推导 $P(k + 3)$。你可以设想进行步长为 $4$ 或 $10$ 的跳跃,甚至进行数值加倍的跳跃;也就是说,你可以证明某个命题 $P(n)$ 对于每个 $2$ 的 $n$ 次幂成立,即,
\[P(1) \text{ 成立,且 } \forall n \in \mathbb{N} \centerdot P(n) \implies P(2^n)\]

再次强调,所有这些都依赖于某种``公式''或``规则''告诉我们\emph{下一个}要考虑的实例是什么。因此,\emph{我们无法在所有质数集合上进行归纳}。如果你试图证明某个事实对每个质数成立,不要指望使用归纳法!你必须有某种``规则''给出,``如果 $k$ 是一个质数,那么下一个质数是……''。如果你知道这样的规则,数学界会非常愿意聆听你的见解!这将回答许多关于质数的未解之谜,并且你将成为历史上最著名的数学家。没开玩笑!
