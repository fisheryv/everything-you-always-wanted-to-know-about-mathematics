% !TeX root = ../../../book.tex
\subsection{奇偶归纳}

让我们从一个观察开始,引入这一节的主题,并进入这种方法的第一个应用示例。考虑以下完全平方数序列:
\[1, 4, 9, 16, 25, 36, 49, 64, 81, 100, 121, 144, \dots\]
观察当这些数除以 $8$ 时的余数;在以下分数中,分子表示余数:
\[0+\frac{1}{8}, 0+\frac{4}{8}, 1+\frac{1}{8}, 2+\frac{0}{8}, 3+\frac{1}{8}, 4+\frac{2}{8}, 6+\frac{1}{8},  \dots\]
注意我们保留了未简化的分数,如 $\frac{4}{8}$ 和 $\frac{2}{8}$,以保持分母为 $8$,从而明确余数。这些余数遵循以下模式:
\[1, 4, 1, 0, 1, 2, 1, \dots\]
似乎每隔一个余数就是 $1$。实际上,当奇数的平方除以 $8$ 时,余数总是 $1$。这很有趣!你可能会好奇这种模式是否永远成立。探索这个想法的合理方式是直接通过归纳法证明该命题,并观察结果。如果证明成功,我们就证明了这一事实;如果失败,我们或许可以找出失败的原因。这是数学发现的一个实用建议:如果你想验证某事是否为\verb|真|,不妨尝试证明它,并观察结果!

\subsubsection*{示例}

在继续阅读之前,建议你自己先仔细研究一下这个问题的细节。你需要弄清楚如何仅对奇数进行归纳,而不是对所有自然数归纳。我们将展示这个命题的证明,并在之后讨论这种方法的工作原理,但你应当先自己尝试解决这个问题!……

\begin{example}[奇数平方除以 $8$ 的余数]

   \textbf{声明}:设 $O$ 为奇数集,即
   \[O = \{n \in \mathbb{N} \mid \exists m \in \mathbb{N} \cup \{0\} \centerdot n = 2m+1\}\]

   设 $P(n)$ 为命题``$n^2$ 比 $8$ 的倍数大 $1$'',则
   \[\forall n \in O \centerdot P(n)\]

    \begin{proof}
        设 $P(n)$ 如上定义,我们通过对 $n$ 应用归纳法证明 $\forall n \in O \centerdot P(n)$。

        \textbf{基本情况}:不难发现 $1^2=1$ 且 $1=0 \cdot 8 + 1$ (即 $1$ 比 $8$ 的倍数大 $1$) ,因此 $P(1)$ 成立。

        \textbf{归纳假设}:设 $k \in O$ 为任意固定奇数,假设 $P(k)$ 成立。

        \textbf{归纳步骤}:我们的目标是推导出 $P(k+2)$ 成立(这是因为 $k+2$ 是 $k$ 之后的下一个奇数)。

        因为 $k+2$ 为奇数,根据假设,我们知道 $\exists m \in \mathbb{N} \cup \{0\} \centerdot k = 2m+1$。给定此 $m$。

        根据归纳假设,我们知道 $\exists \ell \in N \centerdot l^2=8\ell+1$。给定此 $\ell$。

        利用上述条件可得
        \begin{align*}
            (k + 2)^2 &= k^2 + 4k + 4 \\
            &= (8\ell + 1) + 4(2m + 1) + 4 \\
            &= 8\ell + 8m + 8 + 1 \\
            &= 8(\ell + m + 1) + 1 
        \end{align*}

        因为 $\ell, m \in \mathbb{Z}$,可得 $\ell+m \in \mathbb{Z}$。因此 $(k+2)^2$ 比 $8$ 的倍数大 $1$。故 $P(k + 2)$ 成立。

        根据归纳法,$P(n)$ 对于所有 $n \in O$ 成立。
    \end{proof}
\end{example}

\emph{后续问题}:你能否证明当偶数的平方除以 $8$ 时,其余数不为 $1$?(这将使该命题成为一个\emph{当且仅当}型命题。)你能发现这些偶数平方的余数规律吗?能否证明你的观察?

(提示:你可能不需要使用归纳法来证明这些命题!)

\subsubsection*{方法讨论}

让我们探讨一下为什么这种方法有效。其基本原理与我们之前看到的其他归纳法完全一致,唯一的不同在于归纳步骤。由于奇数``间隔为 2'',我们的目标是证明:
\[\forall k \in O \centerdot P(k) \implies P(k + 2)\]
这体现了与标准归纳法相同的核心思想:利用命题的一个实例推导``下一个''实例成立。区别仅仅在于``下一个''的定义。为严谨起见,我们给出描述这种方法的定理。证明过程留给你来完成。

\begin{theorem}[奇数上的归纳]\label{theorem5.3.5}
    设 $O$ 为奇数集,$P(n)$ 为变量命题。假设

    \begin{enumerate}[label=(\arabic*)]
        \item $P(1)$ 成立
        \item $\forall k \in O \centerdot P(k) \implies P(k + 2)$ 成立
    \end{enumerate}

    则 $\forall n \in O \centerdot P(n)$ 成立。
\end{theorem}

\begin{proof}
    留给读者作为 \ref{sec:section5.3.4} 节的习题 \ref{exc:exercises5.3.2}。
\end{proof}

同理,对偶数进行归纳同样有效。以下定理阐述偶数归纳法,证明过程仍留给你来完成。

\begin{theorem}[偶数上的归纳]\label{theorem5.3.6}
    设 $E$ 为偶数集,$P(n)$ 为变量命题。假设

    \begin{enumerate}[label=(\arabic*)]
        \item $P(2)$ 成立
        \item $\forall k \in O \centerdot P(k) \implies P(k + 2)$ 成立
    \end{enumerate}

    则 $\forall n \in E \centerdot P(n)$ 成立。
\end{theorem}

\begin{proof}
    留给读者作为 \ref{sec:section5.3.4} 节的习题 \ref{exc:exercises5.3.2}。
\end{proof}

\subsubsection*{组合和修改这些方法}

假设我们有一个命题 $P(n)$,并希望证明它对所有自然数 $n$ 成立。该命题及其理论背景可能较为复杂,导致无法用传统归纳法直接证明。这可能源于需要特定代数技巧、缺乏高效证明方法,或存在某些深层原因。无论何种情况,我们可以采用新型归纳法,将证明分为若干部分,从而验证 $P(n)$ 对所有 $n \in \mathbb{N}$ 成立。

这些新型方法可视为``跳跃式''归纳法。例如,证明命题对每个奇数成立时,其本质与传统归纳法相同,仅在归纳步骤中跳过偶数。类似方法也适用于偶数(需调整基本情况,因为第一个偶数是 $2$ 而非 $1$)。若先用``奇数''方法证明,再用``偶数''方法证明,即可覆盖所有自然数。

以下例子正是采用此思路,但你会注意到其跳跃步长为 $3$(而非奇偶归纳法中的 $2$)。此处我们暂不具体陈述或证明相关定理,而是基于对归纳法工作原理的直观理解推进。这些定理和证明与先前内容高度相似。若你希望练习或完善笔记,请自行陈述并证明即将使用的方法!

\begin{example}[$2$ 的幂与 $7$ 的倍数]
    
    \textbf{声明}:对于所有自然数 $n \in \mathbb{N}, 2^n+1$ \emph{不是} $7$ 的倍数。

    (建议先通过探索性计算观察 $2^n + 1$ 除以 $7$ 的余数规律。你会发现余数呈现长度为 $3$ 的循环。这正是本证明的核心内容,但需重新整理命题并设计证明方法。)

    \begin{proof}
        定义集合 $A_1, A_2, A_3$ 为:
        \begin{align*}
            A_1 &= \{n \in \mathbb{N} \mid \exists m \in \mathbb{N} \cup \{0\} \centerdot n = 3m + 1\} = \{1, 4, 7, 10, \dots \} \\
            A_2 &= \{n \in \mathbb{N} \mid \exists m \in \mathbb{N} \cup \{0\} \centerdot n = 3m + 2\} = \{2, 5, 8, 11, \dots \} \\
            A_3 &= \{n \in \mathbb{N} \mid \exists m \in \mathbb{N} \cup \{0\} \centerdot n = 3m \quad\:\:\:\} = \{3, 6, 9, 12, \dots \} 
        \end{align*}

        (也就是说,这三个集合按除以 $3$ 的余数对 $\mathbb{N}$ 进行划分。)

        设 $P(n)$ 为命题``$2^n+1$ 不能被 $3$ 整除''。我们要通过归纳法证明 $\forall n \in \mathbb{N} \centerdot P(n)$ 成立。

        定义命题 $Q(n), R(n), S(n)$ 如下:
        \begin{align*}
            Q(n) \text{\ 为\ } \exists \ell \in \mathbb{N} \cup \{0\} \centerdot 2^n + 1 = 7\ell + 3 \\
            R(n) \text{\ 为\ } \exists \ell \in \mathbb{N} \cup \{0\} \centerdot 2^n + 1 = 7\ell + 5 \\
            S(n) \text{\ 为\ } \exists \ell \in \mathbb{N} \cup \{0\} \centerdot 2^n + 1 = 7\ell + 2 
        \end{align*}

        不难发现
        \[\forall n \in \mathbb{N} \centerdot \big(Q(n) \lor R(n) \lor S(n)\big) \implies P(n)\]

        这是因为 $7$ 的倍数加 $2, 3, 5$ 都不是 $7$ 的倍数。\\
        
        首先,我们通过对 $n$ 应用归纳法来证明 $\forall n \in A_1 \centerdot Q(n)$ 成立。

        \textbf{基本情况}:不难发现 $2^1+1=3 = 0 \cdot 7 + 3$,因此 $Q(1)$ 成立。

        \textbf{归纳假设}:设 $k \in A_1$ 为任意固定元素,假设 $Q(k)$ 成立。

        \textbf{归纳步骤}:我们的目标是推导出 $Q(k+3)$ 成立。

        由 $k \in A_1$ 可知 $\exists m \in \mathbb{N} \centerdot k = 3m + 1$。给定这样的 $m$。

        根据归纳假设,可得 $\exists \ell \in \mathbb{N} \centerdot 2^k + 1 = 7\ell + 3$。给定这样的 $\ell$,这意味着 $2^k = 7\ell + 2$。

        推导可得
        \[2^{k+3} = 2^3 \cdot 2^k = 8 \cdot (7\ell + 2) = 56\ell + 16\]
        \[2^{k+3} + 1 = 56\ell + 17 = 7(8\ell) + 14 + 3 = 7(8\ell + 2) + 3\]

        故 $Q(k + 3)$ 也成立。因此 $\forall n \in A_1 \centerdot Q(n)$。\\

        接着,我们通过对 $n$ 应用归纳法来证明 $\forall n \in A_2 \centerdot R(n)$ 成立。

        \textbf{基本情况}:不难发现 $2^2+1=5 = 0 \cdot 7 + 5$,因此 $R(2)$ 成立。

        \textbf{归纳假设}:设 $k \in A_2$ 为任意固定元素,假设 $R(k)$ 成立。

        \textbf{归纳步骤}:我们的目标是推导出 $R(k+3)$ 成立。

        根据归纳假设,可得 $\exists \ell \in \mathbb{N} \centerdot 2^k + 1 = 7\ell + 5$。给定这样的 $\ell$,这意味着 $2^k = 7\ell + 4$。

        推导可得
        \[2^{k+3} = 2^3 \cdot 2^k = 8 \cdot (7\ell + 4) = 56\ell + 32\]
        \[2^{k+3} + 1 = 56\ell + 33 = 7(8\ell) + 28 + 5 = 7(8\ell + 4) + 5\]

        故 $R(k + 3)$ 也成立。因此 $\forall n \in A_2 \centerdot R(n)$。\\

        最后,我们通过对 $n$ 应用归纳法来证明 $\forall n \in A_3 \centerdot S(n)$ 成立。

        \textbf{基本情况}:不难发现 $2^3+1=9 = 1 \cdot 7 + 2$,因此 $S(3)$ 成立。

        \textbf{归纳假设}:设 $k \in A_3$ 为任意固定元素,假设 $S(k)$ 成立。

        \textbf{归纳步骤}:我们的目标是推导出 $S(k+3)$ 成立。

        根据归纳假设,可得 $\exists \ell \in \mathbb{N} \centerdot 2^k + 1 = 7\ell + 2$。给定这样的 $\ell$,这意味着 $2^k = 7\ell + 1$。

        推导可得
        \[2^{k+3} = 2^3 \cdot 2^k = 8 \cdot (7\ell + 1) = 56\ell + 8\]
        \[2^{k+3} + 1 = 56\ell + 9 = 7(8\ell) + 7 + 2 = 7(8\ell + 1) + 2\]

        故 $S(k + 3)$ 也成立。因此 $\forall n \in A_3 \centerdot S(n)$。\\

        综上,我们证明了对于每个自然数,$Q(n), R(n), S(n)$ 三者中必有一个成立(取决于数字除以 $3$ 的余数)。因此,对于所有自然数 $n$, $2^n + 1$ 都不是 $7$ 的倍数。
    \end{proof}
\end{example}

实际上,我们的证明得出了比原声明\emph{更强}的结论。我们不仅证明了形如 $2^n + 1$ 的数都不是 $7$ 的倍数,还精确解释了其原因。

在本节的练习中,我们安排了一些题目,引导你通过识别``跳跃''和命题来进行类似的证明。在 \ref{sec:section5.7} 节的习题中,我们还包含了一些可能需要此类论证的问题(但不会像这里那样提示整体结构)。

值得注意的是,只要你想实施的``跳跃''遵循某种易于识别的模式,这些方法就能方便地应用于各种情形。之前的例子采用了步长为 $3$ 的跳跃,因此将自然数划分为三个集合并进行归纳。其核心在于拥有一个获取命题``下一个''实例的``公式'':从 $P(k)$ 出发,尝试推导 $P(k + 3)$。类似地,你可以设想步长为 $4$ 或为 $10$ 的归纳,甚至是指数增长的归纳——例如证明命题 $P(n)$ 对所有 $2$ 的 $n$ 次幂成立:
\[P(1) \text{\ 成立,且\ } \forall n \in \mathbb{N} \centerdot P(n) \implies P(2^n)\]

再次强调,这一切都依赖于某种指明\emph{下一个}实例的``公式''或``规则''。因此,\emph{我们无法直接在全体质数集上进行归纳}。若想证明某个命题对所有质数成立,别指望使用归纳法!除非你有某种``规则''指明``若 $k$ 是质数,则下一个质数是……''。倘若知晓这样的规则,数学界必将洗耳恭听!这将解答众多关于质数的未解之谜,使你跻身史上最著名数学家之列——绝非戏言!
