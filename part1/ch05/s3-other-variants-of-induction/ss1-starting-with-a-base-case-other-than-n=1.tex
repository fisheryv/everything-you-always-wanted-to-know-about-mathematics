% !TeX root = ../../../book.tex
\subsection{$n = 1$ 以外的基本情况}\label{sec:section5.3.1}

归纳法证明需要一个基本情况,但起点不一定非得是 $n = 1$。例如,若命题 $P(n)$ 在 $n = 1$ 和 $n = 2$ 时为\verb|真|,在 $n = 3$ 和 $n = 4$ 时为\verb|假|,而从 $n = 5$ 开始又全为\verb|真|,该如何证明?我们可以分别验证 $n = 1, 2, 3, 4$ 的情况,再用归纳法证明 $n \ge 5$ 的所有情形。这种方法之所以可行,是因为集合 $\mathbb{N} - \{1, 2, 3, 4\}$ 同样是一个\emph{归纳集}。就像多米诺骨牌一样,跳过前几个,从 $n = 5$ 推倒第一块,后续骨牌便会按预期依次倒下。

实际上,这种思路甚至可扩展到\emph{负数}。想象在数轴上向左滑动:若有一排从 $-3$ 开始编号的多米诺骨牌,从 $n = -3$ 推倒后,骨牌会以相同方式依次倒下。

这里的核心思想是:存在一排向右无限延伸且无间隙的多米诺骨牌。第一个骨牌的具体编号无关紧要——无论起始数字如何,骨牌终将依次倒下。这正是下面定理的核心理念。

\begin{theorem}[任意基本情况的归纳]\label{theorem5.3.1}
    设 $P(n)$ 为变量命题。令 $M \in \mathbb{Z}$ 为任意固定整数。

    设 $S = \{z \in \mathbb{Z} \mid z \ge M\}$。

    假设
    \begin{enumerate}[label=(\arabic*)]
        \item $P(M)$ 成立
        \item $\forall k \in S \centerdot P(k) \implies P(k + 1)$ 成立
    \end{enumerate}

    则 $\forall n \in S \centerdot P(n)$ 成立。
\end{theorem}

该定理表明:若要证明命题对所有大于等于某特定值 $M$ 的整数成立,可从 $M$ 开始应用归纳法。我们将 $M$ 称为\textbf{基本情况 (BC)},再对 $\ge M$ 的整数实施\textbf{归纳假设 (IH)}和\textbf{归纳步骤 (IS)}。其余步骤与常规归纳法完全一致。

\subsubsection*{严格证明}

为了更清晰地阐述并确保理解的完整性,我们将严格地\emph{证明}这个定理。希望之前的讨论——尤其是多米诺骨牌的比喻——能帮助你直观理解这一过程。虽然掌握证明不会立即提升你运用归纳法的能力,但我们相信,阅读并理解这一过程将助你更深入地掌握归纳法与证明技巧,同时深化对相关数学的理解。具体地,我们将使用数学归纳原理 (PMI) 来证明这种归纳法的变体!

\begin{proof}
    令 $P(n)$ 为变量命题。设 $M \in \mathbb{Z}$ 为任意固定整数。

    设 $S = \{z \in \mathbb{Z} \mid z \ge M\}$。

    假设
    \begin{enumerate}[label=(\arabic*)]
        \item $P(M)$ 成立
        \item $\forall k \in S \centerdot P(k) \implies P(k + 1)$ 成立
    \end{enumerate}

    我们的目标是证明 $\forall n \in S \centerdot P(n)$ 成立。

    定义命题 $Q(n)$ 为
    \[Q(n) \iff P(n+M-1)\]

    请注意,通过代数运算,我们可以得到如下不等式
    \[n \ge 1 \iff n+M-1 \ge M\]
    
    这意味着我们将目标转换为证明 $\forall n \in \mathbb{N} \centerdot Q(n)$ 成立。(这样做将证明 $\forall n \in S \centerdot P(n)$。)
    
    我们通过对 $n$ 采用归纳法来证明该结论。

    \textbf{基本情况}:由假设可知 $P(M)$ 成立。注意到 $n + M - 1 = M \iff n = 1$。这意味着 $Q(1)$ 成立。

    \textbf{归纳假设}:设 $k \in \mathbb{N}$ 为任意固定自然数,并假设 $Q(k)$ 成立。

    \textbf{归纳步骤}:由 $Q(k)$ 成立可知 $P(k + M - 1)$ 成立。

    由于 $k \in \mathbb{N}$,可知 $k \ge 1$。因此,$k + M - 1 \ge M$。

    因此,根据假设条件 (2),可以推导出 $P((k+M-1)+1)$ 成立,即 $P(k + M)$ 成立。

    因此 $Q(k + 1)$ 成立。

    根据数学归纳原理,我们推导出 $\forall n \in \mathbb{N} \centerdot Q(n)$ 成立。

    因此,根据 $Q(n)$ 的定义,我们得到 $\forall n \in S \centerdot P(n)$ 成立。
\end{proof}

建议尽量理解该证明的细节,但总体只需记住一个直观概念:我们只是在``移动''基本情况的起点,而归纳过程的原理保持不变。

\subsubsection*{示例}

让我们来看看这种改进证法的实际应用。事实上,接下来的例子正是引言中提及的情形:某命题在某些较小数值上成立,在另一些较小数值上不成立,但从某个特定点开始对所有数值均成立。\\

\begin{example}[比较 $2^n$ 与 $n^2$ 的大小]

    \textbf{声明}:
    \[2^n > n^2 \iff n \in \{0,1\} \cup \{z \in \mathbb{N} \mid z \ge 5\}\]

    也就是说,仅当整数 $z=0,1,5,6,7,\dots$ 时 $2^n > n^2$。

    (命题的构思过程留待读者探索。通常情况下,正如本节练习所示,此类不等式问题会直接询问``命题对哪些 $n$ 成立'',在这种情况下,需要先通过初步推理确定命题,再运用归纳法证明。)

    \begin{proof}
        设 $P(n)$ 为命题 $2^n > n^2$。

        首先,考察如下情况:
        \begin{align*}
            & 2^0 > 0^2 \iff \enspace 1>0 \enspace \qquad \text{所以\ } P(0) \text{\ 为真}\\
            & 2^1 > 1^2 \iff \enspace 2>1 \enspace \qquad \text{所以\ } P(1) \text{\ 为真}\\
            & 2^2 > 2^2 \iff \enspace 4>4 \enspace \qquad \text{所以\ } P(2) \text{\ 为假}\\
            & 2^3 > 3^2 \iff \enspace 8>9 \enspace \qquad \text{所以\ } P(3) \text{\ 为假}\\
            & 2^4 > 4^2 \iff 16>16 \qquad \text{所以\ } P(4) \text{\ 为假}
        \end{align*}

        注意,当 $z \le -1$ 时,我们有 $2^z < 1$ 且 $z^2 \ge 1$,所以 $2^z \ngtr z^2$。因此对于所有 $n \le -1, P(n)$ 为\verb|假|。

        接下来,定义 $S$ 为集合 $S = \{z \in \mathbb{N} \mid z \ge 5\}$。

        我们在 $n$ 上应用归纳法证明 $\forall n \in S \centerdot P(n)$ 成立。

        \textbf{基本情况}:不难发现 $P(5)$ 成立,因为 $2^5=32$ 且 $5^2=25$,显然 $32 > 25$。

        \textbf{归纳假设}:设 $k \in \mathbb{N}$ 为任意固定自然数,假设 $P(k)$ 成立。

        \textbf{归纳步骤}:由 $k \in S$ 可知 $k \ge 5$ 或 $k > 4$。
        
        因此 $k-1>3$, $(k-1)^2>9$,显然 $(k-1)^2>2$。

        考察如下不等式处理:
        \begin{align*}
            (k-1)^2 > 2 &\implies (k-1)^2-2>0 \\
            &\implies k^2-2k-1>0 \\
            &\implies k^2>2k+1 \\
            &\implies 2k^2>k^2+2k+1 \\
            &\implies 2k^2>(k+1)^2 
        \end{align*}

        由于已知第一个不等式成立,故可以推导出上面最后一个不等式成立。

        (注:此推导实为 \ref{sec:section4.9.9} 节习题 \ref{exc:exercises4.9.2} 的解,通过逆向操作从目标不等式回溯至已知条件。)

        由归纳假设 $P(k)$ 可知 $k^2 < 2^k$,由此可得
        \[2k^2 < 2 \cdot 2^k = 2^{k+1}\]

        应用不等式的传递性可得
        \[(k + 1)^2 < 2k^2 < 2^{k+1}\]

        因此 $P(k+1)$ 成立。

        根据数学归纳原理,$\forall n \in S \centerdot P(n)$ 成立。

        综上,我们考虑了每个 $z \in \mathbb{Z}$。$P(z)$ 在 $z \le -1$ 时为\verb|假|,在 $z = 0, 1$ 时为\verb|真|,在 $z = 2, 3, 4$ 时为\verb|假|,在 $z \ge 5$ 时为\verb|真|。上述结果共同证明了该命题。
    \end{proof}
\end{example}

此证明具有一定挑战性——命题采用``当且仅当''形式,需要验证所有整数情形,但最终我们完成了证明。

\clearpage