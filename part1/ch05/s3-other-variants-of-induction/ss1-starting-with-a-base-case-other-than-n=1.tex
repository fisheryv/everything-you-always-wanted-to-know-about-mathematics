% !TeX root = ../../../book.tex
\subsection{$n = 1$ 以外的基本情况}\label{sec:section5.3.1}

在进行归纳法证明时,我们需要一个基本情况,但这个起始点并不一定非得是 $n = 1$。比如,如果我们有一个命题 $P(n)$,当 $n = 1$ 和 $n = 2$ 时为\verb|真|,但 $n = 3$ 和 $n = 4$ 时为\verb|假|,然后从 $n = 5$ 开始又都为\verb|真|。我们该如何证明这些情况呢?我们可以分别验证 $n = 1,2,3,4$ 的情况,然后用归纳法证明 $n \ge 5$ 的所有其他情况。这种方法是可行的,因为集合 $\mathbb{N} - \{1, 2, 3, 4\}$ 同样是一个\emph{归纳集}。就像多米诺骨牌一样,我们可以跳过前几个,从 $n = 5$ 开始推倒,剩下的骨牌就会按照预期的方式依次倒下。

实际上,我们甚至可以将这种讨论扩展到\emph{负数}。想象一下,在数轴上向左滑动,我们有一排从 $-3$ 开始编号的多米诺骨牌。从 $n = -3$ 开始推倒这些骨牌,它们会以类似之前的方式依次倒下。

核心思想是,我们有一条向右无限延伸、没有间隙的多米诺骨牌队列。这条队列的第一个骨牌被标记的具体数字并不重要,这样一来,无论如何编号,这些骨牌最终都会依次倒下。这正是下一个定理所要表达的核心理念。

\begin{theorem}[任意基本情况的归纳]\label{theorem5.3.1}
    设 $P(n)$ 为变量命题。令 $M \in \mathbb{Z}$ 是任意且固定的。

    设 $S = \{z \in \mathbb{Z} \mid z \ge M\}$。

    假设
    \begin{enumerate}[label=(\arabic*)]
        \item $P(M)$ 成立
        \item $\forall k \in S \centerdot P(k) \implies P(k + 1)$ 成立
    \end{enumerate}

    则 $\forall n \in S \centerdot P(n)$ 成立。
\end{theorem}

这个定理正是我们要讨论的内容:如果我们想证明某个命题对所有大于或等于某个特定值(定理中的 $M$)的数都成立,我们可以从这个特定值开始应用归纳法。我们把这个值设为\textbf{基本情况(BC)},然后对所有大于或等于这个值的数应用\textbf{归纳假设(IH)}和\textbf{归纳步骤(IS)}。除此之外,其他部分的处理方式与常规归纳法完全相同。

\subsubsection*{严格证明}

为了更好地说明和确保理解的完整性,我们将严格地\emph{证明}这个定理。我们希望之前的讨论---特别是多米诺骨牌的比喻---能帮助你直观地理解这个过程。虽然掌握这个证明的过程不会立竿见影地提升你使用归纳法的能力,但我们相信,阅读并尝试理解这个过程将帮助你更深入地掌握归纳法以及证明技巧,也会让你对这里涉及的数学有更深的感悟。具体而言,我们将利用数学归纳原理(PMI)来证明这个归纳法的变体!

\begin{proof}
    令 $P(n)$ 为变量命题。设 $M \in \mathbb{Z}$ 是任意且固定的。

    设 $S = \{z \in \mathbb{Z} \mid z \ge M\}$。

    假设
    \begin{enumerate}[label=(\arabic*)]
        \item $P(M)$ 成立
        \item $\forall k \in S \centerdot P(k) \implies P(k + 1)$ 成立
    \end{enumerate}

    我们的目标是证明 $\forall n \in S \centerdot P(n)$ 成立。

    定义命题 $Q(n)$
    \[Q(n) \iff P(n+M-1)\]
    请注意,通过代数运算,我们可以得到如下不等式
    \[n \ge 1 \iff n+M-1 \ge M\]
    这意味着我们将目标转换为证明 $\forall n \in \mathbb{N} \centerdot Q(n)$ 成立。(这样做将证明 $\forall n \in S \centerdot P(n)$。)
    
    我们通过对 $n$ 采用归纳法来证明这一点。

    \textbf{基本情况}:根据假设,我们知道 $P(M)$ 成立。 注意 $n + M - 1 = M \iff n = 1$。这意味着 $Q(1)$ 成立。

    \textbf{归纳假设}:设 $k \in \mathbb{N}$ 是任意固定的,假设 $Q(k)$ 成立。

    \textbf{归纳步骤}:由于 $Q(k)$ 成立,我们知道 $P(k + M - 1)$ 成立。

    又由于 $k \in \mathbb{N}$,我们知道 $k \ge 1$。因此,$k + M - 1 \ge M$。

    因此,根据假设条件(2),我们可以推导出 $P((k+M-1)+1)$ 成立,即 $P(k + M)$ 成立。

    这告诉我们 $Q(k + 1)$ 成立。

    根据数学归纳原理,我们推导出 $\forall n \in \mathbb{N} \centerdot Q(n)$ 成立。

    因此,根据 $Q(n)$ 的定义,我们得到 $\forall n \in S \centerdot P(n)$ 成立。
\end{proof}

正如我们所提到的,尽量去理解这个证明的具体细节,但总体上,你只需要记住这样一个直观的概念:我们只是在``移动''基本情况的起点。归纳过程的基本原理是相同的。

\subsubsection*{示例}

让我们来看看这种改进版证法在实际中的应用。事实上,我们接下来展示的例子正是我们在介绍这个方法时提到的情况:某个命题在一些较小的数值上是成立的,对于另一些较小的数值则不成立,但从某个特定点之后,对于所有数值都是成立的。\\

\begin{example}[比较 $2^n$ 与 $n^2$ 的大小]

    \textbf{声明}:
    \[2^n > n^2 \iff n \in \{0,1\} \cup \{z \in \mathbb{N} \mid z \ge 5\}\]
    也就是说,只有整数 $z=0,1,5,6,7,\dots$ 时 $2^n > n^2$。
\end{example}

(我们将把如何构思出这样一个命题的过程留给你去探索和尝试。通常情况下,如你在本节的练习中所见,这类不等式问题可能会附带一个问题:``这个命题对于哪些 $n$ 成立?''在这种情况下,你需要先进行一些初步的推理工作来确定你的命题,然后才能开始使用归纳法进行证明。)

\begin{proof}
    设 $P(n)$ 为命题 $2^n > n^2$。

    首先,考察如下情况:
    \begin{align*}
        & 2^0 > 0^2 \iff 1>0 & \text{所以 } P(0) \text{ 为真}\\
        & 2^1 > 1^2 \iff 2>1 & \text{所以 } P(1) \text{ 为真}\\
        & 2^2 > 2^2 \iff 4>4 &  \text{所以 } P(2) \text{ 为假}\\
        & 2^3 > 3^2 \iff 8>9 &  \text{所以 } P(3) \text{ 为假}\\
        & 2^4 > 4^2 \iff 16>16 & \text{所以 } P(4) \text{ 为假}\\
    \end{align*}
    注意,当 $z \le -1$ 时,我们有 $2^z < 1$ 且 $z^2 \ge 1$,所以 $2^z \ngtr z^2$。因此对于所有 $n \le -1, P(n)$ 为\verb|假|。

    接下来,定义 $S$ 为集合 $S = \{z \in \mathbb{N} \mid z \ge 5\}$。

    我们要在 $n$ 上应用归纳法证明 $\forall n \in S \centerdot P(n)$ 成立。

    \textbf{基本情况}:不难发现 $P(5)$ 成立,因为 $2^5=32$ 且 $5^2=25$,显然 $32 > 25$。

    \textbf{归纳假设}:设 $k \in \mathbb{N}$ 是任意固定的,假设 $P(k)$ 成立。

    \textbf{归纳步骤}:因为 $k \in S$,我们知道 $k \ge 5$ 或 $k > 4$。
    
    因此 $k-1>3$ 所以 $(k-1)^2>9$,自然 $(k-1)^2>2$。

    考察如下不等式处理:
    \begin{align*}
        (k-1)^2 > 2 &\implies (k-1)^2-2>0 \\
        &\implies k^2-2k-1>0 \\
        &\implies k^2>2k+1 \\
        &\implies 2k^2>k^2+2k+1 \\
        &\implies 2k^2>(k+1)^2 \\
    \end{align*}

    因为我们知道第一个不等式成立,我们可以推导出上面最后一个不等式成立。

    (注:如果你还没有注意到,这一连串的推理其实是 \ref{sec:section4.9.9} 节习题 \ref{exc:exercises4.9.2} 的解答!为了解答这个问题,我们进行了一些初步的探索,从所需证明的不等式出发,然后``逆向操作''直至找到一个显而易见的真理。在这里的书写中,我们从那个显然的事实出发,逐步推导至期望的结论。)

    根据归纳假设 $P(k)$,我们知道 $k^2 < 2^k$,这告诉我们
    \[2k^2 < 2 \cdot 2^k = 2^{k+1}\]
    应用不等式的传递性,我们可以推出
    \[(k + 1)^2 < 2k^2 < 2^{k+1}\]
    所以 $P(k+1)$ 成立。

    根据数学归纳原理,$\forall n \in S \centerdot P(n)$ 成立。

    综上,我们考虑了每个 $z \in \mathbb{Z}$。我们观察到 $P(z)$ 在 $z \le -1$ 时为\verb|假|,在 $z = 0, 1$ 时为\verb|真|,在 $z = 2, 3, 4$ 时为\verb|假|,在 $z \ge 5$ 时为\verb|真|。上述结果共同证明了该声明。
\end{proof}

这个证明其实相当复杂。你有没有注意到,我们的命题是用``当且仅当''来表述的,因此我们在证明过程中必须考虑所有整数?这确实很有挑战性,但我们成功了!
