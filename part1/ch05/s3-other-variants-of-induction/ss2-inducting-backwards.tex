% !TeX root = ../../../book.tex
\subsection{反向归纳}

这种归纳法的变体特别适用于当命题 $P(n)$ 对于某个特定值之前的所有 $n$ 都成立的情况。如果用多米诺骨牌做比喻,这就像是想象我们的无限长的多米诺骨牌向左延伸,而不是向右。正如前一节所讨论的,多米诺骨牌的编号方式其实并不重要。现在,我们还可以看到它们向哪个方向倒下也无关紧要;它们都将遵循同样的原理!下面的定理总结了这一观察。

\begin{theorem}[反向归纳]\label{theorem5.3.3}
    设 $P(n)$ 为变量命题。令 $M \in \mathbb{Z}$ 为任意固定的。

    设 $S = {z \in \mathbb{Z} \mid z \le M}$。

    假设
    \begin{enumerate}[label=(\arabic*)]
        \item $P(M)$ 成立
        \item $\forall k \in S \centerdot P(k) \implies P(k - 1)$ 成立
    \end{enumerate}

    则 $\forall n \in S \centerdot P(n)$ 成立。
\end{theorem}

注意该定理与定理 \ref{theorem5.3.1} 的区别。

\subsubsection*{严格证明}

在我们目前的进展中,我们已经有足够的信心让你来证明一些重要的定理了。具体来说,我们希望你证明上面提到的这个改进版的数学归纳原理,即定理 \ref{theorem5.3.3}!我们希望你亲自动手处理这些细节,而不是仅仅看我们为你提供的演示,从长远来看这对你更加有益。此外,我们想到的这个证明的细节与我们之前给你展示的定理 \ref{theorem5.3.1}(在第 \ref{sec:section5.3.1} 节)的证明细节非常相似。

在数学书中,将证明留作``读者练习''是非常常见的做法。我们这样做是为了帮助你逐渐适应这种现象!$\smiley{}$

\begin{proof}
    留给读者作为 \ref{sec:section5.3.4} 节的习题 \ref{exc:exercises5.3.1}。
\end{proof}

我们不会展示这种方法的实操示例,因为我们认为它与我们已经见过的标准归纳法没有区别。实际上,如果你仔细分析了上面的证明细节,你甚至可以看出如何通过稍作修改我们已经见过的一些例子来为这一节构思一个例子!(比如我们反转一个不等式……)
