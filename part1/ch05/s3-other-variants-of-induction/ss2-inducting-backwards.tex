% !TeX root = ../../../book.tex
\subsection{反向归纳}

这种归纳法的变体特别适用于命题 $P(n)$ 在某个特定值之前的所有 $n$ 都成立的情况。用多米诺骨牌来比喻,这就像想象无限长的多米诺骨牌向左延伸,而非向右延伸。正如前一节所讨论的,多米诺骨牌的编号方式无关紧要。现在,我们还将看到它们倒下的方向也无关紧要;它们都遵循相同的原理!下面的定理总结了这一观察。

\begin{theorem}[反向归纳]\label{theorem5.3.3}
    设 $P(n)$ 为变量命题。令 $M \in \mathbb{Z}$ 为任意固定整数。

    设 $S = {z \in \mathbb{Z} \mid z \le M}$。

    假设
    \begin{enumerate}[label=(\arabic*)]
        \item $P(M)$ 成立
        \item $\forall k \in S \centerdot P(k) \implies P(k - 1)$ 成立
    \end{enumerate}

    则 $\forall n \in S \centerdot P(n)$ 成立。
\end{theorem}

请注意该定理与定理 \ref{theorem5.3.1} 的区别。

\subsubsection*{严格证明}

在目前的进展中,我们有足够的信心让你证明一些重要定理。具体来说,我们希望你自己证明上述改进版的数学归纳原理(定理 \ref{theorem5.3.3})。亲自处理这些细节而非单纯观看演示,对你的长期发展更有益。此外,该证明的细节与我们之前展示的定理 \ref{theorem5.3.1}(见第 \ref{sec:section5.3.1} 节)非常相似。

数学书中常将证明留作``读者练习''。我们采纳这一做法是为了帮助你逐渐熟悉此类形式!$\smiley{}$

\begin{proof}
    留给读者作为 \ref{sec:section5.3.4} 节的习题 \ref{exc:exercises5.3.1}。
\end{proof}

我们不会展示这种方法的实际示例,因为它与我们见过的标准归纳法没有本质区别。事实上,如果你仔细分析上述证明细节,可以看出如何通过稍作修改已有例子来为本节构思示例!(例如反转一个不等式……)
