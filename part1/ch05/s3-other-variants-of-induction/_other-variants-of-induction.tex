% !TeX root = ../../../book.tex
\section{归纳法的其他变体}

现在我们已经非常熟练地掌握了归纳法,并通过多个例子感受到了其用途,接下来我们将介绍这种方法的两种变体。首先要明确的是,使用归纳法来证明某个命题对所有自然数 $n \in \mathbb{N}$ 成立并没有什么``特别''的地方。当然,自然数集合 $\mathbb{N}$ 本身是非常特殊的!这里我们想表达的是,归纳法同样可以用来证明某个命题对于某个不同类型集合 $S$ 中的每一个元素 $n$ 都成立。在后续的讨论和示例中,我们将详细介绍这些集合。

% !TeX root = ../../../book.tex
\subsection{$n = 1$ 以外的基本情况}\label{sec:section5.3.1}

归纳法证明需要一个基本情况,但起点不一定非得是 $n = 1$。例如,若命题 $P(n)$ 在 $n = 1$ 和 $n = 2$ 时为\verb|真|,在 $n = 3$ 和 $n = 4$ 时为\verb|假|,而从 $n = 5$ 开始又全为\verb|真|,该如何证明?我们可以分别验证 $n = 1, 2, 3, 4$ 的情况,再用归纳法证明 $n \ge 5$ 的所有情形。这种方法之所以可行,是因为集合 $\mathbb{N} - \{1, 2, 3, 4\}$ 同样是一个\emph{归纳集}。就像多米诺骨牌一样,跳过前几个,从 $n = 5$ 推倒第一块,后续骨牌便会按预期依次倒下。

实际上,这种思路甚至可扩展到\emph{负数}。想象在数轴上向左滑动:若有一排从 $-3$ 开始编号的多米诺骨牌,从 $n = -3$ 推倒后,骨牌会以相同方式依次倒下。

这里的核心思想是:存在一排向右无限延伸且无间隙的多米诺骨牌。第一个骨牌的具体编号无关紧要——无论起始数字如何,骨牌终将依次倒下。这正是下面定理的核心理念。

\begin{theorem}[任意基本情况的归纳]\label{theorem5.3.1}
    设 $P(n)$ 为变量命题。令 $M \in \mathbb{Z}$ 为任意固定整数。

    设 $S = \{z \in \mathbb{Z} \mid z \ge M\}$。

    假设
    \begin{enumerate}[label=(\arabic*)]
        \item $P(M)$ 成立
        \item $\forall k \in S \centerdot P(k) \implies P(k + 1)$ 成立
    \end{enumerate}

    则 $\forall n \in S \centerdot P(n)$ 成立。
\end{theorem}

该定理表明:若要证明命题对所有大于等于某特定值 $M$ 的整数成立,可从 $M$ 开始应用归纳法。我们将 $M$ 称为\textbf{基本情况 (BC)},再对 $\ge M$ 的整数实施\textbf{归纳假设 (IH)}和\textbf{归纳步骤 (IS)}。其余步骤与常规归纳法完全一致。

\subsubsection*{严格证明}

为了更清晰地阐述并确保理解的完整性,我们将严格地\emph{证明}这个定理。希望之前的讨论——尤其是多米诺骨牌的比喻——能帮助你直观理解这一过程。虽然掌握证明不会立即提升你运用归纳法的能力,但我们相信,阅读并理解这一过程将助你更深入地掌握归纳法与证明技巧,同时深化对相关数学的理解。具体地,我们将使用数学归纳原理 (PMI) 来证明这种归纳法的变体!

\begin{proof}
    令 $P(n)$ 为变量命题。设 $M \in \mathbb{Z}$ 为任意固定整数。

    设 $S = \{z \in \mathbb{Z} \mid z \ge M\}$。

    假设
    \begin{enumerate}[label=(\arabic*)]
        \item $P(M)$ 成立
        \item $\forall k \in S \centerdot P(k) \implies P(k + 1)$ 成立
    \end{enumerate}

    我们的目标是证明 $\forall n \in S \centerdot P(n)$ 成立。

    定义命题 $Q(n)$ 为
    \[Q(n) \iff P(n+M-1)\]

    请注意,通过代数运算,我们可以得到如下不等式
    \[n \ge 1 \iff n+M-1 \ge M\]
    
    这意味着我们将目标转换为证明 $\forall n \in \mathbb{N} \centerdot Q(n)$ 成立。(这样做将证明 $\forall n \in S \centerdot P(n)$。)
    
    我们通过对 $n$ 采用归纳法来证明该结论。

    \textbf{基本情况}:由假设可知 $P(M)$ 成立。注意到 $n + M - 1 = M \iff n = 1$。这意味着 $Q(1)$ 成立。

    \textbf{归纳假设}:设 $k \in \mathbb{N}$ 为任意固定自然数,并假设 $Q(k)$ 成立。

    \textbf{归纳步骤}:由 $Q(k)$ 成立可知 $P(k + M - 1)$ 成立。

    由于 $k \in \mathbb{N}$,可知 $k \ge 1$。因此,$k + M - 1 \ge M$。

    因此,根据假设条件 (2),可以推导出 $P((k+M-1)+1)$ 成立,即 $P(k + M)$ 成立。

    因此 $Q(k + 1)$ 成立。

    根据数学归纳原理,我们推导出 $\forall n \in \mathbb{N} \centerdot Q(n)$ 成立。

    因此,根据 $Q(n)$ 的定义,我们得到 $\forall n \in S \centerdot P(n)$ 成立。
\end{proof}

建议尽量理解该证明的细节,但总体只需记住一个直观概念:我们只是在``移动''基本情况的起点,而归纳过程的原理保持不变。

\subsubsection*{示例}

让我们来看看这种改进证法的实际应用。事实上,接下来的例子正是引言中提及的情形:某命题在某些较小数值上成立,在另一些较小数值上不成立,但从某个特定点开始对所有数值均成立。\\

\begin{example}[比较 $2^n$ 与 $n^2$ 的大小]

    \textbf{声明}:
    \[2^n > n^2 \iff n \in \{0,1\} \cup \{z \in \mathbb{N} \mid z \ge 5\}\]

    也就是说,仅当整数 $z=0,1,5,6,7,\dots$ 时 $2^n > n^2$。

    (命题的构思过程留待读者探索。通常情况下,正如本节练习所示,此类不等式问题会直接询问``命题对哪些 $n$ 成立'',在这种情况下,需要先通过初步推理确定命题,再运用归纳法证明。)

    \begin{proof}
        设 $P(n)$ 为命题 $2^n > n^2$。

        首先,考察如下情况:
        \begin{align*}
            & 2^0 > 0^2 \iff \enspace 1>0 \enspace \qquad \text{所以\ } P(0) \text{\ 为真}\\
            & 2^1 > 1^2 \iff \enspace 2>1 \enspace \qquad \text{所以\ } P(1) \text{\ 为真}\\
            & 2^2 > 2^2 \iff \enspace 4>4 \enspace \qquad \text{所以\ } P(2) \text{\ 为假}\\
            & 2^3 > 3^2 \iff \enspace 8>9 \enspace \qquad \text{所以\ } P(3) \text{\ 为假}\\
            & 2^4 > 4^2 \iff 16>16 \qquad \text{所以\ } P(4) \text{\ 为假}
        \end{align*}

        注意,当 $z \le -1$ 时,我们有 $2^z < 1$ 且 $z^2 \ge 1$,所以 $2^z \ngtr z^2$。因此对于所有 $n \le -1, P(n)$ 为\verb|假|。

        接下来,定义 $S$ 为集合 $S = \{z \in \mathbb{N} \mid z \ge 5\}$。

        我们在 $n$ 上应用归纳法证明 $\forall n \in S \centerdot P(n)$ 成立。

        \textbf{基本情况}:不难发现 $P(5)$ 成立,因为 $2^5=32$ 且 $5^2=25$,显然 $32 > 25$。

        \textbf{归纳假设}:设 $k \in \mathbb{N}$ 为任意固定自然数,假设 $P(k)$ 成立。

        \textbf{归纳步骤}:由 $k \in S$ 可知 $k \ge 5$ 或 $k > 4$。
        
        因此 $k-1>3$, $(k-1)^2>9$,显然 $(k-1)^2>2$。

        考察如下不等式处理:
        \begin{align*}
            (k-1)^2 > 2 &\implies (k-1)^2-2>0 \\
            &\implies k^2-2k-1>0 \\
            &\implies k^2>2k+1 \\
            &\implies 2k^2>k^2+2k+1 \\
            &\implies 2k^2>(k+1)^2 
        \end{align*}

        由于已知第一个不等式成立,故可以推导出上面最后一个不等式成立。

        (注:此推导实为 \ref{sec:section4.9.9} 节习题 \ref{exc:exercises4.9.2} 的解,通过逆向操作从目标不等式回溯至已知条件。)

        由归纳假设 $P(k)$ 可知 $k^2 < 2^k$,由此可得
        \[2k^2 < 2 \cdot 2^k = 2^{k+1}\]

        应用不等式的传递性可得
        \[(k + 1)^2 < 2k^2 < 2^{k+1}\]

        因此 $P(k+1)$ 成立。

        根据数学归纳原理,$\forall n \in S \centerdot P(n)$ 成立。

        综上,我们考虑了每个 $z \in \mathbb{Z}$。$P(z)$ 在 $z \le -1$ 时为\verb|假|,在 $z = 0, 1$ 时为\verb|真|,在 $z = 2, 3, 4$ 时为\verb|假|,在 $z \ge 5$ 时为\verb|真|。上述结果共同证明了该命题。
    \end{proof}
\end{example}

此证明具有一定挑战性——命题采用``当且仅当''形式,需要验证所有整数情形,但最终我们完成了证明。

\clearpage

% !TeX root = ../../../book.tex
\subsection{反向归纳}

这种归纳法的变体特别适用于命题 $P(n)$ 在某个特定值之前的所有 $n$ 都成立的情况。用多米诺骨牌来比喻,这就像想象无限长的多米诺骨牌向左延伸,而非向右延伸。正如前一节所讨论的,多米诺骨牌的编号方式无关紧要。现在,我们还将看到它们倒下的方向也无关紧要;它们都遵循相同的原理!下面的定理总结了这一观察。

\begin{theorem}[反向归纳]\label{theorem5.3.3}
    设 $P(n)$ 为变量命题。令 $M \in \mathbb{Z}$ 为任意固定整数。

    设 $S = {z \in \mathbb{Z} \mid z \le M}$。

    假设
    \begin{enumerate}[label=(\arabic*)]
        \item $P(M)$ 成立
        \item $\forall k \in S \centerdot P(k) \implies P(k - 1)$ 成立
    \end{enumerate}

    则 $\forall n \in S \centerdot P(n)$ 成立。
\end{theorem}

请注意该定理与定理 \ref{theorem5.3.1} 的区别。

\subsubsection*{严格证明}

在目前的进展中,我们有足够的信心让你证明一些重要定理。具体来说,我们希望你自己证明上述改进版的数学归纳原理(定理 \ref{theorem5.3.3})。亲自处理这些细节而非单纯观看演示,对你的长期发展更有益。此外,该证明的细节与我们之前展示的定理 \ref{theorem5.3.1}(见第 \ref{sec:section5.3.1} 节)非常相似。

数学书中常将证明留作``读者练习''。我们采纳这一做法是为了帮助你逐渐熟悉此类形式!$\smiley{}$

\begin{proof}
    留给读者作为 \ref{sec:section5.3.4} 节的习题 \ref{exc:exercises5.3.1}。
\end{proof}

我们不会展示这种方法的实际示例,因为它与我们见过的标准归纳法没有本质区别。事实上,如果你仔细分析上述证明细节,可以看出如何通过稍作修改已有例子来为本节构思示例!(例如反转一个不等式……)


% !TeX root = ../../../book.tex
\subsection{奇偶归纳}

让我们从一个观察开始,引入这一节的主题,并进入这种方法的第一个应用示例。考虑以下完全平方数序列:
\[1, 4, 9, 16, 25, 36, 49, 64, 81, 100, 121, 144, \dots\]
观察当这些数除以 $8$ 时的余数;在以下分数中,分子表示余数:
\[0+\frac{1}{8}, 0+\frac{4}{8}, 1+\frac{1}{8}, 2+\frac{0}{8}, 3+\frac{1}{8}, 4+\frac{2}{8}, 6+\frac{1}{8},  \dots\]
注意我们保留了未简化的分数,如 $\frac{4}{8}$ 和 $\frac{2}{8}$,以保持分母为 $8$,从而明确余数。这些余数遵循以下模式:
\[1, 4, 1, 0, 1, 2, 1, \dots\]
似乎每隔一个余数就是 $1$。实际上,当奇数的平方除以 $8$ 时,余数总是 $1$。这很有趣!你可能会好奇这种模式是否永远成立。探索这个想法的合理方式是直接通过归纳法证明该命题,并观察结果。如果证明成功,我们就证明了这一事实;如果失败,我们或许可以找出失败的原因。这是数学发现的一个实用建议:如果你想验证某事是否为\verb|真|,不妨尝试证明它,并观察结果!

\subsubsection*{示例}

在继续阅读之前,建议你自己先仔细研究一下这个问题的细节。你需要弄清楚如何仅对奇数进行归纳,而不是对所有自然数归纳。我们将展示这个命题的证明,并在之后讨论这种方法的工作原理,但你应当先自己尝试解决这个问题!……

\begin{example}[奇数平方除以 $8$ 的余数]

   \textbf{声明}:设 $O$ 为奇数集,即
   \[O = \{n \in \mathbb{N} \mid \exists m \in \mathbb{N} \cup \{0\} \centerdot n = 2m+1\}\]

   设 $P(n)$ 为命题``$n^2$ 比 $8$ 的倍数大 $1$'',则
   \[\forall n \in O \centerdot P(n)\]

    \begin{proof}
        设 $P(n)$ 如上定义,我们通过对 $n$ 应用归纳法证明 $\forall n \in O \centerdot P(n)$。

        \textbf{基本情况}:不难发现 $1^2=1$ 且 $1=0 \cdot 8 + 1$ (即 $1$ 比 $8$ 的倍数大 $1$) ,因此 $P(1)$ 成立。

        \textbf{归纳假设}:设 $k \in O$ 为任意固定奇数,假设 $P(k)$ 成立。

        \textbf{归纳步骤}:我们的目标是推导出 $P(k+2)$ 成立(这是因为 $k+2$ 是 $k$ 之后的下一个奇数)。

        因为 $k+2$ 为奇数,根据假设,我们知道 $\exists m \in \mathbb{N} \cup \{0\} \centerdot k = 2m+1$。给定此 $m$。

        根据归纳假设,我们知道 $\exists \ell \in N \centerdot l^2=8\ell+1$。给定此 $\ell$。

        利用上述条件可得
        \begin{align*}
            (k + 2)^2 &= k^2 + 4k + 4 \\
            &= (8\ell + 1) + 4(2m + 1) + 4 \\
            &= 8\ell + 8m + 8 + 1 \\
            &= 8(\ell + m + 1) + 1 
        \end{align*}

        因为 $\ell, m \in \mathbb{Z}$,可得 $\ell+m \in \mathbb{Z}$。因此 $(k+2)^2$ 比 $8$ 的倍数大 $1$。故 $P(k + 2)$ 成立。

        根据归纳法,$P(n)$ 对于所有 $n \in O$ 成立。
    \end{proof}
\end{example}

\emph{后续问题}:你能否证明当偶数的平方除以 $8$ 时,其余数不为 $1$?(这将使该命题成为一个\emph{当且仅当}型命题。)你能发现这些偶数平方的余数规律吗?能否证明你的观察?

(提示:你可能不需要使用归纳法来证明这些命题!)

\subsubsection*{方法讨论}

让我们探讨一下为什么这种方法有效。其基本原理与我们之前看到的其他归纳法完全一致,唯一的不同在于归纳步骤。由于奇数``间隔为 2'',我们的目标是证明:
\[\forall k \in O \centerdot P(k) \implies P(k + 2)\]
这体现了与标准归纳法相同的核心思想:利用命题的一个实例推导``下一个''实例成立。区别仅仅在于``下一个''的定义。为严谨起见,我们给出描述这种方法的定理。证明过程留给你来完成。

\begin{theorem}[奇数上的归纳]\label{theorem5.3.5}
    设 $O$ 为奇数集,$P(n)$ 为变量命题。假设

    \begin{enumerate}[label=(\arabic*)]
        \item $P(1)$ 成立
        \item $\forall k \in O \centerdot P(k) \implies P(k + 2)$ 成立
    \end{enumerate}

    则 $\forall n \in O \centerdot P(n)$ 成立。
\end{theorem}

\begin{proof}
    留给读者作为 \ref{sec:section5.3.4} 节的习题 \ref{exc:exercises5.3.2}。
\end{proof}

同理,对偶数进行归纳同样有效。以下定理阐述偶数归纳法,证明过程仍留给你来完成。

\begin{theorem}[偶数上的归纳]\label{theorem5.3.6}
    设 $E$ 为偶数集,$P(n)$ 为变量命题。假设

    \begin{enumerate}[label=(\arabic*)]
        \item $P(2)$ 成立
        \item $\forall k \in O \centerdot P(k) \implies P(k + 2)$ 成立
    \end{enumerate}

    则 $\forall n \in E \centerdot P(n)$ 成立。
\end{theorem}

\begin{proof}
    留给读者作为 \ref{sec:section5.3.4} 节的习题 \ref{exc:exercises5.3.2}。
\end{proof}

\subsubsection*{组合和修改这些方法}

假设我们有一个命题 $P(n)$,并希望证明它对所有自然数 $n$ 成立。该命题及其理论背景可能较为复杂,导致无法用传统归纳法直接证明。这可能源于需要特定代数技巧、缺乏高效证明方法,或存在某些深层原因。无论何种情况,我们可以采用新型归纳法,将证明分为若干部分,从而验证 $P(n)$ 对所有 $n \in \mathbb{N}$ 成立。

这些新型方法可视为``跳跃式''归纳法。例如,证明命题对每个奇数成立时,其本质与传统归纳法相同,仅在归纳步骤中跳过偶数。类似方法也适用于偶数(需调整基本情况,因为第一个偶数是 $2$ 而非 $1$)。若先用``奇数''方法证明,再用``偶数''方法证明,即可覆盖所有自然数。

以下例子正是采用此思路,但你会注意到其跳跃步长为 $3$(而非奇偶归纳法中的 $2$)。此处我们暂不具体陈述或证明相关定理,而是基于对归纳法工作原理的直观理解推进。这些定理和证明与先前内容高度相似。若你希望练习或完善笔记,请自行陈述并证明即将使用的方法!

\begin{example}[$2$ 的幂与 $7$ 的倍数]
    
    \textbf{声明}:对于所有自然数 $n \in \mathbb{N}, 2^n+1$ \emph{不是} $7$ 的倍数。

    (建议先通过探索性计算观察 $2^n + 1$ 除以 $7$ 的余数规律。你会发现余数呈现长度为 $3$ 的循环。这正是本证明的核心内容,但需重新整理命题并设计证明方法。)

    \begin{proof}
        定义集合 $A_1, A_2, A_3$ 为:
        \begin{align*}
            A_1 &= \{n \in \mathbb{N} \mid \exists m \in \mathbb{N} \cup \{0\} \centerdot n = 3m + 1\} = \{1, 4, 7, 10, \dots \} \\
            A_2 &= \{n \in \mathbb{N} \mid \exists m \in \mathbb{N} \cup \{0\} \centerdot n = 3m + 2\} = \{2, 5, 8, 11, \dots \} \\
            A_3 &= \{n \in \mathbb{N} \mid \exists m \in \mathbb{N} \cup \{0\} \centerdot n = 3m \quad\:\:\:\} = \{3, 6, 9, 12, \dots \} 
        \end{align*}

        (也就是说,这三个集合按除以 $3$ 的余数对 $\mathbb{N}$ 进行划分。)

        设 $P(n)$ 为命题``$2^n+1$ 不能被 $3$ 整除''。我们要通过归纳法证明 $\forall n \in \mathbb{N} \centerdot P(n)$ 成立。

        定义命题 $Q(n), R(n), S(n)$ 如下:
        \begin{align*}
            Q(n) \text{\ 为\ } \exists \ell \in \mathbb{N} \cup \{0\} \centerdot 2^n + 1 = 7\ell + 3 \\
            R(n) \text{\ 为\ } \exists \ell \in \mathbb{N} \cup \{0\} \centerdot 2^n + 1 = 7\ell + 5 \\
            S(n) \text{\ 为\ } \exists \ell \in \mathbb{N} \cup \{0\} \centerdot 2^n + 1 = 7\ell + 2 
        \end{align*}

        不难发现
        \[\forall n \in \mathbb{N} \centerdot \big(Q(n) \lor R(n) \lor S(n)\big) \implies P(n)\]

        这是因为 $7$ 的倍数加 $2, 3, 5$ 都不是 $7$ 的倍数。\\
        
        首先,我们通过对 $n$ 应用归纳法来证明 $\forall n \in A_1 \centerdot Q(n)$ 成立。

        \textbf{基本情况}:不难发现 $2^1+1=3 = 0 \cdot 7 + 3$,因此 $Q(1)$ 成立。

        \textbf{归纳假设}:设 $k \in A_1$ 为任意固定元素,假设 $Q(k)$ 成立。

        \textbf{归纳步骤}:我们的目标是推导出 $Q(k+3)$ 成立。

        由 $k \in A_1$ 可知 $\exists m \in \mathbb{N} \centerdot k = 3m + 1$。给定这样的 $m$。

        根据归纳假设,可得 $\exists \ell \in \mathbb{N} \centerdot 2^k + 1 = 7\ell + 3$。给定这样的 $\ell$,这意味着 $2^k = 7\ell + 2$。

        推导可得
        \[2^{k+3} = 2^3 \cdot 2^k = 8 \cdot (7\ell + 2) = 56\ell + 16\]
        \[2^{k+3} + 1 = 56\ell + 17 = 7(8\ell) + 14 + 3 = 7(8\ell + 2) + 3\]

        故 $Q(k + 3)$ 也成立。因此 $\forall n \in A_1 \centerdot Q(n)$。\\

        接着,我们通过对 $n$ 应用归纳法来证明 $\forall n \in A_2 \centerdot R(n)$ 成立。

        \textbf{基本情况}:不难发现 $2^2+1=5 = 0 \cdot 7 + 5$,因此 $R(2)$ 成立。

        \textbf{归纳假设}:设 $k \in A_2$ 为任意固定元素,假设 $R(k)$ 成立。

        \textbf{归纳步骤}:我们的目标是推导出 $R(k+3)$ 成立。

        根据归纳假设,可得 $\exists \ell \in \mathbb{N} \centerdot 2^k + 1 = 7\ell + 5$。给定这样的 $\ell$,这意味着 $2^k = 7\ell + 4$。

        推导可得
        \[2^{k+3} = 2^3 \cdot 2^k = 8 \cdot (7\ell + 4) = 56\ell + 32\]
        \[2^{k+3} + 1 = 56\ell + 33 = 7(8\ell) + 28 + 5 = 7(8\ell + 4) + 5\]

        故 $R(k + 3)$ 也成立。因此 $\forall n \in A_2 \centerdot R(n)$。\\

        最后,我们通过对 $n$ 应用归纳法来证明 $\forall n \in A_3 \centerdot S(n)$ 成立。

        \textbf{基本情况}:不难发现 $2^3+1=9 = 1 \cdot 7 + 2$,因此 $S(3)$ 成立。

        \textbf{归纳假设}:设 $k \in A_3$ 为任意固定元素,假设 $S(k)$ 成立。

        \textbf{归纳步骤}:我们的目标是推导出 $S(k+3)$ 成立。

        根据归纳假设,可得 $\exists \ell \in \mathbb{N} \centerdot 2^k + 1 = 7\ell + 2$。给定这样的 $\ell$,这意味着 $2^k = 7\ell + 1$。

        推导可得
        \[2^{k+3} = 2^3 \cdot 2^k = 8 \cdot (7\ell + 1) = 56\ell + 8\]
        \[2^{k+3} + 1 = 56\ell + 9 = 7(8\ell) + 7 + 2 = 7(8\ell + 1) + 2\]

        故 $S(k + 3)$ 也成立。因此 $\forall n \in A_3 \centerdot S(n)$。\\

        综上,我们证明了对于每个自然数,$Q(n), R(n), S(n)$ 三者中必有一个成立(取决于数字除以 $3$ 的余数)。因此,对于所有自然数 $n$, $2^n + 1$ 都不是 $7$ 的倍数。
    \end{proof}
\end{example}

实际上,我们的证明得出了比原声明\emph{更强}的结论。我们不仅证明了形如 $2^n + 1$ 的数都不是 $7$ 的倍数,还精确解释了其原因。

在本节的练习中,我们安排了一些题目,引导你通过识别``跳跃''和命题来进行类似的证明。在 \ref{sec:section5.7} 节的习题中,我们还包含了一些可能需要此类论证的问题(但不会像这里那样提示整体结构)。

值得注意的是,只要你想实施的``跳跃''遵循某种易于识别的模式,这些方法就能方便地应用于各种情形。之前的例子采用了步长为 $3$ 的跳跃,因此将自然数划分为三个集合并进行归纳。其核心在于拥有一个获取命题``下一个''实例的``公式'':从 $P(k)$ 出发,尝试推导 $P(k + 3)$。类似地,你可以设想步长为 $4$ 或为 $10$ 的归纳,甚至是指数增长的归纳——例如证明命题 $P(n)$ 对所有 $2$ 的 $n$ 次幂成立:
\[P(1) \text{\ 成立,且\ } \forall n \in \mathbb{N} \centerdot P(n) \implies P(2^n)\]

再次强调,这一切都依赖于某种指明\emph{下一个}实例的``公式''或``规则''。因此,\emph{我们无法直接在全体质数集上进行归纳}。若想证明某个命题对所有质数成立,别指望使用归纳法!除非你有某种``规则''指明``若 $k$ 是质数,则下一个质数是……''。倘若知晓这样的规则,数学界必将洗耳恭听!这将解答众多关于质数的未解之谜,使你跻身史上最著名数学家之列——绝非戏言!


% !TeX root = ../../../book.tex
\subsection{习题}\label{sec:section5.3.4}

\subsubsection*{温故知新}

以口头或书面的形式简要回答以下问题。这些问题全都基于你刚刚阅读的内容,如果忘记了具体定义、概念或示例,可以回顾相关内容。确保在继续学习之前能够自信地作答这些问题,这将有助于你的理解和记忆!

\begin{enumerate}[label=(\arabic*)]
    \item 多米诺骨牌类比如何描述一个基本情况不是 $1$ 的归纳证明?
    \item 提供一个证明模板,用于证明命题 $P(n)$ 对所有大于或等于 $7$ 的奇数都成立。
    \item 为什么我们不能``对质数进行归纳''?
\end{enumerate}

\subsubsection*{小试牛刀}

尝试解答以下问题。这些题目需动笔书写或口头阐述答案,旨在帮助你熟练运用新概念、定义及符号。题目难度适中,确保掌握它们将大有裨益!

\begin{enumerate}[label=(\arabic*)]
    \item 证明定理 \ref{theorem5.3.3}。 \label{exc:exercises5.3.1}
    \item 证明定理 \ref{theorem5.3.5} 和定理 \ref{theorem5.3.6}。 \label{exc:exercises5.3.2}
    \item 提出一个定理,描述如何对所有 $5$ 的倍数进行归纳推理,并证明这个定理。
    \item 考虑不等式 $n^3 < 3^{n-1}$。
        \begin{enumerate}[label=(\alph*)]
            \item 证明对于所有 $n \ge 6$,该不等式成立。
            \item 证明对于所有 $\{1,2,3,4,5\}$,该不等式不成立。(这一问很简单)
            \item 证明对于所有 $n \le 0$,该不等式成立。
        \end{enumerate}
    \item 定义数列
        \[x_1 = 2, x_2 = 2, \forall n \in \mathbb{N} - \{1, 2\} \centerdot x_n = x_{n-2} + 1\]
        设 $P(n)$ 为命题
        \[x_n = \frac{1}{2}(n+1)+\frac{1}{4}(1+(-1)^n)\]
        \begin{enumerate}[label=(\alph*)]
            \item 设 $O$ 为奇数集。用归纳法证明 $\forall n \in O \centerdot P(n)$。
            \item 设 $E$ 为偶数集。用归纳法证明 $\forall n \in E \centerdot P(n)$。
        \end{enumerate}
    \item 考虑下列命题
        \[\sum_{k=1}^{n} (-1)^{k-1}k^2 = (-1)^{k-1}\sum_{k=1}^{n} k\]
        也就是说,我们声明
        \[1^2 - 2^2 + 3^2 - 4^2 + \dots + (1)^{n-1}n^2 = (-1)^{n-1}(1 + 2 + 3 + \dots + n)\]
        对于所有 $n \in \mathbb{N}$ 成立。
        \begin{enumerate}[label=(\alph*)]
            \item 证明上述公式对于 $n=1$ 和 $n=2$ 成立。
            \item 证明上述公式如果对于某个 $k$ 成立,则对于 $k+2$ 也成立。
            \item 直观地解释为什么 (a) 和 (b) 证明了该命题。
        \end{enumerate}
\end{enumerate}
