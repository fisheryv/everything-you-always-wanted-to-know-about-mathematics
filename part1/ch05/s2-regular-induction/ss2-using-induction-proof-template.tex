% !TeX root = ../../../book.tex
\subsection{使用归纳法:证明模板}

根据前文的考察,我们设计了一个标准的``\textbf{归纳证明}''模板。(该模板也可以并入上一章的证明策略中,进一步丰富我们的数学工具箱!)该模板确保证明过程条理清晰、逻辑严密、易于理解:

\begin{itemize}
    \item 首先,定义一个命题 $P(n)$,明确告诉读者我们的证明目标。
    \item 然后,验证\textbf{基本情况 (BC)},确保其满足数学归纳原理 (PMI) 的第一个条件。
    \item 接下来,验证条件陈述 $\forall k \in \mathbb{N} \centerdot P(k) \implies P(k+1)$,从而证明归纳原理的第二个条件。为此,我们将采用直接证明策略,分为以下两个步骤:
        \begin{itemize}
            \item 首先,提出\textbf{归纳假设 (IH)},假设存在任意固定自然数 $k$ 使得 $P(k)$ 成立。
            \item 然后,执行\textbf{归纳步骤 (IS)},基于这个假设推导出 $P(k+1)$ 也成立。
        \end{itemize} 
    \item 通过上述步骤——\textbf{基本情况 (BC)、归纳假设 (IH) 和归纳步骤 (IS)},我们验证了数学归纳原理的所有条件,并可以得出结论:$\forall n \in \mathbb{N} \centerdot P(n)$。
    \\
    最后,总结结论,明确告知读者我们已经完成了证明。
\end{itemize}

\subsubsection*{\textcolor{blue}{``归纳证明''模板}}

\setlength{\fboxrule}{2pt}
\setlength\fboxsep{5mm}
\begin{center}
\noindent \fcolorbox{blue}{white}{%
    \parbox{0.85\textwidth}{%
        \linespread{1.5}\selectfont
        \textbf{目标:} 证明 $\forall n \in \mathbb{N} \centerdot P(n)$
        \begin{proof}\\
            设 $P(n)$ 为命题``$\underline{\qquad\qquad\qquad}$''。

            我们对 $n$ 采用归纳法证明 $\forall n \in \mathbb{N} \centerdot P(n)$。

            \textbf{基本情况}:$P(1)$ 成立,因为 $\underline{\qquad\qquad\qquad}$。

            \textbf{归纳假设}:设 $k \in \mathbb{N}$ 为任意固定自然数,假设 $P(k)$ 成立。

            \textbf{归纳步骤}:推导出 $P(k+1)$ 也成立。

            根据数学归纳原理可得 $\forall n \in \mathbb{N} \centerdot P(n)$。
        \end{proof}
    }
}
\end{center}

\subsubsection*{常见问题及说明}

以下是一些忠告和建议。这些建议基于我们对优秀归纳论证应具备特质的理解,并结合了多年来观察到的学生常见错误。

\begin{itemize}
    \item \textbf{务必明确定义命题。}
    
    问题描述有时会直接给出命题定义,但并非所有情况都会明确以 $P(n)$ 的形式呈现。若后续需要引用该命题却未显式定义,则引用 $P(n)$ 将失去意义。因此,若计划引用某个命题,请确保先行定义。

    为求简洁,可采用表述:``设 $P(n)$ 为前文所定义的命题。''(但需确保变量 $n$ 与原始命题一致!

    \item \textbf{明确声明使用数学归纳法并指明归纳变量。}
    
    未来可能遇到涉及多个变量的归纳证明。即便证明整体遵循归纳结构,读者也可能未能察觉。因此,建议开宗明义地声明使用归纳法,避免误解。

    \item \textbf{基本情况尽可能详细清晰。}
    
    切勿仅仅陈述 $P(1)$ 的内容,并期待读者自行理解其为何成立。证明责任在证明者,而非读者!

    同样,避免仅仅写出 $P(1)$ 并标注 $\checkmark$,这并未完成任何证明!

    如果命题 $P(1)$ 是一个方程(这是常见情形),必须展示等式两边的推导过程,而非简单列出等式让读者自定理解。

    \item \textbf{归纳假设和归纳步骤共同运用直接证明策略证明 $\implies$ 关系。}
    
    归纳假设引入任意固定自然数 $k$,并假设 $P(k)$ 成立。以此为基础推导 $P(k+1)$,从而证明归纳法条件 (2) 的条件命题。

    此处必须明确变量 $k$ 的范围!仅声明``假设 $P(k)$''并不充分。这里 $k$ 是什么?是自然数吗?应表述为:``设 $k \in \mathbb{N}$ 并假设 $P(k)$ 成立''。对数学读者而言,``设 $k \in \mathbb{N}$''已隐含``任意固定''之意。

    \item \textbf{在归纳假设中明确写出 $P(k)$ 的含义大有帮助。}
    
    首先,这有助于读者理解假设内容,更好地跟进后续证明。

    其次,这能明确归纳步骤的目标——证明 $P(k+1)$。若在此步骤遇到困难(如考试或作业中),可尝试在纸上分别写下 $P(k)$ 和 $P(k+1)$ 的完整形式。分析二者之间的关联:从 $P(k)$ 向下演绎,从 $P(k+1)$ 向上分析,寻找衔接点。

    \item \textbf{归纳步骤中必须应用归纳假设!}
    
    若未使用归纳假设,则无需采用归纳法。

    应用归纳假设时必须明确指出,不要指望读者自行识别。

    \item \textbf{最后,明确陈述结论}
    
    明确告知读者所完成的证明。\\
\end{itemize}

以上探讨了如何撰写严谨的归纳证明,接下来让我们实践应用。
