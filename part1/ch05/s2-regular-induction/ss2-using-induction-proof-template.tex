% !TeX root = ../../../book.tex
\subsection{使用归纳法:证明模板}

根据前文的考察,我们可以设计一个标准的``\textbf{归纳证明}''模板。(这个模板也可以并入上一章的证明策略中,进一步丰富我们的数学工具箱!)值得注意的是,模板中的每个步骤都旨在确保证明的条理清晰、逻辑严密且易于理解:

\begin{itemize}
    \item 首先,我们需要定义一个命题 $P(n)$,明确告诉读者我们的证明目标。
    \item 然后,我们需要验证\textbf{基本情况(BC)},确保满足数学归纳原理(PMI)的第一个条件。
    \item 接下来,我们要验证条件陈述 $\forall k \in \mathbb{N} \centerdot P(k) \implies P(k+1)$,以证明归纳原理中的第二个条件。为此,我们将采用直接证明策略,分为以下两个步骤:
        \begin{itemize}
            \item 首先,我们提出\textbf{归纳假设(IH)},假设存在任意固定自然数 $k$ 使得 $P(k)$ 成立。
            \item 然后,我们进行\textbf{归纳步骤(IS)},基于这个假设推导出 $P(k+1)$ 也成立。
        \end{itemize} 
    \item 在这些步骤---\textbf{基本情况(BC)、归纳假设(IH)和归纳步骤(IS)}---之中,我们验证了数学归纳原理的所有条件,并可以得出结论:$\forall n \in \mathbb{N} \centerdot P(n)$。
    
    最后,我们总结此结论,告诉读者我们已经完成了证明。
\end{itemize}

\subsubsection*{\textcolor{blue}{``归纳证明''模板}}

\setlength{\fboxrule}{2pt}
\setlength\fboxsep{5mm}
\begin{center}
\noindent \fcolorbox{blue}{white}{%
    \parbox{0.85\textwidth}{%
        \linespread{1.5}\selectfont
        \textbf{目标:} 证明 $\forall n \in \mathbb{N} \centerdot P(n)$
        \begin{proof}\\
            设 $P(n)$ 为命题 ``$\underline{\qquad\qquad\qquad}$''。\\
            我们对 $n$ 采用归纳法证明 $\forall n \in \mathbb{N} \centerdot P(n)$。\\
            \textbf{基本情况}:$P(1)$ 成立,因为 $\underline{\qquad\qquad\qquad}$。\\
            \textbf{归纳假设}:设 $k \in \mathbb{N}$ 是任意固定的,假设 $P(k)$ 成立。\\
            \textbf{归纳步骤}:推导出 $P(k+1)$ 也成立。\\
            根据数学归纳原理可得$\forall n \in \mathbb{N} \centerdot P(n)$。
        \end{proof}
    }
}
\end{center}

\subsubsection*{常见问题及说明}

以下是一些忠告和建议。这些建议基于我们对质量优秀、结构严谨的归纳论证应具备的特质的理解,同时也包括了我们多年来观察到学生们经常犯的一些常见错误。

\begin{itemize}
    \item \textbf{一定要定义命题。}
    
    有时候,问题或练习的描述中会直接给出命题的定义。但并不是所有情况下命题都会明确以 $P(n)$ 的形式出现。如果后文中需要引用这个命题,而它未被显式定义,那么引用 $P(n)$ 就没有任何意义。因此,如果你打算引用某个命题,请确保首先对其进行定义。

    为了表达简洁,你可以使用这样的表述:``设 $P(n)$ 为前文中定义的命题。''(但请确保变量 $n$ 确实是之前命题中使用的,以保证表述的一致性!)

    \item \textbf{明确表明你正在使用数学归纳法,并指出是对哪个变量应用归纳。}
    
    在未来,你可能会遇到涉及多个变量的归纳证明。此外,仅仅因为你的整个证明遵循某种归纳结构,并不意味着读者就理解你正在使用归纳法。因此,最好在一开始就告诉读者你正在使用归纳法,这可以避免很多不必要的误解。

    \item \textbf{基本情况尽可能详细清晰。}
    
    不要仅仅说明 $P(1)$ 是什么,并期待读者能理解其为何成立。这个责任在证明者,而非读者!

    同样,不要只是单纯地写出命题 $P(1)$ 并在旁边打一个 $\checkmark$。这样做并没有证明任何东西!

    如果命题 $P(1)$ 是一个方程(这是常见的情况),你需要展示方程两边为什么相等,而不是简单地写下方程并期望读者自己理解。

    \item \textbf{归纳假设和归纳步骤共同运用直接证明策略证明 $\implies$ 关系。}
    
    归纳假设引入了一个任意且固定的自然数 $k$,并假设 $P(k) \implies P(k+1)$。这是我们的假设基础。我们接着用这个假设来推导 $P(k+1)$,从而证明归纳法中条件(2)的条件陈述。

    这里一定要明确指出变量 $k$ 的范围!像``假设 $P(k)$''这样的陈述是不够的。$k$ 是什么?是自然数吗?正确的表述应该是``设 $k \in \mathbb{N}$ 并假设 $P(k)$ 成立'',这对数学读者来说,``设 $k \in \mathbb{N}$''隐含着``设 $k \in \mathbb{N}$(任意且固定)''。

    \item \textbf{在归纳假设中明确写出 $P(k)$ 的含义大有帮助。}
    
    首先,这种表述有助于读者理解你的假设,并更好地跟进证明的其他部分。

    此外,这也有助于你明确如何证明 $P(k+1)$,这是你在此步骤中的目标。如果你在解决这一步骤时感到困难(可能是在考试或作业中),不妨在纸上先写下 $P(k)$ 的含义,在下面写下 $P(k+1)$ 的含义。现在你能看出它们之间可能的联系了吗?尝试从 $P(k)$ 向下分析,从 $P(k+1)$ 向上分析,并在中间找到它们的联系。

    \item \textbf{在归纳步骤中必须调用归纳假设!}
    
    如果你没有使用归纳假设,那么为何还需要使用归纳法呢?

    当你使用归纳假设时,明确指出你正在这样做。不要指望读者能记住或识别这一点。

    \item \textbf{最后,明确给出你的结论}
    
    明确告诉读者你完成了什么。\\
\end{itemize}

以上,我们讨论了如何撰写一个好的归纳证明,接下来让我们动手实践一下。
