% !TeX root = ../../../book.tex
\subsection{定理的陈述与证明}

回顾第 \ref{ch:chapter03} 章所述的\textbf{数学归纳原理}。该原理可通过\textbf{多米诺骨牌}等类比辅助理解。如果你尚未阅读 \ref{sec:section3.8} 节关于如何定义自然数 $\mathbb{N}$ 的选学内容,可能遗漏定理的完整陈述,但这不影响你理解其核心思想并正确应用归纳过程。

\begin{theorem}[数学归纳原理]
    设 $P(n)$ 为依赖于自然数 $n$ 的``事实''或``观察''。若
    \begin{enumerate}
        \item $P(1)$ 为\verb|真|。
        \item 给定任意 $k \in \mathbb{N}$,如果 $P(k)$ 为\verb|真|,可推导出 $P(k+1)$ 为\verb|真|。
    \end{enumerate}
    则陈述 $P(n)$ 对所有自然数 $n \in \mathbb{N}$ 均成立。
\end{theorem}

这里的表述冗长且模糊——依赖于自然数的``事实''听起来像是一个\textbf{变量命题};``如果……那么可推导出……''听起来像是一个\textbf{条件陈述}。这些表述都是为了阐明某些逻辑基础,我们可以借助上一章的概念和符号体系优化定理表述。建议你先尝试自行改写,并回顾定理的证明思路(即使跳过选学内容也无妨)。我们将沿用 \ref{sec:section3.8.2} 节的证明方法,但此次将采用逻辑符号工具。准备好了吗?那我们开始吧!

\begin{theorem}[数学归纳原理]\label{theorem5.2.2}
    设 $P(n)$ 为变量命题。若
    \begin{enumerate}[label=(\arabic*)]
        \item $P(1)$ 为\verb|真|。
        \item $\forall k \in \mathbb{N} \centerdot P(k) \implies P(k+1)$ 为\verb|真|。
    \end{enumerate}
    则 $\forall n \in \mathbb{N} \centerdot P(n)$ 为\verb|真|。
\end{theorem}

新表述完整保留了核心思想——某个初始事实成立,并且每一个事实都能推导出下一个事实,从而令所有事实均成立——但采用逻辑符号使其更加精炼。请确认新旧表述的等价性后再继续阅读。

现在,我们的目标是\emph{证明}该定理的有效性。换言之,我们要证明数学归纳法是有效的证明方法!此前通过真值表证明了条件命题与其逆否命题的逻辑等价性,这为我们提供了一种证明策略。

证明前,建议阅读 \ref{sec:section3.8} 节关于如何定义自然数的内容。该节引入的\textbf{归纳集}定义至关重要:我们证明了 $\mathbb{N}$ 是``最小''归纳集,这意味着 $\mathbb{N}$ 是所有归纳集的子集。此性质正是归纳法的基石。下文将给出关键定义的符号化版本(略去部分集合论细节),但阅读 \ref{sec:section3.8} 节有助于全面理解。

\begin{definition}
    设 $I$ 为集合。若 $I$ 满足以下条件:
    \begin{enumerate}
        \item $1 \in I$
        \item 对于任意元素 $k$, $k \in I \implies k + 1 \in I$
    \end{enumerate}
    则称 $I$ 为\dotuline{归纳集}。
\end{definition}

\begin{definition}
    所有\dotuline{自然数}的集合定义为
    \[\mathbb{N}:=\{x \mid \text{对于每个归纳集\ }I, x \in I\}\]
    换言之,$\mathbb{N}$ 是最小的归纳集:
    \[\mathbb{N} = \bigcap_{I \in \{S \mid S \text{为归纳集}\}} I\] 
\end{definition}

\begin{proof}
    设 $P(n)$ 为定义在自然数上的变量命题。假设定理中给出的两个条件成立,即
    \begin{enumerate}[label=(\arabic*)]
        \item $P(1)$ 为\verb|真|
        \item $\forall k \in \mathbb{N} \centerdot P(k) \implies P(k+1)$ 为\verb|真|。
    \end{enumerate}

    设 $S$ 为 $P(n)$ 为\verb|真|的实例的集合。也就是说,定义
    \[S = \{n \in \mathbb{N} \mid P(n) \text{\ 为真}\}\]

    根据定义(使用集合构建符),$S \subseteq \mathbb{N}$。

    条件 (1) 确保 $1 \in S$。

    条件 (2) 确保 $\forall k \in \mathbb{N} \centerdot k \in S \implies k+1 \in S$。

    这两个条件共同确保 $S$ 为\emph{归纳集}。根据上面 $\mathbb{N}$ 的定义,我们知道 $\mathbb{N} \subseteq S$。

    因此,通过双向包含论证可得 $S = \mathbb{N}$。这意味着陈述 $P(n)$ 对于\emph{每个}自然数 $n$ 都成立,
    
    即 $\forall n \in \mathbb{N} \centerdot P(n)$ 为\verb|真|!
\end{proof}

虽然理解该证明背后的集合论原理不是使用归纳法或编写归纳证明的必要条件,但探究其逻辑根源可以增进数学理解,或激发对数理逻辑逻辑和集合论的兴趣。

通过重新表述数学归纳原理 (PMI),我们获得关键认知:验证归纳证明的核心在于依据定理条件确认 (1) 和 (2) 成立(即证明命题 $P(n)$ 的``真值集''构成归纳集)。
