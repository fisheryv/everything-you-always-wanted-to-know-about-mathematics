% !TeX root = ../../../book.tex
\subsection{习题}

\subsubsection*{温故知新}

以口头或书面的形式简要回答以下问题。这些问题全都基于你刚刚阅读的内容,所以如果忘记了具体的定义、概念或示例,可以回去重读相关部分。确保在继续学习之前能够自信地回答这些问题,这将有助于你的理解和记忆!

\begin{enumerate}[label=(\arabic*)]
    \item 数学归纳原理是什么?它是如何被证明的?
    \item 什么是数学归纳法的基本情况?它与数学归纳原理有什么关系?
    \item 归纳假设和归纳步骤有什么关系?它们与数学归纳原理又有什么关系?
    \item 为什么在归纳步骤中调用归纳假设很重要?
\end{enumerate}

\subsubsection*{小试牛刀}

尝试回答以下问题。这些题目要求你实际动笔写下答案,或(对朋友/同学)口头陈述答案。目的是帮助你练习使用新的概念、定义和符号。题目都比较简单,确保能够解决这些问题将对你大有帮助!

\begin{enumerate}[label=(\arabic*)]
    \item 证明,对于所有 $n \in \mathbb{N}$
        \[\sum_{i=1}^{n} i^3 = \bigg[\frac{n(n+1)}{2}\bigg]^2\]
    \item 证明奇数的平方都比 $8$ 的倍数多 $1$。也就是说证明,对于所有 $n \in \mathbb{N}$
        \[(2n+1)^2-1 \text{是 8 的整数倍}\]
    \item 考虑声明:对于所有 $n \in \mathbb{N}, 7^n-4^n$ 是 $3$ 的整数倍。
    
    用逻辑符号重写该声明,然后用归纳法证明它。
    \item 斐波那契数列定义为
    \[f_0=1 \quad\text{且}\quad f_1=1 \quad\text{且}\quad \forall n=\mathbb{N}-\{1\} \centerdot f_n=f_{n-1}+f_{n-2}\]
    用归纳法证明对于所有 $n \in \mathbb{N}$ 下列声明都成立:
    \begin{enumerate}[label=(\alph*)]
        \item $\displaystyle\sum_{i=1}^{n} f_i = f_{n+2}-1$
        \item $\displaystyle\sum_{i=1}^{n} f_{2i-1} = f_{2n}$
        \item $f_{4n}$ 是 $3$ 的整数倍
        \item \textbf{挑战 1}:($n$ 是 $3$ 的整数倍) $\implies$ ($f_n$ 为偶数)
        \item \textbf{挑战 2}:($n$ 不是 $3$ 的整数倍) $\implies$ ($f_n$ 为奇数)
    \end{enumerate}
\end{enumerate}