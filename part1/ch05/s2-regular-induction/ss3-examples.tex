% !TeX root = ../../../book.tex
\subsection{示例}

\begin{example}[奇数之和是平方数]

    \textbf{声明}:前 $n$ 个奇数之和为 $n^2$。

    (注意:我们之前在 \ref{sec:section1.4.3} 节中见过这个谜题,并在 \ref{sec:section2.3.4} 节中引导你使用归纳法详细分析它。现在,我们将在这里给出该命题的良好证明。)
\end{example}

\begin{proof}
    设 $P(n)$ 为命题
    \[1+3+5+ \dots +2n-1 = \sum_{i=1}^n (2i-1) = n^2\]
    我们通过对 $n$ 用归纳法证明 $\forall n \in \mathbb{N} \centerdot P(n)$。

    \textbf{基本情况}:考虑 $n=1$,不难发现
    \[\sum_{i=1}^1 (2i-1) = 1 \quad \text{且} \quad 1=1^2\]
    故
    \[\sum_{i=1}^1 (2i-1) = 1^2\]
    因此 $P(1)$ 成立。

    \textbf{归纳假设}:设 $k \in \mathbb{N}$ 是任意固定的。假设 $P(k)$ 成立,这意味着
    \[\sum_{i=1}^k (2i-1) = k^2\]

    \textbf{归纳步骤}:考虑 $k+1$,通过分离出第 $k+1$ 项可得
    \[\sum_{i=1}^{k+1} (2i-1) = 2(k + 1) - 1 + \sum_{i=1}^k (2i-1) = 2k + 1 + \sum_{i=1}^k (2i-1)\]
    利用归纳假设,替换上面等式右边的求和,将其化简为
    \[\sum_{i=1}^{k+1} (2i-1) = 2k+1+k^2\]
    因式分解后得
    \[\sum_{i=1}^{k+1} (2i-1) = (k+1)^2\]
    因此 $P(k+1)$ 成立。

    根据数学归纳原理,我们可以得出结论 $\forall n \in \mathbb{N} \centerdot P(n)$。
\end{proof}

下面是几何级数求和公式的良好归纳法证明。\\

\begin{example}[几何级数求和公式]
    
    \textbf{声明}:对于所有 $q \in \mathbb{R}-\{0,1\}$ 且 $n \in \mathbb{N}$,下列公式成立:
    \[\sum_{i=0}^{n-1}q^i = 1+q+q^2+\dots+q^{n-1} = \frac{q^n-1}{q-1}\]
\end{example}

\begin{proof}
    设 $q \in \mathbb{R}-\{0,1\}$ 是任意固定的。定义 $P(n)$ 为命题
    \[\sum_{i=0}^{n-1}q^i = \frac{q^n-1}{q-1}\]
    我们通过对 $n$ 用归纳法证明 $\forall n \in \mathbb{N} \centerdot P(n)$。

    \textbf{基本情况}:考虑 $n=1$,不难发现
    \[\sum_{i=0}^{n-1}q^i = \sum_{i=0}^{0}q^i = q^0\]
    因为 $q \neq 0$,所以 $q^0=1$,因此
    \[\sum_{i=0}^{n-1}q^i = \sum_{i=0}^{0}q^i = q^0 = 1\]
    同时
    \[\frac{q^n-1}{q-1} = \frac{q-1}{q-1}\]
    因为 $q \neq 1$,所以
    \[\frac{q^n-1}{q-1} = \frac{q-1}{q-1} = 1\]
    因此 $P(1)$ 成立。

    \textbf{归纳假设}:设 $k \in \mathbb{N}$ 是任意固定的。假设 $P(k)$ 成立,这意味着
    \[\sum_{i=0}^{k-1}q^i = \frac{q^k-1}{q-1}\]

    \textbf{归纳步骤}:我们要证明 $P(k+1)$ 成立,即
    \[\sum_{i=0}^{k}q^i = \frac{q^{k+1}-1}{q-1}\]
    \begin{align*}
        \sum_{i=0}^{k}q^i &= \Bigg(\sum_{i=0}^{k-1}q^i\Bigg)+q^k &\qquad \text{求和运算符的定义}\\
        &= \frac{q^k-1}{q-1} + q^k &\qquad \text{代入归纳假设}\\
        &= \frac{q^k-1+q^k(q-1)}{q-1} &\qquad \text{通分}\\
        &= \frac{q^k-1+q^{k+1}-q^k}{q-1} &\qquad \text{展开}\\ 
        &= \frac{q^{k+1}-1}{q-1} &\qquad \text{合并同类项}
    \end{align*}
    这表明 $P(k+1)$ 成立。

    根据数学归纳原理,我们可以得出结论 $\forall n \in \mathbb{N} \centerdot P(n)$。
\end{proof}

\emph{扩展问题}:为什么上例中我们需要声明 $q \notin \{0, 1\}$?

$q=0$ 会发生什么?我们的证明会在哪一步出问题?该求和公式还成立吗?如果成立,请证明它;如果不成立,尝试修正它。

同理,请尝试回答 $q=1$ 的情况。
