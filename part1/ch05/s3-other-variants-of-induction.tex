% !TeX root = ../../book.tex
\section{归纳法的其他变体}

现在我们已经非常熟练地掌握了归纳法,并通过多个例子感受到了其用途,接下来我们将介绍这种方法的两种变体。首先要明确的是,使用归纳法来证明某个命题对所有自然数 $n \in \mathbb{N}$ 成立并没有什么``特别''的地方。当然,自然数集合 $\mathbb{N}$ 本身是非常特殊的!这里我们想表达的是,归纳法同样可以用来证明某个命题对于某个不同类型集合 $S$ 中的每一个元素 $n$ 都成立。在后续的讨论和示例中,我们将详细介绍这些集合。

\subsection{$n = 1$ 以外的基本情况}\label{sec:section5.3.1}

在进行归纳法证明时,我们需要一个基本情况,但这个起始点并不一定非得是 $n = 1$。比如,如果我们有一个命题 $P(n)$,当 $n = 1$ 和 $n = 2$ 时为\verb|真|,但 $n = 3$ 和 $n = 4$ 时为\verb|假|,然后从 $n = 5$ 开始又都为\verb|真|。我们该如何证明这些情况呢?我们可以分别验证 $n = 1,2,3,4$ 的情况,然后用归纳法证明 $n \ge 5$ 的所有其他情况。这种方法是可行的,因为集合 $\mathbb{N} - \{1, 2, 3, 4\}$ 同样是一个\emph{归纳集}。就像多米诺骨牌一样,我们可以跳过前几个,从 $n = 5$ 开始推倒,剩下的骨牌就会按照预期的方式依次倒下。

实际上,我们甚至可以将这种讨论扩展到\emph{负数}。想象一下,在数轴上向左滑动,我们有一排从 $-3$ 开始编号的多米诺骨牌。从 $n = -3$ 开始推倒这些骨牌,它们会以类似之前的方式依次倒下。

核心思想是,我们有一条向右无限延伸、没有间隙的多米诺骨牌队列。这条队列的第一个骨牌被标记的具体数字并不重要,这样一来,无论如何编号,这些骨牌最终都会依次倒下。这正是下一个定理所要表达的核心理念。

\begin{theorem}[任意基本情况的归纳]\label{theorem5.3.1}
    设 $P(n)$ 为变量命题。令 $M \in \mathbb{Z}$ 是任意且固定的。

    设 $S = \{z \in \mathbb{Z} \mid z \ge M\}$。

    假设
    \begin{enumerate}[label=(\arabic*)]
        \item $P(M)$ 成立
        \item $\forall k \in S \centerdot P(k) \implies P(k + 1)$ 成立
    \end{enumerate}

    则 $\forall n \in S \centerdot P(n)$ 成立。
\end{theorem}

这个定理正是我们要讨论的内容:如果我们想证明某个命题对所有大于或等于某个特定值(定理中的 $M$)的数都成立,我们可以从这个特定值开始应用归纳法。我们把这个值设为\textbf{基本情况(BC)},然后对所有大于或等于这个值的数应用\textbf{归纳假设(IH)}和\textbf{归纳步骤(IS)}。除此之外,其他部分的处理方式与常规归纳法完全相同。

\subsubsection*{严格证明}

为了更好地说明和确保理解的完整性,我们将严格地\emph{证明}这个定理。我们希望之前的讨论---特别是多米诺骨牌的比喻---能帮助你直观地理解这个过程。虽然掌握这个证明的过程不会立竿见影地提升你使用归纳法的能力,但我们相信,阅读并尝试理解这个过程将帮助你更深入地掌握归纳法以及证明技巧,也会让你对这里涉及的数学有更深的感悟。具体而言,我们将利用数学归纳原理(PMI)来证明这个归纳法的变体!

\begin{proof}
    令 $P(n)$ 为变量命题。设 $M \in \mathbb{Z}$ 是任意且固定的。

    设 $S = \{z \in \mathbb{Z} \mid z \ge M\}$。

    假设
    \begin{enumerate}[label=(\arabic*)]
        \item $P(M)$ 成立
        \item $\forall k \in S \centerdot P(k) \implies P(k + 1)$ 成立
    \end{enumerate}

    我们的目标是证明 $\forall n \in S \centerdot P(n)$ 成立。

    定义命题 $Q(n)$
    \[Q(n) \iff P(n+M-1)\]
    请注意,通过代数运算,我们可以得到如下不等式
    \[n \ge 1 \iff n+M-1 \ge M\]
    这意味着我们将目标转换为证明 $\forall n \in \mathbb{N} \centerdot Q(n)$ 成立。(这样做将证明 $\forall n \in S \centerdot P(n)$。)
    
    我们通过对 $n$ 采用归纳法来证明这一点。

    \textbf{基本情况}:根据假设,我们知道 $P(M)$ 成立。 注意 $n + M - 1 = M \iff n = 1$。这意味着 $Q(1)$ 成立。

    \textbf{归纳假设}:设 $k \in \mathbb{N}$ 是任意固定的,假设 $Q(k)$ 成立。

    \textbf{归纳步骤}:由于 $Q(k)$ 成立,我们知道 $P(k + M - 1)$ 成立。

    又由于 $k \in \mathbb{N}$,我们知道 $k \ge 1$。因此,$k + M - 1 \ge M$。

    因此,根据假设条件(2),我们可以推导出 $P((k+M-1)+1)$ 成立,即 $P(k + M)$ 成立。

    这告诉我们 $Q(k + 1)$ 成立。

    根据数学归纳原理,我们推导出 $\forall n \in \mathbb{N} \centerdot Q(n)$ 成立。

    因此,根据 $Q(n)$ 的定义,我们得到 $\forall n \in S \centerdot P(n)$ 成立。
\end{proof}

正如我们所提到的,尽量去理解这个证明的具体细节,但总体上,你只需要记住这样一个直观的概念:我们只是在``移动''基本情况的起点。归纳过程的基本原理是相同的。

\subsubsection*{示例}

让我们来看看这种改进版证法在实际中的应用。事实上,我们接下来展示的例子正是我们在介绍这个方法时提到的情况:某个命题在一些较小的数值上是成立的,对于另一些较小的数值则不成立,但从某个特定点之后,对于所有数值都是成立的。\\

\begin{example}[比较 $2^n$ 与 $n^2$ 的大小]

    \textbf{声明}:
    \[2^n > n^2 \iff n \in \{0,1\} \cup \{z \in \mathbb{N} \mid z \ge 5\}\]
    也就是说,只有整数 $z=0,1,5,6,7,\dots$ 时 $2^n > n^2$。
\end{example}

(我们将把如何构思出这样一个命题的过程留给你去探索和尝试。通常情况下,如你在本节的练习中所见,这类不等式问题可能会附带一个问题:``这个命题对于哪些 $n$ 成立?''在这种情况下,你需要先进行一些初步的推理工作来确定你的命题,然后才能开始使用归纳法进行证明。)

\begin{proof}
    设 $P(n)$ 为命题 $2^n > n^2$。

    首先,考察如下情况:
    \begin{align*}
        & 2^0 > 0^2 \iff 1>0 & \text{所以 } P(0) \text{ 为真}\\
        & 2^1 > 1^2 \iff 2>1 & \text{所以 } P(1) \text{ 为真}\\
        & 2^2 > 2^2 \iff 4>4 &  \text{所以 } P(2) \text{ 为假}\\
        & 2^3 > 3^2 \iff 8>9 &  \text{所以 } P(3) \text{ 为假}\\
        & 2^4 > 4^2 \iff 16>16 & \text{所以 } P(4) \text{ 为假}\\
    \end{align*}
    注意,当 $z \le -1$ 时,我们有 $2^z < 1$ 且 $z^2 \ge 1$,所以 $2^z \ngtr z^2$。因此对于所有 $n \le -1, P(n)$ 为\verb|假|。

    接下来,定义 $S$ 为集合 $S = \{z \in \mathbb{N} \mid z \ge 5\}$。

    我们要在 $n$ 上应用归纳法证明 $\forall n \in S \centerdot P(n)$ 成立。

    \textbf{基本情况}:不难发现 $P(5)$ 成立,因为 $2^5=32$ 且 $5^2=25$,显然 $32 > 25$。

    \textbf{归纳假设}:设 $k \in \mathbb{N}$ 是任意固定的,假设 $P(k)$ 成立。

    \textbf{归纳步骤}:因为 $k \in S$,我们知道 $k \ge 5$ 或 $k > 4$。
    
    因此 $k-1>3$ 所以 $(k-1)^2>9$,自然 $(k-1)^2>2$。

    考察如下不等式处理:
    \begin{align*}
        (k-1)^2 > 2 &\implies (k-1)^2-2>0 \\
        &\implies k^2-2k-1>0 \\
        &\implies k^2>2k+1 \\
        &\implies 2k^2>k^2+2k+1 \\
        &\implies 2k^2>(k+1)^2 \\
    \end{align*}

    因为我们知道第一个不等式成立,我们可以推导出上面最后一个不等式成立。

    (注:如果你还没有注意到,这一连串的推理其实是 \ref{sec:section4.9.9} 节练习 \ref{ex:exercises4.9.2} 的解答!为了解答这个问题,我们进行了一些初步的探索,从所需证明的不等式出发,然后``逆向操作''直至找到一个显而易见的真理。在这里的书写中,我们从那个显然的事实出发,逐步推导至期望的结论。)

    根据归纳假设 $P(k)$,我们知道 $k^2 < 2^k$,这告诉我们
    \[2k^2 < 2 \cdot 2^k = 2^{k+1}\]
    应用不等式的传递性,我们可以推出
    \[(k + 1)^2 < 2k^2 < 2^{k+1}\]
    所以 $P(k+1)$ 成立。

    根据数学归纳原理,$\forall n \in S \centerdot P(n)$ 成立。

    综上,我们考虑了每个 $z \in \mathbb{Z}$。我们观察到 $P(z)$ 在 $z \le -1$ 时为\verb|假|,在 $z = 0, 1$ 时为\verb|真|,在 $z = 2, 3, 4$ 时为\verb|假|,在 $z \ge 5$ 时为\verb|真|。上述结果共同证明了该声明。
\end{proof}

这个证明其实相当复杂。你有没有注意到,我们的命题是用``当且仅当''来表述的,因此我们在证明过程中必须考虑所有整数?这确实很有挑战性,但我们成功了!

\subsection{反向归纳}

这种归纳法的变体特别适用于当命题 $P(n)$ 对于某个特定值之前的所有 $n$ 都成立的情况。如果用多米诺骨牌做比喻,这就像是想象我们的无限长的多米诺骨牌向左延伸,而不是向右。正如前一节所讨论的,多米诺骨牌的编号方式其实并不重要。现在,我们还可以看到它们向哪个方向倒下也无关紧要;它们都将遵循同样的原理!下面的定理总结了这一观察。

\begin{theorem}[反向归纳]\label{theorem5.3.3}
    设 $P(n)$ 为变量命题。令 $M \in \mathbb{Z}$ 为任意固定的。

    设 $S = {z \in \mathbb{Z} \mid z \le M}$。

    假设
    \begin{enumerate}[label=(\arabic*)]
        \item $P(M)$ 成立
        \item $\forall k \in S \centerdot P(k) \implies P(k - 1)$ 成立
    \end{enumerate}

    则 $\forall n \in S \centerdot P(n)$ 成立。
\end{theorem}

注意该定理与定理 \ref{theorem5.3.1} 的区别。

\subsubsection*{严格证明}

在我们目前的进展中,我们已经有足够的信心让你来证明一些重要的定理了。具体来说,我们希望你证明上面提到的这个改进版的数学归纳原理,即定理 \ref{theorem5.3.3}!我们希望你亲自动手处理这些细节,而不是仅仅看我们为你提供的演示,从长远来看这对你更加有益。此外,我们想到的这个证明的细节与我们之前给你展示的定理 \ref{theorem5.3.1}(在第 \ref{sec:section5.3.1} 节)的证明细节非常相似。

在数学书中,将证明留作``读者练习''是非常常见的做法。我们这样做是为了帮助你逐渐适应这种现象!$\smiley{}$

\begin{proof}
    留给读者作为 \ref{sec:section5.3.4} 节的练习 \ref{exc:exercises5.3.1}。
\end{proof}

我们不会展示这种方法的实操示例,因为我们认为它与我们已经见过的标准归纳法没有区别。实际上,如果你仔细分析了上面的证明细节,你甚至可以看出如何通过稍作修改我们已经见过的一些例子来为这一节构思一个例子!(比如我们反转一个不等式……)

\subsection{奇偶归纳}

让我们从一个观察开始引出这一节的内容,这将带领我们进入这种方法的第一个示例应用。考虑这样一个完全平方数序列:
\[1, 4, 9, 16, 25, 36, 49, 64, 81, 100, 121, 144, \dots\]
看看当我们将这些数除以 $8$ 时会发生什么;特别是观察余数(每种情况下分数的分子表示余数):
\[0+\frac{1}{8}, 0+\frac{4}{8}, 1+\frac{1}{8}, 2+\frac{0}{8}, 3+\frac{1}{8}, 4+\frac{2}{8}, 6+\frac{1}{8},  \dots\]
注意我们保留了像 $\frac{4}{8}$ 和 $\frac{2}{8}$ 这样未简化的分数,保持分母为 $8$,以表示余数。这些余数遵循以下模式:
\[1, 4, 1, 0, 1, 2, 1, \dots\]
看起来每隔一个余数就是 $1$。实际上,当我们将一个奇数的平方除以 $8$ 时,余数似乎总是 $1$。这很有趣!你可能会好奇这种模式是否会持续下去。探索这个想法的一个合理方式是直接尝试通过归纳法证明这个命题,并看一下结果如何。如果证明成功,那么我们就成功地发现并证明了一个事实。如果证明失败,我们可能能找出失败的原因。这是进行数学发现的一个很好的通用建议:如果你想验证某事是否为\verb|真|,不妨尝试去证明它,看看会发生什么!

\subsubsection*{示例}

在继续阅读之前,试着自己先仔细研究一下这个问题的细节。在这个过程中,你需要弄清楚如何仅对奇数进行归纳,而不是像我们之前那样对所有自然数进行归纳。我们将会展示这个命题的证明,并在之后讨论这种方法的工作原理,但你绝对应该先自己尝试解决这个问题!…… \\

\begin{example}[奇数平方和除以 $8$ 的余数]

   \textbf{声明}:设 $O$ 为奇数集,即
   \[\O = \{n \in \mathbb{N} \mid \exists m \in \mathbb{N} \cup \{0\} \centerdot n = 2m+1\}\]
   设 $P(n)$ 为命题``$n^2$ 比 $8$ 的倍数大 $1$'',则
   \[\forall n \in O \centerdot P(n)\]
\end{example}

\begin{proof}
    设 $P(n)$ 如题目定义,我们通过对 $n$ 应用归纳法证明 $\forall n \in O \centerdot P(n)$。

    \textbf{基本情况}:不难发现 $1^2=1$ 且 $1=0 \cdot 8 + 1$ (即 $1$ 比 $8$ 的倍数大 $1$) ,因此 $P(1)$ 成立。

    \textbf{归纳假设}:设 $k \in O$ 是任意固定的,假设 $P(k)$ 成立。

    \textbf{归纳步骤}:我们的目标是推导出 $P(k+2)$ 成立(这是因为 $k+2$ 是 $k$ 之后的下一个奇数)。

    因为 $k+2$ 为奇数,根据假设,我们知道 $\exists m \in \mathbb{N} \cup \{0\} \centerdot k = 2m+1$。给定此 $m$。

    根据归纳假设,我们知道 $\exists \ell \in N \centerdot l^2=8\ell+1$。给定此 $\ell$。

    利用上面的条件可得
    \begin{align*}
        (k + 2)^2 &= k^2 + 4k + 4 \\
        &= (8\ell + 1) + 4(2m + 1) + 4 \\
        &= 8\ell + 8m + 8 + 1 \\
        &= 8(\ell + m + 1) + 1 \\
    \end{align*}
    因为 $\ell, m \in \mathbb{Z}$,我们知道 $\ell+m \in \mathbb{Z}$。因此 $(k+2)^2$ 比 $8$ 的倍数大 $1$。所以 $P(k + 2)$ 成立。

    根据归纳法,$P(n)$ 对于所有 $n \in O$ 成立。
\end{proof}

\emph{后续问题}:你能否证明当偶数的平方除以 $8$ 时,其余数不为 $1$?(这会使该命题成为一个\emph{当且仅当}陈述。)你能发现这些偶数平方的余数有什么规律吗?你能证明你的观察吗?

(提示:你可能不需要使用归纳法来证明这些命题!)

\subsubsection*{方法讨论}

让我们探讨一下为什么这种方法有效。其基本原理与我们之前看到的其他归纳法完全一样。唯一的不同在于归纳步骤。由于奇数``间隔为2'',我们的目标是证明:
\[\forall k \in O \centerdot P(k) \implies P(k + 2)\]
这体现了与标准归纳法相同的思想:取命题的一个实例,并用它来推导``下一个''实例的成立。这里的不同之处在于``下一个''的定义。为了完整性,我们将给出一个描述这种方法的定理。再次强调,我们将把证明的具体细节留给你来完成。

\begin{theorem}[奇数上的归纳]\label{theorem5.3.5}
    设 $O$ 为奇数集,

    设 $P(n)$ 为变量命题。假设

    \begin{enumerate}[label=(\arabic*)]
        \item $P(1)$ 成立
        \item $\forall k \in O \centerdot P(k) \implies P(k + 2)$ 成立
    \end{enumerate}

    则 $\forall n \in O \centerdot P(n)$ 成立。
\end{theorem}

\begin{proof}
    留给读者作为 \ref{sec:section5.3.4} 节的练习 \ref{exc:exercises5.3.2}。
\end{proof}

同理,我们可以发现对偶数进行归纳同样有效。这里定理阐述了对偶数进行归纳。再次,我们把具体的证明过程留给你。

\begin{theorem}[偶数上的归纳]\label{theorem5.3.6}
    设 $E$ 为偶数集,

    设 $P(n)$ 为变量命题。假设

    \begin{enumerate}[label=(\arabic*)]
        \item $P(2)$ 成立
        \item $\forall k \in O \centerdot P(k) \implies P(k + 2)$ 成立
    \end{enumerate}

    则 $\forall n \in E \centerdot P(n)$ 成立。
\end{theorem}

\begin{proof}
    留给读者作为 \ref{sec:section5.3.4} 节的练习 \ref{exc:exercises5.3.2}。
\end{proof}


\subsubsection*{组合和修改这些方法}

假设我们有一个命题 $P(n)$,我们想要证明这个命题对于所有的自然数 $n$ 都成立。这个命题及其背后的理论可能相当复杂,使得我们无法用传统的归纳法来证明它。这可能是由于某种代数技巧的需要,或者我们找不到一种高效的证明方法,又或者是有一些深层的原因使得我们无法这样做。不管是什么原因,我们可以采用一些新型的归纳法,将证明分成几个部分,从而证明对所有 $n \in \mathbb{N}$,命题 $P(n)$ 都成立。

这些新型的方法可以被看作是``跳跃式''归纳法。例如,证明命题对每一个奇数成立的方法本质上和传统的归纳法相同,只是在归纳步骤中我们跳过了偶数。同样的方法也适用于偶数的证明(虽然我们会稍微调整一下基本情况,因为 $2$ 是第一个偶数,不是 $1$)。如果我们先用``奇数''方法证明,然后再用``偶数''方法证明,我们就可以证明这个命题对所有自然数都成立。

下面的例子正是采用了这种方法,但你会注意到它实际上是以 $3$ 为步长进行``跳跃''(而不是像奇偶归纳那样以 $2$ 为步长)。我们这里不会具体陈述和证明这些方法的定理(也不会要求你这么做)。此时,我们更依赖于对归纳法运作方式的直觉,这些定理和证明与我们之前见过的非常相似。如果你想要练习,或者想要为你的笔记和记录保留这些内容,尽管去陈述和证明我们即将使用的方法的定理吧!\\

\begin{example}[ $2$ 的幂与 $7$ 的倍数]
    
    \textbf{声明}:对于所有自然数 $n \in \mathbb{N}, 2^n+1$ \emph{不是} $7$ 的倍数。
\end{example}

(这里,我们建议做一些探索性的计算,来找出当表达式 $2^n + 1$ 除以 $7$ 的余数的规律。你会发现这些余数形成了一个长度为 $3$ 的循环。真是太棒了!这实际上就是我们这里要证明的内容;只不过最初这个命题并没有以这种方式提出,因此我们需要做一些额外的工作,重新整理这个命题并设计出一个证明方法。)

\begin{proof}
    定义集合 $A_1, A_2, A_3$ 为:
    \begin{align*}
        A_1 &= \{n \in \mathbb{N} \mid \exists m \in \mathbb{N} \cup \{0\} \centerdot n = 3m + 1\} = \{1, 4, 7, 10, \dots \} \\
        A_2 &= \{n \in \mathbb{N} \mid \exists m \in \mathbb{N} \cup \{0\} \centerdot n = 3m + 2\} = \{2, 5, 8, 11, \dots \} \\
        A_3 &= \{n \in \mathbb{N} \mid \exists m \in \mathbb{N} \cup \{0\} \centerdot n = 3m \quad\:\:\:\} = \{3, 6, 9, 12, \dots \} \\
    \end{align*}
    (也就是说,这三个集合根据除以 $3$ 时的余数对 $\mathbb{N}$ 进行划分。)

    设 $P(n)$ 为命题``$2^n+1$ 不能被 $3$ 整除''。我们要通过归纳法证明 $\forall n \in \mathbb{N} \centerdot P(n)$ 成立。

    定义命题 $Q(n), R(n), S(n)$ 如下:
    \begin{align*}
        Q(n) \text{ 为 } \exists \ell \in \mathbb{N} \cup \{0\} \centerdot 2^n + 1 = 7\ell + 3 \\
        R(n) \text{ 为 } \exists \ell \in \mathbb{N} \cup \{0\} \centerdot 2^n + 1 = 7\ell + 5 \\
        S(n) \text{ 为 } \exists \ell \in \mathbb{N} \cup \{0\} \centerdot 2^n + 1 = 7\ell + 2 \\
    \end{align*}
    不难发现
    \[\forall n \in \mathbb{N} \centerdot \big(Q(n) \lor R(n) \lor S(n)\big) \implies P(n)\]
    这是因为 $7$ 的倍数加 $3$ 不是 $7$ 的倍数,$7$ 的倍数加 $5$ 和加 $2$ 也不是。\\
    
    首先,我们通过对 $n$ 应用归纳法来证明 $\forall n \in A_1 \centerdot Q(n)$ 成立。

    \textbf{基本情况}:不难发现 $2^1+1=3 = 0 \cdot 7 + 3$,因此 $Q(1)$ 成立。

    \textbf{归纳假设}:设 $k \in A_1$ 是任意固定的,假设 $Q(k)$ 成立。

    \textbf{归纳步骤}:我们的目标是推导出 $Q(k+3)$ 成立。

    因为 $k \in A_1$,我们知道 $\exists m \in \mathbb{N} \centerdot k = 3m + 1$。给定这样的 $m$。

    根据归纳假设,我们有 $\exists \ell \in \mathbb{N} \centerdot 2^k + 1 = 7\ell + 3$。给定这样的 $\ell$,这意味着 $2^k = 7\ell + 2$。

    我们能够推导出
    \[2^{k+3} = 2^3 \cdot 2^k = 8 \cdot (7\ell + 2) = 56\ell + 16\]
    因此
    \[2^{k+3} + 1 = 56\ell + 17 = 7(8\ell) + 14 + 3 = 7(8\ell + 2) + 3\]
    所以 $Q(k + 3)$ 也成立。因此 $\forall n \in A_1 \centerdot Q(n)$。\\

    接着,我们通过对 $n$ 应用归纳法来证明 $\forall n \in A_2 \centerdot R(n)$ 成立。

    \textbf{基本情况}:不难发现 $2^2+1=5 = 0 \cdot 7 + 5$,因此 $R(2)$ 成立。

    \textbf{归纳假设}:设 $k \in A_2$ 是任意固定的,假设 $R(k)$ 成立。

    \textbf{归纳步骤}:我们的目标是推导出 $R(k+3)$ 成立。

    根据归纳假设,我们有 $\exists \ell \in \mathbb{N} \centerdot 2^k + 1 = 7\ell + 5$。给定这样的 $\ell$,这意味着 $2^k = 7\ell + 4$。

    我们能够推导出
    \[2^{k+3} = 2^3 \cdot 2^k = 8 \cdot (7\ell + 4) = 56\ell + 32\]
    因此
    \[2^{k+3} + 1 = 56\ell + 33 = 7(8\ell) + 28 + 5 = 7(8\ell + 4) + 5\]
    所以 $R(k + 3)$ 也成立。因此 $\forall n \in A_2 \centerdot R(n)$。\\

    最后,我们通过对 $n$ 应用归纳法来证明 $\forall n \in A_3 \centerdot S(n)$ 成立。

    \textbf{基本情况}:不难发现 $2^3+1=9 = 1 \cdot 7 + 2$,因此 $S(3)$ 成立。

    \textbf{归纳假设}:设 $k \in A_3$ 是任意固定的,假设 $S(k)$ 成立。

    \textbf{归纳步骤}:我们的目标是推导出 $S(k+3)$ 成立。

    根据归纳假设,我们有 $\exists \ell \in \mathbb{N} \centerdot 2^k + 1 = 7\ell + 2$。给定这样的 $\ell$,这意味着 $2^k = 7\ell + 1$。

    我们能够推导出
    \[2^{k+3} = 2^3 \cdot 2^k = 8 \cdot (7\ell + 1) = 56\ell + 8\]
    因此
    \[2^{k+3} + 1 = 56\ell + 9 = 7(8\ell) + 7 + 2 = 7(8\ell + 1) + 2\]
    所以 $S(k + 3)$ 也成立。因此 $\forall n \in A_3 \centerdot S(n)$。\\

    综上,我们证明了对于每个自然数,$Q(n)$ 或 $R(n)$ 或 $S(n)$ 三者必有一个成立(取决于数字除以 $3$ 的余数)。因此,每个自然数都具有 $2^n + 1$ 不是 $7$ 的倍数的性质。
\end{proof}

实际上,我们证明中得出的结论比声明中提出的\emph{更强}。我们不仅证明了没有一个形如 $2^n + 1$ 的数是 $7$ 的倍数,还准确解释了这些数为何不是 $7$ 的倍数。

在本节的练习中,我们设计了一些练习,通过识别``跳跃''和声明来引导你进行类似的证明。在 \ref{sec:section5.7} 本章的练习中,我们还包含了一些可能需要这种论证的问题(但我们不会像这里那样告诉你论证的整体结构)。

值得注意的是,只要你想进行的``跳跃''遵循某种易于识别的模式,你可以很容易地将这些方法应用到任何情况中。在前面的例子中,我们进行了步长为 $3$ 的跳跃,因此将所有自然数分成三个集合,并在这些集合中跳跃。基本上,这依赖于我们有一个``公式''来获取命题的``下一个''实例:从 $P(k)$ 开始,并尝试推导 $P(k + 3)$。你可以设想进行步长为 $4$ 或 $10$ 的跳跃,甚至进行数值加倍的跳跃;也就是说,你可以证明某个命题 $P(n)$ 对于每个 $2$ 的 $n$ 次幂成立,即,
\[P(1) \text{ 成立,且 } \forall n \in \mathbb{N} \centerdot P(n) \implies P(2^n)\]

再次强调,所有这些都依赖于某种``公式''或``规则''告诉我们\emph{下一个}要考虑的实例是什么。因此,\emph{我们无法在所有素数集合上进行归纳}。如果你试图证明某个事实对每个素数成立,不要指望使用归纳法!你必须有某种``规则''给出,``如果 $k$ 是一个素数,那么下一个素数是……''。如果你知道这样的规则,数学界会非常愿意聆听你的见解!这将回答许多关于素数的未解之谜,并且你将成为历史上最著名的数学家。没开玩笑!

\subsection{问题和练习}\label{sec:section5.3.4}

口头或书面简要回答以下问题。这些题目全都基于你刚刚阅读的部分,因此如果你无法想起特定的定义、概念或示例,请返回重新阅读相应部分。确保自己在继续之前可以自信地回答这些问题,这将有助于你的理解和记忆!

\begin{enumerate}[label=(\arabic*)]
    \item 多米诺骨牌类比如何描述一个基本情况不是 $1$ 的归纳法证明?
    \item 提供一个证明模板,用于证明一个命题 $P(n)$,该命题对所有大于或等于 $7$ 的奇数都成立。
    \item 为什么我们不能``对素数进行归纳''?
\end{enumerate}

\subsubsection*{试一试}

尝试回答以下简答题。这些题目要求你实际动笔写一写,或(对朋友/同学)口头描述一些东西。目的是让你练习使用新概念、定义和符号。别担心,这些题本来就很简单。确保能够解决这些问题将对你有所帮助!

\begin{enumerate}[label=(\arabic*)]
    \item 证明定理 \ref{theorem5.3.3}。 \label{exc:exercises5.3.1}
    \item 证明定理 \ref{theorem5.3.5} 和定理 \ref{theorem5.3.6}。 \label{exc:exercises5.3.2}
    \item 提出一个定理,该定理描述了如何对所有 $5$ 的倍数进行归纳推理的方法。并证明这个定理。
    \item 考虑不等式 $n^3 < 3^{n-1}$。
        \begin{enumerate}[label=(\alph*)]
            \item 证明对于所有 $n \ge 6$ 该不等式都成立。
            \item 证明对于所有 $\{1,2,3,4,5\}$ 该不等式都不成立。(这一问很简单)
            \item 证明对于所有 $n \le 0$ 该不等式都成立。
        \end{enumerate}
    \item 定义数列
        \[x_1 = 2, x_2 = 2, \forall n \in \mathbb{N} - \{1, 2\} \centerdot x_n = x_{n-2} + 1\]
        设 $P(n)$ 为命题
        \[x_n = \frac{1}{2}(n+1)+\frac{1}{4}(1+(-1)^n)\]
        \begin{enumerate}[label=(\alph*)]
            \item 设 $O$ 为奇数集。用归纳法证明 $\forall n \in O \centerdot P(n)$。
            \item 设 $E$ 为偶数集。用归纳法证明 $\forall n \in E \centerdot P(n)$。
        \end{enumerate}
    \item 考虑下列声明
        \[\sum_{k=1}^{n} (-1)^{k-1}k^2 = (-1)^{k-1}\sum_{k=1}^{n} k\]
        也就是说,我们声明
        \[1^2 - 2^2 + 3^2 - 4^2 + \dots + (1)^{n-1}n^2 = (-1)^{n-1}(1 + 2 + 3 + \dots + n)\]
        对于所有 $n \in \mathbb{N}$ 成立。
        \begin{enumerate}[label=(\alph*)]
            \item 证明上面公式对于 $n=1$ 和 $n=2$ 成立。
            \item 证明上面公式只要对于 $k$ 成立,则 $k+2$ 也成立。
            \item 直观地解释为什么(a)和(b)证明了该声明。
        \end{enumerate}
\end{enumerate}
