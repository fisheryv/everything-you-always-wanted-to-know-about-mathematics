% !TeX root = ../../../book.tex
\subsection{比较``常规''归纳法与强归纳法}

我们需要重申介绍强归纳法时强调的关键点,这个重要经验值得反复强调:

\begin{quotation}
    \emph{当需要通过归纳法证明命题时,应当优先选用强归纳法。}
\end{quotation}

原因在于常规归纳法与强归纳法本质等价:二者可以相互推导。使用强归纳假设并无不妥,因为它在逻辑上是成立的。在证明过程中,你往往无法预知归纳步骤需要引用\emph{哪些}或\emph{多少}先前的归纳假设。若采用较弱的假设,却发现依赖了未经证明的``事实'',将遗憾的造成缺陷!反之,采用最强的假设才能有备无患。尽管这看似有些小题大作(例如实际只需 $P(k)$ 推导 $P(k+1)$),但这无关紧要——证明命题才是最终目标。

随着数学能力的提升,你将更敏锐地区分常规归纳与强归纳的论证场景。特别当处理递归定义的数列时,往往需要强归纳法,但该方法的应用范围远不止于此。分析命题实例间的依赖关系至关重要:若发现某个实例依赖多个先前实例,则几乎可以肯定需要强归纳论证。

\clearpage