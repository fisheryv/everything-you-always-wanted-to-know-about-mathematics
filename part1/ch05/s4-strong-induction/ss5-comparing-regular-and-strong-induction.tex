% !TeX root = ../../../book.tex
\subsection{比较``常规''归纳法和强归纳法}

我们想再次强调一下之前介绍强归纳法时提到的一个关键点。这是一个重要的教训,值得在这里重申:

\begin{quotation}
    \emph{每当我们需要通过归纳法来证明某个命题时,我们最好总是使用强归纳法。}
\end{quotation}

原因在于,常规归纳法和强归纳法是互相包含的;每一种方法都意味着另一种方法。在进行归纳证明时,使用强归纳假设``没有坏处'',因为我们知道这样做是可以的。在进行证明时,你可能无法预见归纳步骤中需要引用\emph{哪个}或\emph{多少个}归纳假设。如果做了一个较弱的假设,却发现自己引用了那些未正式证明的``真理'',那将是非常遗憾的!相反,你不妨做出最强的假设,以备不时之需。尽管这可能显得有些小题大作(例如你实际上只需要 $P(k)$ 来推导 $P(k+1)$),但这并不重要,对吧?关键是要证明当前的命题,只要达到了这个目的,你就成功了。

随着你在数学领域的进步,你可能会更好地识别常规归纳和强归纳论证的区别。特别是,你可能会注意到什么时候确实需要强归纳法。通常,这种情况发生在处理递归定义的数列时,但也可能出现在许多其他地方。当你尝试解决一个问题时,看看你的命题实例之间存在什么样的依赖关系。如果你注意到一个实例依赖于多个之前的实例,你几乎百分百需要使用强归纳论证。
