% !TeX root = ../../../book.tex
\subsection{引言}

回顾一下 \ref{sec:section2.4} 节的例子。在那里,我们考察了用多米诺骨牌密铺 $2 \times n$ 矩形棋盘的方法数量,并且我们玩了取走游戏。在处理这些例子的归纳论证时,我们发现情况与之前的归纳论证略有不同。当我们证明类似于
\[\sum_{k=1}^{n} \frac{n(n+1)}{2}\]
对于每个 $n \in \mathbb{N}$ 都成立,我们可以在归纳步骤中引用前一个情况并调用归纳假设,如下所示:
\[\sum_{k=1}^{n+1} k=(n+1)+\sum_{k=1}^{n}k = n+1+\frac{n(n+1)}{2} = \frac{(n+1)(n+2)}{2} \]
当然,我们当时没有把这些部分称为``\textbf{归纳假设(IH)}''或``\textbf{归纳步骤(IS)}'',但这正是我们接下里要做的事情。

然而,当我们思考多米诺密铺的例子时,我们发现需要引用前两个实例的事实。具体来说,要找到一个 $2 \times n$ 棋盘的密铺数量,我们不仅需要知道 $2 \times (n-1)$ 棋盘的密铺数量,还需要知道 $2 \times (n-2)$ 棋盘的密铺数量。这本质上是不同的!归纳论证的什么特性让我们能够这样做?这如何遵循我们描述的``多米诺类比''?或者``猴子 Mojo 类比''?它是否真的遵循?

当我们思考取走游戏时,情况甚至``更为''不同,不是吗?在制定玩家 $2$ 的获胜策略时,我们注意到玩家 $2$ 应该在另一堆上模仿玩家 $1$ 的动作。也就是说,如果玩家 $1$ 从左边的堆中移走 $3$ 块石头,那么玩家 $2$ 应该从右边的堆中移走 $3$ 块石头,以保证胜利。这在无论玩家 $1$ 移走多少石头的情况下都成立。从这个意义上讲,我们需要玩家 $2$ 在任意大小的堆(包括 $n$)上都有必胜策略,以保证玩家 $2$ 在大小为 $n + 1$ 的堆上有获胜策略。这需要在我们的归纳假设中包含很多假设。我们怎么知道我们可以这样做?
