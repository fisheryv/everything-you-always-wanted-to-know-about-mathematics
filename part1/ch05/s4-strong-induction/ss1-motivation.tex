% !TeX root = ../../../book.tex
\subsection{引言}

回顾第 \ref{sec:section2.4} 节的例子。我们曾探讨用多米诺骨牌铺满 $2 \times n$ 矩形棋盘的方法数量,并分析了取走游戏。在归纳论证中,这些案例与先前有所不同。例如,证明
\[\sum_{k=1}^{n} \frac{n(n+1)}{2}\]
对于每个 $n \in \mathbb{N}$ 都成立,我们可在归纳步骤中引用前一情形并应用归纳假设:
\[\sum_{k=1}^{n+1} k=(n+1)+\sum_{k=1}^{n}k = n+1+\frac{n(n+1)}{2} = \frac{(n+1)(n+2)}{2} \]
尽管当时并未明确标注``\textbf{归纳假设 (IH)}''或``\textbf{归纳步骤 (IS)}'',但这正是我们接下里要做的事情。

然而,在多米诺密铺的例子中,我们需要引用前两个实例:计算 $2 \times n$ 棋盘的密铺方案数时,不仅依赖 $2 \times (n-1)$ 棋盘的密铺方案数,还需要 $2 \times (n-2)$ 棋盘的密铺方案数。这与常规归纳法存在本质差异!归纳论证的何种特性允许我们这样做?它如何契合``多米诺类比''或``猴子 Mojo 类比''类比?是否真正符合?

在取走游戏中,差异更为显著。设计玩家 $2$ 的必胜策略时,玩家 $2$ 需要在另一堆上模仿玩家 $1$ 的操作:若玩家 $1$ 从左堆取走 $3$ 块石头,玩家 $2$ 则从右堆取走 $3$ 块石头以确保胜利。无论玩家 $1$ 取走多少石头,此策略均成立。因此,要证明玩家 $2$ 在大小为 $n + 1$ 的堆上必胜,需要假设其对任意大小不超过 $n$ 的堆均有必胜策略。这要求归纳假设覆盖多种情形——我们如何确认其合理性?
