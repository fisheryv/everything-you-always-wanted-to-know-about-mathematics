% !TeX root = ../../../book.tex
\subsection{习题}

\subsubsection*{温故知新}

以口头或书面的形式简要回答以下问题。这些问题全都基于你刚刚阅读的内容,所以如果忘记了具体的定义、概念或示例,可以回去重读相关部分。确保在继续学习之前能够自信地回答这些问题,这将有助于你的理解和记忆!

\begin{enumerate}[label=(\arabic*)]
    \item 强归纳法和普通归纳法的区别是什么?
    \item 你如何判断何时需要使用强归纳法?
    \item 为什么我们总是可以选择使用强归纳法,而不是在常规归纳法和强归纳法之间做选择?
    \item 在质因数分解的例子中,使用的归纳假设(IH)有什么特别之处?与我们在其他例子中证明递归定义的数列公式相比,有什么不同?
\end{enumerate}

\subsubsection*{小试牛刀}

尝试回答以下问题。这些题目要求你实际动笔写下答案,或(对朋友/同学)口头陈述答案。目的是帮助你练习使用新的概念、定义和符号。题目都比较简单,确保能够解决这些问题将对你大有帮助!

\begin{enumerate}[label=(\arabic*)]
    \item 定义数列
        \[x_1=2 \;\text{ 且 }\; x_2=3 \;\text{ 且 }\; \forall n \in \mathbb{N}-\{1,2\} \centerdot x_n=3x_{n-1}-2x_{n-2}\]
        证明
        \[\forall n \in \mathbb{N} \centerdot x_n = 2^{n-1} + 1\]
    \item 数列 $a_n$ 定义为 $a_0 = 0, a_1 = 1$ 且
        \[\forall n \in \mathbb{N}-\{1\} \centerdot a_n=5x_{n-1}-6x_{n-2}\]
        即 $\langle a_n \rangle = \langle 0, 1, 5, 19, 65, 211, \dots \rangle$。
        
        证明对于所有 $n \in \mathbb{N} \cup \{0\}, a_n = 3^n-2^n$。
    \item 设 $a_1 \in \mathbb{Z}$ 是任意且固定的。定义数列
        \[\forall n \in \mathbb{N} - \{1\} \centerdot a_n = \sum_{k=1}^{n-1}k^2a_k\]
        证明
        \[a_1 \text{ 为偶数} \implies \forall n \in \mathbb{N} \centerdot a_n \text{ 为偶数}\]
    \item 定义数列 $\langle t_n \rangle$
        \[t_1=t_2=2 \;\text{ 且 }\; \forall n \in \mathbb{N}-\{1,2\} \centerdot t_n=\frac{1}{2t_{n-2}}(t_{n-1}-4)(t_{n-1}-6)\]
        证明 $\forall n \in \mathbb{N} \centerdot t_n=2$。
    \item 你之前可能见过\textbf{三角不等式};它说的是
        \[\forall x,y \in \mathbb{R} \centerdot |x+y| \le |x| + |y|\]
        (其中 $|x|$ 表示 $x$ 的绝对值)。证明该不等式不仅仅对 $2$ 个变量成立,而是对 $n$ 个变量都成立;也就是说,证明如果我们有实数 $x_i$ 组成的数列,即 $\forall i \in \mathbb{N} \centerdot x_i \in \mathbb{R}$,那么
        \[\forall i \in \mathbb{N} \centerdot \vert \sum_{i=1}^{n} x_i \vert \le \sum_{i=1}^{n} |x_i|\]
        (注意:要证明 $n = 2$ 的情况。而不能仅仅只是假设!)
    \item 回想一下 \ref{sec:section2.4.1} 节,我们曾讨论过如何用多米诺骨牌密铺 $2 \times n$ 的矩形棋盘。现在,我们来研究一个类似的问题,如何用条状三连块(即 $3 \times 1$ 的矩形块)密铺 $3 \times n$ 矩形棋盘。试着找出其中的归纳关系,并定义一个数列来表示密铺 $3 \times n$ 棋盘的方法数。

    (注意:不要尝试找到一个封闭形式或者证明它!这需要的技术超出了我们目前的讨论范围。如果你感兴趣,可以查找\textbf{递推关系}。如果愿意,可以尝试根据你读到的内容为这个问题找到一个封闭形式。你能通过归纳法证明它吗?)
\end{enumerate}