% !TeX root = ../../../book.tex
\subsection{习题}

\subsubsection*{温故知新}

以口头或书面的形式简要回答以下问题。这些问题全都基于你刚刚阅读的内容,如果忘记了具体定义、概念或示例,可以回顾相关内容。确保在继续学习之前能够自信地作答这些问题,这将有助于你的理解和记忆!

\begin{enumerate}[label=(\arabic*)]
    \item 强归纳法与常规归纳法有何区别?
    \item 如何判断何时需要使用强归纳法?
    \item 为什么总是可以选择使用强归纳法,而非在常规归纳法和强归纳法之间做选择?
    \item 在质因数分解的例子中,使用的归纳假设 (IH) 有何特别之处?与其他证明递归定义数列公式的例子相比,有何不同?
\end{enumerate}

\subsubsection*{小试牛刀}

尝试解答以下问题。这些题目需动笔书写或口头阐述答案,旨在帮助你熟练运用新概念、定义及符号。题目难度适中,确保掌握它们将大有裨益!

\begin{enumerate}[label=(\arabic*)]
    \item 定义数列
        \[x_1=2, \quad x_2=3, \quad \forall n \in \mathbb{N}-\{1,2\} \centerdot x_n=3x_{n-1}-2x_{n-2}\]
        证明
        \[\forall n \in \mathbb{N} \centerdot x_n = 2^{n-1} + 1\]
    \item 数列 $a_n$ 定义为 $a_0 = 0, a_1 = 1$ 且
        \[\forall n \in \mathbb{N}-\{1\} \centerdot a_n=5x_{n-1}-6x_{n-2}\]
        即 $\langle a_n \rangle = \langle 0, 1, 5, 19, 65, 211, \dots \rangle$。
        
        证明:对于所有 $n \in \mathbb{N} \cup \{0\}, a_n = 3^n-2^n$。
    \item 设 $a_1 \in \mathbb{Z}$ 为任意且固定整数。定义数列
        \[\forall n \in \mathbb{N} - \{1\} \centerdot a_n = \sum_{k=1}^{n-1}k^2a_k\]
        证明
        \[a_1 \text{\ 为偶数} \implies \forall n \in \mathbb{N} \centerdot a_n \text{\ 为偶数}\]
    \item 定义数列 $\langle t_n \rangle$
        \[t_1=t_2=2, \quad \forall n \in \mathbb{N}-\{1,2\} \centerdot t_n=\frac{1}{2t_{n-2}}(t_{n-1}-4)(t_{n-1}-6)\]
        证明 $\forall n \in \mathbb{N} \centerdot t_n=2$。
    \item 你之前可能见过\textbf{三角不等式};即
        \[\forall x,y \in \mathbb{R} \centerdot |x+y| \le |x| + |y|\]
        (其中 $|x|$ 表示 $x$ 的绝对值)。证明该不等式不仅对 $2$ 个变量成立,而且对 $n$ 个变量也成立;也就是说,证明对于实数 $x_i$ 组成的数列,即 $\forall i \in \mathbb{N} \centerdot x_i \in \mathbb{R}$,下面不等式成立
        \[\forall i \in \mathbb{N} \centerdot \vert \sum_{i=1}^{n} x_i \vert \le \sum_{i=1}^{n} |x_i|\]
        (注意:需要单独证明 $n = 2$ 的情况,不能仅仅只是假设!)
    \item 回顾 \ref{sec:section2.4.1} 节中关于用多米诺骨牌密铺 $2 \times n$ 棋盘的讨论。现研究类似问题:如何用 $3 \times 1$ 条形三连块密铺 $3 \times n$ 棋盘。请尝试建立归纳关系,并定义数列表示不同密铺方法数。

    (注意:无需尝试求解封闭形式或严格证明,这超出了当前范围。若感兴趣,可查阅\textbf{递推关系}相关文献。如果愿意,可以基于文献内容尝试推导该问题的封闭形式,并思考能否通过归纳法证明。)
\end{enumerate}