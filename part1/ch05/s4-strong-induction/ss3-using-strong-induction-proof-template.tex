% !TeX root = ../../../book.tex
\subsection{使用强归纳法:证明模板}

该模板与常规归纳法的模板非常相似,因为这两个定理(以及相应的应用技术)之间的唯一区别在于归纳假设(IH)。

\subsubsection*{\textcolor{blue}{``强归纳证明''模板}}

\setlength{\fboxrule}{2pt}
\setlength\fboxsep{5mm}
\begin{center}
\noindent \fcolorbox{blue}{white}{%
    \parbox{0.85\textwidth}{%
        \linespread{1.5}\selectfont
        \textbf{目标:} 证明 $\forall n \in \mathbb{N} \centerdot P(n)$
        \begin{proof}\\
            设 $P(n)$ 为命题 ``$\underline{\qquad\qquad\qquad}$''。\\
            我们对 $n$ 采用归纳法证明 $\forall n \in \mathbb{N} \centerdot P(n)$。\\
            \textbf{基本情况}:$P(1)$ 成立,因为 $\underline{\qquad\qquad\qquad}$。\\
            \textbf{归纳假设}:设 $k \in \mathbb{N}$ 是任意固定的,假设 $\forall i \in [k] \centerdot P(k)$ 成立。\\
            \textbf{归纳步骤}:推导出 $P(k+1)$ 也成立。\\
            根据数学归纳原理可得$\forall n \in \mathbb{N} \centerdot P(n)$。
        \end{proof}
    }
}
\end{center}

我们对常规归纳法所做的所有重要观察和建议在这里同样适用。我们必须确保定义一个命题,指出我们在特定变量上应用(强)归纳法,标记我们的步骤,并得出结论。

我们想要提出的一个新建议是对旧建议的改进。在使用常规归纳法时,我们必须确保在使用归纳假设(IH)时引用它。在这里,我们的归纳假设中会有许多命题实例,所以我们必须小心,引用证明中使用到的命题实例!你会在下面的例子中看到这一点。
