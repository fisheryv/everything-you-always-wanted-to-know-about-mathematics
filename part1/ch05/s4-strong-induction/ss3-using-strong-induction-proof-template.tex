% !TeX root = ../../../book.tex
\subsection{使用强归纳法:证明模板}

该模板与常规归纳法高度相似,二者的核心差异仅在于归纳假设 (IH) 的表述形式。

\subsubsection*{\textcolor{blue}{``强归纳证明''模板}}

\setlength{\fboxrule}{2pt}
\setlength\fboxsep{5mm}
\begin{center}
\noindent \fcolorbox{blue}{white}{%
    \parbox{0.85\textwidth}{%
        \linespread{1.5}\selectfont
        \textbf{目标:} 证明 $\forall n \in \mathbb{N} \centerdot P(n)$
        \begin{proof}\\
            设 $P(n)$ 为命题``$\underline{\qquad\qquad\qquad}$''。\\
            我们对 $n$ 采用归纳法证明 $\forall n \in \mathbb{N} \centerdot P(n)$。\\
            \textbf{基本情况}:$P(1)$ 成立,因为 $\underline{\qquad\qquad\qquad}$。\\
            \textbf{归纳假设}:设 $k \in \mathbb{N}$ 为任意固定自然数,假设 $\forall i \in [k] \centerdot P(k)$ 成立。\\
            \textbf{归纳步骤}:推导出 $P(k+1)$ 也成立。\\
            根据数学归纳原理可得 $\forall n \in \mathbb{N} \centerdot P(n)$。
        \end{proof}
    }
}
\end{center}

常规归纳法的重要注意事项在此同样适用:必须明确定义命题,指明归纳变量,清晰标注证明步骤并给出结论。

需要特别强调的是归纳假设的使用规范。常规归纳法要求在使用归纳假设 (IH) 时显式引用,而强归纳法的归纳假设包含多个命题实例,因此必须明确指出证明中所引用的具体实例。下面的示例将展示这一要点。
