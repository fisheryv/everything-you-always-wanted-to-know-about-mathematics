% !TeX root = ../../../book.tex
\subsection{定理陈述与证明}\label{sec:section5.4.2}

我们的目标是陈述并证明一个改进版的数学归纳原理,能够反映多米诺骨牌覆盖和取走游戏这类例子。它们代表了一种归纳论证,其中我们可能需要
\begin{enumerate}[label=(\arabic*)]
    \item 引用多个先前实例来证明声明的后续实例;
    \item 引用某个未知的先前实例来证明后续实例。
\end{enumerate}
这个定理将涵盖这两种风格的论证。首先让我们看看定理的陈述,然后讨论它的含义。

\begin{theorem}[强数学归纳原理(Strong PMI)]\label{theorem5.4.1}
    设 $P(n)$ 为变量命题,假设
    \begin{enumerate}[label=(\arabic*)]
        \item $P(1)$ 成立
        \item $\forall k \in \mathbb{N} \centerdot \big(\forall i \in [k] \centerdot P(i)\big) \implies P(k+1)$ 成立
    \end{enumerate}
    则 $\forall n \in \mathbb{N} \centerdot P(n)$ 成立。
\end{theorem}

哇哦,这是什么意思?我们在这里以逻辑符号的形式呈现定理,然后再以更详细的方式讨论,这样做可能会给你增加了一些额外工作,但我们认为你应该可以应付过来。尽管条件 (2) 显然比较难一些,但请试着解析这两个条件。它说了什么?大声读出来,用一句话写下来,思考一下。将它与我们在上一节中陈述和证明的常规数学归纳原理比较一下。为什么我们称此为`'`强归纳法''?这些定理有何不同?它们的假设不同吗?它们的结论呢?花几分钟时间思考这些问题。然后再继续阅读……

好的,让我们来解释一下这个定理。请注意,\textbf{强归纳法}(定理 \ref{theorem5.4.1})和\textbf{常规归纳法}(定理 \ref{theorem5.2.2})之间的\emph{唯一}区别在于条件 (2),它决定了我们在证明的归纳假设部分要做什么。设定(我们有一个变量命题)和条件 (1)(基本情况)以及结论(对于每个 $n \in \mathbb{N}, P(n)$ 都成立)是相同的。现在让我们比较一下条件 (2)。

常规归纳法要求对于每个 $k \in \mathbb{N}, P(k)$ 足以让我们推导出 $P(k+1)$。如果我们能做到这一点(多米诺骨牌效应),并且我们有一个基本情况,那么 $P(n)$ 对于每个 $n \in \mathbb{N}, P(n)$ 都成立。这就是我们在归纳假设和归纳步骤中所做的:假设 $P(k)$ 成立,并使用它推导出 $P(k + 1)$ 必然成立。

让我们重写强归纳法的条件 (2) 来看看它说了什么:
\[\forall k \in \mathbb{N} \centerdot \big(P(1) \land P(2) \land P(3) \land \dots \land P(k)) \implies P(k+1)\]
也就是说,强归纳法要求所有之前的命题实例($P(1), P(2), (3), \dots$ 一直到 $P(k)$)一起才足以让我们推导出 $P(k+1)$。这个定理似乎在说,``嘿,不用担心只能通过 $P(k)$ 来得到 $P(k+1)$; 实际上你可以使用从 $P(1)$ 到 $P(k)$ 的所有陈述来得到 $P(k+1)$ 成立!所需的结论 --- $\forall n \in \mathbb{N} \centerdot P(n)$ --- 仍然成立!'''是不是很棒?

现在有三方面需要讨论:
\begin{enumerate}[label=(\arabic*)]
    \item 为什么这种方法是有效的;
    \item 什么时候需要使用它;
    \item 如何使用它。
\end{enumerate}
我们先快速解决问题 (3),然后再给你展示一些例子。强归纳法证明和常规归纳法证明之间的唯一区别将在于\textbf{归纳假设}和\textbf{归纳步骤}。当使用强归纳法时,在归纳假设中我们假设 $P(1), P(2), \dots$ 直到 $P(k)$ 都成立,然后使用它们来推导出 $P(k+1)$ 必然成立。在归纳步骤中,我们只需要小心指出我们使用了归纳假设的哪些假设。

为了解决问题 (2) --- 何时使用强归纳法 --- 我们将展示几个例子。在处理这些例子时,我们将准确指出为什么常规归纳法证明会失效。通过研究这些实例,我们希望能培养出在未来识别这些情况的直觉。也就是说,我们将学习到哪些类型的声明在其证明中需要强归纳假设。

再来看问题 (1),因为这是最紧迫的。在我们快速上手并开始使用某个证明技术之前,我们要确保它在数学上是有效的!如果你像我们一样,你会想,``这个定理怎么可能是真的?它说我们需要知道更多 $P(n)$ 实例之间的关系。为什么我们允许在归纳假设中做出这么多假设并能够在以后使用它们?''

\subsubsection*{改良版多米诺骨牌类比与启发图}

我们先从改进第 \ref{ch:chapter02} 章中的多米诺骨牌类比开始,然后展示一个关于强归纳法如何工作的\textbf{启发图},以便你能够更好地理解。之后,我们将严格证明上述定理。

回想一下常规归纳法是如何遵循多米诺骨牌类比的。我们只需要知道多米诺骨牌 $n$ 会倒向多米诺骨牌 $n+1$,就能保证整个队列会倒下。而在强归纳法中,我们需要知道包括多米诺骨牌 $n$ 在内的所有多米诺骨牌都已经倒下,并且撞向多米诺骨牌 $n+1$,将其撞倒,这样才能保证整个队列倒下。这就好像随着队列的延续,多米诺骨牌变得越来越重,因此需要一连串的多米诺骨牌相互撞击才能够产生足够的动量,推倒下一个更重的多米诺骨牌。

换一种方式来解释。想象一下连接我们所有命题的推理链。\textbf{基本情况(BC)}会告诉我们 $P(1)$ 为\verb|真|。这就意味着 $P(2)$ 成立。(在强数学归纳原理的条件 (2) 中应用 $n = 1$。)知道这两个命题成立意味着 $P(3)$ 成立。(在强数学归纳原理的条件 (2) 中应用 $n=2$。)知道这三个命题成立将意味着 $P(4)$ 成立。依此类推:
\[\underbrace{\underbrace{\overbrace{\Big(P(1)\Big)}^{\text{基本情况为真}} \overbrace{\implies}^{\text{应用归纳步骤}} \overbrace{\Big(P(2)\Big)}^{\text{为真}}}_{\text{已知 } P(1) \land P(2)} \underbrace{\implies}_{\text{应用归纳步骤}} \underbrace{\Big(P(3)\Big)}_{\text{为真}}}_{\text{已知 } P(1) \land P(2) \land P(3)} \implies\Big(P(4)\Big) \implies \Big(P(5)\Big)\]
在某种程度上,这说明了为什么这种方法总体上是有效的。我们证明了 $P(1)$ 成立,就像常规归纳法一样。但随后,为了``得出'' $P(2)$ 的真值,第一步--- $P(1) \implies P(2)$ ---在强归纳法和常规归纳法中是一样的。(在强数学归纳原理和数学归纳原理的条件 (2) 中应用 $n = 1$。这种情况下二者是相同的。)从此刻开始,当我们使用强归纳法时,我们只是利用了所有先前命题都成立的事实;我们不妨利用它们来继续推导下一个命题的真值!常规归纳法不关注这一点。它说,``好吧,很好,所有先前的实例都成立。我们实际上不需要它们来证明下一个实例;我们只需要紧接着的前一个实例。''

这里有另一种稍微不同的方式来解释这种``推理链''。这实际上也直接暗示了我们很快将看到的证明!假设我们正在进行强归纳过程,并且我们已经证明了直到 $P(n)$ 的所有内容;也就是说,$P(1), P(2),\dots$ 直到 $P(n)$ 都是为\verb|真|。让我们把这些实例打包在一起,并将它们标记为一个大命题 $Q(n)$。(换一种方式思考,我们将所有这些多米诺骨牌绑定在一起,变成一个巨大的多米诺骨牌。)下一步是使用此实例来证明下一个实例,这听起来更像是我们熟悉的常规归纳法。这基本上就是我们要在证明中做的事情!我们将重新构建整个强归纳法过程,使其表述为一个常规归纳法过程。

\subsubsection*{严格证明}

正如前文所提到的,接下来的证明将运用数学归纳原理(PMI)。 (事实上,我们还会使用上一节中讲解的归纳证明模板!)从这个角度来看,我们实际上是在证明这个命题:
\[\text{PMI} \implies \text{SPMI}\]
让我们开始吧!

\begin{proof}
    设 $P(n)$ 为变量命题,假设
    \begin{enumerate}[label=(\arabic*)]
        \item $P(1)$ 成立
        \item $\forall k \in \mathbb{N} \centerdot \big(\forall i \in [k] \centerdot P(i)\big) \implies P(k+1)$ 成立
    \end{enumerate}
    我们的目标是证明 $\forall n \in \mathbb{N} \centerdot P(n)$。

    定义命题 $Q(n)$ 为
    \[Q(n) \iff \forall i \in [n] \centerdot P(i)\]
    (也就是说,$Q(n)$ 表示所有命题 $P(1), P(2), \dots$ 直到 $P(n)$ 都为\verb|真|。)

    我们通过对 $n$ 应用归纳法来证明 $\forall n \in \mathbb{N} \centerdot Q(n)$。

    \textbf{基本情况}:根据命题 $Q(n)$ 的定义,我们有 $Q(1) \iff P(1)$。条件 (1) 告诉我们 $P(1)$ 成立,那么 $Q(1)$ 也一定成立。

    \textbf{归纳假设}:设 $k \in \mathbb{N}$ 是任意固定的,假设 $Q(k)$ 成立。

    \textbf{归纳步骤}:根据 $Q(n)$ 的定义,我们有
    \[Q(n) \iff \forall i \in [k] \centerdot P(i)\]
    (再次强调,这说明 $P(1), P(2), \dots$ 直到 $P(k)$ 都成立。)

    根据条件 (2) 我们可以推导出 $P(k+1)$ 成立。

    这意味着 $\forall i \in [k+1] \centerdot P(i)$(也就是说我们以前知道 $P(1), P(2), \dots$ 直到 $P(k)$ 都成立,现在我们得到 $P(k+1) 也成立$)。

    根据 $Q(k+1)$ 的定义,这意味着 $Q(k+1)$ 成立。而这正是归纳步骤的目标。

    因此,根据数学归纳原理(PMI),我们推导出 $\forall n \in \mathbb{N} \centerdot Q(n)$ 成立。

    根据 $Q(n)$ 的定义,我们有
    \[\forall n \in \mathbb{N} \centerdot Q(n) \implies P(n)\]
    (也就是说,$Q(n)$ 表示 $P(1), \dots$ 到 $P(n)$ 的所有实例都成立,那么最后一个实例 $P(n)$ 显然成立。)

    由于我们刚刚证明了 $Q(n)$ 对于每个 $n \in \mathbb{N}$ 都成立,因此我们可以推断 $P(n)$ 对于每个 $n \in \mathbb{N}$ 也成立,即
    \[\forall n \in \mathbb{N} \centerdot P(n)\]
    这就是我们要证的目标,所以证明完毕。
\end{proof}

\subsubsection*{证明总结和惊人的等价性}

看看我们完成了什么:我们用常规归纳法证明了强归纳法是有效的。这表明数学归纳原理(PMI)定理\emph{蕴含}强数学归纳原理(SPMI)定理:
\[\text{PMI} \implies \text{SPMI}\]

当然,这也可以反过来!如果我们已经通过其他方法证明了强归纳法的有效性,那么普通归纳法也必然是有效的。也就是说,我们还知道:
\[\text{SPMI} \implies \text{PMI}\]
换句话说:如果我们已经掌握了强归纳法作为一种有效的证明技术,那么每当我们想用常规归纳法证明某事时,我们只需使用强归纳法来实现我们的目标。从这个意义上讲,强归纳法``包含''了常规归纳法。

这两个观察结果共同告诉我们关于数学归纳原理(PMI)和强数学归纳原理(SPMI)在数学世界中的一个重要事实。我们现在已经证明它们是等价的:
\[\text{PMI} \iff \text{SPMI}\]
每个定理都蕴含着另一个定理。

现在,出于将这些技术应用于证明的实际目的,这种等价性可能看起来不太重要,但它确实告诉我们一些有用的信息,即:
\begin{quotation}
    \emph{每当我们需要通过归纳法来证明某个命题时,我们最好总是使用强归纳法。}
\end{quotation}

思考几分钟。阅读定理陈述及其证明并考虑一下。在接下来的例子中牢记这一点。等你你学习了下面的证明模板,请回到前一节关于常规归纳法的例子并尝试应用强归纳法。它有效吗?看起来有什么不同吗?试试看!我们将在下面的例子中讨论常规/强归纳法的比较,所以让我们继续看看如何使用强归纳法。
