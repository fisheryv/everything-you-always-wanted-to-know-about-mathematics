% !TeX root = ../../../book.tex
\subsection{定理陈述与证明}\label{sec:section5.4.2}

我们的目标是陈述并证明一个改进版的数学归纳原理,以涵盖多米诺骨牌密铺和取走游戏所体现的归纳论证模式。这类论证的特点在于:
\begin{enumerate}[label=(\arabic*)]
    \item 引用多个先前实例来证明后续实例;
    \item 引用任意先前实例来证明后续实例。
\end{enumerate}
以下定理将统一描述这两种论证形式。先给出定理陈述,再分析其内涵。

\begin{theorem}[强数学归纳原理 (Strong PMI)]\label{theorem5.4.1}
    设 $P(n)$ 为变量命题,假设
    \begin{enumerate}[label=(\arabic*)]
        \item $P(1)$ 成立
        \item $\forall k \in \mathbb{N} \centerdot \big(\forall i \in [k] \centerdot P(i)\big) \implies P(k+1)$ 成立
    \end{enumerate}
    则 $\forall n \in \mathbb{N} \centerdot P(n)$ 成立。
\end{theorem}

定理的条件 (2) 具有深刻含义。建议先尝试解析这两个条件:将其转化为自然语言表述,比较其与常规数学归纳原理(定理 \ref{theorem5.2.2})的异同。思考为何称其为``强归纳法''?两者的假设与结论有何区别?建议暂停,思考后再继续阅读……

强归纳法(定理 \ref{theorem5.4.1})与常规归纳法(定理 \ref{theorem5.2.2})的核心差异在于条件 (2)。两者的命题框架、条件 (1)(基础情形)及最终结论均相同。具体比较条件 (2):

常规归纳法要求:若 $P(k)$ 成立,则能推出 $P(k+1)$ 成立。这对应多米诺骨牌类比——假设第 $k$ 块骨牌倒下可以推出第 $k+1$ 块倒下。

强归纳法则可表述为:
\[\forall k \in \mathbb{N} \centerdot \big(P(1) \land P(2) \land P(3) \land \dots \land P(k)) \implies P(k+1)\]
也就是说,强归纳法要求之前的所有命题实例($P(1), P(2), (3), \dots$ 直到 $P(k)$) 共同蕴含 $P(k+1)$。这意味着在推导 $P(k+1)$ 时,可自由使用从 $P(1)$ 到 $P(k)$ 的全部结论,而结论 $\forall n \in \mathbb{N} \centerdot P(n)$ 依然成立。

现在有三个关键问题需要讨论:
\begin{enumerate}[label=(\arabic*)]
    \item 为什么该方法有效;
    \item 什么时候使用它;
    \item 如何使用它。
\end{enumerate}

我们先快速解决问题 (3),然后再展示一些例子。强归纳法与常规归纳法的证明仅在\textbf{归纳假设}和\textbf{归纳步骤}存在差异:使用强归纳法时,假设 $P(1), P(2), \ldots, P(k)$ 全部成立,并据此推导 $P(k+1)$。在归纳步骤中需要明确说明所使用的假设。

问题 (2) 将通过后续示例阐释,这些例子将揭示常规归纳法失效的原因,帮助培养识别适用情形的直觉。

问题 (1) 最为关键。在使用证明技术前,必须确保其数学有效性。读者可能质疑:``该定理要求更强的关联性,为何允许在归纳假设中引入更多前提?''

\subsubsection*{改良版多米诺骨牌类比与启发图}

我们从改进第 \ref{ch:chapter02} 章中的多米诺骨牌类比入手,随后展示一幅关于强归纳法工作原理的\textbf{启发图},以助理解。最后,我们将严格证明相关定理。

回顾常规归纳法的多米诺骨牌类比:仅需确认第 $n$ 张骨牌会撞倒第 $n+1$ 张骨牌,即可保证整列骨牌倒下。而在强归纳法中,必须确保第 $n+1$ 张骨牌之前的所有骨牌(包含第 $n$ 张)均已倒下并共同撞击第 $n+1$ 张骨牌,方能保证整列倒下。这如同骨牌随序列延续而愈发沉重,需要借助前方连续倾倒的多张骨牌共同累积动量,方能推倒后续更重的骨牌。

另一种理解方式是考察命题间的推理链:\textbf{基本情况 (BC)} 证明 $P(1)$ 成立。这意味着 $P(2)$ 成立(在强数学归纳原理条件 (2) 中取 $n = 1$)。已知 $P(1)$ 与 $P(2)$ 成立可推出 $P(3)$ 成立(取 $n=2$)。同理,已知前三个命题成立将推出 $P(4)$ 成立,依此类推:
\[\underbrace{\underbrace{\overbrace{\Big(P(1)\Big)}^{\text{基本情况为真}} \overbrace{\implies}^{\text{应用归纳步骤}} \overbrace{\Big(P(2)\Big)}^{\text{为真}}}_{\text{已知 } P(1) \land P(2)} \underbrace{\implies}_{\text{应用归纳步骤}} \underbrace{\Big(P(3)\Big)}_{\text{为真}}}_{\text{已知 } P(1) \land P(2) \land P(3)} \implies\Big(P(4)\Big) \implies \Big(P(5)\Big)\]
某种程度上,此过程揭示了方法的有效性:与常规归纳法相同,我们先证 $P(1)$ 成立。推导 $P(2)$ 时,步骤 $P(1) \implies P(2)$ 在两种归纳法中完全一致(取 $n=1$ 时,强归纳原理与数学归纳原理的条件 (2) 等价)。此后使用强归纳法时,我们充分利用了先前所有命题成立的事实——既然已有这些结论,自然可用于推导后续命题的真值。而常规归纳法则忽略此优势,仅依赖紧邻的前一个实例进行推导。

关于``推理链''的另一种理解直接关联后续证明:假设强归纳过程已证得 $P(1)$ 至 $P(n)$ 全部成立。将这些命题整体打包为一个复合命题 $Q(n)$——形象地说,如同将多张骨牌组合成一张巨型骨牌。下一步即是利用 $Q(n)$ 证明下一实例 $P(n+1)$,此过程已转化为熟悉的常规归纳模式。这正是证明的核心思路:通过重构将强归纳过程表述为常规归纳形式。

\subsubsection*{严格证明}

如前所述,我们将运用数学归纳原理 (PMI) 进行证明(实际上会采用上一节介绍的归纳证明模板)。本质上,我们是在证明:
\[\text{PMI} \implies \text{SPMI}\]
现在开始证明。

\begin{proof}
    设 $P(n)$ 为变量命题,假设
    \begin{enumerate}[label=(\arabic*)]
        \item $P(1)$ 成立
        \item $\forall k \in \mathbb{N} \centerdot \big(\forall i \in [k] \centerdot P(i)\big) \implies P(k+1)$ 成立
    \end{enumerate}
    我们的目标是证明 $\forall n \in \mathbb{N} \centerdot P(n)$。

    定义命题 $Q(n)$ 为
    \[Q(n) \iff \forall i \in [n] \centerdot P(i)\]

    (也就是说,$Q(n)$ 表示所有命题 $P(1), P(2), \dots$ 直到 $P(n)$ 都为\verb|真|。)

    我们通过对 $n$ 应用归纳法来证明 $\forall n \in \mathbb{N} \centerdot Q(n)$。

    \textbf{基本情况}:根据命题 $Q(n)$ 的定义,我们有 $Q(1) \iff P(1)$。条件 (1) 告诉我们 $P(1)$ 成立,那么 $Q(1)$ 也一定成立。

    \textbf{归纳假设}:设 $k \in \mathbb{N}$ 为任意固定自然数,假设 $Q(k)$ 成立。

    \textbf{归纳步骤}:根据 $Q(n)$ 的定义,我们有
    \[Q(n) \iff \forall i \in [k] \centerdot P(i)\]

    (再次强调,这说明 $P(1), P(2), \dots$ 直到 $P(k)$ 都成立。)

    根据条件 (2) 我们可以推导出 $P(k+1)$ 成立。

    这意味着 $\forall i \in [k+1] \centerdot P(i)$(也就是说已知 $P(1), P(2), \dots$ 直到 $P(k)$ 都成立,推得 $P(k+1)$ 也成立)。

    根据 $Q(k+1)$ 的定义,这意味着 $Q(k+1)$ 成立。而这正是归纳步骤的目标。

    因此,根据数学归纳原理 (PMI),我们推导出 $\forall n \in \mathbb{N} \centerdot Q(n)$ 成立。

    根据 $Q(n)$ 的定义,我们有
    \[\forall n \in \mathbb{N} \centerdot Q(n) \implies P(n)\]

    (也就是说,$Q(n)$ 表示 $P(1), \dots$ 到 $P(n)$ 的所有实例都成立,那么最后一个实例 $P(n)$ 显然成立。)

    以上证明了 $Q(n)$ 对于每个 $n \in \mathbb{N}$ 都成立,因此可以推断 $P(n)$ 对于每个 $n \in \mathbb{N}$ 也成立,即
    \[\forall n \in \mathbb{N} \centerdot P(n)\]

    证明完毕。
\end{proof}

\subsubsection*{证明总结与惊人的等价性}

我们完成了什么?通过常规归纳法,我们证明了强归纳法的有效性。这表明数学归纳原理 (PMI) \emph{蕴含}强数学归纳原理 (SPMI):
\[\text{PMI} \implies \text{SPMI}\]

这个关系是可逆的!若已通过其他方法证明强归纳法有效,则常规归纳法必然成立:
\[\text{SPMI} \implies \text{PMI}\]

换言之:若掌握强归纳法这一有效证明技术,当需用常规归纳法证明命题时,只需借助强归纳法即可实现。在此意义上,强归纳法``包含''了常规归纳法。

综上可知,数学归纳原理 (PMI) 与强数学归纳原理 (SPMI) 具有深刻的等价性:
\[\text{PMI} \iff \text{SPMI}\]

两者相互蕴含。

尽管这种等价性在具体证明中看似无关紧要,但它揭示了一个实用准则:
\begin{quotation}
    \emph{当需要通过归纳法证明命题时,应当优先选用强归纳法。}
\end{quotation}

请思考几分钟。阅读定理陈述及其证明后,带着这个观点审视后续示例。学习完下一节的证明模板后,请返回常规归纳法的示例并尝试应用强归纳法。它是否奏效?证明过程有何差异?我们将在后续示例中对比常规归纳法与强归纳法,现在继续探讨强归纳法的应用。

\clearpage