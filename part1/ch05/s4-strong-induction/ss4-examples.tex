% !TeX root = ../../../book.tex
\subsection{示例}

我们将在此展示三种不同``类型''的示例。尽管它们都采用前述强归纳法模板,但在引用归纳假设 (IH) 时存在差异。第一个例子是强归纳法的直接应用,因此我们先分析它,再讨论其他例子的不同之处。\\

\begin{example}[递归定义数列的封闭公式]
    
    \textbf{声明}:定义序列 $S_n$ 为
    \[s_0 = 1 \text{\ 且\ } \forall n \in \mathbb{N} \centerdot s_n = 1 + \sum_{i=0}^{n-1} s_i\]

    对于每个 $n \in \mathbb{N} \cup \{0\}$,找到并证明 $s_n$ 的封闭公式。

    \begin{proof}
        令 $P(n)$ 为``$s_n = 2^n$''。我们通过对 $n$ 应用归纳法来证明 $n \in \mathbb{N} \cup \{0\} \centerdot P(n)$。

        \textbf{基本情况}:当 $n=0$ 时,易得 $s_0=1$ 且 $2^0=1$,所以 $s_0=2^0$,因此 $P(0)$ 成立。

        \textbf{归纳假设}:设 $k \in \mathbb{N} \cup \{0\}$ 为任意固定元素,假设 $P(0) \land P(1) \land \dots \land P(k)$ 成立。

        \textbf{归纳步骤}:易得
        \begin{align*}
            s_{k+1} &= 1+\sum_{i=0}^{k} s_i &s_{k+1}\text{\ 的定义}\\
            &= 1+\sum_{i=0}^{k} 2^i &\text{利用归纳假设:} P(0) \land P(1) \land \dots \land P(k)\\
            &= 1 + (2^{k+1}-1) &\text{标准答案(见练习\ } \ref{exc:exercises2.7.1} \text{)}\\
            &= 2^{k+1}
        \end{align*}

        因此 $P(k+1)$ 成立。根据归纳法 $\forall n \in \mathbb{N} \cup \{0\} \centerdot P(n)$ 成立。
    \end{proof}
\end{example}

注意此例要求使用归纳假设 (IH) 的全部实例——这正是强归纳法的必要性所在:若不能假设所有前置实例成立,则无法推导出后续结果!

与下例不同,本例中我们明确使用了归纳假设 (IH) 的全部实例。而在下例中,虽然也会使用归纳假设 (IH),却无法预先指定具体实例。你将在后续内容中理解这一区别!\\

\begin{example}\label{ex:example5.4.3}
    首先,回顾质数和自然数的基本概念。

    \textbf{质数}:\textbf{质数}是集合
    \[P = \{n \in \mathbb{N} \mid n > 1 \land (n = ab) \implies (a = 1 \lor a = n)\}\]
    的元素。也就是说,质数的因子只有 $1$ 和它本身。

    \textbf{质因数分解}:给定 $x \in \mathbb{N}$, $x$ 的\textbf{质因数分解}是将 $x$ 表示为质数的乘积(允许重复),且该乘积等于 $x$。

    例如,$6$ 的质因数分解为 $2 \cdot 3$, $252$ 的质因数分解为 $2 \cdot 2 \cdot 3 \cdot 3 \cdot 7$。

    接下来,我们陈述并证明每个自然数都存在质因数分解。

    \textbf{声明}:令 $F(n)$ 为命题``$n$ 存在质因数分解''。我们声明 $\forall n \in \mathbb{N} - \{1\} \centerdot F(n)$。

    \begin{proof}
        我们通过对 $n$ 应用强归纳法来证明 $\forall n \in \mathbb{N} - \{1\} \centerdot F(n)$。

        \textbf{基本情况}:$F(2)$ 成立,因为 $2 = 2$ 是自然数 $2$ 的质因数分解。

        \textbf{归纳假设}:设 $k \in \mathbb{N} - \{1\}$ 为任意固定元素,假设 $\forall i \in [k]-\{1\} \centerdot F(i)$ 成立。(即 $F(2) \land F(3) \land \dots \land F(k)$ 成立。)

        \textbf{归纳步骤}:考虑 $k+1$。我们想要找到 $k+1$ 的质因数分解。根据 $k+1$ 本身是否为质数,分两种情况讨论:

        \textbf{情况 1}:如果 $k+1$ 本身是一个质数,则 $k+1$ 是 $k+1$ 的质因数分解,从而证明 $F(k+1)$ 成立。

        \textbf{情况 2}:如果 $k+1$ 不是质数,则存在 $a, b \in \mathbb{N} - \{1\}$ 使得 $k+1 = a \cdot b$。由于 $a, b \neq 1$,所以必有 $1 < a < k+1$ 且 $1 < b < k+1$。即 $2 \le a \le k$ 且 $2 \le b \le k$。

        因此,根据归纳假设,$F(a)$ 和 $F(b)$ 成立。也就是说 $a$ 和 $b$ 都存在质因数分解。将这两个质因数分解相乘即可得到 $a \cdot b = k+1$ 的质因数分解。这表明 $F(k+1)$ 成立。

        无论是上面哪种情况,我们都推导出 $F(k+1)$ 成立。

        根据归纳法,可得 $\forall n \in \mathbb{N} - \{1\} \centerdot F(n)$ 成立。
    \end{proof}
\end{example}

注意,在这个证明中我们应用了归纳假设 (IH),但并未指定具体引用的``先前实例''。我们只能依赖具有特定属性的某个 $a$ 和 $b$。这与之前的例子不同,但清楚地表明此处需要使用强归纳法。$k$ 的质因数分解无法帮助我们直接得到 $k+1$ 的质因数分解。思考一下:知道 $14 = 2 \cdot 7$ 能帮助我们得到 $15 = 3 \cdot 5$ 吗?知道 $16 = 2^4$ 能帮助我们判断 $17$ 是质数吗?

我们刚刚证明的这个结论非常重要:它表明每个自然数 $n > 1$ 都存在质因数分解。进一步,质因数分解具有\textbf{唯一性}:每个自然数\emph{有且只有}一种质因数分解(忽略因子顺序)。例如,$6 = 2 \cdot 3$ 与 $6 = 3 \cdot 2$ 视为相同;$252 = 2 \cdot 2 \cdot 3 \cdot 3 \cdot 7$ 等价于 $252 = 7 \cdot 3^2 \cdot 2^2$。

但上述证明并未未涉及唯一性证明!我们仅依赖 $a$ 和 $b$ 的存在性。谁能否认存在其他满足条件的 $c$ 和 $d$ 呢?思考:如何证明质因数分解具有唯一性?你将采用何种方法?

下一个例子将涉及我们之前讨论过的数列——\textbf{斐波那契数列}。具体来说,我们将陈述并证明该数列的\textbf{封闭形式}。斐波那契数列通常以递归形式定义,而``封闭形式''指的是可直接代入并计算出结果的表达式。例如,要计算 $f_{100}$,使用递归定义需要逐步计算所有前项:这需要计算 $f_{99}$ 和 $f_{98}$,进而需要计算 $f_{97}$,依此类推……然而,通过封闭形式,只需代入 $n$ 的值即可直接计算出 $f_{100}$。\\

\begin{example}[斐波那契数列的封闭形式]

    \textbf{声明}:斐波那契数列的标准定义如下:
    \[f_0 = 0, \quad f_1 = 1, \quad \forall n \in \mathbb{N}-\{1\} \centerdot f_n = f_{n-1} + f_{n-2}\]

    定义 $\varphi = \frac{1+\sqrt{5}}{2}$。则对于所有 $n \in \mathbb{N}-\{1\}$ 下面等式均成立:
    \[f_n = \frac{1}{\sqrt{5}}\big(\varphi^n-(1-\varphi)^n\big)\]

    \begin{proof}
        设 $f_n$ 和 $\varphi$ 如声明所定义。

        我们首先证明下面等式成立:
        \[1+\varphi=\varphi^2 \tag{*1}\]

        不难发现
        \begin{align*}
            \varphi^2 &= \Big(\frac{1+\sqrt{5}}{2}\Big)^2 = \frac{1+2\sqrt{5}+5}{4} = \frac{6+2\sqrt{5}}{4} \\
            &= \frac{3+\sqrt{5}}{2} = 1+\frac{1+\sqrt{5}}{2} = 1+\varphi \\
        \end{align*}

        接着,我们利用这个等式证明下面的等式:
        \[2-\varphi=(1-\varphi)^2 \tag{*2}\]

        利用 (*1) 的结论,可以推出
        \[(1-\varphi)^2 = 1-2\varphi+\varphi^2 = 1-2\varphi+(\varphi+1) = 2-\varphi\]

        令 $P(n)$ 为命题
        \[f_n = \frac{1}{\sqrt{5}}\big(\varphi^n-(1-\varphi)^n\big)\]

        我们通过对 $n$ 应用归纳法来证明 $\forall n \in \mathbb{N} \cup \{0\} \centerdot P(n)$。

        \textbf{基本情况}:$f(0) = 0$ 且
        \[\frac{1}{\sqrt{5}}\big(\varphi^0-(1-\varphi)^0\big) = \frac{1}{\sqrt{5}} (1-1) = 0\]

        故 $P(0)$ 成立。

        \textbf{归纳假设}:设 $k \in \mathbb{N} \cup \{0\}$ 为任意固定元素,假设 $\forall i \in [k] \cup \{0\} \centerdot P(i)$ 成立。

        \textbf{归纳步骤}:我们的目标是推导出 $P(k+1)$ 成立。

        \textbf{情况 1}:假设 $k=0$,易得 $f_1=1$ 且
        \[\frac{1}{\sqrt{5}}\big(\varphi^1-(1-\varphi)^1\big) = \frac{1}{\sqrt{5}} (2\varphi-1) = \frac{1}{\sqrt{5}} (1+\sqrt{5}-1) = \frac{1}{\sqrt{5}} (\sqrt{5}) = 1\]

        故 $P(1)$ 成立。

        \textbf{情况 2}:假设 $k \ge 1$,则
        \begin{align*}
            f_{k+1} &= f_k+f_{k-1} & \text{因为 } k \ge 1, \textbf{ 根据定义} \\
            &= \frac{1}{\sqrt{5}}\big(\varphi^k-(1-\varphi)^k\big) + \frac{1}{\sqrt{5}}\big(\varphi^{k-1}-(1-\varphi)^{k-1}\big) & \text{归纳假设 } P(k), P(k-1)\\
            &= \frac{1}{\sqrt{5}}\big(\varphi^k+\varphi^{k-1}-(1-\varphi)^k-(1-\varphi)^{k-1}\big) & \textbf{化简}\\
            &= \frac{1}{\sqrt{5}}\big(\varphi^{k-1}(\varphi+1)-(1-\varphi)^{k-1}((1-\varphi)+1)\big) & \textbf{提取公因式}\\
            &= \frac{1}{\sqrt{5}}\big(\varphi^{k-1} \cdot \varphi^2-(1-\varphi)^{k-1}(2-\varphi)\big) & \textbf{根据(*1)}\\
            &= \frac{1}{\sqrt{5}}\big(\varphi^{k+1} -(1-\varphi)^{k-1}(1-\varphi)^2\big) & \textbf{根据(*2)}\\
            &= \frac{1}{\sqrt{5}}\big(\varphi^{k+1} -(1-\varphi)^{k+1}\big) & 
        \end{align*}

        故 $P(k+1)$ 成立。

        根据归纳法,我们得出 $\forall n \in \mathbb{N} \cup \{0\} \centerdot P(n)$。
    \end{proof}
\end{example}

\subsubsection*{多基本情况讨论}

注意,在前面的例子中,我们需要在归纳步骤 (IS) 中考虑两种情况。由于斐波那契数列是递归定义的,每一项都依赖于前两项,因此不能仅凭 $P(0)$ 成立就推导出 $P(1)$ 成立。(你可以尝试验证:这将被迫引用未定义的 $f_{-1}$!)接下来,我们可以利用 $P(0)$ 和 $P(1)$ 成立推导出 $P(2)$,再用 $P(1)$ 和 $P(2)$ 推导出 $P(3)$,依此类推……也就是说,在整个归纳法推理开始前,需要添加一个额外的基本情况。

处理这一问题有两种有效方法,我们已展示其中一种。另一种方法是预先识别这种情况,并在基本情况步骤中设置两个基本情况。为了说明这一点,下面展示采用第二种方法时证明的相关部分:

\begin{proof}

    $\dots$

    $\dots$

    \textbf{基本情况}:$f(0) = 0$ 且
    \[\frac{1}{\sqrt{5}}\big(\varphi^0-(1-\varphi)^0\big) = \frac{1}{\sqrt{5}} (1-1) = 0\]

    故 $P(0)$ 成立。

    并且 $f(1) = 1$ 且
    \[\frac{1}{\sqrt{5}}\big(\varphi^1-(1-\varphi)^1\big) = \frac{1}{\sqrt{5}} (2\varphi-1) = \frac{1}{\sqrt{5}} (1+\sqrt{5}-1) = \frac{1}{\sqrt{5}} (\sqrt{5}) = 1\]
    
    故 $P(1)$ 成立。

    \textbf{归纳假设}:设 $k \in \mathbb{N}$ 为任意固定自然数,假设 $\forall i \in [k] \cup \{0\} \centerdot P(i)$ 成立。

    \textbf{归纳步骤}:我们的目标是推导出 $P(k+1)$ 成立。

    $\dots$

    $\dots$
\end{proof}

我们将特殊的 $P(1)$ 移至基本情况 (BC) 部分,因此需要调整归纳假设和归纳步骤的量化表述。后续论证中不再使用 $k = 0$,故归纳假设仅考虑 $k \ge 1$。但由于 $P(0)$ 已知成立,仍可将其包含在归纳假设中。

两个证明本质相同,唯一区别在于呈现方式,且这些区别十分微小。你可以自由选择偏好的证明风格(如果有的话)。但需注意:这些区别虽然微小却很微妙,易被忽略!若证明涉及多个基本情况,务必在归纳步骤中证明高于这些基本情形的值。否则可能无意中断言无效的逻辑推论(例如第二个证明中:若允许 $k=0$,将引用不存在的 $f_{-1}$,导致证明有缺陷)。

此类区别常见于递归定义数列的表达式证明,其中每项依赖于前若干项。本章节及章末练习包含许多此类案例,请务必牢记!

\subsubsection*{需要证明 $n=2$ 的情况}

在强归纳法的证明中,通常需要先验证 $n = 1$ 和 $n = 2$ 成立,再建立归纳假设。尤其是证明涉及 $n$ 个变量的不等式或等式时,$n = 1$ 的情形往往较简单,而 $n = 2$ 的情形则需更深入的分析。后续的归纳证明可以通过引用 $n = 2$ 的结论完成,这意味着归纳假设需设定 $k \ge 2$。

以下通过示例说明。值得注意的是,$n = 2$ 的情形已被证明——这正是集合的德摩根定律之一!\\

\begin{example}[集合的广义德摩根定律:]

    \textbf{声明}:设 $U$ 为全集。对于每个 $i \in \mathbb{N}$,设集合 $A_i \subseteq U$。则对于所有 $n \in \mathbb{N}$,下列等式成立:
    \[\overline{\bigcup_{i=1}^{n} A_i} = \bigcap_{i=1}^{n} \overline{A_i}\]
    
    上面命题可以等价地写作:对于所有 $n \in \mathbb{N}$
    \[\overline{A_1 \cup A_2 \cup \dots \cup A_n} = \overline{A_1} \cap \overline{A_2} \cap \dots \cap \overline{A_n}\]

    \begin{proof}
        设 $U$ 及 $A_1, A_2, \dots$ 如声明定义。

        设 $P(n)$ 为命题
        \[\overline{\bigcup_{i=1}^{n} A_i} = \bigcap_{i=1}^{n} \overline{A_i}\]

        我们通过对 $n$ 应用归纳法来证明 $\forall n \in \mathbb{N} \centerdot P(n)$。

        \textbf{基本情况}:显然 $\overline{A_1} = \overline{A_1}$,故 $P(1)$ 成立。

        并且根据集合的德摩根定律(见定理 \ref{theorem4.6.9})我们有 $\overline{A_1 \cup A_2} = \overline{A_1} \cap \overline{A_2}$,故 $P(2)$ 成立。

        \textbf{归纳假设}:设 $k \in \mathbb{N}-\{1\}$ 为任意固定元素,假设 $\forall i \in [k] \centerdot P(i)$ 成立。

        \textbf{归纳步骤}:我们的目标是推导出 $P(k+1)$ 成立。

        首先,根据并集的定义,可得
        \[\bigcup_{i=1}^{k+1} A_i = A_{k+1} \cup \bigcup_{i=1}^k A_i\]

        不妨设
        \[B_k = \bigcup_{i=1}^k A_i\]

        则
        \begin{align*}
            \overline{\bigcup_{i=1}^{k+1} A_i} &= \overline{A_{k+1} \cup B_k} & \text{根据\ } B_k \text{\ 的定义} \\
            &= \overline{A_{k+1}} \cap \overline{B_k} & \text{根据基本情况\ } P(2) \text{\ (即德摩根定律)}\\
            &= \overline{A_{k+1}} \cap \overline{\bigcup_{i=1}^k A_i} & \text{根据\ } B_k \text{\ 的定义} \\
            &= \overline{A_{k+1}} \cap \bigcap_{i=1}^k \overline{A_i} & \text{根据归纳假设\ } P(k) \\
            &= \bigcap_{i=1}^{k+1} \overline{A_i} & \text{化简}
        \end{align*}

        故 $P(k+1)$ 成立。

        根据归纳法,$\forall n \in \mathbb{N} \centerdot P(n)$。
    \end{proof}
\end{example}