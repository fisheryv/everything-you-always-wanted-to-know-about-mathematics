% !TeX root = ../../../book.tex
\subsection{示例}

我们将在这里看到三种不同``类型''的例子。尽管它们都使用我们刚刚介绍的强归纳法模板,但它们在引用归纳假设(IH)中的假设时有所不同。第一个例子是强归纳法的直接应用,所以让我们先来看看它,然后讨论其他例子有什么不同。\\

\begin{example}[递归定义数列的公式]
    
    \textbf{声明}:序列 $S_n$ 定义为
    \[s_0 = 1 \;\text{ 且 }\; \forall n \in \mathbb{N} \centerdot s_n = 1 + \sum_{i=0}^{n-1} s_i\]
    对于每个 $n \in \mathbb{N} \cup \{0\}$,找到并证明 $s_n$ 的封闭公式。
\end{example}

\begin{proof}
    令 $P(n)$ 为 ``$s_n = 2^n$''。我们通过对 $n$ 应用归纳法来证明 $n \in \mathbb{N} \cup \{0\} \centerdot P(n)$。

    \textbf{基本情况}:当 $n=0$ 时,不难得到 $s_0=1$ 且 $2^0=1$,所以 $s_0=2^0$,因此 $P(0)$ 成立。

    \textbf{归纳假设}:设 $k \in \mathbb{N} \cup \{0\}$ 是任意固定的,假设 $P(0) \land P(1) \land \dots \land P(k)$ 成立。

    \textbf{归纳步骤}:不难得到

    \begin{align*}
        s_{k+1} &= 1+\sum_{i=0}^{k} s_i & s_{k+1}\text{ 的定义}\\
        &= 1+\sum_{i=0}^{k} 2^i & \text{利用归纳假设:}P(0) \land P(1) \land \dots \land P(k)\\
        &= 1 + (2^{k+1}-1) & \text{标准答案(见练习 } \ref{exc:exercises2.7.1} \text{)}\\
        &= 2^{k+1}
    \end{align*}

    因此 $P(k+1)$ 成立。所以,根据归纳法 $\forall n \in \mathbb{N} \cup \{0\} \centerdot P(n)$ 成立。
\end{proof}

请注意,这个例子要求我们使用归纳假设(IH)中的所有实例。是不是很惊人?确实,我们在这里需要强归纳法。如果不知道之前的所有实例是否成立,我们就无法推导出下一个实例!

与下一个例子不同的是,在这个例子中,我们确切知道使用了归纳假设(IH)的哪些实例(即全部实例)。而在下一个例子中,我们将也会使用归纳假设(IH),但无法确切地指出使用了哪个实例。你会明白我们的意思的!\\

\begin{example}\label{ex:example5.4.3}
    首先,我们需要向你介绍(或者提醒你)一些关于质数和自然数的概念。

    \textbf{质数}:\textbf{质数}是下面集合的元素
    \[P = \{n \in \mathbb{N} \mid n > 1 \land (n = ab) \implies (a = 1 \lor a = n)\}\]
    也就是说,质数的因子只有 $1$ 和它本身。

    \textbf{质因数分解}:给定 $x \in \mathbb{N}$,$x$ 的\textbf{质因数分解}为一组质数的乘积,可以重复,乘积的结果等于 $x$。

    例如,$6$ 的质因数分解为 $2 \cdot 3$,$252$ 的质因数分解为 $2 \cdot 2 \cdot 3 \cdot 3 \cdot 7$。

    我们接下来将陈述并证明每个自然数都有一个质因数分解。

    \textbf{声明}:令 $F(n)$ 为命题 ``$n$ 由质因数分解''。我们声明 $\forall n \in \mathbb{N} - \{1\} \centerdot F(n)$。
\end{example}

\begin{proof}
    我们通过对 $n$ 应用归纳法来证明 $\forall n \in \mathbb{N} - \{1\} \centerdot F(n)$。

    \textbf{基本情况}:$F(2)$ 成立,因为 $2 = 2$ 是自然数 $2$ 的质因数分解。

    \textbf{归纳假设}:设 $k \in \mathbb{N} - \{1\}$ 是任意固定的,假设 $\forall i \in [k]-\{1\} \centerdot F(i)$ 成立。(也就是说,假设 $F(2) \land F(3) \land \dots \land F(k)$ 成立。)

    \textbf{归纳步骤}:考虑 $k+1$。我们想要找到 $k+1$ 的质因数分解。根据 $k+1$ 本身是否为质数,有两种情况:

    \textbf{情况 1}:如果 $k+1$ 本身是一个质数,则 $k+1$ 是 $k+1$ 的质因数分解,从而证明 $F(k+1)$ 成立。

    \textbf{情况 2}:如果 $k+1$ 不是质数,则存在 $a, b \in \mathbb{N} - \{1\}$ 使得 $k+1 = a \cdot b$。由于 $a, b \neq 1$,所以必定有 $1 < a < k+1$ 且 $1 < b < k+1$。即 $2 \le a \le k$ 且 $2 \le b \le k$。

    因此,根据归纳假设,$F(a)$ 和 $F(b)$ 成立。也就是说 $a$ 和 $b$ 都存在质因数分解。将这两个质因数分解相乘就得到 $a \cdot b = k+1$ 的质因数分解。这表明 $F(k+1)$ 成立。

    无论是上面哪种情况,我们都推导出 $F(k+1)$ 成立。

    根据归纳法,我们得到 $\forall n \in \mathbb{N} - \{1\} \centerdot F(n)$。
\end{proof}

注意,在这个证明中我们应用了归纳假设(IH),但我们并不知道具体引用的是哪个``之前的实例''。我们只能依赖某个具有特定属性的 $a$ 和 $b$。这与之前的例子不同,但清楚表明我们这里需要使用强归纳法。关于 $k$ 的质因数分解无法帮助我们找到 $k+1$ 的质因数分解。想一想:知道 $14 = 2 \cdot 7$ 能帮助我们得到 $15 = 3 \cdot 5$ 吗?知道 $16 = 2^4$ 能帮助我们得到 $17$ 是质数吗?

我们刚刚证明的这个结果非常重要:它表明每个自然数都有一个质因数分解。质因数分解的\textbf{唯一性}也成立,即每个自然数都\emph{有且只有}一个质因数分解。当然,因数分解与``因数的顺序''无关。也就是说,$6 = 2 \cdot 3$ 和 $6 = 3 \cdot 2$ 实际上是相同的因数分解。同样地,$252 = 2 \cdot 2 \cdot 3 \cdot 3 \cdot 7$ 是 $252$ 的唯一分解;这与写作 $252 = 7 \cdot 3^2 \cdot 2^2$ 没有区别。

然而,上述证明中没有涉及这一事实!我们只是利用了某个 $a$ 和 $b$ 的存在来推导出一些结论。谁能说我们不能用具有相同属性的其他 $c$ 和 $d$ 呢?想一想。你能证明质因数分解的唯一性吗?你会使用什么方法?

下一个例子将涉及\textbf{斐波那契数列},这是我们之前讨论过的数列。具体来说,我们将陈述并证明该数列的\textbf{封闭形式},斐波那契数列通常是递归定义的。所谓``封闭形式'',指的是可以直接代入并计算的表达式。例如,要找到 $f_100$,使用递归定义的数列,我们必须计算到那一点为止的所有数列:我们需要 $f_99$ 和 $f_98$,这意味着我们需要 $f_97$,这意味着……然而,通过封闭形式,我们只需``代入 $n$''从而直接计算出 $f_100$。\\

\begin{example}[斐波那契数列的封闭形式]

    \textbf{声明}:斐波那契数列的标准定义如下:
    \[f_0 = 0 \;\text{ 且 }\; f_1 = 1 \;\text{ 且 }\; \forall n \in \mathbb{N}-\{1\} \centerdot f_n = f_{n-1} + f_{n-2}\]
    定义 $\varphi = \frac{1+\sqrt{5}}{2}$。则下面等式对于所有 $n \in \mathbb{N}-\{1\}$ 都成立:
    \[f_n = \frac{1}{\sqrt{5}}\big(\varphi^n-(1-\varphi)^n\big)\]
\end{example}

\begin{proof}
    设 $f_n$ 和 $\varphi$ 如声明所定义。

    我们首先证明下面的等式成立:
    \[1+\varphi=\varphi^2 \tag{*1}\]
    不难发现
    \begin{align*}
        \varphi^2 &= \Big(\frac{1+\sqrt{5}}{2}\Big)^2 = \frac{1+2\sqrt{5}+5}{4} = \frac{6+2\sqrt{5}}{4} \\
        &= \frac{3+\sqrt{5}}{2} = 1+\frac{1+\sqrt{5}}{2} = 1+\varphi \\
    \end{align*}
    接着,我们利用这个等式证明下面的等式:
    \[2-\varphi=(1-\varphi)^2 \tag{*2}\]
    利用 (*1) 的结论,很容易推出
    \[(1-\varphi)^2 = 1-2\varphi+\varphi^2 = 1-2\varphi+(\varphi+1) = 2-\varphi\]

    令 $P(n)$ 为命题
    \[f_n = \frac{1}{\sqrt{5}}\big(\varphi^n-(1-\varphi)^n\big)\]
    我们通过对 $n$ 应用归纳法来证明 $\forall n \in \mathbb{N} \cup \{0\} \centerdot P(n)$。

    \textbf{基本情况}:$f(0) = 0$ 且
    \[\frac{1}{\sqrt{5}}\big(\varphi^0-(1-\varphi)^0\big) = \frac{1}{\sqrt{5}} (1-1) = 0\]
    因此,$P(0)$ 成立。

    \textbf{归纳假设}:设 $k \in \mathbb{N} \cup \{0\}$ 是任意固定的,假设 $\forall i \in [k] \cup \{0\} \centerdot P(i)$ 成立。

    \textbf{归纳步骤}:我们的目标是推导出 $P(k+1)$ 成立。

    \textbf{情况 1}:假设 $k=0$,我们可以很容易推导出 $f_1=1$ 且
    \[\frac{1}{\sqrt{5}}\big(\varphi^1-(1-\varphi)^1\big) = \frac{1}{\sqrt{5}} (2\varphi-1) = \frac{1}{\sqrt{5}} (1+\sqrt{5}-1) = \frac{1}{\sqrt{5}} (\sqrt{5}) = 1\]
    这表明 $P(1)$ 成立。
    \textbf{情况 2}:假设 $k \ge 1$,则
    \begin{align*}
        f_{k+1} &= f_k+f_{k-1} & \text{因为 } k \ge 1, \textbf{ 根据定义} \\
        &= \frac{1}{\sqrt{5}}\big(\varphi^k-(1-\varphi)^k\big) + \frac{1}{\sqrt{5}}\big(\varphi^{k-1}-(1-\varphi)^{k-1}\big) & \text{归纳假设 } P(k), P(k-1)\\
        &= \frac{1}{\sqrt{5}}\big(\varphi^k+\varphi^{k-1}-(1-\varphi)^k-(1-\varphi)^{k-1}\big) & \textbf{化简}\\
        &= \frac{1}{\sqrt{5}}\big(\varphi^{k-1}(\varphi+1)-(1-\varphi)^{k-1}((1-\varphi)+1)\big) & \textbf{提取公因式}\\
        &= \frac{1}{\sqrt{5}}\big(\varphi^{k-1} \cdot \varphi^2-(1-\varphi)^{k-1}(2-\varphi)\big) & \textbf{根据(*1)}\\
        &= \frac{1}{\sqrt{5}}\big(\varphi^{k+1} -(1-\varphi)^{k-1}(1-\varphi)^2\big) & \textbf{根据(*2)}\\
        &= \frac{1}{\sqrt{5}}\big(\varphi^{k+1} -(1-\varphi)^{k+1}\big) & 
    \end{align*}
    因此, $P(k+1)$ 成立。

    根据归纳法,我们得出 $\forall n \in \mathbb{N} \cup \{0\} \centerdot P(n)$。
\end{proof}

\subsubsection*{多基本情况讨论}

注意,在前面的例子中,我们需要在归纳步骤(IS)中考虑两种情况。由于斐波那契数列是递归定义的,每一项都依赖于前两项,所以我们不能仅凭 $P(0)$ 成立来推导出 $P(1)$ 成立。我们必须单独证明 $P(1)$ 成立。(你可以回头试试,你会发现自己不得不引用 $f_{-1}$,这是一个未定义的项!)接下来,我们可以利用 $P(0)$ 和 $P(1)$ 成立来推导出 $P(2)$,然后再用 $P(1)$ 和 $P(2)$ 来推导出 $P(3)$ ……也就是说,我们确实需要在整个归纳法的推理开始之前,添加一个额外的基本情况。

处理这个问题有两种合法的方法,我们刚刚展示了其中一种。另一种方法是提前认识到这种情况,并在基本情况步骤中提出两个基本情况。为了说明这一点,让我们展示一下如果我们采用第二种方法,证明的相关部分会有什么不同:

\begin{proof}

    $\dots$

    $\dots$

    \textbf{基本情况}:$f(0) = 0$ 且
    \[\frac{1}{\sqrt{5}}\big(\varphi^0-(1-\varphi)^0\big) = \frac{1}{\sqrt{5}} (1-1) = 0\]
    因此,$P(0)$ 成立。

    并且 $f(1) = 1$ 且
    \[\frac{1}{\sqrt{5}}\big(\varphi^1-(1-\varphi)^1\big) = \frac{1}{\sqrt{5}} (2\varphi-1) = \frac{1}{\sqrt{5}} (1+\sqrt{5}-1) = \frac{1}{\sqrt{5}} (\sqrt{5}) = 1\]
    因此,$P(1)$ 成立。

    \textbf{归纳假设}:设 $k \in \mathbb{N}$ 是任意固定的,假设 $\forall i \in [k] \cup \{0\} \centerdot P(i)$ 成立。

    \textbf{归纳步骤}:我们的目标是推导出 $P(k+1)$ 成立。

    $\dots$

    $\dots$
\end{proof}

我们将特殊的 $P(1)$ 情况移到了基本情况(BC)部分。因此,我们必须修改归纳假设和归纳步骤中的量化过程。在随后的论证中,我们不再使用 $k = 0$,因此在归纳假设中,我们只取任意满足 $k \ge 1$ 的 $k$。然而,我们已经知道 $P(0)$ 成立,所以我们仍然可以在归纳假设中包含它。

就是这样!这两个证明在本质上是相同的。唯一的区别在于它们的呈现方式,即使如此,这些区别也很小。你可以自由决定在你的证明中更喜欢使用哪种风格(如果有的话)。不过,我们想提醒你,这些区别虽然很小,但也很微妙,有时容易被忽略!如果你发现自己包含了许多基本情况,请确保在开始你的归纳步骤时,寻求证明一个高于这些基本情况的值!你不希望无意中断言某些实际上并不成立的逻辑推论。(例如,回顾上面的第二个证明。如果我们允许 $k = 0$ 作为归纳步骤中的一个情况,我们将无意中引用 $f_{-1}$,而它并不存在。因此,我们会说一些不正确的话,证明将是有缺陷的,尽管不会完全失败。)

这种区别通常发生在你被要求证明某个递归定义数列的表达式时,其中序列中的每一项由几个前面的项定义。本节和本章末尾的练习中有许多这种类型的例子。在做这些练习时请牢记这一点!

\subsubsection*{需要证明 $n=2$ 的情况}

在强归纳法的证明中,常常需要先证明 $n = 1$ 和 $n = 2$ 的情况再进行归纳假设。特别是,当你需要证明某个不等式或等式对 $n$ 个变量成立时,$n = 1$ 的情况通常比较简单,而 $n = 2$ 的情况则更为复杂,需要更多的工作。然后,剩下的归纳证明则可以通过引用 $n = 2$ 的情况来完成。当然,这意味着在归纳假设中需要取 $k \ge 2$。

我们来看一个例子来说明这个问题。幸运的是,我们已经证明了这个命题在 $n = 2$ 的情况下成立;实际上,它就是集合的德摩根定律之一!\\

\begin{example}[集合的广义德摩根定律:]

    \textbf{声明}:设 $U$ 为全集。对于每个 $i \in \mathbb{N}$,设 $A_i \subseteq U$ 为集合。则下面等式等于所有 $n \in \mathbb{N}$ 成立:
    \[\overline{\bigcup_{i=1}^{n} A_i} = \bigcup_{i=1}^{n} \overline{A_i}\]
    换种写法,该声明说的是对于所有 $n \in \mathbb{N}$
    \[\overline{A_1 \cup A_2 \cup \dots \cup A_n} = \overline{A_1} \cap \overline{A_2} \cap \dots \cap \overline{A_n}\]
\end{example}

\begin{proof}
    设 $U$ 和 $A_1, A_2, \dots$ 如声明定义。

    设 $P(n)$ 为命题
    \[\overline{\bigcup_{i=1}^{n} A_i} = \bigcup_{i=1}^{n} \overline{A_i}\]
    我们通过对 $n$ 应用归纳法来证明 $\forall n \in \mathbb{N} \centerdot P(n)$。

    \textbf{基本情况}:显然 $\overline{A_1} = \overline{A_1}$,所以 $P(1)$ 成立。

    并且根据集合的德摩根定律(见定理 \ref{theorem4.6.9})我们有 $\overline{A_1 \cup A_2} = \overline{A_1} \cap \overline{A_2}$,所以 $P(2)$ 成立。

    \textbf{归纳假设}:设 $k \in \mathbb{N}-\{1\}$ 是任意固定的,假设 $\forall i \in [k] \centerdot P(i)$ 成立。

    \textbf{归纳步骤}:我们的目标是推导出 $P(k+1)$ 成立。

    首先,根据并集的定义,我们有
    \[\bigcup_{i=1}^{k+1} A_i = A_{k+1} \cup \bigcup_{i=1}^k A_i\]

    不妨设
    \[B_k = \bigcup_{i=1}^k A_i\]

    则
    \begin{align*}
        \overline{\bigcup_{i=1}^{k+1} A_i} &= \overline{A_{k+1} \cup B_k} & \text{根据 } B_k \text{ 的定义} \\
        &= \overline{A_{k+1}} \cap \overline{B_k} & \text{根据基本情况 } P(2) \text{(即德摩根定律)}\\
        &= \overline{A_{k+1}} \cap \overline{\bigcup_{i=1}^k A_i} & \text{根据 } B_k \text{ 的定义} \\
        &= \overline{A_{k+1}} \cap \bigcap_{i=1}^k \overline{A_i} & \text{根据归纳假设 } P(k) \\
        &= \bigcap_{i=1}^{k+1} \overline{A_i} & \text{化简}
    \end{align*}

    因此 $P(k+1)$ 成立。

    根据归纳法,$\forall n \in \mathbb{N} \centerdot P(n)$。
\end{proof}
