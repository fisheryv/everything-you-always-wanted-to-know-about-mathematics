% !TeX root = ../../book.tex
\section{总结}

现在,我们终于将\textbf{归纳法}建立在了坚实的数学基础之上!我们已经为此努力了相当长一段时间,因此当我们最终达到这一点时,我们希望全面呈现它。我们仔细地正式阐述并证明了数学归纳原理,并考察了其在实际中的几个应用示例。接着,我们利用数学归纳原理证明了更加通用的\textbf{强}归纳法原理。在此过程中,我们指出任何归纳证明实际上都\emph{可以是}强归纳证明,因为一种技术``包含''了另一种技术。此外,我们后来在讨论 $\mathbb{N}$ 的良序原则时证明了这两种归纳原则是\emph{逻辑等价}的(也与良序原则等价)。

我们探索了几种归纳法的变体,并为每一种变体提供了一两个示例。稍后我们用到了一种非常有用的技术即``最小罪犯''论证,这是一种归纳证明,其中归纳步骤证明了所需条件陈述的\emph{逆否命题}。

对于所有这些归纳法的变体,我们提供了一些证明模板。未来请参考这些模板,使用它们来使你的证明结构清晰、条理分明且易于理解。这不仅会使读者更容易理解你的证明,还会强调这些证明技术背后的重要概念。这些模板并非无的放矢,而是坚实地基于基本原理!

下面的练习将带给你大量使用各种归纳论证的实践机会。我们设计了一些比第 \ref{ch:chapter02} 章中的问题更具挑战性的问题,因为我们现在已经深入研究了归纳原理,并对使用它解决问题充满信心。此外,你在这些问题中证明的一些结果是有趣且有用的知识,我们甚至可能在本书后续的内容中引用这些结果!