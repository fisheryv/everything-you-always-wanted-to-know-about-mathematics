% !TeX root = ../../../book.tex
\subsection{``最小罪犯''论证}\label{sec:section5.5.1}

\subsubsection*{利用逆否命题}

请记住,条件陈述逻辑等价于其逆否陈述。而且,关于归纳法的定理陈述中都包含有条件陈述。它们通常出现在条件 (2) 中,表示归纳假设 (IH) 和归纳步骤 (IS) 的作用。那么,如果我们考虑这些条件陈述的逆否形式,会发生什么呢?这不会改变定理的真实性,但确实会影响我们如何将归纳法用作证明技术。让我们来探讨一下!

以下是强归纳法的条件陈述:
\[\forall k \in \mathbb{N} \centerdot \big(\forall i \in [k] \centerdot P(i)\big) \implies P(k+1)\]
两边同时取否并调换方向,我们便得到了其逆否形式:
\[\forall k \in \mathbb{N} \centerdot \neg P(k+1) \implies \big(\exists i \in [k] \centerdot \neg P(i)\big)\]
在应用强归纳法时,我们要证明 $P(1) \dots$ 到 $P(k)$ 一起可以推出 $P(k+1)$。这个陈述的新版本采用了一种不同的方法:假设 $P(k+1)$ 为假,然后推导出之前的某个实例也为假。

\subsubsection*{工作机制}

从技术角度讲,这里并没有什么新内容!这种方法有效是因为条件陈述和它的逆否命题逻辑等价。然而,这种方法让人感觉有些不爽。倒着推论,假设我们的命题在某处失败,从而证明它在\emph{更早的}地方也失败,这种方法听起来很奇怪。这难道不是我们要做的事情的反面吗?这种方法的关键有两点:
\begin{enumerate}[label=(\arabic*)]
    \item 我们已经建立了一个基本情况;
    \item 这种``更早失败''的论证是针对\emph{任意} $k$ 提出的。
\end{enumerate}

我们的思路是这样的。假设我们有一个命题 $P(n)$,并且我们想要证明 $\forall n \in \mathbb{N} \centerdot P(n)$。首先,我们证明 $P(1)$ 成立。这很好。接下来,我们假设对于某个\emph{任意} $k \in \mathbb{N}, P(k+1)$ 失败。(注意 $k + 1 \ge 2$,所以我们并不是假设 $P(1)$ 失败,因为我们已经知道它成立。)通过一些推理,我们得出一个更早的实例也失败。假设对于某个满足 $1 \le \ell \le k$ 的 $\ell, P(\ell)$ 失败。

我们刚才提出的论点适用于\emph{任意} $k$,因此同样的论点也适用于我们生成的新值 $\ell$。这保证了在某些满足 $1 \le m \le \ell-1$ 的 $m$ 上,命题 $P(m)$ 会失败。然后,我们可以将同样的论点重新应用于 $m$ 的值,这样不断重复下去……你可能已经看出这个过程会如何发展下去了。最终,我们会``用完''命题可能失败的所有先前实例,最终必须回到 $P(1)$。而我们已经知道 $P(1)$ 是成立的!

这里的核心思想可以总结如下:如果我们有一个有效的基本情况,并且没有最小的失败实例,那么命题在所有情况下都成立。这就是``最小罪犯''(Minimal Criminal)这个短语的由来。(当然,这个名称既形象又有趣。) ``罪犯''(Criminal)指的是命题失败的实例,而证明以下推论
\[\forall k \in \mathbb{N} \centerdot \neg P(k+1) \implies \exists i \in [k] \centerdot \neg P(i)\]
相当于表明没有``最小''的这种实例。

另一个表达相同思想的短语是``无最小反例''(No Least Counterexample)。你可能在其他书中看到过这个短语,要知道它指的是同一个思想。它传达了一个观点,即没有一个反例使得所有之前的实例都成立。另外,这种方法的另一个术语是``无穷递降''(Infinite Descent)。虽然这个短语不那么直观,但它实际上描述了我们方法的工作机制。通过证明我们总能找到一个更小的反例,我们表明存在一个命题失败的``向后''实例序列。然而,这个序列不能是``无穷递降''的,因为我们最终会遇到 $P(1)$,而我们已经证明 $P(1)$ 是有效的。请注意,这两个术语都可以使用。我们选择``最小罪犯''是因为它更有趣。

\subsubsection*{证明模板}

我们先简要展示一下如何编写这样一个证明模板,然后直接给出一个有趣事实的示例证明。这里没有什么特别新颖的内容。我们只是将直接证明策略应用于 $\implies$ 陈述,只不过这个语句是我们之前见过的陈述的逆否命题。

\subsubsection*{\textcolor{blue}{``最小罪犯论证证明''模板}}

\setlength{\fboxrule}{2pt}
\setlength\fboxsep{5mm}
\begin{center}
\noindent \fcolorbox{blue}{white}{%
    \parbox{0.8\textwidth}{%
        \linespread{1.5}\selectfont
        \textbf{目标:} 证明 $\forall n \in \mathbb{N} \centerdot P(n)$
        \begin{proof}\\
            设 $P(n)$ 为命题 ``$\underline{\qquad\qquad\qquad}$''。\\
            我们对 $n$ 采用归纳法(``最小罪犯''论证)证明 $\forall n \in \mathbb{N} \centerdot P(n)$。\\
            \textbf{基本情况}:$P(1)$ 成立,因为 $\underline{\qquad\qquad\qquad}$。\\
            \textbf{归纳假设}:设 $k \in \mathbb{N}$ 是任意固定的,假设 $P(k)$ 不成立。\\
            \textbf{归纳步骤}:推导出 $\exists \ell \in \mathbb{N}$ 满足 $1 \le \ell \le k$ 使得 $P(\ell)$ 不成立。\\
            由此可得 $\forall n \in \mathbb{N} \centerdot P(n)$。
        \end{proof}
    }
}
\end{center}

如果你担心忘记这个模板的技术细节,只需记住核心思想:
\begin{quotation}
    所谓``最小罪犯''论证,是通过应用归纳证明中通常的归纳假设 (IH) 和归纳步骤 (IS) 的\textbf{逆否命题}来进行的。
\end{quotation}

\subsubsection*{示例}

以下结果本身就很值得关注。(事实上,我们将在 \ref{sec:section7.6.3} 节讨论无限集的``大小''时用到它,是不是很有趣?)我们鼓励你在开始证明之前先试着理解这个命题。试着理解它为什么是正确的以及它是如何运作的。用小的 $n$ 值来检验。然后,当你阅读证明时,看看你的草稿是否反映了你可能观察到的模式。\\

\begin{example}[将自然数唯一地表示为乘积]
    
    \textbf{声明:}每个 $n \in \mathbb{N}$ 都可以\emph{唯一地}表示为 $2$ 的幂乘以一个奇数。即
    \[\forall n \in \mathbb{N} \centerdot \exists m, \ell \in \mathbb{N} \cup \{0\} \centerdot n = 2^m \cdot (2\ell + 1)\]
    并且存在唯一的 $\ell, m$ 满足此等式。
\end{example}

\begin{proof}
    我们通过对 $n$ 应用归纳法来证明这个命题;具体来说,我们使用``最小罪犯''论证法。

    \textbf{基本情况}:不难发现 $n = 1$ 可以表示为 $1 = 2^0 \cdot (2 \cdot 0 + 1)$。此外,这是唯一的表示,因为任何其他 $2$ 的幂都会使乘积大于等于 $2$,而任何其他奇数都会使乘积至少为 $3$。

    \textbf{归纳假设}:设 $k \in \mathbb{N}$ 是任意固定的,假设 $P(k+1)$ 不成立,即 $k+1$ 没有这样的表示,或者有多个这样的表示。我们将根据 $k+1$ 的奇偶性分为两种情况。

    \textbf{情况 1}:假设 $k+1$ 为偶数。这意味着 $\frac{k+1}{2} \in \mathbb{N}$。

    首先,假设 $k+1$ \emph{没有}这样的表示,那么 $\frac{k+1}{2}$ 也没有。因为如果 $\frac{k+1}{2}$ 有这样的表示,我们可以乘以 $2$ (将 $2$ 的幂加 $1$)从而得到 $k+1$ 的表示。

    因此,在这种情况下 $P(\frac{k+1}{2})$ 不成立(因为没有这样的表示)。

    其次,假设 $k+1$ 有至少\emph{两种}这样的表示:
    \[k + 1 = 2^{m_1}(2\ell_1 + 1) \qquad k + 1 = 2^{m_2}(2\ell_2 + 1)\]
    假设上面两种表示是不同的,即 $(m_1, \ell_1) \ne (m_2, \ell_2)$。因为 $k+1$ 为偶数,我们知道 $m_1, m_2 > 1$,将 $2$ 的幂降 $1$ 得
    \[\frac{k+1}{2}=2^{m_1-1}(2\ell_1+1) \qquad \frac{k+1}{2}=2^{m_2-1}(2\ell_2+1)\]
    $\frac{k+1}{2}$ 的两种表示方式(注意 $m_1-1, m_2-1 \ge 0$)。这是两种\emph{不同}的表示方式,因为根据之前的假设 $(m_1-1, \ell_1) \ne (m_2-1, \ell_2)$。

    因此,在这种情况下 $P(\frac{k+1}{2})$ 不成立(因为不唯一)。

    \textbf{情况 2}:假设 $k+1$ 为奇数。这意味着 $\exists \ell \in \mathbb{N} \cup \{0\} \centerdot k+1 = 2\ell + 1$。我们当然可以将 $k+1$ 表示为:
    \[k+1 = 2^0 \cdot (2\ell + 1)\]
    同时,这是唯一的表示方法。采用 $2$ 的不同次幂会使乘积为偶数(但 $k+1$ 为奇数)。而采用不同奇数因子会改变乘积。因此,这种情况产生了矛盾。$\hashx$

    根据归纳法,$\forall n \in \mathbb{N} \centerdot P(n)$ 成立。
\end{proof}

有趣吧?实际上,这个证明比我们一开始提到的要复杂一些。具体来说,基于奇偶性的情况使得这个问题有点复杂。其中一个情况(偶数的情况)遵循``最小罪犯''论证法。另一个情况(奇数的情况)实际上可以真正证明出来。在这个证明中,我们假设 $P(k+1)$ 不成立,但后来发现当 $k+1$ 为奇数时实际上是成立的。这就是矛盾所在。回想起来,这似乎有点迂回,但它允许我们将整个证明作为``最小罪犯''论证来呈现,而不是分别进行两个独立的证明,一个针对奇数,一个针对偶数。

此外,我们不仅要证明这些表示的存在性,还要证明它们的\emph{唯一性}。这就是为什么在 $k+1$ 为偶数的情况下需要进行两方面的考虑。为了证明这些表示的存在性,我们必须证明 $k+1$ 不可能没有表示;为了证明唯一性,我们必须证明 $k+1$ 不可能有两个或以上表示。
