% !TeX root = ../../../book.tex
\subsection{``最小罪犯''论证}\label{sec:section5.5.1}

\subsubsection*{利用逆否命题}

请记住,条件陈述与其逆否形式逻辑等价。由于归纳法定理的陈述通常包含条件陈述(尤其出现在条件 (2) 中,用以表达归纳假设(IH) 与归纳步骤 (IS) 的作用),若考虑这些条件陈述的逆否形式会如何?这虽不会改变定理的真实性,却会影响归纳法作为证明技术的应用方式。让我们深入探讨!

以下是强归纳法的条件陈述:
\[\forall k \in \mathbb{N} \centerdot \big(\forall i \in [k] \centerdot P(i)\big) \implies P(k+1)\]

两边同时取否并调换方向,便得到了其逆否形式:
\[\forall k \in \mathbb{N} \centerdot \neg P(k+1) \implies \big(\exists i \in [k] \centerdot \neg P(i)\big)\]

在应用强归纳法时,我们需要证明 $P(1) \dots$ 到 $P(k)$ 共同蕴涵 $P(k+1)$。而上述新形式则采用不同思路:假设 $P(k+1)$ 为\verb|假|,进而推导出存在某个先前命题实例也为\verb|假|。

\subsubsection*{工作机制}

从技术角度讲,这里并没有什么新内容!这种方法有效是因为条件陈述与其逆否形式逻辑等价。然而,这种方法可能让人感觉有些不自然。逆向推理——假设命题在某处失败,并证明它在\emph{更早的}地方也失败,这听起来有些奇怪。这难道不是与我们的目标相反吗?这种方法的关键有两点:
\begin{enumerate}[label=(\arabic*)]
    \item 我们已经建立了一个基本情况;
    \item 这种``更早失败''的论证是针对\emph{任意} $k$ 提出的。
\end{enumerate}

我们的思路如下。假设有一个命题 $P(n)$,需要证明 $\forall n \in \mathbb{N} \centerdot P(n)$。首先,证明 $P(1)$ 成立,这为证明奠定了基础。接下来,假设对于某个\emph{任意} $k \in \mathbb{N}$, $P(k+1)$ 不成立。(注意 $k + 1 \ge 2$,因此我们没有假设 $P(1)$ 不成立,因为 $P(1)$ 已证成立。)通过推理,推导出对于某个满足 $1 \le \ell \le k$ 的 $\ell$, $P(\ell)$ 不成立。

上述论证适用于\emph{任意} $k$,因此同样适用于新值 $\ell$。这确保了存在某个满足 $1 \le m \le \ell-1$ 的 $m$,使得 $P(m)$ 不成立。然后,可以将同样的论证应用于 $m$,并重复此过程……可以看出,该过程会持续进行。最终,我们将穷尽所有可能失败的先前实例,必须回到 $P(1)$,而 $P(1)$ 已被证明成立。

这里的核心思想可以总结如下:如果我们有一个有效的基本情况,且不存在最小的失败实例,那么命题在所有情况下都成立。这就是``最小罪犯 (Minimal Criminal)''的由来(当然,这个名称既形象又有趣)。``罪犯 (Criminal)''指命题失败的实例,而证明以下推论
\[\forall k \in \mathbb{N} \centerdot \neg P(k+1) \implies \exists i \in [k] \centerdot \neg P(i)\]
等价于表明不存在``最小''的此类实例。

另一个表达相同思想的短语是``无最小反例 (No Least Counterexample)''。你可能在其他书籍中见过这个短语,它表达的是相同的思想:不存在一个反例,使得所有更小的实例都成立。此外,另一个术语是``无穷递降 (Infinite Descent)''。虽然这个短语不够直观,但它描述了方法的工作机制:通过证明总能找到更小的反例,我们构建了一个命题失败的递减实例序列。然而,该序列不可能是``无穷递降''的,因为我们最终会到达 $P(1)$,而 $P(1)$ 已被证明成立。注意,这两个术语均可使用,我们选择``最小罪犯''是因为它更有趣。

\subsubsection*{证明模板}

我们首先简要展示该证明模板的编写方法,随后给出一个有趣命题的示例证明。本模板虽无特别创新之处,但其核心在于将直接证明策略应用于蕴含($\implies$)陈述,而该陈述恰好是已知命题的逆否形式。

\subsubsection*{\textcolor{blue}{``最小罪犯论证证明''模板}}

\setlength{\fboxrule}{2pt}
\setlength\fboxsep{5mm}
\begin{center}
\noindent \fcolorbox{blue}{white}{%
    \parbox{0.8\textwidth}{%
        \linespread{1.5}\selectfont
        \textbf{目标:} 证明 $\forall n \in \mathbb{N} \centerdot P(n)$
        \begin{proof}\\
            设 $P(n)$ 为命题``$\underline{\qquad\qquad\qquad}$''。\\
            我们对 $n$ 采用归纳法(``最小罪犯''论证)证明 $\forall n \in \mathbb{N} \centerdot P(n)$。\\
            \textbf{基本情况}:$P(1)$ 成立,因为 $\underline{\qquad\qquad\qquad}$。\\
            \textbf{归纳假设}:设 $k \in \mathbb{N}$ 为任意固定元素,假设 $P(k)$ 不成立。\\
            \textbf{归纳步骤}:推导出 $\exists \ell \in \mathbb{N}$ 满足 $1 \le \ell \le k$ 使得 $P(\ell)$ 不成立。\\
            由此可得 $\forall n \in \mathbb{N} \centerdot P(n)$。
        \end{proof}
    }
}
\end{center}

如果你担心忘记该模板的技术细节,只需记住其核心思想:
\begin{quotation}
    所谓``最小罪犯''论证,是通过应用归纳证明中通常归纳假设 (IH) 和归纳步骤 (IS) 的\textbf{逆否形式}完成推理。
\end{quotation}

\subsubsection*{示例}

以下结果本身就很值得关注。(事实上,我们将在 \ref{sec:section7.6.3} 节讨论无限集的``大小''时用到它,这很有趣,不是吗?)我们鼓励你在开始证明前先尝试理解这个命题:思考其正确性及运作机制,并用较小的 $n$ 值验证。当你阅读证明时,观察你的草稿是否捕捉到了所发现的模式。

\begin{example}[将自然数唯一地表示为乘积]
    
    \textbf{声明:}每个 $n \in \mathbb{N}$ 都可以\emph{唯一地}表示为 $2$ 的幂乘以一个奇数。即
    \[\forall n \in \mathbb{N} \centerdot \exists m, \ell \in \mathbb{N} \cup \{0\} \centerdot n = 2^m \cdot (2\ell + 1)\]

    并且存在唯一的 $\ell, m$ 满足此等式。

    \begin{proof}
        我们通过对 $n$ 应用归纳法来证明这个命题;具体来说,我们使用``最小罪犯''论证法。

        \textbf{基本情况}:不难发现 $n = 1$ 可以表示为 $1 = 2^0 \cdot (2 \cdot 0 + 1)$。并且,这是唯一的表示,因为任何其他 $2$ 的幂都会使乘积大于等于 $2$,而任何其他奇数都会使乘积至少为 $3$。

        \textbf{归纳假设}:设 $k \in \mathbb{N}$ 为任意固定自然数,假设 $P(k+1)$ 不成立,即 $k+1$ 不存在这样的表示,或者存在多个这样的表示。我们将根据 $k+1$ 的奇偶性分两种情况讨论。

        \textbf{情况 1}:假设 $k+1$ 为偶数。这意味着 $\frac{k+1}{2} \in \mathbb{N}$。

        首先,假设 $k+1$ \emph{没有}这样的表示,那么 $\frac{k+1}{2}$ 也没有。因为如果 $\frac{k+1}{2}$ 存在这样的表示,我们可以乘以 $2$ (将 $2$ 的幂加 $1$)从而得到 $k+1$ 的表示。

        因此,在这种情况下 $P(\frac{k+1}{2})$ 不成立(因为没有这样的表示)。

        其次,假设 $k+1$ 存在至少\emph{两种}这样的表示:
        \[k + 1 = 2^{m_1}(2\ell_1 + 1) \qquad k + 1 = 2^{m_2}(2\ell_2 + 1)\]

        假设上面两种表示是不同的,即 $(m_1, \ell_1) \ne (m_2, \ell_2)$。由 $k+1$ 为偶数可知 $m_1, m_2 > 1$,将 $2$ 的幂降 $1$ 可得 $\frac{k+1}{2}$ 的两种表示方式
        \[\frac{k+1}{2}=2^{m_1-1}(2\ell_1+1) \qquad \frac{k+1}{2}=2^{m_2-1}(2\ell_2+1) \quad (m_1-1, m_2-1 \ge 0)\]

        根据之前的假设 $(m_1-1, \ell_1) \ne (m_2-1, \ell_2)$,这是两种\emph{不同}的表示方式。

        因此,在这种情况下 $P(\frac{k+1}{2})$ 不成立(因为不唯一)。

        \textbf{情况 2}:假设 $k+1$ 为奇数。这意味着 $\exists \ell \in \mathbb{N} \cup \{0\} \centerdot k+1 = 2\ell + 1$。
        
        此时可以将 $k+1$ 表示为:
        \[k+1 = 2^0 \cdot (2\ell + 1)\]

        这是唯一的表示方法。采用 $2$ 的不同次幂会使乘积为偶数(但 $k+1$ 为奇数)。而采用不同奇数因子会改变乘积。因此,这种情况产生了矛盾。$\hashx$

        根据归纳法,$\forall n \in \mathbb{N} \centerdot P(n)$ 成立。
    \end{proof}
\end{example}

这很有趣,不是吗?实际上,这个证明比我们一开始提到的要复杂一些。具体来说,分奇偶性的讨论使其略显棘手。情况 1(偶数)遵循``最小罪犯''论证;而情况 2(奇数)实际上可以直接证明。证明中假设 $P(k+1)$ 不成立,但当 $k+1$ 为奇数时它却成立——这一矛盾构成了关键。回想起来,方法虽然迂回,但它允许我们用统一的``最小罪犯''论证替代两个分奇偶性的独立证明。

此外,我们需要同时证明该表示的\emph{存在性}与\emph{唯一性}。这解释了情况 1 的双重分析:为了证明存在性,需排除无表示的可能;为了证明唯一性,需排除多重表示的可能。
