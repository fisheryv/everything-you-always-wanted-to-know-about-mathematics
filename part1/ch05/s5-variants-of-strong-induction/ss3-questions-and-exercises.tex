% !TeX root = ../../../book.tex
\subsection{习题}

\subsubsection*{温故知新}

以口头或书面的形式简要回答以下问题。这些问题全都基于你刚刚阅读的内容,如果忘记了具体定义、概念或示例,可以回顾相关内容。确保在继续学习之前能够自信地作答这些问题,这将有助于你的理解和记忆!

\begin{enumerate}[label=(\arabic*)]
    \item ``最小罪犯 (Minimal Criminal)''论证与强归纳法 (Strong Induction) 证明有何区别?
    \item 我们已证明每个自然数 $n \in \mathbb{N}$ 均可表示为 $2$ 的幂与一个奇数的乘积。这种表示法还具有哪些其他特性?
    \item 我们已证明自然数集 $\mathbb{N}$ 是良序的。你认为整数集 $\mathbb{Z}$ 也具有此性质吗?有理数集 $\mathbb{Q}$ 呢?实数集 $\mathbb{R}$ 呢?
    \item 数学归纳法 (PMI)、强归纳法 (PSI) 和良序原理 (WOP) 在逻辑上等价,其具体含义是什么?
\end{enumerate}

\subsubsection*{小试牛刀}

尝试解答以下问题。这些题目需动笔书写或口头阐述答案,旨在帮助你熟练运用新概念、定义及符号。题目难度适中,确保掌握它们将大有裨益!

\begin{enumerate}[label=(\arabic*)]
    \item 请证明良序原理。这也是习题 \ref{exc:exercises5.7.21}。快来试试吧!
    \item 请证明 $\sqrt{3}$ 为无理数。\\
        (\textbf{提示}:为了引出矛盾而假设 $\sqrt{3} = \frac{a}{b}$,其中 $a, b \in \mathbb{N}$ 且该分数为最简形式。使用递降法 (descent argument) 反驳此处最简形式的假设。)
    \item 使用良序原理证明每个大于 $1$ 的自然数都可以表示为 $2$ 和 $3$ 的非负整数倍之和。\\
        例如:$2 = 2, \quad 8 = 6 + 2, \quad 101 = 3 \times 33 + 2$。
    \item 考虑方程:$4x^4 + 2y^4 = z^4$。本题将通过良序原理证明此方程在 $(x, y, z) \in \mathbb{N}^3$ 时\textbf{无解}。
        \begin{enumerate}[label=(\alph*)]
            \item 为了引出矛盾而假设 $(x, y, z) \in \mathbb{N}^3$ 是一个解,并进一步假设在所有解中,这个解的 $x$ 值是最小的。也就是说,我们定义
            \[T = \{x \in \mathbb{N} \mid \exists y, z \in \mathbb{N} \centerdot 4x^4 + 2y^4 = z^4\}\]
            并假设 $T$ 非空(即方程有解),因此 $T$ 存在一个最小元素。
            \item 证明 $z$ 为偶数。\\
            \textbf{提示}:在本部分和接下来的两部分中,可以使用以下事实——某个自然数 $m$ 的倍数和/差也是 $m$ 的倍数。
            \item 证明 $y$ 为偶数。
            \item 证明 $x$ 为偶数。
            \item 据此推导出存在另一个解 $(a, b, c)$,其第一个变量的值更小,即 $a < x$。
            \item 解释这一结论如何推出矛盾,从而证明方程无解。
        \end{enumerate}
\end{enumerate}