% !TeX root = ../../../book.tex
\subsection{习题}

\subsubsection*{温故知新}

以口头或书面的形式简要回答以下问题。这些问题全都基于你刚刚阅读的内容,所以如果忘记了具体的定义、概念或示例,可以回去重读相关部分。确保在继续学习之前能够自信地回答这些问题,这将有助于你的理解和记忆!

\begin{enumerate}[label=(\arabic*)]
    \item ``最小罪犯 (Minimal Criminal)''论证和强归纳法 (Strong Induction) 证明之间有什么区别?
    \item 我们证明了每个自然数 $n \in \mathbb{N}$ 都可以写成 $2$ 的幂和一个奇数的乘积。这种表示法还有什么其他特性?
    \item 我们证明了自然数集 $\mathbb{N}$ 是良序的。你认为整数集 $\mathbb{Z}$ 也有此性质吗?有理数集 $\mathbb{Q}$ 呢?实数集 $\mathbb{R}$ 呢?
    \item 数学归纳法 (PMI)、强归纳法 (PSI) 和良序原理 (WOP) 在逻辑上是等价的,这是什么意思?
\end{enumerate}

\subsubsection*{小试牛刀}

尝试回答以下问题。这些题目要求你实际动笔写下答案,或(对朋友/同学)口头陈述答案。目的是帮助你练习使用新的概念、定义和符号。题目都比较简单,确保能够解决这些问题将对你大有帮助!

\begin{enumerate}[label=(\arabic*)]
    \item 请证明良序原理。这也是习题 \ref{exc:exercises5.7.21}。快来试试吧!
    \item 请证明 $\sqrt{3}$ 为无理数。\\
        (\textbf{提示}:为了得到矛盾而假设 $\sqrt{3} = \frac{a}{b}$,其中 $a, b \in \mathbb{N}$ 且分数是最简形式。使用递降法 (descent argument) 来反驳这里最简形式的假设。)
    \item 使用良序原理证明每个自然数(除了 $1$)都可以写成 $2$ 和 $3$ 的非负整数倍之和。\\
        例如:$2 = 2, 8 = 6 + 2, 101 = 3 \times 33 + 2$。
    \item 考虑以下方程:$4x^4 + 2y^4 = z^4$。在这个问题中,你将通过良序原理来论证此方程在 $(x, y, z) \in \mathbb{N}^3$ 时\textbf{无解}。
        \begin{enumerate}[label=(\alph*)]
            \item 为了得到矛盾而假设 $(x, y, z) \in \mathbb{N}^3$ 是一个解,并进一步假设在所有解中,这个解的 $x$ 值是最小的。也就是说,我们定义
            \[T = \{x \in \mathbb{N} \mid \exists y, z \in \mathbb{N} \centerdot 4x^4 + 2y^4 = z^4\}\]
            并预设这个集合是非空的(即方程有解),因此 $T$ 有一个最小元素。
            \item 推导出 $z$ 为偶数。\\
            \textbf{提示}:在本部分和接下来的两部分中,你可以使用以下事实:某个自然数 $m$ 的倍数和/差也是 $m$ 的倍数。
            \item 推导出 $y$ 为偶数。
            \item 推导出 $x$ 为偶数。
            \item 据此推导出存在另一个解 $(a, b, c)$,其第一个变量的值更小,即 $a < x$。
            \item 解释为什么这证明了无解。
        \end{enumerate}
\end{enumerate}