% !TeX root = ../../../book.tex
\subsection{$\mathbb{N}$ 的良序原理} \label{sec:section5.5.2}

\subsubsection*{引言}

我们知道自然数集 $\mathbb{N}$ 上的关系``$\le$''具有以下性质:对于任意两个元素 $x, y \in \mathbb{N}$,必有 $x \le y$ 或 $y \le x$(当 $x = y$ 时两者同时成立)。此外:
\[\forall x, y, z \in \mathbb{N} \cdot (x \le y \land y \le z) \implies x \le z\]
且 $\forall x \in \mathbb{N}$ 有 $x \le x$。这使得 $\mathbb{N}$ 成为\textbf{有序集},我们称``$\le$''是 $\mathbb{N}$ 上的\emph{顺序关系}(详见 \ref{sec:section6.3} 节)。

此外,这种关系实际上是\textbf{良序关系}。良序关系的关键特征是不存在\emph{无穷递降链}:在 $\mathbb{N}$ 中不可能存在序列 $a_1, a_2, a_3, \dots$ 满足 $a_1 > a_2 > a_3 > \cdots$(注意此处不等式是严格大于)。因为从任意 $a_1 \in \mathbb{N}$ 开始``递降'',最终必然到达 $1$ 而终止。

本节不会深入讨论良序关系的一般理论(这属于集合论或形式逻辑范畴),而是聚焦于其在 $\mathbb{N}$ 上的应用。我们将阐述良序原理,引导读者完成证明,并揭示其与数学归纳法的联系。

\subsubsection*{陈述与证明}

\begin{theorem}\label{theorem5.5.2}
    $\mathbb{N}$ 的所有非空子集均有最小元素。其逻辑表述为:
    \[\forall S \in \mathcal{P}(\mathbb{N}) \centerdot [S \ne \varnothing \implies (\exists \ell \in S \centerdot (\forall x \in S \centerdot \ell \le x))]\]
\end{theorem}

想一想这与我们之前提到的自然数中不存在无穷递降链的关系。若存在无穷递降链,取链中元素构成集合 $S$,则 $S$ \textbf{没有}最小元素。因为对于任意 $a_n \in S$,存在 $a_{n+1} \in S$ 满足 $a_{n+1} < a_n$。

我们希望由你完成这个定理的证明,因为详细的推理过程将会大有裨益。证明分为几个步骤,关键之处在于其采用\textbf{归纳法}完成。通过这种方式证明良序原理,展示了数学归纳原理\emph{蕴含}良序原理。

\begin{proof}
    用归纳法证明。留作习题 \ref{exc:exercises5.7.21}。
\end{proof}

需额外指出的是,子集 $S \subseteq \mathbb{N}$ 的最小元素必\emph{唯一}。假设存在两个最小元 $\ell$ 和 $m$,由最小元素的定义有 $\ell \le m$ 且 $m \le \ell$,这意味着 $\ell = m$,所以它们实际上是同一元素!

\subsubsection*{归纳法、强归纳法与良序原理}

如前所述,使用归纳法证明良序原理表明数学归纳原理蕴含良序原理。接下来的定理说明,这两个原理\textbf{逻辑等价},它们互相蕴含。此外,强归纳原理与良序原理也互相蕴含。事实上,这三个定理在逻辑上等价!

\begin{theorem}
    以下三个命题逻辑等价:
    \begin{itemize}
        \item 数学归纳原理
        \item 强归纳原理
        \item 良序原理
    \end{itemize}
\end{theorem}

\begin{proof}
    使用以下简写表示各定理:
    \begin{itemize}
        \item \textbf{PMI}:数学归纳原理 (Principle of Mathematical Induction)
        \item \textbf{PSI}:强归纳原理 (Principle of Strong Induction)
        \item \textbf{WOP}:良序原理 (Well-Ordering Principle)
    \end{itemize}

    根据先前对 PSI 和 WOP 的证明,可以推断出
    \[\text{PMI} \implies \text{PSI} \quad\text{且}\quad \text{PMI} \implies \text{WOP}\]

    我们在 \ref{sec:section5.4.2} 节已证
    \[\text{PSI} \implies \text{PMI}\]

    因此可得
    \[\text{PMI} \iff \text{PSI} \quad\text{且}\quad \text{PMI} \implies \text{WOP}\]

    要完成证明,需要证明 $\text{WOP} \implies \text{PMI}$。由此可得 $\text{WOP} \iff \text{PMI}$。通过上述等价性,可以推断出这三个定理逻辑等价。

    为了证明这一点,我们假设 WOP 是成立,并用其来证明 PMI。(可参考定理 \ref{theorem5.2.2} 中 PMI 的陈述,以理解我们这样做的原因。)

    设命题 $P(n)$ 定义在 $n \in \mathbb{N}$ 上。假设 $P(1)$ 为\verb|真|,并且对于所有 $k \in \mathbb{N} \centerdot P(k) \implies P(k+1)$。我们需要证明对于所有的 $n \in \mathbb{N} \centerdot P(n)$ 都成立。

    定义集合 $F$ 为 $P(n)$ 的``失败实例''集合。即定义:
    \[F = \{n \in \mathbb{N} \mid P(n) \text{\ 为假}\}\]

    为了证明 $\forall n \in \mathbb{N} \centerdot P(n)$,假设 $F \ne \varnothing$。

    由于使用了集合构建符,所以 $F \subseteq \mathbb{N}$。根据之前的假设,$\exists f \in F$。给定该 $f$。

    基于这两个条件,WOP 适用于集合 $F$,这意味着 $F$ 存在一个最小元素。设 $\ell$ 为这个最小元素。可知 $\ell \in F$,并且
    \[\forall x \in F \centerdot \ell \le x\]

    考虑 $\ell = 1$ 的情况。这是不可能的,因为上面假设 $P(1)$ 成立,所以 $1 \notin F$。

    考虑 $\ell \ge 2$ 的情况。上面的假设指出
    \[\forall k \in \mathbb{N} \centerdot P(k) \implies P(k+1)\]

    这逻辑等价于其逆否命题
    \[\forall k \in \mathbb{N} - \{1\} \centerdot \neg P(k) \implies \neg P(k-1)\]

    将这一点应用到元素 $\ell \in \mathbb{N} - \{1\}$ 上,可以推断出 $\neg P(\ell - 1)$ 也成立。换句话说,$P(\ell-1)$ 为\verb|假|。

    这意味着 $\ell-1 \in F$。然而,这与 $\ell$ 是 $F$ 的最小元素相矛盾,因为 $\ell - 1 < \ell$。$\hashx$

    因此,$F = \varnothing$,这意味着 $\forall n \in \mathbb{N} \centerdot P(n)$。

    以上证明了 PMI 成立。
\end{proof}

回顾证明的核心部分:为证 $P(n)$ 对所有 $n$ 成立,假设其存在反例 $f \in F$。此时可能认为:若 $P(f)$ 不成立,则 $P(f-1)$ 不成立,进而 $P(f-2)$ 不成立,……\textbf{依此类推}至 $P(1)$,但 $P(1)$ 为\verb|真|。然而,这种``依次类推''的论证正是 PMI 和 WOP 旨在解决的问题!无法用模糊的``继续下去''证明此过程可行。因此需借助 WOP 找出 $F$ 的最小元。引入 $f \in F$ 看似未直接使用,实则 $f$ 的存在性保证了 $F \ne \varnothing$,从而能够应用 WOP。

\clearpage
