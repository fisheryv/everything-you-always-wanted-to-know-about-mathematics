% !TeX root = ../../../book.tex
\subsection{$\mathbb{N}$ 的良序原理} \label{sec:section5.5.2}

\subsubsection*{引言}

我们都知道自然数 $\mathbb{N}$ 上的关系``$\le$''。对于任意两个元素 $x, y \in \mathbb{N}$,总有以下两种情况之一:要么 $x \le y$,要么 $y \le x$(当且仅当 $x = y$ 时,两者皆满足)。我们还知道:
\[\forall x, y, z \in \mathbb{N} \centerdot (x \le y \land y \le z) \implies x \le z\]
且 $\forall x \in \mathbb{N}$,总有 $x \le x$。这使得 $\mathbb{N}$ 成为一个\textbf{有序}集,我们称``$\le$''是 $\mathbb{N}$ 上的一种\emph{顺序关系}。(详见 \ref{sec:section6.3} 节。)

此外,这种关系实际上是一种\textbf{良序关系}。我们不会正式定义这个术语,但良序关系的一个关键特点是不存在\emph{无穷递降链}。想象一下在 $\mathbb{N}$ 中是否存在一个无穷序列 $a_1, a_2, a_3, \dots$,使得 $a_1 > a_2 > a_3 > \dots$?这是不可能的!(注意这些不等式是严格的。)意思是说,从某个数字 $a_1 \in \mathbb{N}$ 开始,如果我们``递降'',最终会到达 $1$,我们不能再``递降''了。

我们不会全面讨论良序关系 --- 这可以在集合论或形式逻辑的课程中讨论 --- 而是专注于这一概念在 $\mathbb{N}$ 上的应用。这是一个很有用的性质,我们以后也会涉及到。在本节中,我们将阐述良序原理,并请你帮助我们证明它,然后展示它与数学归纳法之间的关系。

\subsubsection*{陈述与证明}

\begin{theorem}\label{theorem5.5.2}
    $\mathbb{N}$ 的所有非空子集都有一个最小元素。用逻辑形式表示为
    \[\forall S \in \mathcal{P}(\mathbb{N}) \centerdot [S \ne \varnothing \implies (\exists \ell \in S \centerdot (\forall x \in S \centerdot \ell \le x))]\]
\end{theorem}

想一想这与我们之前提到的自然数中不存在无穷递降链的关系。如果确实存在无穷递降链,我们可以定义 $S$ 为链中所有元素的集合。这个集合将\textbf{没有}最小元素。假设集合中有一个元素 $a_n$,那么 $a_{n+1}$ 也在集合中,且 $a_{n+1} < a_n$。因此,这个集合不会有最小元素。

我们希望你来证明这个定理,因为我们认为完成详细推理过程会大有裨益。证明分为几个步骤。一个关键点是,这个证明是通过\textbf{归纳法}完成的!也就是说,通过这种方式证明良序原理,我们可以展示数学归纳原理\emph{蕴含}了良序原理。

\begin{proof}
    用归纳法证明。留给读者作为习题 \ref{exc:exercises5.7.21}。
\end{proof}

一个简单的额外发现是,任何子集 $S \subseteq \mathbb{N}$ 的最小元素必须是\emph{唯一的}。也就是说,不可能存在两个(或更多)最小元素。假设集合 $S$ 中确实有两个最小元素,分别是 $\ell$ 和 $m$。根据最小元素的定义,我们知道 $\ell \le m$ 且 $m \le \ell$。这意味着 $\ell = m$,所以它们实际上是同一元素!

\subsubsection*{归纳法、强归纳法与良序原理}

如前所述,因为我们使用归纳法证明了良序原理,这说明数学归纳原理蕴含良序原理。接下来的定理实际上表明,这两个定理是\textbf{逻辑等价}的,它们互相蕴含。此外,它还表明强归纳原理也蕴含良序原理,反之亦然。事实上,这三个定理在逻辑上是等价的!

\begin{theorem}
    以下三个命题全都逻辑等价:
    \begin{itemize}
        \item 数学归纳原理
        \item 强归纳原理
        \item 良序原理
    \end{itemize}
\end{theorem}

\begin{proof}
    我们可以用以下简写来表示每个定理:
    \begin{itemize}
        \item \textbf{PMI}:数学归纳原理 (Principle of Mathematical Induction)
        \item \textbf{PSI}:强归纳原理 (Principle of Strong Induction)
        \item \textbf{WOP}:良序原理 (Well-Ordering Principle)
    \end{itemize}
    根据我们对 PSI 和 WOP 的证明,我们可以推断出
    \[\text{PMI} \implies \text{PSI} \quad\text{且}\quad \text{PMI} \implies \text{WOP}\]
    我们在 \ref{sec:section5.4.2} 节证明了
    \[\text{PSI} \implies \text{PMI}\]
    因此我们得到
    \[\text{PMI} \iff \text{PSI} \quad\text{且}\quad \text{PMI} \implies \text{WOP}\]
    要完成这个证明,我们需要证明 $\text{WOP} \implies \text{PMI}$。这样我们就证明了 $\text{WOP} \iff \text{PMI}$。通过上述等价性,我们可以推断出这三个定理在逻辑上是等价的。

    为了证明这一点,我们假设 WOP 是成立,并用它来证明 PMI。(可以回顾一下定理 \ref{theorem5.2.2} 中 PMI 的陈述,以理解我们这样做的原因。)

    假设有一个命题 $P(n)$,定义在 $n \in \mathbb{N}$ 上。假设 $P(1)$ 为\verb|真|,并且对于所有的 $k \in \mathbb{N} \centerdot P(k) \implies P(k+1)$。我们需要证明对于所有的 $n \in \mathbb{N} \centerdot P(n)$ 都成立。


    定义集合 $F$ 为 $P(n)$ 的``失败实例''集合。即定义:
    \[F = \{n \in \mathbb{N} \mid P(n) \text{ 为假}\}\]
    为了证明 $\forall n \in \mathbb{N} \centerdot P(n)$,我们将为了引出矛盾而假设 $F \ne \varnothing$。

    由于我们使用了集合构建符,所以 $F \subseteq \mathbb{N}$。根据之前的假设,$\exists f \in F$。给定该 $f$。

    基于这两个条件,WOP 适用于集合 $F$,这意味着 $F$ 有一个最小元素。设 $\ell$ 为这个最小元素。我们知道 $\ell \in F$,并且
    \[\forall x \in F \centerdot \ell \le x\]
    考虑 $\ell = 1$ 的情况。这是不可能的,因为我们上面的假设说 $P(1)$ 成立,所以 $1 \notin F$。

    考虑 $\ell \ge 2$ 的情况。我们上面的假设说
    \[\forall k \in \mathbb{N} \centerdot P(k) \implies P(k+1)\]
    这逻辑等价与其逆否命题
    \[\forall k \in \mathbb{N} - \{1\} \centerdot \neg P(k) \implies \neg P(k-1)\]
    将这一点应用到元素 $\ell \in \mathbb{N} - \{1\}$ 上,我们可以推断出 $\neg P(\ell - 1)$ 也成立。换句话说,$P(\ell-1)$ 为\verb|假|。

    这意味着 $\ell-1 \in F$。然而,这与 $\ell$ 是 $F$ 的最小元素相矛盾,因为 $\ell - 1 < \ell$。$\hashx$

    因此,必定是 $F = \varnothing$,而这意味着 $\forall n \in \mathbb{N} \centerdot P(n)$。

    这证明了 PMI 成立。
\end{proof}

让我们来看一下这个证明的核心部分。为了证明 $P(n)$ 对所有 $n$ 都成立,我们假设它对某个特定的 $n$ 会失败,即这个元素 $f \in F$。从这里开始,你可能会想,``如果 $P(f)$ 失败,那么 $P(f-1)$ 也会失败,接着 $P(f-2)$ 也会失败,……\textbf{依次类推},直到 $P(1)$,但是我们知道 $P(1)$ 为\verb|真|。''然而,这种``依次类推''的论证正是 PMI 和 WOP 所要解决的问题!你不能用一个含糊的``继续下去''来证明你可以这样做。这就是为什么我们要使用 WOP 来找出 $F$ 的最小元素。你可能会觉得我们引入 $f \in F$ 然后不再使用它很奇怪。其实,我们需要 $f$ 的存在来证明 $F \ne \varnothing$,从而能够应用 WOP。
