% !TeX root = ../../../book.tex
\section{强归纳法的变体}

如同常规归纳法存在多种变体(例如使用不同的基础情况、在不同集合上归纳或反向归纳等),强归纳法也有一个值得讨论的重要变体。你将发现,``最小罪犯''论证本质上是一种基于\emph{反证法}的强归纳证明。这种论证在归纳证明中颇为常见,理解其运作机制非常有用!

此外,我们将陈述并证明自然数集的一个重要性质——著名的\textbf{良序原理}。之所以在本节介绍它,是因为该原理与归纳法及强归纳法有着深刻的联系。

% !TeX root = ../../../book.tex
\subsection{``最小罪犯''论证}\label{sec:section5.5.1}

\subsubsection*{利用逆否命题}

请记住,条件陈述逻辑等价于其逆否陈述。而且,关于归纳法的定理陈述中都包含有条件陈述。它们通常出现在条件 (2) 中,表示归纳假设 (IH) 和归纳步骤 (IS) 的作用。那么,如果我们考虑这些条件陈述的逆否形式,会发生什么呢?这不会改变定理的真实性,但确实会影响我们如何将归纳法用作证明技术。让我们来探讨一下!

以下是强归纳法的条件陈述:
\[\forall k \in \mathbb{N} \centerdot \big(\forall i \in [k] \centerdot P(i)\big) \implies P(k+1)\]
两边同时取否并调换方向,我们便得到了其逆否形式:
\[\forall k \in \mathbb{N} \centerdot \neg P(k+1) \implies \big(\exists i \in [k] \centerdot \neg P(i)\big)\]
在应用强归纳法时,我们要证明 $P(1) \dots$ 到 $P(k)$ 一起可以推出 $P(k+1)$。这个陈述的新版本采用了一种不同的方法:假设 $P(k+1)$ 为假,然后推导出之前的某个实例也为假。

\subsubsection*{工作机制}

从技术角度讲,这里并没有什么新内容!这种方法有效是因为条件陈述和它的逆否命题逻辑等价。然而,这种方法让人感觉有些不爽。倒着推论,假设我们的命题在某处失败,从而证明它在\emph{更早的}地方也失败,这种方法听起来很奇怪。这难道不是我们要做的事情的反面吗?这种方法的关键有两点:
\begin{enumerate}[label=(\arabic*)]
    \item 我们已经建立了一个基本情况;
    \item 这种``更早失败''的论证是针对\emph{任意} $k$ 提出的。
\end{enumerate}

我们的思路是这样的。假设我们有一个命题 $P(n)$,并且我们想要证明 $\forall n \in \mathbb{N} \centerdot P(n)$。首先,我们证明 $P(1)$ 成立。这很好。接下来,我们假设对于某个\emph{任意} $k \in \mathbb{N}, P(k+1)$ 失败。(注意 $k + 1 \ge 2$,所以我们并不是假设 $P(1)$ 失败,因为我们已经知道它成立。)通过一些推理,我们得出一个更早的实例也失败。假设对于某个满足 $1 \le \ell \le k$ 的 $\ell, P(\ell)$ 失败。

我们刚才提出的论点适用于\emph{任意} $k$,因此同样的论点也适用于我们生成的新值 $\ell$。这保证了在某些满足 $1 \le m \le \ell-1$ 的 $m$ 上,命题 $P(m)$ 会失败。然后,我们可以将同样的论点重新应用于 $m$ 的值,这样不断重复下去……你可能已经看出这个过程会如何发展下去了。最终,我们会``用完''命题可能失败的所有先前实例,最终必须回到 $P(1)$。而我们已经知道 $P(1)$ 是成立的!

这里的核心思想可以总结如下:如果我们有一个有效的基本情况,并且没有最小的失败实例,那么命题在所有情况下都成立。这就是``最小罪犯''(Minimal Criminal)这个短语的由来。(当然,这个名称既形象又有趣。) ``罪犯''(Criminal)指的是命题失败的实例,而证明以下推论
\[\forall k \in \mathbb{N} \centerdot \neg P(k+1) \implies \exists i \in [k] \centerdot \neg P(i)\]
相当于表明没有``最小''的这种实例。

另一个表达相同思想的短语是``无最小反例''(No Least Counterexample)。你可能在其他书中看到过这个短语,要知道它指的是同一个思想。它传达了一个观点,即没有一个反例使得所有之前的实例都成立。另外,这种方法的另一个术语是``无穷递降''(Infinite Descent)。虽然这个短语不那么直观,但它实际上描述了我们方法的工作机制。通过证明我们总能找到一个更小的反例,我们表明存在一个命题失败的``向后''实例序列。然而,这个序列不能是``无穷递降''的,因为我们最终会遇到 $P(1)$,而我们已经证明 $P(1)$ 是有效的。请注意,这两个术语都可以使用。我们选择``最小罪犯''是因为它更有趣。

\subsubsection*{证明模板}

我们先简要展示一下如何编写这样一个证明模板,然后直接给出一个有趣事实的示例证明。这里没有什么特别新颖的内容。我们只是将直接证明策略应用于 $\implies$ 陈述,只不过这个语句是我们之前见过的陈述的逆否命题。

\subsubsection*{\textcolor{blue}{``最小罪犯论证证明''模板}}

\setlength{\fboxrule}{2pt}
\setlength\fboxsep{5mm}
\begin{center}
\noindent \fcolorbox{blue}{white}{%
    \parbox{0.8\textwidth}{%
        \linespread{1.5}\selectfont
        \textbf{目标:} 证明 $\forall n \in \mathbb{N} \centerdot P(n)$
        \begin{proof}\\
            设 $P(n)$ 为命题 ``$\underline{\qquad\qquad\qquad}$''。\\
            我们对 $n$ 采用归纳法(``最小罪犯''论证)证明 $\forall n \in \mathbb{N} \centerdot P(n)$。\\
            \textbf{基本情况}:$P(1)$ 成立,因为 $\underline{\qquad\qquad\qquad}$。\\
            \textbf{归纳假设}:设 $k \in \mathbb{N}$ 是任意固定的,假设 $P(k)$ 不成立。\\
            \textbf{归纳步骤}:推导出 $\exists \ell \in \mathbb{N}$ 满足 $1 \le \ell \le k$ 使得 $P(\ell)$ 不成立。\\
            由此可得 $\forall n \in \mathbb{N} \centerdot P(n)$。
        \end{proof}
    }
}
\end{center}

如果你担心忘记这个模板的技术细节,只需记住核心思想:
\begin{quotation}
    所谓``最小罪犯''论证,是通过应用归纳证明中通常的归纳假设 (IH) 和归纳步骤 (IS) 的\textbf{逆否命题}来进行的。
\end{quotation}

\subsubsection*{示例}

以下结果本身就很值得关注。(事实上,我们将在 \ref{sec:section7.6.3} 节讨论无限集的``大小''时用到它,是不是很有趣?)我们鼓励你在开始证明之前先试着理解这个命题。试着理解它为什么是正确的以及它是如何运作的。用小的 $n$ 值来检验。然后,当你阅读证明时,看看你的草稿是否反映了你可能观察到的模式。\\

\begin{example}[将自然数唯一地表示为乘积]
    
    \textbf{声明:}每个 $n \in \mathbb{N}$ 都可以\emph{唯一地}表示为 $2$ 的幂乘以一个奇数。即
    \[\forall n \in \mathbb{N} \centerdot \exists m, \ell \in \mathbb{N} \cup \{0\} \centerdot n = 2^m \cdot (2\ell + 1)\]
    并且存在唯一的 $\ell, m$ 满足此等式。
\end{example}

\begin{proof}
    我们通过对 $n$ 应用归纳法来证明这个命题;具体来说,我们使用``最小罪犯''论证法。

    \textbf{基本情况}:不难发现 $n = 1$ 可以表示为 $1 = 2^0 \cdot (2 \cdot 0 + 1)$。此外,这是唯一的表示,因为任何其他 $2$ 的幂都会使乘积大于等于 $2$,而任何其他奇数都会使乘积至少为 $3$。

    \textbf{归纳假设}:设 $k \in \mathbb{N}$ 是任意固定的,假设 $P(k+1)$ 不成立,即 $k+1$ 没有这样的表示,或者有多个这样的表示。我们将根据 $k+1$ 的奇偶性分为两种情况。

    \textbf{情况 1}:假设 $k+1$ 为偶数。这意味着 $\frac{k+1}{2} \in \mathbb{N}$。

    首先,假设 $k+1$ \emph{没有}这样的表示,那么 $\frac{k+1}{2}$ 也没有。因为如果 $\frac{k+1}{2}$ 有这样的表示,我们可以乘以 $2$ (将 $2$ 的幂加 $1$)从而得到 $k+1$ 的表示。

    因此,在这种情况下 $P(\frac{k+1}{2})$ 不成立(因为没有这样的表示)。

    其次,假设 $k+1$ 有至少\emph{两种}这样的表示:
    \[k + 1 = 2^{m_1}(2\ell_1 + 1) \qquad k + 1 = 2^{m_2}(2\ell_2 + 1)\]
    假设上面两种表示是不同的,即 $(m_1, \ell_1) \ne (m_2, \ell_2)$。因为 $k+1$ 为偶数,我们知道 $m_1, m_2 > 1$,将 $2$ 的幂降 $1$ 得
    \[\frac{k+1}{2}=2^{m_1-1}(2\ell_1+1) \qquad \frac{k+1}{2}=2^{m_2-1}(2\ell_2+1)\]
    $\frac{k+1}{2}$ 的两种表示方式(注意 $m_1-1, m_2-1 \ge 0$)。这是两种\emph{不同}的表示方式,因为根据之前的假设 $(m_1-1, \ell_1) \ne (m_2-1, \ell_2)$。

    因此,在这种情况下 $P(\frac{k+1}{2})$ 不成立(因为不唯一)。

    \textbf{情况 2}:假设 $k+1$ 为奇数。这意味着 $\exists \ell \in \mathbb{N} \cup \{0\} \centerdot k+1 = 2\ell + 1$。我们当然可以将 $k+1$ 表示为:
    \[k+1 = 2^0 \cdot (2\ell + 1)\]
    同时,这是唯一的表示方法。采用 $2$ 的不同次幂会使乘积为偶数(但 $k+1$ 为奇数)。而采用不同奇数因子会改变乘积。因此,这种情况产生了矛盾。$\hashx$

    根据归纳法,$\forall n \in \mathbb{N} \centerdot P(n)$ 成立。
\end{proof}

有趣吧?实际上,这个证明比我们一开始提到的要复杂一些。具体来说,基于奇偶性的情况使得这个问题有点复杂。其中一个情况(偶数的情况)遵循``最小罪犯''论证法。另一个情况(奇数的情况)实际上可以真正证明出来。在这个证明中,我们假设 $P(k+1)$ 不成立,但后来发现当 $k+1$ 为奇数时实际上是成立的。这就是矛盾所在。回想起来,这似乎有点迂回,但它允许我们将整个证明作为``最小罪犯''论证来呈现,而不是分别进行两个独立的证明,一个针对奇数,一个针对偶数。

此外,我们不仅要证明这些表示的存在性,还要证明它们的\emph{唯一性}。这就是为什么在 $k+1$ 为偶数的情况下需要进行两方面的考虑。为了证明这些表示的存在性,我们必须证明 $k+1$ 不可能没有表示;为了证明唯一性,我们必须证明 $k+1$ 不可能有两个或以上表示。


% !TeX root = ../../../book.tex
\subsection{$\mathbb{N}$ 的良序原理} \label{sec:section5.5.2}

\subsubsection*{引言}

我们都知道自然数 $\mathbb{N}$ 上的关系``$\le$''。对于任意两个元素 $x, y \in \mathbb{N}$,总有以下两种情况之一:要么 $x \le y$,要么 $y \le x$(当且仅当 $x = y$ 时,两者皆满足)。我们还知道:
\[\forall x, y, z \in \mathbb{N} \centerdot (x \le y \land y \le z) \implies x \le z\]
且 $\forall x \in \mathbb{N}$,总有 $x \le x$。这使得 $\mathbb{N}$ 成为一个\textbf{有序}集,我们称``$\le$''是 $\mathbb{N}$ 上的一种\emph{顺序关系}。(详见 \ref{sec:section6.3} 节。)

此外,这种关系实际上是一种\textbf{良序关系}。我们不会正式定义这个术语,但良序关系的一个关键特点是不存在\emph{无穷递降链}。想象一下在 $\mathbb{N}$ 中是否存在一个无穷序列 $a_1, a_2, a_3, \dots$,使得 $a_1 > a_2 > a_3 > \dots$?这是不可能的!(注意这些不等式是严格的。)意思是说,从某个数字 $a_1 \in \mathbb{N}$ 开始,如果我们``递降'',最终会到达 $1$,我们不能再``递降''了。

我们不会全面讨论良序关系 --- 这可以在集合论或形式逻辑的课程中讨论 --- 而是专注于这一概念在 $\mathbb{N}$ 上的应用。这是一个很有用的性质,我们以后也会涉及到。在本节中,我们将阐述良序原理,并请你帮助我们证明它,然后展示它与数学归纳法之间的关系。

\subsubsection*{陈述与证明}

\begin{theorem}\label{theorem5.5.2}
    $\mathbb{N}$ 的所有非空子集都有一个最小元素。用逻辑形式表示为
    \[\forall S \in \mathcal{P}(\mathbb{N}) \centerdot [S \ne \varnothing \implies (\exists \ell \in S \centerdot (\forall x \in S \centerdot \ell \le x))]\]
\end{theorem}

想一想这与我们之前提到的自然数中不存在无穷递降链的关系。如果确实存在无穷递降链,我们可以定义 $S$ 为链中所有元素的集合。这个集合将\textbf{没有}最小元素。假设集合中有一个元素 $a_n$,那么 $a_{n+1}$ 也在集合中,且 $a_{n+1} < a_n$。因此,这个集合不会有最小元素。

我们希望你来证明这个定理,因为我们认为完成详细推理过程会大有裨益。证明分为几个步骤。一个关键点是,这个证明是通过\textbf{归纳法}完成的!也就是说,通过这种方式证明良序原理,我们可以展示数学归纳原理\emph{蕴含}了良序原理。

\begin{proof}
    用归纳法证明。留给读者作为习题 \ref{exc:exercises5.7.21}。
\end{proof}

一个简单的额外发现是,任何子集 $S \subseteq \mathbb{N}$ 的最小元素必须是\emph{唯一的}。也就是说,不可能存在两个(或更多)最小元素。假设集合 $S$ 中确实有两个最小元素,分别是 $\ell$ 和 $m$。根据最小元素的定义,我们知道 $\ell \le m$ 且 $m \le \ell$。这意味着 $\ell = m$,所以它们实际上是同一元素!

\subsubsection*{归纳法、强归纳法与良序原理}

如前所述,因为我们使用归纳法证明了良序原理,这说明数学归纳原理蕴含良序原理。接下来的定理实际上表明,这两个定理是\textbf{逻辑等价}的,它们互相蕴含。此外,它还表明强归纳原理也蕴含良序原理,反之亦然。事实上,这三个定理在逻辑上是等价的!

\begin{theorem}
    以下三个命题全都逻辑等价:
    \begin{itemize}
        \item 数学归纳原理
        \item 强归纳原理
        \item 良序原理
    \end{itemize}
\end{theorem}

\begin{proof}
    我们可以用以下简写来表示每个定理:
    \begin{itemize}
        \item \textbf{PMI}:数学归纳原理 (Principle of Mathematical Induction)
        \item \textbf{PSI}:强归纳原理 (Principle of Strong Induction)
        \item \textbf{WOP}:良序原理 (Well-Ordering Principle)
    \end{itemize}
    根据我们对 PSI 和 WOP 的证明,我们可以推断出
    \[\text{PMI} \implies \text{PSI} \quad\text{且}\quad \text{PMI} \implies \text{WOP}\]
    我们在 \ref{sec:section5.4.2} 节证明了
    \[\text{PSI} \implies \text{PMI}\]
    因此我们得到
    \[\text{PMI} \iff \text{PSI} \quad\text{且}\quad \text{PMI} \implies \text{WOP}\]
    要完成这个证明,我们需要证明 $\text{WOP} \implies \text{PMI}$。这样我们就证明了 $\text{WOP} \iff \text{PMI}$。通过上述等价性,我们可以推断出这三个定理在逻辑上是等价的。

    为了证明这一点,我们假设 WOP 是成立,并用它来证明 PMI。(可以回顾一下定理 \ref{theorem5.2.2} 中 PMI 的陈述,以理解我们这样做的原因。)

    假设有一个命题 $P(n)$,定义在 $n \in \mathbb{N}$ 上。假设 $P(1)$ 为\verb|真|,并且对于所有的 $k \in \mathbb{N} \centerdot P(k) \implies P(k+1)$。我们需要证明对于所有的 $n \in \mathbb{N} \centerdot P(n)$ 都成立。


    定义集合 $F$ 为 $P(n)$ 的``失败实例''集合。即定义:
    \[F = \{n \in \mathbb{N} \mid P(n) \text{ 为假}\}\]
    为了证明 $\forall n \in \mathbb{N} \centerdot P(n)$,我们将为了引出矛盾而假设 $F \ne \varnothing$。

    由于我们使用了集合构建符,所以 $F \subseteq \mathbb{N}$。根据之前的假设,$\exists f \in F$。给定该 $f$。

    基于这两个条件,WOP 适用于集合 $F$,这意味着 $F$ 有一个最小元素。设 $\ell$ 为这个最小元素。我们知道 $\ell \in F$,并且
    \[\forall x \in F \centerdot \ell \le x\]
    考虑 $\ell = 1$ 的情况。这是不可能的,因为我们上面的假设说 $P(1)$ 成立,所以 $1 \notin F$。

    考虑 $\ell \ge 2$ 的情况。我们上面的假设说
    \[\forall k \in \mathbb{N} \centerdot P(k) \implies P(k+1)\]
    这逻辑等价与其逆否命题
    \[\forall k \in \mathbb{N} - \{1\} \centerdot \neg P(k) \implies \neg P(k-1)\]
    将这一点应用到元素 $\ell \in \mathbb{N} - \{1\}$ 上,我们可以推断出 $\neg P(\ell - 1)$ 也成立。换句话说,$P(\ell-1)$ 为\verb|假|。

    这意味着 $\ell-1 \in F$。然而,这与 $\ell$ 是 $F$ 的最小元素相矛盾,因为 $\ell - 1 < \ell$。$\hashx$

    因此,必定是 $F = \varnothing$,而这意味着 $\forall n \in \mathbb{N} \centerdot P(n)$。

    这证明了 PMI 成立。
\end{proof}

让我们来看一下这个证明的核心部分。为了证明 $P(n)$ 对所有 $n$ 都成立,我们假设它对某个特定的 $n$ 会失败,即这个元素 $f \in F$。从这里开始,你可能会想,``如果 $P(f)$ 失败,那么 $P(f-1)$ 也会失败,接着 $P(f-2)$ 也会失败,……\textbf{依次类推},直到 $P(1)$,但是我们知道 $P(1)$ 为\verb|真|。''然而,这种``依次类推''的论证正是 PMI 和 WOP 所要解决的问题!你不能用一个含糊的``继续下去''来证明你可以这样做。这就是为什么我们要使用 WOP 来找出 $F$ 的最小元素。你可能会觉得我们引入 $f \in F$ 然后不再使用它很奇怪。其实,我们需要 $f$ 的存在来证明 $F \ne \varnothing$,从而能够应用 WOP。


% !TeX root = ../../../book.tex
\subsection{习题}

\subsubsection*{温故知新}

以口头或书面的形式简要回答以下问题。这些问题全都基于你刚刚阅读的内容,所以如果忘记了具体的定义、概念或示例,可以回去重读相关部分。确保在继续学习之前能够自信地回答这些问题,这将有助于你的理解和记忆!

\begin{enumerate}[label=(\arabic*)]
    \item ``最小罪犯 (Minimal Criminal)''论证和强归纳法 (Strong Induction) 证明之间有什么区别?
    \item 我们证明了每个自然数 $n \in \mathbb{N}$ 都可以写成 $2$ 的幂和一个奇数的乘积。这种表示法还有什么其他特性?
    \item 我们证明了自然数集 $\mathbb{N}$ 是良序的。你认为整数集 $\mathbb{Z}$ 也有此性质吗?有理数集 $\mathbb{Q}$ 呢?实数集 $\mathbb{R}$ 呢?
    \item 数学归纳法 (PMI)、强归纳法 (PSI) 和良序原理 (WOP) 在逻辑上是等价的,这是什么意思?
\end{enumerate}

\subsubsection*{小试牛刀}

尝试回答以下问题。这些题目要求你实际动笔写下答案,或(对朋友/同学)口头陈述答案。目的是帮助你练习使用新的概念、定义和符号。题目都比较简单,确保能够解决这些问题将对你大有帮助!

\begin{enumerate}[label=(\arabic*)]
    \item 请证明良序原理。这也是习题 \ref{exc:exercises5.7.21}。快来试试吧!
    \item 请证明 $\sqrt{3}$ 为无理数。\\
        (\textbf{提示}:为了得到矛盾而假设 $\sqrt{3} = \frac{a}{b}$,其中 $a, b \in \mathbb{N}$ 且分数是最简形式。使用递降法 (descent argument) 来反驳这里最简形式的假设。)
    \item 使用良序原理证明每个自然数(除了 $1$)都可以写成 $2$ 和 $3$ 的非负整数倍之和。\\
        例如:$2 = 2, 8 = 6 + 2, 101 = 3 \times 33 + 2$。
    \item 考虑以下方程:$4x^4 + 2y^4 = z^4$。在这个问题中,你将通过良序原理来论证此方程在 $(x, y, z) \in \mathbb{N}^3$ 时\textbf{无解}。
        \begin{enumerate}[label=(\alph*)]
            \item 为了得到矛盾而假设 $(x, y, z) \in \mathbb{N}^3$ 是一个解,并进一步假设在所有解中,这个解的 $x$ 值是最小的。也就是说,我们定义
            \[T = \{x \in \mathbb{N} \mid \exists y, z \in \mathbb{N} \centerdot 4x^4 + 2y^4 = z^4\}\]
            并预设这个集合是非空的(即方程有解),因此 $T$ 有一个最小元素。
            \item 推导出 $z$ 为偶数。\\
            \textbf{提示}:在本部分和接下来的两部分中,你可以使用以下事实:某个自然数 $m$ 的倍数和/差也是 $m$ 的倍数。
            \item 推导出 $y$ 为偶数。
            \item 推导出 $x$ 为偶数。
            \item 据此推导出存在另一个解 $(a, b, c)$,其第一个变量的值更小,即 $a < x$。
            \item 解释为什么这证明了无解。
        \end{enumerate}
\end{enumerate}