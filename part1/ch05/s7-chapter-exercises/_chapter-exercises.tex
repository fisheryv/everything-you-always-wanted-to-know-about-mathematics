% !TeX root = ../../../book.tex
\section{本章习题} \label{sec:section5.7}

本节习题涵盖本章全部内容,并涉及先前知识点及部分数学假设。我们不要求你解答\textbf{所有}题目,但解决得越多,收获越大!请牢记:真正\emph{掌握}数学必须亲自\emph{实践}。尝试动手解题,仔细阅读并思考题意。撰写证明并与朋友讨论,检验其说服力。持续练习如何清晰、准确、有条理地\emph{书写}思路。完成证明后要反复修改以臻完善。最重要的是,坚持\emph{钻研}数学!

标有 $\blacktriangleright$ 的简答题只需解释或陈述答案,无需严格证明。

特别具有挑战性的问题标记为 $\bigstar$。\\

\begin{exercise} \label{exc:exercises5.7.1}
    证明
    \[\forall n \in \mathbb{N} \centerdot \sum_{k=1}^{n} k^3 = \bigg(\sum_{k=1}^{n} k\bigg)^2\]
\end{exercise}

\begin{exercise}
    找出使下列不等式成立的所有\emph{自然数}。先给出\textbf{结论},再进行\textbf{证明}(必要时可使用归纳法)。
    \begin{tasks}[label=(\alph*)](2)
        \task $3^n > n^4$
        \task $(n-3)^2 > (n-2)^3$
        \task $3^n < n!$
        \task $4^n > n^4$
    \end{tasks}
\end{exercise}

\begin{exercise}
    下面的``证明''有什么问题?

    \textbf{声明}:所有偶数都是 $2$ 的幂。

    \begin{center}
        \noindent
            \parbox{0.85\textwidth}{%
                \linespread{1.5}\selectfont
                通过对 $n$ 应用归纳法来证明。

                首先,$2=2^1$ 是 $2$ 的幂。

                接着,假设 $k \in \mathbb{N}$ 满足 $k \ge 4$ 且 $k$ 为偶数。
                
                进一步假设所有小于 $k$ 的偶数均为 $2$ 的幂。

                因为 $k$ 为偶数,考虑 $\frac{k}{2}$,根据假设,$\frac{k}{2}$ 是 $2$ 的幂,所以存在某个 $j$ 使得 $\frac{k}{2}=2^j$。

                这表明 $k = 2^{j+1}$,所以 $k$ 是 $2$ 的幂。 
            }
    \end{center}
\end{exercise}

\begin{exercise}
    如果一个数 $n \in \mathbb{N}$ 可表示为 $x$ 和 $y$ 的非负倍数之和,那么我们称其``在 $(x, y)$ 的土地上是特殊的''。 例如,$11$ 在 $(3, 5)$ 的土地上是特殊的,因为 $11 = 5 + 2 \cdot 3$。同理,$15$ 也是特殊的,因为 $15 = 3 \cdot 5 + 0 \cdot 3$。然而,$7$ 则不是特殊的。\\
    对于以下各组 $(x, y)$,给出结论并证明:确定在相应土地上所有特殊数构成的集合 $S_{x,y}$。
    \begin{tasks}(4)
        \task $(2, 3)$
        \task $(3, 5)$
        \task $(4, 9)$
        \task $(7, 6)$
    \end{tasks}
\end{exercise}

\begin{exercise}
    证明:对于任意 $n \in \mathbb{N}$,以及满足 $\forall i \in [n] \centerdot 0 \le x_i \le 1$ 的任意实数 $x_1, x_2, \dots, x_n$,下面不等式成立
    \[\prod_{i=1}^{n} (1-x_i) \ge 1-\sum_{i=1}^{n}x_i\]
    这就是所谓的\textbf{伯努利不等式}。
\end{exercise}

\begin{exercise}
    设 $P(n)$ 是关于变量 $n$ 的命题,$n$ 可以取任意\textbf{整数}值。\\
    对于以下每种情况,给定``基本情况''和``归纳推论''。请确定并解释在这些假设下能够必然推出的命题实例。\\
    例如,若基本情况为 $P(3)$,归纳推论为 $\forall n \in \mathbb{N} \centerdot P(n) \implies P(n+1)$,则正确答案为``$P(n)$ 对于每个 $n \in \mathbb{N}$ 且 $n \ge 3$ 成立。''
    \begin{enumerate}[label=(\alph*)]
        \item 基本情况:$P(-3)$ \qquad\quad\enspace\,\, 推论:$\forall n \in \mathbb{Z} \centerdot P(n) \implies P(n+1)$
        \item 基本情况:$P(1) \land P(2)$ \quad\;\; 推论:$\forall n \in \mathbb{N} \centerdot P(n) \implies P(2n)$
        \item 基本情况:$P(0)$ \qquad\qquad\quad 推论:$\forall n \in \mathbb{Z} \centerdot P(n) \implies \big(P(n-1) \land P(n+1)\big)$
        \item 基本情况:$P(-1) \land P(0)$ \quad 推论:$\forall n \in \mathbb{Z} \centerdot P(n) \implies P(n+2)$
    \end{enumerate}
\end{exercise}

\begin{exercise}
    证明:对于任意不相等的整数 $x,y \in \mathbb{Z}\ (x \ne y)$, $x^n-y^n$ 是 $x-y$ 的倍数,其中 $n \in \mathbb{N} \cup \{0\}$。
\end{exercise}

\begin{exercise}
    找出使下列不等式成立的自然数 $n$ 的集合。
    \begin{tasks}[label=(\alph*)](3)
        \task $n! > 2^n$
        \task $n! > 3^n$
        \task $n! > 5^n$
    \end{tasks}
\end{exercise}

\begin{exercise}
    $\bigstar$ 证明上一个问题的推广命题:
    \[\forall m \in \mathbb{N} - \{1\} \centerdot \exists B_m \in \mathbb{N} \centerdot \forall n \in \mathbb{N} \centerdot n \ge B_m \implies n! > m^n\]
\end{exercise}

\begin{exercise}
    \textbf{斐波那契数列}定义为 $f_0=0, f_1=1$,且对于所有 $n \ge 2, f_n = f_{n-1}+f_{n-2}$。\\
    对于该数列,证明下列命题成立:
    \begin{enumerate}[label=(\alph*)]
        \item $\displaystyle \forall n \in \mathbb{N} \cup \{0\} \centerdot f_n < 2^n$
        \item $\displaystyle \forall n \in \mathbb{N} \centerdot f_{n-1}f_{n+1} = f_n^2+(-1)^n$
        \item $\displaystyle \forall n \in \mathbb{N} \centerdot 1 \le \frac{f_{n+1}}{f_n} \le 2$
        \item $\displaystyle \forall n \in \mathbb{N} \centerdot \sum_{k=1}^{n} f_{2k}= f_{2k+1}-1$
        \item $\displaystyle \forall n \in \mathbb{N} \centerdot \sum_{k=1}^{n} f_k^2= f_{n}f_{n+1}$
    \end{enumerate}
\end{exercise}

\begin{exercise}\label{exc:exercises5.7.15}
    在问题 \ref{exc:exercises5.7.1} 中,你已证明前 $n$ 个自然数的立方和等于其和的平方。\\
    本题要求证明其逆命题,即满足该性质的\emph{唯一}数列是 $\langle 1, 2, \dots, n \rangle$。命题重述如下:\\
    \textbf{声明}:设 $\langle a_i \rangle$ 为实数数列,即 $\forall i \in \mathbb{N} \centerdot a_i \in \mathbb{R}$。假设该数列具有如下性质
    \[\forall n \in \mathbb{N} \centerdot \sum_{k=1}^{n} a_k^3 = \bigg(\sum_{k=1}^{n} a_k\bigg)^2\]
    通过对 $n$ 应用归纳法证明 $\forall n \in \mathbb{N} \centerdot a_n = n$。
\end{exercise}

\begin{exercise}
    证明:
    \begin{enumerate}[label=(\alph*)]
        \item $\forall n \in \mathbb{N} \centerdot 7^n+7 < 7^{n+1}$。
        \item $\forall n \in \mathbb{N} \centerdot 3^n+3 < 3^{n+1}$。
        \item 找出满足 $\forall n \in \mathbb{N} \centerdot r^n+r < r^{n+1}$ 的实数 $r$ 的集合 $S$,并用归纳法证明你的结论。
    \end{enumerate}
\end{exercise}

\begin{exercise}
    证明:对于所有 $n \in \mathbb{N}, 2^{3^n}+1$ 是 $3^{n+1}$ 的倍数。
\end{exercise}

\begin{exercise}
    $\bigstar$ 假设 $x + \frac{1}{x}$ 为整数,证明对于所有 $n \in \mathbb{Z}, x^n + \frac{1}{x^n}$ 也为整数。\\
    (\textbf{注意}:本题要求对全体整数 $n \in \mathbb{Z}$ 成立,而不仅仅是自然数 $n \in \mathbb{N}$!)
\end{exercise}

\begin{exercise}
    通过对 $n$ 应用归纳法证明:对于所有 $n \in \mathbb{N}, n^3+5n$ 是 $6$ 的倍数。
\end{exercise}

\begin{exercise}
    证明,下面等式对于所有 $n \in \mathbb{N}$ 成立:
    \[\sum_{k=n}^{2n-1} 2k+1=3n^2\]
\end{exercise}

\begin{exercise}
    对于所有 $n \in \mathbb{N} \cup \{0\}$ 定义:
    \[s_n=(3+\sqrt{5})^n+(3-\sqrt{5})^n\]
    证明所有 $s_n$ 均为整数,且 $s_n$ 实际上是 $2^n$ 的倍数。
\end{exercise}

\begin{exercise}
    $\blacktriangleright$ 本题将证明\textbf{调和级数}
    \[\sum_{k=1}^{\infty} \frac{1}{k} = 1+\frac{1}{2}+\frac{1}{3}+\frac{1}{4}+\dots\]
    是\textbf{发散的}。也就是说,我们要证明所有项之总和不趋近于任何有限极限。

    我们断言以下不等式对于所有自然数 $n$ 都成立:
    \[\sum_{k=1}^{2^n} \frac{1}{k} > \frac{n+1}{2} \tag{$\gamma$}\]
    \begin{enumerate}[label=(\alph*)]
        \item 证明当 $n=1$ 时 $(\gamma)$ 成立。
        \item 假设 $m \in \mathbb{N}$ 为任意且固定自然数,并假设当 $n=m$ 时 $(\gamma)$ 成立 \\
            推导出当 $n=m+1$ 时 $(\gamma)$ 也成立,并明确指出何处引用了 $n=m$ 的归纳假设。
        \item 根据以上结论,解释调和级数不收敛的原因。
    \end{enumerate}
\end{exercise}

\begin{exercise}
    证明以下不等式对所有自然数 $n$ 成立:
    \[\sum_{k=1}^{n} \frac{1}{\sqrt{k}} = 1+\frac{1}{\sqrt{2}}+\frac{1}{\sqrt{3}}+\dots+\frac{1}{\sqrt{n}} \ge \sqrt{n}\]
    据此推导无穷级数 $\displaystyle \sum_{k=1}^{\infty} \frac{1}{\sqrt{k}}$ 不收敛。
\end{exercise}

\begin{exercise}
    证明以下不等式对所有自然数 $n$ 成立:
    \[\prod_{i=1}^{n} \Big(1+\frac{1}{i^2}\Big) = \Big(1+\frac{1}{1^2}\Big)\Big(1+\frac{1}{2^2}\Big) \dots \Big(1+\frac{1}{n^2}\Big) < 4-\frac{1}{n}\]
\end{exercise}

\begin{exercise}\label{exc:exercises5.7.21}
    本题将证明自然数的\textbf{良序原理}(定理 \ref{theorem5.5.2} 的复述):
    \[\forall S \in \mathcal{P}(\mathbb{N}) \centerdot [S \ne \varnothing \implies (\exists \ell \in S \centerdot (\forall x \in S \centerdot \ell \le x))]\]
    也就是说,每个非空自然数子集均有\textbf{最小元素}。

    你将通过归纳法证明给定集合 $S$ 是否包含 $n$ 这个元素。

    我们会为你提供证明框架,并引导你完成剩余部分:

    设 $S \subseteq \mathbb{N}$ 为任意且固定自然数。对于任意 $n \in \mathbb{N}$,定义 $P(n)$ 为命题
    \[n \in S \implies [\exists \ell \in S \centerdot (\forall x \in S \centerdot \ell le x)]\]
    \begin{enumerate}[label=(\alph*)]
        \item 证明 $P(1)$ 成立。(提示:最小自然数是什么?)
        \item 设 $k \in \mathbb{N}$ 为任意且固定自然数。用逻辑符号写出归纳假设:$P(i)$ 对于 $1$ 到 $k$(含)之间的所有 $i$ 都成立。 \\
        (提示:这一步应该很容易;只需写一个``与''陈述。想想它的含义。)\\
        接下来假设 $k + 1 \in S$,定义 $T = S - {k + 1}$,分三种情况讨论:
        \item 若 $T = \varnothing$,证明 $S$ 具有最小元素。
        \item 若 $T \ne \varnothing$ 且 $\forall x \in S \centerdot x \ge k + 1$,证明 $S$ 具有最小元素。
        \item 若 $T \ne \varnothing$ 且 $\exists x \in S \centerdot x < k + 1$,证明 $S$ 具有最小元素。\\
        (提示:这里你需要使用来自 (b) 的归纳假设!)\\
        由于 $S$ 在任何情况下都具有最小元素,故 $P(k + 1)$ 成立。通过归纳,$\forall n \in \mathbb{N} \centerdot P(n)$。
        \item 让我们来说明为什么该证明确实有效!考虑任意 $S \subseteq N$,且 $S \ne \varnothing$。我们如何知道 $S$ 具有最小元素?也就是说,声明 $P(n)$ 的哪个实例保证成立?\\
        (提示:若 $S = \varnothing$ ……,则证明失败)
        \item $[\text{附加题}]$ 为什么不直接对 $S$ 的大小进行归纳?此方法为何不能证明良序原理?
    \end{enumerate}
\end{exercise}

\begin{exercise}
    设 $W$ 是由括号组成的合法字符串集合。$W$ 中的任意元素 $w$ 满足以下条件之一:
    \begin{enumerate}[i]
        \item $w$ 为字符串 ``$()$''
        \item $\exists x \in W$,使得 $w$ 是字符串 ``$(x)$''(即 $w$ 是由字符串 $x$ 周围加括号形成的字符串)
        \item $\exists x, y \in W$,使得 $w$ 是字符串 ``$xy$''(即 $w$ 是将字符串 $y$ 拼接在字符串 $x$ 后面而形成的字符串)
    \end{enumerate}
    例如,``$()()$'' 是 $W$ 中的合法字符串,因为它由有效字符串 ``$()$'' 在其后拼接 ``$()$'' 而成。但 ``$(( )$'' 不是 $W$ 中的合法字符串,因为它不满足上述任何条件。

    (再看一个更复杂的例子,请分析并解释为什么 ``$(()(()))$'' 是合法字符串。)

    证明关于此系统的以下陈述:
    \begin{enumerate}[label=(\alph*)]
        \item 证明每个元素 $w \in W$ 包含\textbf{偶数}个括号。\\
        (提示:使用最小罪犯论证。假设 $w$ 是\emph{最小}奇数长度的字符串……)
        \item 对于 $w \in W$,令 $L(w)$ 为 $w$ 中出现的左括号数量,令 $R(w)$ 为右括号数量。\\
        证明 $\forall w \in W \centerdot L(w) = R(w)$。\\
        (提示:对字符串的长度进行归纳。)
    \end{enumerate}
\end{exercise}

\begin{exercise}
    下面的``错误证明''声称所有笔都是同一种颜色。问题到底出在哪里?

    \begin{spoof}
        我们声称所有笔都是同一种颜色。为了证明这一点,我们将证明任意大小的笔的集合中只有一种颜色。我们将使用归纳法来证明这一点。

        首先,考虑一个只有一支笔的情况。由于只有一支笔,它当然和自己是同一种颜色。

        现在,假设任意 $n$ 支笔的集合只有一种颜色。

        接着,拿出任意 $n+1$ 支笔,将它们排成一排,并从左到右编号为 $1$ 到 $n + 1$。

        先观察前 $n$ 支笔,即编号为 $1,2,3, \dots ,n$ 的笔。根据假设,这组笔中只有一种颜色(尽管我们还不知道具体是什么颜色)。

        再观察后 $n$ 支笔,即编号为 $2,3, \dots ,n+1$ 的笔。根据假设,这组笔中也只有一种颜色。

        由于编号为 $2$ 的笔同时属于这两个集合,因此,无论编号为 $2$ 的笔是什么颜色,它也是这两组笔的颜色。所以,所有 $n + 1$ 支笔都是同一种颜色。

        通过归纳法,这表明任意大小的笔的集合都只有一种颜色。由此,我们可以推断世界上所有的笔都是同一种颜色。
    \end{spoof}
\end{exercise}

\begin{exercise}
    $n$ 边形是指具有 $n$ 条边的凸多边形(``凸''指图形无凹陷,即其内部任意两点的连线仍完全包含在图形内部)。例如,$3$ 边形为三角形,$4$ 边形为四边形。\\
    用归纳法证明:$n$ 边形顶点间可画出 $\frac{n(n-3)}{2}$ 条对角线(不含多边形的\emph{边},仅计算\emph{内部}对角线)。
\end{exercise}