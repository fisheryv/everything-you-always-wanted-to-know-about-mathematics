% !TeX root = ../../../book.tex
\section{引言}

尽管此前已经讨论过数学归纳法,此时再加入本章似乎略显多余。然而,本章的安排仍具深意。后续内容将阐明我们为何需要重新审视这一主题。

首先,此前对归纳法的处理在数学严谨性上尚有不足。其次,第 \ref{ch:chapter02} 章遗留了若干有待解决的问题——例如取物游戏与汉诺塔问题中,其归纳论证似乎比求和公式 $\sum_{k \in [n]} k=\frac{n(n+1)}{2}$ 等例证使用了更强的假设。我们将在此探讨这些差异的本质。再者,许多重要而有趣的实例亟待考察,它们既能深化对数学语言的理解,本身也是值得掌握的结论。最后,本章将与你共同证明的定理(见 \ref{sec:section4.9.6} 节)完美呈现了等价性的证明范式:通过双向蕴涵关系串联三个定理,这将是我们首次实践 \ref{sec:section4.9.6} 节指出的``以下等价……''风格定理的经典证明策略。

% !TeX root = ../../../book.tex
\subsection{目标}

鉴于我们之前已经探讨过归纳法,现在再次回到这一主题上,本章将不再赘述常规的引言部分。我们将通过几个要点来概述本章的核心目标。

\textbf{学完本章后,你应该能够……}

\begin{itemize}
    \item 阐述数学归纳法原理,解释其证明过程与自然数集的联系。
    \item 阐述强数学归纳法原理,对比其与标准归纳法的异同并阐述证明方法。
    \item 应用归纳法证明各种命题,并能判别何时需采用强归纳法。
    \item 理解多种归纳法变体,并阐释其适用场景。
    \item 阐述良序原理,解释其与数学归纳法之间的联系
\end{itemize}
