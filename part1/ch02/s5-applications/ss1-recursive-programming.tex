% !TeX root = ../../../book.tex
\subsection{递归程序}\label{sec:section2.5.1}

数学归纳法背后的概念也大量应用于计算机科学中。回想一下我们当初是如何推导出 $\sum_{k=1}^{n}k^2$ 的公式的。一旦我们找到了用更小的立方和剩余项来表示立方数的方法,我们就会一遍又一遍地重复这个替代过程,直到我们到达``最简单''的情况,即我们在开始计算问题:$2^3=1+3+3+1$ 时第一次观察到的情况。递归编程利用了这一技术:为了解决``大''问题,确定问题如何依赖于``小''案例,并化简问题,直到达到一个简单的已知案例。

此类技术的一个经典示例是编写代码来计算\emph{阶乘} $n!$,它被定义为前 $n$ 个自然数的乘积:
\[n! = 1 \cdot 2 \cdot 3 \dots (n - 1) \cdot n\]
这是一个简单的定义,我们作为人类可以直观地理解,但告诉计算机如何执行该产品的方式并不完全相同。(尝试一下!怎么用计算机代码表达``继续执行,直到达到 $n$''?)事实上,对函数进行编程的一种更有效的方法,以及对数学归纳定义进行建模的方法,是让一个程序\emph{递归地调用自身},直到达到``最简''情况。在阶乘种,这种情况就是 $1! = 1$。对于 $n$ 的任意其他值,我们可以简单地不断应用以下知识:
\[n! = (n - 1)! \cdot n\]
来计算 $n!$。下面的\emph{伪代码}描述了这一思想:

\begin{verbatim}
    factorial(n):
        if n = 1
            return 1
        else
            return n * factorial(n-1)
    end
\end{verbatim}

我们知道 $1! = 1$,因此如果要求程序计算该值,则会立即返回正确的结果。对于任意大于 $1$ 的 $n$ 值,程序都会引用\emph{自身},并说:``为我计算 $(n-1)!$,然后我在后面乘上 $n$ 就得到答案了。''为了计算 $(n-1)!$,程序会再次询问输入是否为 $1$;如果不为 $1$,则会继续调用自身并说:``为我计算 $(n-2)!$,然后我在后面乘上 $(n-1)$ 。'' 这个过程一直持续到程序返回 $1! = 1$。从那之后,程序就知道如何求出 $2! = 1 \times 2$,然后是 $3! = 2! \times 3$,依此类推,直到 $n! = (n - 1)! \times n$。

递归编程的另一个经典例子是\emph{斐波那契数}。可能你之前在数学课上见过这个数列。(事实上,我们在上一节多米诺骨牌密铺中提到过该数列!)你可能还听说过斐波那契数列以一些有趣和奇怪的方式出现在自然界中。(该数列最早是由比萨的意大利数学家莱昂纳多·斐波那契(Leonardo Fibonacci)在研究兔子种群增长时``发现''的。)斐波那契数列的前两项均为 $1$,数列中的任意数字都被定义为前两个数字之和。也就是说,如果我们用 $F(n)$ 表示第 $n$ 个斐波那契数,那么
\[F(1) = 1 \text{ 且 }F(2) = 1, \text{ 对于任意 } n \ge 3, F(n) = F(n - 1) + F(n - 2)\]
那么,$F(5)$ 等于多少?$F(100)$ 呢?$F(10000)$ 呢?这可以通过递归程序很容易地处理。思路是一样的:如果程序遇到``简单情况''之一,即 $F(1)$ 或 $F(2)$,则立即返回正确值 $1$。否则,将调用自身来计算前两个数字,然后将它们相加。阅读下面的伪代码并思考它是如何工作的。如果我们使用这个程序来计算 $F(10)$ 会发生什么?怎样才能得出答案呢?

\begin{verbatim}
    Fibonacci(n):
        if n = 1 or n = 2
            return 1
        else
            return Fibonacci(n-1) + Fibonacci(n-2)
    end
\end{verbatim}

上面程序与 \verb|factorial| 程序的思路相同(让程序调用自身来计算函数``较小''情况的值,直到达到已知值),但这里有一些更深层次的内容。但这里有一些更深层次的内容。如果我们将 $n = 10$ 输入到程序中,它会识别出还不知道应该输出什么值,于是会调用自身来计算 \verb|Fibonacci(9)| 和 \verb|Fibonacci(8)|。对程序的每次调用,都会再次识别出该值未知。因此,会再次调用自身来计算 \verb|Fibonacci(8)| 和 \verb|Fibonacci(7)|,以及 \verb|Fibonacci(7)| 和 \verb|Fibonacci(6)|。没错,程序使用相同的输入值多次调用自身。为了计算 $F(9)$,我们需要知道 $F(8)$ 和 $F(7)$,但同时,为了计算 $F(8)$,我们还需要知道 $F(7) $ 和 $F(6)$。这样,我们最终多次调用程序 \verb|Fibonacci|。

尝试比较 \verb|Fibonacci| 程序和 \verb|factorial| 程序,特别是关于我们在本章中研究的归纳过程。他们使用类似的思路吗? 它们与我们介绍数学归纳法时用到的``多米诺骨牌''类比有何关系?将多米诺骨牌 $n$ 上的``事实''视为对 $n!$ 的正确计算或 $F(n)$。这个类比在每种情况下如何发挥作用?所有多米诺骨牌都会倒下吗?在你继续阅读时,请记住这些问题。所有这些想法背后都有一些非常强大的数学基础。

\clearpage
