% !TeX root = ../../../book.tex
\subsection{递归程序}\label{sec:section2.5.1}

数学归纳法背后的概念在计算机科学中有着广泛应用。回想我们推导 $\sum_{k=1}^{n}k^2$ 公式的过程:一旦找到用较小立方和及剩余项表示立方数的方法,就会反复进行替换,直到抵达最初观察到的``最简''情况——即计算起点 $2^3=1+3+3+1$。递归编程正是运用了这一思想:通过确定``大''问题如何依赖``小''案例来简化问题,直至达到已知的简单案例。

此类技术的经典示例是计算\emph{阶乘} $n!$ 的代码,其定义为前 $n$ 个自然数的乘积:
\[n! = 1 \cdot 2 \cdot 3 \dots (n - 1) \cdot n\]
这个定义对人类而言直观易懂,但指导计算机执行该乘积却并非易事(尝试思考:如何在代码中表达``持续执行直至达到 $n$''?)。实际上,更有效的函数编程方法以及对数学归纳定义的建模,是让程序\emph{递归调用自身}直至``最简''情况。对于阶乘而言,该情况即 $1! = 1$。对于任意其他 $n$ 值,可反复应用关系:
\[n! = (n - 1)! \cdot n\]
来计算 $n!$。以下\emph{伪代码}体现了这一思想:

\begin{verbatim}
    factorial(n):
        if n = 1
            return 1
        else
            return n * factorial(n-1)
    end
\end{verbatim}

我们已知 $1! = 1$,若需计算此值,将直接返回正确结果。对于任意 $n > 1$,程序会调用\emph{自身}:``请计算 $(n-1)!$,然后乘以 $n$ 得到答案。''为计算 $(n-1)!$,程序再次检查输入是否为 $1$;若不为 $1$,则继续调用自身:``请计算 $(n-2)!$,然后乘以 $(n-1)$。''此过程持续至程序返回 $1! = 1$。随后,程序便能计算 $2! = 1 \times 2$,进而 $3! = 2! \times 3$,依此类推,最终得到 $n! = (n - 1)! \times n$。

\clearpage

递归编程的另一个经典例子是\emph{斐波那契数列}。你可能在数学课上见过这个数列(事实上,我们在上一节讨论多米诺骨牌密铺时提到过它),或者听说过它以有趣而奇妙的方式出现在自然界中——该数列最早由意大利比萨数学家莱昂纳多·斐波那契 (Leonardo Fibonacci) 在研究兔子种群增长时``发现''。斐波那契数列的前两项均为 $1$,后续每一项都定义为前两项之和。若用 $F(n)$ 表示第 $n$ 个斐波那契数,则有:
\[F(1) = 1 \text{\ 且\ }F(2) = 1, \text{ 对于任意\ } n \ge 3, F(n) = F(n - 1) + F(n - 2)\]
那么,$F(5)$ 等于多少?$F(100)$ 或 $F(10000)$ 呢?递归程序可以轻松解决这个问题。思路一致:当函数遇到基本情况 $F(1)$ 或 $F(2)$ 时,直接返回 $1$;否则,递归调用自身计算前两项并求和。阅读以下伪代码并思考其工作原理:若用此程序计算 $F(10)$,过程会如何展开?

\begin{verbatim}
    Fibonacci(n):
        if n = 1 or n = 2
            return 1
        else
            return Fibonacci(n-1) + Fibonacci(n-2)
    end
\end{verbatim}

该程序与 \verb|factorial| 函数思路相似(通过递归调用计算更小规模的问题直至达到已知解),但蕴含更深层的机制。若输入 $n = 10$,程序发现结果未知,将调用自身计算 \verb|Fibonacci(9)| 和 \verb|Fibonacci(8)|。每次调用遇到未知值时,又会触发新的递归计算——例如为求 $F(9)$ 需计算 $F(8)$ 和 $F(7)$,而为求 $F(8)$ 又需计算 $F(7)$ 和 $F(6)$。这导致相同输入值被多次重复计算。

请比较 \verb|Fibonacci| 与 \verb|factorial| 程序,思考它们如何体现归纳过程的核心思想。两者是否采用类似策略?与介绍数学归纳法时用到的``多米诺骨牌''类比有何关联?不妨将骨牌 $n$ 的``事实''视为 $n!$ 的正确计算或 $F(n)$ 的值,这个类比在两种情况下如何成立?所有骨牌是否必然倒下?在继续阅读之前,请带着这些问题深入思考——其背后蕴藏着强大的数学基础。

\clearpage
