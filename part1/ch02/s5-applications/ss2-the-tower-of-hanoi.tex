% !TeX root = ../../../book.tex
\subsection{汉诺塔}

我们稍作休息来玩个游戏吧。虽然严格来说不算完全放松——因为它本质上是关于\emph{归纳}的练习,所以与主题密切相关——但至少是个游戏!\emph{汉诺塔}作为经典智力游戏广受欢迎,部分得益于其简洁的规则和道具。然而它的解法可绝不简单!

假设有三根垂直立柱,以及三个大小不同的圆盘(\textcolor{blue}{蓝色}、\textcolor{olivegreen}{绿色}、\textcolor{red}{红色})叠放其中,初始状态如下:
\begin{center}
    \begin{tikzpicture}[line width=4mm,line cap=round,xscale=3,brown!30]
        \def\sequence{3/1,2/1,1/1}
        % init colors
        \foreach[count=\j] \c in {red,olivegreen,blue}
            \gset col[\j]={\c};
        \edef\numdisks{\j}
        % init positions and draw support
        \foreach \j in {1,2,3}{
            \gset pos[\j]=0
            \draw (\j,-.4) -- +(0,3);
        }
        \draw (.5,-.4) -- +(3,0);

        % draw
        \foreach[count=\k] \i/\j in \sequence{
            \edef\delta{\i*0.4/3}
            \draw[draw={\get col[\i]}](\j-\delta,\get pos[\j]) -- (\j+\delta,\get pos[\j]);
            \ginc pos[\j]+={.4}
        }
    \end{tikzpicture}
\end{center}
目标是将所有圆盘移至另一立柱(左中右皆可),但需遵守两条规则:
\begin{enumerate}
    \item 每次只能移动一个圆盘,且必须从某立柱顶端移至另一立柱顶端;
    \item 任何圆盘不可置于比它小的圆盘之上。
\end{enumerate}
规则虽简,破解却难!建议用硬币或扑克牌模拟(亦可购买专用套件)。你能破解这个问题吗?用了多少步?这是最优解吗?原因何在?

既然这是归纳游戏,我们深入探究:解决此谜题需多少步?(一步指单个圆盘的移动)更关键的是,如何确定\emph{最小步数}?以三个圆盘为例,若反复移动最小圆盘 $100$ 次后再求解,显然非最优。那么如何证明某种解法的步数已达最小值?

为了解决这个问题,我们将尝试\emph{递归分解}解法。此过程能回答更一般的问题:$n$ 个圆盘在三柱汉诺塔中的最小移动步数是多少?以 $3$ 个圆盘引入是为了便于理解,但通过深入分析可解决一般情况。为确保理解一致,先演示 $3$ 个圆盘的解法:

\begin{center}
    \begin{tikzpicture}[line width=2.6mm,line cap=round,xscale=1.5,yscale=0.5,brown!30]
        \def\sequencelist{{3/1,2/1,1/1}, {3/1,2/1,1/3}, {3/1,2/2,1/3}, {3/1,2/2,1/2}, {3/3,2/2,1/2}, {3/3,2/2,1/1}, {3/3,2/3,1/1}, {3/3,2/3,1/3}}
        % init colors
        \foreach[count=\j] \c in {red,olivegreen,blue}
            \gset col[\j]={\c};
        \edef\numdisks{\j}
        \foreach[count=\n] \sequence in \sequencelist {
            % init positions and draw support
            \foreach \j in {1,2,3}{
                \gset pos[\j]=0
                \draw (\j,-0.4-5*\n) -- +(0, 3);
            }
            \draw (0.5,-0.52-5*\n) -- +(3,0);
            % draw
            \foreach[count=\k] \i/\j in \sequence{
                \edef\delta{\i*0.4/3}
                \draw[draw={\get col[\i]}](\j-\delta,\get pos[\j]-5*\n) -- (\j+\delta,\get pos[\j]-5*\n);
                \ginc pos[\j]+={0.52}
            }
        }
        \node[black,left] at (0,-5.52){开始};
        \foreach \m in {1, ..., 7} {
            \node[black,left] at (0,-5.52-5*\m){第 \m 步};
        }   
    \end{tikzpicture}
\end{center}

请注意,最大的圆盘在多数解法中几乎是``无关紧要''的。由于我们可以在其上放置任何其他圆盘,我们只需将其他圆盘移到不同立柱上以``露出''最大圆盘,将其移至唯一的空立柱,再将其他圆盘移回即可。本质上,我们重复执行相同过程(将两个较小圆盘从一个立柱移至另一个立柱)两次,中间移动一次最大圆盘。若最大圆盘不存在,我们所做的正是解决两次 $2$ 盘问题!(仔细思考这一点,确保理解上文。可设想最大的\textcolor{blue}{蓝色}圆盘不存在,操作一遍上图的步骤。)

这表明解决 $3$ 盘问题需两次迭代解决 $2$ 盘问题,并增加一次移动最大圆盘的操作。一般而言,这揭示了问题的\emph{递归}性质:为得到 $n$ 盘问题的最优解,只需遵循解决 $(n-1)$ 盘问题的最优过程,移动一次最大的第 $n$ 个圆盘,再解决一次 $(n-1)$ 盘问题。

了解解法后,让我们确定所需步数。由于该问题需用\emph{递归}算法,任何关于最优解的\emph{证明}都需使用\emph{归纳法}。因此需要确定`起点'',即问题的``最小''或``最简''形式。对汉诺塔而言,最简形式是 $1$ 盘问题——只需一步将圆盘移至另一个立柱即可。若用 $M(n)$ 表示解决 $n$ 盘问题的最少移动次数,则 $M(1)=1$。根据前述观察:
\[\underbrace{M(2)}_{2 \text{盘问题}}= \underbrace{M(1)}_{1 \text{盘问题}}+ \underbrace{1}_{\text{移动最大圆盘}}+ \underbrace{M(1)}_{1 \text{盘问题}}= 1 + 1 + 1 = 3\]
进而可得:
\[M(3) = M(2) + 1 + M(2) = 3 + 1 + 3 = 7\]
和
\[M(4) = M(3) + 1 + M(3) = 7 + 1 + 7 = 15\]
依此类推。你发现规律了吗?这些数均为 $2$ 的幂减 $1$,即 $M(n)=2^n-1$。需指出,观察到规律不等于证明规律——它仅在前 $4$ 种情况成立,而归纳法将证明其普遍性。发现 $M(n)=2^n-1$ 本身已是重要洞见。你可以尝试求解以下递推关系:
\[M(n) = 2M(n - 1) + 1 \quad \text{且} \quad M(1) = 1\]
看看能否推导出公式 $M(n) = 2^n -1$。此闭合式优于递推式,因为 $M(n)$ 仅依赖 $n$ 而非前项(例如 $M(n-1)$)。此类关系称为\emph{递推关系},通常求解困难!

我们已知 $M(n)=2^n-1$,但验证工作留给你来完成。虽可代入数值检验,但这并非\emph{证明}。请尝试用归纳法严格证明!我们已完成了主体工作,但还需要你严谨地将一切串联在一起。请记住,你需要明确每张多米诺骨牌上的``事实''是什么,确保多米诺骨牌 $1$ 会倒下,然后对多米诺骨牌 $n$ 倒下会引起多米诺骨牌 $(n+1)$ 倒下进行一般性论证。试着写出这个证明。这些细节对你来说是否合理?向朋友展示你的证明,看看他们是否理解。你还需要向他们做出解释或指导他们完成证明阅读吗?思考最佳的解释方法与步骤,确保书面版本清晰准确。

\clearpage
