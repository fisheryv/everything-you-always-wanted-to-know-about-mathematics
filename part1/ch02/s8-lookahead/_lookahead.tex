% !TeX root = ../../../book.tex
\section{展望}

本章介绍了\textbf{数学归纳法}的概念。我们观察了归纳思想如何引导解题思路,并探讨了如何通过\emph{归纳证明}对此思路进行\emph{严格验证}。鉴于现有数学工具的局限,我们暂时借助非技术性类比来描述这一过程。某种程度上,这好比请朋友向从未接触过高尔夫的人解释挥杆动作:他们可提供挥杆``感受''的心理意象,但若不亲身实践,如何真正理解挥杆机制?如何学习调整动作或区分不同球杆的用法?同样,通过剖析原理与刻意练习,我们期望深入理解数学归纳法,从而能准确运用它、识别适用场景,并学会将其\emph{适配}至新情境。多米诺骨牌类比虽有助于引导直觉,但需谨记其并非数学本质。它亦无法完美解释某些案例——例如当某张骨牌的倒下不仅依赖相邻骨牌,还受之前多张骨牌影响的情形。

下一章将探讨严格表述和证明数学归纳法所需的基础概念。我们将研究\emph{数理逻辑}的相关思想,学习如何分解复杂的数学命题、如何从基础组件构建精妙的陈述,并引入新符号与简记法来压缩冗长的表述,形成简洁而精确的数学语言。在此基础上,我们将探索更基础的证明策略,并将其应用于本课程\emph{所有后续内容}——包括归纳技术本身!同时,我们还将学习\emph{集合论}的核心思想,其构成了数学各分支的基石。这不仅有助于未来系统组织思想,还能为\emph{自然数}的严格定义提供框架。掌握这两大数学分支的概念后,我们便能在坚实的基础之上构建数学归纳法,并持续正确地运用它。
