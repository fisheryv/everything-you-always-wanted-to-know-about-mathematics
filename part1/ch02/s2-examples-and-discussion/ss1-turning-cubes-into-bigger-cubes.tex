% !TeX root = ../../../book.tex
\subsection{构建更大的立方体}

为了引入数学归纳法的整体方法,让我们考察一道几何题并一同解决它。这个例子经过精心挑选,旨在说明当问题具有特定结构时,数学归纳法如何与之关联;具体而言,某些真理、事实或洞察\emph{取决于}、\emph{依赖于}或可从``先前的''事实\emph{推导}得出。这种对先前案例(或多个案例)的依赖使得过程具有\emph{归纳性},当我们观察到这种现象时,应用\emph{归纳法}通常是最佳策略。

\subsubsection*{$1$ 阶立方体到 $2$ 阶立方体}

让我们考察立方数,特别是如何用前一个立方数描述当前立方数。想象一个 $1 \times 1 \times 1$ 的单位立方体。如何通过添加单位立方体构建尺寸为 $2 \times 2 \times 2$ 的``下一个''立方体?需要添加多少块?从算术上看,答案显而易见:$2^3 = 8$ 且 $1^3 = 1$,因此需添加 $7$ 个单位立方体。这虽是一个具体答案,但未说明如何排列这些单位立方体,也未揭示构建\emph{更大}立方体的方法。最终目标是回答:从 $100 \times 100 \times 100$ 立方体构建 $101 \times 101 \times 101$ 立方体需多少块,而无需繁琐计算?即给定 $n \times n \times n$ 立方体,需添加多少单位立方体才能构建出 $(n+1) \times (n+1) \times (n+1)$ 立方体?为此,让我们仔细分析初始案例并用一般性论证来解答它。

给定一个单位立方体,且需添加 $7$ 个单位立方体,让我们确定这些单位立方体的位置以构成 $2 \times 2 \times 2$ 立方体。(为简便起见,称大小为 $n \times n \times n$ 的立方体为 $n$ 阶立方体。本例中 $n$ 仅取自然数,即非负整数。)观察 $1$ 阶和 $2$ 阶立方体的图示,尝试解释如何从前者构建后者。

\begin{center}
    \begin{tikzpicture}
        \pic {annotated cuboid};

        \foreach \x in {0,1}
            \foreach \y in {0,1}
                \foreach \z in {0,1}
                    \pic [fill=white] at (4+\x,\y,\z) {annotated cuboid};
        % \pic [very thick,densely dashed,draw=blue] at (5,0) {annotated cuboid={width=30, height=5, depth=10, opacity=0.2}};
    \end{tikzpicture}
\end{center}

我们采用以下合理解释,因其能指导推导从 $n$ 阶立方体构建 $(n+1)$ 阶立方体的一般方法,且兼具数学的简洁优雅。从 $1$ 阶立方体出发,将三个暴露的面各``扩展''一个单位(本例中为一个单位立方体)。此步骤占用 $7$ 个块中的 $3$ 个:$2^3 = 1^3 + 3 + \underline{\qquad}$。接下来需要补充什么?

\begin{center}
    \begin{tikzpicture}
        \pic {annotated cuboid};
        \pic at (1,-1,0) {annotated cuboid};
        \pic at (0,-1,1) {annotated cuboid};
    \end{tikzpicture}
\end{center}

新添加的单位立方体在每对块之间产生``间隙'',每个间隙可填充一个单位立方体。此步骤又占用 $3$ 个单位立方体:$2^3 = 1^3 + 3 + 3 + \underline{\qquad}$。接下来呢?

\begin{center}
    \begin{tikzpicture}
        \pic {annotated cuboid};
        \foreach \x in {0,1}
            \foreach \y in {0,1}
                    \pic [fill=white] at (\x,\y,0) {annotated cuboid};
        \pic [fill=white] at (0,0,1) {annotated cuboid};
        \pic [fill=white] at (0,1,1) {annotated cuboid};
        \pic [fill=white] at (1,0,1) {annotated cuboid};
        % \pic [very thick,densely dashed,draw=blue] at (5,0) {annotated cuboid={width=30, height=5, depth=10, opacity=0.2}};
    \end{tikzpicture}
\end{center}

仅剩一个单位立方体位于顶角。添加此单位立方体即完成 $2$ 阶立方体的构建,并得到构建过程的数学描述:

\begin{center}
    \begin{tikzpicture}
        \pic {annotated cuboid};

        \pic [densely dashed] at (3, 0) {annotated cuboid};
        \pic [very thick,draw=blue] at (4.2,0,0) {annotated cuboid};
        \pic [very thick,draw=blue] at (3,1.2,0) {annotated cuboid};
        \pic [very thick,draw=blue] at (3,0,1.4) {annotated cuboid};

        \pic [densely dashed] at (7.5,1,0) {annotated cuboid};
        \pic [densely dashed] at (8.5,0,0) {annotated cuboid};
        \pic [densely dashed] at (7.5,0,1) {annotated cuboid};
        \pic [very thick,draw=red] at (8.7,1.2,-0.1) {annotated cuboid};
        \pic [very thick,draw=red] at (7.3,1,1.4) {annotated cuboid};
        \pic [very thick,draw=red] at (8.7,-0.2,1) {annotated cuboid};

        \foreach \x in {0,1}
            \foreach \y in {0,1}
                \pic [densely dashed, fill=white] at (12+\x,\y,0) {annotated cuboid};
        \pic [densely dashed, fill=white] at (12,0,1) {annotated cuboid};
        \pic [densely dashed, fill=white] at (12,1,1) {annotated cuboid};
        \pic [densely dashed, fill=white] at (13,0,1) {annotated cuboid};
        \pic [very thick,draw=olivegreen] at (13.3,1,1.4) {annotated cuboid};
    \end{tikzpicture}
\end{center}

\begin{center}
    \large $2^3 = 1^3+\textcolor{blue}{3}+\textcolor{red}{3}+\textcolor{olivegreen}{1}$
\end{center}

\subsubsection*{$2$ 阶立方体到 $3$ 阶立方体}

现在我们对这一过程的描述可能有了更清晰的认知,但为了确保理解透彻,让我们再分析两个例子。

我们从 $2$ 阶立方体出发,构建一个 $3$ 阶立方体。(如果你手边有不同尺寸的魔方,甚至可以亲手尝试!)遵循与之前类似的步骤,只需调整相应数字即可。初始结构如图所示:

\begin{center}
    \begin{tikzpicture}[scale=1]
        \pic {annotated cuboid};
        \foreach \x in {0,1}
            \foreach \y in {0,1}
                \foreach \z in {0,1}
                    \pic [fill=white] at (\x,\y,\z) {annotated cuboid};
        \foreach \x in {0,1,2}
            \foreach \y in {0,1,2}
                \foreach \z in {0,1,2}
                    \pic [fill=white] at (\x+4,\y,\z) {annotated cuboid};
    \end{tikzpicture}
\end{center}

此时需要``放大'' $2$ 阶立方体的三个外露面,但放大比例与之前($1$ 阶立方体)\emph{不同},因为初始立方体更大。具体而言,每个面需添加 $2 \times 2$ 的\emph{正方形}(此前添加的是 $1 \times 1$ 正方形)。因此构建方程为:
\[3^2 = 2^3+3\cdot2^2+\underline{\qquad}\]

\begin{center}
    \begin{tikzpicture}[scale=1]
        \foreach \x in {0,1}
            \foreach \y in {0,1}
                \pic [fill=white] at (\x,\y,2) {annotated cuboid};
        \foreach \x in {0,1}
            \foreach \z in {0,1}
                \pic [fill=white] at (\x,2,\z) {annotated cuboid};
        \foreach \y in {0,1}
            \foreach \z in {0,1}
                \pic [fill=white] at (2,\y,\z) {annotated cuboid};
    \end{tikzpicture}
\end{center}

完成此步后,需用 $2 \times 1$ 的模块填充放大面之间的空隙(此前使用 $1 \times 1$ 模块)。此时构建方程为:
\[3^2 = 2^3+3\cdot2^2+3\cdot2+\underline{\qquad}\]

\begin{center}
    \begin{tikzpicture}[scale=1]
        \foreach \x in {0,1,2}
            \foreach \y in {0,1,2}
                \foreach \z in {0,1}
                    \pic [fill=white] at (\x,\y,\z) {annotated cuboid};
        \foreach \x in {0,1}
            \foreach \y in {0,1,2}
                \pic [fill=white] at (\x,\y,2) {annotated cuboid};
        \foreach \y in {0,1}
            \pic [fill=white] at (2,\y,2) {annotated cuboid};
    \end{tikzpicture}
\end{center}

最后仅剩顶角需要填充。完整的构建流程及对应方程为:

\begin{center}
    \begin{tikzpicture}[scale=1]
        \foreach \x in {0,1}
            \foreach \y in {0,1}
                \foreach \z in {0,1}
                    \pic [very thick, fill=white] at (\x,\y,\z) {annotated cuboid};

        \foreach \x in {0,1}
            \foreach \y in {0,1}
                \foreach \z in {0,1}
                    \pic [densely dashed, fill=white] at (\x+6,\y,\z) {annotated cuboid};
        \foreach \x in {0,1}
            \foreach \y in {0,1}
                \pic [very thick,fill=white,draw=blue] at (\x+6,\y,2.4) {annotated cuboid};
        \foreach \x in {0,1}
            \foreach \z in {0,1}
                \pic [very thick,fill=white,draw=blue] at (\x+6,2.3,\z) {annotated cuboid};
        \foreach \y in {0,1}
            \foreach \z in {0,1}
                \pic [very thick,fill=white,draw=blue] at (8.3,\y,\z) {annotated cuboid};

        \foreach \x in {0,1}
            \foreach \y in {0,1}
                \pic [densely dashed,fill=white] at (\x,\y-5,2) {annotated cuboid};
        \foreach \x in {0,1}
            \foreach \z in {0,1}
                \pic [densely dashed,fill=white] at (\x,-3,\z) {annotated cuboid};
        \foreach \y in {0,1}
            \foreach \z in {0,1}
                \pic [densely dashed,fill=white] at (2,\y-5,\z) {annotated cuboid};
        \foreach \x in {0,1}
            \pic [very thick,draw=red,fill=white] at (\x-0.3,-3,2.4) {annotated cuboid};
        \foreach \y in {0,1}
            \pic [very thick,draw=red,fill=white] at (2.3,\y-5,2.4) {annotated cuboid};
        \foreach \z in {0,1}
            \pic [very thick,draw=red,fill=white] at (2.3,-3,\z-0.4) {annotated cuboid};

        \foreach \x in {0,1,2}
            \foreach \y in {0,1,2}
                \foreach \z in {0,1}
                    \pic [densely dashed,fill=white] at (\x+6,\y-5,\z) {annotated cuboid};
        \foreach \x in {0,1}
            \foreach \y in {0,1,2}
                \pic [densely dashed,fill=white] at (\x+6,\y-5,2) {annotated cuboid};
        \foreach \y in {0,1}
            \pic [densely dashed,fill=white] at (8,\y-5,2) {annotated cuboid};
        \pic [very thick,draw=olivegreen] at (8.3,-3,2.4) {annotated cuboid};
    \end{tikzpicture}
\end{center}

\begin{center}
    \large $3^3 = 2^3+\textcolor{blue}{3 \cdot 2^2}+\textcolor{red}{3 \cdot 2}+\textcolor{olivegreen}{1}$
\end{center}

\subsubsection*{$n$ 阶立方体到 $n+1$ 阶立方体}

你知道这个过程如何推广吗?如果我们从 $n$ 阶立方体开始,该如何构造一个 $(n + 1)$ 阶立方体?只需遵循前两个案例的相同步骤即可。首先,通过添加三个\emph{正方形}块来扩大三个暴露面。每个正方形块应该多大?我们希望其尺寸与暴露面相同,因此它们是 $n \times n$ 的正方形块,每个面包含 $n^2$ 个单位立方体:

\begin{center}
    \begin{tikzpicture}[scale=0.20]
        \pic [densely dashed] {annotated cuboid={width=30, height=30, depth=30}};
        \pic at (2,0,0) {annotated cuboid={width=2, height=30, depth=30}};
        \pic at (0,2,0) {annotated cuboid={width=30, height=2, depth=30}};
        \pic at (0,0,3.6) {annotated cuboid={width=30, height=30, depth=3}};
    \end{tikzpicture}
\end{center}
\[(n+1)^3 = n^3+3n^2+\underline{\qquad}\]

接下来,用行块填充放大面之间的间隙。这些行的长度是多少?它们位于新添加正方形块的边缘,因此尺寸均为 $n \times 1$,每个间隙包含 $n$ 个单位立方体:

\begin{center}
    \begin{tikzpicture}[scale=0.20]
        \pic [densely dashed] {annotated cuboid={width=30, height=30, depth=30}};
        \pic [densely dashed,fill=white] at (1,0,0) {annotated cuboid={width=2, height=30, depth=30}};
        \pic [densely dashed,fill=white] at (0,1,0) {annotated cuboid={width=30, height=2, depth=30}};
        \pic [densely dashed,fill=white] at (0,0,1) {annotated cuboid={width=30, height=30, depth=3}};
        \pic at (1,2,3.6) {annotated cuboid={width=30, height=2, depth=3}};
        \pic at (2,1,3.6) {annotated cuboid={width=2, height=30, depth=3}};
        \pic at (2.2,2,2.25) {annotated cuboid={width=2, height=2, depth=30}};
    \end{tikzpicture}
\end{center}
\[(n+1)^3 = n^3+3n^2+3n+\underline{\qquad}\]

最后只剩顶角需要填充!因此,
\[(n+1)^3 = n^3+3n^2+3n+1\]

``等一下!''你可能会说,``我们早就知道这个结果了。''某种程度上确实如此;上述等式是代数恒等式,展开左侧乘积并合并同类项便能轻松得到:
\begin{align*}
    (n + 1)^3 &= (n + 1) \cdot (n + 1)^2\\
    &= (n + 1) \cdot (n^2 + 2n + 1)\\
    &= (n^3 + 2n^2 + n) + (n^2 + 2n + 1) \\
    &= n^3 + 3n^2 + 3n + 1
\end{align*}

那么我们真正取得了什么成果?关键在于以几何和视觉方式推导这个恒等式,揭示了其背后的\emph{归纳}本质。我们试图说明如何从已知``事实''(较小的立方数 $n^3$)推导出新的``事实''(立方数 $(n + 1)^3$),并阐明具体过程。将此与我们研究奇数之和为完全平方数的方法对比:当时对奇数之和的观察已隐含归纳过程,尽管未明确表述。现在请回顾此前的讨论,尝试通过正方形块的几何视角,写出如何用 $n^2$ 表示 $(n + 1)^2$。它是否仍像``明显的''代数恒等式?(若你有余力,思考如何用 $n^4$ 表示 $(n + 1)^4$。这背后是否存在几何直觉?更高次幂又如何?)

这种方法的优势在于,我们能以更小的立方数(直至 $1$)描述任意立方数;即每当表达式中出现立方数时,都可用更小的立方数和剩余项表示其值。此外,每个表达式和剩余项均具有依赖于具体立方数的内在结构。因此,通过上述导出的表达式迭代替换立方数(如 $(n + 1)^3$),直至无法替换为止,将生成具有内在对称性的方程。此概念最好通过实践说明,让我们一探究竟。从任意 $n$ 对应的表达式出发:
\[(n+1)^3 = n^3+3n^2+3n+1\]
类似地,
\[n^3 = (n-1)^3+3(n-1)^2+3(n-1)+1\]

此前对 $n^3$ 的一般论证已证明此式成立,因其仅依赖于 $n \ge 1$ 的条件。遵循相同逻辑步骤,将 $n$ 替换为 $n - 1$,即得上述第二个表达式,也就是 $(n - 1)^3$ 的展开。(此过程对任意 $n$ 均适用吗?思考一下:当 $n \le 0$ 时论证是否合理?例如,物理上能否构建 $(-2) \times (-2) \times (-2)$ 的立方体?)

因此,我们可以替换上一行中的 $n^3$ 项:

\begin{center}
    \begin{tabular}{rcccccccc}
        $(n+1)^3=$ &     & $\cancel{n^3}$ & $+$ &   $3n^2$   & $+$ &   $3n$   & $+$ & $1$\\
                   & $+$ & $(n-1)^3$      & $+$ & $3(n-1)^2$ & $+$ & $3(n-1)$ & $+$ & $1$\\
    \end{tabular}
\end{center}

这仍是一个代数恒等式,但直接展开左侧乘积合并项绝非易事。此处我们反复利用结构特性,导出了全新表达式。继续替换过程!将 $(n - 1)^3$ 替换后得到:

\begin{center}
    \begin{tabular}{rcccccccc}
        $(n+1)^3=$ &     &                    &     &   $3n^2$   & $+$ &   $3n$   & $+$ & $1$\\
                   &     & $\cancel{(n-1)^3}$ & $+$ & $3(n-1)^2$ & $+$ & $3(n-1)$ & $+$ & $1$\\
                   & $+$ & $(n-2)^3$          & $+$ & $3(n-2)^2$ & $+$ & $3(n-2)$ & $+$ & $1$\\
    \end{tabular}
\end{center}

你已预见最终结果了吗?重复此替换过程将使表达式逐列增长,揭示其深刻的数学对称性。但过程终止于何处?为写出简洁的迭代形式并解释各项,需明确终点。回顾立方数研究的起点:我们得出 $2^3 = 1^3 + 3 + 3 + 1$。作为归纳过程的\emph{第一步},它应是逆向构建的\emph{最后一步},由此可得:

\begin{center}
    \begin{tabular}{rcccccccc}
        $(n+1)^3=$ &     &       &     &   $3n^2$   & $+$ &   $3n$   & $+$ & $1$\\
                   &     &       & $+$ & $3(n-1)^2$ & $+$ & $3(n-1)$ & $+$ & $1$\\
                   &     &       & $+$ & $3(n-2)^2$ & $+$ & $3(n-2)$ & $+$ & $1$\\
                   &     &       & $+$ & $3(n-3)^2$ & $+$ & $3(n-3)$ & $+$ & $1$\\
                   &     &       &     & $\vdots$   & $+$ & $\vdots$ & $+$ & $\vdots$\\
                   &     &       & $+$ & $3 \cdot 2^2$ & $+$ & $3 \cdot 2$ & $+$ & $1$\\
                   & $+$ & $1^3$ & $+$ & $3 \cdot 1^2$ & $+$ & $3 \cdot 1$ & $+$ & $1$\\
    \end{tabular}
\end{center}

这\emph{绝对}是意想不到的恒等式!此类式子不仅优美,更能借助已有知识简化表达式。为了演示这一点,对上述各列应用求和符号,将同类项合并为简洁形式:
\[(n+1)^3 = 1^3+3 \cdot \sum_{k=1}^{n}k^2+3 \cdot \sum_{k=1}^{n}k+\sum_{k=1}^{n}1\]
上一章已用多种方法证明:
\[\sum_{k=1}^{n}k = \frac{n(n+1)}{2}\]
将其代入最右两项,简化为:
\[(n+1)^3 = 1^3+3 \cdot \sum_{k=1}^{n}k^2+\frac{3n(n+1)}{2}+n\]

这说明了什么?经过代数推导后,我们完成了何种目标?此前已得前 $n$ 个自然数之和的结论,自然延伸至问题:前 $n$ 个自然数的平方和是多少?答案\emph{呼之欲出}!只需分离求和项并稍作代数变形:
\begin{align*}
    (n+1)^3-1-n-\frac{3n(n+1)}{2} &= 3 \cdot \sum_{k=1}^{n}k^2 \\
    \frac{1}{3}(n+1)^3 - \frac{1}{3}(n+1) - \frac{n(n+1)}{2} &= \sum_{k=1}^{n}k^2
\end{align*}

至此,我们成功推导出前 $n$ 个自然数的平方和公式!当然,左侧表达式可进一步简化,你可以亲自验证一下是否会得到以下表达式:

\[\sum_{k=1}^{n}k^2 = \frac{1}{6}n(n+1)(2n+1) \]

\subsubsection*{``依此类推''并不严谨!}

基于这些工作,我们想指出两点``启示''。首先,归纳论证是发现新颖有趣的数学思想与结论的有效方式。你是否思考过这个问题与奇数之和的联系?如果没有,我们强烈建议你立即尝试,并探索将其推广到四维或五维``立方体''。这种方法不仅能帮助发现有趣结果,对培养抽象思维和掌握归纳过程也极具指导意义。其次,我们需要坦诚说明:目前\emph{尚未}从技术上严格\emph{证明}前 $n$ 个自然数平方和的公式。尽管推导过程看似合理且得出了``正确答案'',但存在一个明显缺陷——省略号的使用!

在展开 $(n + 1)^3$ 推导各列求和项时,用 $\vdots$ 表示中间项虽有助于直观理解,但这种表达方式缺乏数学严谨性。我们如何确信所有中间项都符合预期?如何保证立方体图示能完美对应数学表达式?``递降至 $1$''的确切含义又是什么?

举个例子,考虑下面数列:
\[1,2,3,4,\dots 100\]
人们通常会理解为``$1$ 到 $100$ 之间的所有自然数(含 $1$ 和 $100$)''。这看似合理,但若\emph{实际}指代的是:
\[1, 2, 3, 4, 7, 10, 11, 12, 14, \dots , 100\]
呢?该数列由英文拼写不含字母``i''的数字组成。这并不总是显而易见的。

关键在于:日常交流中使用 $1,2,3,\dots,100$ 无碍,因为听众能准确理解意图。但在数学论证中,我们不能假定读者能凭直觉领会我们的思想;必须尽可能做到\emph{明确}而\emph{严谨}。

你或许认为我们过分挑剔,但更重要的是:存在严格的数学方法可使该论证更加\emph{精确},从而构成完整证明。目前的工作虽有助于建立直观理解,但仍需进一步完善才能确保论证完全可靠。通常需要借助其他数学工具来实现严格证明,我们将在后续章节探讨这些内容后回归本主题。现在,让我们通过新的案例练习这种直观论证方式,并学习判断何时适用归纳法。
