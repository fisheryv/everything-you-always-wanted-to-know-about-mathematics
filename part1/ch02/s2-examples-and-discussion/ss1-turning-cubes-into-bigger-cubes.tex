% !TeX root = ../../../book.tex
\subsection{构建更大的立方体}

为了引出数学归纳法的整体方法,让我们看一道几何题并一起解决它。这个例子是精心挑选出来的,旨在说明当问题具有特定类型的结构时,数学归纳法如何与之关联;具体来说就是,某些真理、事实或洞察\emph{取决于}、\emph{依赖于}或可以从``先前的''事实\emph{推导}得出。这种对先前案例(或多个案例)的依赖使得过程具有\emph{归纳性},当我们观察到这种现象时,应用\emph{归纳法}几乎总是一个好主意。

\subsubsection*{$1$ 阶立方体到 $2$ 阶立方体}

让我们来考察一下立方数,尤其是,让我们试着用前一个立方数来描述一个立方数。想象一个 $1 \times 1 \times 1$ 的立方体,让它作为单位块。我们如何通过添加 $1 \times 1 \times 1$ 的块来构建尺寸为 $2 \times 2 \times 2$ 的``下一个最大''立方体?我们需要添加多少个?从算术上讲,我们知道答案:$2^3 = 8$ 且 $1^3 = 1$,因此我们需要添加 $7$ 个块才能得到正确的体积。好吧,这是一个具体的答案,但它并没有完全告诉我们如何排列这 $7$ 个块来构成一个立方体,也没有让我们深入了解如何回答构建\emph{更大}立方体这个问题。最终,我们想回答的是,需要多少块才能从 $100 \times 100 \times 100$ 的立方体构建出 $101 \times 101 \times 101$ 的立方体,而无需执行大量繁琐的计算;也就是说,我们希望最终找到问题的答案:给定一个 $n \times n \times n$ 的立方体,我们需要添加多少块才能将其构建为 $(n+ 1) \times (n+ 1) \times (n+ 1)$ 的立方体?考虑到这一点,让我们仔细思考这个最初的案例,并尝试用一般性的论点来回答它。

给定一个单位块,并且我们知道必须向其添加 $7$ 个块,让我们试着确定这 $7$ 个块应该放置在哪里,以形成 $2 \times 2 \times 2$ 的立方体。(为了简单起见,对于 $n$ 的任意值,我们把大小为 $n \times n \times n$ 的立方体称为 $n$ 阶立方体。在这个例子中,$n$ 的值只取自然数,即非负整数。)查看下面 $1$ 阶立方体和 $2$ 阶立方体的图片,并试着解释如何从一个立方体构建另一个立方体。

\begin{center}
    \begin{tikzpicture}
        \pic {annotated cuboid};

        \foreach \x in {0,1}
            \foreach \y in {0,1}
                \foreach \z in {0,1}
                    \pic [fill=white] at (4+\x,\y,\z) {annotated cuboid};
        % \pic [very thick,densely dashed,draw=blue] at (5,0) {annotated cuboid={width=30, height=5, depth=10, opacity=0.2}};
    \end{tikzpicture}
\end{center}

这是我们想要使用的一个合理的解释,因为它能指导我们给出从 $n$ 阶立方体构建 $(n+1)$ 阶立方体的一般解释,并且它是一种数学上优雅且简单的解释。从上面的 $1$ 阶立方体开始,将 $3$ 个暴露的面``放大''适当的量,在本例中为 $1$ 块。到目前为止,这占 $7$ 个块中的 $3$ 个:$2^3 = 1^3+3+\underline{\qquad}$。现在还缺哪里?

\begin{center}
    \begin{tikzpicture}
        \pic {annotated cuboid};
        \pic at (1,-1,0) {annotated cuboid};
        \pic at (0,-1,1) {annotated cuboid};
    \end{tikzpicture}
\end{center}

我们刚刚添加的块在每对块之间都产生了``间隙'',并且每个``间隙''都可以用一个块填充。这占了 $7$ 个块中的 $3$ 个:$2^3 = 1^3+3+3+\underline{\qquad}$。接下来呢?

\begin{center}
    \begin{tikzpicture}
        \pic {annotated cuboid};
        \foreach \x in {0,1}
            \foreach \y in {0,1}
                    \pic [fill=white] at (\x,\y,0) {annotated cuboid};
        \pic [fill=white] at (0,0,1) {annotated cuboid};
        \pic [fill=white] at (0,1,1) {annotated cuboid};
        \pic [fill=white] at (1,0,1) {annotated cuboid};
        % \pic [very thick,densely dashed,draw=blue] at (5,0) {annotated cuboid={width=30, height=5, depth=10, opacity=0.2}};
    \end{tikzpicture}
\end{center}

只剩下一个块需要填充,位于最顶角。添加这个块就完成了 $2$ 阶立方体的构建,并且我们还得到了如何使用以下图形和方程以数学的方式描述我们的构建过程:

\begin{center}
    \begin{tikzpicture}
        \pic {annotated cuboid};

        \pic [densely dashed] at (3, 0) {annotated cuboid};
        \pic [very thick,draw=blue] at (4.2,0,0) {annotated cuboid};
        \pic [very thick,draw=blue] at (3,1.2,0) {annotated cuboid};
        \pic [very thick,draw=blue] at (3,0,1.4) {annotated cuboid};

        \pic [densely dashed] at (7.5,1,0) {annotated cuboid};
        \pic [densely dashed] at (8.5,0,0) {annotated cuboid};
        \pic [densely dashed] at (7.5,0,1) {annotated cuboid};
        \pic [very thick,draw=red] at (8.7,1.2,-0.1) {annotated cuboid};
        \pic [very thick,draw=red] at (7.3,1,1.4) {annotated cuboid};
        \pic [very thick,draw=red] at (8.7,-0.2,1) {annotated cuboid};

        \foreach \x in {0,1}
            \foreach \y in {0,1}
                \pic [densely dashed, fill=white] at (12+\x,\y,0) {annotated cuboid};
        \pic [densely dashed, fill=white] at (12,0,1) {annotated cuboid};
        \pic [densely dashed, fill=white] at (12,1,1) {annotated cuboid};
        \pic [densely dashed, fill=white] at (13,0,1) {annotated cuboid};
        \pic [very thick,draw=olivegreen] at (13.3,1,1.4) {annotated cuboid};
    \end{tikzpicture}
\end{center}

\begin{center}
    \large $2^3 = 1^3+\textcolor{blue}{3}+\textcolor{red}{3}+\textcolor{olivegreen}{1}$
\end{center}

\subsubsection*{$2$ 阶立方体到 $3$ 阶立方体}

现在我们可能对如何描述这个过程有了更好的了解,但让我们多考察两个案例,以确保我们有完整的想法。

让我们从 $2$ 阶立方体开始,构造一个 $3$ 阶立方体。(如果碰巧你手上有各种尺寸的魔方,你甚至可以手动尝试一下!)我们可以遵循与上一个案例类似的步骤,只需适当更改数字即可。从相似的图形开始

\begin{center}
    \begin{tikzpicture}[scale=1]
        \pic {annotated cuboid};
        \foreach \x in {0,1}
            \foreach \y in {0,1}
                \foreach \z in {0,1}
                    \pic [fill=white] at (\x,\y,\z) {annotated cuboid};
        \foreach \x in {0,1,2}
            \foreach \y in {0,1,2}
                \foreach \z in {0,1,2}
                    \pic [fill=white] at (\x+4,\y,\z) {annotated cuboid};
    \end{tikzpicture}
\end{center}

可见我们需要``放大'' $2$ 阶立方体的三个暴露面,但在这种情况下,我们需要放大的量与以前($1$ 阶立方体)\emph{不同},因为我们现在使用的是更大的初始立方体。具体来说,每个面必须放大 $2 \times 2$ 的\emph{正方形}块(而在之前的情况下,我们添加了 $1 \times 1$ 的正方形块)因此,此添加过程的方程是
\[3^2 = 2^3+3\cdot2^2+\underline{\qquad}\]

\begin{center}
    \begin{tikzpicture}[scale=1]
        \foreach \x in {0,1}
            \foreach \y in {0,1}
                \pic [fill=white] at (\x,\y,2) {annotated cuboid};
        \foreach \x in {0,1}
            \foreach \z in {0,1}
                \pic [fill=white] at (\x,2,\z) {annotated cuboid};
        \foreach \y in {0,1}
            \foreach \z in {0,1}
                \pic [fill=white] at (2,\y,\z) {annotated cuboid};
    \end{tikzpicture}
\end{center}

这样做之后,我们发现需要使用 $2 \times 1$ 的块来填充这些放大的面之间的间隙(而在之前的情况下,我们添加了 $1 \times 1$ 的块)。到目前为止,添加过程的方程是
\[3^2 = 2^3+3\cdot2^2+3\cdot2+\underline{\qquad}\]

\begin{center}
    \begin{tikzpicture}[scale=1]
        \foreach \x in {0,1,2}
            \foreach \y in {0,1,2}
                \foreach \z in {0,1}
                    \pic [fill=white] at (\x,\y,\z) {annotated cuboid};
        \foreach \x in {0,1}
            \foreach \y in {0,1,2}
                \pic [fill=white] at (\x,\y,2) {annotated cuboid};
        \foreach \y in {0,1}
            \pic [fill=white] at (2,\y,2) {annotated cuboid};
    \end{tikzpicture}
\end{center}

这样做之后,我们看到只剩下顶角需要填充。因此,我们可以描述我们的构建过程及其相应的方程:

\begin{center}
    \begin{tikzpicture}[scale=1]
        \foreach \x in {0,1}
            \foreach \y in {0,1}
                \foreach \z in {0,1}
                    \pic [very thick, fill=white] at (\x,\y,\z) {annotated cuboid};

        \foreach \x in {0,1}
            \foreach \y in {0,1}
                \foreach \z in {0,1}
                    \pic [densely dashed, fill=white] at (\x+6,\y,\z) {annotated cuboid};
        \foreach \x in {0,1}
            \foreach \y in {0,1}
                \pic [very thick,fill=white,draw=blue] at (\x+6,\y,2.4) {annotated cuboid};
        \foreach \x in {0,1}
            \foreach \z in {0,1}
                \pic [very thick,fill=white,draw=blue] at (\x+6,2.3,\z) {annotated cuboid};
        \foreach \y in {0,1}
            \foreach \z in {0,1}
                \pic [very thick,fill=white,draw=blue] at (8.3,\y,\z) {annotated cuboid};

        \foreach \x in {0,1}
            \foreach \y in {0,1}
                \pic [densely dashed,fill=white] at (\x,\y-5,2) {annotated cuboid};
        \foreach \x in {0,1}
            \foreach \z in {0,1}
                \pic [densely dashed,fill=white] at (\x,-3,\z) {annotated cuboid};
        \foreach \y in {0,1}
            \foreach \z in {0,1}
                \pic [densely dashed,fill=white] at (2,\y-5,\z) {annotated cuboid};
        \foreach \x in {0,1}
            \pic [very thick,draw=red,fill=white] at (\x-0.3,-3,2.4) {annotated cuboid};
        \foreach \y in {0,1}
            \pic [very thick,draw=red,fill=white] at (2.3,\y-5,2.4) {annotated cuboid};
        \foreach \z in {0,1}
            \pic [very thick,draw=red,fill=white] at (2.3,-3,\z-0.4) {annotated cuboid};

        \foreach \x in {0,1,2}
            \foreach \y in {0,1,2}
                \foreach \z in {0,1}
                    \pic [densely dashed,fill=white] at (\x+6,\y-5,\z) {annotated cuboid};
        \foreach \x in {0,1}
            \foreach \y in {0,1,2}
                \pic [densely dashed,fill=white] at (\x+6,\y-5,2) {annotated cuboid};
        \foreach \y in {0,1}
            \pic [densely dashed,fill=white] at (8,\y-5,2) {annotated cuboid};
        \pic [very thick,draw=olivegreen] at (8.3,-3,2.4) {annotated cuboid};
    \end{tikzpicture}
\end{center}

\begin{center}
    \large $3^3 = 2^3+\textcolor{blue}{3 \cdot 2^2}+\textcolor{red}{3 \cdot 2}+\textcolor{olivegreen}{1}$
\end{center}

\subsubsection*{$n$ 阶立方体到 $n+1$ 阶立方体}

你知道这个过程如何泛化吗?如果我们从 $n$ 阶立方体开始怎么办?我们如何构造一个 $(n + 1)$ 阶立方体?我们按照前两个案例中使用的相同步骤进行操作。首先,我们通过添加三个\emph{正方形}块来放大三个暴露面。每个正方形块有多大?我们希望每个正方形块的大小与暴露面的大小相同,因此它们是 $n \times n$ 的正方形块,每个面有 $n^2$ 个单位块:

\begin{center}
    \begin{tikzpicture}[scale=0.20]
        \pic [densely dashed] {annotated cuboid={width=30, height=30, depth=30}};
        \pic at (2,0,0) {annotated cuboid={width=2, height=30, depth=30}};
        \pic at (0,2,0) {annotated cuboid={width=30, height=2, depth=30}};
        \pic at (0,0,3.6) {annotated cuboid={width=30, height=30, depth=3}};
    \end{tikzpicture}
\end{center}

\[(n+1)^3 = n^3+3n^2+\underline{\qquad}\]

接下来,我们要用行块填充这些放大面之间的间隙。这些行有多长?它们都位于我们刚刚添加的正方形块的边缘,因此它们的大小均为 $n \times 1$,每个间隙有 $n$ 个块:

\begin{center}
    \begin{tikzpicture}[scale=0.20]
        \pic [densely dashed] {annotated cuboid={width=30, height=30, depth=30}};
        \pic [densely dashed,fill=white] at (1,0,0) {annotated cuboid={width=2, height=30, depth=30}};
        \pic [densely dashed,fill=white] at (0,1,0) {annotated cuboid={width=30, height=2, depth=30}};
        \pic [densely dashed,fill=white] at (0,0,1) {annotated cuboid={width=30, height=30, depth=3}};
        \pic at (1,2,3.6) {annotated cuboid={width=30, height=2, depth=3}};
        \pic at (2,1,3.6) {annotated cuboid={width=2, height=30, depth=3}};
        \pic at (2.2,2,2.25) {annotated cuboid={width=2, height=2, depth=30}};
    \end{tikzpicture}
\end{center}

\[(n+1)^3 = n^3+3n^2+3n+\underline{\qquad}\]

最后就只剩下顶角需要填充了!所以,

\[(n+1)^3 = n^3+3n^2+3n+1\]

``等一下!'' 你可能会说,``我们早就知道这个结果了。'' 某种程度上,是的;上面的等式是一个代数恒等式,我们也可以通过展开左侧的乘积再合并同类项轻松得到它:

\begin{align*}
    (n + 1)^3 &= (n + 1) \cdot (n + 1)^2\\
    &= (n + 1) \cdot (n^2 + 2n + 1)\\
    &= (n^3 + 2n^2 + n) + (n^2 + 2n + 1) \\
    &= n^3 + 3n^2 + 3n + 1
\end{align*}

那么我们真正取得了什么成果呢?其实,以几何和视觉方式推导出这个恒等式其背后的要点是,它展示了这个恒等式如何表示某种\emph{归纳}过程。我们试图解释如何从先前已知的``事实''(下一个最小立方数,$n^3$)推导出该``事实''(立方数,$(n + 1)^3$),并正确解释如何做到这一点。将此与我们研究奇数之和为完全平方数时使用的方法进行比较。我们对技术之和的观察也隐含了一个归纳过程,尽管我们当时没有这样描述,但我们鼓励你现在思考一下这个问题。回顾一下我们之前的讨论,并尝试通过查看正方形块来写出如何用 $n^2$ 来写出 $(n + 1)^2$。它看起来像``明显的''代数恒等式吗?(如果你雄心勃勃,想一想用 $n^4$ 来写出 $(n + 1)^4$ 会发生什么。这背后有任何几何直觉吗?更高次幂呢?)

这种方法的好处是,我们知道如何用更小的立方数(一直到 $1$)来描述一个立方数;也就是说,每当我们在表达式中看到立方数时,我们都知道如何用更小的立方数和一些剩余项来写出该值。此外,这些表达式和剩余项中的每一个都具有某种固有结构,取决于具体讨论的立方数。因此,通过我们上面导出的表达式迭代地替换任意立方数(例如 $(n + 1)^3$),持续进行下去直到无法再替换为止,应该会产生一个具有一定内在对称性的方程。这个想法最好通过实际行动来说明,所以让我们看看会发生什么。让我们从之前推导出的表达式开始,对于 $n$ 的某个任意值,

\[(n+1)^3 = n^3+3n^2+3n+1\]

接着我们就知道一个类似的表达

\[n^3 = (n-1)^3+3(n-1)^2+3(n-1)+1\]

当我们给出 $n^3$ 的上述表达式的一般论证时,我们证明了这个方程成立,因为这仅依赖于 $n \ge 1$ 的事实。我们可以遵循相同的逻辑步骤,在整个过程中将 $n$ 替换为 $n - 1$,并最终得到上面第二个表达式,也就是 $(n - 1)^3$ 的表达式。(对于 $n$ 的任意值,这种情况都会继续下去吗?思考一下。当 $n \le 0$ 时,我们的论证有意义吗?比如说,从不同的立方体构造 $(-2) \times (- 2) \times (-2)$ 的立方体,这在物理上有意义吗?)

因此,我们可以替换上面一行中的 $n^3$ 项

\begin{center}
    \begin{tabular}{rcccccccc}
        $(n+1)^3=$ &     & $\cancel{n^3}$ & $+$ &   $3n^2$   & $+$ &   $3n$   & $+$ & $1$\\
                   & $+$ & $(n-1)^3$      & $+$ & $3(n-1)^2$ & $+$ & $3(n-1)$ & $+$ & $1$\\
    \end{tabular}
\end{center}

这也是一个代数恒等式,但我们肯定不会轻易地想到通过展开左侧的乘积并合并同类项来写出这个恒等式。这里,我们一遍又一遍地利用结果的结构,并得到我们原本不会想到的新表达式。让我们继续这个替换过程,看看它会带我们去到哪里!接下来,我们将 $(n - 1)^3$ 替换为相应的表达式,并得到

\begin{center}
    \begin{tabular}{rcccccccc}
        $(n+1)^3=$ &     &                    &     &   $3n^2$   & $+$ &   $3n$   & $+$ & $1$\\
                   &     & $\cancel{(n-1)^3}$ & $+$ & $3(n-1)^2$ & $+$ & $3(n-1)$ & $+$ & $1$\\
                   & $+$ & $(n-2)^3$          & $+$ & $3(n-2)^2$ & $+$ & $3(n-2)$ & $+$ & $1$\\
    \end{tabular}
\end{center}

也许你已经看清最终会去到哪里?我们可以一遍又一遍地进行这个替换过程,上面式子的列数将不断增长,向我们表明这里发生了一些深刻的、数学上对称的事情。但这个过程在哪里终止呢?我们想要写出这个迭代过程的简洁版本,并能够解释出现的每一项,因此必须知道它在哪里结束。还记得我们研究立方数的第一步吗?我们弄清楚了如何得到 $2^3 = 1^3 + 3 + 3 + 1$。由于这是我们构建此归纳过程的\emph{第一步},因此它应该是我们向后构建的\emph{最后一步},据此,我们可以写出

\begin{center}
    \begin{tabular}{rcccccccc}
        $(n+1)^3=$ &     &       &     &   $3n^2$   & $+$ &   $3n$   & $+$ & $1$\\
                   &     &       & $+$ & $3(n-1)^2$ & $+$ & $3(n-1)$ & $+$ & $1$\\
                   &     &       & $+$ & $3(n-2)^2$ & $+$ & $3(n-2)$ & $+$ & $1$\\
                   &     &       & $+$ & $3(n-3)^2$ & $+$ & $3(n-3)$ & $+$ & $1$\\
                   &     &       &     & $\vdots$   & $+$ & $\vdots$ & $+$ & $\vdots$\\
                   &     &       & $+$ & $3 \cdot 2^2$ & $+$ & $3 \cdot 2$ & $+$ & $1$\\
                   & $+$ & $1^3$ & $+$ & $3 \cdot 1^2$ & $+$ & $3 \cdot 1$ & $+$ & $1$\\
    \end{tabular}
\end{center}

这\emph{绝对}是我们做梦都想不到的恒等式!像这样的式子除了看起来比较漂亮之外,还可以让我们应用之前的知识,简化该表达式。为了了解如何做到这一点,让我们对上面的列应用求和符号,将一列同类项求和写成更简单的表达式:

\[(n+1)^3 = 1^3+3 \cdot \sum_{k=1}^{n}k^2+3 \cdot \sum_{k=1}^{n}k+\sum_{k=1}^{n}1\]

上一章中,我们通过几种不同的方法证明出

\[\sum_{k=1}^{n}k = \frac{n(n+1)}{2}\]

将该式应用于上面表达式最右边的两项,可以化简为

\[(n+1)^3 = 1^3+3 \cdot \sum_{k=1}^{n}k^2+\frac{3n(n+1)}{2}+n\]

这告诉我们什么?在所有这些代数运算之后,我们完成了什么?我们之前证明了前 $n$ 个自然数之和的结果,所以接下来自然要问:前 $n$ 个自然数的平方和是多少?我们如何回答这个问题呢?这是一个恶作剧问题,因为\emph{我们已经得到了}!让我们对上面的方程分离求和项再执行一两步代数步骤即可得到:

\begin{align*}
    (n+1)^3-1-n-\frac{3n(n+1)}{2} &= 3 \cdot \sum_{k=1}^{n}k^2 \\
    \frac{1}{3}(n+1)^3 - \frac{1}{3}(n+1) - \frac{n(n+1)}{2} &= \sum_{k=1}^{n}k^2
\end{align*}

这就是我们所完成的:我们推导出了前 $n$ 个自然数的平方和公式!当然,上面一行左边的表达式不是特别好看,我们可以进一步简化,你可以亲自验证一下是否会得到以下表达式:

\[\sum_{k=1}^{n}k^2 = \frac{1}{6}n(n+1)(2n+1) \]

\subsubsection*{``依此类推''并不严谨!}

基于所有这些工作,我们想指出一些``寓意''。第一个寓意是,归纳论证是发现新的、有趣的数学思想和结论的好方法。你有没有想过这个问题与奇数之和有什么关系?如果没有,我们强烈建议你现在就尝试一下,并思考将其进一步推广到四维或五维``立方体''。除了带给你其他有趣的结果之外,它对于学习抽象思维和应用归纳过程也具有难以置信的指导意义。第二个寓意更像是一种承认:我们还\emph{没有}从技术上\emph{证明}上面的前 $n$ 个自然数平方和的公式。看起来我们的推导是有效的,并得到了``正确答案'',但有一个明显的问题:省略号!

在展开 $(n + 1)^3$ 得到每列的求和项时,在这些列中间写出 $\vdots$ 有助于引导我们的直觉,但\emph{这在不是严谨的数学技术}。我们如何\emph{知道}中间所有项都符合我们的预期?我们如何确定所有立方体图形都能完美地转化为我们写下的数学表达式?``一直递降到 $1$''到底是什么意思?

举个例子,考虑下面的数字列表:
\[1,2,3,4,\dots 100\]
你可能将其解释为``$1$ 到 $100$ 之间的所有自然数(含 $1$ 和 $100$)''。这似乎很合理。但万一我们\emph{实际}指的是下面这个数列呢?
\[1, 2, 3, 4, 7, 10, 11, 12, 14, \dots , 100\]
为什么是这个数列?这当然有可能,我们指的是 $1$ 到 $100$ 的自然数中,英文拼写不含字母``i''的数字的列表。这不是很明显吗?

重点是:当与朋友交流并\emph{表达}一些想法时,写 $1,2,3, \dots, 100$ 没有问题,可以确保受众\emph{确切地}知道你的意思。但总的来说,我们不能假设读者会自然而然地凭直觉理解我们试图传达的内容;我们应该尽可能做到\emph{明确}和\emph{严谨}。

现在你可能会觉得我们在吹毛求疵,但更重要的一点是,有一种数学方法可以使这个论证更加\emph{精确},从而构成一个完全有效的\emph{证明}。到目前为止,我们所做的一切都有助于引导我们的直觉,但我们还需要做更多的工作来确保我们的论点完全令人信服。一般来说,要使此类论证变得严格,还需要一些其他概念,我们将在下一章中研究这些概念,然后再回到这个主题。然而,与此同时,让我们再看一个例子来练习这种直观的论证风格,并识别归纳法何时是一种适用的技术。
