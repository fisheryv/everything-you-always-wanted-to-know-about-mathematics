% !TeX root = ../../../book.tex
\subsection{习题}

\subsubsection*{温故知新}

以口头或书面的形式简要回答以下问题。这些问题全都基于你刚刚阅读的内容,如果忘记了具体定义、概念或示例,可以回顾相关内容。确保在继续学习之前能够自信地作答这些问题,这将有助于你的理解和记忆!

\begin{enumerate}[label=(\arabic*)]
    \item 归纳过程具有哪些特征?
    \item 如何证明 $\sum_{k=1}^{n}k = \frac{n(n+1)}{2}$ 成立?该方法如何体现归纳思想?(若不记得,请重读 \ref{sec:section1.4.2} 节!)
    \item 为何能将上一题求和公式中的 $n$ 直接``替换''为 $n+1$ 并保证其成立?若替换为 $n - 1$ 是否同样可行?
    \item 通过代数运算推导前 $n$ 个自然数平方和的表达式,即验证:
    \[\frac{1}{3}(n+1)^3-\frac{1}{3}(n+1)-\frac{n(n+1)}{2} = \frac{1}{6}n(n+1)(2n+1)\]
    \item 试说明在平面中添加第 $(n+1)$ 条直线恰能产生 $n+1$ 个新区域的论证过程。
    \item 为何不能通过平方前 $n$ 个自然数之和的公式来求其平方和?这种方法的错误本质是什么?
\end{enumerate}

\subsubsection*{小试牛刀}

尝试解答以下问题。这些题目需动笔书写或口头阐述答案,旨在帮助你熟练运用新概念、定义及符号。题目难度适中,确保掌握它们将大有裨益!

\begin{enumerate}[label=(\arabic*)]
    \item 在平面中绘制 $5$ 条满足原题条件的直线,验证是否得到 $16$ 个区域。你能否进一步验证 $6$ 条直线产生 $22$ 个区域吗?
    \item 为数列 $1, 2, 3, 4, \dots , 100$ 提供除``从 $1$ 到 $100$ 的自然数''外的另一种解释。(参考示例:$1$ 到 $100$ 间所有英文拼写不含字母``i''的数字。)
    \item 仿照立方的推导方法,建立 $(n + 1)^4$ 与 $n^4$ 的代数关系式。\\
    (\textbf{挑战题:}能否为所得表达式赋予\emph{几何}解释?)
    \item \textbf{挑战题:}将``直线划分平面区域''问题拓展至三维空间!考虑 $n$ 个平面划分三维空间的情形,将产生多少区域?假设没有两个平面平行,且没有三个或三个以上平面共线。(思考这些条件如何对应原题的约束。)
\end{enumerate}