% !TeX root = ../../../book.tex
\subsection{习题}

\subsubsection*{温故知新}

以口头或书面的形式简要回答以下问题。这些问题全都基于你刚刚阅读的内容,所以如果忘记了具体的定义、概念或示例,可以回去重读相关部分。确保在继续学习之前能够自信地回答这些问题,这将有助于你的理解和记忆!

\begin{enumerate}[label=(\arabic*)]
    \item 归纳过程有哪些特征?
    \item 我们如何证明 $\sum_{k=1}^{n}k = \frac{n(n+1)}{2}$ 是正确的?我们的方法是如何归纳的?(如果你不记得了,请重读第 \ref{sec:section1.4.2} 节!)
    \item 为什么我们可以把上一个问题中提到的求和公式,用 $n+1$ ``替换'' $n$,并且知道它仍然成立?我们也可以将 $n$ 替换为 $n - 1$ 吗?
    \item 通过代数步骤获得前 $n$ 个自然数平方和的最终表达式;也就是说,验证
    \[\frac{1}{3}(n+1)^3-\frac{1}{3}(n+1)-\frac{n(n+1)}{2} = \frac{1}{6}n(n+1)(2n+1)\]
    \item 试着回忆一下向平面中添加第 $(n+1)$ 条直线正好会创建 $n+1$ 个新区域的论点。你能为朋友证明这个论点并说服他/她它是有效的吗?
    \item 求前 $n$ 个自然数的平方和,为什么不能把前 $n$ 个自然数之和的公式平方呢?为什么这是错误的?
\end{enumerate}

\subsubsection*{小试牛刀}

尝试回答以下问题。这些题目要求你实际动笔写下答案,或(对朋友/同学)口头陈述答案。目的是帮助你练习使用新的概念、定义和符号。题目都比较简单,确保能够解决这些问题将对你大有帮助!

\begin{enumerate}[label=(\arabic*)]
    \item 在平面中画 $5$ 条直线(满足原题的两个条件)并验证是否有 $16$ 个区域。你还能验证 $6$ 条线产生 $22$ 个区域吗?
    \item 给出序列 $1, 2, 3, 4, \dots , 100$ 的另一种解释,而不仅仅是从 $1$ 到 $100$ 的所有自然数。(回想一下我们给出的例子:$1$ 到 $100$ 之间所有英文拼写中不含字母"i"的数字。)
    \item 提出一个将 $(n + 1)^4$ 与 $n^4$ 联系起来的代数表达式,就像我们对立方所做的那样。\\ 
    (\textbf{挑战题:}你能为刚刚推导出的表达式给出\emph{几何}解释吗?)
    \item \textbf{挑战题:}让我们将``平面上的线''这题提升一个维度!考虑三维空间中有 $n$ 个平面。会创建多少个区域?假设没有两个平面平行,并且没有三个或以上平面相交于一条直线。(想想这两个条件如何直接类比于``线''那题的给定条件。)
\end{enumerate}