% !TeX root = ../../book.tex
\section{导论}

本章朝着更彻底地研究数学证明并学习构建我们自己的数学证明迈出了一大步,介绍了我们见到的第一个重要的\textbf{证明技术}。正如后文所述,本章旨在作为开胃菜,初尝什么是\textbf{数学归纳法}以及如何使用它。接下来的几章中,我们将严格定义归纳法并\emph{证明}该技术在数学上是可行的。没错,我们真的会去证明它如何有效以及为什么有效!不过,现在我们还要继续研究一些有趣的数学难题,这些精挑细选的问题都使用了归纳技术。

\subsection{目标}

以下简短内容将向你展示本章如何融入本书的体系。这部分内容会描述我们之前的工作将如何发挥作用,还会激发我们为什么要研究本章出现的主题,并告诉你我们的目标,以及你在阅读时应该记住什么来实现这些目标。现在,我们将通过一系列陈述为你总结本章的主要目标,以及本章结束时你应该获得的技能和知识。以下各节将更详细地重申这些想法,但这里将为你提供一个简短的列表以供将来参考。当学完本章后,请返回此列表,看看你是否理解所有这些目标。你明白为什么我们在这里概述它们很重要吗?你能定义我们使用的所有术语吗?你能应用我们描述的技术吗?

\textbf{学完本章后,你应该能够...}

\begin{itemize}
    \item 定义什么是归纳论证,以及将给出的论证分类为归纳论证或非归纳论证。
    \item 根据要解决的问题的结构来决定何时使用归纳论证。
    \item 通过类比启发式地描述数学归纳法。
    \item 通过比较和对比来识别和描述不同类型的归纳论证,并识别产生这些相似点和差异点的相应问题的基本结构。
\end{itemize}

\subsection{承上}

与上一章一样,我们假设你只熟悉基本代数和算术,以及视觉、几何直觉,对除此之外的更高等的数学知之甚少。但我们会频繁使用求和和求积符号,因此,如果你觉得自己的符号技能有所欠缺,请回看第 \ref{sec:section1.3.5} 节。

\subsection{动机}

回顾一下 \ref{sec:section1.4.3} 节中的问题,我们证明了前 $n$ 个奇数之和恰好等于 $n^2$。我们首先通过将求和项(奇数)排列为正方形相继变大的``角块'',从几何角度观察到这种模式。然而,我们证明该结论的第一种方法似乎并不依赖此观察,而是以\emph{代数的}方式利用了先前的结论(关于偶数\emph{和}奇数之和);也就是说,我们对一些方程进行了一些复杂的操作(乘法和减法等等),然后 --- 瞧!--- 得到了我们预期的结果。你对这种方法有何看法?是不是感觉很满足呢?在某种程度上,它与我们一开始的几何解释不太相符,所以它的效果如此之理想可能会令人惊讶。(也许这种方法有\emph{不同的}几何解释。你能找到吗?)

我们的第二种方法是对最初的几何观察进行建模。我们将视觉形式转化为代数形式;具体来说,求和与正方形的面积有关,而求和项与该正方形的特定部分有关。我们在同一问题的不同解释之间建立了\emph{对应关系},找到了一种将一种解释与另一种解释联系起来的方法,这样我们就可以使用任何一种解释,并知道我们正在证明总体结果。视觉解释的好处是,它使我们能够利用称为\textbf{数学归纳法}的通用证明策略,有时简称为\textbf{归纳法}。(\emph{归纳法}一词也有一些非数学含义,例如在电磁学或哲学论证中,但在本书的上下文中,当我们说\emph{归纳法}时,我们指的是\emph{数学归纳法}。)归纳法到底是什么?它是如何工作的?我们什么时候可以使用这个策略?我们如何使策略适应特定的问题?在某些情况下是否有更有用的策略变体?这些都是我们希望在本章中回答的问题。

我们要谈论的第一个主题是我们在最后一段中没有问的一个问题,即``\emph{为什么}要用归纳法?\emph{为什么}要关注它?'' 基于第 \ref{sec:section1.4.3} 节中的问题,数学归纳法似乎并非完全必要,因为不用归纳法,可能也有其他方法可以给出证明。根据背景的不同,这很可能是正确的,但我们想从一开始就明确的一点是,\emph{归纳法非常有用!}在许多情况下,归纳证明是最简明的方法,并且它是一种众所周知的通用策略,可以应用于各种此类情况。此外,问题需要具有某种特定\text{结构}才能应用归纳法,即结果的一``部分''依赖于``前一部分''。(当然,``部分''和``依赖性''取决于上下文。)认识到归纳法的适用性,并实际经历随后的证明过程,通常会告诉我们一些有关问题的内在结构的信息。即便归纳证明失败也是如此!也许问题的某个特定部分``破坏''了归纳过程,识别该特定部分可能会有所帮助且让我们富有洞察力。

我们希望首先通过一些说明性的例子来激发这些观点,然后我们再提供数学归纳法的完整\emph{定义},以展示该方法在一般情况下是如何工作的。(完全\text{严格}的定义要推迟到稍后的章节才能给出,需要等到我们定义和研究了一些相关概念之后,例如集合论和逻辑陈述与蕴涵。不过,就目前而言,我们给出的定义足以解决一些有趣的难题,并让我们可以将归纳法作为一种通用证明策略进行讨论。)

\subsection{忠告}

请注意,我们仍在朝着数学严谨的目标迈进,或者说在本书和课程的范围与时间安排内尽可能地实现这一目标。我们在本章中提出的一些主张将在稍后得到澄清并在技术上得到证明,这需要自然数和一些基本数理逻辑为基础。一切都有恰当的安排!

尽管如此,本章仍然非常重要,因为我们将继续介绍解决数学问题的过程,应用我们现有的知识和技术来发现新事实并向他人做出解释。 此外,数学归纳法是一种基本的证明技术,很可能会出现在所有其他数学课程中!这是因为它的实用性以及归纳性质在整个数学世界中的普遍性决定的。