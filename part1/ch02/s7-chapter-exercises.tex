% !TeX root = ../../book.tex
\section{本章习题}

以下是一些问题,帮助你熟悉归纳式证明。我们不需要完全严格的证明,只要你能很好地描述过程并写下步骤即可。等我们掌握了数学归纳原理 (PMI) 和相应的证明策略后,再回头来严格证明这些问题。

\begin{exercise} \label{exc:exercises2.7.1}
    证明以下求和公式对于每个自然数都成立,对于 $n=0$ 也成立。
    \[\sum_{i=0}^{n}2^i=2^{n+1}-1\]
    后续问题:用这个结果来说明在 $2^n$ 支球队的单赛淘汰赛中需要进行多少场比赛才能确定获胜者。(例如,NCAA 疯狂三月锦标赛就使用这种赛制,其中 $n = 6$。)
\end{exercise}

\begin{exercise}
    证明对于每个大于等于 $2$ 的自然数 $n$, $3^n \ge 2^{n+1}$。
\end{exercise}

\begin{exercise}
    对于哪些自然数 $n$,下列不等式成立?先陈述结论,然后再证明它。
    \begin{enumerate}
        \item $2^n \ge (n + 1)^2$
        \item $2^n \ge n!$
        \item $3^{n+1} > n^4$
        \item $n^3 + (n + 1)^3 > (n + 2)^3$
    \end{enumerate}
\end{exercise}

\begin{exercise}
    \textbf{末日游戏}:两名玩家轮流从日历中命名日期。每一回合中,玩家可以增加月份或日期,但不能同时增加。起始位置为 1 月 1 日,说出 12 月 31 日的人获胜。确定第一个玩家的必胜策略。
    例如,玩家 1 获胜的一系列动作如下:
    \begin{itemize}
        \item 玩家 1: 1 月 10 日;
        \item 玩家 2: 3 月 10 日;
        \item 玩家 1: 8 月 10 日;
        \item 玩家 2: 8 月 25 日;
        \item 玩家 1: 8 月 28 日;
        \item 玩家 2: 11 月 28 日;
        \item 玩家 1: 11 月 30 日;
        \item 玩家 2: 12 月 30 日;
        \item 玩家 1: 12 月 31 日。
    \end{itemize}
    我们所说的\emph{必胜}策略是指玩家 1 遵循的一种游戏方法,无论玩家 2 做什么,都可以\emph{保证}获胜。
\end{exercise}

\begin{exercise}
    找到并证明\emph{几何级数}求和公式,几何级数定义如下:
    \[\sum_{i=0}^{n-1}q^i\]
    其中 $q$ 为实数,$n$ 为自然数。(提示:留意 $q = 1$ 的情况。)
\end{exercise}

\begin{exercise}
    写一个依赖于 $n$ 的句子,使得该句子对于从 $1$ 到 $99$(含)的所有 $n$ 值都为真,但当 $n = 100$ 时该句子为假。
\end{exercise}

\begin{exercise}
    下面``错误证明''证明了对于所有 $n$, $a^n=1$。请指出问题在哪里?
    \begin{spoof}
        设 $a$ 为非零实数。请注意 $a^0 = 1$。另请注意我们可以归纳地写出
        \[a^{n+1} = a^n \cdot a = a^n \cdot \frac{a^n}{a^{n-1}} = 1 \cdot \frac{1}{1} = 1\]
    \end{spoof}
\end{exercise}

\begin{exercise}
    未来社会中,只有两种面额的货币:一种价值 $3$ Brendan 的硬币,一种价值 $8$ Brendan 的硬币。还有一项全国性法令,店主只能收取可以使用这两种硬币\textbf{精确支付}的价格。

    店主可能向你收取的一杯咖啡的法定价格是多少?
    
    \textbf{提示:}尝试一些较小的值,看看会发生什么。
\end{exercise}

\begin{exercise}
    对于某个任意自然数 $n$,考虑大小为 $2^n \times 2^n$ 的棋盘。从棋盘上移除\textbf{任意}一个方格。 是否可以用 L 形三联骨牌来密铺剩余的方格?

    如果你的答案是肯定的,请证明这一点。

    如果您的答案是否定的,请提供反例论证。(也就是说,找到一个 $n$ 使得任何方法都无法密铺棋盘,并说明为什么会出现这种情况。)
\end{exercise}

\begin{exercise}
    考虑一个 $n \times n$ 的正方形网格。该网格内存在多少个任意大小的子方格?例如,当 $n = 2$ 时,答案为 $5$:有 $4$ 个$1 \times 1$ 方格和 $1$ 个 $2 \times 2$ 方格。找到你的答案的公式并尝试证明它是正确的。
\end{exercise}

\begin{exercise}
    证明,在至少有 $2$ 人的队列中,如果第一个人是女性,最后一个人是男性,那么在队列的某个位置一定存在一个男性紧邻女性身后。
\end{exercise}

\begin{exercise}
    对于每个自然数 $n$,证明 $n^3 - n$ 是 3 的倍数。
\end{exercise}

\begin{exercise}
    \textbf{二进制 $n$ 元组}是由 \verb|0| 和 \verb|1| 组成的有序字符串,字符串中共有 $n$ 个数字。提供一个\emph{归纳论证}来解释为什么有 $2^n$ 个可能的二进制 $n$ 元组。
\end{exercise}

\begin{exercise}
    回想一下,\textbf{斐波那契数}是通过设 $f_0 = 0$ 和 $f_1 = 1$,然后对于每个 $n \ge 2$,设 $f_n = f_{n-1} + f_{n-2}$ 来定义的。这会产生序列 $0, 1, 1, 2, 3, 5, 8, 13, 21, 34, \dots$

    你可能不知道,斐波那契数列也有\emph{封闭形式};也就是说,除了上面给出的常规递归定义外,还有一个特定\emph{公式}来定义它。那就是:
    \[f_n = \frac{1}{\sqrt 5}\Bigg[\Bigg(\frac{1+\sqrt 5}{2}\Bigg)^n - \Bigg(\frac{1-\sqrt 5}{2}\Bigg)^n\Bigg]\]
    证明该公式对于所有 $n \ge 0$ 都正确。
\end{exercise}

\begin{exercise}
    再次考虑斐波那契数 $f_n$,证明以下内容:
    \begin{enumerate}
        \item $\displaystyle{\sum_{i=0}^{n}f_i = f_{n+2} - 1}$
        \item $\displaystyle{\sum_{i=0}^{n}f_i^2 = f_n \cdot f_{n+1}}$
        \item $\displaystyle{f_{n-1} \cdot f_{n+1} - f_n^2 = (-1)^n}$
        \item $\displaystyle{f_{m+n} = f_n \cdot f_{n+1} + f_{m-1} \cdot f_n}$
        \item $\displaystyle{f_n^2 + f_{n+1}^2 = f_{2n+1}}$
    \end{enumerate}
\end{exercise}

\begin{exercise}
    尝试提供一个归纳论证来解释为什么每个 $n ≥ 2$ 的自然数都可以写成质数的乘积。你能证明该乘积的\emph{唯一性}吗?也就是说,你能解释为什么\emph{只有唯一一种方法}可以将自然数分解为质数吗?
\end{exercise}

\begin{exercise}
    证明
    \[\sum_{k=1}^{n} k \cdot k! = 1 \cdot 1! + 2 \cdot 2! + 3 \cdot 3! + \dots + n \cdot n! = (n+1)!-1\]
\end{exercise}

\clearpage

\begin{exercise}
    下面的``错误证明''得出所有的笔颜色都一样,问题出在哪儿?
    \begin{spoof}
        考虑一组数量为 $1$ 的笔。由于只有 $1$ 支笔,所以它的颜色肯定与自身相同。

        假设任意一组 $n$ 支笔在组内只有一种颜色。(注意:我们已经解释了为什么这个假设对于 $n = 1$ 是有效的,所以我们可以做出这个假设。)取任意一组 $n + 1$ 支笔。将它们排列在桌子上,从左到右用 $1$ 到 $n + 1$ 编号。查看其中的前 $n$ 个,即查看笔 $1,2,3, \dots , n$。这是一组 $n$ 支笔,因此根据假设,该组只有一种颜色。(我们还不知道是什么颜色。)然后,查看最后 $n$ 支笔;即查看笔 $2,3, \dots ,n+1$。这也是一组 $n$ 支笔,因此根据假设,该组也只有一种颜色。而 $2$ 号笔恰好属于这两个组。因此,无论 $2$ 号笔的颜色是什么,这也是\dotuline{两组}中每支笔的颜色。因此,所有 $n+1$ 支笔具有相同的颜色。

        根据归纳法,这表明任何一组笔,无论多少,都只有一种颜色。那么,纵观世界上有限的钢笔,我们应该只能找到一种颜色。
    \end{spoof}
\end{exercise}

\begin{exercise}
    $\star$ 这题\emph{极其难解},摘自著名数学家陶哲轩(Terence Tao)的博客(\href{https://terrytao.wordpress.com/2011/04/07/the-blue-eyed-islanders-puzzle-repost/}{详见链接})

    有一座岛屿,岛上居住着一个部落。这个部落有 $1000$ 人,有着不同颜色的眼睛。然而,他们的宗教信仰禁止他们知道自己眼睛的颜色,甚至禁止讨论这个话题;因此,每个居民都可以(并且确实可以)看到所有其他居民眼睛的颜色,但无法知道自己眼睛的颜色(不考虑反射表面)。如果部落成员确实发现了自己眼睛的颜色,那么他们的宗教信仰就会迫使他们第二天中午在村庄广场举行自杀仪式,让所有人围观。所有的部落成员都是高度逻辑和虔诚的,他们都知道其他人也是高度逻辑和虔诚的(并且他们都知道他们都知道其他人是高度逻辑和虔诚的,等等)。

    (就这个逻辑谜题而言,``高度逻辑的''意味着能够从岛民可用的信息和观察中逻辑推断出的任何结论,该岛民将自动知晓。)

    事实证明,在这 $1000$ 名岛民中,有 $100$ 人是蓝眼睛,$900$ 人是棕眼睛,尽管岛民最初并没有意识到这些统计数据(当然,他们每个人只能看到 $1000$ 名部落居民中的 $999$ 人)。

    一天,一名蓝眼睛的外国人来到岛上,并赢得了部落的完全信任。

    一天晚上,他向整个部落发表讲话,感谢他们的热情款待。

    然而,由于不了解当地习俗,这名外国人在称呼中错误地提及了眼睛的颜色,并表示\emph{在世界的其他角落看到另一个像我这样的蓝眼睛的人是多么不同寻常}。

    这种失礼(如果有的话)会对部落带来什么影响?
\end{exercise}