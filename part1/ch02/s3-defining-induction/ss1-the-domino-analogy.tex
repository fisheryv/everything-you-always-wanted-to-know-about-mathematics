% !TeX root = ../../../book.tex
\subsection{多米诺骨牌类比}

假设我们有一副特殊的多米诺骨牌,它包含无穷多张骨牌!每张骨牌可以书写任意内容,而非标准的点数。这些骨牌沿无限延伸的桌面排列成无限长的一行。从侧面观察时,每张骨牌下方标有位置标签:

\begin{center}
    \begin{tikzpicture}
        \foreach \x in {1,...,5}
        {
            \pic [fill=white] at (\x, 0, 0) {annotated cuboid={width=3, height=30, depth=10}};
            \node[below] at (\x, -3){\tiny $n=\x$};
        }
        \node[anchor=center] at (6.5, -1.5){\LARGE $\dots \cdot$};
        % \pic [very thick,densely dashed,draw=blue] at (5,0) {annotated cuboid={width=30, height=5, depth=10, opacity=0.2}};
    \end{tikzpicture}
\end{center}

对于这个特定的例子,我们需要验证公式
\[\sum_{k=1}^{n}k^2 = \frac{1}{6}n(n+1)(2n+1)\]
为此,设想每张多米诺骨牌记载一个特定``事实''。具体来说,我们可以想象第一张多米诺骨牌上写有表达式
\[\sum_{k=1}^{1}k^2 = \frac{1}{6}(1)(1)(3)\]
第二张多米诺骨牌上写有表达式
\[\sum_{k=1}^{2}k^2 = \frac{1}{6}(2)(3)(5)\]
推而广之,第 $n$ 张骨牌写有如下``事实'':
\[\sum_{k=1}^{n}k^2 = \frac{1}{6}n(n+1)(2n+1)\]
由于多米诺骨牌具有连锁倾倒的特性,我们约定骨牌倒下即表示其记载的``事实''是\emph{真实命题}。由此将多米诺骨牌的物理解释与公式有效性的数学解释联系起来。

我们已手动验证 $n=1$ 的情形:$1^2=\frac{1}{6}(1)(2)(3)$,故第一张骨牌记载的命题为真,必将倒下。同理验证 $n=2$ 后,第二张骨牌也会倒下:

\begin{center}
    \begin{tikzpicture}
        \foreach \x in {1,2}
        {
            \pic [fill=white, rotate=-30, anchor=south] at (\x, -0.15, 0) {annotated cuboid={width=3, height=32, depth=10}};
            \node[below] at (\x-1.5, -3){\tiny $n=\x$};
        }
        \foreach \x in {3,4,5}
        {
            \pic [fill=white] at (\x, 0, 0) {annotated cuboid={width=3, height=30, depth=10}};
            \node[below] at (\x, -3){\tiny $n=\x$};
        }
        \node[anchor=center] at (6.5, -1.5){\LARGE $\dots \cdot$};
        % \pic [very thick,densely dashed,draw=blue] at (5,0) {annotated cuboid={width=30, height=5, depth=10, opacity=0.2}};
    \end{tikzpicture}
\end{center}
然而,若继续逐个检验,又会陷入原先的困境——我们不可能验证\emph{每张}骨牌。真正的需求是捕捉多米诺效应的精髓:一张骨牌倒下将触发下一张倒下。这要求我们建立相邻骨牌所载``事实''的数学关联。

让我们看看前两张多米诺骨牌的情况。既然知道骨牌 $1$ 倒下,我们能否在不重写所有求和项的情况下确保骨牌 $2$ 倒下?两块骨牌上的陈述有何关联?每个陈述都是自然数的平方和,且第二张骨牌的陈述正好多出一项。因此,利用骨牌 $1$ 上已知的\emph{真实陈述},可以\emph{验证}骨牌 $2$ 上陈述的真实性:
\[\sum_{k=1}^{2}k^2 = 1^2+2^2=1+2^2=5=\frac{1}{6}(2)(3)(5)\]
尽管节省的唯一``工作''只是免于计算 $1^2=1$,但让我们在更大数字上应用此过程以凸显其优势。\emph{假设}骨牌 $10$ 已经倒下(其求和的完整验证已在前文给出),这意味着我们\emph{知道}
\[\sum_{k=1}^{10}k^2 =\frac{1}{6}(10)(11)(21)=285\]
是一个\emph{真实陈述}。利用它来验证骨牌 $11$ 的陈述:
\[\sum_{k=1}^{11}k^2 =\frac{1}{6}(11)(12)(23)\]
骨牌 $11$ 上的求和公式有 $11$ 项,前 $10$ 项正是骨牌 $10$ 上的求和!因此只需分离第 $11$ 项并代入已知结果:
\begin{align*}
    \sum_{k=1}^{11}k^2 &= (1^2+2^2+\dots+10^2)+11^2\\
    &=\sum_{k=1}^{10}k^2+11^2\\
    &=385+121\\
    &=506\\
    &=\frac{1}{6}3036=\frac{1}{6}(11)(12)(23)
\end{align*}
节省的工作量显而易见!既然已知前 $10$ 项之和,何必重新计算?

现在设想对\emph{所有} $n$ 值\emph{同时}实施此过程!若能证明每当骨牌 $n$ 倒下,骨牌 $(n + 1)$ \emph{必然}倒下,这意味着什么?回顾无穷骨牌序列:已知骨牌 $1$ 因手动检验而倒下,加之``骨牌 $n$ 撞倒骨牌 $(n + 1)$''的普适性验证,可推得骨牌 $1$ 撞倒骨牌 $2$,骨牌 $2$ 撞倒骨牌 $3$,骨牌 $3$ 撞倒骨牌 $4$,…… 如此传递下去,整列骨牌终将全部倒下!本质上,整个过程可归结为\emph{两步}:

\begin{enumerate}[label=(\arabic*)]
    \item 确保第一张骨牌倒下;
    \item 确保每张骨牌都能撞倒下一张骨牌。
\end{enumerate}
仅凭这两步,便能\emph{保证}所有骨牌倒下,从而\emph{证明}每个公式对\emph{任意}自然数 $n$ 成立。

我们已经完成步骤 (a),现在需要完成步骤 (b)。此前已针对特定案例(骨牌 $1$ 撞倒骨牌 $2$、骨牌 $10$ 撞倒骨牌 $11$)执行了此操作,现在将其推广到任意 $n$ 值。我们\emph{假设}:对于某个\emph{特定}但\emph{任意}的 $n$,多米诺骨牌 $n$ 会倒下,这意味着方程
\[\sum_{k=1}^{n}k^2=\frac{1}{6}n(n+1)(2n+1)\]
为\emph{真实陈述}。现在需将其关联到骨牌 $(n+1)$ 的陈述,并应用上述等式信息。将 $n+1$ 项的和拆分为 $n$ 项和与末项:
\[\sum_{k=1}^{n+1}k^2 = (1^2+2^2+\dots+n^2+(n+1)^2)=\sum_{k=1}^{n}k^2+(n+1)^2\]
根据骨牌 $n$ 倒下的假设(即其命题为真),可得
\[\sum_{k=1}^{n+1}k^2 = \frac{1}{6}n(n+1)(2n+1)+(n+1)^2\]
这与骨牌 $(n+1)$ 的命题是否一致?骨牌 $(n+1)$ 的``事实''与骨牌 $n$ 类似,只是将``$n$''替换为``$n + 1$'':
\[\sum_{k=1}^{n+1}k^2 = \frac{1}{6}\big(n+1\big)\big((n+1)+1\big)\big((2(n+1)+1)\big)=\frac{1}{6}(n+1)(n+1)(2n+3)\]
目前还不清楚我们推导出的表达式是否实际上等于上面的式子。我们可以尝试化简该表达式,并将其分解为与上面表达式``类似''的新表达式,但展开两个表达式并比较所有项可能会更容易。(这基于这样的一般思想:展开因式分解后的多项式比进行因式分解要容易得多。)对于第一个表达式,我们有
\begin{align*}
    \frac{1}{6}n(n + 1)(2n + 1) + (n + 1)^2 &=\frac{1}{6}n(2n^2 + 3n + 1) + (n^2 + 2n + 1)\\
    &= \frac{1}{3}n^3 + \frac{1}{2}n^2 + \frac{1}{6}n + n^2 + 2n + 1 \\
    &= \frac{1}{3}n^3 + \frac{3}{2}n^2 + \frac{13}{6}n + 1
\end{align*}
对于第二个表达式,我们有
\begin{align*}
    \frac{1}{6}(n+1)(n + 2)(2n + 3) &=\frac{1}{6}(n+1)(2n^2 + 7n+6)\\
    &= \frac{1}{6}\big[(2n^3 + 7n^2 + 6n) + (2n^2 + 7n + 6)\big] \\
    &= \frac{1}{3}n^3 + \frac{3}{2}n^2 + \frac{13}{6}n + 1
\end{align*}
可见两式相等!此外,请注意,这比尝试整理其中一个表达式并将其``变形''为另一个表达式要容易得多。我们通过展开两式并最终得到相同的表达来证明它们是相同的。现在,让我们回顾并总结我们所取得的成果:
\begin{enumerate}
    \item 我们将证明公式
    \[\sum_{k=1}^{n+1}k^2 = \frac{1}{6}n(n+1)(2n+1)+(n+1)^2\]
    对于\emph{所有} $n$ 值成立类比为推倒无穷多的多米诺骨牌。
    \item 通过手工计算验证骨牌 $1$ 的命题成立,骨牌 $1$ 倒下;
    \item \emph{假设}骨牌 $n$ 的命题为真,由骨牌 $n$ 命题成立推出骨牌 $(n+1)$ 命题成立,从而证明骨牌 $n$ 会撞倒骨牌 $(n+1)$。
    \item 由此保证所有骨牌都会倒下,因此公式对\emph{所有} $n$ 都成立。
\end{enumerate}
此方法是否严谨?是否已\emph{严格证明}公式对所有自然数 $n$ 都成立?若存在 $n$ 使得公式失效,这对多米诺骨牌体系意味着什么?

请记住,这里的多米诺骨牌类比只是理解归纳法工作原理的一个直观指引,并非建立在严格的数学基础之上。建立严格的数学基础将是接下来几章的目标。现在,让我们回顾本章讨论的另一个例子:直线划分平面区域。同样,在推导公式 $R(n)$ 时使用省略号显得繁琐,我们希望避免这种做法。让我们尝试将多米诺骨牌类比应用于此问题。

设想我们定义 $R(n)$ 为 $n$ 条直线在平面上划分出的不同区域的数量,这些直线满足互不平行且任意三条(或更多)直线不共点。进一步设想,我们在代表第 $n$ 步的骨牌上写下``$R(n) = 1 + \frac{n(n+1)}{2}$''这一``事实''。能否按照与之前相同的逻辑来验证所有骨牌都会倒下?

首先,需要验证骨牌 $1$ 是否会倒下。这等同于验证命题``$R(1) = 1+\frac{1(2)}{2} = 1+1 = 2$''是否成立。这显然成立,正如我们之前验证过的:一条直线将平面划分为两个区域。其次,需要证明对于\emph{任意} $n$,第 $n$ 块骨牌倒下必定导致第 $(n + 1)$ 块骨牌倒下。也就是说,我们\emph{假设}``$R(n) = 1 + \frac{n(n+1)}{2}$''对某个特定的 $n$ 成立,然后\emph{证明}``$R(n + 1) = 1 + \frac{(n+1)(n +2)}{2}$''也必然成立。如何证明?沿用之前的思路,建立 $R(n + 1)$ 与 $R(n)$ 的关系。向\emph{任意}满足条件的 $n$ 条直线的图形中添加一条新直线,通过几何分析,我们得到关系式 $R(n+1) = R(n) + n + 1$。利用此关系以及第 $n$ 块骨牌倒下的假设,可得:
\[R(n + 1) = R(n) + n + 1 = 1 +\frac{n(n+1)}{2}+ n + 1\]
这个结果是否与骨牌 $(n + 1)$ 上的表达式一致?通过化简比较即可验证:
\[1 +\frac{n(n+1)}{2}+ n + 1=2+n+\frac{n^2+n}{2} = \frac{1}{2}n^2+\frac{3}{2}n+2\]
以及
\[1 + \frac{(n+1)(n +2)}{2} = 1+\frac{n^2+3n+2}{2} =  \frac{1}{2}n^2+\frac{3}{2}n+2\]
结果完全相同!因此,我们证明了对于\emph{任意} $n$,骨牌 $n$ 倒下\emph{必然}导致骨牌 $(n+1)$ 倒下。

思考一下,使用这种``多米诺骨牌技术''进行的证明,与我们之前为推导该公式所采用的方法有何不同?我们在本节中是否使用了省略号?为何这种证明方式更优?我们是否曾用多米诺骨牌归纳技术来推导公式本身?
