% !TeX root = ../../../book.tex
\subsection{类比多米诺骨牌}

假设我们有一副多米诺骨牌。这是一副特殊的多米诺骨牌,里面有无穷多的骨牌!我们可以想象在上面有任何我们想要的内容,而不是标准的点数。我们还假设它们沿着无限延申的桌面排列成无限长的一行。从侧面看多米诺骨牌,可以看到每张牌下面有一个标签,以便知道其在行中的位置:

\begin{center}
    \begin{tikzpicture}
        \foreach \x in {1,...,5}
        {
            \pic [fill=white] at (\x, 0, 0) {annotated cuboid={width=3, height=30, depth=10}};
            \node[below] at (\x, -3){\tiny $n=\x$};
        }
        \node[anchor=center] at (6.5, -1.5){\LARGE $\dots \cdot$};
        % \pic [very thick,densely dashed,draw=blue] at (5,0) {annotated cuboid={width=30, height=5, depth=10, opacity=0.2}};
    \end{tikzpicture}
\end{center}

对于这个特定的例子,要验证公式
\[\sum_{k=1}^{n}k^2 = \frac{1}{6}n(n+1)(2n+1)\]
我们想象每张多米诺骨牌上都写有一个特定的``事实''。具体来说,我们可以想象第一张多米诺骨牌上写有表达式
\[\sum_{k=1}^{1}k^2 = \frac{1}{6}(1)(1)(3)\]
第二张多米诺骨牌上写有表达式
\[\sum_{k=1}^{2}k^2 = \frac{1}{6}(2)(3)(5)\]
总的来说,我们想象第 $n$ 张多米诺骨牌上写着以下``事实'':
\[\sum_{k=1}^{n}k^2 = \frac{1}{6}n(n+1)(2n+1)\]
由于多米诺骨牌本来就是要互相推倒、互相撞倒的,所以让我们假设每当多米诺骨牌倒下时,就意味着上面的相应``事实''是一个\emph{真实陈述}。这就将多米诺骨牌的物理解释与我们推导公式有效性的数学解释联系起来。

我们手动检验了 $n = 1$ 的求和:$1^2=\frac{1}{6}(1)(2)(3)$,因此,第一张多米诺骨牌上的事实是一个真实陈述,所以我们知道第一张多米诺骨牌一定会倒下。我们还手动检验了 $n = 2$ 的求和,因此我们知道第二张多米诺骨牌也会倒下:

\begin{center}
    \begin{tikzpicture}
        \foreach \x in {1,2}
        {
            \pic [fill=white, rotate=-30, anchor=south] at (\x, -0.15, 0) {annotated cuboid={width=3, height=32, depth=10}};
            \node[below] at (\x-1.5, -3){\tiny $n=\x$};
        }
        \foreach \x in {3,4,5}
        {
            \pic [fill=white] at (\x, 0, 0) {annotated cuboid={width=3, height=30, depth=10}};
            \node[below] at (\x, -3){\tiny $n=\x$};
        }
        \node[anchor=center] at (6.5, -1.5){\LARGE $\dots \cdot$};
        % \pic [very thick,densely dashed,draw=blue] at (5,0) {annotated cuboid={width=30, height=5, depth=10, opacity=0.2}};
    \end{tikzpicture}
\end{center}
然而,继续这样下去会让我们回到与之前相同的问题:我们不想检查\emph{每一张}多米诺骨牌以确保它倒下。我们真的很想封装多米诺骨牌的物理概念 --- 即,当多米诺骨牌倒下时,会碰倒下一张多米诺骨牌,依此类推 --- 并以某种方式将相邻多米诺骨牌上的``事实''联系起来。

让我们看看前两张多米诺骨牌的情况。知道骨牌 $1$ 倒下,我们能在不重写所有求和项的情况下保证骨牌 $2$ 倒下吗?两块多米诺骨牌上的陈述有何关联?每个陈述都是自然数的平方和,第二张多米诺骨牌上的陈述正好多了一项。因此,既然已经知道骨牌 $1$ 已经倒下,我们可以使用骨牌 $1$ 上写的\emph{真实陈述}来\emph{验证}骨牌 $2$ 上陈述的真实性:
\[\sum_{k=1}^{2}k^2 = 1^2+2^2=1+2^2=5=\frac{1}{6}(2)(3)(5)\]
现在,这可能看起来有点愚蠢,因为我们节省的唯一``工作''是不必``进行算术运算''来计算 $1^2 = 1$。让我们在数字较大的情况下使用此过程,以便充分说明这种做法的好处。\emph{假设}骨牌 $10$ 已经倒下。(如果你担心这个假设,我们在前面给出了完整的求和计算,你可以在那里验证。)这意味着我们\emph{知道}
\[\sum_{k=1}^{10}k^2 =\frac{1}{6}(10)(11)(21)=285\]
是一个\emph{真实陈述}。我们用它来验证骨牌 11 上写的陈述,即
\[\sum_{k=1}^{11}k^2 =\frac{1}{6}(11)(12)(23)\]
骨牌 $11$ 上的求和公式有 $11$ 项,前 $10$ 项正是骨牌 $10$ 上的求和!由于我们对该求和有所了解,因此我们只需将第 $11$ 项与其他求和项分开,并应用我们对其他项的已知信息:

\begin{align*}
    \sum_{k=1}^{11}k^2 &= (1^2+2^2+\dots+10^2)+11^2\\
    &=\sum_{k=1}^{10}k^2+11^2\\
    &=385+121\\
    &=506\\
    &=\frac{1}{6}3036=\frac{1}{6}(11)(12)(23)
\end{align*}
看看我们节省的工作!如果我们已经对求和的前 $10$ 项有所了解,为什么还要费力去计算它们呢?

现在,想象一下我们是否可以\emph{同时}对\emph{所有} $n$ 值执行此过程!也就是说,假设我们可以证明,只要多米诺骨牌 $n$ 倒下,我们就可以\emph{保证}多米诺骨牌 $(n + 1)$ 倒下。这会告诉我们什么?再思考一下无穷多的多米诺骨牌。我们知道骨牌 1 会倒下,因为我们手动检验了该值。然后,因为我们对所有 $n$ 值都验证了``骨牌 $n$ 撞倒骨牌 $(n + 1)$''的步骤,所以我们知道 骨牌 $1$ 会撞倒骨牌 $2$,骨牌 $2$ 又撞倒骨牌 $3$,骨牌 $3$ 有撞倒骨牌 $4 \dots$,一直持续下去,整行多米诺骨牌都会倒下!本质上,我们可以将整行多米诺骨牌分解为\emph{两}步:

\begin{enumerate}[label=(\arabic*)]
    \item 确保第一张多米诺骨牌倒下;
    \item 确保每张多米诺骨牌都能撞倒后面的多米诺骨牌。
\end{enumerate}
只需这两个步骤,我们就可以\emph{保证}每一张多米诺骨牌都会倒下,从而\emph{证明}上面写的所有事实都是真的。这将证明我们推导的公式对于\emph{每一个}自然数 $n$ 都有效。

我们已经完成了步骤(a),所以现在我们必须完成步骤(b)。我们已经在前面的段落中针对特定案例(骨牌 $1$ 推倒骨牌 $2$骨牌 $10$ 推倒骨牌 $11$)执行了此操作,因此让我们试着遵循这些案例的步骤将其推广到任意 $n$ 值。我们\emph{假设},对于某个\emph{特定}但\emph{任意}的 $n$ 值,多米诺骨牌 $n$ 会倒下,这告诉我们方程
\[\sum_{k=1}^{n}k^2=\frac{1}{6}n(n+1)(2n+1)\]
为\emph{真实陈述}。现在,我们想要将其与骨牌 $(n + 1)$ 上的陈述联系起来,并应用上面等式中给出的信息。让我们像之前一样,将 $n+1$ 项的和写为 $n$ 项之和加上最后一项:
\[\sum_{k=1}^{n+1}k^2 = (1^2+2^2+\dots+n^2+(n+1)^2)=\sum_{k=1}^{n}k^2+(n+1)^2\]
接下来,我们可以利用多米诺骨牌 $n$ 已经倒下的假设(这告诉我们骨牌上的事实为真)得到
\[\sum_{k=1}^{n+1}k^2 = \frac{1}{6}n(n+1)(2n+1)+(n+1)^2\]
这与骨牌 $(n+1)$ 上的事实一样吗?我们先来看看这个式子是什么,然后再进行比较。骨牌 $(n + 1)$ 上的``事实''与骨牌 $n$ 上的事实类似,只是将所有 ``$n$'' 的地方都替换为 ``$n + 1$'':
\[\sum_{k=1}^{n+1}k^2 = \frac{1}{6}\big(n+1\big)\big((n+1)+1\big)\big((2(n+1)+1)\big)=\frac{1}{6}(n+1)(n+1)(2n+3)\]
目前还不清楚我们推导出的表达式是否实际上等于上面的式子。我们可以尝试化简我们推导出的表达式,并将其分解为``看起来像''上面表达式的新表达式,但展开两个表达式并比较所有项可能会更容易。(这是出于这样的一般思想:展开因式分解后的多项式比进行因式分解要容易得多。)对于第一个表达式,我们得到

\begin{align*}
    \frac{1}{6}n(n + 1)(2n + 1) + (n + 1)^2 &=\frac{1}{6}n(2n^2 + 3n + 1) + (n^2 + 2n + 1)\\
    &= \frac{1}{3}n^3 + \frac{1}{2}n^2 + \frac{1}{6}n + n^2 + 2n + 1 \\
    &= \frac{1}{3}n^3 + \frac{3}{2}n^2 + \frac{13}{6}n + 1
\end{align*}
对于第二个表达式,我们得到
\begin{align*}
    \frac{1}{6}(n+1)(n + 2)(2n + 3) &=\frac{1}{6}(n+1)(2n^2 + 7n+6)\\
    &= \frac{1}{6}\big[(2n^3 + 7n^2 + 6n) + (2n^2 + 7n + 6)\big] \\
    &= \frac{1}{3}n^3 + \frac{3}{2}n^2 + \frac{13}{6}n + 1
\end{align*}
看呐,它们是相等的!此外,请注意,这比尝试整理其中一个表达式并将其``变形''为另一个表达式要容易得多。我们通过展开两式并最终找到相同的表达来证明它们是相同的。现在,让我们回顾并评估我们所取得的成果:

\begin{enumerate}
    \item 我们将证明公式
    \[\sum_{k=1}^{n+1}k^2 = \frac{1}{6}n(n+1)(2n+1)+(n+1)^2\]
    对于\emph{所有} $n$ 值的有效性类比为推倒无穷多的多米诺骨牌。
    \item 我们通过手工检验与该情况相对应的公式来验证多米诺骨牌 $1$ 会倒下。
    \item 我们通过\emph{假设}骨牌 $n$ 上的事实为真,并使用该信息来证明骨牌 $(n + 1)$ 上的事实也一定为真,从而证明了骨牌 $n$ 会倒下并撞倒骨牌 $(n+1)$。
    \item 这保证了所有多米诺骨牌都会倒下,因此该公式对于\emph{所有} $n$ 值都成立!
\end{enumerate}
这项技术有说服力吗?你认为我们已经\emph{严格证明}了该公式对于所有自然数 $n$ 都有效吗?如果有一个 $n$ 值使公式不成立怎么办?这对我们的多米诺骨牌体系意味着什么?

请记住,这里的多米诺骨牌类比只是展示归纳法如何工作的一个很好的直观指引,并不是建立在严格的数学基础上的。这将是接下来几章的目标。现在,让我们回顾一下本章中讨论的另一个示例:平面上的线。同样,在推导公式 $R(n)$ 时使用省略号很麻烦,我们希望避免使用省略号。让我们试着将多米诺骨牌类比应用于这道题。

想象一下,我们定义表达式 $R(n)$ 表示由 $n$ 条直线创建的平面中不同区域的数量,其中这些直线两两不平行,也没有三条或以上直线相交于一点。另外,想象一下,我们在骨牌 $n$ 上写下 ``$R(n) = 1 + \frac{n(n+1)}{2}$'' 这一``事实''。我们是否可以按照与上面相同的步骤来验证所有多米诺骨牌都会倒下?

首先,我们需要检验骨牌 $1$ 是否一定会倒下。这相当于验证以下陈述:``$R(1) = 1+\frac{1(2)}{2} = 1+1 = 2$'' 是否为真?这当然为真,我们之前验证过。一条线会将平面分为两个区域。其次,我们需要证明对于\emph{任意} $n$ 值,骨牌 $n$ 都会撞倒骨牌 $(n + 1)$。也就是说,我们\emph{假设} ``$R(n) = 1 + \frac{n(n+1)}{2}$'' 对于某个 $n$ 值是成立,然后\emph{证明} ``$R(n + 1) = 1 + \frac{(n+1)(n +2)}{2}$'' 也必然成立。我们应该怎么做?让我们继续沿用之前的论证方法,将 $R(n + 1)$ 与 $R(n)$ 联系起来。向\emph{任意}具有 $n$ 条直线的图形中添加一条新的直线(符合题目对于直线的要求),通过研究其几何影响,我们证明了 $R(n+ 1) = R(n) +n+ 1$。利用这些知识和我们对骨牌 $n$ 会倒下的假设,我们可以知道
\[R(n + 1) = R(n) + n + 1 = 1 +\frac{n(n+1)}{2}+ n + 1\]
这与骨牌 $(n + 1)$ 上的表达式一样吗?让我们化简这两个表达式来验证它们是否相同。可得
\[1 +\frac{n(n+1)}{2}+ n + 1=2+n+\frac{n^2+n}{2} = \frac{1}{2}n^2+\frac{3}{2}n+2\]
和
\[1 + \frac{(n+1)(n +2)}{2} = 1+\frac{n^2+3n+2}{2} =  \frac{1}{2}n^2+\frac{3}{2}n+2\]
瞧,它们是相同的!因此,我们证明了,对于\emph{任意} $n$ 值,骨牌 $n$ \emph{保证}撞倒骨牌 $(n+1)$。

想想我们用``多米诺骨牌技术''所做的事情与我们之前为推导出刚刚证明的表达式所做的事情之间的差异。我们在本节中用过省略号吗?为什么这种证明方式更好?我们曾经用过多米诺骨牌归纳技术推导过公式吗?
