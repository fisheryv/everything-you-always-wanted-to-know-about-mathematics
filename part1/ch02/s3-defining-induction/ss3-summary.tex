% !TeX root = ../../../book.tex
\subsection{总结}

让我们重新思考一下前面两个示例所完成的工作以及我们给出的类比。在我们对每个题目的初步思考中,我们发现题目存在某种\emph{结构},其中一个``事实''依赖于``前一个事实''。对于立方数,我们找到了一种用 $n^3$ 的\emph{项}来表示 $(n + 1)^3$ 的方法;就平面上的线而言,我们描述了当向具有 $n$ 条直线的图形中添加一条新的直线时,会添加多少个区域。根据这些观察,我们一遍又一遍地应用这些已知的知识,直到我们得到一个我们确认的``事实'',一般是较``小''的 $n$ 值(在这两题中,$n = 1$)。这让我们能够推导出适用于\emph{任意} $n$ 值的通用公式、方程或表达式。

这项工作对于推导这些表达式来说很有趣且至关重要,但还\emph{不足以证明}这些表达式的有效性。在进行上述工作时,我们发现了归纳过程的存在,并利用其结构推导了相关表达式。这实际上有两个好处:我们找到了要证明的表达式,并且通过认识题目的归纳行为,我们意识到通过\emph{数学归纳法}来证明表达式是严谨且有效的。

对于实际的``归纳证明'',我们遵循两个主要步骤。首先,我们确定了一个``起始值'',我们可以手动检验公式/方程。其次,我们\emph{假设} $n$ 的某个特定值使得相应的公式成立,然后使用这一知识来证明相应公式对 $n + 1$ 也必然成立。在这两个步骤之间,我们可以放心地说``所有多米诺骨牌都会倒下'',因此,这些公式对于 $n$ 的所有相关值都成立。

\subsubsection*{一个问题:梯子的``尽头''是什么?}

你可能还存有疑虑,我们在这里尝试预测一下你的担忧。(我们之所以提到这一点,是因为这是一个常见的观察结果。如果你\emph{没有}考虑到这一点,请试着想象一下这个想法来自哪里。)你可能会说,``嘿,现在我想我清楚 Mojo 是如何攀登天梯了,但他如何才能真正\emph{爬到顶端}呢?这是一个无穷阶梯,对吗?那他永远无法到达顶端……不是吗?''

某种程度上,你是对的。既然这个神奇的梯子会\emph{永远}持续下去,那么它就真的没有尽头,Mojo 永远不会到达``顶端''。然而,这不是重点;我们不关心梯子的任何``\emph{顶端}''(不仅仅是因为\emph{没有}顶端)。我们只需要知道 Mojo 实际上到达了\emph{每一个可能的}阶梯。他不必超越所有人,站在梯子的顶端,俯视自己的来路。那不是目标!

我们可以这样想这个问题:假设你对我们正在证明的某些特定事实保佑浓厚的兴趣。假设这个事实是事实 $\#18,458,789,572,311,000,574,003$。(某个巨大的数字。具体是多少无关紧要。)它对应的阶梯在梯子很远很远的地方,你关心的只是 Mojo 是否能在他的旅程中到达那里。他会到达吗?……你打赌他会!这可能需要很长时间(要走多少步呢?),但在这个猴子和梯子的神奇世界里,谁会在乎时间呢!你知道他最终会到达那里,这就是你想要的。现在,想象一下,对于每个事实,在那个神奇的世界里都有一个人只关心这个事实。当然,每个人都会高兴地知道 Mojo 将在他的旅程中达到他们关心的阶梯。没有人关心他是否能登上顶端;那不是人们关心的事。与此同时,在我们这个正常的、非魔法的世界里,我们对\emph{那个}世界上的每个人最终都会高兴这一事实感到非常高兴。整个无限攀爬梯子的过程被浓缩为两步,只需要这两步,我们就可以放心,梯子上的每一个台阶都会到达。每一个编号的事实都是真的。

也可以用多米诺骨牌的类比来思考这个问题。我们是否关心多米诺骨牌是否存在某个``终点'',倒在某处墙上?当然不关心; 这条多米诺骨牌链会永远持续下去。每一张多米诺骨牌最终都会倒下,我们甚至不在乎这需要多长时间。同样地,我们知道 Doug 会到达每个院子;我们不在乎他``何时''到达\emph{某个}院子,只关心他到达了\emph{所有}院子。
