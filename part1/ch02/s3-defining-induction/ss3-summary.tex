% !TeX root = ../../../book.tex
\subsection{总结}

回顾前两个示例的工作及我们的类比,可以发现每个问题都具有特定的\emph{结构}:某个``事实''依赖于``前一个事实''。对于立方数,我们找到了用 $n^3$ 表示 $(n + 1)^3$ 的方法;对于平面分割问题,我们刻画了向 $n$ 条直线的图形添加新直线时新增的区域个数。基于这些观察,我们反复应用已知关系,直至抵达一个可验证的``基础事实''——通常对应较小的 $n$ 值(两例中均为 $n = 1$)。这一过程使我们能够推导出适用于\emph{任意} $n$ 的通用公式或表达式。

尽管这项工作对公式推导至关重要且富有启发性,但它本身\emph{不足以证明}公式的有效性。在进行上述工作时,我们发现了归纳过程的存在,并利用其结构推导了相关表达式。这实际上有两个好处:不仅发现了待证公式,还让我们意识到采用\emph{数学归纳法}进行严格证明的可行性。

实际的``归纳证明''包含两个核心步骤:首先,验证公式在某个``起始值''成立;其次,\emph{假设}公式对某个特定 $n$ 成立,并以此证明其对 $n + 1$ 必然成立。完成这两步后,我们即可断言``所有多米诺骨牌都会倒下''——公式对所有相关 $n$ 值都成立。

\subsubsection*{一个问题:梯子的``尽头''是什么?}

你可能仍存疑虑,我们尝试在此预测你的担忧。(之所以提及这一点,是因为这是一个常见疑问。若你\emph{未曾}考虑这一点,请试着想象其来源。)你或许会说:``等等,现在我明白 Mojo 如何攀登天梯了,但他如何真正\emph{抵达顶端}呢?这是个无穷阶梯,对吗?那他永远无法到达终点……不是吗?''

某种意义上,你是对的。既然这个神奇阶梯将\emph{永远}延伸,它便没有真正的终点,Mojo 永无法抵达``顶端''。然而,这并非关键;我们不在意任何``\emph{顶端}''(不仅仅是因为\emph{不存在}顶端),只需确认 Mojo 能踏足\emph{每一个}台阶。他不必凌驾所有台阶立于顶端俯视来路——那不是目的!知道 Mojo 实际上到达了\emph{每一个可能的}阶梯。他不必超越所有人,站在梯子的顶端,俯视自己的来路。那不是目标!

不妨这样思考:假设你对某个待证事实抱有浓厚兴趣,例如
\[\text{事实\ } \#18,458,789,572,311,000,574,003 \text{\ (具体数值无关紧要)}\]
它对应遥不可及的台阶,而你只关心 Mojo 能否抵达。他会到达吗?他当然会!这或许需要漫长的时光(多少步呢?),但这在猴子与梯子的神奇世界,谁又在乎时间呢?你知道他终将抵达,这就够了。试想每个事实在神奇世界里都有专属关注者,每位关注着都将因 Mojo 踏足其关切之阶而欣喜。无人在意他能否登顶——那并非焦点。与此同时,在现实世界中,我们因\emph{所有}关注者终将如愿而欣慰。无限攀登的过程被简化为两步:仅凭此两步,我们便确信阶梯的\emph{每一级}皆可达,每个编号的事实皆成立。

亦可类比多米诺骨牌:我们是否在意骨牌链存在``终点'',最终撞上墙壁?当然不。骨牌链将永续延伸,每张牌终会倒下,时间长短无关紧要。同理,我们知晓 Doug 终将抵达\emph{所有}院子——何时抵达\emph{某个}院子无关紧要,唯有抵达\emph{全部}院子方为关键。
