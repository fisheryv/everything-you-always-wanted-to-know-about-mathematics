% !TeX root = ../../../book.tex
\section{定义归纳}

为了将数学归纳法定义为证明技术,我们想强调,上一节中的例子使用了问题结构的某些直观概念来给出问题的``解'',我们在\emph{解}上加了引号,表明我们还没有正式证明它。从这个意义上讲,我们提出以下问题:如果\emph{给定}我们前面推导的公式并要求验证它怎么办?如果我们没有通过任何直观的步骤来推导公式,只是有人告诉我们它是正确的怎么办?我们如何验证他们的说法?之所以问这个问题,是因为我们现在确实面临着这种情况,除非告诉我们公式的人采用与我们相同的直觉论证。

假如一个持怀疑态度的朋友说:``嘿,我听说过一个计算前 $n$ 个自然数平方和的公式。有人告诉我,它们加起来等于 $\frac{1}{6}n(n+1)(2n+1)$。我验证了前两个自然数,全都正确,所以它一定是正确的。应该传播出去!'' 作为一个理性思考者,同时也是好朋友的身份,你点点头说:``我确实听说了,但让我们确保这个公式对每个数字都是正确的。'' 你将如何进行?你的朋友说的没错,前几个值确实``完美匹配'':

\begin{align*}
    1^2 &= \enspace 1 = \frac{1}{6}(1)(2)(3) \\
    1^2 + 2^2 &= \enspace 5 = \frac{1}{6}(2)(3)(5) \\
    1^2 + 2^2 + 3^2 &= 14 = \frac{1}{6}(3)(4)(7) \\
    1^2 + 2^2 + 3^2 + 4^2 &= 30 = \frac{1}{6}(4)(5)(9)
\end{align*}
如果我们愿意的话,我们甚至可以手动检验 $n$ 的更大的值:
\[1^2 + 2^2 + 3^2 + 4^2 + 5^2 + 6^2 + 7^2 + 8^2 + 9^2 + 10^2 = 385 = \frac{1}{6}(10)(11)(21)\]
但请记住,该公式声称对于\emph{任意} $n$ 值都有效。手动检验每个结果将花费大量时间,因为自然数有\emph{无穷}多个。无论我们检验多少个独立的 $n$ 值,总会有更大的值,我们怎么\emph{知道}公式对于某些大值不会失效?从数学和时间上来讲,我们需要一个更加\emph{有效}的方法,以某种方式只需几步即可验证所有 $n$ 值。我们先在心里埋下一颗种子(这是即将推出的数学归纳法的严格版本),在这里我们将从广义上解释该过程是如何工作的。

% !TeX root = ../../../book.tex
\subsection{多米诺骨牌类比}

假设我们有一副特殊的多米诺骨牌,它包含无穷多张骨牌!每张骨牌可以书写任意内容,而非标准的点数。这些骨牌沿无限延伸的桌面排列成无限长的一行。从侧面观察时,每张骨牌下方标有位置标签:

\begin{center}
    \begin{tikzpicture}
        \foreach \x in {1,...,5}
        {
            \pic [fill=white] at (\x, 0, 0) {annotated cuboid={width=3, height=30, depth=10}};
            \node[below] at (\x, -3){\tiny $n=\x$};
        }
        \node[anchor=center] at (6.5, -1.5){\LARGE $\dots \cdot$};
        % \pic [very thick,densely dashed,draw=blue] at (5,0) {annotated cuboid={width=30, height=5, depth=10, opacity=0.2}};
    \end{tikzpicture}
\end{center}

对于这个特定的例子,我们需要验证公式
\[\sum_{k=1}^{n}k^2 = \frac{1}{6}n(n+1)(2n+1)\]
为此,设想每张多米诺骨牌记载一个特定``事实''。具体来说,我们可以想象第一张多米诺骨牌上写有表达式
\[\sum_{k=1}^{1}k^2 = \frac{1}{6}(1)(1)(3)\]
第二张多米诺骨牌上写有表达式
\[\sum_{k=1}^{2}k^2 = \frac{1}{6}(2)(3)(5)\]
推而广之,第 $n$ 张骨牌写有如下``事实'':
\[\sum_{k=1}^{n}k^2 = \frac{1}{6}n(n+1)(2n+1)\]
由于多米诺骨牌具有连锁倾倒的特性,我们约定骨牌倒下即表示其记载的``事实''是\emph{真实命题}。由此将多米诺骨牌的物理解释与公式有效性的数学解释联系起来。

我们已手动验证 $n=1$ 的情形:$1^2=\frac{1}{6}(1)(2)(3)$,故第一张骨牌记载的命题为真,必将倒下。同理验证 $n=2$ 后,第二张骨牌也会倒下:

\begin{center}
    \begin{tikzpicture}
        \foreach \x in {1,2}
        {
            \pic [fill=white, rotate=-30, anchor=south] at (\x, -0.15, 0) {annotated cuboid={width=3, height=32, depth=10}};
            \node[below] at (\x-1.5, -3){\tiny $n=\x$};
        }
        \foreach \x in {3,4,5}
        {
            \pic [fill=white] at (\x, 0, 0) {annotated cuboid={width=3, height=30, depth=10}};
            \node[below] at (\x, -3){\tiny $n=\x$};
        }
        \node[anchor=center] at (6.5, -1.5){\LARGE $\dots \cdot$};
        % \pic [very thick,densely dashed,draw=blue] at (5,0) {annotated cuboid={width=30, height=5, depth=10, opacity=0.2}};
    \end{tikzpicture}
\end{center}
然而,若继续逐个检验,又会陷入原先的困境——我们不可能验证\emph{每张}骨牌。真正的需求是捕捉多米诺效应的精髓:一张骨牌倒下将触发下一张倒下。这要求我们建立相邻骨牌所载``事实''的数学关联。

让我们看看前两张多米诺骨牌的情况。既然知道骨牌 $1$ 倒下,我们能否在不重写所有求和项的情况下确保骨牌 $2$ 倒下?两块骨牌上的陈述有何关联?每个陈述都是自然数的平方和,且第二张骨牌的陈述正好多出一项。因此,利用骨牌 $1$ 上已知的\emph{真实陈述},可以\emph{验证}骨牌 $2$ 上陈述的真实性:
\[\sum_{k=1}^{2}k^2 = 1^2+2^2=1+2^2=5=\frac{1}{6}(2)(3)(5)\]
尽管节省的唯一``工作''只是免于计算 $1^2=1$,但让我们在更大数字上应用此过程以凸显其优势。\emph{假设}骨牌 $10$ 已经倒下(其求和的完整验证已在前文给出),这意味着我们\emph{知道}
\[\sum_{k=1}^{10}k^2 =\frac{1}{6}(10)(11)(21)=285\]
是一个\emph{真实陈述}。利用它来验证骨牌 $11$ 的陈述:
\[\sum_{k=1}^{11}k^2 =\frac{1}{6}(11)(12)(23)\]
骨牌 $11$ 上的求和公式有 $11$ 项,前 $10$ 项正是骨牌 $10$ 上的求和!因此只需分离第 $11$ 项并代入已知结果:
\begin{align*}
    \sum_{k=1}^{11}k^2 &= (1^2+2^2+\dots+10^2)+11^2\\
    &=\sum_{k=1}^{10}k^2+11^2\\
    &=385+121\\
    &=506\\
    &=\frac{1}{6}3036=\frac{1}{6}(11)(12)(23)
\end{align*}
节省的工作量显而易见!既然已知前 $10$ 项之和,何必重新计算?

现在设想对\emph{所有} $n$ 值\emph{同时}实施此过程!若能证明每当骨牌 $n$ 倒下,骨牌 $(n + 1)$ \emph{必然}倒下,这意味着什么?回顾无穷骨牌序列:已知骨牌 $1$ 因手动检验而倒下,加之``骨牌 $n$ 撞倒骨牌 $(n + 1)$''的普适性验证,可推得骨牌 $1$ 撞倒骨牌 $2$,骨牌 $2$ 撞倒骨牌 $3$,骨牌 $3$ 撞倒骨牌 $4$,…… 如此传递下去,整列骨牌终将全部倒下!本质上,整个过程可归结为\emph{两步}:

\begin{enumerate}[label=(\arabic*)]
    \item 确保第一张骨牌倒下;
    \item 确保每张骨牌都能撞倒下一张骨牌。
\end{enumerate}
仅凭这两步,便能\emph{保证}所有骨牌倒下,从而\emph{证明}每个公式对\emph{任意}自然数 $n$ 成立。

我们已经完成步骤 (a),现在需要完成步骤 (b)。此前已针对特定案例(骨牌 $1$ 撞倒骨牌 $2$、骨牌 $10$ 撞倒骨牌 $11$)执行了此操作,现在将其推广到任意 $n$ 值。我们\emph{假设}:对于某个\emph{特定}但\emph{任意}的 $n$,多米诺骨牌 $n$ 会倒下,这意味着方程
\[\sum_{k=1}^{n}k^2=\frac{1}{6}n(n+1)(2n+1)\]
为\emph{真实陈述}。现在需将其关联到骨牌 $(n+1)$ 的陈述,并应用上述等式信息。将 $n+1$ 项的和拆分为 $n$ 项和与末项:
\[\sum_{k=1}^{n+1}k^2 = (1^2+2^2+\dots+n^2+(n+1)^2)=\sum_{k=1}^{n}k^2+(n+1)^2\]
根据骨牌 $n$ 倒下的假设(即其命题为真),可得
\[\sum_{k=1}^{n+1}k^2 = \frac{1}{6}n(n+1)(2n+1)+(n+1)^2\]
这与骨牌 $(n+1)$ 的命题是否一致?骨牌 $(n+1)$ 的``事实''与骨牌 $n$ 类似,只是将``$n$''替换为``$n + 1$'':
\[\sum_{k=1}^{n+1}k^2 = \frac{1}{6}\big(n+1\big)\big((n+1)+1\big)\big((2(n+1)+1)\big)=\frac{1}{6}(n+1)(n+1)(2n+3)\]
目前还不清楚我们推导出的表达式是否实际上等于上面的式子。我们可以尝试化简该表达式,并将其分解为与上面表达式``类似''的新表达式,但展开两个表达式并比较所有项可能会更容易。(这基于这样的一般思想:展开因式分解后的多项式比进行因式分解要容易得多。)对于第一个表达式,我们有
\begin{align*}
    \frac{1}{6}n(n + 1)(2n + 1) + (n + 1)^2 &=\frac{1}{6}n(2n^2 + 3n + 1) + (n^2 + 2n + 1)\\
    &= \frac{1}{3}n^3 + \frac{1}{2}n^2 + \frac{1}{6}n + n^2 + 2n + 1 \\
    &= \frac{1}{3}n^3 + \frac{3}{2}n^2 + \frac{13}{6}n + 1
\end{align*}
对于第二个表达式,我们有
\begin{align*}
    \frac{1}{6}(n+1)(n + 2)(2n + 3) &=\frac{1}{6}(n+1)(2n^2 + 7n+6)\\
    &= \frac{1}{6}\big[(2n^3 + 7n^2 + 6n) + (2n^2 + 7n + 6)\big] \\
    &= \frac{1}{3}n^3 + \frac{3}{2}n^2 + \frac{13}{6}n + 1
\end{align*}
可见两式相等!此外,请注意,这比尝试整理其中一个表达式并将其``变形''为另一个表达式要容易得多。我们通过展开两式并最终得到相同的表达来证明它们是相同的。现在,让我们回顾并总结我们所取得的成果:
\begin{enumerate}
    \item 我们将证明公式
    \[\sum_{k=1}^{n+1}k^2 = \frac{1}{6}n(n+1)(2n+1)+(n+1)^2\]
    对于\emph{所有} $n$ 值成立类比为推倒无穷多的多米诺骨牌。
    \item 通过手工计算验证骨牌 $1$ 的命题成立,骨牌 $1$ 倒下;
    \item \emph{假设}骨牌 $n$ 的命题为真,由骨牌 $n$ 命题成立推出骨牌 $(n+1)$ 命题成立,从而证明骨牌 $n$ 会撞倒骨牌 $(n+1)$。
    \item 由此保证所有骨牌都会倒下,因此公式对\emph{所有} $n$ 都成立。
\end{enumerate}
此方法是否严谨?是否已\emph{严格证明}公式对所有自然数 $n$ 都成立?若存在 $n$ 使得公式失效,这对多米诺骨牌体系意味着什么?

请记住,这里的多米诺骨牌类比只是理解归纳法工作原理的一个直观指引,并非建立在严格的数学基础之上。建立严格的数学基础将是接下来几章的目标。现在,让我们回顾本章讨论的另一个例子:直线划分平面区域。同样,在推导公式 $R(n)$ 时使用省略号显得繁琐,我们希望避免这种做法。让我们尝试将多米诺骨牌类比应用于此问题。

设想我们定义 $R(n)$ 为 $n$ 条直线在平面上划分出的不同区域的数量,这些直线满足互不平行且任意三条(或更多)直线不共点。进一步设想,我们在代表第 $n$ 步的骨牌上写下``$R(n) = 1 + \frac{n(n+1)}{2}$''这一``事实''。能否按照与之前相同的逻辑来验证所有骨牌都会倒下?

首先,需要验证骨牌 $1$ 是否会倒下。这等同于验证命题``$R(1) = 1+\frac{1(2)}{2} = 1+1 = 2$''是否成立。这显然成立,正如我们之前验证过的:一条直线将平面划分为两个区域。其次,需要证明对于\emph{任意} $n$,第 $n$ 块骨牌倒下必定导致第 $(n + 1)$ 块骨牌倒下。也就是说,我们\emph{假设}``$R(n) = 1 + \frac{n(n+1)}{2}$''对某个特定的 $n$ 成立,然后\emph{证明}``$R(n + 1) = 1 + \frac{(n+1)(n +2)}{2}$''也必然成立。如何证明?沿用之前的思路,建立 $R(n + 1)$ 与 $R(n)$ 的关系。向\emph{任意}满足条件的 $n$ 条直线的图形中添加一条新直线,通过几何分析,我们得到关系式 $R(n+1) = R(n) + n + 1$。利用此关系以及第 $n$ 块骨牌倒下的假设,可得:
\[R(n + 1) = R(n) + n + 1 = 1 +\frac{n(n+1)}{2}+ n + 1\]
这个结果是否与骨牌 $(n + 1)$ 上的表达式一致?通过化简比较即可验证:
\[1 +\frac{n(n+1)}{2}+ n + 1=2+n+\frac{n^2+n}{2} = \frac{1}{2}n^2+\frac{3}{2}n+2\]
以及
\[1 + \frac{(n+1)(n +2)}{2} = 1+\frac{n^2+3n+2}{2} =  \frac{1}{2}n^2+\frac{3}{2}n+2\]
结果完全相同!因此,我们证明了对于\emph{任意} $n$,骨牌 $n$ 倒下\emph{必然}导致骨牌 $(n+1)$ 倒下。

思考一下,使用这种``多米诺骨牌技术''进行的证明,与我们之前为推导该公式所采用的方法有何不同?我们在本节中是否使用了省略号?为何这种证明方式更优?我们是否曾用多米诺骨牌归纳技术来推导公式本身?


% !TeX root = ../../../book.tex
\subsection{其他类比}

多米诺骨牌类比非常流行,但它并不是归纳法工作方式的唯一描述。根据你的阅读内容或交谈对象,可能会学到不同的类比,或其他类型的描述。这里,我们将描述以前听说过的两个。思考这些类比本质上的等价性,这将有助于巩固你对归纳法的理解(至少就我们所开发的而言)。

\subsubsection*{神奇的数学猴子 Mojo}

想象一个无穷天梯,直矗云霄。梯子有无数级,按 $1, 2, 3$ 的顺序依次编号。我们的朋友 Mojo 恰好站在梯子旁。他是一只聪明的猴子,对数学很感兴趣,但也有点神奇,因为他真的可以爬上这个无穷天梯!

如果 Mojo 到达了阶梯上的某一级,则意味着与该数字对应的事实为真。我们怎样才能确保他爬完整个梯子?单独检查每个阶梯的效率很低。想象一下:我们必须站在地面上确保他到达第 $1$ 级,然后我们必须稍微抬起头来确保他到达了第 $2$ 级,然后是第 $3$ 级,依此类推……相反,我们在 Mojo 开始攀爬之前确认了两个细节。他要开始攀爬了吗?也就是说,他会爬上第 $1$ 级吗?如果是这样,那就太好了!另外,阶梯之间的距离是否足够近,以便无论他在哪里,\emph{总能}到达下一个阶梯?如果是这样,那就更棒了!这些与我们在多米诺骨牌类比中建立的条件完全相同。为了确保 Mojo 到达\emph{每个}阶梯,我们只需要知道他到达了第 $1$ 个阶梯,并且他总是可以到达下一个阶梯。

\subsubsection*{归纳鸭 Doug}

再来认识一下 Doug。他是一只鸭子。他喜欢面包,所以他会去每个人的院子里寻找更多的面包。这些院子都沿数学镇的归纳街而建,房子的编号是 $1, 2, 3, \dots$ 以此类推。

Doug 从 $1$ 号院子开始寻找面包。没有找到任何东西,所以他依旧很饿。还能去哪里找?隔壁还有 $2$ 号院子!Doug 朝那边走去,肚子咕咕叫。他在那里也没找到面包,所以他必须继续寻找。此时他已经知道 $1$ 号院子没有面包,所以唯一去向就是隔壁的 $3$ 号院子。我想你已经明白事情的发展方向了…… 

如果我们跟踪 Doug 的进展,我们可能想知道他最终是否到达了每一个院子。假设我们已经提前知道\emph{没人}有面包。这意味着,每当 Doug 在某个院子里时,他一定会去隔壁院子,继续寻找食物。这意味着他一定会挨家挨户地去寻找!也就是说,无论我们住在哪栋房子里,无论我们门前的数字多大,在某个时点我们一定会看到 Doug 在我们的后院闲逛。(不幸的是,他会一直饿着肚子!可怜的 Doug。)


% !TeX root = ../../../book.tex
\subsection{总结}

回顾前两个示例的工作及我们的类比,可以发现每个问题都具有特定的\emph{结构}:某个``事实''依赖于``前一个事实''。对于立方数,我们找到了用 $n^3$ 表示 $(n + 1)^3$ 的方法;对于平面分割问题,我们刻画了向 $n$ 条直线的图形添加新直线时新增的区域个数。基于这些观察,我们反复应用已知关系,直至抵达一个可验证的``基础事实''——通常对应较小的 $n$ 值(两例中均为 $n = 1$)。这一过程使我们能够推导出适用于\emph{任意} $n$ 的通用公式或表达式。

尽管这项工作对公式推导至关重要且富有启发性,但它本身\emph{不足以证明}公式的有效性。在进行上述工作时,我们发现了归纳过程的存在,并利用其结构推导了相关表达式。这实际上有两个好处:不仅发现了待证公式,还让我们意识到采用\emph{数学归纳法}进行严格证明的可行性。

实际的``归纳证明''包含两个核心步骤:首先,验证公式在某个``起始值''成立;其次,\emph{假设}公式对某个特定 $n$ 成立,并以此证明其对 $n + 1$ 必然成立。完成这两步后,我们即可断言``所有多米诺骨牌都会倒下''——公式对所有相关 $n$ 值都成立。

\subsubsection*{一个问题:梯子的``尽头''是什么?}

你可能仍存疑虑,我们尝试在此预测你的担忧。(之所以提及这一点,是因为这是一个常见疑问。若你\emph{未曾}考虑这一点,请试着想象其来源。)你或许会说:``等等,现在我明白 Mojo 如何攀登天梯了,但他如何真正\emph{抵达顶端}呢?这是个无穷阶梯,对吗?那他永远无法到达终点……不是吗?''

某种意义上,你是对的。既然这个神奇阶梯将\emph{永远}延伸,它便没有真正的终点,Mojo 永无法抵达``顶端''。然而,这并非关键;我们不在意任何``\emph{顶端}''(不仅仅是因为\emph{不存在}顶端),只需确认 Mojo 能踏足\emph{每一个}台阶。他不必凌驾所有台阶立于顶端俯视来路——那不是目的!知道 Mojo 实际上到达了\emph{每一个可能的}阶梯。他不必超越所有人,站在梯子的顶端,俯视自己的来路。那不是目标!

不妨这样思考:假设你对某个待证事实抱有浓厚兴趣,例如
\[\text{事实\ } \#18,458,789,572,311,000,574,003 \text{\ (具体数值无关紧要)}\]
它对应遥不可及的台阶,而你只关心 Mojo 能否抵达。他会到达吗?他当然会!这或许需要漫长的时光(多少步呢?),但这在猴子与梯子的神奇世界,谁又在乎时间呢?你知道他终将抵达,这就够了。试想每个事实在神奇世界里都有专属关注者,每位关注着都将因 Mojo 踏足其关切之阶而欣喜。无人在意他能否登顶——那并非焦点。与此同时,在现实世界中,我们因\emph{所有}关注者终将如愿而欣慰。无限攀登的过程被简化为两步:仅凭此两步,我们便确信阶梯的\emph{每一级}皆可达,每个编号的事实皆成立。

亦可类比多米诺骨牌:我们是否在意骨牌链存在``终点'',最终撞上墙壁?当然不。骨牌链将永续延伸,每张牌终会倒下,时间长短无关紧要。同理,我们知晓 Doug 终将抵达\emph{所有}院子——何时抵达\emph{某个}院子无关紧要,唯有抵达\emph{全部}院子方为关键。


% !TeX root = ../../../book.tex
\subsection{习题}\label{sec:section2.3.4}

\subsubsection*{温故知新}

以口头或书面的形式简要回答以下问题。这些问题全都基于你刚刚阅读的内容,如果忘记了具体定义、概念或示例,可以回顾相关内容。确保在继续学习之前能够自信地作答这些问题,这将有助于你的理解和记忆!

\begin{enumerate}[label=(\arabic*)]
    \item 多米诺骨牌、Mojo 和 Doug 的类比为何等价?能否定义一个``函数''来描述它们之间的关系,实现类比间的相互转换?
    \item 找一位未接触过数学归纳法的朋友,尝试向他解释这一概念。你是否在解释中使用了上述类比?这些类比是否有帮助?
    \item 为什么我们对立方体的研究不足以证明求和公式?为何仍需完成后续推导?
    \item 思考多米诺骨牌的类比:若骨牌无限延伸是否会导致某些骨牌永不倒下?尝试用类比解释这一现象的含义。
\end{enumerate}

\subsubsection*{小试牛刀}

尝试解答以下问题。这些题目需动笔书写或口头阐述答案,旨在帮助你熟练运用新概念、定义及符号。题目难度适中,确保掌握它们将大有裨益!

\begin{enumerate}[label=(\arabic*)]
    \item 用数学归纳法证明公式:
    \[\sum_{k=1}^{n}k = \frac{n(n+1)}{2}\]
    \item 用数学归纳法证明公式:
    \[\sum_{k=1}^{n}2k-1 = n^2\]
    \item 用数学归纳法证明公式:
    \[\sum_{k=1}^{n}k^3 = \Bigg(\frac{n(n+1)}{2}\Bigg)^2\]
    \item 设存在一系列由自然数索引的命题,用``$P(n)$''表示第 $n$ 个命题。
    \begin{enumerate}[label=(\alph*)]
        \item 若要证明对所有自然数 $n$,\emph{每个} $P(n)$ 均成立,应该如何操作?
        \item 若只需证明当 $n$ 为\emph{偶数}时 $P(n)$ 成立,应该如何处理?能否通过修改某个类比来描述此方法?
        \item 若只需证明当 $n \geq 4$ 时 $P(n)$ 成立,又该如何处理?能否通过修改某个类比来描述此方法?
    \end{enumerate}
\end{enumerate}
