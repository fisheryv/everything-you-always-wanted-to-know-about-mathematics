% !TeX root = ../../../book.tex
\subsection{其他类比}

多米诺骨牌类比非常流行,但它并不是归纳法工作方式的唯一描述。根据你的阅读内容或交谈对象,可能会学到不同的类比,或其他类型的描述。这里,我们将描述以前听说过的两个。思考这些类比本质上的等价性,这将有助于巩固你对归纳法的理解(至少就我们所开发的而言)。

\subsubsection*{神奇的数学猴子 Mojo}

想象一个无穷天梯,直矗云霄。梯子有无数级,按 $1, 2, 3$ 的顺序依次编号。我们的朋友 Mojo 恰好站在梯子旁。他是一只聪明的猴子,对数学很感兴趣,但也有点神奇,因为他真的可以爬上这个无穷天梯!

如果 Mojo 到达了阶梯上的某一级,则意味着与该数字对应的事实为真。我们怎样才能确保他爬完整个梯子?单独检查每个阶梯的效率很低。想象一下:我们必须站在地面上确保他到达第 $1$ 级,然后我们必须稍微抬起头来确保他到达了第 $2$ 级,然后是第 $3$ 级,依此类推……相反,我们在 Mojo 开始攀爬之前确认了两个细节。他要开始攀爬了吗?也就是说,他会爬上第 $1$ 级吗?如果是这样,那就太好了!另外,阶梯之间的距离是否足够近,以便无论他在哪里,\emph{总能}到达下一个阶梯?如果是这样,那就更棒了!这些与我们在多米诺骨牌类比中建立的条件完全相同。为了确保 Mojo 到达\emph{每个}阶梯,我们只需要知道他到达了第 $1$ 个阶梯,并且他总是可以到达下一个阶梯。

\subsubsection*{归纳鸭 Doug}

再来认识一下 Doug。他是一只鸭子。他喜欢面包,所以他会去每个人的院子里寻找更多的面包。这些院子都沿数学镇的归纳街而建,房子的编号是 $1, 2, 3, \dots$ 以此类推。

Doug 从 $1$ 号院子开始寻找面包。没有找到任何东西,所以他依旧很饿。还能去哪里找?隔壁还有 $2$ 号院子!Doug 朝那边走去,肚子咕咕叫。他在那里也没找到面包,所以他必须继续寻找。此时他已经知道 $1$ 号院子没有面包,所以唯一去向就是隔壁的 $3$ 号院子。我想你已经明白事情的发展方向了…… 

如果我们跟踪 Doug 的进展,我们可能想知道他最终是否到达了每一个院子。假设我们已经提前知道\emph{没人}有面包。这意味着,每当 Doug 在某个院子里时,他一定会去隔壁院子,继续寻找食物。这意味着他一定会挨家挨户地去寻找!也就是说,无论我们住在哪栋房子里,无论我们门前的数字多大,在某个时点我们一定会看到 Doug 在我们的后院闲逛。(不幸的是,他会一直饿着肚子!可怜的 Doug。)
