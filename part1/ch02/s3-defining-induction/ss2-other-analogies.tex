% !TeX root = ../../../book.tex
\subsection{其他类比}

多米诺骨牌类比非常流行,但它并非描述归纳法工作方式的唯一途径。根据阅读材料或交流对象的不同,你可能会接触到不同的类比或其他形式的阐释。在此,我们将介绍两种常见的类比。思考这些类比在本质上的相通之处,将有助于深化你对归纳法的理解(至少基于我们已有的探讨)。

\subsubsection*{神奇的数学猴子 Mojo}

想象一架直冲云霄的无穷天梯,梯级按 $1, 2, 3$ 的顺序无限延伸。我们的朋友 Mojo 伫立在梯子旁。这只聪慧的猴子痴迷数学,更拥有神奇的能力——他能真正攀爬这架无穷天梯!

Mojo 踏上某一级阶梯,即代表与该数字对应的事实成立。如何确保他爬完整架梯子?逐一检查每级阶梯效率低下:我们需要先确认他抵达第 $1$ 级,再验证他到达第 $2$ 级,接着是第 $3$ 级……如此往复。更高效的方法是在 Mojo 攀爬前确认两点:第一,他是否踏上起点(即第 $1$ 级)?若是,那就太好了!第二,阶梯间距是否足够近,使他无论身处何处,\emph{总能}登上下一级?若满足此条件,那就更棒了!这些要求与多米诺骨牌类比中的条件完全一致。要确保 Mojo 抵达\emph{每一级}阶梯,只需确认他踏上第 $1$ 级并能持续迈向下一级。

\subsubsection*{归纳鸭 Doug}

再来认识一下 Doug ——一只酷爱面包的鸭子。为了觅食,他会造访数学镇归纳街上每户人家的院子。这些院子沿街排列,门牌号依次为 $1, 2, 3, \dots$。

Doug 从 $1$ 号院子开始搜寻面包。一无所获的他饥肠辘辘,便转向隔壁的 $2$ 号院子。再次空手而归的他只能继续前行。此时已知 $1$ 号院无面包,因此唯一的选择只能是隔壁的 $3$ 号院……我想你已预见到了结局。

如果我们追踪 Doug 的行迹,我们或许好奇他是否终将踏足每个院子。假设已知\emph{所有院子均无面包},这意味着每当 Doug 身处某院,必将前往隔壁院子继续寻找。由此,他必定会挨家挨户探索!换言之,无论你住在编号多大的房子,终将在某一刻目睹 Doug 在你的后院徘徊(遗憾的是,他永远腹中空空——可怜的 Doug!)。
