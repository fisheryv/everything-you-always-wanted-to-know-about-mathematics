% !TeX root = ../../../book.tex
\subsection{习题}\label{sec:section2.3.4}

\subsubsection*{温故知新}

以口头或书面的形式简要回答以下问题。这些问题全都基于你刚刚阅读的内容,所以如果忘记了具体的定义、概念或示例,可以回去重读相关部分。确保在继续学习之前能够自信地回答这些问题,这将有助于你的理解和记忆!

\begin{enumerate}[label=(\arabic*)]
    \item 多米诺骨牌、Mojo 和 Doug 类比是如何等价的?你能给出``函数''来描述它们的关系,将一种类比转换成另一种类比吗?
    \item 找一个没学过数学归纳法的朋友,试着向他描述一下数学归纳法。你发现自己使用了其中的类比吗?有帮助吗?
    \item 为什么我们对立方体的研究未能证明求和公式?为什么我们还需要完成所有这些工作?
    \item 想想多米诺骨牌的类比。多米诺骨牌永远持续下去是一个问题吗?这是否意味着有些多米诺骨牌永远不会倒下?尝试用类比来描述这意味着什么。
\end{enumerate}

\subsubsection*{小试牛刀}

尝试回答以下问题。这些题目要求你实际动笔写下答案,或(对朋友/同学)口头陈述答案。目的是帮助你练习使用新的概念、定义和符号。题目都比较简单,确保能够解决这些问题将对你大有帮助!

\begin{enumerate}[label=(\arabic*)]
    \item 通过归纳步骤来证明该公式
    \[\sum_{k=1}^{n}k = \frac{n(n+1)}{2}\]
    \item 通过归纳步骤来证明该公式
    \[\sum_{k=1}^{n}2k-1 = n^2\]
    \item 通过归纳步骤来证明该公式
    \[\sum_{k=1}^{n}k^3 = \Bigg(\frac{n(n+1)}{2}\Bigg)^2\]
    \item 假设我们有一系列由自然数索引的事实。我们使用表达式 ``$P(n)$'' 表示第 $n$ 个事实。
    \begin{enumerate}[label=(\alph*)]
        \item 如果我们想证明对于每个自然数 $n$,\emph{每个}事实都为真,我们应该怎么做呢?
        \item 如果我们想证明只有 $n$ 为\emph{偶数}时对应的陈述才为真,那该怎么办?我们能做到吗?你能用我们给出的一个类比稍作修改来描述你的方法吗?
        \item 如果我们想证明只有 $n$ 大于等于 $4$ 时才对应的陈述才为真,那该怎么办?我们能做到吗?你能用我们给出的一个类比稍作修改来描述你的方法吗?
    \end{enumerate}
\end{enumerate}
