% !TeX root = ../../../book.tex
\subsection{习题}\label{sec:section2.3.4}

\subsubsection*{温故知新}

以口头或书面的形式简要回答以下问题。这些问题全都基于你刚刚阅读的内容,如果忘记了具体定义、概念或示例,可以回顾相关内容。确保在继续学习之前能够自信地作答这些问题,这将有助于你的理解和记忆!

\begin{enumerate}[label=(\arabic*)]
    \item 多米诺骨牌、Mojo 和 Doug 的类比为何等价?能否定义一个``函数''来描述它们之间的关系,实现类比间的相互转换?
    \item 找一位未接触过数学归纳法的朋友,尝试向他解释这一概念。你是否在解释中使用了上述类比?这些类比是否有帮助?
    \item 为什么我们对立方体的研究不足以证明求和公式?为何仍需完成后续推导?
    \item 思考多米诺骨牌的类比:若骨牌无限延伸是否会导致某些骨牌永不倒下?尝试用类比解释这一现象的含义。
\end{enumerate}

\subsubsection*{小试牛刀}

尝试解答以下问题。这些题目需动笔书写或口头阐述答案,旨在帮助你熟练运用新概念、定义及符号。题目难度适中,确保掌握它们将大有裨益!

\begin{enumerate}[label=(\arabic*)]
    \item 用数学归纳法证明公式:
    \[\sum_{k=1}^{n}k = \frac{n(n+1)}{2}\]
    \item 用数学归纳法证明公式:
    \[\sum_{k=1}^{n}2k-1 = n^2\]
    \item 用数学归纳法证明公式:
    \[\sum_{k=1}^{n}k^3 = \Bigg(\frac{n(n+1)}{2}\Bigg)^2\]
    \item 设存在一系列由自然数索引的命题,用``$P(n)$''表示第 $n$ 个命题。
    \begin{enumerate}[label=(\alph*)]
        \item 若要证明对所有自然数 $n$,\emph{每个} $P(n)$ 均成立,应该如何操作?
        \item 若只需证明当 $n$ 为\emph{偶数}时 $P(n)$ 成立,应该如何处理?能否通过修改某个类比来描述此方法?
        \item 若只需证明当 $n \geq 4$ 时 $P(n)$ 成立,又该如何处理?能否通过修改某个类比来描述此方法?
    \end{enumerate}
\end{enumerate}
