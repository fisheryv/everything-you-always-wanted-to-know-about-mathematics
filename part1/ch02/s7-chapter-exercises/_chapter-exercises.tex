% !TeX root = ../../../book.tex
\section{本章习题}

以下问题,有助于你熟悉归纳证明。我们暂不要求完全严格的证明,只需清晰地描述过程并写出步骤即可。待掌握数学归纳原理 (PMI) 及相应证明策略后,可重新严格证明这些问题。

\begin{exercise} \label{exc:exercises2.7.1}
    证明以下求和公式对所有自然数 $n$(包括 $n=0$)成立:
    \[\sum_{i=0}^{n}2^i=2^{n+1}-1\]
    后续问题:利用此结果说明在 $2^n$ 支球队的单场淘汰赛中,需进行多少场比赛才能决出冠军?(例如 NCAA 疯狂三月锦标赛就采用此赛制,其中 $n=6$。)
\end{exercise}

\begin{exercise}
    证明对于每个大于等于 $2$ 的自然数 $n$,均有 $3^n \ge 2^{n+1}$。
\end{exercise}

\begin{exercise}
    判断下列不等式对哪些自然数 $n$ 成立?先陈述结论,再予以证明。
    \begin{enumerate}
        \item $2^n \ge (n + 1)^2$
        \item $2^n \ge n!$
        \item $3^{n+1} > n^4$
        \item $n^3 + (n + 1)^3 > (n + 2)^3$
    \end{enumerate}
\end{exercise}

\begin{exercise}
    \textbf{末日游戏}:两名玩家轮流在日历上命名日期。每回合玩家可增加月份或日期(但不可同时增加)。从 1 月 1 日开始,说出 12 月 31 日者获胜。确定先手玩家的必胜策略。例如,以下是玩家 1 获胜的过程:
    \begin{itemize}
        \item 玩家 1: 1 月 10 日;
        \item 玩家 2: 3 月 10 日;
        \item 玩家 1: 8 月 10 日;
        \item 玩家 2: 8 月 25 日;
        \item 玩家 1: 8 月 28 日;
        \item 玩家 2: 11 月 28 日;
        \item 玩家 1: 11 月 30 日;
        \item 玩家 2: 12 月 30 日;
        \item 玩家 1: 12 月 31 日。
    \end{itemize}
    此处\emph{必胜策略}指:无论玩家 2 如何应对,玩家 1 均可依此策略\emph{保证获胜}。
\end{exercise}

\begin{exercise}
    推导并证明\emph{几何级数}求和公式。几何级数定义如下:
    \[\sum_{i=0}^{n-1}q^i\]
    其中 $q$ 为实数,$n$ 为自然数。(提示:注意 $q = 1$ 的情形。)
\end{exercise}

\begin{exercise}
    构造一个与 $n$ 相关的命题,使其对 $n=1$ 至 $n=99$ 均为真,但当 $n=100$ 时为假。
\end{exercise}

\begin{exercise}
    以下``错误证明''声称对于所有 $n$ 有 $a^n=1$,请指出其错误所在:
    \begin{spoof}
        设 $a$ 为非零实数。已知 $a^0 = 1$。我们可以归纳地写出
        \[a^{n+1} = a^n \cdot a = a^n \cdot \frac{a^n}{a^{n-1}} = 1 \cdot \frac{1}{1} = 1\]
    \end{spoof}
\end{exercise}

\begin{exercise}
    某未来社会仅流通两种硬币:$3$ Brendan 和 $8$ Brendan。法令规定,商品价格必须能用这两种硬币\textbf{精确支付}。\\
    那么一杯咖啡的合法价格可能为多少?\\
    \textbf{提示:}尝试一些较小数值并观察其中规律。
\end{exercise}

\clearpage

\begin{exercise}
    对于任意自然数 $n$,考虑大小为 $2^n \times 2^n$ 的棋盘。若移除棋盘上\textbf{任意}一个方格,能否用 L 形三格骨牌密铺剩余部分?\\
    如果答案是肯定的,请证明这一点。\\
    如果答案是否定的,请提供反例论证。(即找到一个 $n$ 值证明其不可能密铺,并解释原因。)
\end{exercise}

\begin{exercise}
    考虑一个 $n \times n$ 的正方形网格。该网格中包含多少不同大小的子方格?例如 $n=2$ 时答案为 $5$:含有 $4$ 个 $1 \times 1$ 方格和 $1$ 个 $2 \times 2$ 方格。试推导通项公式并证明其正确性。
\end{exercise}

\begin{exercise}
    证明:在不少于 $2$ 人的队列中,若队首为女性、队末为男性,则必存在某位置上一男性紧邻一女性之后。
\end{exercise}

\begin{exercise}
    证明:对于任意自然数 $n, n^3 - n$ 是 $3$ 的倍数。
\end{exercise}

\begin{exercise}
    \textbf{二进制 $n$ 元组}是由 \verb|0| 和 \verb|1| 组成的有序字符串,字符串中共有 $n$ 个数字。用\emph{归纳法}论证其总数恰为 $2^n$。
\end{exercise}

\begin{exercise}
    斐波那契数列定义为 $f_0 = 0$, $f_1 = 1$,且对于 $n \ge 2$ 有 $f_n = f_{n-1} + f_{n-2}$。这会产生序列 $0, 1, 1, 2, 3, 5, 8, 13, 21, 34, \dots$。\\
    你可能不知道,斐波那契数列也存在\emph{封闭形式};也就是说,除了上面给出的常规递归定义外,还有一个特定\emph{公式}来定义它,即:
    \[f_n = \frac{1}{\sqrt 5}\Bigg[\Bigg(\frac{1+\sqrt 5}{2}\Bigg)^n - \Bigg(\frac{1-\sqrt 5}{2}\Bigg)^n\Bigg]\]
    证明此式对所有 $n \ge 0$ 成立。
\end{exercise}

\begin{exercise}
    再次考察斐波那契数 $f_n$,证明以下结论:
    \begin{enumerate}
        \item $\displaystyle{\sum_{i=0}^{n}f_i = f_{n+2} - 1}$
        \item $\displaystyle{\sum_{i=0}^{n}f_i^2 = f_n \cdot f_{n+1}}$
        \item $\displaystyle{f_{n-1} \cdot f_{n+1} - f_n^2 = (-1)^n}$
        \item $\displaystyle{f_{m+n} = f_n \cdot f_{n+1} + f_{m-1} \cdot f_n}$
        \item $\displaystyle{f_n^2 + f_{n+1}^2 = f_{2n+1}}$
    \end{enumerate}
\end{exercise}

\begin{exercise}
    用归纳法证明:每个 $n \ge 2$ 的自然数可表为质数之积。你能证明该分解的\emph{唯一性}吗?即证明质因数分解方式\emph{只有唯一一种}。
\end{exercise}

\begin{exercise}
    证明:
    \[\sum_{k=1}^{n} k \cdot k! = 1 \cdot 1! + 2 \cdot 2! + 3 \cdot 3! + \dots + n \cdot n! = (n+1)!-1\]
\end{exercise}

\begin{exercise}
    以下``错误证明''得出所有笔颜色相同,其谬误何在?
    \begin{spoof}
        当笔数为 $1$ 时,命题显然成立。

        假设任意一组 $n$ 支笔颜色相同。(注意:我们已经解释了为什么该假设对于 $n = 1$ 成立,所以我们可以做出此假设。)取任意一组 $n + 1$ 支笔。将它们排列在桌子上,从左到右用 $1$ 到 $n + 1$ 编号。查看其中的前 $n$ 个,即查看笔 $1,2,3, \dots , n$。这是一组 $n$ 支笔,因此根据假设,该组颜色相同。(我们还不知道是什么颜色。)然后,查看最后 $n$ 支笔;即查看笔 $2,3, \dots ,n+1$。这也是一组 $n$ 支笔,因此根据假设,该组也颜色相同。而 $2$ 号笔恰好属于这两个组。因此,无论 $2$ 号笔的颜色是什么,其必然是\dotuline{两组}中每支笔的颜色。因此,所有 $n+1$ 支笔具有相同的颜色。

        根据归纳法,这表明任何一组笔,无论数量多少,都只有一种颜色。因此,纵观世界上有限数量的钢笔,我们应该只能找到一种颜色。
    \end{spoof}
\end{exercise}

\begin{exercise}
    $\star$ 本题\emph{难度极高},摘自著名数学家陶哲轩 (Terence Tao) 的博客(\href{https://terrytao.wordpress.com/2011/04/07/the-blue-eyed-islanders-puzzle-repost/}{详见链接})

    有一座小岛,岛上居住着一个部落。这个部落有 $1000$ 人,有着不同颜色的眼睛。然而,他们的宗教信仰禁止他们知道自己眼睛的颜色,甚至禁止讨论这个话题;因此,每个居民都可以(并且确实可以)看到所有其他居民眼睛的颜色,但无法知道自己眼睛的颜色(无反射物)。如果部落成员确实发现了自己眼睛的颜色,那么他们的宗教信仰就会迫使他们第二天中午在村庄广场举行自杀仪式,让所有人围观。所有的部落成员都是高度逻辑和虔诚的,他们都知道其他人也是高度逻辑和虔诚的(并且他们都知道他们都知道其他人是高度逻辑和虔诚的……)。

    (就此逻辑谜题而言,``高度逻辑的''意味着能够从岛民可用的信息和观察中逻辑推断出的任何结论,该岛民将自动知晓。)

    事实证明,在这 $1000$ 名岛民中,有 $100$ 人是蓝眼睛,$900$ 人是棕眼睛,尽管岛民最初并没有意识到这些统计数据(当然,他们每个人只能看到 $1000$ 名部落居民中的 $999$ 人)。

    一天,一名蓝眼睛的外国人来到岛上,并赢得了部落的完全信任。

    某天晚上,他向整个部落发表讲话,感谢他们的热情款待。

    然而,由于不了解当地习俗,这名外国人在讲话中提及了眼睛的颜色,他表示``\emph{在世界的其他角落看到另一个像我这样有着蓝眼睛的人是多么不同寻常啊}''。

    这种失礼(如果有的话)会给部落带来什么影响?
\end{exercise}