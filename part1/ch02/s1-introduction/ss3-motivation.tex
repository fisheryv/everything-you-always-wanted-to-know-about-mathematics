% !TeX root = ../../../book.tex
\subsection{启下}

回顾 \ref{sec:section1.4.3} 节的问题,我们证明了前 $n$ 个奇数之和等于 $n^2$。最初通过几何视角观察这一模式:将奇数项排列为逐渐扩大的正方形``角块''。然而,第一种证明方法似乎并未依赖这一观察,而是以\emph{代数}方式运用了关于偶数与奇数之和的既有结论——通过对若干等式进行乘法、减法等操作,最终得到了预期结果。这种方法是否令人满意?它在某种程度上偏离了最初的几何解释,其有效性或许出人意料。(也许存在\emph{不同的}几何解释?读者可尝试探寻。)

第二种方法则是对几何观察的代数建模。我们将求和与正方形面积建立联系,将求和项对应于图形的特定部分。通过在不同问题解释间构建\emph{对应关系},使几何与代数解释互为支撑,共同指向同一结论。这种视觉化优势在于启发了名为\textbf{数学归纳法}的通用证明策略(简称\textbf{归纳法})。(请注意:\emph{归纳法}在电磁学或哲学等领域另有含义,但本书特指\emph{数学归纳法}。)究竟何为归纳法?其运作机制如何?适用范围是什么?如何针对具体问题调整策略?是否存在更有效的变体?本章将解答这些问题。

首先要探讨的,是此前未提及的核心问题:``\emph{为何}采用归纳法?\emph{为何}重视它?''基于 \ref{sec:section1.4.3} 节的问题,数学归纳法看似并非必需,因为其他方法同样可完成证明。这在一定背景下成立,但需强调:\emph{归纳法极具实用价值!}在众多情形中,它是最简洁的证明途径,且作为通用策略可广泛应用于同类问题。此外,适用归纳法的问题需具备特定\text{结构}——即结果的``后续部分''依赖``前序部分''。(``部分''与``依赖''的具体含义取决于上下文。)识别归纳法的适用性并完成证明过程,常能揭示问题的内在结构。即使归纳证明失败,发现``破坏''归纳步骤的具体环节,往往也能提供深刻洞见。

我们将通过若干示例阐明这些观点,再给出数学归纳法的完整\emph{定义}以展示其通用原理。(\text{严格}的形式化定义将延后至后续章节,待集合论、逻辑陈述与蕴涵等基础概念完备后展开。目前给出的定义已足以解决一些有趣的难题,并支撑归纳法作为通用证明策略的讨论。)
