% !TeX root = ../../../book.tex
\subsection{启下}

回顾一下 \ref{sec:section1.4.3} 节中的问题,我们证明了前 $n$ 个奇数之和恰好等于 $n^2$。我们首先通过将求和项(奇数)排列为正方形相继变大的``角块'',从几何角度观察到这种模式。然而,我们证明该结论的第一种方法似乎并不依赖此观察,而是以\emph{代数的}方式利用了先前的结论(关于偶数\emph{和}奇数之和);也就是说,我们对一些方程进行了一些复杂的操作(乘法和减法等等),然后 --- 瞧!--- 得到了我们预期的结果。你对这种方法有何看法?是不是感觉很满足呢?在某种程度上,它与我们一开始的几何解释不太相符,所以它的效果如此之理想可能会令人惊讶。(也许这种方法有\emph{不同的}几何解释。你能找到吗?)

我们的第二种方法是对最初的几何观察进行建模。我们将视觉形式转化为代数形式;具体来说,求和与正方形的面积有关,而求和项与该正方形的特定部分有关。我们在同一问题的不同解释之间建立了\emph{对应关系},找到了一种将一种解释与另一种解释联系起来的方法,这样我们就可以使用任何一种解释,并知道我们正在证明总体结果。视觉解释的好处是,它使我们能够利用称为\textbf{数学归纳法}的通用证明策略,有时简称为\textbf{归纳法}。(\emph{归纳法}一词也有一些非数学含义,例如在电磁学或哲学论证中,但在本书的上下文中,当我们说\emph{归纳法}时,我们指的是\emph{数学归纳法}。)归纳法到底是什么?它是如何工作的?我们什么时候可以使用这个策略?我们如何使策略适应特定的问题?在某些情况下是否有更有用的策略变体?这些都是我们希望在本章中回答的问题。

我们要谈论的第一个主题是我们在最后一段中没有问的一个问题,即``\emph{为什么}要用归纳法?\emph{为什么}要关注它?'' 基于第 \ref{sec:section1.4.3} 节中的问题,数学归纳法似乎并非完全必要,因为不用归纳法,可能也有其他方法可以给出证明。根据背景的不同,这很可能是正确的,但我们想从一开始就明确的一点是,\emph{归纳法非常有用!}在许多情况下,归纳证明是最简明的方法,并且它是一种众所周知的通用策略,可以应用于各种此类情况。此外,问题需要具有某种特定\text{结构}才能应用归纳法,即结果的一``部分''依赖于``前一部分''。(当然,``部分''和``依赖性''取决于上下文。)认识到归纳法的适用性,并实际经历随后的证明过程,通常会告诉我们一些有关问题的内在结构的信息。即便归纳证明失败也是如此!也许问题的某个特定部分``破坏''了归纳过程,识别该特定部分可能会有所帮助且让我们富有洞察力。

我们希望首先通过一些说明性的例子来激发这些观点,然后我们再提供数学归纳法的完整\emph{定义},以展示该方法在一般情况下是如何工作的。(完全\text{严格}的定义要推迟到稍后的章节才能给出,需要等到我们定义和研究了一些相关概念之后,例如集合论和逻辑陈述与蕴涵。不过,就目前而言,我们给出的定义足以解决一些有趣的难题,并让我们可以将归纳法作为一种通用证明策略进行讨论。)
