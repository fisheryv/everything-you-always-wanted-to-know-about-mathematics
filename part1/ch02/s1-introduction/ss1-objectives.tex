% !TeX root = ../../../book.tex
\subsection{目标}

以下简短内容将向你展示本章如何融入本书的体系。这部分内容会描述我们之前的工作将如何发挥作用,还会激发我们为什么要研究本章出现的主题,并告诉你我们的目标,以及你在阅读时应该记住什么来实现这些目标。现在,我们将通过一系列陈述为你总结本章的主要目标,以及本章结束时你应该获得的技能和知识。以下各节将更详细地重申这些想法,但这里将为你提供一个简短的列表以供将来参考。当学完本章后,请返回此列表,看看你是否理解所有这些目标。你明白为什么我们在这里概述它们很重要吗?你能定义我们使用的所有术语吗?你能应用我们描述的技术吗?

\textbf{学完本章后,你应该能够...}

\begin{itemize}
    \item 定义什么是归纳论证,以及将给出的论证分类为归纳论证或非归纳论证。
    \item 根据要解决的问题的结构来决定何时使用归纳论证。
    \item 通过类比启发式地描述数学归纳法。
    \item 通过比较和对比来识别和描述不同类型的归纳论证,并识别产生这些相似点和差异点的相应问题的基本结构。
\end{itemize}
