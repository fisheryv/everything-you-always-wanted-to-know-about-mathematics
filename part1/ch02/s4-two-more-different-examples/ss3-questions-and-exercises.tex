% !TeX root = ../../../book.tex
\subsection{习题}

\subsubsection*{温故知新}

以口头或书面的形式简要回答以下问题。这些问题全都基于你刚刚阅读的内容,如果忘记了具体定义、概念或示例,可以回顾相关内容。确保在继续学习之前能够自信地作答这些问题,这将有助于你的理解和记忆!

\begin{enumerate}[label=(\arabic*)]
    \item 如何\emph{归纳}这两个例子?它们与立方体、直线的例子在哪些方面相似?在哪些方面不同?
    \item 对于多米诺骨牌密铺问题,计算 $T(n)$ 需要知道前几项值?
    \item $T(n) = T(n - 1) + T(n - 2)$ 与 $T(n + 2) = T(n + 1) + T(n)$ 有什么区别?
    \item 取石子游戏的必胜策略是什么?请与不了解策略的朋友试玩,并采用玩家 $P_2$ 的必胜策略。观察你每次获胜时对方的反应如何?对方是否开始察觉这一策略?
\end{enumerate}

\subsubsection*{小试牛刀}

尝试解答以下问题。这些题目需动笔书写或口头阐述答案,旨在帮助你熟练运用新概念、定义及符号。题目难度适中,确保掌握它们将大有裨益!

\begin{enumerate}[label=(\arabic*)]
    \item $T(5)$ 的值是多少?能否画出所有密铺方案?
    \item 分析两堆各 $4$ 颗石子的取石子游戏:玩家 $P_2$ 是否总有必胜策略?
    \item \textbf{挑战题:}若使用\emph{三堆}等量石子进行游戏,会如何发展?能否为任意一方找到必胜策略?不妨与朋友实际对弈,观察结果!
    \item 探索\emph{斐波那契数列}:它与多米诺骨牌密铺问题中的数列 $T(n)$ 有何关联?
\end{enumerate}