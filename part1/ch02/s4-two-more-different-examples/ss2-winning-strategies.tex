% !TeX root = ../../../book.tex
\subsection{制胜策略}

这个例子将是我们第一个\emph{无需}证明数值公式的归纳问题!这似乎有些奇怪,但正如即将看到的,事实确实如此。实际上在数学中,这种情况比你想象的更为常见:一个问题或数学对象可能蕴含潜在的归纳结构,却并不依赖代数或算术的具体内容。

具体来说,我们将讨论一个\emph{游戏}。它是通常意义上的游戏——有玩家必须遵守的规则以及明确的胜负判定——同时也是数学意义上的游戏:我们可以用数学符号精确描述规则和情境,并抽象地探讨\emph{策略},甚至能\emph{解}出这个游戏。这与棒球比赛截然不同。

现在介绍游戏规则,我们暂且称之为``取石子''。有两名玩家 $P_1$ 和 $P_2$,$P_1$ 先行。桌上有两堆石子,每堆恰好有 $n$ 个($n$ 为自然数)。为区分不同版本,当每堆石子数为 $n$ 时,我们称玩家在进行``$T_n$ 游戏''。每回合玩家可从\emph{任意}一堆取走\emph{任意}数量的石子(不可同时取两堆)。取走\emph{最后}一颗石子者\emph{获胜}。

试着跟朋友玩一下这个游戏。用硬币或糖果代替石子,并试着互换 $P_1$/$P_2$ 角色。尝试制定获胜\emph{策略}——最大化胜率的玩法,推测不同 $n$ 值下的胜负关系。谁``应该''获胜?能否给出\emph{证明}?请务必先亲自体验游戏并尝试给出证明,再继续阅读我们的分析。你可能会惊讶于自己的发现!

与其他示例一样,我们先观察 $n$ 为较小数值的情形再进行推广。当 $n = 1$ 时,游戏很简单:$P_1$ 只能取走其中一堆的唯一石子,随后 $P_2$ 取走另一堆的石子获胜。(注意:无论 $P_1$ 选择哪堆,$P_2$ 总能取走剩余的石子。我们可以说``不失一般性'',假设 $P_1$ 选左堆,因为两堆等价。后文讨论数理逻辑时,我们将深入探讨``不失一般性''这一概念。)

\begin{center}
    \begin{tikzpicture}[line width=0.5mm]
        \foreach \n in {0, 1}{
            \node at (\n, 0)[circle,fill,inner sep=8pt, anchor=west]{};
            \draw (\n, -1) -- +(0.8, 0);
        }
        \draw[-latex] (2.5, -0.4) -- +(2, 0) node[midway,above]{$P_1$ 回合};
        \draw (5, -1) -- +(0.8, 0);
        \node at (6, 0)[circle,fill,inner sep=8pt, anchor=west]{};
        \draw (6, -1) -- +(0.8, 0);
        \draw[-latex] (7.5, -0.4) -- +(2, 0) node[midway,above]{$P_2$ 回合};
        \draw (10, -1) -- +(0.8, 0);
        \draw (11, -1) -- +(0.8, 0);
        \path (10, -0.4) -- +(2, 0) node[midway,above]{$P_2$ 胜!};
    \end{tikzpicture}
\end{center}

当 $n = 2$ 时,可能出现几种情况。考虑玩家 $P_1$ 可能的行动。虽然 $P_1$ 可选择左堆或右堆,但由于最终结果相同且两堆可交换,不妨设(不失一般性)$P_1$ 从左堆取走若干石子。具体而言,可能取一颗或两颗石子。下面对每种情况分别讨论。

\begin{center}
    \begin{tikzpicture}[line width=0.5mm]
        \foreach \x in {0, 1}{
            \node at (\x, 0)[circle,fill,inner sep=8pt, anchor=west]{};
            \node at (\x, 1)[circle,fill,inner sep=8pt, anchor=west]{};
            \draw (\x, -1) -- +(0.8, 0);
        }
        \draw[-latex] (2.5, -0.4) -- +(2, -0.6);
        \draw[-latex] (2.5, 0.6) -- +(2, 0.6);
        \node[anchor=west] at(2.5,0) {$P_1$ 回合};
        \foreach \delta in {1, -2}{
            \foreach \n in {0, 1}{
                \draw (5+\n, \delta) -- +(0.8, 0);
                \node at (6, \delta+1+\n)[circle,fill,inner sep=8pt, anchor=west]{};
            }
        }
        \node at (5, -1)[circle,fill,inner sep=8pt, anchor=west]{};
        \draw[-latex] (7.5, 2) -- +(2, 0) node[midway,above]{$P_2$ 回合};
        \draw[-latex] (7.5, -1) -- +(2, 0) node[midway,above]{$P_2$ 回合};
        \draw (10, 1) -- +(0.8, 0);
        \draw (11, 1) -- +(0.8, 0);
        \path (10, 1.6) -- +(2, 0) node[midway,above]{$P_2$ 胜!};
        \path (10, -1.4) -- +(2, 0) node[midway,above]{???};
    \end{tikzpicture}
\end{center}

若 $P_1$ 取走两颗石子,$P_2$ 应该如何应对?此时 $P_2$ 可以取走另一堆的全部石子从而获胜,因此 $P_1$ 不应采取此行动。但为全面分析游戏,仍需考虑所有可能情况。故在此情形下(对应示意图上半部分),$P_2$ 获胜。此情况较为简单。

若 $P_1$ 仅从左堆取走一颗石子(对应示意图下半部分),$P_2$ 应该如何应对?此时存在多种选择:

\begin{itemize}
    \item 若 $P_2$ 从左堆取走最后一颗石子,则 $P_1$ 可以取走右堆全部石子获胜。
    \item 若 $P_2$ 从右堆取走全部两颗石子,则 $P_1$ 可以取走左堆最后一颗石子获胜。
    \item 若 $P_2$ 仅从右堆取走一颗石子……
\end{itemize}

\begin{center}
    \begin{tikzpicture}[line width=0.5mm]
        \foreach \g in {0, 5, 10} {
            \foreach \x in {0, 1} {
                \node at (\x+\g, 0)[circle,fill,inner sep=8pt, anchor=west]{};
                \draw (\x+\g, -1) -- +(0.8, 0);
            }
        }
        \node at (0, 1)[circle,fill,inner sep=8pt, anchor=west]{};
        \node at (1, 1)[circle,fill,inner sep=8pt, anchor=west]{};
        \draw[-latex] (2.5, -0.4) -- +(2, 0) node[midway,above]{$P_1$ 回合};

        \node at (6, 1)[circle,fill,inner sep=8pt, anchor=west]{};
        \draw[-latex] (7.5, -0.4) -- +(2, 0) node[midway,above]{$P_2$ 回合};
    \end{tikzpicture}
\end{center}

此时局面与 $T_1$ 完全相同,而该情形已分析过:由于此时轮到 $P_1$ 行动,无论其如何操作,$P_2$ 都将获胜。对 $P_2$ 而言,此乃最优应对——\emph{无论} $P_1$ \emph{采取何种行动},$P_2$ 均可获胜!

综上可见:无论 $P_1$ 初始采取何种行动(从任意堆取一或两颗石子),$P_2$ 总能作出\emph{特定回应以保证获胜},且此后无论 $P_1$ 如何应对均无法改变结果。$P_2$ 已占据不败之地。现在让我们进一步探究其他 $n$ 值的情形。

当 $n = 3$ 时,我们再次不失一般性地假设玩家 $P_1$ 首先从左侧石子堆取石子。他可以取走一颗、两颗或三颗石子:
\begin{itemize}
    \item 若 $P_1$ 取走全部三颗石子,则 $P_2$ 将直接取走另一堆石子并获胜。
    \item 若 $P_1$ 取走两颗石子……此时 $P_2$ 应该如何应对?
\end{itemize}
$P_2$ 若取空右侧堆是愚蠢的($P_1$ 随后可取空左侧堆获胜),同样取空左侧堆也不明智($P_1$ 可以立即取空右侧堆获胜),因此需采取折中策略。若 $P_2$ 仅从右侧堆取走一颗石子,$P_1$ 可采取相同策略回应,导致两堆均剩一颗石子且轮次互换——此时 $P_2$ 成为先手方。根据此前分析,$P_2$ 在此局面下必败,因此这是一个糟糕的策略!

\begin{center}
    \begin{tikzpicture}[line width=0.5mm]
        \foreach \g in {0, 5, 10, 15}{
            \foreach \x in {0, 1}{
                \node at (\x+\g, 0)[circle,fill,inner sep=8pt, anchor=west]{};
                \draw (\x+\g, -1) -- +(0.8, 0);
            }
        }
        \foreach \x in {0, 1, 6}{
            \foreach \y in {0, 1}{
                \node at (\x, \y)[circle,fill,inner sep=8pt, anchor=west]{};
                \node at (\x, \y+1)[circle,fill,inner sep=8pt, anchor=west]{};
            }
        }

        \node at (11, 1)[circle,fill,inner sep=8pt, anchor=west]{};

        \draw[-latex] (2.5, -0.4) -- +(2, 0) node[midway,above]{$P_1$ 回合};
        \draw[-latex] (7.5, -0.4) -- +(2, 0) node[midway,above]{$P_2$ 回合};
        \draw[-latex] (12.5, -0.4) -- +(2, 0) node[midway,above]{$P_1$ 回合};
        \path (15, 0.6) -- +(2, 0) node[midway,above]{$P_1$ 胜!};
    \end{tikzpicture}
\end{center}

让我们重新推演:若 $P_2$ 从右侧堆取走两颗石子……局面立现转机!此时每堆仅余一颗石子,$P_1$ 作为先手必败,$P_2$ 锁定胜局!

\begin{center}
    \begin{tikzpicture}[line width=0.5mm]
        \foreach \g in {0, 5, 10}{
            \foreach \x in {0, 1}{
                \node at (\x+\g, 0)[circle,fill,inner sep=8pt, anchor=west]{};
                \draw (\x+\g, -1) -- +(0.8, 0);
            }
        }
        \foreach \x in {0, 1, 6}{
            \foreach \y in {0, 1}{
                \node at (\x, \y)[circle,fill,inner sep=8pt, anchor=west]{};
            }
        }
        \draw[-latex] (2.5, -0.4) -- +(2, 0) node[midway,above]{$P_1$ 回合};
        \draw[-latex] (7.5, -0.4) -- +(2, 0) node[midway,above]{$P_2$ 回合};
        \path (10, 0.6) -- +(2, 0) node[midway,above]{$P_2$ 胜!};
    \end{tikzpicture}
\end{center}

考虑 $n = 4$ 的情形,类似分析依然适用。玩家 $P_1$ 可以从左侧堆取走一至四颗石子。但无论其如何操作,$P_2$ 皆可在另一堆上\emph{模仿}相同动作,将游戏简化为更小规模的已知局面,从而确保胜利!$P_2$ 似乎始终占据主动:他能通过镜像策略回应 $P_1$ 的任何操作。无论 $P_1$ 采取何种行动,$P_2$ 的针对性回应终将导向胜利。因此我们称``$P_2$ 拥有必胜策略''——其具备明确的方法评估局势并选择特定行动以\emph{确保获胜}。

我们要如何证明这一点?如何运用本章的归纳法?目前或许难以窥见全貌。我们究竟需要证明什么?此处的多米诺骨牌或阶梯类比如何具象化?思考这个案例时需领悟:归纳法不仅关乎代数表达式,更体现某种``递推''结构,即大规模情形依赖于小规模情形;我们需先证明初始事实,再论证如何将任意更大规模情形化归为已证事实。这才是多米诺类比的本质。该类比虽能有效阐释部分归纳问题(非全部)且具有直观感染力,但并非普适于\emph{所有}场景。

回顾本章给出的四个案例,辨析其异同。尝试用更精准的数学语言(或自创术语)描述数学归纳法(需超越直观类比)。令人惊叹的是,即使尚未掌握``标准''表述,但你仍能通过此过程深化理解并获得精妙描述!后续我们将严格陈述并证明数学归纳法及其变体。在此之前,需先探索其他数学领域以构建必要的语言、符号与知识基础。不过启程之前,不妨先了解一下数学归纳法的若干经典应用。
