% !TeX root = ../../../book.tex
\subsection{制胜策略}

这个例子将是我们第一个\emph{无需}证明数值公式的归纳问题!这似乎很奇怪,但正如即将看到的那样,这却是一个事实。实际上在数学中,这比你想象的更常见:一个问题或数学对象可能存在某些潜在的归纳结构,但却不依赖代数或算术的内容。

事实上,我们将讨论一个\emph{游戏}。就是通常意义上的游戏——有两个玩家必须遵守的规则,并且有明显的赢家和输家——同时也是数学意义上的游戏,我们可以使用数学符号来制定规则和游戏情境,并以抽象的方式讨论\emph{策略}。我们甚至可以\emph{解}这个游戏。这与棒球比赛非常不同。

现在让我们讨论一下这个游戏的规则,我们暂时将其称为``取石子''。有两个玩家,称为 $P_1$ 和 $P_2$。玩家 $P_1$ 先行。玩家面前的桌子上有两堆石子,每堆正好有 $n$ 个石子,其中 $n$ 是某个自然数。(为了区分游戏的不同版本,当每堆石子有 $n$ 个石子时,我们会说玩家正在``玩 $T_n$''。)在每个玩家的回合中,他们可以从\emph{任意}一堆中取走\emph{任意}数量的石子。但不能同时从两堆石子中取石子。取走\emph{最后}一颗石子的玩家\emph{获胜}。

试着跟朋友玩一下这个游戏。使用硬币或糖果当石子。再试着切换一下角色,你先作为 $P_1$,然后再作为 $P_2$。尝试制定一个获胜\emph{策略},一种最大化获胜机会的游戏方法。尝试猜测不同的 $n$ 值下会发生什么。谁``应该''获胜?你能\emph{证明}吗?说真的,在继续阅读我们的分析之前,先玩玩这个游戏并尝试证明一些结论。你可能会对自己所取得的成就感到惊讶!

与其他示例一样,让我们使用 $n$ 的一些较小值来弄清楚到底发生了什么,然后再尝试进行泛化。当 $n = 1$ 时,这个游戏相当愚蠢。$P_1$ 必须取走其中一堆中唯一的石子,然后 $P_2$ 取走另一堆中唯一剩余的石子。因此,$P_2$ 胜。(请注意,无论 $P_1$ 选择两堆中的哪一堆,$P_2$ 总是会得到另一堆。我们可以说``不失一般性'', $P_1$ 选择左边这堆,因为无论选择哪一堆都无关紧要;情况是等价的,所以为了方便具体讨论,我们不妨说是左边这堆。后面讨论数理逻辑时,我们也会深入探讨``不失一般性''这一思想。)


\begin{center}
    \begin{tikzpicture}[line width=0.5mm]
        \foreach \n in {0, 1}{
            \node at (\n, 0)[circle,fill,inner sep=8pt, anchor=west]{};
            \draw (\n, -1) -- +(0.8, 0);
        }
        \draw[-latex] (2.5, -0.4) -- +(2, 0) node[midway,above]{$P_1$ 回合};
        \draw (5, -1) -- +(0.8, 0);
        \node at (6, 0)[circle,fill,inner sep=8pt, anchor=west]{};
        \draw (6, -1) -- +(0.8, 0);
        \draw[-latex] (7.5, -0.4) -- +(2, 0) node[midway,above]{$P_2$ 回合};
        \draw (10, -1) -- +(0.8, 0);
        \draw (11, -1) -- +(0.8, 0);
        \path (10, -0.4) -- +(2, 0) node[midway,above]{$P_2$ 胜!};
    \end{tikzpicture}
\end{center}

当 $n = 2$ 时,可能会出现几种情况。思考 $P_1$ 可能采取的行动。同样地,$P_1$ 可能选择左堆也可能选择右堆,但因为最终结果是相同的,并且我们可以交换这两堆,所以我们可以说(不失一般性)$P_1$ 从左堆取走若干石子。具体是多少?可能是一颗石子也可能是两颗石子。让我们分别检验每种情况。

\begin{center}
    \begin{tikzpicture}[line width=0.5mm]
        \foreach \x in {0, 1}{
            \node at (\x, 0)[circle,fill,inner sep=8pt, anchor=west]{};
            \node at (\x, 1)[circle,fill,inner sep=8pt, anchor=west]{};
            \draw (\x, -1) -- +(0.8, 0);
        }
        \draw[-latex] (2.5, -0.4) -- +(2, -0.6);
        \draw[-latex] (2.5, 0.6) -- +(2, 0.6);
        \node[anchor=west] at(2.5,0) {$P_1$ 回合};
        \foreach \delta in {1, -2}{
            \foreach \n in {0, 1}{
                \draw (5+\n, \delta) -- +(0.8, 0);
                \node at (6, \delta+1+\n)[circle,fill,inner sep=8pt, anchor=west]{};
            }
        }
        \node at (5, -1)[circle,fill,inner sep=8pt, anchor=west]{};
        \draw[-latex] (7.5, 2) -- +(2, 0) node[midway,above]{$P_2$ 回合};
        \draw[-latex] (7.5, -1) -- +(2, 0) node[midway,above]{$P_2$ 回合};
        \draw (10, 1) -- +(0.8, 0);
        \draw (11, 1) -- +(0.8, 0);
        \path (10, 1.6) -- +(2, 0) node[midway,above]{$P_2$ 胜!};
        \path (10, -1.4) -- +(2, 0) node[midway,above]{???};
    \end{tikzpicture}
\end{center}

如果 $P_1$ 取走两颗石子,$P_2$ 应该如何应对呢?$P_2$ 可以取走另一堆从而获胜,所以 $P_1$ 一开始就不应该采取这一行动。不过,可能 $P_1$ 脑子不清什么的,而且我们需要考虑所有可能的情况来全面分析这场比赛。因此,在这种情况下(上图中的上半行)$P_2$ 获胜。好吧,这就是简单的情况。

如果 $P_1$ 只从左边一堆石子(上图中的下半行)中取走一颗石子怎么办?$P_2$ 应该如何应对?我们现在有几种选择:

\begin{itemize}
    \item 如果 $P_2$ 从左边一堆石子中取走另一颗石子……那么,$P_2$ 取走另一堆全部石子,$P_1$ 获胜。
    \item 如果 $P_2$ 从右边一堆石子中取走全部两颗石子……那么,$P_1$ 取走左边一堆中的最后一颗石子,$P_1$ 获胜。
    \item 然而,如果 $P_2$ 只从右边一堆石子中取走一颗石子……
\end{itemize}

\begin{center}
    \begin{tikzpicture}[line width=0.5mm]
        \foreach \g in {0, 5, 10} {
            \foreach \x in {0, 1} {
                \node at (\x+\g, 0)[circle,fill,inner sep=8pt, anchor=west]{};
                \draw (\x+\g, -1) -- +(0.8, 0);
            }
        }
        \node at (0, 1)[circle,fill,inner sep=8pt, anchor=west]{};
        \node at (1, 1)[circle,fill,inner sep=8pt, anchor=west]{};
        \draw[-latex] (2.5, -0.4) -- +(2, 0) node[midway,above]{$P_1$ 回合};

        \node at (6, 1)[circle,fill,inner sep=8pt, anchor=west]{};
        \draw[-latex] (7.5, -0.4) -- +(2, 0) node[midway,above]{$P_2$ 回合};
    \end{tikzpicture}
\end{center}
现在我们遇到了与 $T_1$ 完全相同的情况,我们已经对此进行了分析!这次又是 $P_1$ 先走的,所以我们知道会发生什么:无论如何 $P_2$ 都会获胜。如果你是玩家 $P_2$,这显然是最好的应对:\emph{无论} $P_1$ \emph{如何行动},你都会赢!

退一步,让我们思考一下这表明了什么:无论 $P_1$ 首先采取什么行动(从任一堆中取走一个或两个石子),$P_2$ 都可以做出\emph{某个可能的回应,保证} $P_2$ 总会获胜,无论 $P_1$ 随后采取什么回应。哇,$P_2$ 稳坐钓鱼台!让我们看看其他 $n$ 值的情况下是否会发生同样的事情。

当 $n = 3$ 时,我们将再次假设(不失一般性)玩家 $P_1$ 从左边石子堆上取石子。他可以取走一颗、两颗或三颗石子:

\begin{itemize}
    \item 如果 $P_1$ 取走全部三颗,那么 $P_2$ 就会完全拿走另一堆并获胜。
    \item 如果 $P_1$ 取走两颗石子……那么 $P_2$ 应该做什么呢?
\end{itemize}
取完左边一堆是愚蠢的(因为 $P_1$ 可以取走整个右边一堆从而获胜),而取走整个右边一堆也同样愚蠢(因为 $P_1$ 可以取走整个左边一堆从而获胜),所以需要介于两者之间。现在,如果 $P_2$ 仅从右侧一堆石子中取走一颗石子,请注意 $P_1$ 可以采用相同的动作做出回应;从而使得两堆石子都只剩下一颗,但先手互换了。在这种情况下,$P_2$ 先行,根据我们之前的分析,$P_2$ 肯定会输。因此这是糟糕的策略!

\begin{center}
    \begin{tikzpicture}[line width=0.5mm]
        \foreach \g in {0, 5, 10, 15}{
            \foreach \x in {0, 1}{
                \node at (\x+\g, 0)[circle,fill,inner sep=8pt, anchor=west]{};
                \draw (\x+\g, -1) -- +(0.8, 0);
            }
        }
        \foreach \x in {0, 1, 6}{
            \foreach \y in {0, 1}{
                \node at (\x, \y)[circle,fill,inner sep=8pt, anchor=west]{};
                \node at (\x, \y+1)[circle,fill,inner sep=8pt, anchor=west]{};
            }
        }

        \node at (11, 1)[circle,fill,inner sep=8pt, anchor=west]{};

        \draw[-latex] (2.5, -0.4) -- +(2, 0) node[midway,above]{$P_1$ 回合};
        \draw[-latex] (7.5, -0.4) -- +(2, 0) node[midway,above]{$P_2$ 回合};
        \draw[-latex] (12.5, -0.4) -- +(2, 0) node[midway,above]{$P_1$ 回合};
        \path (15, 0.6) -- +(2, 0) node[midway,above]{$P_1$ 胜!};
    \end{tikzpicture}
\end{center}

让我们再试一次。如果 $P_2$ 从右边一堆石子中取走两颗石子……瞧!现在,我们每堆中只有一颗石子,$P_1$ 先行,所以我们知道  $P_1$ 一定会输。$P_2$ 再次完胜!

\begin{center}
    \begin{tikzpicture}[line width=0.5mm]
        \foreach \g in {0, 5, 10}{
            \foreach \x in {0, 1}{
                \node at (\x+\g, 0)[circle,fill,inner sep=8pt, anchor=west]{};
                \draw (\x+\g, -1) -- +(0.8, 0);
            }
        }
        \foreach \x in {0, 1, 6}{
            \foreach \y in {0, 1}{
                \node at (\x, \y)[circle,fill,inner sep=8pt, anchor=west]{};
            }
        }
        \draw[-latex] (2.5, -0.4) -- +(2, 0) node[midway,above]{$P_1$ 回合};
        \draw[-latex] (7.5, -0.4) -- +(2, 0) node[midway,above]{$P_2$ 回合};
        \path (10, 0.6) -- +(2, 0) node[midway,above]{$P_2$ 胜!};
    \end{tikzpicture}
\end{center}

思考一下 $n = 4$ 的情况,你会发现完全相同的分析再次出现。你会考虑另一种可能性:玩家 $P_1$ 可以从左边一堆石子中取走一颗、两颗、三颗或四颗。不过,无论 $P_1$ 怎么做,你都会发现 $P_2$ 可以在另一堆上\emph{模仿}相同的动作,将整个游戏简化为之前的\emph{较小}版本,从而保证 $P_2$ 一定获胜!看起来 $P_2$ 一直处于主导地位,因为他可以对 $P_1$ 的任何行为做出回应,在另一堆上做出相同的动作。无论 $P_1$ 做什么,$P_2$ 总会做出回应,这意味着 $P_2$ 一定获胜,无论 $P_1$ 随后的动作是什么。从这个意义上讲,我们说``$P_2$ 有制胜策略''。$P_2$ 有一个清晰且可描述的方法来评估比赛局势并选择特定的行动来\emph{保证获胜}。

我们要如何证明这一点?如何应用本章的归纳法?目前可能很难看出来。我们在这里到底要证明什么?对这个问题的类比中,多米诺骨牌或阶梯是什么?在你开动脑筋思考这个例子时,你应该意识到以下几点:归纳法并不总是与代数式有关;归纳法代表某种``构建''结构,较大的情况依赖于较小的情况;我们必须证明一些初始事实,然后论证如何化简任意更大事实,使其依赖于先前的事实。这才是多米诺骨牌类比的真正目的。碰巧的是,这个类比可以很好地解释某些归纳问题(但不是全部),并且是可视化的,令人印象深刻。但它并不完全适用于\emph{所有}情况。

回顾本章的四个例子,思考它们有何相似之处和不同之处。 尝试使用一些更好的术语(也许是你自己发明的)对数学归纳法进行更精确的数学描述。(这里的意思是比直观的类比更好。你会惊讶地发现,在不真正知道自己``应该''说什么的情况下,你竟然能够很好地描述归纳法,并且在这个过程中你会学到很多东西!)在适当的时候,我们会对数学归纳法及其各种形式进行严格的陈述和证明。与此同时,我们需要探索数学的一些其他领域,以建立必要的语言、符号和知识,以便回过头解决这个问题。不过,在开始之前,我们应该了解一下数学归纳法的一些有用的应用。
