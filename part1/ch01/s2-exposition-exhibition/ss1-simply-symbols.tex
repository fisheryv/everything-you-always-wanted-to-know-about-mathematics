% !TeX root = ../../../book.tex
\subsection{简单符号}

\subsubsection*{数学是门语言}

尽管看上去全是符号(那些写得密密麻麻的教科书所呈现的),但数学不仅仅是我们纸上所用符号的集合。英语基于一组固定的符号(字母表中的 26 个字母加上常见的标点符号,如句点、逗号和括号),但我们可以以特定的方式将这些符号组合在一起,并遵循一定的标准和约定,创作出有意义的单词、短语、句子、段落等;从本质上讲,英语和任何其他语言一样,通过符号集合以及组织这些符号的规则集合,来传达含义。同样的概念也适用于\emph{数学语言}:有一组符号和一组作用于这些符号的规则。

一个区别是,我们在数学中使用的符号集合可能相当大,具体取决于当前讨论的数学分支。数学结构多样性的一个重要部分是我们总是可以创建和定义要使用的新符号。通常,这样做只是为了使内容更简短、更易于阅读。

数学与其他语言之间的另一个主要区别是,我们仔细选择如何\emph{定义}我们的单词及其表达的概念。通常,数学家的大多数争论都围绕定义展开。这可能会令你感到惊讶;似乎数学家就证明和猜想进行辩论才更合理,甚至数学家居然会辩论都是一个新奇的想法!为新发现的概念选择正确的定义和术语是数学发现和阐述的重要组成部分,因为这有助于发现者/发明者向其他感兴趣的人解释他/她的想法。(没有这个过程,数学就不会进步,只是一群孤立的人试图自己发现真理。)

口语的情况与此类似,但似乎没有那么极端。例如,如果你对你的朋友说,``我饿了'',或者``我感觉有点饿了'',或者``天哪,我饿死了'',他们听到的基本上是相同的信息,并给出大致相同的回应。然而,在数学中,我们的定义要精确得多,并且不包含口语允许的细微差别。当然,这两种哲学各有优缺点,但在数学中,我们尽可能追求精确,因此我们希望我们的定义准确且稳定。尽管如此,我们可以掌控这些定义是什么!这就是为什么关于定义的争论在数学世界中如此普遍:为手头的概念选择正确的定义可以使未来使用这些概念的工作变得更加容易和方便。

\subsubsection*{恰当选择定义}

作为一个具体的例子,让我们回到上一节中看到的质数的定义\ref{def:prime}。它说的是:

\begin{definition}
    如果大于 $1$ 的正整数 $p$ 其正因子只有 $1$ 和 $p$,则称 $p$ 为\dotuline{质数}。非质正整数称为\dotuline{合数}。
\end{definition}

这个定义似乎没有什么问题,不是吗?也许你会用不同的措辞或更简洁的表达或使用不同的可变字母或其他,但最终的信息是相同的:质数是具有特定属性的某种类型的数字。而你选择写出该特定类型的数字是什么(大于 $1$ 的正整数)以及该属性是什么(没有除 $1$ 和它本身以外的正因子),你将获得等效的定义。

不过,这个定义背后存在一些微妙的问题:为什么它是那种特定类型的数字?为什么我们如此关心这个特殊的属性——只能被 $1$ 和它本身整除?如果定义略有不同怎么办?事情真的会有那么大的改变吗?我们将用另一个问题来解决这些问题:你如何看待以下质数的替代定义?

\begin{definition}\label{def:prime2}
    如果小于 $-1$ 或大于 $1$ 的整数 $p$ 其正因子只有 $1$ 和 $p$,则称 $p$ 为\textbf{质数}。
\end{definition}

你注意到细微的差别了吗?所有符合之前``质数''定义的数字仍然符合这个定义,但现在负数也适用!具体来说,给定任意数字 $p$ 在旧定义下是质数,$-p$ 现在在新定义下也是质数。这是一个合理的想法吗?负质数有什么问题?

质数的第三个定义怎么样?

\begin{definition}\label{def:prime3}
    如果正整数 $p$ 的正因子只有 $1$ 和 $p$,则称 $p$ 为\textbf{质数}。
\end{definition}

(请记住,按照惯例,$0$ 既不是正数也不是负数。)现在,负数会超出范围,但 $1$ 符合此定义。这合理吗? $1$ 的唯一正因子是 $1$ 和...它本身,对吗?

这就是可能引发争论的地方:也许你不介意让 $1$ 成为质数,但你的朋友会强烈反对。好吧,如果没有确凿的理由,就没有办法说你们中任何一个都是错的,真的;你只是对术语做了不同的选择,它们都没有改变 $1$ 的唯一正因子是 $1$ 和它本身这一固有属性。类似地,请考虑一下:无论你称它们为凉鞋、拖鞋还是人字拖,事实仍然是这些类型的鞋子适合在海滩上穿。

然而,考虑到历史的后见之明和新的愿望,通常一个特定的定义被认为更加合适。未来,我们将研究质数分解,这是一种将每个(正)整数写为质数乘积的方法。例如,$15 = 3 \cdot 5$, $12 = 2 \cdot 2 \cdot 3 = 2^2 \cdot 3$ 和 $142857 = 33 \cdot 11 \cdot 13 \cdot 37$ 都是质因数分解。

这些因数分解也有一个特殊的性质:一般来说,正整数的质因数分解是\textbf{唯一的}!也就是说,只有一种方法可以将正整数写为质数的乘积(因为我们将因子的不同排序视为同一事物,所以 $105 = 3 \cdot 5 \cdot 7$ 和 $105 = 7 \cdot 3 \cdot 5$ 是相同的因数分解)。我们将使用上面给出的第一个定义严格证明这一点。如果我们使用第二个定义或第三个定义会怎么样?这种唯一性还存在吗? 为什么这种唯一性如此重要?最终,结论是,定义应该由逻辑和实用性驱动,并且这可能会随着时间的推移而改变并引发一些争论。

\subsubsection*{数学家的学习模式}

建立清晰准确的定义的另一个好处是你可以像思想者一样获取知识和理解;人类学习的一个主要方面涉及通过日常经验识别模式,接着将想法、概念、词语、事件与这些模式联系起来。然后,人们可以使用这些模式来预测抽象的想法、概念和事件并对其进行理论化。

例如,研究表明,人类婴儿最初缺乏\emph{物体存继性}概念,随着时间的推移逐渐发展起来。如果你给孩子看一个他们喜欢的彩色玩具,然后把它藏在纸箱下面,孩子不太明白这个玩具仍然存在,只是看不见了。他/她会表现得好像该物体不再存在一样。然而,在某些时候,我们知道这不是真的,我们视野之外的物体仍然存在。这究竟是如何发生的?也许是我们见识到许多此类事件的模式,其中一个物体``消失'',然后我们又找到它。

更好的例子可以在自然科学中找到,它们说明了模式识别和抽象思维的另一个方面,这是极其重要的,特别是在数学和科学领域。我们可以想象,尼安德特人不知何故知道,每当他们拿起一块岩石并将其保持在一定距离,然后松开时,岩石就会掉到地上。这种情况可能一次又一次地发生,所以他们``明白''这种现象是自然的必然产物。在发生足够多的事件之后,人们很可能明白这种情况总会发生,或者至少,任何没有发生的情况都会引起极大的困惑和恐惧。(正是这种情绪反应可能有助于解释火山爆发等罕见但强烈的事件如何导致古代文明将此类事件归咎于``神之愤怒'')。

对事件的观察并没有使史前人类进一步理解\emph{为什么}岩石总是会掉落到地面,或者能够\emph{解释}为什么它每次都必然发生。几千年后,人们才开始思考这种现象为何发生以及如何发生,更长时间之后,艾萨克·牛顿(Isaac Newton)最终提出了一个试图解释重力行为的模型(最终为此类现象命名)。有人说,即使是现在,我们仍然没有弄清楚它到底是如何运行的。(如果你好奇的话,可以上网搜索``循环量子引力''并尝试理解这一点)。

正是这种思维上的抽象飞跃——从对某种模式的观察到对该模式的认识论理解——从最好的意义上来说,是真正具有好奇心和智慧的思想家、真正的科学家的特征。你认为谁是更好的昆虫学家:贪婪的读者,他已经记住了世界上所有目前已知的甲虫种类,或者实验室科学家,他检查了多种物种,可以采集新标本并对其分类为甲虫还是非甲虫?这在某种程度上是一个引导性问题,但要点是:\emph{理解}定义及其背后的动机比简单地了解一堆满足某个定义的\emph{实例}要有益得多。

可以说,这对数学更为重要。你能想象一个数学家不知道质数是什么,只能凭记忆列出前 100 个质数并对此沾沾自喜吗?当然不是!数学研究的美妙、通用和魅力部分在于我们检查模式和现象,然后选择如何做出与这些模式相关的适当定义。然后,我们利用对这些模式的新理解来对其他模式和现象做出严格精确的预测。彻底理解定义或概念可以提高预测能力,并且比仅仅了解该定义/概念的示例更为有效。

