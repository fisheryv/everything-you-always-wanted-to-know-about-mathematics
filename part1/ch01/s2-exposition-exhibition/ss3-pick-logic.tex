% !TeX root = ../../../book.tex
\subsection{选择逻辑}

我们已经非常频繁地使用``逻辑''一词及其相关形式,但还没有充分解释其意思。我们意识到这似乎违背了我们迄今为止一直大力倡导的精确性和清晰性,但不幸的是,我们不得不承认,提供\emph{逻辑}的完整定义是极其困难的。

\subsubsection*{一场游戏}

如果你寻找对逻辑的启发式理解,可以尝试从``逻辑谜题''(如数独或数谜)入手来思考它。这些谜题/游戏从一开始就围绕定下的非常具体的规则构建的,然后向解谜者提供一个起始谜面,并期望解谜者严格遵守规则,直到解出谜题。例如,在数独中,规则是 1 到 9 中每个数字在每行、每列和 $3 \times 3$ 框中恰好只出现一次,解谜者需要综合各种情况在网格中放置越来越多的数字,不断缩小``潜在解''的范围,以找到起始谜面的唯一答案。这个解谜过程的一个重要方面是,任何时候都不要(自作聪明地)\emph{猜测};每一步都应该在考虑当前情况和谜题既定规则的情况下进行理性选择,并且在这个框架内,保证谜题是可以解决的(当然,要有足够的时间)。

数学逻辑在某些方面略有不同,但本质是一样的:都有既定的游戏规则,每一步都应该以这些规则和当前知识为指导,除此之外别无其他。这就是我们所说的撰写数学证明应该受\emph{逻辑}支配的意思:从一个真理到另一个真理,每一步都应该遵循约定的规则,并且只参考这些规则或已经证明的事实。我们在证明(以及一般的数学中)中玩的``游戏''或``谜题''并不像数独谜题那么清晰。然而,更令人困惑的是,有时我们会投身一场无法获胜的游戏,却丝毫没有意识到这一点!

这里``无法获胜的游戏''的想法来自 20 世纪奥地利逻辑学家、数学家库尔特·哥德尔 (Kurt Gödel) 的工作成果,这是一项令人震惊、使人惊奇但极其有力的结论。他的\emph{不完备性定理}反映出一个强大的逻辑系统内部的固有问题:有些\textbf{真实}陈述在该系统内却是不\emph{可证明}的。在这里,我们无法透彻详细地解释一些术语(即,\emph{逻辑系统}和\emph{可证明}),但希望你能看到这里出现了一些神奇的事情。这怎么可能呢?如果某事在数学中是\textbf{真的},我们不是应该能够以某种方式证明它是真的吗?否则我们怎么知道这是真的呢?

\subsubsection*{数学简史}

要回答这些自然而生的问题,让我们先退一步,回到数学的一个重要分支——逻辑的起源进行讨论。在整个讨论过程中要记住的一点是,我们无法完全解决出现的每个主题,这可能让人感到不满,我们理解这一点。数学之美部分在于,学习任何一个主题都会带来许多其他问题和概念需要思考,而这些问题和概念又可以通过更多的数学来解决。不过,背景很重要,就本书的背景而言,我们没有足够的时间和空间去讨论所有这些相关的主题。我们并不是试图向你隐瞒任何事情或掩盖某些问题;相反,我们只是在面对现实,确保我们不会强迫你阅读 10,000 页的完整数学史,只是为了理解我们的观点!

在你的数学生涯中,可能会进一步研究我们下面提到的许多数学家(以及他们所做的工作)。到那时,你将通过亲自动手实践从而对这个学科有更深入的理解和欣赏,你也将更有能力去解决其中的问题。而现在,我们仅仅是基于兴趣介绍这些数学家。数学有着丰富而有趣的历史,了解它会很有帮助!在这里,我们将尽力以简洁而有意义的方式来解读逻辑学——它的历史、动机和意义——使之与当前的背景相契合。

19 世纪中后期的数学家和哲学家首先研究了后来演变成现代逻辑的思想,他们对我们在这里试图研究的许多相同问题感兴趣:我们如何知道某件事是\textbf{真}的?我们如何才能表达这个真理呢?我们可以声明什么类型的``事件''为\textbf{真}或为假?这些数学家从根本上分解了数学语言,研究了如何以非常具体的方式组合一组固定的符号来创建更复杂的陈述,但从总体上看,这些陈述仍相当简单。这并不是要贬低他们的努力,毕竟,我们都必须从某个地方开始,而这些人是从头开始的。

首先进行的一项重大工作是探究算术的基础,或者说\textbf{自然数}($1,2,3,4,\dots$)的研究。就像欧几里得(Euclid)研究几何时,先通过建立一系列公认的真理或\textbf{公理},然后从这些给定的假设中推导出真理一样,意大利数学家朱塞佩·皮亚诺(Giuseppe Peano)建立了一套自然数公理,而其他人则从稍微不同的视角对这个主题进行了研究。与此同时,对真理及其证明严谨果断的欣赏,促使得大卫·希尔伯特和其他人提出了欧几里得公理的一些问题,尤其是平行公设。

这项关于几何和算术的工作自然引出了对数学其他领域的进一步、复杂的研究,以及对诸如实分析等领域热切地公理化尝试。卡尔·魏尔斯特拉斯(Karl Weierstrass)在研究这个主题时,提出了一些具有奇特属性的令人震惊的函数示例。例如,尝试定义一个处处不可微的连续函数。(如果你对微积分的这些术语不熟悉,请不用担心;总之,这很难。)最后,理查德·戴德金(Richard Dedekind)能够建立一个严谨的、逻辑的实数定义,完全由自然数推导出来,并且不依赖于必须存在数字连续体这种模糊的物理概念。

后来,这项研究稍微分支出来,变成了集合的研究。这个领域的许多基础工作是乔治·康托尔(Georg Cantor)在 19 世纪末期奠定的。他是第一个真正研究无限集理论的人,提出了无限有不同``大小''这一有争议的观点。也就是说,他证明了某些无限集严格大于其他无限集。这个想法在当时引起了极大的争议,以至于许多数学家都讨厌他!如今,我们意识到康托尔是对的。(这也让你提前窥探我们稍后在 7.6节中将会讨论的内容。举个有趣的例子:奇数集合和偶数集合当然一样大,但它们也和所有整数的集合一样大。然而,所有实数的集合严格大于二者!)

事实上,一些数学家对康托尔的发现感到相当震惊,甚至伟大的伯恩哈德·黎曼(Bernhard Riemann)一开始也认为集合论的发展将成为数学的祸害。但事实并非如此,从诞生起它就蓬勃发展,许多数学家致力于以正确的方式表示所有数学并理解数学的``基础''。某种程度上,你可以将集合论视为对所有数学家正在研究的基本对象的研究,最终,其方式类似于所有化学都是通过将元素周期表中的元素以越来越复杂的方式恰当地组合在一起来完成的。

这些主题的进一步发展是符号逻辑的研究,它比我们迄今为止提到的抽象概念更具体一些,而且我们在本书的开始章节中会频繁地研究这个领域的基本理念。该领域涵盖了如何将数学方程和符号与基于语言的符号和连词结合起来,以做出有意义的数学陈述,并可以通过证明来确认这些陈述的真实性。总的来说,这是数学的一个极其重要的组成部分,尤其是本书。个人观点当然比这更加细致和具体,但总的来说,大多数数学家的心态是,有许多数学真理等待被发现,我们花时间学习我们已经发现的真理,希望揭示更多真理。这就像一个巨大的考古挖掘,研究我们已经出土的骨头和文物将帮助我们预测我们在什么地方会发现什么类型的其他宝藏,以及如何寻找和挖掘。某种程度上,逻辑是从挖掘中一步一步抽象出来的过程:逻辑是对挖掘过程的研究。 它告诉我们如何真正利用数学知识并从中学习,并将其与其他知识相结合,从而证明更多的真理。

请注意,这不是一个精确的类比,抽象逻辑的研究要复杂得多。不过,就本书的目的而言,这是一种合理的思考逻辑的方式。我们将学习符号逻辑的一些基本原理和基本运算,并将这些知识应用到我们撰写证明的研究中。它将帮助我们真正理解证明是什么,它将指导我们构建要编写的证明,它将允许我们批判可能不正确的证明,并最终帮助我们理解数学作为一个整体是如何工作的。

\subsubsection*{逻辑应用:理论计算机科学}

逻辑思想和结果的一个非常重要的应用是计算机科学的发展和研究,特别是理论计算机科学和可计算性理论。这个特殊的数学分支最初是受大卫·希尔伯特二十三个问题——1900 年出版的数学界著名未解决猜想列表——中的第十个问题推动的。第十问题涉及解\textbf{丢番图方程(Diophantine Equations)}, 就是以下形式的方程
\[a_1x_1^{p_1}+a_2x_2^{p_2}+a_3x_3^{p_3}+\dots+a_nx_n^{p_n} = c\]
其中 $a_1, a_2, \dots, a_n$ 和 $c$ 都是给定的常数,$p_1, \dots, p_n$ 为给定的自然数,$x_1, \dots, x_n$ 为要求的使方程成立的变量。

给定这样一个方程,人们可能想知道是否存在解,如果存在,那么存在多少组解。 此外,如果我们给定常数 $a_i$ 和 $c$ 都是有理数,我们想知道是否可以确保存在一组解,其中所有变量 $x_i$ 也都是有理数。关于这个特定问题已经建立了一些理论成果,但是根据 1900 年的陈述,希尔伯特第十问题问的是,是否存在``一个过程,根据该过程可以经过有限数量的操作确定它''是否存在给定方程的解,其中所有变量 $x_i$ 都是有理数。尽管当时还没有算法这个术语的正确概念或定义,但希尔伯特要求的是一种\textbf{算法},该算法接受常数 $a_i$ 和 $c$ 的值,并根据是否存在所需属性的解输出 \textbf{True} 或 \textbf{False}。这个问题的一个重要部分是,该``过程''在输出答案之前执行有限数量的步骤。

一位名叫艾伦·图灵(Alan Turing)的英国剑桥大学学生几年后开始研究这个问题,他想到了一台物理机器,该机器将执行输出所提出问题的答案所需的步骤。他在随后的出版物中描述了他的发明,我们现在称之为\emph{图灵机},这是一种有趣的理论装置,可以用来回答形式逻辑中的一些问题,但也展现了构建现代计算机的许多想法。我们说它是一个理论装置,是因为它的定义的属性决定了它在物理上无法构建和操作,但它很好地处理了一些理论问题,包括前面提到的希尔伯特第十问题。更具体地说,当我们说某件事是可计算的,或者能够在有限数量的步骤中确定时,这台机器为我们的意思提供了正确的定义,这有助于建立正确的算法概念。如果我们在讨论可计算性话题时不提及阿隆佐·丘奇(Alonzo Church),那是不公平的,因为他与图灵同时在研究类似的问题。他们的名字一起出现在丘奇-图灵论文中,该论文将图灵机的工作原理与更理论化、基于形式逻辑的可计算性概念联系在一起。

\subsubsection*{我们将用逻辑做什么?}

虽然集合论和逻辑中的所有这些主题本质上都很有趣并且对数学非常重要,但总的来说,我们根本没有足够的时间和空间来详细讨论它们。相反,我们更关注在撰写和批判数学证明时使用的逻辑概念。

我们将考虑:

\begin{enumerate}
    \item 我们实际上可以陈述和证明什么类型的``事物'',
    \item 我们如何将我们已知为真的``事物''结合起来以产生更复杂的真理,
    \item 我们如何解释我们是怎样得出``事物''确实为真的结论。
\end{enumerate}

由于缺乏更好的术语,这里我们用``事物''一词,因为我们还没有\textbf{数学陈述}的正式定义,而这实际上是我们将要证明的``事物''的类型。从本质上讲,数学陈述是数学和语言中符号和句子的组合,可以验证为\textbf{真}或为\textbf{假},但不能同时既真又假或非真非假。那么,证明就相当于组织一系列步骤和解释,使用为真的数学陈述和句子将这些真理连接在一起,并最终产生特定陈述所需的真理。我们对逻辑的研究将解决如何组合这些步骤并确保我们的证明最终会导出对真理的正确评估。

更具体地说,我们将研究数学陈述到底是什么,以及如何将它们组合起来产生更复杂的陈述。``\emph{与}''和``\emph{或}''这两个词在其中特别重要,因为这两个词允许我们以新的、有意义的方式将两个数学陈述组合在一起。我们还将研究\textbf{条件}数学陈述,即``如果 A,则 B''或``A 蕴含 B''形式的语句。这些是数学陈述的基础,大多数重要的数学定理都是这种形式。这些陈述涉及做出一些\emph{假设(assumption)}或\emph{假说(hypothes)}(包含在陈述 A 中),并使用这些假设的事实得出结论(包含在陈述 B 中)。回顾 \ref{sec:section1.1.1} 节中毕达哥拉斯定理的陈述,注意它是如何以条件陈述的形式出现的。(可以用另一种方式书写吗?尝试以非条件形式重写定理的陈述,并思考在该形式下是否本质上是不同的陈述。找到另一个以条件陈述形式给出的著名数学定理,并尝试进行相同的格式更改。)

数学中的另一个重要思想,也是在证明撰写中经常出现的思想,是\textbf{变量}的概念。有时我们想笼统地讨论一种数学对象,而不为其分配特定的值,这就需要通过引入变量来实现。你可能在之前的数学学习中经常看到这种情况发生,甚至在本书中我们已经使用过变量了。再看看 \ref{sec:section1.1.1} 节中的毕达哥拉斯定理的陈述。字母 $a,b,c$ 代表什么?好吧,我们并没有给出明确说明,但我们知道它们是正实数,表示直角三角形三边的长度。什么三角形?我们并没有给出一个具体的三角形,也没有给出一张具体的图画或类似的东西,但你清除我们在说什么。此外,我们要检查的证明并不取决于这些变量的实际值,而仅仅取决于它们是否是具有某些属性的正实数。 这是非常有用且重要的,某种程度上,它节省了时间,因为我们不必单独考虑宇宙中所有可能的直角三角形(有无穷多个!) 并且可以将整个想法简化为一个紧凑的陈述和证明。

我们可以对变量进行\emph{量化}。这涉及到声明某个陈述对于变量的\emph{任意}潜在值或仅对\emph{某个}特定值是否成立。例如,在毕达哥拉斯定理中,我们不能声称 $a^2+b^2=c^2$ 对任意正实数 $a,b,c$ 成立;我们必须对变量施加额外的假设才能获得我们所做的结果。这是\textbf{全称}量化的一个例子:``对于\emph{所有}具有这个属性和那个属性的数字 $a,b,c$,我们可以保证...''同样地,我们还可以进行\textbf{存在}量化:``\emph{存在}一个具有此属性的数$n$。''

你能想到我们迄今为止已经研究过的使用存在量化的定理/事实吗?再来看一个证明,存在无理数 $a$ 和 $b$ 使得 $a^b$ 是有理数。请注意,我们证明的这个主张属于存在类型:我们声称\emph{存在}两个具有所需属性的数字,然后我们继续证明确实必然存在这样的数字。眼下,这个证明的有趣之处在于它是\emph{非构建性的};也就是说,我们能够在不明确给出数字 $a$ 和 $b$ 实际是什么的情况下证明我们的主张。我们将其缩小到两个选择,但从未声称哪一个是正确的选择,只是其中之一必然有效。
