% !TeX root = ../../../book.tex
\subsection{正确撰写}

数学的另一个有趣的方面是,尽管它本身就是一种语言,但我们依赖外部语言来传达我们所拥有的数学思想和见解。尝试在不使用任何单词的情况下重写我们之前看过的定义和证明。这很难,不是吗?因此,我们希望用来传达数学思想的书面语言遵循与我们所写的数学``句子''相同的标准:我们希望它们\emph{精确}、\emph{合乎逻辑}且\emph{清晰}。

现在,为这三个词下一个精确、合乎逻辑且清晰的定义本身就是一项艰巨的任务。然而,我们都认同理想的证明应该是:

\begin{itemize}
    \item \textbf{精确的:}任何个体陈述都不应是不真实的或可以通过多种方式解释从而导致真相需要商榷;
    \item \textbf{合乎逻辑的:}每一步都应遵循先前的步骤,并有适当的动机和解释;
    \item \textbf{清晰的:}步骤间应该用正确的语法连接和描述,帮助读者了解发生了什么。
\end{itemize}

让我们检视几个无视这些标准并且在某种程度上不符合我们迄今为止的证明定义的``证明''。

\subsubsection*{糟糕``证明''\#1}

首先,我们来一个 $1=2$ 的``证明'',我们知道这肯定有问题。你能找到哪里出错了吗?它违反了哪个标准?精确、合乎逻辑还是清晰?

\begin{proofs}{``证明''}
    假设有两个实数 $x$ 和 $y$,考虑如下等式:
    \begin{align*}
        x &= y \\
        x^2 &= xy &\text{两边同时乘以} y\\
        x^2-y^2 &= xy-y^2 &\text{两边同时减去} y^2\\
        (x+y)(x-y) &= y(x-y) &\text{因式分解} \\
        x + y &= y &\text{两边同时消掉} (x-y)\\
        y + y &= y &\text{因为第一行给定} x=y\\ 
        2y &= y \\
        2 &= 1 &\text{两边同时除以} y
    \end{align*}
\end{proofs}

这里的问题是\emph{精确}。对第四行进行因式分解后,除以公因数 $(x - y)$ 即可得到第五行,这似乎既方便又明智;然而,第一行告诉我们 $x = y$,所以 $x-y = 0$,\textbf{除以零是不允许的}!使用变量 $x$ 和 $y$ 只是一种让你失去踪迹并掩盖除以零的方法。(说到这里,为什么不能除以零?你能想出一个合理的解释吗?从乘法的角度思考一下。)

\subsubsection*{糟糕``证明''\#2}

这是类似``事实''的另一个证明,即 $0 = 36$。

\begin{proofs}{``证明''}
    考虑方程 $x^2+y^2 = 25$。整理并分离 $x$ 可得
    \[x = \sqrt{25-y^2}\]
    两边加3再同时平方得
    \[(x+3)^2=\Big(3+\sqrt{25-y^2}\Big)^2\]
    请注意,$x = -3$ 和 $y = 4$ 是原方程的解,所以最终的方程也应该是成立的。将这组解代入 $x$ 和 $y$ 可得
    \[0 = (-3+3)^2 = (3+\sqrt{25-16})^2 = (3+3)^2 = 36\]
    因此,$0 = 36$。
\end{proofs}

到底发生了什么?你能发现不合逻辑的步骤吗?如果我们使用最后选择的变量 $x$ 和 $y$ 的特定值重写证明步骤,也许会有所帮助:

\begin{align*}
    (-3)^2+4^2 &= 25 \\
    -3 &= \sqrt{25-4^2} \\
    (-3+3)^2 &= \Big(3+\sqrt{25-4^2}\Big)^2 \\
    0 &= 36
\end{align*}

现在很明显了,不是吗?对方程两边取平方根存在一个问题,它取决于 $(-x)^2=x^2$ 这一事实。

当我们解 $z^2=x^2$ 这样的方程时,必须牢记这个方程有两个根:$z = -x$ 和 $z = x$。因此,从方程开始并对两边进行平方是一个完全合乎逻辑的步骤(所得方程的真值与原方程的真值\emph{一致}),但反之却是一个不合逻辑的步骤(平方方程成立并不\emph{一定}等于平方根方程也成立)。这是一个带有\textbf{条件陈述}或\textbf{逻辑蕴涵}的问题,我们稍后会详细讨论这些概念(第 4.5.3 节)。现在,我们可以用下面的代码来总结这个概念:

\[\text{如果} a=b, \text{则} a^2=b^2, \text{反过来,如果} a^2=b^2, \text{则} a=b \text{或} a=-b\]

这说明了为什么在上面``证明''中从 $x^2+y^2 = 25$ 到 $x = \sqrt{25-y^2}$ 这步是不合逻辑的:当有两种可能的选择时,我们立即假设平方根的一种特定选择。如果我们选择负平方根,会发生什么?试着将第二步替换为 $-x = \sqrt{25-y^2}$ 并重写证明,在最后对 $x$ 和 $y$ 使用相同的值。发生了什么? 如果你用 $x = 3, y = -4$ 代替呢?或者 $x=-5, y=0$ 呢?你能描述一下如何确定何时应该使用正根 $x$ 何时应该使用负根 $-x$ 吗?

\subsubsection*{数学使用``兼或''}

既然``或''这个词已经出现,我们先提一下上面句子中\emph{或}的使用。当我们说 ``$a = b$ 或 $a = -b$'' 时,我们的意思是,这两个陈述中\emph{至少}有一个必须为真,甚至可能两者都为真。现在,如果 $a \ne 0$ 且 $b \ne 0$,则只有其中一种结论性陈述可以为真;也就是说,在这种情况下,只有一个根(正或负)是正确的,而不是两者都是正确的。然而,如果 $b = 0$,那么两个结论性陈述说的是同样的事情,$a = 0$,因此规定\emph{或}意味着只有一个陈述可以为真并且不允许它们都为真,这是不合逻辑的。在其他情况下,这种区别会产生更显着的差异。

例如,如果你在餐厅点了一份三明治,服务员问:``您想要薯条还是土豆沙拉?'',这可以理解为,你可以选择其中一项,但不能同时选择两者。这就是\textbf{异或}的示例,因为它阻止你选择两个选项。或者,如果你忘记带书写工具去课堂,准备用老方法记笔记于是询问你的朋友,``我可以借用您的铅笔或钢笔吗?'',这可以理解为,你实际上并不关心提供两个选项中的哪一个,只要至少有一个可用即可。也许你的朋友两者都有,而且其中任何一个都可以。这是\textbf{兼或}的示例,并且这是所有数学示例中假定的解释。

\subsubsection*{不清楚的论证}

最后两个糟糕的``证明''问题在于精确性和逻辑正确性。我们要求良好证明的第三个条件是\emph{清晰}:我们希望文字能够解释证明者在每个步骤中完成的工作以及为什么该工作是相关的。换句话说,我们不希望读者在任何时候停下来问:``这句话是什么意思?''或``那是从哪里来的?''或因困惑而产生的类似问题。如果有帮助的话,考虑写一个证明,向你班上的朋友、将要阅读你作业的评分者或智力相当的家庭成员解释它。重读你自己写的证明,并尝试预测可能出现的问题或可能要求你进行的澄清,然后通过重写来解决这些问题。

证明可能因为多种原因失败或不清晰,首当其冲的是,单词和句子可能无法正确解释证明的步骤和动机,这实际上可能是因为单词太多(使读者负担过重而模糊了证明)或因为单词太少(没有给读到者足够的信息)或者因为所选的词语令人困惑(没有正确解释证明)。这些是证明\emph{语言}的问题。

从数学上讲,就清晰性而言,可能会出现许多问题。也许证明撰写者突然引入一个变量,但没有说明它是什么类型的数字(整数、实数等),或者跳过几个算术/代数步骤,或者使用新的符号而没有提前定义它的含义...这些行为在技术上都没有错误或不合逻辑,但它们肯定会给读者带来困惑。你能想到其他原因让证明不明确吗?尝试想出一种基于语言的原因和一种基于数学的原因。

\subsubsection*{糟糕``证明''\#3}

让我们陈述一个关于多项式函数的简单事实,然后查阅关于该事实的``证明''。仔细阅读论证并尝试找出一些不清楚的句子或数学步骤。

\textbf{事实:}考虑多项式函数 $f(x) = x^4-8x^2+16$。对于任意 $x$,该函数都满足 $f(x) \ge 0$。

\begin{proofs}{``证明''}
    无论 $x$ 的值是多少,我们将其代入 $x$ 的函数 $f$ 中,都可以通过对多项式进行因式分解写出该函数的输出值,如下所示:
    \[f(x) = x^4-8x^2+16 = (x-2)^2(x+2)^2\]
    而任意数字 $z$ 一定要么小于 $-2$,要么大于 $2$,要么严格介于 $-2$ 和 $2$ 之间,要么等于其中之一。当 $z > 2$ 时, $z - 2$ 和 $z + 2$ 都大于 $0$,因此 $f(z) > 0$。当 $z < -2$ 时,两项都为负而负数的平方为正,所以 $f(z) > 0$。当 $-2 < z < 2$ 时,类似情况再次发生,当 $x = 2$ 或 $x = -2$ 时,其中一项为 $0$,所以 $f = 0$。因此,我们要证明的必然成立。
\end{proofs}

这个证明有什么可批评的地方呢?首先,它正确吗?精确吗?符合逻辑吗?清楚吗?哪里不清楚?试着找出那些有点不清楚的陈述,无论是语言上的还是数学上的,并尝试适当地修改它们。在不指出个别错误的情况下,下面提供了上述事实的更好、更清晰的论证。

\begin{proof}
    我们首先对函数 $f(x)$ 进行因式分解,将其视为变量为 $x^2$ 的二次函数
    \[f(x) = (x^2)^2-8x^2+16 = (x^2-4)^2\]
    接下来,我们可以因式分解 $x^2-4=(x+2)(x-2)$ 并将原函数重写为
    \[f(x) = \big((x+2)(x-2)\big)^2 = (x+2)^2(x-2)^2\]
    对于任意实数 $x$,都有 $(x+2)^2 \ge 0$ 且 $(x-2)^2 \ge 0$,因为平方结果一定非负。两个非负项的成绩依然非负,所以 对于任意实数 $x$, $f(x) = (x+2)^2(x-2)^2 \ge 0$。
\end{proof}

第一个``证明''和第二个``证明''有什么区别?你重写的证明也像第二个证明吗?

对第一个``证明''的批评之一是它没有完全解释 $-2 < x < 2$ 的情况;相反,它只是说发生了``类似''的事情,并没有实际执行任何细节。这是数学中的常见情况(证明的某些步骤``留给读者''),这是一种简便技巧,有时可以避免繁琐的算术/代数,并使阅读证明更容易、更快速、更愉悦。但是,要谨慎使用该技巧。作为证明撰写者,确保步骤确实有效非常重要,即使你不打算在证明中呈现它们;你应该向读者提供简短的摘要或提示,说明这些步骤实际上是如何运作的。此外,证明撰写者应尽量不要在对证明最终结果至关重要的步骤上使用此技巧。

在上面的特例中,完全跳过了因式分解的实际步骤,并且只是顺便提及了对 $-2 < x < 2$ 情况的分析,但这些都是证明的重要组成部分!从任何维度上来说,这都是一个简短的证明,展示这些步骤并不代表在简洁性或清晰性方面做出了巨大的牺牲。这再次呼应了证明撰写既是科学也是艺术的观点:选择何时将一些细节验证留给读者可能很棘手。 在这种特殊情况下,展示所有步骤很重要。

尽管如此,我们给出的第二个证明要清楚得多。而且,完全摒弃了第一个``证明''中采用的分情况讨论技术!第一个``证明''中的一个情况存在清晰性问题,但我们没有简单地在重写版本中阐述细节,而是选择完全放弃该技术并使用更简短、更直接的证明。这并不是说第一个证明的技术是错误的。如果我们填补第一个``证明''论证中的空白,我们就会得到一个完全正确的证明。然而,该技术中的一些步骤是多余的。请注意,实际上 $-2 < x < 2$ 和 $x > 2$ 的情况在某种意义上是相同的:在这两种情况下,因子都满足 $(x - 2)^2 > 0$ 和 $(x + 2)^2 > 0$。事实上,第一种情况 $x<-2$ 也是如此!所以,当相同的最终观察结果对应于所有三个情况时,为什么要把论证分成三个不同情况呢?这种情况下,最好将它们合二为一(同样利用当 $x = 2$ 或 $x = -2$ 时,其中一个因子为 $0$ 的知识)。重申一遍,使用分开讨论技术当然没错。但是,它只会给证明增加不必要的长度。

我们在上面的段落中提到了术语``情况''和短语``分情况讨论'',但没有恰当定义或解释我们的意思。现在,我们想推迟对这些术语的讨论,直到我们在第 4 章中彻底讨论逻辑。不过,如果你渴望立即解决这个问题,可以跳到第 1.4.4 节并查看``匈牙利朋友''问题,其中包含一些复杂的分情况讨论。
