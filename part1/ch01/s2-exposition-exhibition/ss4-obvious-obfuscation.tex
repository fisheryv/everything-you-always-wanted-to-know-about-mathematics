% !TeX root = ../../../book.tex
\subsection{明显的混淆}

作为这些逻辑概念的预览,我们将在稍后详细研究其数学细节,让我们举一些现实世界中基于语言的例子来说明这些想法。

\subsubsection*{条件陈述}

首先,让我们研究一下\textbf{条件陈述}。数学定理经常采用条件陈述的形式,但这种类型的陈述也经常出现在日常用语中,有时是隐含的(这只会增加混乱)。例如,人们有时会谈论他们将如何处理彩票奖金,比如
\[\text{如果我中了彩票,那么我就买辆新车。}\]
``那么(then)''之后的语句依赖于``如果(if)''相关的语句。当``如果(if)''部分的条件满足时,保证会发生``那么(then)''部分的操作。

条件语句中与``如果(if)''相关的部分称为\textbf{假说(hypothesis)}(更正式的名称为\textbf{先行词(antecedent)})。与``那么(then)''相关的部分称为\textbf{结论(conclusion)}(更正式的名称为\textbf{结果(consequent)})。

有时条件句的结论更加微妙,甚至句子中的动词时态不包含``如果(if)''。以电影《壮志凌云》中的台词为例:
\[\text{这是机密。我可以告诉你,但那样我就不得不杀了你。}\]
这里,第一部分``我可以告诉你''是一个伪装的假设。与实际电影台词具有相同逻辑含义的说法是``\emph{如果我告诉你,我就不得不杀了你}'';然而,这么说没有原台词富有戏剧性和张力。实际上,在条件陈述的结论中常常不包含``那么(then)''一词。在阅读句子时,你甚至可能不知不觉中在脑海里添加该词。下面是 The Barenaked Ladies 乐队的一首歌中的歌词:
\[\text{如果我有 100 万,我们就不必步行去商店了。}\]
\[\text{如果我有 100 万,我们会乘坐豪华轿车,因为它更贵。}\]
这两行都是条件陈述,但都不包含``那么(then)''一词;它被理解为句子的一部分。

将上述示例与以下句子进行比较,看看有什么不同:
\[\text{只有下雨的时候我才带伞。}\]
这里,说话人不愿意在没有正当理由的情况下随身携带雨伞,而是更愿意确保它有用。这句话与下面类似的句子意思相同吗?
\[\text{如果我带着雨伞,那就是下雨了。}\]
在现代语言用法中,条件的概念可能有点模糊。例如,第一句可以解释为有时可能下雨,但说话人忘记带伞。第二句是一个条件陈述的明确断言:看到我撑着伞走来,你一定会推断这是因为下雨了。在数学中,我们把这两个句子联系起来,说它们有相同的逻辑意义。

这引出了短语``仅当(only if)''的含义,以及随后的短语``\textbf{当且仅当(if and only if)}''。考虑以下两句话:
\[\text{如果中了彩票,我会买辆新车。}\]
\[\emph{只有}\text{中了彩票,我才会买辆新车。}\]
第一句说中彩票保证我会买辆新车,而第二句说买新车的行为保证是因为我刚刚中了彩票。如果这两句话都为真,那么``中彩票''和``买新车''这两个事件在某种意义上是等价的,因为其中一个事件的发生\emph{必然保证}另一个事件的发生。

因此,数学定义通常使用``\textbf{当且仅当}''这一短语。例如,我们可以写``一个整数是偶数,当且仅当它能被 $2$ 整除。''这表明知道一个数具有该性质可以称之为``偶数'',知道一个数是偶数可以得出其整除性质。(不过,有时一个定义只会使用\emph{当(if)},而\emph{仅当(only if)}部分没有说明但能被理解。你可能已经注意到,我们在 \ref{sec:section1.1.2} 节中对质数的定义就是这样做的。)

\subsubsection*{创建更多条件陈述}

从一个条件陈述开始,只要稍加修改,便能生成其他三个内容相同但结构不同的条件陈述。继续使用``彩票/汽车''的例子,让我们考虑原句的以下四个版本:

\begin{enumerate}
    \item 如果我中了彩票,那么我就买辆新车。
    \item 如果我买了辆新车,那么我中了彩票。
    \item 如果我中不了彩票,那么我就不会买新车。
    \item 如果我没有买新车,那么我就没中彩票。
\end{enumerate}

这些句子比较起来怎么样?它们中的任何一个都有相同的逻辑含义吗?假设第一个为真,那么所有这些都为真吗?我们认为,在这种情况下,即使第一句是真的,第二句也可能是假的。也许我在工作中得到了大幅加薪,或者继承了一笔钱,所以决定买辆新车。第三句和第四句呢?它们能以某种方式与其他句子联系在一起吗?这个就留给大家自己讨论和探索吧。对我们研究过的其他条件陈述提出同样的问题,看看你的答案是否也不同,这可能会很有趣。

最后一个条件陈述的例子来自脱口秀演员德米特里·马丁(Demetri Martin)的一个笑话。

\begin{quote}
    我走进一家服装店,一位女士走过来对我说:``如果你需要什么,我是吉尔。''我以前从未见过有条件身份的人。``如果我什么都不需要怎么办!你是谁?''
\end{quote}

上面的例子应该会让你体会到现代语言中条件陈述的不精确或微妙,有时需要进一步解释。在数学中,我们希望此类陈述是严格的、定义明确的且无歧义的。我们稍后将在第 \ref{sec:section4.5.3} 节中进一步研究这一点。不过,就目前而言,以计算机算法解释 \verb|if...then| 的严格方式来思考此类陈述可能会有所帮助。当 \verb|if| 部分的条件满足时,子程序被执行,否则被忽略。同样,\verb|while| 循环只是 \verb|if...then| 语句的序列,只是被压缩成一种简洁的形式。

\subsubsection*{量词}

接下来,让我们看一些量词的例子。当存在一个未知变量是从一组可能的值或表示中提取对象时,我们将使用量词。例如,当我们在毕达哥拉斯定理的陈述中量化变量 $a,b,c$ 时,它们是从表示直角三角形边长的实数集中提取的。对于非数学示例,请考虑以下句子:
\[\text{每个人都被某人爱着。}\]
这里有哪些变量?它们是如何量化的?请小心,因为这句话中实际上有两个量化,两个变量各一个。在这两种情况下,变量都代表世界上所有人集合中的成员,第一个变量是全称量化,而第二个变量是存在量化。这听起来可能令人困惑,所以让我们试着用更详细的措辞改写这个句子:
\[\text{对于世界上的每个人} x\text{,都存在另一个人} y \text{,具有} y \text{爱} x \text{的属性。}\]

你看出来这和第一句话的逻辑意义是一样的吗?当然,对于对话来说,这个内容有点过于冗长和精确,但我们在这里展示它是为了向你揭示潜在的变量和量词。量词的关键短语是``\emph{对于所有(for every)}''(全称量化)和``\emph{存在(there exists)}''(存在量化)。

\subsubsection*{量化顺序很重要!}

现在,让我们看一个与上面示例类似的句子:
\[\text{某人被每个人爱着。}\]
这句话和上面那句话很相似;甚至它们的用词都相同!词序变化对句子的逻辑意义有何影响?这里仍然有两个变量和两个量词,一个是全称量词,一个是存在量词,但是这些量词的应用顺序发生了改变。这句话的更详细表述是:
\[\text{存在某人} x \text{具有对于世界上的每个人} y, y \text{都爱} x \text{的属性。}\]

这和第一句话的意思完全不同!第一个似乎可信,但这个就很奇怪。这个例子应该让你明白保持量化顺序是多么重要,这样你才能真正表达你的真实意思。

\subsubsection*{嵌套量词}

下面的例子说明我们的大脑在处理语言中的量词时有时是多么得快速和轻松,即使这种相互联系可能让人难以理解。当量词一个接一个地跟在后面时,我们称之为\emph{嵌套}。

分析和理解这些句子的能力可能取决于句子的上下文及其试图传达的信息。如果信息有意义并且我们相信它,那么它就更容易理解。关于这一现象,我们所知的最好的例子是伟大的总统演说家亚伯拉罕·林肯(Abraham Lincoln)的以下名言:

\begin{quote}
    你可以一直愚弄一些人,有时也可以愚弄所有人,但你不能一直愚弄所有人。
\end{quote}

这里到处都是量词!我们谈论的是所有人的集合,以及某些被愚弄人的集合,并对这些集合进行量化。尝试用几种不同的措辞重写这个句子,看看它是否听起来更``简单''或更简洁。是否存在另一种表达句子的方式,可以删除部分(或全部)量词而不改变含义?

最后,出于个人兴趣和幽默感,我们将引用鲍勃·迪伦(Bob Dylan)《Talkin World War III Blues》中的一句类似的话,这首歌来自鲍勃·迪伦 1963 年发行的专辑《The Freewheelin' Bob Dylan》:

\begin{quote}
    Half of the people can be part right all of the time\\
    Some of the people can be all right part of the time\\
    But all of the people can't be all right all of the time\\
    I think Abraham Lincoln said that
\end{quote}

稍后我们将更详细地讨论这些主题,那时我们将研究它们的数学动机、含义和用途。目前,我们再怎么强调这些问题在撰写证明中的重要性都不为过。把一堆句子串在一起,却不知道它们是如何连接的,这并不是证据,但一系列结构合理的逻辑陈述和含义才是我们真正想要的。