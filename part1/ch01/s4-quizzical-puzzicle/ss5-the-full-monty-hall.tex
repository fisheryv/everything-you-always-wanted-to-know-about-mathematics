% !TeX root = ../../../book.tex
\subsection{三门问题}

\subsubsection*{问题描述}

这个问题仅涉及基础概率与算术,但多年来却让无数聪明人折戟沉沙。1990 年,玛丽莲·沃斯·萨万特 (Marilyn vos Savant) 在《Parade》杂志专栏发表该问题及其解法后,引发了激烈争论,许多人(包括数学家)致信赞同或反对她(正确)的答案。让我们看看你的见解!

\begin{quote}
    假设你正在参加一档游戏节目,面前有三扇门。其中一扇门后是汽车,其余两扇后是山羊。游戏开始前,汽车和山羊的位置已被随机放置在门后。游戏规则如下:你选定一扇门后,该门暂时保持关闭。主持人蒙蒂·霍尔 (Monty Hall) 知晓门后的情况,他会打开其余两扇门中有山羊的一扇。若两扇门后皆为山羊,他会随机开启一扇。蒙蒂·霍尔打开一扇有山羊的门后,会询问你:是坚持最初的选择,还是切换至另一扇关闭的门?试想:你选择了 1 号门,主持人打开了藏有山羊的 3 号门,然后问你:``是否要换到 2 号门?''改变最初的选择对你有利吗?
\end{quote}

当然,我们假设你更希望赢得汽车而非山羊,且力求最大化获胜概率。值得一提的是,该问题得名于电视节目《\emph{Let's Make a Deal}》的主持人蒙蒂·霍尔 (Monty Hall)。

那么你怎么想?试想自己站在聚光灯下,面对所有观众,当蒙蒂·霍尔问你:``要换到另一扇门吗?'' 你会作何选择?

请仔细思考一下,然后再翻页阅读解答。

\clearpage

\subsubsection*{结论:\emph{坚决}切换}

我们直接给出结论——该结论可能令人惊讶:你应当改变最初的选择!推理过程才是棘手且令人困惑的部分,而如何建立正确的解题思路正是该问题长期困扰求解者的原因。

\subsubsection*{错误论证分析}

首先展示一个常见的错误``解答'',该解答声称换门与否无关紧要。假设你与朋友讨论此题时对方提出如下解释,你会如何回应?该论证是否成立?若不同意,你会如何指出其谬误?

\begin{quote}
    当我选定一扇门后,蒙蒂·霍尔展示了另一扇有山羊的门,此时只剩两扇关闭的门。一扇后有山羊,另一扇后有汽车,因此我最初选择的门后有汽车的概率为 $50\%$,另一扇门后有汽车的概率也为 $50\%$。因此,换门不换门没有区别,还不如坚持最初的选择。
\end{quote}

上面的解释能说服你吗?让我们揭示其根本缺陷。解决此问题的关键在于计算两个概率值:坚持原选择获胜的概率,以及换门后获胜的概率。唯有准确计算并比较这两个值,才能彻底解决这个难题。

上述论证将两个概率均视为 $50\%$,但其推理存在根本缺陷。你认为坚持原选择获胜的真实概率是多少?关键在于:展示有山羊的门的行为并不会影响最初选择的门后的物体。请重点理解以下陈述:

\begin{quote}
    因为有三扇门,所以一开始选择正确的门的概率是 $\frac{1}{3}$,看到另一扇门后面有山羊\emph{并不能改变这一事实}。
\end{quote}
这正是上面错误论证的核心症结,也是``解答''本题时最常见的误区。

接下来计算换门后的胜率,并将其与 $\frac{1}{3}$ 进行比较。事实上,有多种方法可以计算此概率。一种简洁的推导是:只要初始选择的门后是山羊(概率 $\frac{2}{3}$),换门必然获胜(赢得汽车)。因为此时两扇未选门中藏着山羊与汽车,主持人必定展示有山羊的门,剩余那扇门后必定是汽车。因此换门策略的胜率为 $\frac{2}{3}$。

\subsubsection*{枚举可能性}

上述解释可能无法令你信服,我们不妨尝试实际枚举门后山羊与汽车的可能排列,并具体分析每种情况下切换选择的结果。首先请注意,门的编号并无实质影响,因为所有选择都是随机的;也就是说,无论汽车停在 1 号门、2 号门还是 3 号门后,我们最初选中汽车的概率始终是 $\frac{1}{3}$。因此可 \textbf{WOLOG}(此缩写意为``不失一般性'')假设汽车位于 1 号门后,山羊分别在 2 号门和 3 号门后。需要强调的是,这是我们自己设定的条件,参赛者并不知晓(否则必然直接选择 1 号门!)。基于此设定,我们考察初始选择的全部三种情况:

\begin{center}
    \begin{tabular}{ c|c|c } 
     1 号门 & 2 号门  & 3 号门 \\ 
     \hline 
     汽车   & 山羊    & 山羊 \\
    \end{tabular}
\end{center}

\begin{center}
    \begin{tabular}{ c|c|c|c } 
     我们的选择 & 主持人展示 & 换门结果 & 不换门结果 \\ 
     \hline 
     1 号门    & 2 号门 \:或\: 3 号门  & 山羊 & 汽车 \\
     2 号门    & 3 号门            & 汽车 & 山羊 \\
     3 号门    & 2 号门            & 汽车 & 山羊 \\
    \end{tabular}
\end{center}

关键发现在于:当我们初始选中有汽车的门时,主持人可以随机开启任意一扇有山羊的门。但无论开启哪扇门,切换选择都将失败,而坚持选择会获胜。这种情况仅占 $\frac{1}{3}$,即初始选中汽车的概率。由于表中所有情况概率均等,可以得出结论:切换策略的胜率为 $\frac{2}{3}$,而坚持策略的胜率仅为 $\frac{1}{3}$。

现在是否感觉问题更清晰了?不妨向亲友提出这个问题并观察他们的反应:有多少人答对?多少人能正确解释?多少人误答``概率相同''?又有多少人此前已经接触过此题?

\subsubsection*{泛化到多门多车情形}

让我们将这个游戏节目问题泛化,分析切换策略是否仍然有效。假设共有 $n$ 扇门和 $m$ 辆汽车,因此有 $n - m$ 只山羊。为进行有效分析,需满足 $m \le n - 2$,原因如下:

\begin{itemize}
    \item 如果 $m = n$,则无论切换与否都必然获胜,无需讨论。
    \item 如果 $m = n-1$,当初始选择有山羊的门时,主持人\emph{无法}展示另一扇有山羊的门,游戏规则无法成立,切换策略也就毫无意义。
\end{itemize}

现在,有了这些变量,游戏的新规则如下:我们随机选择一扇门。主持人从\emph{其余}门中随机打开一扇有山羊的门。此时可选择坚持最初选择或切换至\emph{任意}未打开的门。关键问题是:最优策略是什么?切换是否有利?答案是否取决于 $m$ 和 $n$?

我们将用与原题中第一种方法大致相同的方式来处理这道修改后的问题。由于 $m,n$ 为变量,我们无法枚举所有情况,只能采用逻辑推理来推断坚持和切换的胜率。首要观察与原始问题一致:\emph{坚持策略}的胜率等于初始选中汽车的概率。若初始选中有汽车的门(概率 $\frac{m}{n}$),坚持必胜;若选中有山羊的门(概率 $\frac{n-m}{n}$),坚持必败。故坚持策略的胜率为 $\frac{m}{n}$。

为了计算切换策略的胜率,需要分两种情况讨论。请注意,当 $m \geq 2$ 时,初始选中有汽车的门后切换仍可能获胜。考虑到这一点,我们需要分两种不同情况讨论:(a) 初始选中有山羊的门的情况,(b) 初始选中有汽车的门的情况。每种情况都会给主持人留下不同数量的选择,进而留下不同数量的切换策略和获胜方式,所以此处需要分开处理。

\begin{enumerate}[label=(\alph*)]
    \item \textbf{初始选中有山羊的门}。现在还剩下 $n - m - 1$ 扇有山羊的门,主持人随机选择其中一扇打开。从我们的角度来看,切换给我们留下了 $n-2$ 个选项(我们不能切换到已打开的门和最初选择的门),其中 $m$ 个是汽车。因此,在这种\emph{特定}情况下,切换后获胜的概率为 $\frac{m}{n-2}$。\\
    由于最初有 $n - m$ 只山羊,所以这种情况发生的概率为 $\frac{n-m}{n}$。因此,这种情况对切换后总获胜概率的贡献为
    \[\frac{n-m}{n} \cdot \frac{m}{n-2} = \frac{nm-m^2}{n(n-2)}\]
    (思考一下为什么我们要把这些概率相乘而不是相加?我们要如何将此概率与下一种情况的概率结合起来?)
    \item \textbf{初始选中有汽车的门}。现在还剩下 $n - m$ 扇有山羊的门,主持人随机选择其中一扇打开。从我们的角度来看,切换给我们留下了 $n-2$ 个选项,其中 $m - 1$ 个是汽车。因此,在这种\emph{特定}情况下,切换后获胜的概率为 $\frac{m-1}{n-2}$。\\
    由于最初有 $m$ 辆车,所以这种情况发生的概率为 $\frac{m}{n}$。因此,这种情况对切换后总获胜概率的贡献为
    \[\frac{m}{n} \cdot \frac{m-1}{n-2} = \frac{m^2-m}{n(n-2)}\]
\end{enumerate}

由于这两种情况是独立发生的(即它们不可能同时发生),我们需要将这两个概率相加,从而得到切换策略的总胜率:
\begin{align*}
    \frac{nm-m^2}{n(n-2)} + \frac{m^2-m}{n(n-2)} &= \frac{nm - m^2 + m^2 - m}{n(n-2)} \\
    &= \frac{nm - m}{n(n-2)} \\
    &= \frac{m(n - 1)}{n(n-2)} \\
    &= \frac{m}{n} \cdot \frac{n-1}{n-2}
\end{align*}
将此结果与坚持策略的胜率 $\frac{m}{n}$ 比较。因为 $ n - 1 > n - 2$,所以 $\frac{n-1}{n-2} > 1$,可得不等式:
\[\frac{m}{n} < \frac{m}{n} \cdot \underbrace{\frac{n-1}{n-2}}_{>1}\]
切换策略的胜率\emph{严格大于}(即总是大于)坚持策略的胜率。因此,随机切换到另一扇门总是更优策略!

\subsubsection*{具体应用}

此问题的原始版本对应 $n = 3$ 且 $m = 1$,因此可以将这些值代入来验证结果的正确性。根据推导公式,坚持策略的胜率为 $\frac{1}{3}$,而切换策略的胜率为 $\frac{1(3-1)}{3(1)} = \frac{2}{3}$,与之前的结论完全吻合!

\subsubsection*{泛化:留给你的问题}

当 $m$ 和 $n$ 取其他值时会发生什么?能否使``始终切换''策略显著优于``始终坚持''策略?两种策略的胜率差异最大能达到多少?最小又能缩小至何种程度?是否存在使两者胜率相等的情况?

该问题的另一变体是主持人开启多于一扇有山羊的门。具体设定如下:共有 $n$ 扇门,其中 $m$ 扇后有汽车;玩家首次选择后,主持人从剩余门中随机开启 $p$ 扇有山羊的门;此时玩家可选择坚持初始选择,或切换到未开启的 $n-p-1$ 扇门中的任意一扇。在此情形下最优策略是什么?需对 $m,n,p$ 施加何种约束条件以保证游戏成立?最优策略是始终切换,还是取决于 $p$ 的取值?``始终切换''与``始终坚持''策略的胜率差异范围如何?

\subsubsection*{解题心得}

直觉和快速决策有时能\emph{引导}我们找到正确答案,但务必检查这些仓促判断是否基于合理的逻辑。本题中,``$50/50$''的概率初看似乎成立,但仔细推敲后便发现其论证存在缺陷——关键在于正确解读题目并严格遵循游戏步骤的顺序。概率分析应按照游戏实际发生的流程进行,而非从结果倒推。

概率问题往往极具迷惑性,需格外审慎对待。这也揭示了一个深刻启示:看似简单的问题往往最难攻克。切勿因表述简短或易于理解而轻视问题的复杂性!

关于蒙蒂·霍尔问题及其心理学背景,请参见 \href{http://www.usd.edu/~xtwang/Papers/MontyHallPaper.pdf}{Krauss, Stefan and Wang, X. T. (2003). ``The Psychology of the Monty Hall Problem: Discovering Psychological Mechanisms for Solving a Tenacious Brain Teaser'', \emph{Journal of Experimental Psychology}: General 132(1)}。