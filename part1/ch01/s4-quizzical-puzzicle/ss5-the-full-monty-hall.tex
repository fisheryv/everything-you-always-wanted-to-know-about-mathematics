% !TeX root = ../../../book.tex
\subsection{三门问题}

\subsubsection*{问题描述}

这个问题只涉及基本的概率和算术,但多年不断有非常聪明的人在这道题上折戟沉沙。事实上,1990 年,玛丽莲·沃斯·萨万特 (Marilyn vos Savant) 在《Parade》杂志的专栏中发表了这个问题及其解法,引发了一场争论,许多人(包括数学家)写信表示赞同或反对她(应该说是正确的)答案。让我们看看你的想法!

\begin{quote}
    假设你正在参加一个游戏节目,有三扇门供你选择。其中一扇门后面是汽车;其余的后面是山羊。游戏前,汽车和山羊被随机放置在门后。游戏节目规则如下:选择一扇门后,该门暂时保持关闭状态。游戏节目主持人蒙蒂·霍尔(Monty Hall)知道门后有什么,他会打开剩下两扇门中的一扇,并且打开的门后面一定是山羊。如果剩下的两扇门后面都是山羊,他会随机打开一扇。蒙蒂·霍尔打开一扇有山羊的门后,他会问你是继续选择最初选择的门还是切换到剩下的另一扇门。想象一下,加入你选择了 1 号门,主持人打开了 3 号装有山羊的门。然后他问你``你想换到 2 号门吗?'' 改变你最初的选择对你有利吗?
\end{quote}

当然,我们假设你更愿意赢得一辆汽车而不是一只山羊,并且你希望最大化赢得那辆车的机会。 另外,值得一提的是,这个问题的名字来源于电视游戏节目《\emph{Let's Make a Deal}》的主持人蒙蒂·霍尔(Monty Hall)。

所以你怎么看?想象一下你站在电视观众面前的舞台上,当蒙蒂·霍尔问你:``你想换到另一扇门吗?'' 你会怎么选择?

请仔细思考一下,然后再翻页阅读解答。

\clearpage

\subsubsection*{解答:始终切换}

我们先直接给出答案,因为这可能会让你大吃一惊:你绝对应该改变最初的选择!推理并得到这个答案是棘手和令人困惑的部分,而建立正确的解释该题的方法是困扰求解者这么长时间的部分原因。

\subsubsection*{分析错误论证}

首先让我们向你演示一个错误的``解答'',该解答声称切不切换无关紧要。想象一下你和你的朋友听到了这个问题,他/她给了你这么一个解释。你会如何回应?你同意吗?为什么?如果不是,你会如何指出他们的错误?他们的解释有什么问题?

\begin{quote}
    当我选择了一扇门,蒙蒂·霍尔向我展示了另一扇后面有山羊的门,那么就只有两扇门没有打开。其中一扇是山羊,另一扇是汽车,所以汽车在我选择的门后面的可能性是 $50/50$,汽车在另一扇门后面的可能性也是 $50/50$。所以,换不换都无所谓,还不如坚持最初的选择。
\end{quote}

上面的解释说服你了吗?让我们试着找出这个论证的问题。为了解决这个难题,我们需要解决的主要问题是计算出两个数字:坚持我们最初的选择赢得汽车的概率,以及切换到另一扇门后赢得汽车的概率。我们需要识别这两个值并进行比较;只有这样我们才能明确地解决这个难题。

上面的论证似乎通过说这两个概率都是 $50\%$ 来解决这问题,但反对者如何解释这种情况存在问题。你认为坚持最初的选择赢得汽车的概率有多大?本质上讲,这相当于蒙蒂·霍尔没有向我们展示另一扇后面有山羊的门。如果我们坚持对门的最初选择,我们甚至可能不会看到另一扇后面有山羊的门,因为这不会影响我们最初选择的门后面的物体。让我们重申这个想法以强调其重要性:

\begin{quote}
    因为有三扇门,所以一开始选择正确的门的概率是 $\frac{1}{3}$,看到另一扇门后面有山羊\emph{并不能改变这一事实}。
\end{quote}
这就是上面论证的问题,事实上,这是``解答''这个问题时最常见的错误之一。

下一步是计算切换后赢得汽车的概率,并将其与 $\frac{1}{3}$ 进行比较。事实上,有多种方法可以实现这一目标。一种简洁的方法是,每当我们初次选择的门后碰巧是山羊时,切换就会成功(赢得汽车)。这种情况下,两扇未选择的门按某种顺序隐藏了山羊和汽车,游戏节目主持人被迫向我们展示山羊;因此,汽车隐藏在剩下的门后面,切换就会获胜。由于我们的初次选择在 $\frac{2}{3}$ 的情况下会选到有山羊的门,因此我们得出结论,$\frac{2}{3}$ 的情况下切换会赢得汽车。

\subsubsection*{枚举可能性}

上面的解释可能无法令你满意,所以让我们尝试实际枚举(明确计算)门后山羊和汽车的可能排列,并写下如果我们在每种情况下进行切换会发生什么。首先要注意的是,门的编号是无关紧要的,因为所有选择都是随机做出的;也就是说,无论汽车停在印有``1 号''的门后,还是``2 号''或``3 号'',结果都是一样的:我们仍然有 $\frac{1}{3}$ 的概率选到那扇有车的门。 因此,我们可以假设\textbf{WOLOG}(记住这个缩写的意思是``不失一般性'')汽车位于 1 号门后面,山羊站在 2 号门和 3 号门后面。当然,这是我们对问题的强加,我们不能说玩家知道这一点(否则他/她每次都会选择1 号门!)。按照这种安排,让我们检验一开始可以做出的所有 $3$ 个选择,看看在每种情况下切换或坚持会有什么效果:

\begin{center}
    \begin{tabular}{ c|c|c } 
     1 号门 & 2 号门  & 3 号门 \\ 
     \hline 
     汽车   & 山羊    & 山羊 \\
    \end{tabular}
\end{center}

\begin{center}
    \begin{tabular}{ c|c|c|c } 
     我们的选择 & 主持人展示 & 换门结果 & 不换门结果 \\ 
     \hline 
     1 号门    & 2 号门 \:或\: 3 号门  & 山羊 & 汽车 \\
     2 号门    & 3 号门            & 汽车 & 山羊 \\
     3 号门    & 2 号门            & 汽车 & 山羊 \\
    \end{tabular}
\end{center}

一个重要的观察是,当我们最初选中有汽车的门时,主持人可以选择剩余的门中的任意一个展示山羊,并且他会\emph{随机}做出选择。然而,无论选择展示哪扇门,我们都会因为切换而失败,因为坚持而获胜。尽管如此,这种情况只占 $\frac{1}{3}$,即我们最初选中了后面有汽车的门。由于上表中每一行的可能性相同,我们可以得出结论,$\frac{2}{3}$ 的情况下我们会因为切换而获胜,而 $\frac{1}{3}$ 的情况下我们会因为坚持而获胜。

现在这个问题是不是更合理了?尝试向你的朋友和家人提出这个问题,并测试他们的反应。有多少人给出了正确答案?有多少人能正确给出解释?有多少人错误地说``无所谓''?有多少人之前已经听说过这个问题?

\subsubsection*{泛化到许多门许多车}

让我们泛化一下这个游戏节目的情况,并尝试证明切换是否仍是一个好主意。具体来说,假设总共有 $n$ 扇门和 $m$ 辆汽车,因此有 $n - m$ 只山羊。为了分析这一点,我们需要指定 $m \le n - 2$。思考为什么这是必要的:

\begin{itemize}
    \item 如果 $m = n$,那么无论是否切换,我们总是会赢。因此,这种情况没有什么可以证明的。
    \item 如果 $m = n-1$,那么如果我们一开始碰巧选中后面有山羊的门,主持人就\emph{无法}向我们展示另一扇有山羊的门。因此,游戏规则遭到破坏,切换不切换也就毫无意义。
\end{itemize}

现在,有了这些变量,游戏的新规则如下:我们选择 $n$ 扇门中的一扇。主持人从所有\emph{其他}门中找出藏有山羊的门,并随机打开其中一扇门。然后,我们可以选择坚持原来的选择或切换到\emph{任意}其他门。现在的策略是什么?我们应该切换吗?我们应该坚持吗?答案完全取决于 $m$ 和 $n$ 吗?为什么?

我们将用与原题中第一种方法大致相同的方式来处理这个修改后的问题。我们不可能枚举这个版本中的所有情况,因为 $m$ 和 $n$ 是未知变量。相反,我们需要应用逻辑推理来推断坚持和切换的获胜概率。第一个关键观察与我们之前所做的完全相同:\emph{坚持}获胜的概率正是首次选中藏有汽车的门的概率。当我们首次选中一扇后面有汽车的门时,无论主持人打开其他哪个门,坚持我们当前的选择都会胜利。此外,当我们一开始选中了藏有山羊的门时,坚持就会造成损失。因此,坚持最初选择并获胜的唯一方法是从 $n$ 扇门中选中后面有汽车的 $m$ 扇门中的一扇。这个概率正好是 $\frac{m}{n}$。

为了确定切换后的获胜概率,我们需要仔细考虑每个相关步骤的概率。请注意,由于 $m$ 可能 $m \ge 2$,因此我们可能一开始选中了一扇有汽车的门,然后改变我们的选择,却仍然获胜。考虑到这一点,我们这里需要分两种不同情况讨论:(a) 当我们一开始选中带有山羊的门时会发生什么,(b)当我们一开始选中带有汽车的门时会发生什么。每种情况都会给主持人留下不同数量的选择,随后给我们留下不同数量的切换和获胜的方式,所以我们应该分开处理。

\begin{enumerate}[label=(\alph*)]
    \item 假设我们一开始选中了一扇带有山羊的门。现在还剩下 $n - m - 1$ 扇藏有山羊的门,主持人随机选择其中一扇打开。从我们的角度来看,切换给我们留下了 $n-2$ 个选项(我们不能切换到打开的门或我们的最初选择),其中 $m$ 个是汽车。因此,在这种\emph{特定}情况下,切换后获胜的概率为 $\frac{m}{n-2}$。\\
    由于最初有 $n - m$ 只山羊,所以这种情况发生的概率为 $\frac{n-m}{n}$。因此,这种情况对切换后总获胜概率的贡献为
    \[\frac{n-m}{n} \cdot \frac{m}{n-2} = \frac{nm-m^2}{n(n-2)}\]
    (思考一下为什么我们要把这些概率相乘。为什么我们需要这样做?为什么我们不把它们加在一起?我们要如何将此概率与下一种情况相关的概率结合起来?)
    \item 接下来,假设我们一开始选中了一扇带有车的门。现在还剩下 $n - m$ 扇藏有山羊的门,主持人随机选择其中一扇打开。 从我们的角度来看,切换给我们留下了 $n-2$ 个选项,其中 $m - 1$ 个是汽车。因此,在这种\emph{特定}情况下,切换后获胜的概率为 $\frac{m-1}{n-2}$。\\
    由于最初有 $m$ 辆车,所以这种情况发生的概率为 $\frac{m}{n}$。因此,这种情况对切换后总获胜机会的贡献为
    \[\frac{m}{n} \cdot \frac{m-1}{n-2} = \frac{m^2-m}{n(n-2)}\]
\end{enumerate}

由于这两种情况是单独发生的(即它们不可能同时发生),我们需要将这些概率加在一起。从而得到从最初选择切换到另一扇随机门后赢得汽车的总概率:

\begin{align*}
    \frac{nm-m^2}{n(n-2)} + \frac{m^2-m}{n(n-2)} &= \frac{nm - m^2 + m^2 - m}{n(n-2)} \\
    &= \frac{nm - m}{n(n-2)} \\
    &= \frac{m(n - 1)}{n(n-2)} \\
    &= \frac{m}{n} \cdot \frac{n-1}{n-2}
\end{align*}

我们将结果写成这种形式的分数是有原因的。我们想将这个概率与坚持最初选择的获胜概率 $\frac{m}{n}$ 进行比较。我们看到,切换后获胜概率实际上是坚持最初选择获胜概率的倍数,并且因子 $\frac{n-1}{n-2} > 1$,因为 $ n - 1 > n - 2$。写成不等式:
\[\frac{m}{n} < \frac{m}{n} \cdot \underbrace{\frac{n-1}{n-2}}_{>1}\]
因此,切换后的获胜概率\emph{严格大于}(即总是大于)坚持的获胜概率。我们永远应该随机切换到另一扇门!

\subsubsection*{具体应用}

这个问题的原始版本是 $n = 3$ 且 $m = 1$ 的特定情况,因此我们可以代入检查我们的结果是否正确。我们推导出来的公式告诉我们,坚持获胜的概率是 $\frac{1}{3}$,而切换后获胜的概率是 $\frac{1(3-1)}{3(1)} = \frac{2}{3}$,跟我们之前的发现完美一致!

\subsubsection*{泛化:留给你的问题}

如果 $m$ 和 $n$ 取其他值会发生什么?你能否让``始终切换''策略明显优于``始终坚持''策略?也就是说,两种策略的获胜概率之间可以有多大差异?我们能把它做得小到什么程度?有可能让他们相等吗?

该问题的另一个变种版本是主持人打开多于一扇门,露出多只山羊。具体来说,假设总共有 $n$ 扇门,其中 $m$ 扇有汽车,并且在你初次选择后,主持人从其他门种随机打开 $p$ 扇有山羊的门,之后你可以选择切换到剩余 $n-p-1$ 扇门中的任意一扇门,或者坚持你的最初选择。这个游戏中最好的策略是什么? 你需要对 $m,n,p$ 施加什么样的条件才能确保游戏可以进行?你应该总是切换,还是取决于$p$?对于``始终切换''和``始终坚持''策略,我们的获胜几率差异有多大/多小?

\subsubsection*{解题心得}

直觉和快速决策有时有助于\emph{引导}我们找到正确答案,但检查这些仓促的判断以确保它们基于合情合理的论点始终很重要。在这个题目中,说概率是``$50/50$''一开始可能是有道理的,但在进一步仔细思考并重新评估情况后,我们意识到这个论点存在缺陷。具体来说,该缺陷与正确解释问题并按照适当的顺序遵循游戏节目的步骤有关。最好按照游戏进行时发生的顺序来评估概率,而不是从结束出发向后推。

一般来说,涉及概率的问题都非常棘手,需要仔细分析,因此记住这一点很重要。这里还有一个更大的教训是,往往最简单的问题是最难解决的。永远不要因为语句简短或易于理解而误以为问题很容易解决!

有关蒙蒂·霍尔问题及其相关心理学的更多信息,请点击\href{http://www.usd.edu/~xtwang/Papers/MontyHallPaper.pdf}{此连接}查看以下论文: Krauss, Stefan and Wang, X. T. (2003). ``The Psychology of the Monty Hall Problem: Discovering Psychological Mechanisms for Solving a Tenacious Brain Teaser'', \emph{Journal of Experimental Psychology}: General 132(1)。