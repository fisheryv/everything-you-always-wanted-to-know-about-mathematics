% !TeX root = ../../../book.tex
\subsection{消失的钞票}

\subsubsection*{问题描述}

这个经典的谜题包含在一个关于分摊酒店房间费用的故事中:

\begin{quote}
    三个朋友自驾旅行,深夜入住酒店。值班店员说当晚只剩一间空房,三人合住需付 $30$ 美元。他们疲惫不堪,便同意合住,每人各付 $10$ 美元预付款。店员道谢后递过钥匙,三人随即去车上取行李。此时,前来换班的另一名店员发现前一名店员多收了房费:实际只需 $25$ 美元。于是他从收银机取出一张 $5$ 美元钞票,递给值班服务生说:``把钱退给 29 号房的客人,我们多收费了。''服务生点头后走向三人房间。客人开门时对退款又惊又喜。为公平起见,一人将钞票换成五张 $1$ 美元纸币,每人取回 $1$ 美元,剩余 $2$ 美元作为小费给了服务生。服务生道谢后离开。

    现在,三人每人实际支付 $9$ 美元房费,加上 $2$ 美元小费,总计 $29$ 美元。但他们最初支付了 $30$ 美元……消失的 $1$ 美元去哪儿了?!
\end{quote}

请仔细思考一下,然后再翻页阅读解答。

\clearpage

\subsubsection*{解答:追踪资金流向}

你搞明白了吗?其实没有任何东西凭空``消失''。这个谜题的目的就是迷惑读者,误导他们去寻找不存在的事物。题目中的数字经过精心设计,使``消失的金额''仅为 $1$ 美元,让读者误以为发生了什么神秘事件。但通过细致的逻辑分析,你会发现最终提问本身并不合理:它利用了读者对情境的误解,使其忽视逻辑推理。若数字差异变得更大,人们便不会执着于寻找``消失的钞票''。

首先,让我们分析一下在这个特殊案例中到底发生了什么。关键在于厘清资金的实际流向。我们可将参与者分为两组:朋友群体(记为 $F$)和酒店员工群体(包括店员与服务生,记为 $H$)。现在,让我们重现资金转移步骤:

\begin{enumerate}
    \item $F$ 支付 $H \quad 30$ 美元(原始房费)
    \item $H$ 退还 $F \quad \enspace 5$ 美元(多收房费退款)
    \item $F$ 支付 $H \quad \enspace 2$ 美元(服务生小费)
    \item 净变化:$F$ 向 $H$ 支付 $30 \text{\ 美元} -5 \text{\ 美元} + 2 \text{\ 美元} = 27 \text{\ 美元}$
\end{enumerate}

这样就清晰了:退款 $5$ 美元使实际房费变为 $25$ 美元,三人每人付了 $9$ 美元,再加上服务生的小费,共计 $27$ 美元。谜题错误地将小费与房费相加,但 $27$ 美元已包含全部支出。通过追踪群体间的资金流动,我们能准确还原交易过程。

\subsubsection*{泛化:改变数字}

让我们应用上述方法来修改问题,通过改变数字消除对``消失的钞票''的情感依赖,同时放大金额差异。首先定义变量表示各步骤的金额。与其``测试''特定数值,不如引入变量实现``一次性全面验证''。

设 $3n$ 表示三人首次支付的房费总额($n$ 为每人支付金额)。退款金额设为 $3r + 2$,其中 $r$ 为每人实际退款,$2$ 为给服务生的小费。下面用变量重述该问题:

\begin{quote}
    三个朋友自驾旅行,深夜入住酒店。值班店员说当晚只剩一间空房,三人合住需付 $3n$ 美元。他们疲惫不堪,便同意合住,每人各付 $n$ 美元预付款。店员道谢后递过钥匙,三人随即去车上取行李。此时,前来换班的另一名店员发现前一名店员多收了房费:实际只需 $3n - (3r + 2)$ 美元。于是他从收银机取出一张 $3r + 2$ 美元钞票,递给值班服务生说:``把钱退给 29 号房的客人,我们多收费了。''服务生点头后走向三人房间。客人开门时对退款又惊又喜。为公平起见,三人每人取回 $r$ 美元,剩余 $2$ 美元作为小费给了服务生。服务生道谢后离开。

    现在,三人每人实际支付 $n-r$ 美元房费,加上 $2$ 美元小费,总计 $3(n-r)+2$ 美元。但他们最初支付了 $3n$ 美元……消失的 $3n - [3(n - r) + 2]=3r - 2$ 美元去哪儿了?!
\end{quote}

现在问题是否更清晰了?正如我们之前解释的那样,差异源于原文将小费加入退款后的房费,再与初始 $3n$ 美元房费进行比较。正确的比较应该是,实际房费支出 $3(n-r) = 3n - 3r$,与退费后房费与小费之和 $[3n-(3r+2)] + 2 = 3n-3r$ 进行比较。两者完全相等!

\subsubsection*{泛化:留给你的问题}

谜题的原始陈述中,$n=10, r=1$,因此``消失的钞票''神奇地变为 $3r-2=1$。如果我们选择更大的数值——例如 $n=100, r=10$——那么 $300$ 美元的房间实际花费 $268$ 美元,服务生会退还三人 $32$ 美元,他们每人拿回 $10$ 美元,服务生保留 $2$ 美元,差额就变成了 $28$ 美元。真的会有人相信 $28$ 美元在此交易中凭空消失了吗?如果我们使用更大的 $n$ 和 $r$ 值呢?你能把差额扩大到多大?或缩小到多小?给定所需的差额(以美元为单位),你能找到满足条件的 $n$ 和 $r$ 的值吗?有多少种方法可以做到这一点?

\subsubsection*{解题心得}

逻辑和理性思维在解决难题时至关重要,因为情绪容易误导人。如果我们最初将这个谜题表述为``消失的 $28$ 美元'',你会有同样的反应吗?在试图回溯并弄清真相之前,你是否会感到短暂的困惑?
