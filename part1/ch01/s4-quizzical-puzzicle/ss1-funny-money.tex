% !TeX root = ../../../book.tex
\subsection{消失的钞票}

\subsubsection*{问题描述}

这个经典的谜题包含在一个关于分摊酒店房间费用的故事中:

\begin{quote}
    三个朋友正在进行公路旅行,一天深夜他们在一家酒店停下来寻找房间休息一下。值班人员说,当晚只有一间空房,三人挤在一起要$30$ 美元。三人实在太困了,于是他们同意挤一间房间,于是每人在柜台上放了一张 $10$ 美元钞票作为预付款。店员向他们表示感谢,递给他们钥匙,然后三人就去车上拿行李了。这时,前来换班的另一名店员,发现前一名店员犯了一个错误,多收了三人的房费:应该只收 $25$ 美元。于是他从收银机里拿出一张 $5$ 美元钞票,递给值班的服务生,说:``把钱送给 29 号房间的客人,多收他们钱了。''侍者点点头,就往三人的房间走去。当三人开门时,他们对退款又惊又喜。为了公平起见,一人将钱换成五张 $1$ 美元钞票,然后每人拿走 $1$ 美元,将剩余的 $2$ 美元给服务生作小费。服务生亲切地感谢了他们,然后回去工作了。

    现在,三人每人付了 $9$ 美元房费,再加上 $2$ 美元小费,总共 $29$ 美元。但他们最初给了店员 $30$ 美元...消失的 $1$ 美元去哪儿了?!
\end{quote}

请仔细思考一下,然后再翻页阅读解答。

\clearpage

\subsubsection*{解答:仔细追踪资金}

你搞明白了吗?你有没有意识到,其实并没有东西凭空``消失''?这个谜题的目的就是迷惑并误导读者去寻找并不存在的东西。题目所涉及的数字都经过精心挑选,使得``消失的总金额''只有 $1$ 美元这么小,以至于读者会误以为发生了什么神秘的事情,但对事件进行仔细地、逻辑地分析后,你会意识到最后的问题并不是一个真正公平的问题; 不是一个真正公平的问题;这道题利用了读者对情况的误解,并试图忽视推理,认识到这点是解决这道题的关键。当数字发生巨大变化,最终的差异变得更大时,读者就不再有激情去寻找``消失的钞票''了。

首先,让我们分析一下在这个特殊案例中实际发生了什么。关键是仔细追踪资金的实际去向。忘记其中涉及的个体,将其划分为两个不同的群体会有所帮助:一群朋友,我们称之为 $F$,以及酒店服务生/店员,我们称之为 $H$。现在,让我们重现一下故事中的步骤并描述每一步资金的来源和去向:

\begin{enumerate}
    \item $F$ 到达并给 $H \quad 30$ 美元(房间的原始费用)
    \item $H$ 退还 $F \quad 5$ 美元(多收房费退款)
    \item $F$ 给 $H \quad 2$ 美元(给服务生小费)
    \item 净变化:$F$ 给了 $H \quad 30 \text{美元} -5 \text{美元} + 2 \text{美元} = 27 \text{美元}$
\end{enumerate}

这样才更合理,不是吗?退款是 $5$ 美元,所以房间实际花费 $25$ 美元,而三人每人付了 $9$ 美元,再加上给服务生的小费,这是不合理的。他们三人付出的 $27$ 美元包括了小费。故事后面的问题暗示我们应该在费用中加上小费,实际上,这确实是他们付出的一部分。通过将朋友与服务生/店员聚集在一起,我们可以切实追踪到钱是如何转手的。

\subsubsection*{泛化:改变数字}

让我们用上面提到的方法来改变这个问题;具体来说,让我们试着改变问题中的数字,来消除对``消失的钞票''的情感依恋,并使差异更大。首先,我们要定义一些变量来表示上述每个步骤中使用的美元金额。我们可以尝试通过``测试''这些美元金额的特定值来解决这个问题,看看会发生什么,而通过引入变量并稍后用代入特定值,从本质上``一次尝试所有事情''会更有效。

对于 $n$ 的某个特定值,我们令 $3n$ 代表酒店房间的初始费用(三人第一次到达时支付的金额)。我们这么设是因为我们希望三人平摊费用。接下来,我们要定义一个变量表示他们收到的退款。已知他们希望平分这笔钱,并留出一些剩余的钱给服务员做小费,假设退款金额用 $3r + 2$ 这种形式表示。变量 $r$ 代表每个人各自从酒店收到的退款,$2$ 代表给服务员的小费。现在,让我们用这些变量而不是原始值来重述这个问题。

\begin{quote}
    三个朋友正在进行公路旅行,一天深夜他们在一家酒店停下来寻找房间休息一下。值班人员说,当晚只有一间空房,三人挤在一起要$3n$ 美元。三人实在太困了,于是他们同意挤一间房间,于是每人在柜台上放了一张 $n$ 美元钞票作为预付款。店员向他们表示感谢,递给他们钥匙,然后三人就去车上拿行李了。这时,前来换班的另一名店员,发现前一名店员犯了一个错误,多收了三人的房费:应该只收 $3n - (3r + 2)$ 美元。于是他从收银机里拿出一张 $3r + 2$ 美元钞票,递给值班的服务生,说:``把钱送给 29 号房间的客人,多收他们钱了。''侍者点点头,就往三人的房间走去。当三人开门时,他们对退款又惊又喜。为了公平起见,三人每人拿走 $r$ 美元,将剩余的 $2$ 美元给服务生作小费。服务生亲切地感谢了他们,然后回去工作了。

    现在,三人每人付了 $n-r$ 美元房费,再加上 $2$ 美元小费,总共 $3(n-r)+2$ 美元。但他们最初给了店员 $3n$ 美元...消失的 $3n - [3(n - r) + 2]=3r - 2$ 美元去哪儿了?!
\end{quote}

现在你看到发生了什么吗?正如我们之前解释的那样,之所以会出现这种差异,是因为问题中将 $2$ 美元小费添加到退还的房费中,并将其与初始的 $3n$ 美元费用进行比较。实际应该是退费后的实际房费支出,即 $3(n-r) = 3n - 3r$,与退费后的房费和小费之和,即 $3n-(3r+2)+2 = 3n-3r$,进行比较。二者没有任何差异!

\subsubsection*{泛化:留给你的问题}

谜题的原始陈述中,$n=10, r=1$,因此``消失的钞票''神奇地变成了 $3r-2=1$。如果我们选择了更大的数值 --- 比如 $n=100, r=10$ --- 后来 $300$ 美元的房间实际应该花 $268$ 美元,服务生会给三人带来 $32$ 美元,他们每人拿回 $10$ 美元,服务生保留$2$ 美元,差额就变成了 $28$ 美元。真的会有人相信 $28$ 美元在这些交易中消失了吗?如果我们使用更大的 $n$ 值和 $r$ 值呢?你能把差异扩大到多大?缩小到多小?给定所需的差异(以美元为单位),你能找到满足条件的 $n$ 和 $r$ 的值吗?有多少种方法可以做到这一点?

\subsubsection*{解题心得}

逻辑和理性思维在解决难题时至关重要,因为有时很容易被情绪误导。如果我们最初将这个谜题表述为``消失的 $28$ 美元'',你会有同样的反应吗?在试图回溯并发现到底发生了什么之前,你是否会短暂地感到困惑?
