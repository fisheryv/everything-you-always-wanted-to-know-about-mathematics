% !TeX root = ../../book.tex
\section{大罗洞观}\label{sec:section1.2}

\subsection{简单符号}

\subsubsection*{数学是门语言}

尽管看上去全是符号(那些写得密密麻麻的教科书所呈现的),但数学不仅仅是我们纸上所用符号的集合。英语基于一组固定的符号(字母表中的 26 个字母加上常见的标点符号,如句点、逗号和括号),但我们可以以特定的方式将这些符号组合在一起,并遵循一定的标准和约定,创作出有意义的单词、短语、句子、段落等;从本质上讲,英语和任何其他语言一样,通过符号集合以及组织这些符号的规则集合,来传达含义。同样的概念也适用于\emph{数学语言}:有一组符号和一组作用于这些符号的规则。

一个区别是,我们在数学中使用的符号集合可能相当大,具体取决于当前讨论的数学分支。数学结构多样性的一个重要部分是我们总是可以创建和定义要使用的新符号。通常,这样做只是为了使内容更简短、更易于阅读。

数学与其他语言之间的另一个主要区别是,我们仔细选择如何\emph{定义}我们的单词及其表达的概念。通常,数学家的大多数争论都围绕定义展开。这可能会令你感到惊讶;似乎数学家就证明和猜想进行辩论才更合理,甚至数学家居然会辩论都是一个新奇的想法!为新发现的概念选择正确的定义和术语是数学发现和阐述的重要组成部分,因为这有助于发现者/发明者向其他感兴趣的人解释他/她的想法。(没有这个过程,数学就不会进步,只是一群孤立的人试图自己发现真理。)

口语的情况与此类似,但似乎没有那么极端。例如,如果你对你的朋友说,``我饿了'',或者``我感觉有点饿了'',或者``天哪,我饿死了'',他们听到的基本上是相同的信息,并给出大致相同的回应。然而,在数学中,我们的定义要精确得多,并且不包含口语允许的细微差别。当然,这两种哲学各有优缺点,但在数学中,我们尽可能追求精确,因此我们希望我们的定义准确且稳定。尽管如此,我们可以掌控这些定义是什么!这就是为什么关于定义的争论在数学世界中如此普遍:为手头的概念选择正确的定义可以使未来使用这些概念的工作变得更加容易和方便。

\subsubsection*{恰当选择定义}

作为一个具体的例子,让我们回到上一节中看到的质数的定义\ref{def:prime}。它说的是:

\begin{definition}
    如果大于 $1$ 的正整数 $p$ 其正因子只有 $1$ 和 $p$,则称 $p$ 为\dotuline{质数}。非质正整数称为\dotuline{合数}。
\end{definition}

这个定义似乎没有什么问题,不是吗?也许你会用不同的措辞或更简洁的表达或使用不同的可变字母或其他,但最终的信息是相同的:质数是具有特定属性的某种类型的数字。而你选择写出该特定类型的数字是什么(大于 $1$ 的正整数)以及该属性是什么(没有除 $1$ 和它本身以外的正因子),你将获得等效的定义。

不过,这个定义背后存在一些微妙的问题:为什么它是那种特定类型的数字?为什么我们如此关心这个特殊的属性 --- 只能被 $1$ 和它本身整除?如果定义略有不同怎么办?事情真的会有那么大的改变吗?我们将用另一个问题来解决这些问题:你如何看待以下质数的替代定义?

\begin{definition}\label{def:prime2}
    如果小于 $-1$ 或大于 $1$ 的整数 $p$ 其正因子只有 $1$ 和 $p$,则称 $p$ 为\textbf{质数}。
\end{definition}

你注意到细微的差别了吗?所有符合之前``质数''定义的数字仍然符合这个定义,但现在负数也适用!具体来说,给定任意数字 $p$ 在旧定义下是质数,$-p$ 现在在新定义下也是质数。这是一个合理的想法吗?负质数有什么问题?

质数的第三个定义怎么样?

\begin{definition}\label{def:prime3}
    如果正整数 $p$ 的正因子只有 $1$ 和 $p$,则称 $p$ 为\textbf{质数}。
\end{definition}

(请记住,按照惯例,$0$ 既不是正数也不是负数。)现在,负数会超出范围,但 $1$ 符合此定义。这合理吗? $1$ 的唯一正因子是 $1$ 和...它本身,对吗?

这就是可能引发争论的地方:也许你不介意让 $1$ 成为质数,但你的朋友会强烈反对。好吧,如果没有确凿的理由,就没有办法说你们中任何一个都是错的,真的;你只是对术语做了不同的选择,它们都没有改变 $1$ 的唯一正因子是 $1$ 和它本身这一固有属性。类似地,请考虑一下:无论你称它们为凉鞋、拖鞋还是人字拖,事实仍然是这些类型的鞋子适合在海滩上穿。

然而,考虑到历史的后见之明和新的愿望,通常一个特定的定义被认为更加合适。未来,我们将研究质数分解,这是一种将每个(正)整数写为质数乘积的方法。例如,$15 = 3 \cdot 5$, $12 = 2 \cdot 2 \cdot 3 = 2^2 \cdot 3$ 和 $142857 = 33 \cdot 11 \cdot 13 \cdot 37$ 都是质因数分解。

这些因数分解也有一个特殊的性质:一般来说,正整数的质因数分解是\textbf{唯一的}!也就是说,只有一种方法可以将正整数写为质数的乘积(因为我们将因子的不同排序视为同一事物,所以 $105 = 3 \cdot 5 \cdot 7$ 和 $105 = 7 \cdot 3 \cdot 5$ 是相同的因数分解)。我们将使用上面给出的第一个定义严格证明这一点。如果我们使用第二个定义或第三个定义会怎么样?这种唯一性还存在吗? 为什么这种唯一性如此重要?最终,结论是,定义应该由逻辑和实用性驱动,并且这可能会随着时间的推移而改变并引发一些争论。

\subsubsection*{数学家的学习模式}

建立清晰准确的定义的另一个好处是你可以像思想者一样获取知识和理解;人类学习的一个主要方面涉及通过日常经验识别模式,接着将想法、概念、词语、事件与这些模式联系起来。然后,人们可以使用这些模式来预测抽象的想法、概念和事件并对其进行理论化。

例如,研究表明,人类婴儿最初缺乏\emph{物体存继性}概念,随着时间的推移逐渐发展起来。如果你给孩子看一个他们喜欢的彩色玩具,然后把它藏在纸箱下面,孩子不太明白这个玩具仍然存在,只是看不见了。他/她会表现得好像该物体不再存在一样。然而,在某些时候,我们知道这不是真的,我们视野之外的物体仍然存在。这究竟是如何发生的?也许是我们见识到许多此类事件的模式,其中一个物体``消失'',然后我们又找到它。

更好的例子可以在自然科学中找到,它们说明了模式识别和抽象思维的另一个方面,这是极其重要的,特别是在数学和科学领域。我们可以想象,尼安德特人不知何故知道,每当他们拿起一块岩石并将其保持在一定距离,然后松开时,岩石就会掉到地上。这种情况可能一次又一次地发生,所以他们``明白''这种现象是自然的必然产物。在发生足够多的事件之后,人们很可能明白这种情况总会发生,或者至少,任何没有发生的情况都会引起极大的困惑和恐惧。(正是这种情绪反应可能有助于解释火山爆发等罕见但强烈的事件如何导致古代文明将此类事件归咎于``神之愤怒'')。

对事件的观察并没有使史前人类进一步理解\emph{为什么}岩石总是会掉落到地面,或者能够\emph{解释}为什么它每次都必然发生。几千年后,人们才开始思考这种现象为何发生以及如何发生,更长时间之后,艾萨克·牛顿(Isaac Newton)最终提出了一个试图解释重力行为的模型(最终为此类现象命名)。有人说,即使是现在,我们仍然没有弄清楚它到底是如何运行的。(如果你好奇的话,可以上网搜索``循环量子引力''并尝试理解这一点)。

正是这种思维上的抽象飞跃 --- 从对某种模式的观察到对该模式的认识论理解 --- 从最好的意义上来说,是真正具有好奇心和智慧的思想家、真正的科学家的特征。你认为谁是更好的昆虫学家:贪婪的读者,他已经记住了世界上所有目前已知的甲虫种类,或者实验室科学家,他检查了多种物种,可以采集新标本并对其分类为甲虫还是非甲虫?这在某种程度上是一个引导性问题,但要点是:\emph{理解}定义及其背后的动机比简单地了解一堆满足某个定义的\emph{实例}要有益得多。

可以说,这对数学更为重要。你能想象一个数学家不知道质数是什么,只能凭记忆列出前 100 个质数并对此沾沾自喜吗?当然不是!数学研究的美妙、通用和魅力部分在于我们检查模式和现象,然后选择如何做出与这些模式相关的适当定义。然后,我们利用对这些模式的新理解来对其他模式和现象做出严格精确的预测。彻底理解定义或概念可以提高预测能力,并且比仅仅了解该定义/概念的示例更为有效。


\subsection{正确撰写}

数学的另一个有趣的方面是,尽管它本身就是一种语言,但我们依赖外部语言来传达我们所拥有的数学思想和见解。尝试在不使用任何单词的情况下重写我们之前看过的定义和证明。这很难,不是吗?因此,我们希望用来传达数学思想的书面语言遵循与我们所写的数学``句子''相同的标准:我们希望它们\emph{精确}、\emph{合乎逻辑}且\emph{清晰}。

现在,为这三个词下一个精确、合乎逻辑且清晰的定义本身就是一项艰巨的任务。然而,我们都认同理想的证明应该是:

\begin{itemize}
    \item \textbf{精确的:}任何个体陈述都不应是不真实的或可以通过多种方式解释从而导致真相需要商榷;
    \item \textbf{合乎逻辑的:}每一步都应遵循先前的步骤,并有适当的动机和解释;
    \item \textbf{清晰的:}步骤间应该用正确的语法连接和描述,帮助读者了解发生了什么。
\end{itemize}

让我们检视几个无视这些标准并且在某种程度上不符合我们迄今为止的证明定义的``证明''。

\subsubsection*{糟糕``证明''\#1}

首先,我们来一个 $1=2$ 的``证明'',我们知道这肯定有问题。你能找到哪里出错了吗?它违反了哪个标准?精确、合乎逻辑还是清晰?

\begin{proofs}{``证明''}
    假设有两个实数 $x$ 和 $y$,考虑如下等式:
    \begin{align*}
        x &= y \\
        x^2 &= xy &\text{两边同时乘以} y\\
        x^2-y^2 &= xy-y^2 &\text{两边同时减去} y^2\\
        (x+y)(x-y) &= y(x-y) &\text{因式分解} \\
        x + y &= y &\text{两边同时消掉} (x-y)\\
        y + y &= y &\text{因为第一行给定} x=y\\ 
        2y &= y \\
        2 &= 1 &\text{两边同时除以} y
    \end{align*}
\end{proofs}

这里的问题是\emph{精确}。对第四行进行因式分解后,除以公因数 $(x - y)$ 即可得到第五行,这似乎既方便又明智;然而,第一行告诉我们 $x = y$,所以 $x-y = 0$,\textbf{除以零是不允许的}!使用变量 $x$ 和 $y$ 只是一种让你失去踪迹并掩盖除以零的方法。(说到这里,为什么不能除以零?你能想出一个合理的解释吗?从乘法的角度思考一下。)

\subsubsection*{糟糕``证明''\#2}

这是类似``事实''的另一个证明,即 $0 = 36$。

\begin{proofs}{``证明''}
    考虑方程 $x^2+y^2 = 25$。整理并分离 $x$ 可得
    \[x = \sqrt{25-y^2}\]
    两边加3再同时平方得
    \[(x+3)^2=\Big(3+\sqrt{25-y^2}\Big)^2\]
    请注意,$x = -3$ 和 $y = 4$ 是原方程的解,所以最终的方程也应该是成立的。将这组解代入 $x$ 和 $y$ 可得
    \[0 = (-3+3)^2 = (3+\sqrt{25-16})^2 = (3+3)^2 = 36\]
    因此,$0 = 36$。
\end{proofs}

到底发生了什么?你能发现不合逻辑的步骤吗?如果我们使用最后选择的变量 $x$ 和 $y$ 的特定值重写证明步骤,也许会有所帮助:

\begin{align*}
    (-3)^2+4^2 &= 25 \\
    -3 &= \sqrt{25-4^2} \\
    (-3+3)^2 &= \Big(3+\sqrt{25-4^2}\Big)^2 \\
    0 &= 36
\end{align*}

现在很明显了,不是吗?对方程两边进行平方根运算存在一个问题,它取决于 $(-x)^2=x^2$ 这一事实。

当我们解 $z^2=x^2$ 这样的方程时,必须牢记这个方程有两个根:$z = -x$ 和 $z = x$。因此,从方程开始并对两边进行平方是一个完全合乎逻辑的步骤(所得方程的真值与原方程的真值\emph{一致}),但反之却是一个不合逻辑的步骤(平方方程成立并不\emph{一定}等于平方根方程也成立)。这是一个带有\textbf{条件陈述}或\textbf{逻辑蕴涵}的问题,我们稍后会详细讨论这些概念(第 4.5.3 节)。现在,我们可以用下面的代码来总结这个概念:

\[\text{如果} a=b, \text{则} a^2=b^2, \text{反过来,如果} a^2=b^2, \text{则} a=b \text{或} a=-b\]

这说明了为什么在上面``证明''中从 $x^2+y^2 = 25$ 到 $x = \sqrt{25-y^2}$ 这步是不合逻辑的:当有两种可能的选择时,我们立即假设平方根的一种特定选择。如果我们选择负平方根,会发生什么?试着将第二步替换为 $-x = \sqrt{25-y^2}$ 并重写证明,在最后对 $x$ 和 $y$ 使用相同的值。发生了什么? 如果你用 $x = 3, y = -4$ 代替呢?或者 $x=-5, y=0$ 呢?你能描述一下如何确定何时应该使用正根 $x$ 何时应该使用负根 $-x$ 吗?

\subsubsection*{数学使用``包含或''}

既然``或''这个词已经出现,我们先提一下上面句子中\emph{或}的使用。当我们说 ``$a = b$ 或 $a = -b$'' 时,我们的意思是,这两个陈述中\emph{至少}有一个必须为真,甚至可能两者都为真。现在,如果 $a \ne 0$ 且 $b \ne 0$,则只有其中一种结论性陈述可以为真;也就是说,在这种情况下,只有一个根(正或负)是正确的,而不是两者都是正确的。然而,如果 $b = 0$,那么两个结论性陈述说的是同样的事情,$a = 0$,因此规定\emph{或}意味着只有一个陈述可以为真并且不允许它们都为真,这是不合逻辑的。在其他情况下,这种区别会产生更显着的差异。

例如,如果你在餐厅点了一份三明治,服务员问:``您想要薯条还是土豆沙拉?'',这可以理解为,你可以选择其中一项,但不能同时选择两者。这就是\textbf{异或}的示例,因为它阻止你选择两个选项。或者,如果你忘记带书写工具去课堂,准备用老方法记笔记于是询问你的朋友,``我可以借用您的铅笔或钢笔吗?'',这可以理解为,你实际上并不关心提供两个选项中的哪一个,只要至少有一个可用即可。也许你的朋友两者都有,而且其中任何一个都可以。这是\textbf{包含或}的示例,并且这是所有数学示例中假定的解释。

\subsubsection*{不清楚的论证}

最后两个糟糕的``证明''问题在于精确性和逻辑正确性。我们要求良好证明的第三个条件是\emph{清晰}:我们希望文字能够解释证明者在每个步骤中完成的工作以及为什么该工作是相关的。换句话说,我们不希望读者在任何时候停下来问:``这句话是什么意思?''或``那是从哪里来的?''或因困惑而产生的类似问题。如果有帮助的话,考虑写一个证明,向你班上的朋友、将要阅读你作业的评分者或智力相当的家庭成员解释它。重读你自己写的证明,并尝试预测可能出现的问题或可能要求你进行的澄清,然后通过重写来解决这些问题。

证明可能因为多种原因失败或不清晰,首当其冲的是,单词和句子可能无法正确解释证明的步骤和动机,这实际上可能是因为单词太多(使读者负担过重而模糊了证明)或因为单词太少(没有给读到者足够的信息)或者因为所选的词语令人困惑(没有正确解释证明)。这些是证明\emph{语言}的问题。

从数学上讲,就清晰性而言,可能会出现许多问题。也许证明撰写者突然引入一个变量,但没有说明它是什么类型的数字(整数、实数等),或者跳过几个算术/代数步骤,或者使用新的符号而没有提前定义它的含义...这些行为在技术上都没有错误或不合逻辑,但它们肯定会给读者带来困惑。你能想到其他原因让证明不明确吗?尝试想出一种基于语言的原因和一种基于数学的原因。

\subsubsection*{糟糕``证明''\#3}

让我们陈述一个关于多项式函数的简单事实,然后查阅关于该事实的``证明''。仔细阅读论证并尝试找出一些不清楚的句子或数学步骤。

\textbf{事实:}考虑多项式函数 $f(x) = x^4-8x^2+16$。对于任意 $x$,该函数都满足 $f(x) \ge 0$。

\begin{proofs}{``证明''}
    无论 $x$ 的值是多少,我们将其代入 $x$ 的函数 $f$ 中,都可以通过对多项式进行因式分解写出该函数的输出值,如下所示:
    \[f(x) = x^4-8x^2+16 = (x-2)^2(x+2)^2\]
    而任意数字 $z$ 一定要么小于 $-2$,要么大于 $2$,要么严格介于 $-2$ 和 $2$ 之间,要么等于其中之一。当 $z > 2$ 时, $z - 2$ 和 $z + 2$ 都大于 $0$,因此 $f(z) > 0$。当 $z < -2$ 时,两项都为负而负数的平方为正,所以 $f(z) > 0$。当 $-2 < z < 2$ 时,类似情况再次发生,当 $x = 2$ 或 $x = -2$ 时,其中一项为 $0$,所以 $f = 0$。因此,我们要证明的必然成立。
\end{proofs}

这个证明有什么可批评的地方呢?首先,它正确吗?精确吗?符合逻辑吗?清楚吗?哪里不清楚?试着找出那些有点不清楚的陈述,无论是语言上的还是数学上的,并尝试适当地修改它们。在不指出个别错误的情况下,下面提供了上述事实的更好、更清晰的论证。

\begin{proof}
    我们首先对函数 $f(x)$ 进行因式分解,将其视为变量为 $x^2$ 的二次函数
    \[f(x) = (x^2)^2-8x^2+16 = (x^2-4)^2\]
    接下来,我们可以因式分解 $x^2-4=(x+2)(x-2)$ 并将原函数重写为
    \[f(x) = \big((x+2)(x-2)\big)^2 = (x+2)^2(x-2)^2\]
    对于任意实数 $x$,都有 $(x+2)^2 \ge 0$ 且 $(x-2)^2 \ge 0$,因为平方结果一定非负。两个非负项的成绩依然非负,所以 对于任意实数 $x$, $f(x) = (x+2)^2(x-2)^2 \ge 0$。
\end{proof}

第一个``证明''和第二个``证明''有什么区别?你重写的证明也像第二个证明吗?

对第一个``证明''的批评之一是它没有完全解释 $-2 < x < 2$ 的情况;相反,它只是说发生了``类似''的事情,并没有实际执行任何细节。这是数学中的常见情况(证明的某些步骤``留给读者''),这是一种简便技巧,有时可以避免繁琐的算术/代数,并使阅读证明更容易、更快速、更愉悦。但是,要谨慎使用该技巧。作为证明撰写者,确保步骤确实有效非常重要,即使你不打算在证明中呈现它们;你应该向读者提供简短的摘要或提示,说明这些步骤实际上是如何运作的。此外,证明撰写者应尽量不要在对证明最终结果至关重要的步骤上使用此技巧。

在上面的特例中,完全跳过了因式分解的实际步骤,并且只是顺便提及了对 $-2 < x < 2$ 情况的分析,但这些都是证明的重要组成部分!从任何维度上来说,这都是一个简短的证明,展示这些步骤并不代表在简洁性或清晰性方面做出了巨大的牺牲。这再次呼应了证明撰写既是科学也是艺术的观点:选择何时将一些细节验证留给读者可能很棘手。 在这种特殊情况下,展示所有步骤很重要。

尽管如此,我们给出的第二个证明要清楚得多。而且,完全摒弃了第一个``证明''中采用的分情况讨论技术!第一个``证明''中的一个情况存在清晰性问题,但我们没有简单地在重写版本中阐述细节,而是选择完全放弃该技术并使用更简短、更直接的证明。这并不是说第一个证明的技术是错误的。如果我们填补第一个``证明''论证中的空白,我们就会得到一个完全正确的证明。然而,该技术中的一些步骤是多余的。请注意,实际上 $-2 < x < 2$ 和 $x > 2$ 的情况在某种意义上是相同的:在这两种情况下,因子都满足 $(x - 2)^2 > 0$ 和 $(x + 2)^2 > 0$。事实上,第一种情况 $x<-2$ 也是如此!所以,当相同的最终观察结果对应于所有三个情况时,为什么要把论证分成三个不同情况呢?这种情况下,最好将它们合二为一(同样利用当 $x = 2$ 或 $x = -2$ 时,其中一个因子为 $0$ 的知识)。重申一遍,使用分开讨论技术当然没错。但是,它只会给证明增加不必要的长度。

我们在上面的段落中提到了术语``情况''和短语``分情况讨论'',但没有恰当定义或解释我们的意思。现在,我们想推迟对这些术语的讨论,直到我们在第 4 章中彻底讨论逻辑。不过,如果你渴望立即解决这个问题,可以跳到第 1.4.4 节并查看``匈牙利朋友''问题,其中包含一些复杂的分情况讨论。

\subsection{选择逻辑}

我们已经非常频繁地使用``逻辑''一词及其相关形式,但还没有充分解释其意思。我们意识到这似乎违背了我们迄今为止一直大力倡导的精确性和清晰性,但不幸的是,我们不得不承认,提供\emph{逻辑}的完整定义是极其困难的。

\subsubsection*{游戏}

如果您你寻找对逻辑的启发式理解,可以尝试从``逻辑谜题''(如数独或数谜)入手来思考它。这些谜题/游戏从一开始就围绕定下的非常具体的规则构建的,然后向解谜者提供一个起始谜面,并期望解谜者严格遵守规则,直到解出谜题。例如,在数独中,规则是 1 到 9 中每个数字在每行、每列和 $3 \times 3$ 框中恰好只出现一次,解谜者需要综合各种情况在网格中放置越来越多的数字,不断缩小``潜在解''的范围,以找到起始谜面的唯一答案。这个解谜过程的一个重要方面是,任何时候都不要(自作聪明地)\emph{猜测};每一步都应该在考虑当前情况和谜题既定规则的情况下进行理性选择,并且在这个框架内,保证谜题是可以解决的(当然,要有足够的时间)。

数学逻辑在某些方面略有不同,但本质是一样的:都有既定的游戏规则,每一步都应该以这些规则和当前知识为指导,除此之外别无其他。这就是我们所说的撰写数学证明应该受\emph{逻辑}支配的意思:从一个真理到另一个真理,每一步都应该遵循约定的规则,并且只参考这些规则或已经证明的事实。我们在证明(以及一般的数学中)中玩的``游戏''或``谜题''并不像数独谜题那么清晰。然而,更令人困惑的是,有时我们会投身一场无法获胜的游戏,却丝毫没有意识到这一点!

这里``无法获胜的游戏''的想法来自 20 世纪奥地利逻辑学家、数学家库尔特·哥德尔 (Kurt Gödel) 的工作成果,这是一项令人震惊、使人惊奇但极其有力的结论。他的\emph{不完备性定理}反映出一个强大的逻辑系统内部的固有问题:有些\textbf{真实}陈述在该系统内却是不\emph{可证明}的。在这里,我们无法透彻详细地解释一些术语(即,\emph{逻辑系统}和\emph{可证明}),但希望你能看到这里出现了一些神奇的事情。这怎么可能呢?如果某事在数学中是\textbf{真的},我们不是应该能够以某种方式证明它是真的吗?否则我们怎么知道这是真的呢?

\subsubsection*{数学简史}

要回答这些自然而生的问题,让我们先退一步,回到数学的一个重要分支 --- 逻辑的起源进行讨论。在整个讨论过程中要记住的一点是,我们无法完全解决出现的每个主题,这可能让人感到不满,我们理解这一点。数学之美部分在于,学习任何一个主题都会带来许多其他问题和概念需要思考,而这些问题和概念又可以通过更多的数学来解决。不过,背景很重要,就本书的背景而言,我们没有足够的时间和空间去讨论所有这些相关的主题。我们并不是试图向你隐瞒任何事情或掩盖某些问题;相反,我们只是在面对现实,确保我们不会强迫你阅读 10,000 页的完整数学史,只是为了理解我们的观点!

在你的数学生涯中,可能会进一步研究我们下面提到的许多数学家(以及他们所做的工作)。到那时,你将通过亲自动手实践从而对这个学科有更深入的理解和欣赏,你也将更有能力去解决其中的问题。而现在,我们仅仅是基于兴趣介绍这些数学家。数学有着丰富而有趣的历史,了解它会很有帮助!在这里,我们将尽力以简洁而有意义的方式来解读逻辑学 --- 它的历史、动机和意义 --- 使之与当前的背景相契合。

19 世纪中后期的数学家和哲学家首先研究了后来演变成现代逻辑的思想,他们对我们在这里试图研究的许多相同问题感兴趣:我们如何知道某件事是\textbf{真}的?我们如何才能表达这个真理呢?我们可以声明什么类型的``事件''为\textbf{真}或为假?这些数学家从根本上分解了数学语言,研究了如何以非常具体的方式组合一组固定的符号来创建更复杂的陈述,但从总体上看,这些陈述仍相当简单。这并不是要贬低他们的努力,毕竟,我们都必须从某个地方开始,而这些人是从头开始的。

首先进行的一项重大工作是探究算术的基础,或者说\textbf{自然数}($1,2,3,4,\dots$)的研究。就像欧几里得(Euclid)研究几何时,先通过建立一系列公认的真理或\textbf{公理},然后从这些给定的假设中推导出真理一样,意大利数学家朱塞佩·皮亚诺(Giuseppe Peano)建立了一套自然数公理,而其他人则从稍微不同的视角对这个主题进行了研究。与此同时,对真理及其证明严谨果断的欣赏,促使得大卫·希尔伯特和其他人提出了欧几里得公理的一些问题,尤其是平行公设。

这项关于几何和算术的工作自然引出了对数学其他领域的进一步、复杂的研究,以及对诸如实分析等领域热切地公理化尝试。卡尔·魏尔斯特拉斯(Karl Weierstrass)在研究这个主题时,提出了一些具有奇特属性的令人震惊的函数示例。例如,尝试定义一个处处不可微的连续函数。(如果你对微积分的这些术语不熟悉,请不用担心;总之,这很难。)最后,理查德·戴德金(Richard Dedekind)能够建立一个严谨的、逻辑的实数定义,完全由自然数推导出来,并且不依赖于必须存在数字连续体这种模糊的物理概念。

后来,这项研究稍微分支出来,变成了集合的研究。这个领域的许多基础工作是乔治·康托尔(Georg Cantor)在 19 世纪末期奠定的。他是第一个真正研究无限集理论的人,提出了无限有不同``大小''这一有争议的观点。也就是说,他证明了某些无限集严格大于其他无限集。这个想法在当时引起了极大的争议,以至于许多数学家都讨厌他!如今,我们意识到康托尔是对的。(这也让你提前窥探我们稍后在 7.6节中将会讨论的内容。举个有趣的例子:奇数集合和偶数集合当然一样大,但它们也和所有整数的集合一样大。然而,所有实数的集合严格大于二者!)

事实上,一些数学家对康托尔的发现感到相当震惊,甚至伟大的伯恩哈德·黎曼(Bernhard Riemann)一开始也认为集合论的发展将成为数学的祸害。但事实并非如此,从诞生起它就蓬勃发展,许多数学家致力于以正确的方式表示所有数学并理解数学的``基础''。某种程度上,你可以将集合论视为对所有数学家正在研究的基本对象的研究,最终,其方式类似于所有化学都是通过将元素周期表中的元素以越来越复杂的方式恰当地组合在一起来完成的。

这些主题的进一步发展是符号逻辑的研究,它比我们迄今为止提到的抽象概念更具体一些,而且我们在本书的开始章节中会频繁地研究这个领域的基本理念。该领域涵盖了如何将数学方程和符号与基于语言的符号和连词结合起来,以做出有意义的数学陈述,并可以通过证明来确认这些陈述的真实性。总的来说,这是数学的一个极其重要的组成部分,尤其是本书。个人观点当然比这更加细致和具体,但总的来说,大多数数学家的心态是,有许多数学真理等待被发现,我们花时间学习我们已经发现的真理,希望揭示更多真理。这就像一个巨大的考古挖掘,研究我们已经出土的骨头和文物将帮助我们预测我们在什么地方会发现什么类型的其他宝藏,以及如何寻找和挖掘。某种程度上,逻辑是从挖掘中一步一步抽象出来的过程:逻辑是对挖掘过程的研究。 它告诉我们如何真正利用数学知识并从中学习,并将其与其他知识相结合,从而证明更多的真理。

请注意,这不是一个精确的类比,抽象逻辑的研究要复杂得多。不过,就本书的目的而言,这是一种合理的思考逻辑的方式。我们将学习符号逻辑的一些基本原理和基本运算,并将这些知识应用到我们撰写证明的研究中。它将帮助我们真正理解证明是什么,它将指导我们构建要编写的证明,它将允许我们批判可能不正确的证明,并最终帮助我们理解数学作为一个整体是如何工作的。

\subsubsection*{逻辑应用:理论计算机科学}

逻辑思想和结果的一个非常重要的应用是计算机科学的发展和研究,特别是理论计算机科学和可计算性理论。这个特殊的数学分支最初是受大卫·希尔伯特二十三个问题 --- 1900 年出版的数学界著名未解决猜想列表 --- 中的第十个问题推动的。第十问题涉及解\textbf{丢番图方程(Diophantine Equations)}, 就是以下形式的方程
\[a_1x_1^{p_1}+a_2x_2^{p_2}+a_3x_3^{p_3}+\dots+a_nx_n^{p_n} = c\]
其中 $a_1, a_2, \dots, a_n$ 和 $c$ 都是给定的常数,$p_1, \dots, p_n$ 为给定的自然数,$x_1, \dots, x_n$ 为要求的使方程成立的变量。

给定这样一个方程,人们可能想知道是否存在解,如果存在,那么存在多少组解。 此外,如果我们给定常数 $a_i$ 和 $c$ 都是有理数,我们想知道是否可以确保存在一组解,其中所有变量 $x_i$ 也都是有理数。关于这个特定问题已经建立了一些理论成果,但是根据 1900 年的陈述,希尔伯特第十问题问的是,是否存在``一个过程,根据该过程可以经过有限数量的操作确定它''是否存在给定方程的解,其中所有变量 $x_i$ 都是有理数。尽管当时还没有算法这个术语的正确概念或定义,但希尔伯特要求的是一种\textbf{算法},该算法接受常数 $a_i$ 和 $c$ 的值,并根据是否存在所需属性的解输出 \textbf{True} 或 \textbf{False}。这个问题的一个重要部分是,该``过程''在输出答案之前执行有限数量的步骤。

一位名叫艾伦·图灵(Alan Turing)的英国剑桥大学学生几年后开始研究这个问题,他想到了一台物理机器,该机器将执行输出所提出问题的答案所需的步骤。他在随后的出版物中描述了他的发明,我们现在称之为\emph{图灵机},这是一种有趣的理论装置,可以用来回答形式逻辑中的一些问题,但也展现了构建现代计算机的许多想法。我们说它是一个理论装置,是因为它的定义的属性决定了它在物理上无法构建和操作,但它很好地处理了一些理论问题,包括前面提到的希尔伯特第十问题。更具体地说,当我们说某件事是可计算的,或者能够在有限数量的步骤中确定时,这台机器为我们的意思提供了正确的定义,这有助于建立正确的算法概念。如果我们在讨论可计算性话题时不提及阿隆佐·丘奇(Alonzo Church),那是不公平的,因为他与图灵同时在研究类似的问题。他们的名字一起出现在丘奇-图灵论文中,该论文将图灵机的工作原理与更理论化、基于形式逻辑的可计算性概念联系在一起。

\subsubsection*{我们将用逻辑做什么?}

虽然集合论和逻辑中的所有这些主题本质上都很有趣并且对数学非常重要,但总的来说,我们根本没有足够的时间和空间来详细讨论它们。相反,我们更关注在撰写和批判数学证明时使用的逻辑概念。

我们将考虑:

\begin{enumerate}
    \item 我们实际上可以陈述和证明什么类型的``事物'',
    \item 我们如何将我们已知为真的``事物''结合起来以产生更复杂的真理,
    \item 我们如何解释我们是怎样得出``事物''确实为真的结论。
\end{enumerate}

由于缺乏更好的术语,这里我们用``事物''一词,因为我们还没有\textbf{数学陈述}的正式定义,而这实际上是我们将要证明的``事物''的类型。从本质上讲,数学陈述是数学和语言中符号和句子的组合,可以验证为\textbf{真}或为\textbf{假},但不能同时既真又假或非真非假。那么,证明就相当于组织一系列步骤和解释,使用为真的数学陈述和句子将这些真理连接在一起,并最终产生特定陈述所需的真理。我们对逻辑的研究将解决如何组合这些步骤并确保我们的证明最终会导出对真理的正确评估。

更具体地说,我们将研究数学陈述到底是什么,以及如何将它们组合起来产生更复杂的陈述。``\emph{与}''和``\emph{或}''这两个词在其中特别重要,因为这两个词允许我们以新的、有意义的方式将两个数学陈述组合在一起。我们还将研究\textbf{条件}数学陈述,即``如果 A,则 B''或``A 蕴含 B''形式的语句。这些是数学陈述的基础,大多数重要的数学定理都是这种形式。这些陈述涉及做出一些\emph{假设(assumption)}或\emph{假说(hypothes)}(包含在陈述 A 中),并使用这些假设的事实得出结论(包含在陈述 B 中)。回顾 \ref{sec:section1.1.1} 节中毕达哥拉斯定理的陈述,注意它是如何以条件陈述的形式出现的。(可以用另一种方式书写吗?尝试以非条件形式重写定理的陈述,并思考在该形式下是否本质上是不同的陈述。找到另一个以条件陈述形式给出的著名数学定理,并尝试进行相同的格式更改。)

数学中的另一个重要思想,也是在证明撰写中经常出现的思想,是\textbf{变量}的概念。有时我们想笼统地讨论一种数学对象,而不为其分配特定的值,这就需要通过引入变量来实现。你可能在之前的数学学习中经常看到这种情况发生,甚至在本书中我们已经使用过变量了。再看看 \ref{sec:section1.1.1} 节中的毕达哥拉斯定理的陈述。字母 $a,b,c$ 代表什么?好吧,我们并没有给出明确说明,但我们知道它们是正实数,表示直角三角形三边的长度。什么三角形?我们并没有给出一个具体的三角形,也没有给出一张具体的图画或类似的东西,但你清除我们在说什么。此外,我们要检查的证明并不取决于这些变量的实际值,而仅仅取决于它们是否是具有某些属性的正实数。 这是非常有用且重要的,某种程度上,它节省了时间,因为我们不必单独考虑宇宙中所有可能的直角三角形(有无穷多个!) 并且可以将整个想法简化为一个紧凑的陈述和证明。

我们可以对变量进行\emph{量化}。这涉及到声明某个陈述对于变量的\emph{任意}潜在值或仅对\emph{某个}特定值是否成立。例如,在毕达哥拉斯定理中,我们不能声称 $a^2+b^2=c^2$ 对任意正实数 $a,b,c$ 成立;我们必须对变量施加额外的假设才能获得我们所做的结果。这是\textbf{全称}量化的一个例子:``对于\emph{所有}具有这个属性和那个属性的数字 $a,b,c$,我们可以保证...''同样地,我们还可以进行\textbf{存在}量化:``\emph{存在}一个具有此属性的数$n$。''

你能想到我们迄今为止已经研究过的使用存在量化的定理/事实吗?再来看一个证明,存在无理数 $a$ 和 $b$ 使得 $a^b$ 是有理数。请注意,我们证明的这个主张属于存在类型:我们声称\emph{存在}两个具有所需属性的数字,然后我们继续证明确实必然存在这样的数字。眼下,这个证明的有趣之处在于它是\emph{非构建性的};也就是说,我们能够在不明确给出数字 $a$ 和 $b$ 实际是什么的情况下证明我们的主张。我们将其缩小到两个选择,但从未声称哪一个是正确的选择,只是其中之一必然有效。

\subsection{明显的混淆}

作为这些逻辑概念的预览,我们将在稍后详细研究其数学细节,让我们举一些现实世界中基于语言的例子来说明这些想法。

\subsubsection*{条件陈述}

首先,让我们研究一下\textbf{条件陈述}。数学定理经常采用条件陈述的形式,但这种类型的陈述也经常出现在日常用语中,有时是隐含的(这只会增加混乱)。例如,人们有时会谈论他们将如何处理彩票奖金,比如
\[\text{如果我中了彩票,那么我就买辆新车。}\]
``那么(then)''之后的语句依赖于``如果(if)''相关的语句。当``如果(if)''部分的条件满足时,保证会发生``那么(then)''部分的操作。

条件语句中与``如果(if)''相关的部分称为\textbf{假说(hypothesis)}(更正式的名称为\textbf{先行词(antecedent)})。与``那么(then)''相关的部分称为\textbf{结论(conclusion)}(更正式的名称为\textbf{结果(consequent)})。

有时条件句的结论更加微妙,甚至句子中的动词时态不包含``如果(if)''。以电影《壮志凌云》中的台词为例:
\[\text{这是机密。我可以告诉你,但那样我就不得不杀了你。}\]
这里,第一部分``我可以告诉你''是一个伪装的假设。与实际电影台词具有相同逻辑含义的说法是``\emph{如果我告诉你,我就不得不杀了你}'';然而,这么说没有原台词富有戏剧性和张力。实际上,在条件陈述的结论中常常不包含``那么(then)''一词。在阅读句子时,你甚至可能不知不觉中在脑海里添加该词。下面是 The Barenaked Ladies 乐队的一首歌中的歌词:
\[\text{如果我有 100 万,我们就不必步行去商店了。}\]
\[\text{如果我有 100 万,我们会乘坐豪华轿车,因为它更贵。}\]
这两行都是条件陈述,但都不包含``那么(then)''一词;它被理解为句子的一部分。

将上述示例与以下句子进行比较,看看有什么不同:
\[\text{只有下雨的时候我才带伞。}\]
这里,说话人不愿意在没有正当理由的情况下随身携带雨伞,而是更愿意确保它有用。这句话与下面类似的句子意思相同吗?
\[\text{如果我带着雨伞,那就是下雨了。}\]
在现代语言用法中,条件的概念可能有点模糊。例如,第一句可以解释为有时可能下雨,但说话人忘记带伞。第二句是一个条件陈述的明确断言:看到我撑着伞走来,你一定会推断这是因为下雨了。在数学中,我们把这两个句子联系起来,说它们有相同的逻辑意义。

这引出了短语``仅当(only if)''的含义,以及随后的短语``\textbf{当且仅当(if and only if)}''。考虑以下两句话:
\[\text{如果中了彩票,我会买辆新车。}\]
\[\emph{只有}\text{中了彩票,我才会买辆新车。}\]
第一句说中彩票保证我会买辆新车,而第二句说买新车的行为保证是因为我刚刚中了彩票。如果这两句话都为真,那么``中彩票''和``买新车''这两个事件在某种意义上是等价的,因为其中一个事件的发生\emph{必然保证}另一个事件的发生。

因此,数学定义通常使用``\textbf{当且仅当}''这一短语。例如,我们可以写``一个整数是偶数,当且仅当它能被 $2$ 整除。''这表明知道一个数具有该性质可以称之为``偶数'',知道一个数是偶数可以得出其整除性质。(不过,有时一个定义只会使用\emph{当(if)},而\emph{仅当(only if)}部分没有说明但能被理解。你可能已经注意到,我们在 \ref{sec:section1.1.2} 节中对质数的定义就是这样做的。)

\subsubsection*{创建更多条件陈述}

从一个条件陈述开始,只要稍加修改,便能生成其他三个内容相同但结构不同的条件陈述。继续使用``彩票/汽车''的例子,让我们考虑原句的以下四个版本:

\begin{enumerate}
    \item 如果我中了彩票,那么我就买辆新车。
    \item 如果我买了辆新车,那么我中了彩票。
    \item 如果我中不了彩票,那么我就不会买新车。
    \item 如果我没有买新车,那么我就没中彩票。
\end{enumerate}

这些句子比较起来怎么样?它们中的任何一个都有相同的逻辑含义吗?假设第一个为真,那么所有这些都为真吗?我们认为,在这种情况下,即使第一句是真的,第二句也可能是假的。也许我在工作中得到了大幅加薪,或者继承了一笔钱,所以决定买辆新车。第三句和第四句呢?它们能以某种方式与其他句子联系在一起吗?这个就留给大家自己讨论和探索吧。对我们研究过的其他条件陈述提出同样的问题,看看你的答案是否也不同,这可能会很有趣。

最后一个条件陈述的例子来自脱口秀演员德米特里·马丁(Demetri Martin)的一个笑话。

\begin{quote}
    我走进一家服装店,一位女士走过来对我说:``如果你需要什么,我是吉尔。''我以前从未见过有条件身份的人。``如果我什么都不需要怎么办!你是谁?''
\end{quote}

上面的例子应该会让你体会到现代语言中条件陈述的不精确或微妙,有时需要进一步解释。在数学中,我们希望此类陈述是严格的、定义明确的且无歧义的。我们稍后将在第 \ref{sec:section4.5.3} 节中进一步研究这一点。不过,就目前而言,以计算机算法解释 \verb|if...then| 的严格方式来思考此类陈述可能会有所帮助。当 \verb|if| 部分的条件满足时,子程序被执行,否则被忽略。同样,\verb|while| 循环只是 \verb|if...then| 语句的序列,只是被压缩成一种简洁的形式。

\subsubsection*{量词}

接下来,让我们看一些量词的例子。当存在一个未知变量是从一组可能的值或表示中提取对象时,我们将使用量词。例如,当我们在毕达哥拉斯定理的陈述中量化变量 $a,b,c$ 时,它们是从表示直角三角形边长的实数集中提取的。对于非数学示例,请考虑以下句子:
\[\text{每个人都被某人爱着。}\]
这里有哪些变量?它们是如何量化的?请小心,因为这句话中实际上有两个量化,两个变量各一个。在这两种情况下,变量都代表世界上所有人集合中的成员,第一个变量是全称量化,而第二个变量是存在量化。这听起来可能令人困惑,所以让我们试着用更详细的措辞改写这个句子:
\[\text{对于世界上的每个人} x\text{,都存在另一个人} y \text{,具有} y \text{爱} x \text{的属性。}\]

你看出来这和第一句话的逻辑意义是一样的吗?当然,对于对话来说,这个内容有点过于冗长和精确,但我们在这里展示它是为了向你揭示潜在的变量和量词。量词的关键短语是``\emph{对于所有(for every)}''(全称量化)和``\emph{存在(there exists)}''(存在量化)。

\subsubsection*{量化顺序很重要!}

现在,让我们看一个与上面示例类似的句子:
\[\text{某人被每个人爱着。}\]
这句话和上面那句话很相似;甚至它们的用词都相同!词序变化对句子的逻辑意义有何影响?这里仍然有两个变量和两个量词,一个是全称量词,一个是存在量词,但是这些量词的应用顺序发生了改变。这句话的详细版本是:
\[\text{存在某人} x \text{具有对于世界上的每个人} y, y \text{都爱} x \text{的属性。}\]

这和第一句话的意思完全不同!第一个似乎可信,但这个就很奇怪。这个例子应该让你明白保持量化顺序是多么重要,这样你才能真正表达你的真实意思。

\subsubsection*{嵌套量词}

下面的例子说明我们的大脑在处理语言中的量词时有时是多么得快速和轻松,即使这种相互联系可能让人难以理解。当量词一个接一个地跟在后面时,我们称之为\emph{嵌套}。

分析和理解这些句子的能力可能取决于句子的上下文及其试图传达的信息。如果信息有意义并且我们相信它,那么它就更容易理解。关于这一现象,我们所知的最好的例子是伟大的总统演说家亚伯拉罕·林肯(Abraham Lincoln)的以下名言:

\begin{quote}
    你可以一直愚弄一些人,有时也可以愚弄所有人,但你不能一直愚弄所有人。
\end{quote}

这里到处都是量词!我们谈论的是所有人的集合,以及某些被愚弄人的集合,并对这些集合进行量化。尝试用几种不同的措辞重写这个句子,看看它是否听起来更``简单''或更简洁。是否存在另一种表达句子的方式,可以删除部分(或全部)量词而不改变含义?

最后,出于个人兴趣和幽默感,我们将引用鲍勃·迪伦(Bob Dylan)《Talkin World War III Blues》中的一句类似的话,这首歌来自鲍勃·迪伦 1963 年发行的专辑《The Freewheelin' Bob Dylan》:

\begin{quote}
    Half of the people can be part right all of the time\\
    Some of the people can be all right part of the time\\
    But all of the people can't be all right all of the time\\
    I think Abraham Lincoln said that
\end{quote}

稍后我们将更详细地讨论这些主题,那时我们将研究它们的数学动机、含义和用途。目前,我们再怎么强调这些问题在撰写证明中的重要性都不为过。把一堆句子串在一起,却不知道它们是如何连接的,这并不是证据,但一系列结构合理的逻辑陈述和含义才是我们真正想要的。