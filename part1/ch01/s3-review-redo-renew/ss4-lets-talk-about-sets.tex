% !TeX root = ../../../book.tex
\subsection{集合漫谈}

我们已经提到了一些特定类型的数字,但我们想具体定义我们将来要使用的数字集。这些数字集合都由一个特定字母用黑板粗体表示。\textbf{自然数}(也称为整体数(whole numbers)或计数数(counting numbers))之所以被称为自然数,是因为当我们计数物体时,用自然数感觉很``自然''。自然数可以写做
\[\mathbb{N} = \{1, 2, 3, 4, 5, \dots\}\]
(自然数有一个更具体更技术性的定义,我们将在稍后解释。)\\
我们用 $\mathbb{N}$ 表示 ``自然(natural)''

使用 $\mathbb{N}$,我们可以定义一个相关的数字集合:所有\textbf{整数}的集合,它包含了自然数、$0$ 和负自然数。整数可以写做
\[\mathbb{Z} = \{\dots, -3, -2, -1, 0, 1, 2, 3, \dots\}\]
字母 $\mathbb{Z}$ 来自德语单词 \emph{Zahlen},意思是``数字''。

从整数集合,我们可以定义\textbf{有理数}的集合。这些数字可以表示为整数的比率,但它们似乎没有像集合 $\mathbb{N}$ 和 $\mathbb{Z}$ 那样自然的``列表'',所以我们不能像上面那样书写这个集合。为此,我们使用一个非常常见的集合表示法,如下所示:
\[\mathbb{Q} = \Big\{\frac{a}{b} \mid a,b \in \mathbb{Z} \text{ 且 } b \ne 0\Big\}\]
读作:
\begin{quote}
    ``有理数集是所有 $\frac{a}{b}$ 形式的数的集合,其中 $a$ 和 $b$ 都是整数,且 $b$ 不为零。''
\end{quote}
这表达了有理数是分数的必要信息,其中分子和分母都是整数(但分母不能为 $0$,因为除以 $0$ 是不允许的)。我们使用字母 $\mathbb{Q}$ 表示有理数,是因为 $\mathbb{R}$ 已经被用于表示实数,而 $\mathbb{Q}$ 是上一个可用的字母。此外,$\mathbb{Q}$ 包含所有整数的商,所以这也是有道理的!

\textbf{实数} $\mathbb{R}$ 有一个非常技术性的定义,遗憾的是,我们无法在本书中全面深入研究。(这恰恰表明,从数学上定义这个集合是多么困难!) 目前,思考实数的一种方法是通过\textbf{数轴}。实数是数轴上所有的数字,而 $\mathbb{N}$、$\mathbb{Z}$ 和 $\mathbb{Q}$ 中的数字是数轴上的特定数字,它们并不构成整条数轴。某种程度上,$\mathbb{R}$ 是 $\mathbb{Q}$ 的``补全'',即``填补有理数之间的空白''。
