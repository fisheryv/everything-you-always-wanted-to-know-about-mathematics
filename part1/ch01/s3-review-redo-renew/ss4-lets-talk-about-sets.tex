% !TeX root = ../../../book.tex
\subsection{集合漫谈}

我们已经介绍了几种特定类型的数字,现在将明确定义后续将使用的数字集合。这些集合通常用黑板粗体字母表示。\textbf{自然数}——或称正整数 (whole numbers) 或计数数 (counting numbers)——之所以被称为自然数,是因为计数物体时用自然数感觉很``自然''。自然数可以写做
\[\mathbb{N} = \{1, 2, 3, 4, 5, \dots\}\]
(自然数有更严谨的数学定义,后文将详细说明。)\\
符号 $\mathbb{N}$ 取自``自然 (Natural)''的首字母。

基于 $\mathbb{N}$ 可定义所有\textbf{整数}的集合,它包含自然数、$0$ 和负自然数。整数可以写做
\[\mathbb{Z} = \{\dots, -3, -2, -1, 0, 1, 2, 3, \dots\}\]
符号 $\mathbb{Z}$ 源于德语单词 \emph{Zahlen},意为``数字''。

从整数集可以定义\textbf{有理数}集。这类数可以表示为整数之比,但无法像集合 $\mathbb{N}$ 和 $\mathbb{Z}$ 那样自然``罗列'',所以我们不能像上面那样书写这个集合。为此,我们使用一个非常常见的集合表示法,如下所示:
\[\mathbb{Q} = \left\{\frac{a}{b} \mid a,b \in \mathbb{Z} \text{ 且 } b \ne 0\right\}\]
读作:
\begin{quote}
    ``有理数集是所有形如 $\frac{a}{b}$ 的数的集合,其中 $a$ 和 $b$ 为整数,且 $b$ 不为零。''
\end{quote}
这表明有理数是分子分母均为整数的分数(分母不能为 $0$,因为除以 $0$ 是不允许的)。我们使用字母 $\mathbb{Q}$ 表示有理数,一方面是因为 $\mathbb{R}$ 已被用于表示实数,而 $\mathbb{Q}$ 是上一个可用的字母;另一方面,$\mathbb{Q}$ 包含所有整数的商 (Quotient),所以这也是有道理的!

\textbf{实数} $\mathbb{R}$ 的严格定义涉及高深理论,遗憾的是,我们无法在本书中全面深入研究。(这恰恰印证了其数学定义的复杂性!)现阶段可以通过\textbf{数轴}来理解实数。实数是数轴上的所有数字,而 $\mathbb{N}$, $\mathbb{Z}$ 和 $\mathbb{Q}$ 仅是数轴上的特定数字,它们并不构成完整数轴。某种程度上,$\mathbb{R}$ 是 $\mathbb{Q}$ 的``补全'',即``填补有理数之间的空隙''。
