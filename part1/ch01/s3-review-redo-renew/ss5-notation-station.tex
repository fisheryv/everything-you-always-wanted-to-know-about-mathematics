% !TeX root = ../../../book.tex
\subsection{符号加油站}\label{sec:section1.3.5}

一种流行且方便的求和与求积的写法是使用缩写符号,将多个项或因子写成一种通用的形式。例如,如果我们想谈论前 $500$ 个自然数之和,该怎么写呢?写出总和的全部 $500$ 项会很乏味,所以 $1+2+3+\dots+499+500$ 是更常见的写法。(事实上,我们已经使用过这样的省略号。 你明白我们的意思吗?)这种写法很流行,并且确实表达了观点,但一些数学家对中间多余的省略号感到不满。我们推迟到现在才讨论这个问题,是因为符号通常很难学习和理解。我们并没有立即用新符号轰炸你,而是诉诸我们对 ``$\dots$'' 作用的直观理解。

既然我们已经提出来了,让我们看看如何避免使用省略号。为了写出我们上面提到的求和,我们将使用以下符号:
\[1+2+3+\dots+499+500 = \sum_{i=1}^{500}i\]
大写西格玛 $\sum$ 来自对应英文字母 S 的希腊字母,代表``求和''。\textbf{索引} $i$ 告诉我们求和的各个项的值。在 $\sum$ 符号下面写 $i = 1$,在 $\sum$ 符号上面写 $500$ 意味着我们让 $i$ 表示 $1$ 到 $500$(含)之间的所有自然数值。我们将这些值代入求和项的通用表达式,在本例中就是 $i$。因此,根据要求,求和项为 $1,2,3,\dots,500$。通过改变求和项表达式和/或索引的值,试着找到一些其他的写法。如果我们想求前 $500$ 个偶自然数之和怎么办?想求 $500$ 以内(含)所有偶自然数又该怎么办呢?尝试用上面的符号样式写出这些求和。

与此相关的是 $\prod$ 表示法。如果我们想查看前 $500$ 个自然数的乘积,我们将遵循相同的约定来识别索引值和通项:
\[1+2+3+\dots+499+500 = \prod_{i=1}^{500}i\]
大写派 $\prod$ 来自对应英文字母 P 的希腊字母,代表``求积''。再次尝试通过更改求积项和/或索引值以不同的方式表达上面公式。如果我们想求前 $500$ 个偶自然数的乘积怎么办?想求 $500$ 以内(含)所有偶自然数之积又该怎么办呢?尝试用上面的符号样式写出这些求积。

\subsubsection*{问题}

\begin{problem}
    用自然语言来描述以下等式的含义:
    \[\sum_{i=1}^{n}i^2 = \frac{n(n+1)(2n+1)}{6}\]
\end{problem}
\begin{problem}
    用适当的符号表达 $2$ 的前 $n$ 次幂的和与积,从 $2^0=1$ 开始。你能证明这个求和和求积公式吗?
\end{problem}
\begin{problem}
    考虑 $17$ 到 $33$(含)之间的所有奇数之和。索引 $0$ 开始,用求和符号写出这个求和公式。现在试着将索引从 $1$ 开始,重写求和公式。现在试着将索引从 $8$ 开始,重写求和公式,然后将索引从 $9$ 开始。以下哪一个感觉``更自然''?为什么?
\end{problem}