% !TeX root = ../../../book.tex
\subsection{符号加油站}\label{sec:section1.3.5}

一种流行且方便的求和与求积写法是使用缩写符号,将多个项或因子表示为通用形式。例如,若要表示前 $500$ 个自然数之和,完整写出所有 $500$ 项十分繁琐,因此常写作 $1+2+3+\dots+499+500$(事实上我们此前已使用过这种省略号表达)。尽管这种写法通俗易懂,但部分数学家认为省略号不够严谨。我们推迟到现在才讨论这个问题,是因为符号通常难以学习和理解。所以我们并未立即引入新符号,而是诉诸我们对``$\dots$''作用的直观理解。

现在我们将介绍避免省略号的表达方式。对于上述求和,可表示为:
\[1+2+3+\dots+499+500 = \sum_{i=1}^{500}i\]
大写希腊字母 $\Sigma$ 对应英文字母 \verb|S|,表示``求和 (Sum)''。\textbf{索引} $i$ 确定求和项的值。在 $\sum$ 符号下面写 $i = 1$,上面写 $500$ 表示 $i$ 取 $1$ 到 $500$ 之间(含端点)的所有自然数。将其代入通项表达式 $i$。因此,根据要求,求和项为 $1,2,3,\dots,500$。通过改变通项表达式和索引范围,试着找出其他写法。如何表示前 $500$ 个偶数之和?如何表示 $500$(含)以内所有偶数之和?尝试用上面的求和符号写出这些求和。

类似地,$\prod$ 符号表示求积。前 $500$ 个自然数的乘积可表示为:
\[1+2+3+\dots+499+500 = \prod_{i=1}^{500}i\]
大写希腊字母 $\Pi$ 对应英文字母 \verb|P|,表示``求积 (Product)''。再次尝试通改变通项表达式和索引范围,以不同方式表达上面公式。如何表示前 $500$ 个偶数之积?如何表示 $500$(含)以内所有偶数之积?尝试用上面的求积符号写出这些求积。

\subsubsection*{习题}

\begin{problem}
    用自然语言描述等式含义:
    \[\sum_{i=1}^{n}i^2 = \frac{n(n+1)(2n+1)}{6}\]
\end{problem}
\begin{problem}
    用适当的符号表示从 $2^0=1$ 开始,$2$ 的前 $n$ 次幂之和与之积。能否证明其求和与求积公式?
\end{problem}
\begin{problem}
    求 $17$ 到 $33$ 之间(含端点)所有奇数之和。分别以索引起点 $0,1,8,9$ 用求和符号表示该求和公式。哪一个感觉``更自然''?为什么?
\end{problem}