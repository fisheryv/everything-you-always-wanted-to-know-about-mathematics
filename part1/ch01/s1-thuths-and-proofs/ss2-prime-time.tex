% !TeX root = ../../../book.tex
\subsection{质数时间}\label{sec:section1.1.2}

当我们讨论证明这个主题时,让我们看一下另一个不同定理的证明。作为提醒(或简要介绍),我们来谈谈\emph{质数}。

\subsubsection*{定义、示例和使用}

\begin{definition}\label{def:prime}
    如果大于 $1$ 的正整数 $p$ 其正因子只有 $1$ 和 $p$,则称 $p$ 为\dotuline{质数}。非质正整数称为\dotuline{合数}。
\end{definition}

质数已被证明在数学的所有分支中都非常重要,而不仅仅是对整数及其性质的研究,即\textbf{数论}。数学中最著名的\textbf{猜想}(迄今为止既没有被证明也没有被证伪的定理的猜测)当数\emph{黎曼猜想}。其结论已被证明与全体整数中质数的分布密切相关。围绕这个主题已经写了很多书。此外,大多数现代密码学都是基于将巨大质数相乘,因为它们的乘积很难逆向分解为两个巨大质数因子。现在你知道了:每次你用信用卡在 iTunes 上购买歌曲时,某些计算机都会将两个大质数相乘!

前几个质数是 $2, 3, 5, 7, 11, 13, 17, 19, 23,\dots$(记住,$1$ 不满足定义)。质数有多少个?他们相距多远?存在模式吗?回答这些问题可能很有趣,但也很困难(有时甚至是不可能的!)。这里,我们将回答其中一个问题:质数是否有无穷多个?

\subsubsection*{定理和证明}

\begin{theorem}[质数无限性]
    质数有无穷多个。
\end{theorem}

\begin{proof}
    假设质数有有限多个,并按升序列出:$p_1, p_2, p_3, \dots, p_k$,所以 $p_k$ 是所有质数中最大的。定义新数
    \[N = (p_1 \cdot p_2 \cdot p_3 \cdot \dots \cdot p_k) + 1\]
    $N$ 一定能被某个质数整除。然而,它一定不能被 $p_1$ 或 $p_2$ 或 $\dots$ 或 $p_k$ 整除,因为根据 $N$ 的定义都会有余数 $1$。因此 $N$ 可以被列表中未列出的其他质数整除。

    如果 $N$ 本身是合数(即不是质数),那么我们就发现了一些新质数 $p < N$,它不在我们的所有质数列表中。如果 $N$ 本身是质数,那么我们就有了一个新质数 $N > p_k$,所以 $p_k$ 实际上并不是最大的质数。不管哪种情况,我们保证有一个新的质数不会出现在给定的 $k$ 个质数列表中。因此,质数必定有无穷多个。
\end{proof}

对于这个``证明''你怎么看?你确信吗?感觉与我们迄今为止看到的其他论证有点不同,不是吗?尝试向同学解释这个证明与上一节毕达哥拉斯定理的``证明 1''有何不同。我们很快将揭示:这里的``证明''实际上是一个完全正确的\emph{证明},没有引号!
    