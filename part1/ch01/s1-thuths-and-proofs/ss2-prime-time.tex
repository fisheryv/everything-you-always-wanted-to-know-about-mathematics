% !TeX root = ../../../book.tex
\subsection{质数时间}\label{sec:section1.1.2}

当我们讨论证明这个主题时,让我们看一下另一个重要定理的证明。首先简要回顾\emph{质数}的定义。

\subsubsection*{定义、示例与应用}

\begin{definition}\label{def:prime}
    若正整数 $p>1$ 的正因子只有 $1$ 和 $p$ 本身,则称 $p$ 为\dotuline{质数}。非质正整数称为\dotuline{合数}。
\end{definition}

质数在数学的诸多分支中都具有核心地位,而不仅限于研究整数性质的\textbf{数论}领域。数学中最著名的\textbf{猜想}(迄今为止既没有被证明也没有被证伪的命题)当数\emph{黎曼猜想}。该猜想已被证实与质数在整数中的分布规律密切相关,相关研究著作浩如烟海。此外,大多数现代密码学的基础都是大质数的乘积特性 --- 将两个大质数相乘容易,但将乘积逆向分解为两个大质数却极为困难。现在你知道了:每当你用信用卡在 iTunes 上购买歌曲时,背后的计算机系统都会将两个大质数相乘!

前几个质数依次为 $2, 3, 5, 7, 11, 13, 17, 19, 23,\dots$(注意,$1$ 不满足定义)。那么质数有多少个?相邻质数相距多远?是否存在分布规律?这些问题虽引人入胜却极难解答(有时甚至是不可能的!)。本节将聚焦其中一个问题:质数是否有无穷多个?

\subsubsection*{定理与证明}

\begin{theorem}[质数无穷性]
    质数有无穷多个。
\end{theorem}

\begin{proof}
    假设质数有有限多个,按升序列出为:$p_1, p_2, p_3, \dots, p_k$,所以 $p_k$ 为最大质数。现构造新数
    \[N = (p_1 \cdot p_2 \cdot p_3 \cdot \dots \cdot p_k) + 1\]
    $N$ 必有质因子。然而,它一定不能被 $p_1$ 或 $p_2$ 或 $\dots$ 或 $p_k$ 整除,因为根据 $N$ 的定义,除以上述质数都会得到余数 $1$。因此 $N$ 可以被列表中未列出的其他质数整除。

    如果 $N$ 是合数,那么我们就发现了一些新质数 $p < N$,它不在所有质数列表中。
    
    如果 $N$ 是质数,那么我们就得到了一个新质数 $N > p_k$,所以 $p_k$ 实际上并不是最大的质数。无论哪种情况,我们必然有一个新的质数不在给定的 $k$ 个质数的列表中,这与有限性假设矛盾。因此,质数必定有无穷多个。
\end{proof}

对于这个``证明''你怎么看?是否令你信服?这里的论证方式与我们迄今为止看到的其他论证方法有所不同。不妨尝试向同学解释本节的证明与上一节毕达哥拉斯定理的``证明 1''有何不同。我们很快将揭示:此处的``证明''实际上是一个完全正确的\emph{证明}!
