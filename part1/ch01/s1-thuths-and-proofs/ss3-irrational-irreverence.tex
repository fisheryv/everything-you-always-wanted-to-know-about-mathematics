% !TeX root = ../../../book.tex
\subsection{无理取闹}

现在让我们讨论另一种类型的数字:\textbf{有理}数。你可能知道``分数''、``商''或``比率''都是有理数。

\subsubsection*{定义与示例}

下面是\emph{有理}数的精确定义:

\begin{definition}
    实数 $r$ 是\dotuline{有理}数当且仅当它可以表示为两个整数之比 $r = \frac{a}{b}$,其中 $a$ 和 $b$ 都是整数(且 $b \ne 0$)。

    一个实数不是有理数就是\dotuline{无理数}。
\end{definition}

这个定义中没有任何内容表明有理数必须只有一种定义中的表示形式;它只要求有理数至少有一种定义中的表示。例如,$1.5$ 是有理数因为 $1.5 = \frac{3}{2} = \frac{12}{8} = \frac{30}{20}$ 等等。一个实数不是有理数就是\textbf{无理}数,这就是完整定义:\emph{非}有理数,即不存在将数字表示为整数之比的形式。你可能知道 $\sqrt{2}$ 是一个无理数,但你要如何\emph{证明}这一点呢?自己尝试证明一下。我们稍后会重新审视这个问题(见示例 \ref{sec:section4.9.4})。其他你可能知道的无理数还包括 $e, \pi, \varphi$ 和 $\sqrt{n}$,其中 $n$ 为正整数且不为完全平方数。

\subsubsection*{问题}

给定有理数/无理数的定义,我们可能想知道如何组合无理数来产生有理数。尝试自己回答以下问题。如果你的答案为``是'',请尝试找到一个例子,如果你的答案为``否'',请尝试解释为什么要求的情况不可能实现。

\begin{enumerate}
    \item 是否存在无理数 $a$ 和 $b$ 使得 $a \cdot b$ 为有理数?
    \item 是否存在无理数 $a$ 和 $b$ 使得 $a + b$ 为有理数?
    \item 是否存在无理数 $a$ 和 $b$ 使得 $a^b$ 为有理数?
\end{enumerate}

你能找出例子吗?事实证明,这三个问题的答案都是``是''!前两个不太难理解,但第三个有点棘手。

这里,我们给出证明来证明第三个问题的答案为``是''。有趣的是,我们实际上不会给出满足 $a^b$ 为有理数的 $a$ 和 $b$ 的确切数字;我们只是将其范围缩小到两种可能的选择,并证明其中任意一种选择\emph{必定}有效。听起来很有趣,对吧?我们来试试吧。

\begin{proof}
    我们知道 $\sqrt{2}$ 为无理数。考虑数字 $x = \sqrt{2}^{\sqrt{2}}$。有两种可能的情况:
    \begin{itemize}
        \item 如果 $x$ 为有理数,那么我们可以令 $a = \sqrt{2}, b = \sqrt{2}$ 即可得到答案。
        \item 如果 $x$ 为无理数,那么我们可以令 $a = \sqrt{2}^{\sqrt{2}}, b = \sqrt{2}$,则
        \[a^b = \Bigg(\sqrt{2}^{\sqrt{2}}\Bigg)^{\sqrt{2}} = \Big(\sqrt{2}\Big)^{\sqrt{2} \cdot \sqrt{2}} = \Big(\sqrt{2}\Big)^2 = 2\]
        $2$ 为有理数。
    \end{itemize}
    任意一种情况,我们都能找到无理数 $a$ 和 $b$ 使得 $a^b$ 为有理数。因此,这样的数字对一定存在。
\end{proof}

你觉得这个证明怎么样?有说服力吗?它以明确的``是''回答了上面的第三个问题,但它并没有告诉我们\emph{哪}一对 $a, b$ 实际上是正确的,而只是告诉我们其中有一对有效。(事实证明 $\sqrt{2}^{\sqrt{2}}$ 也是无理数,但这一事实需要更多的工作来证明。)

还有很多其他具体例子可以回答这个问题。你能想出任何其他方法吗?(提示:尝试使用 $\log_{10}$ 函数……)
