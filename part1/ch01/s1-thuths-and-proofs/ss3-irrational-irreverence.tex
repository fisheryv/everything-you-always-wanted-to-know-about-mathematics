% !TeX root = ../../../book.tex
\subsection{无理取闹}

现在让我们讨论另一类数字:\textbf{有理数}。你可能知道``分数''、``商''或``比率''指的都是有理数。

\subsubsection*{定义与示例}

以下是\emph{有理数}的精确定义:

\begin{definition}
    实数 $r$ 是\dotuline{有理数}当且仅当它可以表示为两个整数之比 $r = \frac{a}{b}$,其中 $a$ 和 $b$ 均为整数(且 $b \ne 0$)。

    一个实数不是有理数就是\dotuline{无理数}。
\end{definition}

定义并未规定有理数必须具有唯一的表示形式;它只要求有理数至少有一种定义中的表示。例如,$1.5$ 是有理数,因为 $1.5 = \frac{3}{2} = \frac{12}{8} = \frac{30}{20}$ 等等。这一定义具有完备性:一个实数不是有理数就是\textbf{无理数},\emph{非}有理数,即不存在将其表示为整数之比的形式。你可能知道 $\sqrt{2}$ 是一个无理数,但如何\emph{证明}这一点呢?请尝试自行证明。我们稍后会重新讨论这个问题(见示例 \ref{sec:section4.9.4})。其他常见的无理数还包括 $e, \pi, \varphi$ 以及 $\sqrt{n}$(其中 $n$ 为正整数且不为完全平方数)。

\subsubsection*{问题}

基于有理数/无理数的定义,我们可以探讨无理数的组合能否产生有理数。请尝试独立解答以下问题:若答案为"是",请举例说明;若答案为"否",请解释其不可能性。

\begin{enumerate}
    \item 是否存在无理数 $a$ 和 $b$ 使得 $a \cdot b$ 为有理数?
    \item 是否存在无理数 $a$ 和 $b$ 使得 $a + b$ 为有理数?
    \item 是否存在无理数 $a$ 和 $b$ 使得 $a^b$ 为有理数?
\end{enumerate}

你能构造出具体例子吗?事实上,这三个问题的答案均为``是''!前两问较为简单,第三问则略微棘手。

以下我们给出第三问的证明。有趣的是,该证明并不直接给出满足条件的 $a$ 和 $b$;而是将可能性缩小至两种情形,并证明其中\emph{必有}一种成立。听起来很有趣,对吧?我们来试试吧。

\begin{proof}
    已知 $\sqrt{2}$ 为无理数。考虑数字 $x = \sqrt{2}^{\sqrt{2}}$。有两种可能的情况:
    \begin{itemize}
        \item 若 $x$ 为有理数,则令 $a = \sqrt{2}, b = \sqrt{2}$ 即满足条件。
        \item 若 $x$ 为无理数,则令 $a = \sqrt{2}^{\sqrt{2}}, b = \sqrt{2}$,此时
        \[a^b = \Bigg(\sqrt{2}^{\sqrt{2}}\Bigg)^{\sqrt{2}} = \Big(\sqrt{2}\Big)^{\sqrt{2} \cdot \sqrt{2}} = \Big(\sqrt{2}\Big)^2 = 2\]
        $2$ 为有理数。
    \end{itemize}
    无论何种情形,均可找到无理数 $a$ 和 $b$ 使得 $a^b$ 为有理数。因此,这样的数字对必然存在。
\end{proof}

你觉得这个证明怎么样?是否有说服力?它以明确的``是''回答了上面的第三问,但未指明具体\emph{哪一对} $a, b$ 是正确的,而是告诉我们其中必有一对满足。(事实证明 $\sqrt{2}^{\sqrt{2}}$ 确为无理数,但这一事实需要额外证明。)

还有很多其他具体例子可以回答这个问题。你能想出任何其他方法吗?(提示:尝试使用 $\log_{10}$ 函数……)
