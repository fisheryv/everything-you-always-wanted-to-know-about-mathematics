% !TeX root = ../../../book.tex
\section{本章习题}

本节习题涵盖本章全部内容,并涉及先前知识点及部分数学假设。我们不要求你解答\textbf{所有}题目,但解决得越多,收获越大!请牢记:真正\emph{掌握}数学必须亲自\emph{实践}。尝试动手解题,仔细阅读并思考题意。撰写证明并与朋友讨论,检验其说服力。持续练习如何清晰、准确、有条理地\emph{书写}思路。完成证明后要反复修改以臻完善。最重要的是,坚持\emph{钻研}数学!

标有 $\blacktriangleright$ 的简答题只需解释或陈述答案,无需严格证明。

特别具有挑战性的问题标记为 $\bigstar$。

\begin{exercise}
    $\blacktriangleright$ 考虑全集 $U=\mathbb{Z}$。

    令 $P(x)$ 表示命题 ``$1 \le x \le 3$''。

    令 $Q(x)$ 表示命题 ``$\exists k \in \mathbb{Z} \centerdot x=2k$''。

    令 $R(x)$ 表示命题 ``$x^2=4$''。

    令 $S(x)$ 表示命题 ``$x=1$''。

    对于以下每个陈述,用自然语言描述该陈述的含义,然后写出其逻辑否定,并判断其为\verb|真|还是为\verb|假|,以及为什么。

    \begin{enumerate}[label=(\alph*)]
        \item $\forall x \in \mathbb{Z} \centerdot P(x) \implies Q(x)$
        \item $\exists x \in \mathbb{Z} \centerdot R(x) \land P(x)$
        \item $\forall x \in \mathbb{Z} \centerdot R(x) \implies P(x)$
        \item $\forall x \in \mathbb{Z} \centerdot \exists y \in \mathbb{Z} \centerdot x \ne y \land P(x) \land R(y)$
        \item $\forall x \in \mathbb{Z} \centerdot \exists y \in \mathbb{Z} \centerdot (S(x) \lor Q(x)) \land P(y) \land \neg Q(y)$
        \item $\exists x \in \mathbb{Z} \centerdot S(x) \iff P(x) \land \neg Q(x)$
        \item $\exists x \in \mathbb{Z} \centerdot S(x) \iff \neg P(x) \land Q(x)$
    \end{enumerate}
\end{exercise}

\begin{exercise}
    对于以下每个声明,定义一些集合和变量命题,用简洁的符号、逻辑符号表达该声明。然后,写出其逻辑否定。请注意哪个为\verb|真|哪个为\verb|假|。

    \begin{enumerate}[label=(\alph*)]
        \item 每个奇数自然数都是质数。
        \item 存在一个严格大于任意整数平方的实数。
        \item $-1$ 到 $1$ 之间的某个实数具有等于 $-1$ 到 $1$ 之间的某个不同实数的立方的性质。
        \item 质数倍数集合的并集就是自然数集本身。
    \end{enumerate}  
\end{exercise}

\begin{exercise}
    考虑下面定义的集合以及有关这些集合的问题。对于每个问题,如果您的答案是\verb|否|,请提供一个示例来证明这一点。

    \begin{enumerate}[label=(\alph*)]
        \item 设 $S = \{1, 2, 3, 4\}$ 且 $T = \{3, 4, 5, 6, 7, 8\}$。\\
            $\forall s \in S \centerdot \exists t \in T \centerdot s + t = 7$ 为\verb|真|吗?
        \item 设 $S = \{2, 3, 4, 5, 6\}$ 且 $T = \{3, 4, 5, 6\}$。\\
            $\forall s \in S \centerdot \exists t \in T \centerdot s + t = 7$ 为\verb|真|吗?
        \item 设 $S = \mathbb{N}$ 且 $T = \mathbb{Z}$。\\
            $\forall s \in S \centerdot \exists t \in T \centerdot s + t = 7$ 为\verb|真|吗?
        \item 设 $S = \mathbb{Z}$ 且 $T = \mathbb{N}$。\\
            $\forall s \in S \centerdot \exists t \in T \centerdot s + t = 7$ 为\verb|真|吗?
    \end{enumerate}  
\end{exercise}

\begin{exercise}
    考虑下面定义的集合以及有关这些集合的问题。对于每个问题,如果您的答案是\verb|否|,请提供一个示例来证明这一点。

    \begin{enumerate}[label=(\alph*)]
        \item 设 $S = \{1, 2, 3\}, T = \{6, 7, 8, 9\}, V = \{7, 8, 9, 10\}$。\\
            $\exists s \in S \centerdot \forall t \in T \centerdot \exists v \in V \centerdot s + t = v$ 为\verb|真|吗?
        \item 设 $S = \{1, 2, 3\}, T = \{4, 5, 6, 7\}, V = \{5, 6, 7, 9, 10, 11\}$。\\
            $\exists s \in S \centerdot \forall t \in T \centerdot \exists v \in V \centerdot s + t = v$ 为\verb|真|吗?
        \item 设 $S = T = V = \mathbb{N}$。\\
            $\exists s \in S \centerdot \forall t \in T \centerdot \exists v \in V \centerdot s + t = v$ 为\verb|真|吗?
        \item 设 $S = \mathbb{N}, T = \mathbb{Z}$, $V = \mathbb{N}$。\\
            $\exists s \in S \centerdot \forall t \in T \centerdot \exists v \in V \centerdot s + t = v$ 为\verb|真|吗?
    \end{enumerate}  
\end{exercise}

\begin{exercise}
    严格证明或反驳下列声明:
    \[\exists x \in \mathbb{R} \centerdot \forall y \in \mathbb{R} \centerdot x^2 - y^2 \ge 0\]
\end{exercise}

\begin{exercise}
    严格证明或反驳下列声明:
    \[\forall x, y \in \mathbb{Z} \centerdot \exists z \in \mathbb{N} \cup \{0\} \centerdot ((x - y) = z) \lor ((y - x) = z)\]
\end{exercise}

\begin{exercise}
    证明方程 $x^2-y^2=14$ 无整数解(即 $x, y \in \mathbb{Z}$)。
\end{exercise}

\begin{exercise}
    使用\emph{逻辑等价}证明:

    \begin{enumerate}[label=(\alph*)]
        \item $(A \cup B) \cap \bar{A} = B - A$
        \item $A \cap (B - C) = (A \cap B) - (A \cap C)$
    \end{enumerate}  
\end{exercise}

\begin{exercise}
    定义集合
    \[A = \Big\{(x, y) \in \mathbb{R} \times \mathbb{R} \mid \frac{x}{y}+\frac{y}{x} \ge 2\Big\}\]
    \[B = \{(x, y) \in \mathbb{R} \times \mathbb{R} \mid (x>0 \land y>0) \lor (x<0 \lor y<0)\}\]
    是不是 $A \subseteq B$ 呢?如果是,请证明它。如果不是,请举出反例。\\
    是不是 $B \subseteq A$ 呢?如果是,请证明它。如果不是,请举出反例。
\end{exercise}

\begin{exercise}
    设 $P = \{y \in \mathbb{R} \mid y > 0\}$ 为正实数集合,证明如下声明:
    \[\forall \varepsilon \in P \centerdot \exists \delta \in P \centerdot \forall x \in \mathbb{R} \centerdot |x| < \delta \implies|x^2| < \varepsilon\]
\end{exercise}

\begin{exercise}
    下面对 $\mathcal{P}(C \cup D) = \mathcal{P}(C) \cup \mathcal{P}(D)$ 声明的``证明''有什么问题?
    \begin{quote}
        令 $X \in \mathcal{P}(C \cup D)$ 是任意固定的。

        这意味着 $X \subseteq C \cup D$。

        于是,$X \subseteq C \lor X \subseteq D$。

        那么,$X \in \mathcal{P}(C) \lor X \in \mathcal{P}(D)$。
        
        那么,$X \in \mathcal{P}(C) \cup \mathcal{P}(D)$。

        所以,$\mathcal{P}(C \cup D) = \mathcal{P}(C) \cup \mathcal{P}(D)$。
    \end{quote}
\end{exercise}

\begin{exercise}
    假设 $x \in \mathbb{Z}$ 且 $x^2$ 是 $8$ 的倍数。证明 $x$ 是偶数。该命题的逆命题为\verb|真|吗?如果是,请证明;如果不是,请举出反例。
\end{exercise}

\begin{exercise} \label{exc:exercises4.11.13}
    定义命题 $E(z)$ 为``$z$ 为偶数。证明
    \[\forall z \in \mathbb{Z} \centerdot E(z) \iff E(z^3)\]
\end{exercise}

\begin{exercise}
    利用习题 \ref{exc:exercises4.11.13} 的结论证明 $\sqrt[3]{2}$ 为无理数。
\end{exercise}

\begin{exercise}
    设 $P = \{y \in \mathbb{R} \mid y > 0\}$。证明
    \[\bigcap_{x \in P} \Big\{y \in \mathbb{R} \mid 1-\frac{1}{x} < y < 2\Big\} = \{z \in \mathbb{R} \mid 1 < z <2\}\]
\end{exercise}

\begin{exercise}
    定义集合 $A, B, C \subseteq U$。定义
    \[S = \big((A \cap \bar{B}) \cup C \big)-A\]
    \[T = C-(A \cup B)\]
    是不是 $S \subseteq T$ 呢?如果是,请证明它。如果不是,请举出反例。\\
    是不是 $T \subseteq S$ 呢?如果是,请证明它。如果不是,请举出反例。
\end{exercise}

\begin{exercise}
    对于任意 $x \in \mathbb{R}$,定义集合
    \[S_x = \{y \in \mathbb{R} \mid -x \le y \le x\}\]
    \[P = \{y \in \mathbb{R} \mid y > 0\}\]
    证明下列声明:
    \begin{align*}
        &\bigcap_{x \in P} S_x = \{0\} \\
        &\bigcap_{x \in \mathbb{N}} S_x = \{y \in \mathbb{R} \mid -1 \le y \le 1\}
    \end{align*}
\end{exercise}

\begin{exercise} \label{exc:exercises4.11.22}
    设 $P$ 为变量命题
    \[\frac{x^2+4}{x^2+1} < 1+\frac{1}{x}\]
    设 $Q$ 为变量命题
    \[\frac{x^2-4}{x^2+1} > 1-\frac{1}{x}\]
    设 $S = \{x \in \mathbb{R} \mid x > 0\}$ 为正实数集合。

    对于以下每个陈述,确定其为\verb|真|(在这种情况下,提供证明)还是为\verb|假|(在这种情况下,提供反例并证明其有效的原因)。

    \begin{enumerate}[label=(\alph*)]
        \item $\forall x \in S \centerdot P(x) \implies Q(x)$
        \item $\forall x \in S \centerdot Q(x) \implies P(x)$
    \end{enumerate}  
\end{exercise}

\begin{exercise}
    设 $A, B$ 为两个任意集合。证明
    \[A \times B = B \times A \iff (A = B \lor A = \varnothing \lor B = \varnothing)\]
    (不要忘记这里需要证明的是\emph{当且仅当}声明!)
\end{exercise}

\begin{exercise}
    设 $A, B, C, D$ 为任意集合。证明
    \[(A \times B) \cap (C \times D) = \varnothing \iff (A \times B = \varnothing \lor C \times D = \varnothing) \]
\end{exercise}

\begin{exercise}
    设 $B$ 为任意集合。设 $I$ 为索引集,设 $A_i$ 为对于每个 $i \in I$ 的集合。证明以下集合等式:

    \begin{enumerate}[label=(\alph*)]
        \item $\displaystyle \Bigg(\bigcap_{i \in I} A_i \Bigg) - B = \bigcap_{i \in I} (A_i - B)$
        \item $\displaystyle \Bigg(\bigcup_{i \in I} A_i \Bigg) - B = \bigcup_{i \in I} (A_i - B)$
        \item $\displaystyle \Bigg(\bigcap_{i \in I} A_i \Bigg) \times B = \bigcap_{i \in I} (A_i \times B)$
        \item $\displaystyle \Bigg(\bigcup_{i \in I} A_i \Bigg) \times B = \bigcup_{i \in I} (A_i \times B)$
        \item $\displaystyle B - \Bigg(\bigcap_{i \in I} A_i \Bigg) = \bigcup_{i \in I} (B - A_i)$
        \item $\displaystyle B - \Bigg(\bigcup_{i \in I} A_i \Bigg) = \bigcap_{i \in I} (B - A_i)$
    \end{enumerate}  
\end{exercise}

\begin{exercise}\label{exc:exercises4.11.26}
    在本题中,你将证明有理数 $\mathbb{Q}$ 是\textbf{稠密的}。也就是说,我们希望你考虑如下命题:

    \begin{quote}
        \emph{命题:在任意两个不同的有理数之间严格存在另一个有理数。}
    \end{quote}
    用逻辑符号重述这个声明,然后\textbf{证明}它。
\end{exercise}

\begin{exercise}
    这个问题的目的是引入\textbf{唯一性}的概念。如果一个对象具有某种其他对象都不具有的特定属性,则我们说该对象是\textbf{唯一的}。

    也就是说,我们可以说 $x$ 是 $S$ 中具有属性 $P(x)$ 的唯一元素当且仅当
    \[\exists x \in S \centerdot P(x) \land (\forall y \in S \centerdot y \ne x \implies \neg P(y))\]
    请注意,这在逻辑上等价于
    \[\exists x \in S \centerdot P(x) \land (\forall y \in S - \{x\} \centerdot \neg P(y))\]
    此外,我们也可以写出其逆否命题:
    \[\exists x \in S \centerdot P(x) \land (\forall y \in S \centerdot P(y) \implies x = y)\]
    根据以上信息用逻辑符号重述以下声明。然后给出证明。

    \textbf{声明}:方程 $n^3 - n - 6 = 0$ 具有唯一的自然数根。
\end{exercise}

\begin{exercise}
    这个问题提供了新的集合运算的定义(基于其他运算定义),并要求你使用此运算证明若干集合包含和相等性。

    \textbf{定义}:设 $A,B$ 为集合。$A$ 和 $B$ 的\textbf{对称差}用 $A \Delta B$ 表示,定义为
    \[A \Delta B = (A - B) \cup (B - A)\]
    现在,设 $A,B,C$ 为任意集合。证明如下命题:

    \begin{enumerate}[label=(\alph*)]
        \item $A \Delta A = \varnothing$
        \item $A \Delta B = B \Delta A$
        \item $A \Delta \varnothing = \varnothing$
        \item $A \subseteq B \implies A \Delta B = B - A$
        \item $A \Delta (B \Delta C) = (A \Delta B) \Delta C$
        \item $\bar{A} \Delta \bar{B} = A \Delta B$(假设 $A, B \subseteq U$)
        \item $ (A \Delta B) \cap C = (A \cap C) \Delta (B \cap C) = (A \Delta B) - C$
    \end{enumerate} 
\end{exercise}

\begin{exercise}
    在这个问题中,你将证明有关\textbf{素性测试}的有用结论。许多现代密码学都基于将大合数分解为其质因数。下面的命题表明对于 $p$ 的因数,我们只需要检查到 $\sqrt{p}$ 即可判断 $p$ 是否是质数。

    \begin{quote}
        \emph{命题:设 $p$ 为大于等于 $2$ 的自然数。如果 $2$ 和 $\sqrt{p}$(含)之间的自然数都不能整除 $p$,则 $p$ 为质数。}
    \end{quote}
    回想一下:``$\mid$'' 的正式定义为
    \begin{center}
        给定 $a, b \in \mathbb{Z}$,我们写 $a \mid b$ 当且仅当 $\exists k \in \mathbb{Z} \centerdot b = ak$。
    \end{center}
    用逻辑符号重述上述命题。然后证明它。\\
    (提示:思考该命题的逆否形式……)
\end{exercise}

\begin{exercise}
    设 $S, T$ 为集合,其元素也是集合。对于以下陈述,\textbf{证明}其成立,或给出反例:

    \begin{enumerate}[label=(\alph*)]
        \item $\displaystyle \bigcup_{X \in S \cup T} X \subseteq \Bigg(\bigcup_{Y \in S} Y\Bigg) \cup \Bigg(\bigcup_{Z \in T} Z\Bigg)$
        \item $\displaystyle \bigcup_{X \in S \cup T} X \supseteq \Bigg(\bigcup_{Y \in S} Y\Bigg) \cup \Bigg(\bigcup_{Z \in T} Z\Bigg)$
        \item $\displaystyle \bigcap_{X \in S \cup T} X \subseteq \Bigg(\bigcap_{Y \in S} Y\Bigg) \cap \Bigg(\bigcap_{Z \in T} Z\Bigg)$
        \item $\displaystyle \bigcap_{X \in S \cup T} X \supseteq \Bigg(\bigcap_{Y \in S} Y\Bigg) \cap \Bigg(\bigcap_{Z \in T} Z\Bigg)$
    \end{enumerate} 
\end{exercise}