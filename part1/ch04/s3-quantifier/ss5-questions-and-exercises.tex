% !TeX root = ../../../book.tex
\subsection{习题}

\subsubsection*{温故知新}

以口头或书面的形式简要回答以下问题。这些问题全都基于你刚刚阅读的内容,如果忘记了具体定义、概念或示例,可以回顾相关内容。确保在继续学习之前能够自信地作答这些问题,这将有助于你的理解和记忆!

\begin{enumerate}[label=(\arabic*)]
    \item $\forall$ 和 $\exists$ 有什么不同?
    \item 如何朗读下面的陈述?
        \[\forall x \in \mathbb{R} \centerdot \exists y \in \mathbb{R} \centerdot x = y^3\]
    \item 下面的语句为什么不是正确的数学陈述?
        \[\exists y \centerdot y + 3 > 10\]
        以下两个陈述之间有何区别(如果有的话)?
        \[\exists x \in \mathbb{N} \centerdot \exists y \in \mathbb{N} \centerdot x + y = 5\]
        \[\exists x, y \in \mathbb{N} \centerdot x + y = 5\]
        它们是\verb|真|还是\verb|假|?
    \item 以下两个陈述之间有何区别(如果有的话)?
        \[\exists a, b \in \mathbb{Z} \centerdot a \cdot b = -3\]
        \[\exists \heartsuit, \diamondsuit \in \mathbb{Z} \centerdot \heartsuit \cdot \diamondsuit = -3\]
        它们是\verb|真|还是\verb|假|?
    \item 为什么下列语句不是正确的量化陈述?
        \textcolor{red}{\begin{itemize}
            \item $\exists x \centerdot x > 7$
            \item $\forall y \in \mathbb{Z}$
            \item $\forall z > 2 \centerdot z^2 > 4$
            \item $\forall w \in \mathbb{Z} \centerdot w^2 = t \centerdot \exists t \in \mathbb{N}$
        \end{itemize}}
\end{enumerate}

\subsubsection*{小试牛刀}

尝试解答以下问题。这些题目需动笔书写或口头阐述答案,旨在帮助你熟练运用新概念、定义及符号。题目难度适中,确保掌握它们将大有裨益!

\begin{enumerate}[label=(\arabic*)]
    \item 回顾 \ref{sec:section4.3.3} 节,我们用符号表示法表达了哥德巴赫猜想。定义 $X$ 为除 $2$ 之外的所有偶数的集合。请使用符号、量词和集合构造符(可能需要集合运算,取决于具体写法)写出 $X$ 的定义。
    \item 写一个以量词开头的数学陈述的示例:若量词为``$\exists$'',则陈述为\verb|真|;若量词为``$\forall$'',则陈述为\verb|假|。
    \item 写一个命题函数 $P(x)$ 的例子,使得
        \[\forall x \in \mathbb{N} \centerdot P(x)\]
        为\verb|真|,但
        \[\forall x \in \mathbb{Z} \centerdot P(x)\]
        为\verb|假|。
    \item 对于以下每个数学陈述,请用量词将其写成符号形式(首先正确定义可能需要的命题函数)。然后判断该陈述是\verb|真|还是\verb|假|。
        \begin{enumerate}[label=(\alph*)]
            \item 存在一个严格大于每个整数的实数。
            \item 每个整数都满足其平方小于或等于其立方。
            \item 每个自然数的平方根都是实数。
            \item $\mathbb{N}$ 的每个子集都含有数字 $3$。
        \end{enumerate} 
    \item 阅读以下每个量化陈述的符号表示并朗读出来。然后判断该陈述是\verb|真|还是\verb|假|。
        \begin{enumerate}[label=(\alph*)]
            \item $\forall x \in \mathbb{N} \centerdot \exists y \in \mathbb{Z} \centerdot x + y < 0$
            \item $\exists x \in \mathbb{N} \centerdot \forall y \in \mathbb{Z} \centerdot x + y < 0$
            \item $\exists A \in \mathcal{P}(\mathbb{Z}) \centerdot \mathbb{N} \subset A \subset \mathbb{Z}$\\(回想一下,$\subset$ 的意思是``是……的真子集''。)
            \item 设 $P$ 为质数集。\\
                    $\forall x \in P \centerdot \exists t \in \mathbb{Z} \centerdot x = 2t + 1$
            \item $\forall a \in \mathbb{N} \centerdot \exists b \in \mathbb{Z} \centerdot \forall c \in \mathbb{N} \centerdot a + b < c$
            \item $\exists b \in \mathbb{Z} \centerdot \forall a, c \in \mathbb{N} \centerdot a + b < c$
        \end{enumerate} 
\end{enumerate}   