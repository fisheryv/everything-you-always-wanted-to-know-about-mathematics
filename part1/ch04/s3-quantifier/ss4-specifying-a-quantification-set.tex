% !TeX root = ../../../book.tex
\subsection{指定量化集}

关于量词还有一点我们需要强调,每当使用量词时,我们都必须指定一个集合。下面这句话
\[\forall x \centerdot x^2 \ge 0\]
可能``看上去为\verb|真|'',但实际上\textbf{毫无意义}。$x$ 是什么?它从何而来?``对于每一个 $x$ …… ''从哪里获取?如果 $x$ 不是数字怎么办?

我们需要指定对象 $x$ ``来自''哪里,以便我们知道 $x^2 \ge 0$ 是否是一个明确定义、符合语法的短语,更不用说它是否为\verb|真|。如果我们将句子修改为
\[\forall x \in \mathbb{R} \centerdot x^2 \ge 0\]
那么这就是一个定义明确、符合语法(并且正确!)的数学陈述。但是,如果我们将句子修改为
\[\forall x \in \mathbb{C} \centerdot x^2 \ge 0\]
那么这就是一个定义明确但错误的数学陈述!这是因为 $i \in \mathbb{C}$ 但 $i^2 = -1 < 0$。(请记住,本书中我们不会大量使用复数 $\mathbb{C}$ 的集合,但它提供了一些有趣且具有启发性的示例,就像上面这个。)

这里的主要教训是\textbf{上下文}确实很重要。它可以改变陈述的含义及其真值。因此,我们必须始终确保指定一个从中提取变量值的集合。

\subsubsection*{一个例外}

打脸来得好快,我们不得不承认``永远指定量化集''这一规则有一个例外,但这个例外是有充分理由的。考虑以下声明:
\begin{center}
    对于任意集合 $A, B,C$,等式 $(A \cup B) \cap C = (A \cap C) \cup (A \cap B)$ 成立。
\end{center}
这个数学陈述为\verb|真|。(你能证明这一点吗?尝试使用双重包含论证!)

我们如何以符号形式写出这个陈述呢?这是一个\emph{全称}量化(``对于任意……''),因此我们需要使用 ``$\forall$'' 符号。这里的变量(用 $A,B,C$ 表示)是\emph{集合}。他们来自哪里?我们将从什么集合中提取它们?

我们非常确定你想说``所有集合的集合''。但这存在一个大问题!还记得我们在上一章讨论的罗素悖论吗?(请参阅第 \ref{sec:section3.3.5} 节。)所有可能集合的集合本身并\emph{不是}一个集合!因此,我们不能将这个陈述符号化地写成
\[\forall A, B,C \in \underline{\qquad} \centerdot (A \cup B) \cap C = (A \cap C) \cup (A \cap B)\]
因为我们不知道如何用\emph{集合}来填补空白。

由于这个问题,我们将继续使用``对于任意集合 $A,B,C$ ……''之类的短语,而不是用符号形式表达。当我们自己做笔记或在草稿纸上解题时,可以随意写下 ``$\forall A,B,C$'',并且知道它其实代表了集合的量化。然而,当更正式地书写时(例如,书面作业),你应该坚持使用上面的措辞。
