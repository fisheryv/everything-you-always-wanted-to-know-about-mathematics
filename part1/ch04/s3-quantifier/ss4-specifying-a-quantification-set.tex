% !TeX root = ../../../book.tex
\subsection{指定量化集}

关于量词还有一点需要强调:每当使用量词时,都必须指定一个集合。下面这句话
\[\forall x \centerdot x^2 \ge 0\]
可能``看似为\verb|真|'',但实际上\textbf{毫无意义}。$x$ 是什么?它从何而来?``对于每一个 $x$……''中的 $x$ 取自哪里?如果 $x$ 不是数字怎么办?

我们需要指明对象 $x$ 的``来自'',才能判断 $x^2 \ge 0$ 是否是一个定义明确、语法正确的命题,更不必说其真假。若将句子改为
\[\forall x \in \mathbb{R} \centerdot x^2 \ge 0\]
这便是定义清晰、语法正确(且成立)的数学陈述。但如果改为
\[\forall x \in \mathbb{C} \centerdot x^2 \ge 0\]
则成为定义清晰但错误的数学陈述!因为 $i \in \mathbb{C}$ 且 $i^2 = -1 < 0$。(注:本书虽不常用复数集 $\mathbb{C}$,但它能提供如上所示富有启发性的反例。)

关键启示在于\textbf{上下文}至关重要——它能改变陈述的含义与真值。因此必须始终明确变量取值的集合。

\subsubsection*{一个例外}

然而,``永远指定量化集''的规则存在一个例外,且理由充分。考虑以下命题:
\begin{center}
    对于任意集合 $A, B,C$,等式 $(A \cup B) \cap C = (A \cap C) \cup (A \cap B)$ 成立。
\end{center}
该命题为\verb|真|。(你能证明这一点吗?尝试使用双重包含论证!)

如何符号化表达此命题?这是一个\emph{全称}量化(``对于任意……''),需要使用``$\forall$''符号。变量 $A,B,C$ 代表\emph{集合},它们取自何处?

可能你会想到``所有集合的集合''。但这会引发严重问题!回顾前一章的罗素悖论(见第 \ref{sec:section3.3.5} 节),所有可能集合的集合本身\emph{并非集合}!因此无法将该命题符号化为:
\[\forall A, B,C \in \underline{\qquad} \centerdot (A \cup B) \cap C = (A \cap C) \cup (A \cap B)\]
因为我们无法用\emph{集合}填充空白。

鉴于此,我们将继续采用``对于任意集合 $A,B,C$……''等表述,而非符号形式。个人笔记或草稿中可简写为``$\forall A,B,C$'',此时应理解其为对集合的量化。但在正式场合(如书面作业),请使用上述文字表述的方式。

\clearpage
