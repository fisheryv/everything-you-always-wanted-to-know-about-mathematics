% !TeX root = ../../../book.tex
\subsection{用法与符号}
可用``$\forall$''替换的常见短语还包括``对于每个''、``对于任意''、``每当''、``给定任意''等。

可用``$\exists$''替换的常见短语还包括``对于某个''、``至少有一个''、``有''等。

\begin{example}
    我们先看几个简单例子作为实践起点。以下每种情况,我们都尝试用符号表达数学思想,或用更``罗嗦''的文字解释量化陈述。
    \begin{itemize}
        \item ``每个实数的平方都是非负数。''
        
            这是一个简单直白的陈述,事实上,该陈述为\verb|真|。我们可以将其写做:
            \[\forall x \in \mathbb{R} \centerdot x^2 \ge 0\]
            ``分隔点 $\centerdot$''将陈述的量化部分与变量声明部分分开。\\
            另一种写法是:
            \begin{center}
                定义 $S(x)$ 为``$x^2 \ge 0$''。则命题为:$\forall x \in \mathbb{R} \centerdot S(x)$。
            \end{center}
        \item ``存在 $\mathbb{N}$ 的子集以 $7$ 为元素。''
        
            这是一个\emph{存在性}命题,断言存在满足特定性质的对象。我们可以将其写做:
            \[\exists S \in \mathcal{P}(\mathbb{N}) \centerdot 7 \in S\]
            其中 $\mathcal{P}(\mathbb{N})$ 是 $\mathbb{N}$ 的\emph{幂集},即 $\mathbb{N}$ 的所有子集的集合;因此,根据要求,$S \in \mathcal{P}(\mathbb{N})$ 意味着 $S \subseteq \mathbb{N}$。
        \item ``每个整数都有\emph{加法逆元}(即与原数字相加时,结果为 $0$)。''
        
            ``加法逆元''是\emph{环}和\emph{域}等代数结构的通用概念(本书不涉及,抽象代数课程将深入讨论)。\\
            我们可以将此命题写做:
            \[\forall a \in \mathbb{Z} \centerdot \exists b \in \mathbb{Z} \centerdot a + b = 0\]
            读作:
            \begin{center}
                对于任意整数 $a$,存在整数 $b$,使得 $a + b = 0$。
            \end{center}
            或
            \begin{center}
                对于任意整数 $a$,总存在满足 $a + b = 0$ 的整数 $b$。
            \end{center}
            若定义 $I(a, b)$ 为``$a + b = 0$'',可进一步简化为:
            \[\forall \in \mathbb{Z} \centerdot \exists b \in \mathbb{Z} \centerdot I(a, b)\]
    \end{itemize}
\end{example}

\begin{example}
    以下是``$\forall$''的正确用法及等价表述:
    \begin{itemize}
        \item $\forall x \in \mathbb{R} \centerdot x^2 \ge 0$
        \item 对于所有实数 $x$,有 $x^2 \ge 0$。
        \item 每个实数 $x$ 均满足 $x^2 \ge 0$。
        \item 当 $x$ 为实数时,$x^2 \ge 0$。
    \end{itemize}
    类似地,以下是``$\exists$''的正确用法及等价表述:
    \begin{itemize}
        \item $\exists x \in \mathbb{R} \centerdot x^2 - 4x + 4 = 0$
        \item 存在实数 $x$ 使得 $x^2 - 4x + 4 = 0$。
        \item 存在实数 $x$ 满足 $x^2 - 4x + 4 = 0$。
        \item 某个实数 $x$ 具有 $x^2 - 4x + 4 = 0$ 的性质。
    \end{itemize}
\end{example}

\subsubsection*{朗读量化陈述}

\begin{example}
    现在来看一个更具挑战的例子。回顾上一节结尾的短语,并使用新引入的符号重新表达。考虑以下陈述:
    \begin{center}
        对于每个实数 $x$,存在实数 $y$,使得 $y = x^3$。
    \end{center}
    为了用符号表达,定义 $P(x, y)$ 为命题``$y = x^3$'',则该陈述可写做:
    \[\exists y \in \mathbb{R} \centerdot \forall x \in \mathbb{R} \centerdot P(x, y)\]
    逻辑上这是正确且简洁的。现阶段,我们常借助``辅助短语''重述该陈述以提升可读性。朗读时会使用这些短语,因此我们将其写出,帮助读者练习口头解释逻辑符号:
    \begin{center}
        存在实数 $y$,使得对于每个实数 $x$,命题  $P(x, y)$ 成立。
    \end{center}
    其中``使得''就是一个``辅助短语'',连接存在量化与陈述的其余部分。下一小节将详述此类短语的使用时机与方法!
\end{example}

量化部分之间的``分隔点''用于分隔陈述部分以提升可读性,对应口语中的停顿或休止(类似逗号),有时具有语音含义(如 $\exists y \in \mathbb{R}$ 后接``使得'')。

为避免歧义,我们不使用逗号分隔量化,因为逗号已用于其他数学表达。例如:$x, y \in S$ 表示``$x$ 和 $y$ 均为集合 $S$ 的元素''。``分隔点''与此功能不同。

鉴于读者数学基础尚在建立阶段,建议书写时添加``使得''和``为\verb|真|''等辅助短语。这既能提示语义,亦可训练读写此类简洁陈述的能力。请牢记:学习数学语言需要持续练习将自然语言(如汉语)\emph{翻译}为形式符号。例如可将前述示例写做(或默读为):
\[\exists y \in \mathbb{R} \text{\ 使得\ } \forall x \in \mathbb{R} \centerdot P(x, y) \text{\ 为}\verb|真|\]
(注:板书时常用``s.t.''代替``such that''或``使得''以节省书写时间,可见该短语在数学写做中的普遍性——我们已为其设立了专用缩写!)
