% !TeX root = ../../../book.tex
\subsection{用法与符号}
可用``$\forall$''替换的常见短语还包括``对于每个''、``对于任意''、``每当''、``给定任意''等。

可用``$\exists$''替换的常见短语还包括``对于某个''、``至少有一个''、``有''等。

\begin{example}
    我们先看几个简单例子作为实践起点。以下每种情况,我们都尝试用符号表达数学思想,或用更``罗嗦''的文字解释量化陈述。
    \begin{itemize}
        \item ``每个实数的平方都是非负数。''
        
            这是一个简单直白的陈述,事实上,该陈述为\verb|真|。我们可以将其写作:
            \[\forall x \in \mathbb{R} \centerdot x^2 \ge 0\]
            ``分隔点 $\centerdot$''将陈述的量化部分与变量声明部分分开。\\
            另一种写法是:
            \begin{center}
                定义 $S(x)$ 为``$x^2 \ge 0$''。则命题为:$\forall x \in \mathbb{R} \centerdot S(x)$。
            \end{center}
        \item ``存在 $\mathbb{N}$ 的子集以 $7$ 为元素。''
        
            这是一个\emph{存在性}命题,断言存在满足特定性质的对象。我们可以将其写作:
            \[\exists S \in \mathcal{P}(\mathbb{N}) \centerdot 7 \in S\]
            其中 $\mathcal{P}(\mathbb{N})$ 是 $\mathbb{N}$ 的\emph{幂集},即 $\mathbb{N}$ 的所有子集的集合;因此,根据要求,$S \in \mathcal{P}(\mathbb{N})$ 意味着 $S \subseteq \mathbb{N}$。
        \item ``每个整数都有\emph{加法逆元}(即与原数字相加时,结果为 $0$)。''
        
            ``加法逆元''是\emph{环}和\emph{域}等代数结构的通用概念(本书不涉及,抽象代数课程将深入讨论)。\\
            我们可以将此命题写作:
            \[\forall a \in \mathbb{Z} \centerdot \exists b \in \mathbb{Z} \centerdot a + b = 0\]
            读作:
            \begin{center}
                对于任意整数 $a$,存在整数 $b$,使得 $a + b = 0$。
            \end{center}
            或
            \begin{center}
                对于任意整数 $a$,总存在满足 $a + b = 0$ 的整数 $b$。
            \end{center}
            若定义 $I(a, b)$ 为``$a + b = 0$'',可进一步简化为:
            \[\forall \in \mathbb{Z} \centerdot \exists b \in \mathbb{Z} \centerdot I(a, b)\]
    \end{itemize}
\end{example}

\begin{example}
    以下是``$\forall$''的正确用法及等价表述:
    \begin{itemize}
        \item $\forall x \in \mathbb{R} \centerdot x^2 \ge 0$
        \item 对于所有实数 $x$,有 $x^2 \ge 0$。
        \item 每个实数 $x$ 均满足 $x^2 \ge 0$。
        \item 当 $x$ 为实数时,$x^2 \ge 0$。
    \end{itemize}
    类似地,以下是``$\exists$''的正确用法及等价表述:
    \begin{itemize}
        \item $\exists x \in \mathbb{R} \centerdot x^2 - 4x + 4 = 0$
        \item 存在实数 $x$ 使得 $x^2 - 4x + 4 = 0$。
        \item 存在实数 $x$ 满足 $x^2 - 4x + 4 = 0$。
        \item 某个实数 $x$ 具有 $x^2 - 4x + 4 = 0$ 的性质。
    \end{itemize}
\end{example}

\subsubsection*{朗读量化陈述}

\begin{example}
    现在来看一些更难的例子。让我们回顾一下我们在上一节末尾写的短语,并使用这个新的符号来表达它们。考虑一下这个陈述:
    \begin{center}
        对于每个实数 $x$,存在一个实数 $y$,使得 $y = x^3$。
    \end{center}
    为了以符号形式表达这一点,我们将 $P(x, y)$ 定义为命题 ``$y = x^3$'',然后将该陈述写为:
    \[\exists y \in \mathbb{R} \centerdot \forall x \in \mathbb{R} \centerdot P(x, y)\]
    从逻辑上讲,这是正确的,并且相当简洁。目前,我们有时会使用一些``辅助短语''来重写该陈述,以帮助我们更好地阅读出来。特别是,我们在朗读时会使用这些``辅助短语'',因此我们将它们写下来,为你提供一些口头解释逻辑符号的额外练习。我们将上面的陈述朗读为
    \begin{center}
        存在一个实数 $y$,使得对于每个实数 $x$,陈述 $P(x, y)$ 成立。
    \end{center}
    短语``使得''就是一个``辅助短语'',它将存在量化与短语的其余部分联系起来。下一小节包含有关何时以及如何使用此辅助短语的一些重要信息!
\end{example}

上述陈述的量化部分之间的``大点''只是用于分隔陈述的各个部分,使其更易于阅读。它对应于说话中的停顿或休止,就像逗号一样,但有时它具有语音含义(例如 ``$\exists y \in \mathbb{R}$'' 部分后面的``使得'')。

不过,我们不想使用逗号,因为我们已经将它们用于其他含义。例如,
\[x, y \in S\]
表示 ``$x$ 和 $y$ 都是集合 $S$ 的元素''。``大点''是不同的符号。

相对而言,由于我们的数学生涯还很年轻,因此我们鼓励你尽量写下``使得''和``为\verb|真|''等辅助短语来指导你的理解。这会提醒你句子的含义,并帮助你练习以简洁的形式阅读和书写此类陈述。请记住,你在学习一门语言,你需要练习将句子从你知道的一种语言(汉语)\emph{翻译}成另一种语言(数学)。例如,你可能想将上面例子写为(或者,至少在你的脑海里读作):
\[\exists y \in \mathbb{R} \;\text{使得}\; \forall x \in \mathbb{R} \centerdot P(x, y) \;\text{为}\verb|真|\]
(顺带一提,当在白板/黑板或纸上书写时,通常会用 ``s.t.'' 代替 ``such that'' 或``使得'',以节省书写时间。这只是表明 ``such that'' 或``使得''这个短语在数学写作中是多么普遍;我们已经有了一个约定的缩写了!)
