% !TeX root = ../../../book.tex
\subsection{``固定''变量和依赖}\label{sec:section4.3.3}

在讨论量词顺序时,以下示例有助于阐明量词的顺序如何决定表达式中的\textbf{固定变量}。
当我们讨论量词顺序的主题时,我们还要提到以下示例来强调量词的顺序决定何时将变量视为表达式中的\textbf{固定变量}。

考虑命题:``任何大于或等于 $4$ 的偶数均可表示为两个质数之和。''(即著名的\textbf{哥德巴赫猜想}。)其逻辑符号化表达如下:
\begin{align*}
    &\text{令\ } X \text{\ 为除\ } 2 \text{\ 以外的偶数集合。}\\
    &\text{设\ } P \text{\ 为质数集。}\\
    &\text{定义\ } Q(n, a, b) \text{\ 为``}n = a + b\text{''。}\\
    &\text{则命题可表述为:}
\end{align*}
\[\forall n \in X \centerdot \exists a, b \in P \centerdot Q(n, a, b)\]
请注意,此处使用了简写形式。``$\exists a, b \in P$''完全等价于
\[\forall n \in X \centerdot \exists a \in P \centerdot \exists b \in P \centerdot Q(n, a, b)\]
当多个变量从同一集合量化且量词相邻时,合并书写是常见做法。例如:
\[\forall x, y \in \mathbb{Z} \centerdot \exists a, b, c, d \in \mathbb{Z} \centerdot a + b + c + d = x + y \text{\ 且\ } a + b \ne x \text{\ 且\ } c + d \ne y\]
(思考:此命题断言什么?在 $\mathbb{Z}$ 中是否成立?若替换为 $\mathbb{N}$ 或 $\mathbb{R}$ 呢?)

\subsubsection*{量化``固定''变量}

上例中,量词``$\forall n \in X$''的作用是\emph{固定} $n$ 的特定取值,该值将贯穿后续命题。随后,``$\exists a, b \in P$''及其关联性质 $Q(n, a, b)$ 依赖于 $n$ 的\emph{固定}但\emph{任意}取值。

整个命题表明:对任意选择的 $n$,总存在满足 $Q$ 的 $a, b$(注意 $a, b$ 的值可能\emph{依赖于} $n$)。量词顺序揭示了 $a, b$ 对 $n$ 的这种依赖性。

以具体值为例:取 $n = 8 \in X$,存在 $a = 3, b = 5 \in P$ 满足 $3 + 5 = 8$。当 $n = 14$ 时,需选取\emph{不同}的质数对(如 $a = 3, b = 11$ 或 $a = 7, b = 7$)。这就是 $a$ 和 $b$ \emph{依赖于} $n$ 的体现。

为加深理解,请思考下列命题与原命题有何本质区别?
\[\exists n \in X \centerdot \exists a, b \in P \centerdot Q(n, a, b)\]

此命题为\verb|真|还是为\verb|假|?依据是什么?
