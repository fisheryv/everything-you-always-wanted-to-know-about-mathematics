% !TeX root = ../../../book.tex
\subsection{短语``使得''以及量词的顺序}

请注意,辅助短语``使得''总是跟在\emph{存在}量化之后,并且\emph{只能}跟在存在量化之后。这是因为带有 ``$\exists$'' 的声明断言了具有某种属性的对象的存在,而陈述的其余部分是对该特殊属性的描述。因此,这里用``使得''是有合理的,可以帮助我们正确阅读句子。考虑下面这个数学陈述:
\[\exists y \in \mathbb{R} \centerdot \forall x \in \mathbb{R} \centerdot P(x, y)\]
如果我们朗读上面的陈述,但把``使得''这个短语放错了地方,把它放在 ``$\forall$'' 后面而不是 ``$\exists$'' 后面,会发生什么?这会产生这样一句话:
\begin{center}
    \textcolor{red}{$\exists y \in \mathbb{R} \quad \forall x \in \mathbb{R}$ 使得 $P(x, y)$ 为真}
\end{center}
我们前面分析过这可以用两种方式解释,但这两种方式\emph{都不是}真正正确的含义,这就是我们用\textcolor{red}{红色}书写的原因!

一方面,有人可能会说这样的句子根本不符合语法,也没有任何意义,因为 ``使得'' 不属于\emph{全称}量化。这相当于举手说:``我不知道你的意思!''

另一方面,人们可能会稍微解读一下这句话,并认为作者真正的意思是
\[\exists y \in \mathbb{R}, \forall x \in \mathbb{R}, \;\text{使得}\; P(x, y) \;\text{为}\verb|真|\]
或用语言表达
\begin{center}
    对于每个 $y \in \mathbb{R}$,存在 $x \in \mathbb{R}$,使得 $P(x, y)$ 为\verb|真|。
\end{center}
这里,逗号表示短语顺序的倒置,这在英语中很常见。(例如,考虑以下句子:``我笑了 \emph{30 Rock} 的每一集,全心全意地。''这与说``我全心全意地笑了 \emph{30 Rock} 的每一集。'')这句话是等效的,那么, 写作
\[\exists y \in \mathbb{R} \centerdot \forall x \in \mathbb{R}, \;\text{使得}\; P(x, y) \;\text{为}\verb|真|\]
这与我们考虑的原始数学陈述不同,事实上,它实际上是我们在上一节中看到的另一个陈述(参见第 \ref{sec:section4.2.4} 节),它是错误的!回想一下 \ref{sec:section4.2.4} 节中的类似陈述,只是语序颠倒了:
\begin{center}
    存在一个实数 $x$,对于每个实数 $y$,我们有 $y = x^3$。
\end{center}
我们可以用符号将其表示为
\[\exists x \in \mathbb{R} \centerdot \forall y \in \mathbb{R} \;\text{使得}\; P(x, y) \;\text{为}\verb|真|\]
看呐!短语``使得''放置错位置导致对句子的合理语言解释与最初的含义完全相反。哎呀!这就是为什么我们必须\emph{始终且仅在}\textbf{存在量化}之后小心使用``使得''。请记住,我们不会总是写出辅助短语,因此当你在脑海中朗读陈述或读给他人听时,必须谨记正确使用它,以确保获得正确的、预期的解释。

上一节中这个例子的目的是指出词序的重要性。现在我们有了符号来代替这些单词和短语,我们想强调这些符号的顺序也十分重要。我们上面看到的两个数学陈述包含完全相同的单词和符号,但顺序不同,一个为\verb|假|,另一个为\verb|真|。可见,顺序是极其重要的!
