% !TeX root = ../../../book.tex
\subsection{短语``使得''以及量词的顺序}

请注意,辅助短语``使得''总是跟在\emph{存在}量化之后,并且\emph{只能}跟在存在量化之后。这是因为带有``$\exists$''的陈述断言了具有某种属性的对象的存在,而其余部分是对该属性的描述。因此,这里用``使得''是合理的,有助于我们正确阅读句子。考虑以下数学陈述:
\[\exists y \in \mathbb{R} \centerdot \forall x \in \mathbb{R} \centerdot P(x, y)\]
如果我们朗读上述陈述,但将``使得''放错位置,置于``$\forall$''之后而非``$\exists$''之后,会发生什么?这将产生如下语句:
\begin{center}
    \textcolor{red}{$\exists y \in \mathbb{R} \quad \forall x \in \mathbb{R}$ 使得 $P(x, y)$ 为真}
\end{center}
我们之前分析过,这可以用两种方式解释,但两种方式都\emph{不是}真正正确的含义,因此我们用\textcolor{red}{红色}标示!

一方面,有人可能会说这样的句子根本不合语法且毫无意义,因为``使得''不属于\emph{全称}量化。这就像举手说:``我不知道你的意思!''

另一方面,人们可能会稍作解读,认为作者真正的意图是
\[\exists y \in \mathbb{R}, \forall x \in \mathbb{R},\text{使得\ } P(x, y) \text{\ 为}\verb|真|\]
或用语言表达为:
\begin{center}
    对于每个 $y \in \mathbb{R}$,存在 $x \in \mathbb{R}$,使得 $P(x, y)$ 为\verb|真|。
\end{center}
这里,逗号表示语序的颠倒,这在自然语言中很常见。(例如,考虑句子:``《我为喜剧狂》的每一集都让我发笑,真心实意地。''这与``《我为喜剧狂》的每一集都让我真心实意地发笑''是等价的。)那么,写做:
\[\exists y \in \mathbb{R} \centerdot \forall x \in \mathbb{R}, \text{使得\ } P(x, y) \text{\ 为}\verb|真|\]
这与原始数学陈述不同,事实上,它正是我们在上一节中看到的另一个陈述(参见第 \ref{sec:section4.2.4} 节),该陈述为\verb|假|!回想第 \ref{sec:section4.2.4} 节中的类似例子,其语序颠倒为:
\begin{center}
    存在实数 $x$,对于每个实数 $y$,都有 $y = x^3$。
\end{center}
其符号表示为:
\[\exists x \in \mathbb{R} \centerdot \forall y \in \mathbb{R} \text{\ 使得\ } P(x, y) \text{\ 为}\verb|真|\]
看呐!短语``使得''放错位置导致对句子的合理语言解释与原始含义完全相反。哎呀!这就是为什么我们必须\emph{始终且仅在}\textbf{存在量化}之后小心使用``使得''。请记住,我们并不总是写出辅助短语,因此当你在心中默读或朗读给他人听时,必须谨记正确使用它,以确保获得正确且预期的解释。

上一节中这个例子的目的是强调语序的重要性。既然现在我们用符号代替了这些词语和短语,我们要强调符号的顺序也极其重要。上述两个数学陈述包含完全相同的词语和符号,但顺序不同,一个为\verb|假|,另一个为\verb|真|。可见,顺序极其重要!
