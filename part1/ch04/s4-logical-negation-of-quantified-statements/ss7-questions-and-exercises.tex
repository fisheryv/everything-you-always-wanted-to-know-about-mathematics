% !TeX root = ../../../book.tex
\subsection{习题}

\subsubsection*{温故知新}

以口头或书面的形式简要回答以下问题。这些问题全都基于你刚刚阅读的内容,如果忘记了具体定义、概念或示例,可以回顾相关内容。确保在继续学习之前能够自信地作答这些问题,这将有助于你的理解和记忆!

\begin{enumerate}[label=(\arabic*)]
    \item 数学命题的\emph{否定}是什么?陈述与其否定之间有何关系?
    \item 为什么 $\forall$ 命题的否定是 $\exists$ 命题?\\
          为什么 $\exists$ 命题的否定是 $\forall$ 命题?
    \item 什么是非构造性证明?该术语适用于什么类型的命题($\exists$ 或 $\forall$)?
    \item 考虑陈述
        \[\forall x \in S \centerdot P(x)\]
        为什么它的否定不是下面这两个?
        \textcolor{red}{
            \[\forall x \notin S \centerdot P(x)\]
            \[\exists x \notin S \centerdot \neg P(x)\]
        }
\end{enumerate}

\subsubsection*{小试牛刀}

尝试解答以下问题。这些题目需动笔书写或口头阐述答案,旨在帮助你熟练运用新概念、定义及符号。题目难度适中,确保掌握它们将大有裨益!

\begin{enumerate}[label=(\arabic*)]
    \item 对于以下每个陈述,写出其否定形式。原陈述与其否定形式哪个为\verb|真|?
        \begin{tasks}[label=(\alph*)](2)
            \task $\forall x \in \mathbb{R} \centerdot \exists n \in \mathbb{N} \centerdot n > x$
            \task $\exists n \in \mathbb{N} \centerdot \forall x \in \mathbb{R} \centerdot n > x$
            \task $\forall x \in \mathbb{R} \centerdot \exists y \in \mathbb{R} \centerdot y = x^3$
            \task $\exists y \in \mathbb{R} \centerdot \forall x \in \mathbb{R} \centerdot y = x^3$
        \end{tasks}
    \item 对于以下每个陈述,写出其否定形式。原陈述与其否定形式哪个为\verb|真|?
        \begin{tasks}[label=(\alph*)](2)
            \task $\exists S \in \mathcal{P}(\mathbb{N}) \centerdot \forall x \in \mathbb{N} \centerdot x \in S$
            \task $\forall S \in \mathcal{P}(\mathbb{N}) \centerdot \exists x \in \mathbb{N} \centerdot x \in S$
            \task $\forall x \in \mathbb{N} \centerdot \exists S \in \mathcal{P}(\mathbb{N}) \centerdot x \in S$
            \task $\exists x \in \mathbb{N} \centerdot \forall S \in \mathcal{P}(\mathbb{N}) \centerdot x \in S$
        \end{tasks}
    \item 设 $I = \{x \in \mathbb{R} \mid 0 < x < 1\}$。\\
        对于以下每个给定集合,用量词描述 $y \in \mathbb{R}$ 属于该集合的条件。\\
        然后,确定集合的具体形式,并用集合构建符写出答案。\\
        (并尝试通过双重包含论证证明你的结论。)
        \begin{enumerate}[label=(\alph*)]
            \item $\displaystyle{S =\bigcup_{x \in I}\{y \in \mathbb{R} \mid x < y < 2\}}$
            \item $\displaystyle{T =\bigcap_{x \in I}\{y \in \mathbb{R} \mid -x < y < x\}}$
            \item $\displaystyle{V =\bigcup_{x \in I}\{y \in \mathbb{R} \mid -3x < y < 4x\}}$
        \end{enumerate}
    \item 令 $P = \{y \in \mathbb{R} \mid y > 0\}$。考虑以下陈述:
        \[\forall \varepsilon \in P \centerdot \exists \delta \in P \centerdot \forall x \in \{y \in \mathbb{R} \mid -\delta < y < \delta\} \centerdot |x^3| < \varepsilon\]
        写出该命题的逻辑否定形式。\\
        该陈述的含义是什么?其否定形式表示什么?\\
        哪个为\verb|真|?你能证明吗?
    \item 设 $A,B,C,D$ 为任意集合。设 $P(x),Q(x, y),R(x, y, z)$ 为任意变量命题。\\
        写出下列命题的否定形式。
        \begin{enumerate}[label=(\alph*)]
            \item $\forall a \in A \centerdot \exists b \in B \centerdot Q(a, b)$
            \item $\forall a \in A \centerdot \neg P(a)$
            \item $\forall c \in C \centerdot \forall d \in D \centerdot \neg Q(c, d)$
            \item $\forall a_1, a_2 \in A \centerdot \forall d \in D \centerdot R(a_1, a_2, d)$
            \item $\forall b_1, b_2, b_3 \in B \centerdot \neg R(b_1, b_2, b_3)$
            \item $\exists b \in B \centerdot \forall c \in C \centerdot \forall d \in D \centerdot R(d, b, c)$
        \end{enumerate}
\end{enumerate}
