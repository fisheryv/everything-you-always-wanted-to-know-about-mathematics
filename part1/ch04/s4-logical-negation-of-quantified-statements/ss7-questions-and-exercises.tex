% !TeX root = ../../../book.tex
\subsection{习题}

\subsubsection*{温故知新}

以口头或书面的形式简要回答以下问题。这些问题全都基于你刚刚阅读的内容,所以如果忘记了具体的定义、概念或示例,可以回去重读相关部分。确保在继续学习之前能够自信地回答这些问题,这将有助于你的理解和记忆!

\begin{enumerate}[label=(\arabic*)]
    \item 数学命题的\emph{否定}是什么?陈述及其否定是如何相关的?
    \item 为什么 $\forall$ 命题的否定是 $\exists$ 命题?\\
        为什么 $\exists$ 命题的否定是 $\forall$ 命题?
    \item 什么是非构造性证明?该术语适用于什么类型的主张($\exists$ 或 $\forall$)?
    \item 考虑声明
        \[\forall x \in S \centerdot P(x)\]
        为什么它的否定不是下面这两个?
        \textcolor{red}{
            \[\forall x \notin S \centerdot P(x)\]
            \[\exists x \notin S \centerdot \neg P(x)\]
        }
\end{enumerate}

\subsubsection*{小试牛刀}

尝试回答以下问题。这些题目要求你实际动笔写下答案,或(对朋友/同学)口头陈述答案。目的是帮助你练习使用新的概念、定义和符号。题目都比较简单,确保能够解决这些问题将对你大有帮助!

\begin{enumerate}[label=(\arabic*)]
    \item 对于以下每个陈述,写出其否定形式。哪一个 --- 原式形式还是否定形式 --- 为\verb|真|?
        \begin{enumerate}[label=(\alph*)]
            \item $\forall x \in \mathbb{R} \centerdot \exists n \in \mathbb{N} \centerdot n > x$
            \item $\exists n \in \mathbb{N} \centerdot \forall x \in \mathbb{R} \centerdot n > x$
            \item $\forall x \in \mathbb{R} \centerdot \exists y \in \mathbb{R} \centerdot y = x^3$
            \item $\exists y \in \mathbb{R} \centerdot \forall x \in \mathbb{R} \centerdot y = x^3$
        \end{enumerate}
    \item 对于以下每个陈述,写出其否定形式。哪一个 --- 原式形式还是否定形式 --- 为\verb|真|?
        \begin{enumerate}[label=(\alph*)]
            \item $\exists S \in \mathcal{P}(\mathbb{N}) \centerdot \forall x \in \mathbb{N} \centerdot x \in S$
            \item $\forall S \in \mathcal{P}(\mathbb{N}) \centerdot \exists x \in \mathbb{N} \centerdot x \in S$
            \item $\forall x \in \mathbb{N} \centerdot \exists S \in \mathcal{P}(\mathbb{N}) \centerdot x \in S$
            \item $\exists x \in \mathbb{N} \centerdot \forall S \in \mathcal{P}(\mathbb{N}) \centerdot x \in S$
        \end{enumerate}
    \item 设 $I = \{x \in \mathbb{R} \mid 0 < x < 1\}$。\\
        对于以下每个定义的集合,用量词写出确定数字 $y \in \mathbb{R}$ 是否为集合元素的定义条件。\\
        然后,确定集合是什么,并使用集合构建符写出答案。\\
        (并尝试使用双重包含论证来证明你的主张!)
        \begin{enumerate}[label=(\alph*)]
            \item $\displaystyle{S =\bigcup_{x \in I}\{y \in \mathbb{R} \mid x < y < 2\}}$
            \item $\displaystyle{T =\bigcap_{x \in I}\{y \in \mathbb{R} \mid -x < y < x\}}$
            \item $\displaystyle{V =\bigcup_{x \in I}\{y \in \mathbb{R} \mid -3x < y < 4x\}}$
        \end{enumerate}
    \item 令 $P = \{y \in \mathbb{R} \mid y > 0\}$。考虑下面这个声明:
        \[\forall \varepsilon \in P \centerdot \exists \delta \in P \centerdot \forall x \in \{y \in \mathbb{R} \mid -\delta < y < \delta\} \centerdot |x^3| < \varepsilon\]
        写出该命题的逻辑否定形式。\\
        这个陈述表达了什么?它的否定形式表达了什么?\\
        哪一个为\verb|真|?你能证明这一点吗?
    \item 设 $A,B,C,D$ 为任意集合。\\
        设 $P(x),Q(x, y),R(x, y, z)$ 为任意变量命题。\\
        写出下列各命题的否定形式。
        \begin{enumerate}[label=(\alph*)]
            \item $\forall a \in A \centerdot \exists b \in B \centerdot Q(a, b)$
            \item $\forall a \in A \centerdot \neg P(a)$
            \item $\forall c \in C \centerdot \forall d \in D \centerdot \neg Q(c, d)$
            \item $\forall a_1, a_2 \in A \centerdot \forall d \in D \centerdot R(a_1, a_2, d)$
            \item $\forall b_1, b_2, b_3 \in B \centerdot \neg R(b_1, b_2, b_3)$
            \item $\exists b \in B \centerdot \forall c \in C \centerdot \forall d \in D \centerdot R(d, b, c)$
        \end{enumerate}
\end{enumerate}