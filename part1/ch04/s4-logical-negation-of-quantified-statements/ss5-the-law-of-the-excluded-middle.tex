% !TeX root = ../../../book.tex
\subsection{排中律}

你知道吗?让我们实际讨论一下为什么我们可以谈论陈述及其\textbf{逻辑否定}。我们对\textbf{数学/逻辑陈述}的定义中包含这样一个想法:陈述的句子必须仅有一个真值,要么为\verb|真|,要么为\verb|假|。为什么我们可以这样做?在这里我们需要对定义负责!数学家必须为他们的系统设定基本规则 --- \textbf{公理},而我们希望我们的逻辑系统能够确保我们提出的每一个主张都是正确的或错误的,而不是两者兼有或二者皆无。

这种二元性确实是我们系统的\textbf{公理}。它在大多数数学中被广泛采用,被称为``\textbf{排中律}''。这个名字来源于这样一个想法,即每一个主张都要么为\verb|真|要么为\verb|假|,所以这两者之间没有\emph{中间立场};中间的部分被排除在外。

从本质上讲,这使得我们在数学方面的工作卓有成效:每个主张都有一个真值,而我们的目标就是找到该真值。有时,我们必须依靠这个公理,即我们商定的定律,来\emph{确保}某些主张要么为\verb|真|要么为\verb|假|,但不知道具体是哪个真值。下面是一个有趣且颇具代表性的例子。

\begin{proposition}
    存在实数 $a$ 和 $b$ 都为无理数,而 $a^b$ 为有理数。
\end{proposition}

(请记住,有理数是可以表示为整数相除形式的数,无理数是非有理数的实数。你能想到有理数和无理数的一些例子吗?)

\begin{proof}
    我们知道 $\sqrt{2}$ 为无理数。(问题:为什么?你能证明这一点吗?现在就试试。我们也将很快证明这一点!)

    $\sqrt{2}^{\sqrt{2}}$ 要么是有理数,要么是无理数。(这里使用了排中律。)让我们分别考虑这两种情况。
    \begin{itemize}
        \item 假设 $\sqrt{2}^{\sqrt{2}}$ 为有理数。那么令 $a=\sqrt{2}, b=\sqrt{2}$,$a, b$ 都是无理数,而 $a^b$ 为有理数。
        \item 假设 $\sqrt{2}^{\sqrt{2}}$ 为无理数。那么令 $a=\sqrt{2}^{\sqrt{2}}, b=\sqrt{2}$,$a, b$ 都是无理数,而
        \[a^b = \big(\sqrt{2}^{\sqrt{2}}\big)^{\sqrt{2}} = \sqrt{2}^{\sqrt{2} \cdot \sqrt{2}} = (\sqrt{2})^2 = 2\]
        为有理数。
    \end{itemize}
    无论哪种情况,我们都找到了实数 $a, b \in \mathbb{R}$,其中 $a$ 和 $b$ 都为无理数,而 $a^b$ 为有理数。这证明改命题为\verb|真|。
\end{proof}

这是一个\textbf{非构造性}证明的例子。它告诉我们某些东西存在(甚至将其缩小到两种可能性),但实际上并没有\emph{确切}告诉我们哪种可能性是我们要寻找的可能性。正是对排中律的直接运用才构成了这种情况。

(问题:你能证明 $\sqrt{2}^{\sqrt{2}}$ 是无理数吗?这是真的,但没有已知的``简单''方法来证明这一事实。也许你可以找到一个!)

本书中大多数证明都是\textbf{构造性}的(但不是全部)。也许你觉得这样很好,我们也倾向于觉得这样不错。如果我们声称某事物存在,我们应该能够向你\emph{展示}它,对吗?如果我们只是谈论\emph{为什么}某些这样的事物存在于某个地方,而无法指明它,你可能会相信我们,但你一定对此感觉不佳。因此,构造性证明\emph{在主观上会更好一些},我们将尽可能地采用构造性证明。但有时,构造性证明并不显而易见,我们不得不采用非构造性证明,就像我们在上面所做的那样。
