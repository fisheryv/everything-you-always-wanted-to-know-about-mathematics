% !TeX root = ../../../book.tex
\subsection{排中律}

让我们探讨为什么可以谈论陈述及其\textbf{逻辑否定}。我们对\textbf{数学/逻辑陈述}的定义要求每个陈述必须仅有一个真值,要么为\verb|真|,要么为\verb|假|。为什么能这样定义?因为数学家必须为系统设定基本规则——\textbf{公理},而逻辑系统需要确保每个命题要么为真要么为假,不能既真又假或非真非假。

这种二元性是逻辑系统的\textbf{公理}。它在大多数数学中被广泛采用,称为``\textbf{排中律}''。其名称源于这样的思想:每个命题要么为\verb|真|要么为\verb|假|,因此不存在\emph{中间立场};中间状态被排除在外。

本质上,这使得数学工作富有成效:每个命题都有真值,而我们的目标就是确定该真值。有时必须依靠这个公理(即约定的定律)来\emph{确保}某些命题要么为\verb|真|要么为\verb|假|,尽管我们可能不知道具体是哪一个。下面是一个有趣且典型的例子。

\begin{proposition}
    存在实数 $a$ 和 $b$ 均为无理数,而 $a^b$ 为有理数。
\end{proposition}

(请记住:有理数可以表示为两个整数之比,无理数则是不能表示为整数之比的实数。你能举出有理数与无理数的例子吗?)

\begin{proof}
    已知 $\sqrt{2}$ 为无理数。(问题:为什么?你能证明吗?请尝试一下。我们稍后将证明这一点!)

    $\sqrt{2}^{\sqrt{2}}$ 要么是有理数,要么是无理数。(此处应用排中律。)分两种情况讨论:
    \begin{itemize}
        \item 假设 $\sqrt{2}^{\sqrt{2}}$ 为有理数。则令 $a=\sqrt{2}, b=\sqrt{2}$,此时 $a, b$ 均为无理数,而 $a^b$ 为有理数。
        \item 假设 $\sqrt{2}^{\sqrt{2}}$ 为无理数。则令 $a=\sqrt{2}^{\sqrt{2}}, b=\sqrt{2}$,此时 $a, b$ 均为无理数,而
        \[a^b = \big(\sqrt{2}^{\sqrt{2}}\big)^{\sqrt{2}} = \sqrt{2}^{\sqrt{2} \cdot \sqrt{2}} = (\sqrt{2})^2 = 2\]
        为有理数。
    \end{itemize}
    两种情况下,均存在实数 $a, b \in \mathbb{R}$ 满足 $a$ 和 $b$ 均为无理数而 $a^b$ 为有理数。故该命题成立。
\end{proof}

这是一个\textbf{非构造性}证明的范例:它表明某事物存在(并将可能的情形缩小至两种),但未\emph{确切}指出哪种情形成立。正是排中律的直接应用使此成为可能。

(问题:你能证明 $\sqrt{2}^{\sqrt{2}}$ 为无理数吗?该命题为\verb|真|,但尚无已知的``简单''证法。或许你能找到!)

本书大多数证明(非全部)均为\textbf{构造性}证明。你或许认为构造性证明更好,我们也倾向于认同该观点。若声称某事物存在,应该能向你\emph{展示}它;若仅解释其存在却无法指明,你或许会相信,但难免感到缺憾。因此,构造性证明\emph{在主观上更优},我们将尽可能采用构造性证明。但有时构造性证明不易获得,只能诉诸非构造性证明,如上例所示。
