% !TeX root = ../../../book.tex
\subsection{否定全称量化}

全称声明(即 ``$\forall$'')的否定通常是一个存在声明(即``$\exists$''),反之亦然。在我们解决更大的否定任意量化陈述的问题之前,让我们看一个简单的例子。

假设 $S$ 为集合,$R(x)$ 是定义在每个 $x \in S$ 上的数学陈述。该陈述
\[\forall x \in S \centerdot R(x)\]
对集合 $S$ 中变量 $x$ 的每个可能值断言变量命题的真值。它表示,无论我们引用集合 $S$ 的哪个元素 $x$,我们都可以\emph{必然}得出命题 $R(x)$ 为\verb|真|。现在,如果这个陈述为\verb|假|,我们该如何\emph{证明}这一点呢?

如果每个元素 $x \in S$ 都满足某个属性为\verb|假|,则必定\emph{至少}有一个元素\emph{不}满足该属性。为了证明这一点,我们需要找到这样的元素;我们必须定义(或找到)一个元素 $x$ 并解释为什么 $R(x)$ 对于该特定元素不成立。(想想我们如何在语言上理解这个否定。我们在日常语言中一直这样做,甚至没有思考它。)那么,结论是,原始陈述的否定是
\[\exists x \in S \;\text{使得}\; R(x) \;\text{为假}\]

我们引入符号 $\neg$ 表示``\textbf{逻辑否}''或``\textbf{不}''。有了这个符号,我们就可以重写否定陈述
\[\neg\big(\forall x \in S \centerdot R(x)\big)\]
为
\[\exists x \in S \centerdot \neg R(x)\]
上面陈述的结论短语 $\neg R(x)$ 可以简化,具体取决于 $R(x)$ 是什么。例如,如果 $S = \mathbb{R}$ 且 $R(x)$ 为 ``$x^2 \ge 0$'',则否定陈述为
\[\text{``}\exists x \in \mathbb{R} \;\text{使得}\; x^2 < 0 \text{''}\]
因为 ``$x^2 < 0$'' 在逻辑上等价于 ``$\neg(x^2 \ge 0)$''。

但一般来说,我们必须将其保留为 ``$\neg R(x)$'',而不进一步深入命题 $R$。我们还将指出,一般来说,短语 ``$R(x)$ 为\verb|假|'' 和 ``$\neg R(x)$ 为\verb|真|'' 在逻辑上是等价的;他们都断言命题 $R(x)$ 不正确。

我们现在正在发展的这个概念就是\emph{反例}的含义,你以前可能听说过这个术语。为了\emph{反驳}全称量化陈述,我们必须证明存在量化陈述;该证明涉及显式定义集合中不满足指定属性的元素,这就是\textbf{反例}一词的由来。
