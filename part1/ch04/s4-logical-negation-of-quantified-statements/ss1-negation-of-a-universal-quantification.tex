% !TeX root = ../../../book.tex
\subsection{否定全称量化}

全称陈述(即``$\forall$'')的否定通常表现为存在陈述(即``$\exists$''),反之亦然。在探讨更复杂的全称量化陈述否定问题前,我们先分析一个简单示例。

设 $S$ 为集合,$R(x)$ 是定义在 $x \in S$ 上的数学命题。陈述
\[\forall x \in S \centerdot R(x)\]
对集合 $S$ 中任意元素 $x$,命题 $R(x)$ 恒为真。即无论选取 $S$ 中哪个元素 $x$,都能确保 $R(x)$ 为\verb|真|。若该陈述为\verb|假|,我们该如何\emph{证明}这一点呢?

若并非所有元素都满足该属性,则必然存在\emph{至少}一个元素\emph{不}满足该属性。为证明此结论,需要找出这样的元素:即定义(或发现)某个 $x$,并证明 $R(x)$ 对该元素不成立。(思考语言中的否定逻辑——日常交流中我们常不假思索地运用此逻辑。)因此,原陈述的否定可表述为:
\[\exists x \in S \text{\ 使得\ } R(x) \text{\ 为假}\]

我们引入符号 $\neg$ 表示``\textbf{逻辑否定}''或``\textbf{非}''。借此可将否定陈述
\[\neg\big(\forall x \in S \centerdot R(x)\big)\]
重写为
\[\exists x \in S \centerdot \neg R(x)\]
其中 $\neg R(x)$ 可根据 $R(x)$ 的具体形式简化。例如当 $S = \mathbb{R}$ 且 $R(x)$ 为``$x^2 \ge 0$''时,否定陈述为
\[\text{``}\exists x \in \mathbb{R} \text{\ 使得\ } x^2 < 0 \text{''}\]
因为``$x^2 < 0$''逻辑等价于``$\neg(x^2 \ge 0)$''。

一般而言,我们保留``$\neg R(x)$''形式而不深入分解 $R$。需强调的是: ``$R(x)$ 为\verb|假|''与``$\neg R(x)$ 为\verb|真|''逻辑等价,二者均表明命题 $R(x)$ 不成立。

以上阐述的\emph{反例}概念你可能已有所了解。为\emph{反驳}全称量化陈述,需证明对应的存在量化陈述成立;该证明过程要求显式构造集合中不满足指定属性的元素,这便是\textbf{反例}一词的由来。
