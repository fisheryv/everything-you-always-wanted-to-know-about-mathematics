% !TeX root = ../../../book.tex
\subsection{回顾:索引集运算与量词}

回顾第 \ref{sec:section3.6.2} 节,我们在定义 \ref{def:definition3.6.1} 和 \ref{def:definition3.6.2} 中分别介绍了索引集的集合运算(并集和交集)。其核心思想是通过简洁的符号表示整个集合族的并集或交集。

仔细分析这些定义:一个对象属于索引并集的条件是什么?它必须是至少一个组成集合的元素,即存在包含该对象的集合。这本质上是一种\emph{存在量化}。

同理,一个对象属于索引交集的条件是什么?它必须属于所有组成集合,即对于每个集合而言该对象都是其元素。这本质上是一种\emph{全称量化}。

基于上述观察,我们可以用量词符号重新表述索引集运算的定义:

\begin{definition}
    设 $I$ 为索引集,且对于某全集 $U$ 满足 $\forall i \in I, A_i \subseteq U$。则
    \[\bigcup_{i \in I} A_i = \{x \in U \mid \exists k \in I \centerdot x \in A_k\}\]
    \[\bigcap_{i \in I} A_i = \{x \in U \mid \forall i \in I \centerdot x \in A_i\}\]
\end{definition}

建议重做第 \ref{sec:section3.6.2} 节中的例题与习题。现在这些定义是否更加清晰?
