% !TeX root = ../../../book.tex
\subsection{回顾:索引集运算与量词}

回顾一下第 \ref{sec:section3.6.2} 节,我们分别在定义 \ref{def:definition3.6.1} 和 \ref{def:definition3.6.2} 中定义了对索引集执行的集合操作(并集和交集)。主要思想是我们可以使用速记符一次性表达整个集合类的并集/交集。

仔细看看这些定义。例如,一个对象是否是索引并集的元素的特征是什么?该对象必须至少是并集中一个组成集合的元素。也就是说,需要存在某个以该对象为元素的集合。这听起来像是\emph{存在量化},不是吗?

同理,一个对象是否是索引交集的元素的特征是什么?该对象需要是所有组成集合的元素。也就是说,对于所有这些集合,该对象必须是其中的元素。这是一个\emph{全称量化}。

通过这些观察,我们可以使用新的量词符号重写索引集运算的定义:

\clearpage 

\begin{definition}
    假设 $I$ 为索引集,且对于某全集 $U$, $\forall i \in I, A_i \subseteq U$。则
    \[\bigcup_{i \in I} A_i = \{x \in U \mid \exists k \in I \centerdot x \in A_k\}\]
    \[\bigcap_{i \in I} A_i = \{x \in U \mid \forall i \in I \centerdot x \in A_i\}\]
\end{definition}

再次尝试做一下第 \ref{sec:section3.6.2} 节中的一些示例和练习。现在这些定义更有意义了吗?
