% !TeX root = ../../../book.tex
\section{量化陈述的逻辑否定}\label{sec:section4.4}

让我们回到之前使用过的示例陈述。定义 $P(x, y)$ 为 ``$y = x^3$'',然后定义 $Q_1$ 为陈述
\[\text{``}\exists y \in \mathbb{R} \centerdot \forall x \in \mathbb{R} \centerdot P(x, y)\text{''}\]
定义 $Q_2$ 为陈述
\[\text{``}\forall x \in \mathbb{R} \centerdot \exists y \in \mathbb{R} \centerdot P(x, y)\text{''}\]
请记住,$Q_1$ 为\verb|假|,$Q_2$ 为\verb|真|。

我们怎么知道 $Q_1$ 为\verb|假|?它表示存在某个具有特定属性的实数。要声明整个语句为\verb|假|,我们可能必须验证该属性并\emph{不}适用于\emph{每个}实数 $y$,但这需要很长时间!集合 $\mathbb{R}$ 无穷大!一种更有效的方法是证明该陈述的\textbf{否定}形式为\verb|真|。

我们所说的``否定''指的是\textbf{逻辑否},即在逻辑意义上与原始陈述``相反''的陈述。数学陈述的逻辑否具有与原始陈述相反的真值,因此,如果我们得到 $Q_1$ 的否定形式并证明它为\verb|真|,那么我们就证明了 $Q_1$ 本身为\verb|假|。

但是我们如何否定一个陈述呢?当我们注意到我们必须以某种方式证明关于\emph{每个}实数 $y$ 的某些事情时,我们已经有了正确的想法,因为原始陈述声称\emph{存在}。在本节中,我们将探讨如何正确地否定此类陈述。

我们应该注意到,到目前为止我们所讨论的内容背后有一些微妙但深刻的数学概念。为什么一个数学命题要么为\verb|真|,要么为\verb|假|呢?一个厚颜无耻(但完全正确)的回答是,``因为你把`\textbf{数学陈述}'定义为那样,傻逼!''是的,我们确实是这么做的,但我们\emph{为什么}要这么做呢?\verb|真|/\verb|假|二元性对数学为什么是\emph{有益的},或\emph{必要的}?这些都是有意义又难以回答的问题,绝对值得思考。对这些主题的讨论必然会深入研究数学哲学和人类思想,这当然是有趣且有价值的追求,但超出了本书/课程的范畴和目标。我们将依靠我们对真理的共同的、直觉的理解。

% !TeX root = ../../../book.tex
\subsection{否定全称量化}

全称陈述(即``$\forall$'')的否定通常表现为存在陈述(即``$\exists$''),反之亦然。在探讨更复杂的全称量化陈述否定问题前,我们先分析一个简单示例。

设 $S$ 为集合,$R(x)$ 是定义在 $x \in S$ 上的数学命题。陈述
\[\forall x \in S \centerdot R(x)\]
对集合 $S$ 中任意元素 $x$,命题 $R(x)$ 恒为真。即无论选取 $S$ 中哪个元素 $x$,都能确保 $R(x)$ 为\verb|真|。若该陈述为\verb|假|,我们该如何\emph{证明}这一点呢?

若并非所有元素都满足该属性,则必然存在\emph{至少}一个元素\emph{不}满足该属性。为证明此结论,需要找出这样的元素:即定义(或发现)某个 $x$,并证明 $R(x)$ 对该元素不成立。(思考语言中的否定逻辑——日常交流中我们常不假思索地运用此逻辑。)因此,原陈述的否定可表述为:
\[\exists x \in S \text{\ 使得\ } R(x) \text{\ 为假}\]

我们引入符号 $\neg$ 表示``\textbf{逻辑否定}''或``\textbf{非}''。借此可将否定陈述
\[\neg\big(\forall x \in S \centerdot R(x)\big)\]
重写为
\[\exists x \in S \centerdot \neg R(x)\]
其中 $\neg R(x)$ 可根据 $R(x)$ 的具体形式简化。例如当 $S = \mathbb{R}$ 且 $R(x)$ 为``$x^2 \ge 0$''时,否定陈述为
\[\text{``}\exists x \in \mathbb{R} \text{\ 使得\ } x^2 < 0 \text{''}\]
因为``$x^2 < 0$''逻辑等价于``$\neg(x^2 \ge 0)$''。

一般而言,我们保留``$\neg R(x)$''形式而不深入分解 $R$。需强调的是: ``$R(x)$ 为\verb|假|''与``$\neg R(x)$ 为\verb|真|''逻辑等价,二者均表明命题 $R(x)$ 不成立。

以上阐述的\emph{反例}概念你可能已有所了解。为\emph{反驳}全称量化陈述,需证明对应的存在量化陈述成立;该证明过程要求显式构造集合中不满足指定属性的元素,这便是\textbf{反例}一词的由来。


% !TeX root = ../../../book.tex
\subsection{否定存在量化}

类似 
\[\exists x \in S \centerdot R(x)\] 
这样的陈述提出了存在性声明。它表示必须存在某个元素 $x$ 满足属性 $R(x)$。为了反驳这一主张,我们需要证明 $x$ 的任何值实际上都无法满足属性 $R$。因此,我们可以说陈述
\[\neg\big(\exists x \in S \centerdot R(x)\big)\]
在逻辑上等价于陈述 
\[\forall x \in S \centerdot \neg R(x)\]

如果我们考虑如何反驳这种存在性声明,就会发现这是合理的。假设你正在与某位朋友进行辩论,他告诉你某些 kwyjibo 具有它是 Zooqa 的属性。你会如何反驳他/她?你可能会说,``不对!给我任意你想要的 kwyjibo。我知道这不可能是 zooqa,原因如下……''然后你会解释为什么该属性无论如何都不成立。

现在,当你说``给我任意''时,你实际上是在执行全称量化!你是说,\emph{无论}你考虑哪种 kwyjibo,有些事情都是真的;也就是说,对于\emph{每个} kwyjibo,或 $\forall x \in K$(其中 $K$ 是所有 kwyjibo 的集合),某些事情为\verb|真|。

想一想并思考以下为什么我们发现/定义的逻辑否定是合理的。在本章后面,当我们考虑证明技术时,我们将解释考虑\emph{任意} kwyjibo 的策略以及为什么这实际上证明了我们上面刚刚写出的逻辑否定。眼下,我们希望你了解
\[\forall x \in S \centerdot \neg R(x)\] 
和
\[\exists x \in S \centerdot R(x)\]
具有相反的真值。


% !TeX root = ../../../book.tex
\subsection{一般量化陈述的否定}

到目前为止,我们的观察揭示了否定量化陈述的一般方法。第 \pageref{sec:section4.4} 页定义的陈述 $Q_1$ 的形式为
\[\exists y \in \mathbb{R} \centerdot C(y)\]
其中 $C(y)$ 是陈述的其余部分(其内容\emph{依赖于} $y$ 的值)。我们可以将 $C(y)$ 视为被量化变量 $y$ 的某种\emph{属性};该属性内部可能包含其他量词和变量,但本质上它断言了关于 $y$ 的某个事实。

为了否定该陈述,我们采用前述方法写作:
\[\forall y \in \mathbb{R} \centerdot \neg C(y)\]
已知 $C(y)$ 本身是一个全称量化陈述:
\[\forall x \in \mathbb{R} \centerdot y = x^3\]
我们也知道如何否定此类陈述!其否定 $\neg C(y)$ 为:
\[\exists x \in \mathbb{R} \centerdot y \ne x^3\]
此步骤仅应用了上述否定规则。综合起来,$\neg Q_1$ 可表述为:
\[\forall y \in \mathbb{R} \centerdot \exists x \in \mathbb{R} \centerdot y \ne x^3\]
我们可以\emph{证明}该陈述为\verb|真|,从而证实原陈述必定为\verb|假|。

(此证明留作练习。提示:给定任意 $y \in \mathbb{R}$,构造一个 $x$ 值使得 $y \ne x^3$ 成立。注意 $x$ 的选择依赖于 $y$;思考其依赖关系。)

观察否定的形成过程:原陈述是一个\textbf{嵌套量词}序列(即连续多个量化变量),末尾为一个命题。我们将量词序列的一部分视为独立陈述,再将否定从外层量词``传递''至内层量词,最终组合这些否定结果。

遵循同样的思路,我们可以处理包含更长量词序列的陈述。例如:
\[\forall a \in A \centerdot \exists b \in B \centerdot \exists c \in C \centerdot \forall d, e \in D \centerdot Q(a, b, c, d, e)\]
否定时,先从首个量词处拆解,将剩余部分视为仅依赖于 $a$ 的命题 $R(a)$:
\[\forall a \in A \centerdot \big(\underbrace{\exists b \in B \centerdot \exists c \in C \centerdot \forall d, e \in D \centerdot Q(a, b, c, d, e)}_{R(a)}\big)\]
因此其否定可写作:
\[\exists a \in A \centerdot \neg R(a)\]
接下来需确定 $\neg R(a)$ 的表达式。重复上述过程:分离``$\exists b \in B$''并逐步推进……请自行完成推导,确保最终得到原始陈述的逻辑否定:
\[\exists a \in A \centerdot \forall b \in B \centerdot \forall c \in C \centerdot \exists d, e \in D \centerdot \neg Q(a, b, c, d, e)\]

一般而言,否定仅由量词和命题构成的陈述时,只需将每个``$\forall$''替换为``$\exists$'',反之亦然,并否定末尾命题。量化集合保持不变;改变论域并无意义。后续我们将探讨如何否定由其他逻辑连词构成的更复杂的陈述,但首先需要定义并讨论这些连词。


% !TeX root = ../../../book.tex
\subsection{方法总结}

让我们总结一下本节的内容。
\begin{itemize}
    \item \textbf{否定全称量化:} \\
        设 $X$ 为集合,$P(x)$ 为命题。则全称量化的否定,
        \[\neg \big(\forall x \in X \centerdot P(x)\big)\]
        写为
        \[\exists x \in X \centerdot \neg P(x)\]
        用语言表达,我们已经证明了
        \begin{center}
            对于每个 $x \in X, P(x)$ 不都成立。
        \end{center}
        等价于
        \begin{center}
            存在元素 $x \in X$ 使得 $P(x)$ 不成立。
        \end{center}
    \item \textbf{否定存在量化:} \\
        设 $X$ 为集合,$Q(x)$ 为命题。则存在量化的否定,
        \[\neg \big(\exists x \in X \centerdot Q(x)\big)\]
        写为
        \[\forall x \in X \centerdot \neg Q(x)\]
        用语言表达,我们已经证明了
        \begin{center}
            不存在 $x \in X$ 使得 $Q(x)$ 成立。
        \end{center}
        等价于
        \begin{center}
            对于每个元素 $x \in X, Q(x)$ 都不成立。
        \end{center}
\end{itemize}

\subsubsection*{不要更改量化集!}

我们上面提到过,当否定一个陈述时,改变讨论范围是没有意义的。要思考为什么这是合理的,可以举一个现实生活中的例子。

假设我们说``这个书架上的每一本书都是用英语写的''。你如何证明我们在撒谎,我们的陈述实际上为\verb|假|?你必须在\emph{这个书架}上找出一本不是用英文编写的书。你不能从走廊那头的房间里拿一本法国小说说:``瞧,你错了!''这并不能证明我们的主张;因为讨论的领域不同,我们没有对其他房间书架上发生的事情做出任何声明。我们只是对这个\emph{特定}书架做出了断言。

同理,当否定陈述
\[\forall b \in T \centerdot P(b)\]
时,我们在不改变讨论领域(集合 $T$)的情况下得到
\[\exists b \in T \centerdot \neg P(b)\]
最初的声明仅断言了 $T$ 中元素的某些性质,因此它的否定也应仅断言这一点。


% !TeX root = ../../../book.tex
\subsection{排中律}

你知道吗?让我们实际讨论一下为什么我们可以谈论陈述及其\textbf{逻辑否定}。我们对\textbf{数学/逻辑陈述}的定义中包含这样一个想法:陈述的句子必须仅有一个真值,要么为\verb|真|,要么为\verb|假|。为什么我们可以这样做?在这里我们需要对定义负责!数学家必须为他们的系统设定基本规则 --- \textbf{公理},而我们希望我们的逻辑系统能够确保我们提出的每一个主张都是正确的或错误的,而不是两者兼有或二者皆无。

这种二元性确实是我们系统的\textbf{公理}。它在大多数数学中被广泛采用,被称为``\textbf{排中律}''。这个名字来源于这样一个想法,即每一个主张都要么为\verb|真|要么为\verb|假|,所以这两者之间没有\emph{中间立场};中间的部分被排除在外。

从本质上讲,这使得我们在数学方面的工作卓有成效:每个主张都有一个真值,而我们的目标就是找到该真值。有时,我们必须依靠这个公理,即我们商定的定律,来\emph{确保}某些主张要么为\verb|真|要么为\verb|假|,但不知道具体是哪个真值。下面是一个有趣且颇具代表性的例子。

\begin{proposition}
    存在实数 $a$ 和 $b$ 都为无理数,而 $a^b$ 为有理数。
\end{proposition}

(请记住,有理数是可以表示为整数相除形式的数,无理数是非有理数的实数。你能想到有理数和无理数的一些例子吗?)

\begin{proof}
    我们知道 $\sqrt{2}$ 为无理数。(问题:为什么?你能证明这一点吗?现在就试试。我们也将很快证明这一点!)

    $\sqrt{2}^{\sqrt{2}}$ 要么是有理数,要么是无理数。(这里使用了排中律。)让我们分别考虑这两种情况。
    \begin{itemize}
        \item 假设 $\sqrt{2}^{\sqrt{2}}$ 为有理数。那么令 $a=\sqrt{2}, b=\sqrt{2}$,$a, b$ 都是无理数,而 $a^b$ 为有理数。
        \item 假设 $\sqrt{2}^{\sqrt{2}}$ 为无理数。那么令 $a=\sqrt{2}^{\sqrt{2}}, b=\sqrt{2}$,$a, b$ 都是无理数,而
        \[a^b = \big(\sqrt{2}^{\sqrt{2}}\big)^{\sqrt{2}} = \sqrt{2}^{\sqrt{2} \cdot \sqrt{2}} = (\sqrt{2})^2 = 2\]
        为有理数。
    \end{itemize}
    无论哪种情况,我们都找到了实数 $a, b \in \mathbb{R}$,其中 $a$ 和 $b$ 都为无理数,而 $a^b$ 为有理数。这证明改命题为\verb|真|。
\end{proof}

这是一个\textbf{非构造性}证明的例子。它告诉我们某些东西存在(甚至将其缩小到两种可能性),但实际上并没有\emph{确切}告诉我们哪种可能性是我们要寻找的可能性。正是对排中律的直接运用才构成了这种情况。

(问题:你能证明 $\sqrt{2}^{\sqrt{2}}$ 是无理数吗?这是真的,但没有已知的``简单''方法来证明这一事实。也许你可以找到一个!)

本书中大多数证明都是\textbf{构造性}的(但不是全部)。也许你觉得这样很好,我们也倾向于觉得这样不错。如果我们声称某事物存在,我们应该能够向你\emph{展示}它,对吗?如果我们只是谈论\emph{为什么}某些这样的事物存在于某个地方,而无法指明它,你可能会相信我们,但你一定对此感觉不佳。因此,构造性证明\emph{在主观上会更好一些},我们将尽可能地采用构造性证明。但有时,构造性证明并不显而易见,我们不得不采用非构造性证明,就像我们在上面所做的那样。


% !TeX root = ../../../book.tex
\subsection{回顾:索引集运算与量词}

回顾一下第 \ref{sec:section3.6.2} 节,我们分别在定义 \ref{def:definition3.6.1} 和 \ref{def:definition3.6.2} 中定义了对索引集执行的集合操作(并集和交集)。主要思想是我们可以使用速记符一次性表达整个集合类的并集/交集。

仔细看看这些定义。例如,一个对象是否是索引并集的元素的特征是什么?该对象必须至少是并集中一个组成集合的元素。也就是说,需要存在某个以该对象为元素的集合。这听起来像是\emph{存在量化},不是吗?

同理,一个对象是否是索引交集的元素的特征是什么?该对象需要是所有组成集合的元素。也就是说,对于所有这些集合,该对象必须是其中的元素。这是一个\emph{全称量化}。

通过这些观察,我们可以使用新的量词符号重写索引集运算的定义:

\clearpage 

\begin{definition}
    假设 $I$ 为索引集,且对于某全集 $U$, $\forall i \in I, A_i \subseteq U$。则
    \[\bigcup_{i \in I} A_i = \{x \in U \mid \exists k \in I \centerdot x \in A_k\}\]
    \[\bigcap_{i \in I} A_i = \{x \in U \mid \forall i \in I \centerdot x \in A_i\}\]
\end{definition}

再次尝试做一下第 \ref{sec:section3.6.2} 节中的一些示例和练习。现在这些定义更有意义了吗?


% !TeX root = ../../../book.tex
\subsection{习题}

\subsubsection*{温故知新}

以口头或书面的形式简要回答以下问题。这些问题全都基于你刚刚阅读的内容,所以如果忘记了具体的定义、概念或示例,可以回去重读相关部分。确保在继续学习之前能够自信地回答这些问题,这将有助于你的理解和记忆!

\begin{enumerate}[label=(\arabic*)]
    \item 数学命题的\emph{否定}是什么?陈述及其否定是如何相关的?
    \item 为什么 $\forall$ 命题的否定是 $\exists$ 命题?\\
        为什么 $\exists$ 命题的否定是 $\forall$ 命题?
    \item 什么是非构造性证明?该术语适用于什么类型的主张($\exists$ 或 $\forall$)?
    \item 考虑声明
        \[\forall x \in S \centerdot P(x)\]
        为什么它的否定不是下面这两个?
        \textcolor{red}{
            \[\forall x \notin S \centerdot P(x)\]
            \[\exists x \notin S \centerdot \neg P(x)\]
        }
\end{enumerate}

\subsubsection*{小试牛刀}

尝试回答以下问题。这些题目要求你实际动笔写下答案,或(对朋友/同学)口头陈述答案。目的是帮助你练习使用新的概念、定义和符号。题目都比较简单,确保能够解决这些问题将对你大有帮助!

\begin{enumerate}[label=(\arabic*)]
    \item 对于以下每个陈述,写出其否定形式。哪一个 --- 原式形式还是否定形式 --- 为\verb|真|?
        \begin{enumerate}[label=(\alph*)]
            \item $\forall x \in \mathbb{R} \centerdot \exists n \in \mathbb{N} \centerdot n > x$
            \item $\exists n \in \mathbb{N} \centerdot \forall x \in \mathbb{R} \centerdot n > x$
            \item $\forall x \in \mathbb{R} \centerdot \exists y \in \mathbb{R} \centerdot y = x^3$
            \item $\exists y \in \mathbb{R} \centerdot \forall x \in \mathbb{R} \centerdot y = x^3$
        \end{enumerate}
    \item 对于以下每个陈述,写出其否定形式。哪一个 --- 原式形式还是否定形式 --- 为\verb|真|?
        \begin{enumerate}[label=(\alph*)]
            \item $\exists S \in \mathcal{P}(\mathbb{N}) \centerdot \forall x \in \mathbb{N} \centerdot x \in S$
            \item $\forall S \in \mathcal{P}(\mathbb{N}) \centerdot \exists x \in \mathbb{N} \centerdot x \in S$
            \item $\forall x \in \mathbb{N} \centerdot \exists S \in \mathcal{P}(\mathbb{N}) \centerdot x \in S$
            \item $\exists x \in \mathbb{N} \centerdot \forall S \in \mathcal{P}(\mathbb{N}) \centerdot x \in S$
        \end{enumerate}
    \item 设 $I = \{x \in \mathbb{R} \mid 0 < x < 1\}$。\\
        对于以下每个定义的集合,用量词写出确定数字 $y \in \mathbb{R}$ 是否为集合元素的定义条件。\\
        然后,确定集合是什么,并使用集合构建符写出答案。\\
        (并尝试使用双重包含论证来证明你的主张!)
        \begin{enumerate}[label=(\alph*)]
            \item $\displaystyle{S =\bigcup_{x \in I}\{y \in \mathbb{R} \mid x < y < 2\}}$
            \item $\displaystyle{T =\bigcap_{x \in I}\{y \in \mathbb{R} \mid -x < y < x\}}$
            \item $\displaystyle{V =\bigcup_{x \in I}\{y \in \mathbb{R} \mid -3x < y < 4x\}}$
        \end{enumerate}
    \item 令 $P = \{y \in \mathbb{R} \mid y > 0\}$。考虑下面这个声明:
        \[\forall \varepsilon \in P \centerdot \exists \delta \in P \centerdot \forall x \in \{y \in \mathbb{R} \mid -\delta < y < \delta\} \centerdot |x^3| < \varepsilon\]
        写出该命题的逻辑否定形式。\\
        这个陈述表达了什么?它的否定形式表达了什么?\\
        哪一个为\verb|真|?你能证明这一点吗?
    \item 设 $A,B,C,D$ 为任意集合。\\
        设 $P(x),Q(x, y),R(x, y, z)$ 为任意变量命题。\\
        写出下列各命题的否定形式。
        \begin{enumerate}[label=(\alph*)]
            \item $\forall a \in A \centerdot \exists b \in B \centerdot Q(a, b)$
            \item $\forall a \in A \centerdot \neg P(a)$
            \item $\forall c \in C \centerdot \forall d \in D \centerdot \neg Q(c, d)$
            \item $\forall a_1, a_2 \in A \centerdot \forall d \in D \centerdot R(a_1, a_2, d)$
            \item $\forall b_1, b_2, b_3 \in B \centerdot \neg R(b_1, b_2, b_3)$
            \item $\exists b \in B \centerdot \forall c \in C \centerdot \forall d \in D \centerdot R(d, b, c)$
        \end{enumerate}
\end{enumerate}