% !TeX root = ../../../book.tex
\section{量化陈述的逻辑否定}\label{sec:section4.4}

让我们回顾之前使用过的例子。定义 $P(x, y)$ 为``$y = x^3$'',并定义 $Q_1$ 为陈述:
\[\text{``}\exists y \in \mathbb{R} \centerdot \forall x \in \mathbb{R} \centerdot P(x, y)\text{''}\]
定义 $Q_2$ 为陈述:
\[\text{``}\forall x \in \mathbb{R} \centerdot \exists y \in \mathbb{R} \centerdot P(x, y)\text{''}\]
请记住,$Q_1$ 为\verb|假|,$Q_2$ 为\verb|真|。

我们如何知道 $Q_1$ 为\verb|假|?该陈述声称存在某个具有特定属性的实数。要证明整个陈述为\verb|假|,可能需要验证该属性\emph{不}适用于\emph{每个}实数 $y$,但这需要无限长的时间!因为集合 $\mathbb{R}$ 是无限的。一种更有效的方法是证明该陈述的\textbf{否定}形式为\verb|真|。

这里所说的``否定''指的是\textbf{逻辑否定},即在逻辑意义上与原始陈述``相反''的陈述。数学陈述的逻辑否定具有与原始陈述相反的真值。因此,如果我们证明 $Q_1$ 的否定形式为\verb|真|,就相当于证明了 $Q_1$ 本身为\verb|假|。

那么如何否定一个陈述呢?当我们意识到需要证明关于\emph{每个}实数 $y$ 的某些结论时,已接近正确答案——因为原始陈述声称存在某个 $y$。本节将探讨如何正确否定此类量化陈述。

需要说明的是,当前讨论背后涉及微妙而深刻的数学概念。为什么数学命题必须非\verb|真|即\verb|假|?一个俏皮但正确的回答是:``因为你将`\textbf{数学陈述}'定义为如此,傻瓜!''是的,我们确实这样定义,但为何要如此?\verb|真|/\verb|假|二元性为何对数学\emph{有益}或\emph{必要}?这些都是深刻而艰难的问题,值得深入思考。相关讨论必然涉及数学哲学和人类思想的探究,这固然有趣且有价值,但已超出本书的范围和目标。我们将依赖于对真理的共同直觉理解。

% !TeX root = ../../../book.tex
\subsection{否定全称量化}

全称陈述(即``$\forall$'')的否定通常表现为存在陈述(即``$\exists$''),反之亦然。在探讨更复杂的全称量化陈述否定问题前,我们先分析一个简单示例。

设 $S$ 为集合,$R(x)$ 是定义在 $x \in S$ 上的数学命题。陈述
\[\forall x \in S \centerdot R(x)\]
对集合 $S$ 中任意元素 $x$,命题 $R(x)$ 恒为真。即无论选取 $S$ 中哪个元素 $x$,都能确保 $R(x)$ 为\verb|真|。若该陈述为\verb|假|,我们该如何\emph{证明}这一点呢?

若并非所有元素都满足该属性,则必然存在\emph{至少}一个元素\emph{不}满足该属性。为证明此结论,需要找出这样的元素:即定义(或发现)某个 $x$,并证明 $R(x)$ 对该元素不成立。(思考语言中的否定逻辑——日常交流中我们常不假思索地运用此逻辑。)因此,原陈述的否定可表述为:
\[\exists x \in S \text{\ 使得\ } R(x) \text{\ 为假}\]

我们引入符号 $\neg$ 表示``\textbf{逻辑否定}''或``\textbf{非}''。借此可将否定陈述
\[\neg\left(\forall x \in S \centerdot R(x)\right)\]
重写为
\[\exists x \in S \centerdot \neg R(x)\]
其中 $\neg R(x)$ 可根据 $R(x)$ 的具体形式简化。例如当 $S = \mathbb{R}$ 且 $R(x)$ 为``$x^2 \ge 0$''时,否定陈述为
\[\text{``}\exists x \in \mathbb{R} \text{\ 使得\ } x^2 < 0 \text{''}\]
因为``$x^2 < 0$''逻辑等价于``$\neg(x^2 \ge 0)$''。

一般而言,我们保留``$\neg R(x)$''形式而不深入分解 $R$。需强调的是: ``$R(x)$ 为\verb|假|''与``$\neg R(x)$ 为\verb|真|''逻辑等价,二者均表明命题 $R(x)$ 不成立。

以上阐述的\emph{反例}概念你可能已有所了解。为\emph{反驳}全称量化陈述,需证明对应的存在量化陈述成立;该证明过程要求显式构造集合中不满足指定属性的元素,这便是\textbf{反例}一词的由来。


% !TeX root = ../../../book.tex
\subsection{否定存在量化}

类似
\[\exists x \in S \centerdot R(x)\]
这样的陈述提出了存在性断言,表示必须存在某个元素 $x$ 满足属性 $R(x)$。为了反驳这一主张,我们需要证明任何 $x$ 的取值都无法满足 $R$。因此,陈述
\[\neg\big(\exists x \in S \centerdot R(x)\big)\]
在逻辑上等价于
\[\forall x \in S \centerdot \neg R(x)\]

考虑如何反驳这种存在性断言时,这一等价关系便显得合理。假设你正在与朋友辩论,对方声称某些 kwyjibo 具有 zooqa 的属性。你会如何反驳?你可能会说:``不对!给我任意一个你选择的 kwyjibo,我都能证明它不可能是 zooqa,理由如下……'''随后解释该属性为何不成立。

当你说``给我任意''时,实际上是在进行全称量化:你断言\emph{无论}考虑哪个 kwyjibo,都存在一个真命题;即对每个 kwyjibo(或 $\forall x \in K$,其中 $K$ 是所有 kwyjibo 的集合),某个结论为\verb|真|。

请思考为何我们定义的这个逻辑否定是合理的。本章后续讨论证明技术时,我们将解释``考虑{任意}  kwyjibo''的策略如何严格证明上述逻辑等价关系。目前,请理解
\[\forall x \in S \centerdot \neg R(x)\] 
与
\[\exists x \in S \centerdot R(x)\]
具有相反的真值。


% !TeX root = ../../../book.tex
\subsection{一般量化陈述的否定}

到目前为止,我们所做的观察引出了否定量化陈述的一般方法。我们在第 \pageref{sec:section4.4} 页定义的陈述 $Q_1$ 的形式为
\[\exists y \in \mathbb{R} \centerdot C(y)\]
其中 $C(y)$ 是陈述的其余部分(当然,这\emph{取决于} $y$ 的值)。我们将 $C(y)$ 视为量化变量 $y$ 的某些\emph{属性};该属性内部可能有其他量词和变量,但从根本上讲,它只是断言 $y$ 的一些事实。

为了否定这一说法,我们按照上面讨论的方法写做
\[\forall y \in \mathbb{R} \centerdot \neg C(y)\]
现在,我们知道 $C(y)$ 本身就是一个全称量化陈述:
\[\forall x \in \mathbb{R} \centerdot y = x^3\]
我们也知道如何否定这种类型的陈述!其否定形式 $\neg C(y)$ 为
\[\exists x \in \mathbb{R} \centerdot y = x^3\]
这一步仅使用了我们上面看到的另一个否定方法。然后,把它们放在一起,我们可以说 $\neg Q_1$ 为陈述
\[\forall y \in \mathbb{R} \centerdot \exists x \in \mathbb{R} \centerdot y \ne x^3\]
我们可以\emph{证明}这个说法是正确的,从而证明原始陈述一定为\verb|假|。

(我们将此证明留作练习。提示:给定任意 $y \in \mathbb{R}$,定义一个 $x$ 值,该值将迫使 $y \ne x^3$ 为真。请注意,你对 $x$ 的选择取决于 $y$ 的值;它们是如何依赖的?)

看看这个否定是如何产生的:我们认识到原始陈述是一个\textbf{嵌套量词}序列(即连续几个量化变量的序列),末尾有一个变量命题,并且我们看到我们可以将量词序列的一部分视为它自身的陈述。然后,我们将否定从外部量词 ``传递到内部量词'',并将这些否定拼凑在一起。

遵循同样的想法,我们可以弄清楚如何识别具有较长量词序列的陈述。例如,看看下面这个陈述
\[\forall a \in A \centerdot \exists b \in B \centerdot \exists c \in C \centerdot \forall d, e \in D \centerdot Q(a, b, c, d, e)\]
要开始否定它,我们先从第一个量化处断开,并将其余部分视为其自身的命题 $R(a)$,并且它仅依赖于 $a$:
\[\forall a \in A \centerdot \big(\underbrace{\exists b \in B \centerdot \exists c \in C \centerdot \forall d, e \in D \centerdot Q(a, b, c, d, e)}_{R(a)}\big)\]
因此否定可以写成
\[\exists a \in A \centerdot \neg R(a)\]
但我们必须找出 $\neg R(a)$ 的另一种写法。我们需要重复上面的做法!只需将 ``$\exists b \in B$'' 与其余部分分开……接着你就知道是怎么回事了。尝试自己完成这些步骤,并确保最终得到以下结果,这就是原始陈述的逻辑否定:
\[\exists a \in A \centerdot \forall b \in B \centerdot \forall c \in C \centerdot \exists d, e \in D \centerdot \neg Q(a, b, c, d, e)\]

一般来说,我们可以这样说:要否定仅由量词和变量命题组成的命题,只需将每个 ``$\forall$'' 切换为 ``$\exists$'',反之亦然,即可否定命题。不要改变我们量化的任何集合,只改变量词本身和随后的命题;改变讨论范围是没有意义的。稍后,我们将了解如何否定其他类型的陈述,即由其他连词构建的更复杂的陈述。在此之前,我们需要继续定义和讨论其他连词。


% !TeX root = ../../../book.tex
\subsection{方法总结}

本节内容总结如下:
\begin{itemize}
    \item \textbf{否定全称量化:} \\
        设 $X$ 为集合,$P(x)$ 为命题。则全称量化的否定
        \[\neg \big(\forall x \in X \centerdot P(x)\big)\]
        可表示为
        \[\exists x \in X \centerdot \neg P(x)\]
        其含义为:
        \begin{center}
            并非每个 $x \in X, P(x)$ 都成立。
        \end{center}
        等价于
        \begin{center}
            存在某个 $x \in X$ 使得 $P(x)$ 不成立。
        \end{center}
    \item \textbf{否定存在量化:} \\
        设 $X$ 为集合,$Q(x)$ 为命题。则存在量化的否定
        \[\neg \big(\exists x \in X \centerdot Q(x)\big)\]
        可表示为
        \[\forall x \in X \centerdot \neg Q(x)\]
        其含义为:
        \begin{center}
            不存在 $x \in X$ 使得 $Q(x)$ 成立。
        \end{center}
        等价于
        \begin{center}
            对于每个 $x \in X, Q(x)$ 均不成立。
        \end{center}
\end{itemize}

\subsubsection*{不要更改量化集!}

需要强调的是,否定命题时不应改变论域。下面通过现实例子说明其合理性:

假设我们声称:``这个书架上的每本书都是英文书''。要证伪该命题,必须在此书架上找出非英文书籍。若从其他书架取来法语小说,然后声称``此命题为\verb|假|'',这是无效的——原命题仅针对当前书架,未涉及其他书架的书籍。

同理,当否定陈述
\[\forall b \in T \centerdot P(b)\]
时,应在原论域 $T$ 内得到
\[\exists b \in T \centerdot \neg P(b)\]
原命题仅断言 $T$ 中元素的性质,其否定也应限定于此论域。


% !TeX root = ../../../book.tex
\subsection{排中律}

你知道吗?让我们实际讨论一下为什么我们可以谈论陈述及其\textbf{逻辑否定}。我们对\textbf{数学/逻辑陈述}的定义中包含这样一个想法:陈述的句子必须仅有一个真值,要么为\verb|真|,要么为\verb|假|。为什么我们可以这样做?在这里我们需要对定义负责!数学家必须为他们的系统设定基本规则 --- \textbf{公理},而我们希望我们的逻辑系统能够确保我们提出的每一个主张都是正确的或错误的,而不是两者兼有或二者皆无。

这种二元性确实是我们系统的\textbf{公理}。它在大多数数学中被广泛采用,被称为``\textbf{排中律}''。这个名字来源于这样一个想法,即每一个主张都要么为\verb|真|要么为\verb|假|,所以这两者之间没有\emph{中间立场};中间的部分被排除在外。

从本质上讲,这使得我们在数学方面的工作卓有成效:每个主张都有一个真值,而我们的目标就是找到该真值。有时,我们必须依靠这个公理,即我们商定的定律,来\emph{确保}某些主张要么为\verb|真|要么为\verb|假|,但不知道具体是哪个真值。下面是一个有趣且颇具代表性的例子。

\begin{proposition}
    存在实数 $a$ 和 $b$ 都为无理数,而 $a^b$ 为有理数。
\end{proposition}

(请记住,有理数是可以表示为整数相除形式的数,无理数是非有理数的实数。你能想到有理数和无理数的一些例子吗?)

\begin{proof}
    我们知道 $\sqrt{2}$ 为无理数。(问题:为什么?你能证明这一点吗?现在就试试。我们也将很快证明这一点!)

    $\sqrt{2}^{\sqrt{2}}$ 要么是有理数,要么是无理数。(这里使用了排中律。)让我们分别考虑这两种情况。
    \begin{itemize}
        \item 假设 $\sqrt{2}^{\sqrt{2}}$ 为有理数。那么令 $a=\sqrt{2}, b=\sqrt{2}$,$a, b$ 都是无理数,而 $a^b$ 为有理数。
        \item 假设 $\sqrt{2}^{\sqrt{2}}$ 为无理数。那么令 $a=\sqrt{2}^{\sqrt{2}}, b=\sqrt{2}$,$a, b$ 都是无理数,而
        \[a^b = \big(\sqrt{2}^{\sqrt{2}}\big)^{\sqrt{2}} = \sqrt{2}^{\sqrt{2} \cdot \sqrt{2}} = (\sqrt{2})^2 = 2\]
        为有理数。
    \end{itemize}
    无论哪种情况,我们都找到了实数 $a, b \in \mathbb{R}$,其中 $a$ 和 $b$ 都为无理数,而 $a^b$ 为有理数。这证明改命题为\verb|真|。
\end{proof}

这是一个\textbf{非构造性}证明的例子。它告诉我们某些东西存在(甚至将其缩小到两种可能性),但实际上并没有\emph{确切}告诉我们哪种可能性是我们要寻找的可能性。正是对排中律的直接运用才构成了这种情况。

(问题:你能证明 $\sqrt{2}^{\sqrt{2}}$ 是无理数吗?这是真的,但没有已知的``简单''方法来证明这一事实。也许你可以找到一个!)

本书中大多数证明都是\textbf{构造性}的(但不是全部)。也许你觉得这样很好,我们也倾向于觉得这样不错。如果我们声称某事物存在,我们应该能够向你\emph{展示}它,对吗?如果我们只是谈论\emph{为什么}某些这样的事物存在于某个地方,而无法指明它,你可能会相信我们,但你一定对此感觉不佳。因此,构造性证明\emph{在主观上会更好一些},我们将尽可能地采用构造性证明。但有时,构造性证明并不显而易见,我们不得不采用非构造性证明,就像我们在上面所做的那样。


% !TeX root = ../../../book.tex
\subsection{回顾:索引集运算与量词}

回顾第 \ref{sec:section3.6.2} 节,我们在定义 \ref{def:definition3.6.1} 和 \ref{def:definition3.6.2} 中分别介绍了索引集的集合运算(并集和交集)。其核心思想是通过简洁的符号表示整个集合族的并集或交集。

仔细分析这些定义:一个对象属于索引并集的条件是什么?它必须是至少一个组成集合的元素,即存在包含该对象的集合。这本质上是一种\emph{存在量化}。

同理,一个对象属于索引交集的条件是什么?它必须属于所有组成集合,即对于每个集合而言该对象都是其元素。这本质上是一种\emph{全称量化}。

基于上述观察,我们可以用量词符号重新表述索引集运算的定义:

\begin{definition}
    设 $I$ 为索引集,且对于某全集 $U$ 满足 $\forall i \in I, A_i \subseteq U$。则
    \[\bigcup_{i \in I} A_i = \{x \in U \mid \exists k \in I \centerdot x \in A_k\}\]
    \[\bigcap_{i \in I} A_i = \{x \in U \mid \forall i \in I \centerdot x \in A_i\}\]
\end{definition}

建议重做第 \ref{sec:section3.6.2} 节中的例题与习题。现在这些定义是否更加清晰?


% !TeX root = ../../../book.tex
\subsection{习题}

\subsubsection*{温故知新}

以口头或书面的形式简要回答以下问题。这些问题全都基于你刚刚阅读的内容,所以如果忘记了具体的定义、概念或示例,可以回去重读相关部分。确保在继续学习之前能够自信地回答这些问题,这将有助于你的理解和记忆!

\begin{enumerate}[label=(\arabic*)]
    \item 数学命题的\emph{否定}是什么?陈述及其否定是如何相关的?
    \item 为什么 $\forall$ 命题的否定是 $\exists$ 命题?\\
        为什么 $\exists$ 命题的否定是 $\forall$ 命题?
    \item 什么是非构造性证明?该术语适用于什么类型的主张($\exists$ 或 $\forall$)?
    \item 考虑声明
        \[\forall x \in S \centerdot P(x)\]
        为什么它的否定不是下面这两个?
        \textcolor{red}{
            \[\forall x \notin S \centerdot P(x)\]
            \[\exists x \notin S \centerdot \neg P(x)\]
        }
\end{enumerate}

\subsubsection*{小试牛刀}

尝试回答以下问题。这些题目要求你实际动笔写下答案,或(对朋友/同学)口头陈述答案。目的是帮助你练习使用新的概念、定义和符号。题目都比较简单,确保能够解决这些问题将对你大有帮助!

\begin{enumerate}[label=(\arabic*)]
    \item 对于以下每个陈述,写出其否定形式。哪一个 --- 原式形式还是否定形式 --- 为\verb|真|?
        \begin{enumerate}[label=(\alph*)]
            \item $\forall x \in \mathbb{R} \centerdot \exists n \in \mathbb{N} \centerdot n > x$
            \item $\exists n \in \mathbb{N} \centerdot \forall x \in \mathbb{R} \centerdot n > x$
            \item $\forall x \in \mathbb{R} \centerdot \exists y \in \mathbb{R} \centerdot y = x^3$
            \item $\exists y \in \mathbb{R} \centerdot \forall x \in \mathbb{R} \centerdot y = x^3$
        \end{enumerate}
    \item 对于以下每个陈述,写出其否定形式。哪一个 --- 原式形式还是否定形式 --- 为\verb|真|?
        \begin{enumerate}[label=(\alph*)]
            \item $\exists S \in \mathcal{P}(\mathbb{N}) \centerdot \forall x \in \mathbb{N} \centerdot x \in S$
            \item $\forall S \in \mathcal{P}(\mathbb{N}) \centerdot \exists x \in \mathbb{N} \centerdot x \in S$
            \item $\forall x \in \mathbb{N} \centerdot \exists S \in \mathcal{P}(\mathbb{N}) \centerdot x \in S$
            \item $\exists x \in \mathbb{N} \centerdot \forall S \in \mathcal{P}(\mathbb{N}) \centerdot x \in S$
        \end{enumerate}
    \item 设 $I = \{x \in \mathbb{R} \mid 0 < x < 1\}$。\\
        对于以下每个定义的集合,用量词写出确定数字 $y \in \mathbb{R}$ 是否为集合元素的定义条件。\\
        然后,确定集合是什么,并使用集合构建符写出答案。\\
        (并尝试使用双重包含论证来证明你的主张!)
        \begin{enumerate}[label=(\alph*)]
            \item $\displaystyle{S =\bigcup_{x \in I}\{y \in \mathbb{R} \mid x < y < 2\}}$
            \item $\displaystyle{T =\bigcap_{x \in I}\{y \in \mathbb{R} \mid -x < y < x\}}$
            \item $\displaystyle{V =\bigcup_{x \in I}\{y \in \mathbb{R} \mid -3x < y < 4x\}}$
        \end{enumerate}
    \item 令 $P = \{y \in \mathbb{R} \mid y > 0\}$。考虑下面这个声明:
        \[\forall \varepsilon \in P \centerdot \exists \delta \in P \centerdot \forall x \in \{y \in \mathbb{R} \mid -\delta < y < \delta\} \centerdot |x^3| < \varepsilon\]
        写出该命题的逻辑否定形式。\\
        这个陈述表达了什么?它的否定形式表达了什么?\\
        哪一个为\verb|真|?你能证明这一点吗?
    \item 设 $A,B,C,D$ 为任意集合。\\
        设 $P(x),Q(x, y),R(x, y, z)$ 为任意变量命题。\\
        写出下列各命题的否定形式。
        \begin{enumerate}[label=(\alph*)]
            \item $\forall a \in A \centerdot \exists b \in B \centerdot Q(a, b)$
            \item $\forall a \in A \centerdot \neg P(a)$
            \item $\forall c \in C \centerdot \forall d \in D \centerdot \neg Q(c, d)$
            \item $\forall a_1, a_2 \in A \centerdot \forall d \in D \centerdot R(a_1, a_2, d)$
            \item $\forall b_1, b_2, b_3 \in B \centerdot \neg R(b_1, b_2, b_3)$
            \item $\exists b \in B \centerdot \forall c \in C \centerdot \forall d \in D \centerdot R(d, b, c)$
        \end{enumerate}
\end{enumerate}