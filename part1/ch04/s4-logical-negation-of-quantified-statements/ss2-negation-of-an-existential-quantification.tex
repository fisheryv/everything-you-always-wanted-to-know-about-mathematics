% !TeX root = ../../../book.tex
\subsection{否定存在量化}

类似
\[\exists x \in S \centerdot R(x)\]
这样的陈述提出了存在性断言,表示必须存在某个元素 $x$ 满足属性 $R(x)$。为了反驳这一主张,我们需要证明任何 $x$ 的取值都无法满足 $R$。因此,陈述
\[\neg\big(\exists x \in S \centerdot R(x)\big)\]
在逻辑上等价于
\[\forall x \in S \centerdot \neg R(x)\]

考虑如何反驳这种存在性断言时,这一等价关系便显得合理。假设你正在与朋友辩论,对方声称某些 kwyjibo 具有 zooqa 的属性。你会如何反驳?你可能会说:``不对!给我任意一个你选择的 kwyjibo,我都能证明它不可能是 zooqa,理由如下……'''随后解释该属性为何不成立。

当你说``给我任意''时,实际上是在进行全称量化:你断言\emph{无论}考虑哪个 kwyjibo,都存在一个真命题;即对每个 kwyjibo(或 $\forall x \in K$,其中 $K$ 是所有 kwyjibo 的集合),某个结论为\verb|真|。

请思考为何我们定义的这个逻辑否定是合理的。本章后续讨论证明技术时,我们将解释``考虑{任意}  kwyjibo''的策略如何严格证明上述逻辑等价关系。目前,请理解
\[\forall x \in S \centerdot \neg R(x)\] 
与
\[\exists x \in S \centerdot R(x)\]
具有相反的真值。
