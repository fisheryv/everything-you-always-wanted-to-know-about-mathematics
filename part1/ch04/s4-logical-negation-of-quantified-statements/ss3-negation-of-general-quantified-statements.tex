% !TeX root = ../../../book.tex
\subsection{一般量化陈述的否定}

到目前为止,我们的观察揭示了否定量化陈述的一般方法。第 \pageref{sec:section4.4} 页定义的陈述 $Q_1$ 的形式为
\[\exists y \in \mathbb{R} \centerdot C(y)\]
其中 $C(y)$ 是陈述的其余部分(其内容\emph{依赖于} $y$ 的值)。我们可以将 $C(y)$ 视为被量化变量 $y$ 的某种\emph{属性};该属性内部可能包含其他量词和变量,但本质上它断言了关于 $y$ 的某个事实。

为了否定该陈述,我们采用前述方法写做:
\[\forall y \in \mathbb{R} \centerdot \neg C(y)\]
已知 $C(y)$ 本身是一个全称量化陈述:
\[\forall x \in \mathbb{R} \centerdot y = x^3\]
我们也知道如何否定此类陈述!其否定 $\neg C(y)$ 为:
\[\exists x \in \mathbb{R} \centerdot y \ne x^3\]
此步骤仅应用了上述否定规则。综合起来,$\neg Q_1$ 可表述为:
\[\forall y \in \mathbb{R} \centerdot \exists x \in \mathbb{R} \centerdot y \ne x^3\]
我们可以\emph{证明}该陈述为\verb|真|,从而证实原陈述必定为\verb|假|。

(此证明留作练习。提示:给定任意 $y \in \mathbb{R}$,构造一个 $x$ 值使得 $y \ne x^3$ 成立。注意 $x$ 的选择依赖于 $y$;思考其依赖关系。)

观察否定的形成过程:原陈述是一个\textbf{嵌套量词}序列(即连续多个量化变量),末尾为一个命题。我们将量词序列的一部分视为独立陈述,再将否定从外层量词``传递''至内层量词,最终组合这些否定结果。

遵循同样的思路,我们可以处理包含更长量词序列的陈述。例如:
\[\forall a \in A \centerdot \exists b \in B \centerdot \exists c \in C \centerdot \forall d, e \in D \centerdot Q(a, b, c, d, e)\]
否定时,先从首个量词处拆解,将剩余部分视为仅依赖于 $a$ 的命题 $R(a)$:
\[\forall a \in A \centerdot \left(\underbrace{\exists b \in B \centerdot \exists c \in C \centerdot \forall d, e \in D \centerdot Q(a, b, c, d, e)}_{R(a)}\right)\]
因此其否定可写做:
\[\exists a \in A \centerdot \neg R(a)\]
接下来需确定 $\neg R(a)$ 的表达式。重复上述过程:分离``$\exists b \in B$''并逐步推进……请自行完成推导,确保最终得到原始陈述的逻辑否定:
\[\exists a \in A \centerdot \forall b \in B \centerdot \forall c \in C \centerdot \exists d, e \in D \centerdot \neg Q(a, b, c, d, e)\]

一般而言,否定仅由量词和命题构成的陈述时,只需将每个``$\forall$''替换为``$\exists$'',反之亦然,并否定末尾命题。量化集合保持不变;改变论域并无意义。后续我们将探讨如何否定由其他逻辑连词构成的更复杂的陈述,但首先需要定义并讨论这些连词。
