% !TeX root = ../../../book.tex
\subsection{方法总结}

让我们总结一下本节的内容。
\begin{itemize}
    \item \textbf{否定全称量化:} \\
        设 $X$ 为集合,$P(x)$ 为命题。则全称量化的否定,
        \[\neg \big(\forall x \in X \centerdot P(x)\big)\]
        写为
        \[\exists x \in X \centerdot \neg P(x)\]
        用语言表达,我们已经证明了
        \begin{center}
            对于每个 $x \in X, P(x)$ 不都成立。
        \end{center}
        等价于
        \begin{center}
            存在元素 $x \in X$ 使得 $P(x)$ 不成立。
        \end{center}
    \item \textbf{否定存在量化:} \\
        设 $X$ 为集合,$Q(x)$ 为命题。则存在量化的否定,
        \[\neg \big(\exists x \in X \centerdot Q(x)\big)\]
        写为
        \[\forall x \in X \centerdot \neg Q(x)\]
        用语言表达,我们已经证明了
        \begin{center}
            不存在 $x \in X$ 使得 $Q(x)$ 成立。
        \end{center}
        等价于
        \begin{center}
            对于每个元素 $x \in X, Q(x)$ 都不成立。
        \end{center}
\end{itemize}

\subsubsection*{不要更改量化集!}

我们上面提到过,当否定一个陈述时,改变讨论范围是没有意义的。要思考为什么这是合理的,可以举一个现实生活中的例子。

假设我们说``这个书架上的每一本书都是用英语写的''。你如何证明我们在撒谎,我们的陈述实际上为\verb|假|?你必须在\emph{这个书架}上找出一本不是用英文编写的书。你不能从走廊那头的房间里拿一本法国小说说:``瞧,你错了!''这并不能证明我们的主张;因为讨论的领域不同,我们没有对其他房间书架上发生的事情做出任何声明。我们只是对这个\emph{特定}书架做出了断言。

同理,当否定陈述
\[\forall b \in T \centerdot P(b)\]
时,我们在不改变讨论领域(集合 $T$)的情况下得到
\[\exists b \in T \centerdot \neg P(b)\]
最初的声明仅断言了 $T$ 中元素的某些性质,因此它的否定也应仅断言这一点。
