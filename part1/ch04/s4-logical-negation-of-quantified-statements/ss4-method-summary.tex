% !TeX root = ../../../book.tex
\subsection{方法总结}

本节内容总结如下:
\begin{itemize}
    \item \textbf{否定全称量化:} \\
        设 $X$ 为集合,$P(x)$ 为命题。则全称量化的否定
        \[\neg \big(\forall x \in X \centerdot P(x)\big)\]
        可表示为
        \[\exists x \in X \centerdot \neg P(x)\]
        其含义为:
        \begin{center}
            并非每个 $x \in X, P(x)$ 都成立。
        \end{center}
        等价于
        \begin{center}
            存在某个 $x \in X$ 使得 $P(x)$ 不成立。
        \end{center}
    \item \textbf{否定存在量化:} \\
        设 $X$ 为集合,$Q(x)$ 为命题。则存在量化的否定
        \[\neg \big(\exists x \in X \centerdot Q(x)\big)\]
        可表示为
        \[\forall x \in X \centerdot \neg Q(x)\]
        其含义为:
        \begin{center}
            不存在 $x \in X$ 使得 $Q(x)$ 成立。
        \end{center}
        等价于
        \begin{center}
            对于每个 $x \in X, Q(x)$ 均不成立。
        \end{center}
\end{itemize}

\subsubsection*{不要更改量化集!}

需要强调的是,否定命题时不应改变论域。下面通过现实例子说明其合理性:

假设我们声称:``这个书架上的每本书都是英文书''。要证伪该命题,必须在此书架上找出非英文书籍。若从其他书架取来法语小说,然后声称``此命题为\verb|假|'',这是无效的——原命题仅针对当前书架,未涉及其他书架的书籍。

同理,当否定陈述
\[\forall b \in T \centerdot P(b)\]
时,应在原论域 $T$ 内得到
\[\exists b \in T \centerdot \neg P(b)\]
原命题仅断言 $T$ 中元素的性质,其否定也应限定于此论域。
