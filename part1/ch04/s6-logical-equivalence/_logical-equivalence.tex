% !TeX root = ../../../book.tex
\section{逻辑等价}

本节旨在介绍\textbf{逻辑等价}的概念并证明若干基本命题。其核心在于判定复杂的逻辑陈述是否具有``相同的''真值。由于数学陈述可能依赖某些命题变量,我们往往无法对其真值做出具体结论。然而,有时可以证明两个数学陈述对其包含变量的所有可能取值均具有\emph{相同的真值}。这是重要的结论!这意味着无论该真值具体如何,两者恒等。在此意义上,我们称这两个陈述在逻辑上是\textbf{等价的}。

% !TeX root = ../../../book.tex
\subsection{定义与使用}\label{sec:section4.6.1}

以下定义为逻辑等价引入了一个方便的符号:

\begin{definition}
    设 $P$ 和 $Q$ 为数学陈述。我们使用符号``$\iff$''表示``\dotuline{逻辑等价}'',或``具有相同的真值''。

    也就是说,当 $P$ 和 $Q$ 始终具有相同的真值时(无论为\verb|真|还是为\verb|假|),我们记作 ``$P \iff Q$''。

    ``$P \iff Q$''读作``$P$ 逻辑等价于 $Q$''或``$P$ \dotuline{当且仅当} $Q$''。

    此类陈述称为\dotuline{双向条件关系}(或\dotuline{双向蕴含})。
\end{definition}

复用上一节的真值表,为 $\iff$ 添加新列:
\begin{center}
    \begin{tabular}{c|c|c|c|c|c|c}
          $P$      & $Q$      & $\neg P$ &  $\neg P \lor Q$ & $P \implies Q$ & $Q \implies P$ & $P \iff Q$ \\
          \hline
          \verb|T| & \verb|T| & \verb|F| &      \verb|T|    &    \verb|T|    &    \verb|T|    & \verb|T|\\
          \verb|T| & \verb|F| & \verb|F| &      \verb|F|    &    \verb|F|    &    \verb|T|    & \verb|F|\\
          \verb|F| & \verb|T| & \verb|T| &      \verb|T|    &    \verb|T|    &    \verb|F|    & \verb|F|\\
          \verb|F| & \verb|F| & \verb|T| &      \verb|T|    &    \verb|T|    &    \verb|T|    & \verb|T|\\
    \end{tabular}
\end{center}
在 $P \iff Q$ 列中,当且仅当 $P$ 和 $Q$ 真值相同时,该行取值为 \verb|T|(第 $1$ 行二者皆为 \verb|T|,第 $4$ 行二者皆为 \verb|F|)。注意 $P \iff Q$ 为真当且仅当
\[(P \implies Q) \land (Q \implies P)\]
成立。这体现了\textbf{逻辑等价}的本质:$P \iff Q$ 表明 $P \implies Q$ 与 $Q \implies P$ 同时成立。无论 $P$ 取值如何,$Q$ 必与之相同,反之亦然:
\begin{itemize}
    \item 假设 $P$ 为\verb|真|,则 $P \implies Q$ 要求 $Q$ 也必须为\verb|真|。
    \item 假设 $P$ 为\verb|假|,则 $Q \implies P$ 要求 $Q$ 不可能为\verb|真|(否则 $Q \implies P$ 为\verb|假|),因此 $Q$ 也必须为\verb|假|。
\end{itemize}
无论哪种情况,$P$ 和 $Q$ 都具有相同的真值。

\subsubsection*{示例}

\begin{example}
    观察前述真值表第三、第四列,可得如下逻辑等价关系:
    \[(P \implies Q) \iff (\neg P \lor Q)\]
    无论 $P \implies Q$ 的真值是什么(当然,取决于 $P$ 和 $Q$),它都必然与 $\neg P \lor Q$ 具有相同的真值。我们之前已经提及这种等价关系,后续将频繁使用。
\end{example}

\begin{example}
    请看下面的真值表:
    \begin{center}
        \begin{tabular}{c|c|c|c|c|c}
              $P$      & $Q$      & $\neg P$ &  $\neg Q$  & $P \implies Q$ & $\neg Q \implies \neg P$ \\
              \hline
              \verb|T| & \verb|T| & \verb|F| &  \verb|F|  &    \verb|T|    &    \verb|T|    \\
              \verb|T| & \verb|F| & \verb|F| &  \verb|T|  &    \verb|F|    &    \verb|F|    \\
              \verb|F| & \verb|T| & \verb|T| &  \verb|F|  &    \verb|T|    &    \verb|T|    \\
              \verb|F| & \verb|F| & \verb|T| &  \verb|T|  &    \verb|T|    &    \verb|T|    \\
        \end{tabular}
    \end{center}
    可见,无论 $P$ 和 $Q$ 的真值如何,$P \implies Q$ 与 $\neg Q \implies \neg P$ 具有相同的真值。故二者\emph{逻辑等价},可以写作:
    \[(P \implies Q) \iff (\neg Q \implies \neg P)\]
    这正是我们上一节中陈述(但没有证明)的事实:
    \begin{center}
        条件陈述的逆否命题与原陈述逻辑等价。
    \end{center}

    这一事实的另一种证明方法利用了将条件陈述改写为析取形式。回忆一下前例中提到的逻辑等价关系
    \[(P \implies Q) \iff (\neg P \lor Q)\]
    现在,考虑其逆否命题:
    \[\neg Q \implies \neg P\]
    将相同的析取形式应用于该陈述会产生以下等价关系:
    \[(\neg Q \implies \neg P) \iff \big(\neg(\neg Q) \lor \neg P\big)\]
    而 $\neg(\neg Q)$ 等价于 $Q$,并且析取的顺序是无关的(即 $P \lor Q$ 与 $Q \lor P$ 具有相同的真值),因此可得
    \[(\neg Q \implies \neg P) \iff (\neg P \lor Q) \iff (P \implies Q)\]
    这从另一个角度证明,条件陈述与其逆否形式具有相同的真值!
\end{example}

\begin{example}
    本节后续将证明下列逻辑等价对任意命题 $P$、$Q$、$R$ 均成立:
    \begin{align*}
        \neg(P \land Q) &\iff \neg P \lor \neg Q \\
        (P \land Q) \land R &\iff P \land (Q \land R) \\
        P \lor (Q \land R) &\iff (P \lor Q) \land (P \lor R) \\
        \neg (P \implies Q) &\iff P \land \neg Q
    \end{align*}
    以上每个等价式均表明 $\iff$ 两侧的表达式具有相同的真值。你能理解其正确性吗?能否构思证明方法?
\end{example}

\subsubsection*{当且仅当}

逻辑等价与短语``当且仅当''密切相关。说``$P$ 当且仅当 $Q$''意味着我们同时断言``$P$ 当 $Q$''和``$P$ 仅当 $Q$''成立。前一个分句对应 $Q \implies P$,后一个分句对应 $P \implies Q$,因此断言两者均为\verb|真|等价于:
\begin{center}
    $P \iff Q$ 与 $(P \implies Q) \land (Q \implies P)$ 含义相同。
\end{center}

具体而言,当我们说``$P$ 当 $Q$''时,这意味着``若 $Q$ 成立,则 $P$ 成立''。即,
\begin{center}
    $P$ 当 $Q$ 与 $Q \implies P$ 含义相同。
\end{center}

理解另一方向则需要更仔细的分析。``$P$ 仅当 $Q$''断言:若 $P$ 成立,则 $Q$ 必然成立。换言之,$P$ 为真时 $Q$ 必定为真,这等价于 $P \implies Q$。

另一种理解是:``仅当 $Q$ 时 $P$''等同于``$P$ 成立而 $Q$ 不成立的情况不可能发生'',其逻辑表达式为:
\[\neg(P \land \neg Q)\]
后文将讲解并证明\textbf{德摩根定律}(可提前参考 \ref{sec:section4.6.5} 和 \ref{sec:section4.6.6} 节),该定律表明上式等价于:
\[\neg P \lor Q\]
如前所述,此式逻辑等价于 $P \implies Q$。这再次验证``$P$ 仅当 $Q$''意味着 $P \implies Q$。

\subsubsection*{在定义中使用 $\iff$}

我们会经常在\textbf{定义}中使用 ``$\iff$''符号,表示所定义项与所描述性质完全等价。例如:
\begin{center}
    称 $x \in \mathbb{Z}$ 为\textbf{偶数} $\iff \exists k \in \mathbb{Z} \centerdot x = 2k$
\end{center}
也就是说,整数为偶数与它能表示为某个整数的两倍等价。类似地,我们可以定义\textbf{奇数}:
\begin{center}
    称 $x \in \mathbb{Z}$ 为\textbf{奇数} $\iff \exists k \in \mathbb{Z} \centerdot x = 2k+1$
\end{center}
请注意,以上为形式化定义,完整描述了偶数和奇数的本质。后续将用这些定义严格证明整数的性质。每当我们想断言某个整数 $x$ 为偶数,需要证明存在整数 $k$ 满足 $x = 2k$,即通过满足定义中的逻辑等价来验证。

\subsubsection*{双向条件陈述:技术上的区别}

我们还可以用``$\iff$''符号同时表达两个条件陈述。技术上这与断言逻辑等价并\emph{不完全}等同,但两者传达类似的思想,因此我们允许该符号的两种用法。

逻辑等价涉及未定义命题,断言两命题在任何情况下均具有相同的真值。例如:
\[(P \implies Q) \iff (\neg P \lor Q)\]
是逻辑等价的典型例子。若不知 $P$ 和 $Q$ 的具体含义,虽无法确定 $P \implies Q$ 与 $\neg P \lor Q$ 的实质意义,但可知二者真值必然相同。

当``$\iff$''两侧均为不含未定义命题的确定数学陈述时,情况略有不同。例如:
\[\forall x \in \mathbb{R} \centerdot (x > 0) \iff \Big(\frac{1}{x}>0\Big)\]
此断言表明:对于任意实数 $x$,若已知其中一个条件成立($x > 0$ 或 $\frac{1}{x} > 0$),则另一条件必然成立。换言之,若告知某实数为正,可推知其倒数亦为正;反之,若告知某实数的倒数为正,亦可推知该数本身为正。此即\emph{双向性}。(思考:若告知某\emph{负}实数,能否推知其倒数性质?原因何在?)

你看出区别了吗?此处对于任意 $x \in \mathbb{R}$,陈述``$x > 0$''必有确定真值。这与前面给出的例子不同,前述例子中各陈述的真值未知,但我们仍然可以声明两个陈述的真值必然相同。

由于缺乏更广泛的术语,我们称此类陈述为\textbf{双向条件陈述}。其本质是两个``方向相反''的条件陈述:
\[\forall x \in \mathbb{R} \centerdot \Bigg[\Bigg((x > 0) \implies \Big(\frac{1}{x}>0\Big)\Bigg) \land \Bigg(\Big(\frac{1}{x}>0\Big) \implies (x > 0)\Bigg)\Bigg]\]
这就是上面陈述所说的:陈述的每一部分都蕴涵着另一部分。

该术语在其他数学著作中不一定是标准术语,我们在此指出这种技术差异,以便你了解它。若在数理逻辑学家或集合论学家面前使用``逻辑等价''指代此概念,可能会引发困惑。请注意此差异虽然细微,但现阶段学习基础概念时,无需强记技术细节。本书后续内容中,``逻辑等价''与``双向条件''可互换使用,此做法目前可以接受。

``$\iff$''的核心功能是断言两个陈述\emph{真值相同}。``逻辑等价''与``双向条件''之间唯一的区别在于是否包含未定义的命题,此区别甚微,故本书将二者合并处理。


% !TeX root = ../../../book.tex
\subsection{必要条件和充分条件}

数学中偶尔会使用两个术语来表达双向条件陈述的两个方向:\textbf{充分条件}和\textbf{必要条件}。它们与双向条件的``当''和``如果''完全对应。这些术语是由数学家所提问题的自然类型带来的。

\subsubsection*{充分条件:$P$ 当 $Q$}

如果我们发现了一个数学对象的一些有趣的事实或属性(称之为 $P$),我们可能会想,``我们什么时候才能\emph{保证}这样的属性成立?我们是否可以检验某些条件,以便立即给出`肯定'的答案?'' 这就是\textbf{充分条件},能够保证 $P$ 成立的属性。它是``充分''的,因为它``足以''得出 $P$ 的结论;我们不需要任何其他额外信息。

假设我们已经将命题 $Q$ 确定为 $P$ 的充分条件。我们如何逻辑地表达这一点呢?好吧,知道 $Q$ 就足以得出 $P$ 的结论,所以我们可以轻松地将其写为条件陈述:
\begin{center}
    $Q \implies P \qquad$ 意味着 $Q$ 是 $P$ 的充分条件
\end{center}
换句话说,这个条件陈述表达的是:``$P$ 当 $Q$''。

\subsubsection*{必要条件:$P$ 仅当 $Q$}

我们也可能想知道,``我们如何保证 $P$ 为假?我们是否可以检验某些条件从而立即得知这一点?''这就是\textbf{必要条件},即属性 $P$ 成立所必需或必要的属性。这个条件不一定足以得出 $P$ 成立的结论,但为了让 $P$ 有可能成立,这个条件最好也成立。

思考一下这里的逻辑联系。假设我们已知属性 $Q$ 是 $P$ 的必要条件。我们如何符号化地表达 $P$ 和 $Q$ 之间的关系?没错,我们可以使用条件陈述。知道 $P$ 成立就告诉我们 $Q$ 肯定成立;$P$ 必然为真。这可以表示为
\begin{center}
    $P \implies Q \qquad$ 意味着 $Q$ 是 $P$ 的必要条件
\end{center}
换句话说,这个条件陈述表达的是:``$P$ 仅当 $Q$''。

我们也可以从逆否的角度来思考这一点。如果 $Q$ 不成立,则 $P$ 也不成立。即,
\[\neg Q \implies \neg P\]
这是上面条件陈述 $P \implies Q$ 的逆否形式。我们知道这是原陈述的逻辑等价形式。

\subsubsection*{示例}

\begin{example}
    令 $P(x)$ 表示命题 ``$x$ 为能被 $6$ 整除的整数''。对于以下每个条件,确定其是否是 $P(x)$ 成立的\textbf{必要}条件或\textbf{充分}条件(或可能两者都是!)。
    \begin{enumerate}[label=(\arabic*)]
        \item 令 $Q(x)$ 为 ``$x$ 为可被 $3$ 整除的整数''。
            \begin{itemize}
                \item 为了确定 $Q(x)$ 是否是必要条件,我们假设 $P(x)$ 成立。我们也能推导出 $Q(x)$ 成立吗?是的!说一个整数 $x$ 能被 $6$ 整除,意味着它能被 $2$ 和 $3$ 整除。因此,它肯定能被 $3$ 整除,所以 $Q(x)$成立。
                \item 为了确定 $Q(x)$ 是否是充分条件,我们假设 $Q(x)$ 成立。我们也能推导出 $P(x)$ 成立吗?知道 $x$ 是一个能被 $3$ 整除的整数,那么它是否也\emph{一定}能被 $2$ 整除,从而得出它能被 $6$ 整除的结论?我们认为不是!举个反例 $x = 3$;注意 $Q(3)$ 成立,但 $P(3)$ 不成立。
            \end{itemize}
            这说明 $Q(x)$ 只是必要条件,不是充分条件。
        \item 令 $R(x)$ 为 ``$x$ 为可被 $12$ 整除的整数''。\\
            按照与上例类似的推理,我们可以得出结论,$R(x)$ 是 $P(x)$ 的充分条件,但不是必要条件(因为我们可以举出反例 $x = 6$,其中 $P(6)$ 成立,但 $R(6)$ 不成立)。
        \item 令 $S(x)$ 为 ``$x$ 为 $x^2$ 可被 $6$ 整除的整数''。\\
            这个问题留给你来完成……$S(x)$ 是 $P(x)$ 的必要条件吗?是充分条件吗?\\
            请注意我们给定 $x$ 本身是一个整数……
    \end{enumerate}
\end{example}


% !TeX root = ../../../book.tex
\subsection{证明逻辑等价:结合律}

现在,让我们实际\textbf{证明}一些逻辑等价!在此过程中,我们将提升使用量词和逻辑连接词来阅读、理解及编写逻辑陈述的能力。同时,我们将推导出一些基本逻辑结论,以便在后续发展中应用这些结论来构建证明技术。这些技术将成为我们工作的基石,后续所有工作都将涉及证明策略与逻辑概念的某种组合。

让我们从一些比较简单的符号逻辑定律开始。运算满足\emph{结合律}意味着可以自由调整括号位置而不改变结果。例如加法满足结合律:计算 $x + (y + z)$ 等价于计算 $(x + y) + z$,两者结果相同。正因如此,我们可以安全地省略括号,直接写做:
\[x + y + z\]
因为加法顺序不影响最终结果。类似地,逻辑陈述的合取与析取也满足结合律,这正是我们将要证明的内容。

\begin{theorem}
    设 $P, Q, R$ 为逻辑陈述。则
    \[P \land (Q \land R) \iff (P \land Q) \land R\]
    \[P \lor (Q \lor R) \iff (P \lor Q) \lor R\]
\end{theorem}

我们将通过两种不同方法证明该命题:
\begin{enumerate}[label=(\arabic*)]
    \item 真值表法
    \item 语义法(即自然语言描述)
\end{enumerate}
两种方法均有效,加下来我们将展示两种方法,以供选择偏好的证明风格。

\begin{proofs}{证明 1. }
    首先,通过真值表证明合取结合律:
    \begin{center}
        \begin{tabular}{c|c|c|c|c|c|c}
              $P$    & $Q$   & $R$ & $P \land Q$ &  $Q \land R$  & $P \land (Q \land R)$ & $(P \land Q) \land R$ \\
              \hline
              \verb|T| & \verb|T| & \verb|T| &  \verb|T|  &    \verb|T|    &\verb|T| &    \verb|T|    \\
              \verb|T| & \verb|T| & \verb|F| &  \verb|T|  &    \verb|F|    &\verb|F| &    \verb|F|    \\
              \verb|T| & \verb|F| & \verb|T| &  \verb|F|  &    \verb|F|    &\verb|F| &    \verb|F|    \\
              \verb|T| & \verb|F| & \verb|F| &  \verb|F|  &    \verb|F|    &\verb|F| &    \verb|F|    \\
              \verb|F| & \verb|T| & \verb|T| &  \verb|F|  &    \verb|T|    &\verb|F| &    \verb|F|    \\
              \verb|F| & \verb|T| & \verb|F| &  \verb|F|  &    \verb|F|    &\verb|F| &    \verb|F|    \\
              \verb|F| & \verb|F| & \verb|T| &  \verb|F|  &    \verb|F|    &\verb|F| &    \verb|F|    \\
              \verb|F| & \verb|F| & \verb|F| &  \verb|F|  &    \verb|F|    &\verb|F| &    \verb|F|    \\
        \end{tabular}
    \end{center}

    因为 $P \land (Q \land R)$ 和 $(P \land Q) \land R$ 在所有情况下真值相同,因此 $P \land (Q \land R) \iff (P \land Q) \land R$。\\

    再通过真值表证明析取结合律:
    \begin{center}
        \begin{tabular}{c|c|c|c|c|c|c}
              $P$    & $Q$   & $R$ & $P \lor Q$ &  $Q \lor R$  & $P \lor (Q \lor R)$ & $(P \lor Q) \lor R$ \\
              \hline
              \verb|T| & \verb|T| & \verb|T| &  \verb|T|  &    \verb|T|    &\verb|T| &    \verb|T|    \\
              \verb|T| & \verb|T| & \verb|F| &  \verb|T|  &    \verb|T|    &\verb|T| &    \verb|T|    \\
              \verb|T| & \verb|F| & \verb|T| &  \verb|T|  &    \verb|T|    &\verb|T| &    \verb|T|    \\
              \verb|T| & \verb|F| & \verb|F| &  \verb|T|  &    \verb|F|    &\verb|T| &    \verb|T|    \\
              \verb|F| & \verb|T| & \verb|T| &  \verb|T|  &    \verb|T|    &\verb|T| &    \verb|T|    \\
              \verb|F| & \verb|T| & \verb|F| &  \verb|T|  &    \verb|T|    &\verb|T| &    \verb|T|    \\
              \verb|F| & \verb|F| & \verb|T| &  \verb|F|  &    \verb|T|    &\verb|T| &    \verb|T|    \\
              \verb|F| & \verb|F| & \verb|F| &  \verb|F|  &    \verb|F|    &\verb|F| &    \verb|F|    \\
        \end{tabular}
    \end{center}

    因为 $P \lor (Q \lor R)$ 和 $(P \lor Q) \lor R$ 在所有情况下真值相同,因此 $P \lor (Q \lor R) \iff (P \lor Q) \lor R$。
\end{proofs}

\begin{proofs}{证明 2. }
    接下来,我们通过语义分析证明该命题。\\
    首先考虑合取形式:
    \[P \land (Q \land R) \iff (P \land Q) \land R\]
    为了证明符号两侧表达式\emph{逻辑等价},需要证明以下两个条件陈述均为\verb|真|:
    \[P \land (Q \land R) \implies (P \land Q) \land R \qquad (P \land Q) \land R \implies P \land (Q \land R)\]
    \begin{itemize}
        \item[($\implies$)] 首先,证明第一个条件陈述。假设 $P \land (Q \land R)$ 为\verb|真|,则 $P$ 为\verb|真|且 $Q \land R$ 为\verb|真|。根据定义,这意味着 $P$、$Q$、$R$ 均为\verb|真|。此时 $P \land Q$ 为\verb|真|且 $R$ 为\verb|真|,因此 $(P \land Q) \land R$ 为\verb|真|。
        \item[($\impliedby$)] 接着,证明第二个条件陈述。假设 $(P \land Q) \land R$ 为\verb|真|,则 $P \land Q$ 为\verb|真|且 $R$ 为\verb|真|。根据定义,这意味着 $P$、$Q$、$R$ 均为\verb|真|。此时 $P$ 为\verb|真|且 $Q \land R$ 为\verb|真|,因此 $P \land (Q \land R)$ 为\verb|真|。
    \end{itemize}
    由上述证明可知二者逻辑等价。\\ \\
    再考虑析取形式:
    \[P \lor (Q \lor R) \iff (P \lor Q) \lor R\]
    为了证明符号两侧表达式\emph{逻辑等价},需要证明以下两个条件陈述均为\verb|真|:
    \[P \lor (Q \lor R) \implies (P \lor Q) \lor R \qquad (P \lor Q) \lor R \implies P \lor (Q \lor R)\]
    \begin{itemize}
        \item[($\implies$)] 首先,证明第一个条件陈述。假设 $P \lor (Q \lor R)$ 为\verb|真|。这意味着 $P$ 为\verb|真|或 $Q \lor R$ 为\verb|真|。分两种情况讨论:
        \begin{enumerate}
            \item 假设 $P$ 为\verb|真|。这意味着 $P \lor Q$ 为\verb|真|。因此,根据定义 $(P \lor Q) \lor R$ 也为\verb|真|。
            \item 假设 $Q \lor R$ 为\verb|真|。这意味着 $Q$ 为\verb|真|或 $R$ 为\verb|真|。同理,分两种情况讨论:
            \begin{enumerate}[label=(\alph*)]
                \item 假设 $Q$ 为\verb|真|。这意味着 $P \lor Q$ 为\verb|真|。因此,根据定义 $(P \lor Q) \lor R$ 也为\verb|真|。
                \item 假设 $R$ 为\verb|真|。这意味着 $(P \lor Q) \lor R$ 为\verb|真|。
            \end{enumerate}
        \end{enumerate} 
        综上,无论何种情况,都有 $(P \lor Q) \lor R$ 为\verb|真|。因此,该条件陈述为\verb|真|。

        \item[($\impliedby$)] 接着,证明第二个条件陈述。假设 $(P \lor Q) \lor R$ 为\verb|真|。这意味着 $P \lor Q$ 为\verb|真|或 $R$ 为\verb|真|。分两种情况讨论:
        \begin{enumerate}
            \item 假设 $P \lor Q$ 为\verb|真|。这意味着 $P$ 为\verb|真|或 $Q$ 为\verb|真|。同理,分两种情况讨论:
            \begin{enumerate}[label=(\alph*)]
                \item 假设 $P$ 为\verb|真|。这意味着 $P \lor (Q \lor R)$ 为\verb|真|。
                \item 假设 $Q$ 为\verb|真|。这意味着 $Q \lor R$ 为\verb|真|。因此,根据定义,$P \lor (Q \lor R)$ 为\verb|真|。
            \end{enumerate}
            \item 假设 $R$ 为\verb|真|。这意味着 $Q \lor R$ 为\verb|真|。因此,根据定义,$P \lor (Q \lor R)$ 为\verb|真|。
        \end{enumerate}   
        综上,无论何种情况,都有 $P \lor (Q \lor R)$ 为\verb|真|。因此,该条件陈述为\verb|真|。 
    \end{itemize}
    上述两个条件陈述均成立,因此二者逻辑等价。
\end{proofs}

仔细理解上述证明,我们通过这些证明达成了什么?我们已经证明了什么?如何证明的?为什么这种证明有效?

在继续比较这两种证明方法之前,先明确证明的结论:我们证明了逻辑连词``$\land$''和``$\lor$''满足结合律。因此,在处理仅涉及同一种连词的陈述时,括号的顺序不再重要。例如,``$P \land (Q \land R)$''与``$(P \land Q) \land R$''具有相同的含义。今后我们将省略括号,直接写做``$P \land Q \land R$''。

\subsubsection*{反思:真值表与语义证明}

我们先来谈谈真值表。由于 $P,Q,R$ 是逻辑陈述,其真值非真即假。真值表的八行穷尽了这三个陈述的所有真值组合。前三列标示 $P,Q,R$ 的真假,后续两列对应命题中的复合子部分,最后两列则代表需要验证的两个命题。通过对比最后两列,可判定二者是否逻辑等价(即``无论 $P,Q,R$ 取何真值,两个命题真值始终相同'')。若两列真值逐行一致,即可证明逻辑等价。

接下来讨论语义证明。这种证明可能显得冗长,但它是严谨的吗?证明过程是否清晰?逻辑是否严密?请重读证明并思考这些问题。需要强调的是,该证明完全正确。当证明析取(``或''命题)时,分情况讨论至关重要;而通过假设前提为\verb|真|来推导结论,正是验证条件命题的有效方法。我们将进一步分析这些技巧,希望此处的示例对后续学习有所助益。

本节后续内容将采用真值表验证此类简单命题——这种方法更为简洁!如果你需要更具说服力的证明,或希望通过自然语言解读逻辑命题从而进行额外练习,可以详细研究前述示例的语义证明。


% !TeX root = ../../../book.tex
\subsection{证明逻辑等价:分配律}

在算术中,你一定知道乘法分配律。也就是说,我们知道:
\[\forall x, y, z \in \mathbb{R} \centerdot x \cdot (y + z) = x \cdot y + x \cdot z\]
这个符号表示法精确表达了乘法分配律!你能看出它与你熟知的规则一致吗?

接下来,我们将研究并证明两个类似的定律。这些定律表明逻辑连接词``$\land$''和``$\lor$''也满足分配律。

\begin{theorem}
    设 $P, Q, R$ 为逻辑陈述。则
    \[P \land (Q \lor R) \iff (P \land Q) \lor (P \land R)\]
    \[P \lor (Q \land R) \iff (P \lor Q) \land (P \lor R)\]
\end{theorem}

\begin{proof}
    我们使用真值表来验证这两个命题。对于第一个命题,考虑以下真值表:
    \begin{center}
        \begin{tabular}{c|c|c|c|c|c|c|c}
              $P$ & $Q$ & $R$ & $Q \lor R$ & $P \land Q$  & $P \land R$ & $P \land (Q \lor R)$ & $(P \land Q) \lor (P \land R)$ \\
              \hline
              \verb|T| & \verb|T| & \verb|T| &  \verb|T|  &   \verb|T|   &\verb|T| &\verb|T| &   \verb|T|   \\
              \verb|T| & \verb|T| & \verb|F| &  \verb|T|  &   \verb|T|   &\verb|F| &\verb|T| &   \verb|T|   \\
              \verb|T| & \verb|F| & \verb|T| &  \verb|T|  &   \verb|F|   &\verb|T| &\verb|T| &   \verb|T|   \\
              \verb|T| & \verb|F| & \verb|F| &  \verb|F|  &   \verb|F|   &\verb|F| &\verb|F| &   \verb|F|   \\
              \verb|F| & \verb|T| & \verb|T| &  \verb|T|  &   \verb|F|   &\verb|F| &\verb|F| &   \verb|F|   \\
              \verb|F| & \verb|T| & \verb|F| &  \verb|T|  &   \verb|F|   &\verb|F| &\verb|F| &   \verb|F|   \\
              \verb|F| & \verb|F| & \verb|T| &  \verb|T|  &   \verb|F|   &\verb|F| &\verb|F| &   \verb|F|   \\
              \verb|F| & \verb|F| & \verb|F| &  \verb|F|  &   \verb|F|   &\verb|F| &\verb|F| &   \verb|F|   \\
        \end{tabular}
    \end{center}

    注意,最后两列的真值完全相同,这证明了所需的逻辑等价。\\

    对于第二个命题,参考以下真值表:
    \begin{center}
        \begin{tabular}{c|c|c|c|c|c|c|c}
              $P$ & $Q$ & $R$ & $Q \land R$ & $P \lor Q$  & $P \lor R$ & $P \lor (Q \land R)$ & $(P \lor Q) \land (P \lor R)$ \\
              \hline
              \verb|T| & \verb|T| & \verb|T| &  \verb|T|  &   \verb|T|   &\verb|T| &\verb|T| &   \verb|T|   \\
              \verb|T| & \verb|T| & \verb|F| &  \verb|F|  &   \verb|T|   &\verb|T| &\verb|T| &   \verb|T|   \\
              \verb|T| & \verb|F| & \verb|T| &  \verb|F|  &   \verb|T|   &\verb|T| &\verb|T| &   \verb|T|   \\
              \verb|T| & \verb|F| & \verb|F| &  \verb|F|  &   \verb|T|   &\verb|T| &\verb|T| &   \verb|T|   \\
              \verb|F| & \verb|T| & \verb|T| &  \verb|T|  &   \verb|T|   &\verb|T| &\verb|T| &   \verb|T|   \\
              \verb|F| & \verb|T| & \verb|F| &  \verb|F|  &   \verb|T|   &\verb|F| &\verb|F| &   \verb|F|   \\
              \verb|F| & \verb|F| & \verb|T| &  \verb|F|  &   \verb|F|   &\verb|T| &\verb|F| &   \verb|F|   \\
              \verb|F| & \verb|F| & \verb|F| &  \verb|F|  &   \verb|F|   &\verb|F| &\verb|F| &   \verb|F|   \\
        \end{tabular}
    \end{center}

    同理,最后两列的真值完全相同,从而证明了所需的逻辑等价。
\end{proof}


% !TeX root = ../../../book.tex
\subsection{证明逻辑等价:德摩根定律(逻辑)}\label{sec:section4.6.5}

接下来我们将证明一些涉及否定的逻辑等价。以下两条定律以英国数学家\textbf{奥古斯塔斯·德·摩根 (Augustus De Morgan)}的名字命名。他因建立这些逻辑定律并引入\textbf{数学归纳法}这一术语而广受赞誉。我们深深感谢他在数学领域的卓越贡献。

德·摩根逻辑定律描述了否定运算与合取、析取之间的逻辑等价关系。

\begin{theorem}
    设 $P$ 和 $Q$ 为逻辑陈述。则
    \[\neg (P \land Q) \iff \neg P \lor \neg Q\]
    \[\neg (P \lor Q) \iff \neg P \land \neg Q\]
\end{theorem}

\begin{proof}
    我们通过真值表证明第一个命题:
    \begin{center}
        \begin{tabular}{c|c|c|c|c|c|c}
              $P$      & $\neg P$ &   $Q$   &  $\neg Q$  & $P \land Q$ & $\neg (P \land Q)$ & $\neg P \lor \neg Q$ \\
              \hline
              \verb|T| & \verb|F| & \verb|T| &  \verb|F|  &    \verb|T|    &    \verb|F|   & \verb|F| \\
              \verb|T| & \verb|F| & \verb|F| &  \verb|T|  &    \verb|F|    &    \verb|T|   & \verb|T| \\
              \verb|F| & \verb|T| & \verb|T| &  \verb|F|  &    \verb|F|    &    \verb|T|   & \verb|T| \\
              \verb|F| & \verb|T| & \verb|F| &  \verb|T|  &    \verb|F|    &    \verb|T|   & \verb|T| \\
        \end{tabular}
    \end{center}
    
    同理用真值表证明第二个命题:
    \begin{center}
        \begin{tabular}{c|c|c|c|c|c|c}
              $P$      & $\neg P$ &   $Q$    &  $\neg Q$  & $P \lor Q$ & $\neg (P \lor Q)$ & $\neg P \land \neg Q$ \\
              \hline
              \verb|T| & \verb|F| & \verb|T| &  \verb|F|  &    \verb|T|    &    \verb|F|   & \verb|F| \\
              \verb|T| & \verb|F| & \verb|F| &  \verb|T|  &    \verb|T|    &    \verb|F|   & \verb|F| \\
              \verb|F| & \verb|T| & \verb|T| &  \verb|F|  &    \verb|T|    &    \verb|F|   & \verb|F| \\
              \verb|F| & \verb|T| & \verb|F| &  \verb|T|  &    \verb|F|    &    \verb|T|   & \verb|T| \\
        \end{tabular}
    \end{center}
\end{proof}

这两条定律非常有用!事实上,我们可以借助它们证明集合论中的类似结论。


% !TeX root = ../../../book.tex
\subsection{使用逻辑等价:德摩根定律(集合)}\label{sec:section4.6.6}

以下陈述与我们先前讨论的德摩根逻辑定律具有高度相似性。通过后续证明,我们将揭示这种相似性的根源。

\begin{theorem}\label{theorem4.6.9}
    设 $A, B$ 为全集 $U$ 的任意子集,则:
    \[\overline{A \cup B} = \overline{A} \cap \overline{B}\]
    \[\overline{A \cap B} = \overline{A} \cup \overline{B}\]
\end{theorem}

以下证明将运用逻辑等价与德摩根逻辑定律。该方法表明:等式两侧集合的元素性质在逻辑上完全等价,从而同时完成双重包含证明的两个部分。

\begin{proof}
    首先证明第一个等式。任取 $x \in U$,则有:
    \begin{align*}
        x \in \overline{A \cup B} &\iff x \notin A \cup B &\quad \text{补集的定义}\\
        &\iff \neg(x \in A \cup B) &\quad \notin \text{\ 的定义}\\
        &\iff \neg[(x \in A) \lor (x \in B)] &\quad \cup \text{\ 和\ } \lor \text{\ 的定义}\\
        &\iff \neg(x \in A) \land \neg(x \in B) &\quad \text{德摩根逻辑定律}\\
        &\iff (x \notin A) \land (x \notin B) &\quad \notin \text{\ 的定义}\\
        &\iff x \in \overline{A} \land x \in \overline{B} &\quad \text{补集的定义}\\
        &\iff x \in \overline{A} \cap \overline{B} &\quad \land \text{\ 和\ } \cap \text{\ 的定义}
    \end{align*}
    请注意:此处``$\land$''为逻辑运算符,而``$\cap$''为集合运算符。证明核心在于应用德摩根定律,将析取的否定转化为否定的合取。

    上述等价链表明:
    \[x \in \overline{A \cup B} \iff x \in \overline{A} \cap \overline{B}\]
    因此,在全集 $U$ 上,$\overline{A \cup B}$ 与 $\overline{A} \cap \overline{B}$ 的元素性质逻辑等价,因此:
    \[\overline{A \cup B} = \overline{A} \cap \overline{B}\]

    用类似的方法证明第二个等式。任取 $x \in U$,则有:
    \begin{align*}
        x \in \overline{A \cap B} &\iff x \notin A \cap B &\quad \text{补集的定义}\\
        &\iff \neg(x \in A \cap B) &\quad \notin \text{\ 的定义}\\
        &\iff \neg[(x \in A) \land (x \in B)] &\quad \cap \text{\ 和\ } \land \text {\ 的定义}\\
        &\iff \neg(x \in A) \lor \neg(x \in B) &\quad \text{德摩根逻辑定律}\\
        &\iff (x \notin A) \lor (x \notin B) &\quad \notin \text{\ 的定义}\\
        &\iff x \in \overline{A} \lor x \in \overline{B} &\quad \text{补集的定义}\\
        &\iff x \in \overline{A} \cup \overline{B} &\quad \lor \text{\ 和\ } \cup \text{\ 的定义}
    \end{align*}

    上述等价链表明:
    \[x \in \overline{A \cap B} \iff x \in \overline{A} \cup \overline{B}\]
    因此,在全集 $U$ 上,$\overline{A \cap B}$ 与 $\overline{A} \cup \overline{B}$ 的元素性质逻辑等价,因此:
    \[\overline{A \cap B} = \overline{A} \cup \overline{B}\]

    综上,我们证明了定理中所述的两个等式。
\end{proof}


请注意,两个证明具有显著的对称性:方法完全一致,唯一的区别是互换``$\cap$''与``$\cup$''的角色。德摩根逻辑定律为此转换提供了理论基础,使证明简洁而优美。相较于双重包含论证法,此方法更为简明高效(读者可自行尝试验证)。

\clearpage


% !TeX root = ../../../book.tex
\subsection{通过条件陈述证明集合包含}

只要有可能,就多使用我们在上一节中使用的方法,以及德摩根逻辑和集合定律;也就是说,可以随意通过条件陈述和逻辑等价证明集合关系。一般来说,当你证明相等时,你需要确保你的所有主张确实都是 ``$\iff$'' 主张。在上一节中,我们只应用了关于逻辑等价的定义和定理,因此我们肯定证明中 ``$\iff$'' 箭头的所有方向都是成立的。每当你写出这样的证明时,完成后回过头再读一遍,并在每一行问自己:``这真的正确吗?这里的含义是双向的吗?''

让我们看看该技术的另一个实际例子。它会稍微复杂一些,因为给出的主张与德摩根逻辑定律本质上并不相同,因此我们必须定义一些变量命题。不过,我们会引用我们刚刚证明的逻辑定律,并用它来建立集合定律。

\begin{proposition}
    设 $A, B, C$ 为任意集合,且 $A, B, C \subseteq U$,其中 $U$ 为全集。则,
    \[A \cap (B - C) = (A \cap B) - C\]
\end{proposition}

与前面的例子(德摩根集合定律)非常相似,我们将在左侧元素和右侧元素之间建立逻辑等价关系。(同样地,这就像同时证明双重包含证明的两面。)为此,我们只需建立一些变量命题,分别指代 $A, B, C$ 元素的属性。接下来,结果将遵循逻辑法则。

\begin{proof}
    设 $A, B, C$ 为任意集合,且 $A, B, C \subseteq U$,其中 $U$ 为全集。我们定义以下变量命题:
    \begin{center}
        设 $P(x)$ 为 ``$x \in A$'' \\
        设 $Q(x)$ 为 ``$x \in B$'' \\
        设 $R(x)$ 为 ``$x \in C$'' \\
    \end{center}
    设 $x \in U$ 是任意固定元素。有了这些定义,我们可以编写以下逻辑等价链:
    \begin{align*}
        x \in A \cap (B - C) &\iff x \in A \land (x \in B - C) &\quad \cap \text{的定义} \\
        &\iff x \in A \land (x \in B \land x \notin C) &\quad - \text{的定义}\\
        &\iff P(x) \land (Q(x) \land \neg R(x)) &\quad P(x), Q(x), R(x), \notin \text{的定义} \\
        &\iff (P(x) \land Q(x)) \land \neg R(x) &\quad \land \text{分配律} \\
        &\iff (x \in A \land x \in B) \land x \notin C &\quad P(x), Q(x), R(x) \text{的定义}\\
        &\iff (x \in A \cap B \land x \notin C) &\quad \cap \text{的定义} \\
        &\iff x \in (A \cap B) - C &\quad - \text{的定义}
    \end{align*}
    这表明
    \[x \in A \cap (B - C) \iff x \in (A \cap B) - C\]
    对于全集 $U$ 中的任何元素 $x$ 都成立。因此,
    \[A \cap (B - C) = (A \cap B) - C\]
\end{proof}

想想为什么我们需要确保所有这些声明都是正确的\emph{当且仅当}陈述。我们确保等式一侧集合中元素的任意元素 $x$ 也必然是另一侧集合的元素;但是,此外,我们还确保任何不是集合元素的元素 $x$ 也不是另一个集合的元素。双向条件陈述``两个方向都成立'',因此我们同时证明了主张的``是……的元素''和``不是……的元素''。

为了说明我们之前的警告,请考虑以下声明的示例证明,其中 $\iff$ 声明在一个``方向''上不成立。

\begin{proposition}\label{prop:proposition4.6.11}
    设 $X, Y, Z$ 为任意集合,且 $X, Y, Z \subseteq U$,其中 $U$ 为全集。则,以下包含关系成立:
    \[(X \cup Y ) - Z \subseteq X \cup (Y - Z)\]
\end{proposition}

你可能发现了这个命题就是习题 \ref{exc:exercises3.11.17}!在练习中,我们要求你使用包含论证来证明这个主张,取任意 $x \in U$ 并假设它是左侧集合的元素,然后推论它一定也是右侧集合的元素。我们将在这里做(本质上)相同的事情,但论证将用逻辑术语和符号重新表达。我们这样做是为了
\begin{enumerate}
    \item 让我们更多地练习这种类型的论证;
    \item 准确地识别论证中``反''方向不成立的地方。 
\end{enumerate}
请记住,在习题 \ref{exc:exercises3.11.17} 中,我们还要求你找到一个示例来表明 $\supseteq$ 方向\emph{不一定}为\verb|真|。这意味着朝这个方向进行的逻辑论证会在某个地方失败。我们将准确地看到它在哪里失败,并且我们可以用它来帮助我们提出所需的反例。

\begin{proof}
    设 $X,Y,Z$ 为任何集合,且 $X,Y,Z \subseteq U$,其中 $U$ 为全集。设 $x \in U$ 是任意固定元素。我们可以写出以下逻辑等价链:
    \begin{align*}
        x \in (X \cup Y ) - Z &\iff x \in X \cup Y \land x \notin Z &\quad - \text{的定义}\\
        &\iff (x \in X \lor x \in Y ) \land x \notin Z &\quad \cup \text{的定义} \\
        &\iff (x \in X \land x \notin Z) \lor (x \in Y \land x \notin Z) &\quad \text{德摩根逻辑定律} 
    \end{align*}
    
    \setlength{\fboxrule}{2pt}
    \setlength\fboxsep{5mm}
    \begin{center}
    \fcolorbox{red}{white}{%
        \parbox{0.85\textwidth}{%
            \textcolor{red}{\textbf{草稿:}}

            从这里开始,我们可以进一步断言哪些逻辑等价?我们可以化简右侧并将其表示为
            \[x \in X - Z \lor x \in X - Z\]
            因此
            \[x \in (X - Z) \cup (Y - Z)\]
            这不是我们要证明的原始命题,但到目前为止,这个过程有效证明了另一个命题,即
            \[(X \cup Y ) - Z = (X - Z) \cup (Y - Z)\]
            然而,我们要证明命题右边是
            \[X \cup (Y - Z)\]
            但我们并不是要证明相等,而是证明\emph{包含关系}。因此,我们其余证明的目标是证明下面条件声明:
            \[\big((x \in X \land x \notin Z) \lor (x \in Y \land x \notin Z)\big) \implies x \in X \cup (Y - Z)\]
            为了帮助我们弄清楚如何得到上面的公式,让我们在这里做一些临时工作,重写右侧的陈述;然后,我们就可以看到为什么上面公式成立:
            \begin{align*}
                x \in X \cup (Y - Z) &\iff x \in X \lor x \in Y - Z &\quad \cup \text{的定义} \\
                &\iff x \in X \lor (x \in Y \land x \notin Z) &\quad - \text{的定义}
            \end{align*}
            这与我们上面推导的最后一个逻辑等价类似,但这里与左边的项不同。你能看出上面的蕴涵关系吗?想一想,然后继续阅读剩下的证明。
        }
    }
    \end{center}
    现在,我们想证明
    \[\big((x \in X \land x \notin Z) \lor (x \in Y \land x \notin Z)\big) \implies x \in X \cup (Y - Z)\]
    为此,我们假设左侧表达式为\verb|真|。这意味着
    \[x \in X \land x \notin Z\]
    或
    \[x \in Y \land x \notin Z\]
    (或者两者都为\verb|真|)。因此,我们有两种情况:
    \begin{enumerate}
        \item 假设第一个表达式\verb|真|,则 $x \in X \land x \notin Z$。这当然意味着 $x \in X$,因此 $x \in X \lor x \in Y - Z$ 成立。
        \item 假设第二个表达式\verb|真|,则 $x \in Y \land x \notin Z$。这意味着 $x \in Y - Z$,因此 $x \in X \lor x \in Y - Z$ 成立。
    \end{enumerate}
    无论哪种情况,我们都会有 $x \in X \lor x \in Y - Z$ 成立,因此,无论哪种情况,根据 $\cup$ 的定义
    \[x \in X \cup (Y - Z)\]
    都成立。

    综上,这表明对于每个元素 $x \in U$ 
    \[x \in (X \cup Y ) - Z \implies x \in X \cup (Y - Z)\]
    都成立。因此,根据 $\subseteq$ 的定义,我们有
    \[(X \cup Y ) - Z \subseteq X \cup (Y - Z)\]
\end{proof}

认清我们在哪里以及我们想去哪里,帮助我们完成了这个证明。我们没有希望仅使用逻辑等价来完成证明,因为事实上,声明中给出的集合并不总是相等!回顾证明,我们能否识别出逻辑等价无效的步骤,并且我们能否使用它来构建反驳这些集合始终相等这一(错误)主张的反例?

我们已经得到下面这个有效陈述
\[(x \in X \land x \notin Z) \lor (x \in Y \land x \notin Z)\]
并且我们用它推导出下面陈述
\[x \in X \lor (x \in Y \land x \notin Z)\]
从我们证明采用的论证来看,很明显,第一个陈述确实蕴涵了第二个陈述;也就是说,如果我们假设第一个陈述成立,我们可以得出第二个陈述也成立。它们之间唯一的区别在于第一项,并且知道 ``$\land$'' 陈述的两个部分都成立肯定可以让我们得出其中特定的一个成立的结论。

这种推论在另一个方向上不起作用。假设第二个陈述成立。如果正确的项是有效的 --- $x \in Y \land x \notin Z$ --- 那么这也使得第一个陈述成立。然而,由于我们有一个 ``$\lor$'' 陈述,我们必须考虑左边项成立的情况。在这种情况下,仅知道 $x \in X$ 并不能让我们推出 $x \in X \land x \notin Z$ 成立。假设 ``$\land$'' 成立,我们可以推断出其任一部分都成立,但仅知道其中一部分成立并不能告诉我们两者都成立!

我们可以用它来构造一个反例。我们看到,只需保证有某个特定元素 $x \in U$ 满足第二个陈述的左项,即 $x \in X$,但不满足第一个陈述的左项,即 $x \in X \land x \notin Z$。换句话说,我们只需保证有一个元素 $x \in X \cap Z$ 即可。下面的示例恰恰实现了这一点。\\

\begin{example}
    我们声称
    \[(X \cup Y ) - Z \subseteq X \cup (Y - Z)\]
    对于任意集合 $X,Y,Z$ 都成立,但不一定相等。请参阅命题 \ref{prop:proposition4.6.11} 的证明,了解为什么上述包含关系确实成立。
\end{example}

现在,考虑以下示例。我们定义
\begin{align*}
    X &= \{1\} \\
    Y &= \{2\} \\
    Z &= \{1, 2\}
\end{align*}
注意到
\[(X \cup Y ) - Z = (\{1\} \cup \{2\}) - \{1, 2\} = \{1, 2\} - \{1, 2\} = \varnothing\]
且
\[X \cup (Y - Z) = \{1\} \cup (\{2\} - \{1, 2\}) = \{1\} \cup \{\varnothing\} = \{1\}\]
由于 $\varnothing \subset \{1\}$ (真)子集,在这种情况下,我们得出结论:
\[(X \cup Y ) - Z \ne X \cup (Y - Z)\]
这表明上述命题中的相等不一定成立。

这个策略让我们能够以更高效、更严格的方式完成许多涉及集合的证明!我们可以使用我们已经\emph{证明}的逻辑符号和定律,而不是凭直觉使用``与''和``或''的语言学定义。正式出于这个原因,本节中的许多练习都涉及集合。建议你需要回看第 \ref{ch:chapter03} 章以记起相关定义!


% !TeX root = ../../../book.tex
\subsection{习题}

\subsubsection*{温故知新}

以口头或书面的形式简要回答以下问题。这些问题全都基于你刚刚阅读的内容,如果忘记了具体定义、概念或示例,可以回顾相关内容。确保在继续学习之前能够自信地作答这些问题,这将有助于你的理解和记忆!

\begin{enumerate}[label=(\arabic*)]
    \item 什么是逻辑结合律?
    \item 什么是逻辑分配律?
    \item 什么是德摩根逻辑定律?什么是德摩根集合定律?它们之间有什么关系?
    \item 充分条件和必要条件有什么区别?
    \item 当一个条件既是充分条件又是必要条件时,会发生什么?
\end{enumerate}

\subsubsection*{小试牛刀}

尝试解答以下问题。这些题目需动笔书写或口头阐述答案,旨在帮助你熟练运用新概念、定义及符号。题目难度适中,确保掌握它们将大有裨益!

\begin{enumerate}[label=(\arabic*)]
    \item 在前文中,我们使用真值表证明了德摩根逻辑定律。你能给出德摩根逻辑定律的语义证明吗?你能向非数学背景的朋友解释德摩根定律,并使他们相信这些定律成立吗?
    \item 设 $P(x)$ 为变量命题``$x$ 是能被 $10$ 整除的整数''。请为该陈述提供两个必要条件和两个充分条件。
    \item 设 $A,B,C$ 为全集 $U$ 的任意子集。使用逻辑等价和逻辑定律证明以下等式:
        \begin{enumerate}[label=(\alph*)]
            \item $A \cap (B \cup C) = (A \cap B) \cup (A \cap C)$
            \item $(A \cup B) \cap \overline{A} = B - A$
            \item $\overline{\overline{A} \cup B} = A \cap \overline{B}$
            \item $(A - B) \cap \overline{C} = A - (B \cup C)$
        \end{enumerate}
    \item 使用条件陈述和逻辑等价证明包含关系
        \[A - (B \cup C) \subseteq A \cap \overline{B \cap C}\]
        对于任意集合 $A,B,C$ 成立。\\
        然后,找到一个反例证明相等关系不一定成立。\\
        (\textbf{提示:}一般来说,构造严格包含关系的一个有效方法是考虑一侧为空集的情形。)
    \item 设 $D,E,F$ 为任意集合。考虑集合
        \[D - (E - F) \quad \text{和} \quad (D - E) - F\]
        这两个集合之间的关系如何?它们总是相等吗?还是一个总是另一个的子集?\\
        请清晰地陈述你的结论,然后给出证明或提供反例。
\end{enumerate}