% !TeX root = ../../../book.tex
\subsection{证明逻辑等价:分配律}

在算术中,你一定知道乘法分配律。也就是说,我们知道:
\[\forall x, y, z \in \mathbb{R} \centerdot x \cdot (y + z) = x \cdot y + x \cdot z\]
这个符号表示法精确表达了乘法分配律!你能看出它与你熟知的规则一致吗?

接下来,我们将研究并证明两个类似的定律。这些定律表明逻辑连接词``$\land$''和``$\lor$''也满足分配律。

\begin{theorem}
    设 $P, Q, R$ 为逻辑陈述。则
    \[P \land (Q \lor R) \iff (P \land Q) \lor (P \land R)\]
    \[P \lor (Q \land R) \iff (P \lor Q) \land (P \lor R)\]
\end{theorem}

\begin{proof}
    我们使用真值表来验证这两个命题。对于第一个命题,考虑以下真值表:
    \begin{center}
        \begin{tabular}{c|c|c|c|c|c|c|c}
              $P$ & $Q$ & $R$ & $Q \lor R$ & $P \land Q$  & $P \land R$ & $P \land (Q \lor R)$ & $(P \land Q) \lor (P \land R)$ \\
              \hline
              \verb|T| & \verb|T| & \verb|T| &  \verb|T|  &   \verb|T|   &\verb|T| &\verb|T| &   \verb|T|   \\
              \verb|T| & \verb|T| & \verb|F| &  \verb|T|  &   \verb|T|   &\verb|F| &\verb|T| &   \verb|T|   \\
              \verb|T| & \verb|F| & \verb|T| &  \verb|T|  &   \verb|F|   &\verb|T| &\verb|T| &   \verb|T|   \\
              \verb|T| & \verb|F| & \verb|F| &  \verb|F|  &   \verb|F|   &\verb|F| &\verb|F| &   \verb|F|   \\
              \verb|F| & \verb|T| & \verb|T| &  \verb|T|  &   \verb|F|   &\verb|F| &\verb|F| &   \verb|F|   \\
              \verb|F| & \verb|T| & \verb|F| &  \verb|T|  &   \verb|F|   &\verb|F| &\verb|F| &   \verb|F|   \\
              \verb|F| & \verb|F| & \verb|T| &  \verb|T|  &   \verb|F|   &\verb|F| &\verb|F| &   \verb|F|   \\
              \verb|F| & \verb|F| & \verb|F| &  \verb|F|  &   \verb|F|   &\verb|F| &\verb|F| &   \verb|F|   \\
        \end{tabular}
    \end{center}

    注意,最后两列的真值完全相同,这证明了所需的逻辑等价。\\

    对于第二个命题,参考以下真值表:
    \begin{center}
        \begin{tabular}{c|c|c|c|c|c|c|c}
              $P$ & $Q$ & $R$ & $Q \land R$ & $P \lor Q$  & $P \lor R$ & $P \lor (Q \land R)$ & $(P \lor Q) \land (P \lor R)$ \\
              \hline
              \verb|T| & \verb|T| & \verb|T| &  \verb|T|  &   \verb|T|   &\verb|T| &\verb|T| &   \verb|T|   \\
              \verb|T| & \verb|T| & \verb|F| &  \verb|F|  &   \verb|T|   &\verb|T| &\verb|T| &   \verb|T|   \\
              \verb|T| & \verb|F| & \verb|T| &  \verb|F|  &   \verb|T|   &\verb|T| &\verb|T| &   \verb|T|   \\
              \verb|T| & \verb|F| & \verb|F| &  \verb|F|  &   \verb|T|   &\verb|T| &\verb|T| &   \verb|T|   \\
              \verb|F| & \verb|T| & \verb|T| &  \verb|T|  &   \verb|T|   &\verb|T| &\verb|T| &   \verb|T|   \\
              \verb|F| & \verb|T| & \verb|F| &  \verb|F|  &   \verb|T|   &\verb|F| &\verb|F| &   \verb|F|   \\
              \verb|F| & \verb|F| & \verb|T| &  \verb|F|  &   \verb|F|   &\verb|T| &\verb|F| &   \verb|F|   \\
              \verb|F| & \verb|F| & \verb|F| &  \verb|F|  &   \verb|F|   &\verb|F| &\verb|F| &   \verb|F|   \\
        \end{tabular}
    \end{center}

    同理,最后两列的真值完全相同,从而证明了所需的逻辑等价。
\end{proof}
