% !TeX root = ../../../book.tex
\subsection{证明逻辑等价:分配律}

算术中,你一定知道乘法分配率。也就是说,我们知道
\[\forall x, y, z \in \mathbb{R} \centerdot x \cdot (y + z) = x \cdot y + x \cdot z\]
我们用符号表示法写了出来!你明白为什么它表达了你所知的乘法分配律规则吗?

在这里我们将研究并证明两个类似的定律。他们会告诉我们逻辑连词 ``$\land$'' 和 ``$\lor$'' 也满足分配。

\begin{theorem}
    设 $P, Q, R$ 为逻辑陈述。则
    \[P \land (Q \lor R) \iff (P \land Q) \lor (P \land R)\]
    且
    \[P \lor (Q \land R) \iff (P \lor Q) \land (P \lor R)\]
\end{theorem}

\begin{proof}
    我们使用真值表来验证这两个命题。对于第一个命题,请看下面的真值表:
    \begin{center}
        \begin{tabular}{c|c|c|c|c|c|c|c}
              $P$ & $Q$ & $R$ & $Q \lor R$ & $P \land Q$  & $P \land R$ & $P \land (Q \lor R)$ & $(P \land Q) \lor (P \land R)$ \\
              \hline
              \verb|T| & \verb|T| & \verb|T| &  \verb|T|  &   \verb|T|   &\verb|T| &\verb|T| &   \verb|T|   \\
              \verb|T| & \verb|T| & \verb|F| &  \verb|T|  &   \verb|T|   &\verb|F| &\verb|T| &   \verb|T|   \\
              \verb|T| & \verb|F| & \verb|T| &  \verb|T|  &   \verb|F|   &\verb|T| &\verb|T| &   \verb|T|   \\
              \verb|T| & \verb|F| & \verb|F| &  \verb|F|  &   \verb|F|   &\verb|F| &\verb|F| &   \verb|F|   \\
              \verb|F| & \verb|T| & \verb|T| &  \verb|T|  &   \verb|F|   &\verb|F| &\verb|F| &   \verb|F|   \\
              \verb|F| & \verb|T| & \verb|F| &  \verb|T|  &   \verb|F|   &\verb|F| &\verb|F| &   \verb|F|   \\
              \verb|F| & \verb|F| & \verb|T| &  \verb|T|  &   \verb|F|   &\verb|F| &\verb|F| &   \verb|F|   \\
              \verb|F| & \verb|F| & \verb|F| &  \verb|F|  &   \verb|F|   &\verb|F| &\verb|F| &   \verb|F|   \\
        \end{tabular}
    \end{center}

    请注意,最后两列的真值是相同的,从而证明了要求的逻辑等价。\\

    对于第二个命题,请看下面的真值表:
    \begin{center}
        \begin{tabular}{c|c|c|c|c|c|c|c}
              $P$ & $Q$ & $R$ & $Q \land R$ & $P \lor Q$  & $P \lor R$ & $P \lor (Q \land R)$ & $(P \lor Q) \land (P \lor R)$ \\
              \hline
              \verb|T| & \verb|T| & \verb|T| &  \verb|T|  &   \verb|T|   &\verb|T| &\verb|T| &   \verb|T|   \\
              \verb|T| & \verb|T| & \verb|F| &  \verb|F|  &   \verb|T|   &\verb|T| &\verb|T| &   \verb|T|   \\
              \verb|T| & \verb|F| & \verb|T| &  \verb|F|  &   \verb|T|   &\verb|T| &\verb|T| &   \verb|T|   \\
              \verb|T| & \verb|F| & \verb|F| &  \verb|F|  &   \verb|T|   &\verb|T| &\verb|T| &   \verb|T|   \\
              \verb|F| & \verb|T| & \verb|T| &  \verb|T|  &   \verb|T|   &\verb|T| &\verb|T| &   \verb|T|   \\
              \verb|F| & \verb|T| & \verb|F| &  \verb|F|  &   \verb|T|   &\verb|F| &\verb|F| &   \verb|F|   \\
              \verb|F| & \verb|F| & \verb|T| &  \verb|F|  &   \verb|F|   &\verb|T| &\verb|F| &   \verb|F|   \\
              \verb|F| & \verb|F| & \verb|F| &  \verb|F|  &   \verb|F|   &\verb|F| &\verb|F| &   \verb|F|   \\
        \end{tabular}
    \end{center}

    同理,最后两列的真值是相同的,从而证明了要求的逻辑等价。
\end{proof}
