% !TeX root = ../../../book.tex
\subsection{证明逻辑等价:德摩根定律(逻辑)}\label{sec:section4.6.5}

接下来我们将证明一些涉及否定的逻辑等价。以下两条定律以英国数学家\textbf{奥古斯塔斯·德·摩根 (Augustus De Morgan)}的名字命名。他因建立这些逻辑定律并引入\textbf{数学归纳法}这一术语而广受赞誉。我们深深感谢他在数学领域的卓越贡献。

德·摩根逻辑定律描述了否定运算与合取、析取之间的逻辑等价关系。

\begin{theorem}
    设 $P$ 和 $Q$ 为逻辑陈述。则
    \[\neg (P \land Q) \iff \neg P \lor \neg Q\]
    \[\neg (P \lor Q) \iff \neg P \land \neg Q\]
\end{theorem}

\begin{proof}
    我们通过真值表证明第一个命题:
    \begin{center}
        \begin{tabular}{c|c|c|c|c|c|c}
              $P$      & $\neg P$ &   $Q$   &  $\neg Q$  & $P \land Q$ & $\neg (P \land Q)$ & $\neg P \lor \neg Q$ \\
              \hline
              \verb|T| & \verb|F| & \verb|T| &  \verb|F|  &    \verb|T|    &    \verb|F|   & \verb|F| \\
              \verb|T| & \verb|F| & \verb|F| &  \verb|T|  &    \verb|F|    &    \verb|T|   & \verb|T| \\
              \verb|F| & \verb|T| & \verb|T| &  \verb|F|  &    \verb|F|    &    \verb|T|   & \verb|T| \\
              \verb|F| & \verb|T| & \verb|F| &  \verb|T|  &    \verb|F|    &    \verb|T|   & \verb|T| \\
        \end{tabular}
    \end{center}
    
    同理用真值表证明第二个命题:
    \begin{center}
        \begin{tabular}{c|c|c|c|c|c|c}
              $P$      & $\neg P$ &   $Q$    &  $\neg Q$  & $P \lor Q$ & $\neg (P \lor Q)$ & $\neg P \land \neg Q$ \\
              \hline
              \verb|T| & \verb|F| & \verb|T| &  \verb|F|  &    \verb|T|    &    \verb|F|   & \verb|F| \\
              \verb|T| & \verb|F| & \verb|F| &  \verb|T|  &    \verb|T|    &    \verb|F|   & \verb|F| \\
              \verb|F| & \verb|T| & \verb|T| &  \verb|F|  &    \verb|T|    &    \verb|F|   & \verb|F| \\
              \verb|F| & \verb|T| & \verb|F| &  \verb|T|  &    \verb|F|    &    \verb|T|   & \verb|T| \\
        \end{tabular}
    \end{center}
\end{proof}

这两条定律非常有用!事实上,我们可以借助它们证明集合论中的类似结论。
