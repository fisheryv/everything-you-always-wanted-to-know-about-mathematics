% !TeX root = ../../../book.tex
\subsection{习题}

\subsubsection*{温故知新}

以口头或书面的形式简要回答以下问题。这些问题全都基于你刚刚阅读的内容,如果忘记了具体定义、概念或示例,可以回顾相关内容。确保在继续学习之前能够自信地作答这些问题,这将有助于你的理解和记忆!

\begin{enumerate}[label=(\arabic*)]
    \item 什么是逻辑结合律?
    \item 什么是逻辑分配律?
    \item 什么是德摩根逻辑定律?什么是德摩根集合定律?它们之间有什么关系?
    \item 充分条件和必要条件有什么区别?
    \item 当一个条件既是充分条件又是必要条件时,会发生什么?
\end{enumerate}

\subsubsection*{小试牛刀}

尝试解答以下问题。这些题目需动笔书写或口头阐述答案,旨在帮助你熟练运用新概念、定义及符号。题目难度适中,确保掌握它们将大有裨益!

\begin{enumerate}[label=(\arabic*)]
    \item 在前文中,我们使用真值表证明了德摩根逻辑定律。你能给出德摩根逻辑定律的语义证明吗?你能向非数学背景的朋友解释德摩根定律,并使他们相信这些定律成立吗?
    \item 设 $P(x)$ 为变量命题``$x$ 是能被 $10$ 整除的整数''。请为该陈述提供两个必要条件和两个充分条件。
    \item 设 $A,B,C$ 为全集 $U$ 的任意子集。使用逻辑等价和逻辑定律证明以下等式:
        \begin{enumerate}[label=(\alph*)]
            \item $A \cap (B \cup C) = (A \cap B) \cup (A \cap C)$
            \item $(A \cup B) \cap \overline{A} = B - A$
            \item $\overline{\overline{A} \cup B} = A \cap \overline{B}$
            \item $(A - B) \cap \overline{C} = A - (B \cup C)$
        \end{enumerate}
    \item 使用条件陈述和逻辑等价证明包含关系
        \[A - (B \cup C) \subseteq A \cap \overline{B \cap C}\]
        对于任意集合 $A,B,C$ 成立。\\
        然后,找到一个反例证明相等关系不一定成立。\\
        (\textbf{提示:}一般来说,构造严格包含关系的一个有效方法是考虑一侧为空集的情形。)
    \item 设 $D,E,F$ 为任意集合。考虑集合
        \[D - (E - F) \quad \text{和} \quad (D - E) - F\]
        这两个集合之间的关系如何?它们总是相等吗?还是一个总是另一个的子集?\\
        请清晰地陈述你的结论,然后给出证明或提供反例。
\end{enumerate}