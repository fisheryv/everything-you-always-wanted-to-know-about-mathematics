% !TeX root = ../../../book.tex


\subsection{习题}

\subsubsection*{温故知新}

以口头或书面的形式简要回答以下问题。这些问题全都基于你刚刚阅读的内容,所以如果忘记了具体的定义、概念或示例,可以回去重读相关部分。确保在继续学习之前能够自信地回答这些问题,这将有助于你的理解和记忆!

\begin{enumerate}[label=(\arabic*)]
    \item 什么是逻辑结合律?
    \item 什么是逻辑分配律?
    \item 什么是德摩根逻辑定律?什么是德摩根集合定律?它们之间有什么关系?
    \item 充分条件和必要条件有什么区别?
    \item 当一个条件既是充分条件又是必要条件会发生什么?
\end{enumerate}

\subsubsection*{小试牛刀}

尝试回答以下问题。这些题目要求你实际动笔写下答案,或(对朋友/同学)口头陈述答案。目的是帮助你练习使用新的概念、定义和符号。题目都比较简单,确保能够解决这些问题将对你大有帮助!

\begin{enumerate}[label=(\arabic*)]
    \item 上文中,我们使用真值表来证明德摩根逻辑定律。你能给出德摩根逻辑定律的语义证明吗?你能向非数学家朋友解释德摩根定律并让他们相信这些定律是有效的吗?
    \item 设 $P(x)$ 为变量命题 ``$x$ 是能被 $10$ 整除的整数''。为这个陈述提出两个必要条件和两个充分条件。
    \item 设 $A,B,C$ 为任意集合,且 $A,B,C \subseteq U$,其中 $U$ 为全集。使用逻辑等价和逻辑定律来证明以下主张。
        \begin{enumerate}[label=(\alph*)]
            \item $A \cap (B \cup C) = (A \cap B) \cup (A \cap C)$
            \item $(A \cup B) \cap \overline{A} = B - A$
            \item $\overline{\overline{A} \cup B} = A \cap \overline{B}$
            \item $(A - B) \cap \overline{C} = A - (B \cup C)$
        \end{enumerate}
    \item 使用条件陈述和逻辑等价证明包含关系
        \[A - (B \cup C) \subseteq A \cap \overline{B \cap C}\]
        对于任意集合 $A,B,C$ 成立。\\
        然后,找到一个反例证明相等关系不一定成立。\\
        (\textbf{提示:}一般来说,构建集合严格包含的一个有用的思路是,看是否可以将一侧设为空集。)
    \item 设 $D,E,F$ 为任意集合。考虑集合
        \[D - (E - F)\]
        和
        \[(D - E) - F\]
        上面两个集合的关系如何?它们总是相等吗?还是一个总是另一个的子集?\\
        清楚地陈述你的主张,然后证明它们或提供相关的反例。
\end{enumerate}