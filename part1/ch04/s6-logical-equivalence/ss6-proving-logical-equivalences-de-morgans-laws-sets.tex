% !TeX root = ../../../book.tex
\subsection{使用逻辑等价:德摩根定律(集合)}\label{sec:section4.6.6}

以下陈述 ``看起来很像'' 我们上面看到的德摩根逻辑定律中的陈述。当我们看过证明后,就会明白为什么它们看起来如此相似!

\begin{theorem}\label{theorem4.6.9}
    设 $A, B$ 为任意集合,并假设 $A, B \subseteq U$,因此补集运算是在全集 $U$ 上定义的。那么,
    \[\overline{A \cup B} = \overline{A} \cap \overline{B}\]
    且
    \[\overline{A \cap B} = \overline{A} \cup \overline{B}\]
\end{theorem}

我们将使用逻辑等价和德摩根逻辑定律来证明这一定理。我们的方法将证明,在任何一种情况下,等式左侧集合元素的属性在逻辑上等价于右侧集合元素的属性。这同时证明了双重包含论证法的两个部分。

\begin{proof}
    我们先来证明第一个集合相等关系。设 $x \in U$ 是任意固定元素。则,
    \begin{align*}
        x \in \overline{A \cup B} &\iff x \notin A \cup B &\quad \text{补集的定义}\\
        &\iff \neg(x \in A \cup B) &\quad \notin \text{的定义}\\
        &\iff \neg[(x \in A) \lor (x \in B)] &\quad \cup \;\text{和} \lor \text{的定义}\\
        &\iff \neg(x \in A) \land \neg(x \in B) &\quad \text{德摩根逻辑定律}\\
        &\iff (x \notin A) \land (x \notin B) &\quad \notin \text{的定义}\\
        &\iff x \in \overline{A} \land x \in \overline{B} &\quad \text{补集的定义}\\
        &\iff x \in \overline{A} \cap \overline{B} &\quad \land \;\text{和} \cap \text{的定义}
    \end{align*}
    请记住,``$\land$'' 是逻辑运算,而 ``$\cap$'' 是集合运算。我们必须小心我们写的每句话中用哪个运算才有意义。另外,请注意,我们在证明中间使用了德摩根逻辑定律,将析取的否定转换为两个否定的合取。

    这个逻辑等价链表明
    \[x \in \overline{A \cup B} \iff x \in \overline{A} \cap \overline{B}\]
    因此,在全集 $U$ 上,$\overline{A \cup B}$ 中元素的属性在逻辑上等价于 $\overline{A} \cap \overline{B}$ 中元素的属性。因此,
    \[\overline{A \cup B} = \overline{A} \cap \overline{B}\]

    接着我们用类似的方法来证明第二个等式。设 $x \in U$ 是任意固定元素。则,
    \begin{align*}
        x \in \overline{A \cap B} &\iff x \notin A \cap B &\quad \text{补集的定义}\\
        &\iff \neg(x \in A \cap B) &\quad \notin \text{的定义}\\
        &\iff \neg[(x \in A) \land (x \in B)] &\quad \cap \;\text{和} \land \text{的定义}\\
        &\iff \neg(x \in A) \lor \neg(x \in B) &\quad \text{德摩根逻辑定律}\\
        &\iff (x \notin A) \lor (x \notin B) &\quad \notin \text{的定义}\\
        &\iff x \in \overline{A} \lor x \in \overline{B} &\quad \text{补集的定义}\\
        &\iff x \in \overline{A} \cup \overline{B} &\quad \lor \;\text{和} \cup \text{的定义}
    \end{align*}

    这个逻辑等价链表明
    \[x \in \overline{A \cap B} \iff x \in \overline{A} \cup \overline{B}\]
    因此,在全集 $U$ 上,$\overline{A \cap B}$ 中元素的属性在逻辑上等价于 $\overline{A} \cup \overline{B}$ 中元素的属性。因此,
    \[\overline{A \cap B} = \overline{A} \cup \overline{B}\]

    综上,我们证明了定理中所述的两个等式。
\end{proof}

请注意这两个证明之间惊人的相似之处。他们使用完全相同的方法,唯一的区别是将 ``$\cap$'' 翻转为 ``$\cup$'',反之亦然。因为我们已经证明了如何做到这一点(德摩根逻辑定律),所以我们可以引用该结果并使这个证明简短而有趣。你是否同意这比用双重包含论证法证明两个集合相等要容易得多、简洁得多?(尝试一下!)
