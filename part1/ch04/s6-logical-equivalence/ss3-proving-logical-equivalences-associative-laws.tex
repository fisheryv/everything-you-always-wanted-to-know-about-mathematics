% !TeX root = ../../../book.tex
\subsection{证明逻辑等价:结合律}

现在,让我们实际\textbf{证明}一些逻辑等价!在此过程中,我们将提升使用量词和逻辑连接词来阅读、理解及编写逻辑陈述的能力。同时,我们将推导出一些基本逻辑结论,以便在后续发展中应用这些结论来构建证明技术。这些技术将成为我们工作的基石,后续所有工作都将涉及证明策略与逻辑概念的某种组合。

让我们从一些比较简单的符号逻辑定律开始。运算满足\emph{结合律}意味着可以自由调整括号位置而不改变结果。例如加法满足结合律:计算 $x + (y + z)$ 等价于计算 $(x + y) + z$,两者结果相同。正因如此,我们可以安全地省略括号,直接写做:
\[x + y + z\]
因为加法顺序不影响最终结果。类似地,逻辑陈述的合取与析取也满足结合律,这正是我们将要证明的内容。

\begin{theorem}
    设 $P, Q, R$ 为逻辑陈述。则
    \[P \land (Q \land R) \iff (P \land Q) \land R\]
    \[P \lor (Q \lor R) \iff (P \lor Q) \lor R\]
\end{theorem}

我们将通过两种不同方法证明该命题:
\begin{enumerate}[label=(\arabic*)]
    \item 真值表法
    \item 语义法(即自然语言描述)
\end{enumerate}
两种方法均有效,加下来我们将展示两种方法,以供选择偏好的证明风格。

\begin{proofs}{证明 1. }
    首先,通过真值表证明合取结合律:
    \begin{center}
        \begin{tabular}{c|c|c|c|c|c|c}
              $P$    & $Q$   & $R$ & $P \land Q$ &  $Q \land R$  & $P \land (Q \land R)$ & $(P \land Q) \land R$ \\
              \hline
              \verb|T| & \verb|T| & \verb|T| &  \verb|T|  &    \verb|T|    &\verb|T| &    \verb|T|    \\
              \verb|T| & \verb|T| & \verb|F| &  \verb|T|  &    \verb|F|    &\verb|F| &    \verb|F|    \\
              \verb|T| & \verb|F| & \verb|T| &  \verb|F|  &    \verb|F|    &\verb|F| &    \verb|F|    \\
              \verb|T| & \verb|F| & \verb|F| &  \verb|F|  &    \verb|F|    &\verb|F| &    \verb|F|    \\
              \verb|F| & \verb|T| & \verb|T| &  \verb|F|  &    \verb|T|    &\verb|F| &    \verb|F|    \\
              \verb|F| & \verb|T| & \verb|F| &  \verb|F|  &    \verb|F|    &\verb|F| &    \verb|F|    \\
              \verb|F| & \verb|F| & \verb|T| &  \verb|F|  &    \verb|F|    &\verb|F| &    \verb|F|    \\
              \verb|F| & \verb|F| & \verb|F| &  \verb|F|  &    \verb|F|    &\verb|F| &    \verb|F|    \\
        \end{tabular}
    \end{center}

    因为 $P \land (Q \land R)$ 和 $(P \land Q) \land R$ 在所有情况下真值相同,因此 $P \land (Q \land R) \iff (P \land Q) \land R$。\\

    再通过真值表证明析取结合律:
    \begin{center}
        \begin{tabular}{c|c|c|c|c|c|c}
              $P$    & $Q$   & $R$ & $P \lor Q$ &  $Q \lor R$  & $P \lor (Q \lor R)$ & $(P \lor Q) \lor R$ \\
              \hline
              \verb|T| & \verb|T| & \verb|T| &  \verb|T|  &    \verb|T|    &\verb|T| &    \verb|T|    \\
              \verb|T| & \verb|T| & \verb|F| &  \verb|T|  &    \verb|T|    &\verb|T| &    \verb|T|    \\
              \verb|T| & \verb|F| & \verb|T| &  \verb|T|  &    \verb|T|    &\verb|T| &    \verb|T|    \\
              \verb|T| & \verb|F| & \verb|F| &  \verb|T|  &    \verb|F|    &\verb|T| &    \verb|T|    \\
              \verb|F| & \verb|T| & \verb|T| &  \verb|T|  &    \verb|T|    &\verb|T| &    \verb|T|    \\
              \verb|F| & \verb|T| & \verb|F| &  \verb|T|  &    \verb|T|    &\verb|T| &    \verb|T|    \\
              \verb|F| & \verb|F| & \verb|T| &  \verb|F|  &    \verb|T|    &\verb|T| &    \verb|T|    \\
              \verb|F| & \verb|F| & \verb|F| &  \verb|F|  &    \verb|F|    &\verb|F| &    \verb|F|    \\
        \end{tabular}
    \end{center}

    因为 $P \lor (Q \lor R)$ 和 $(P \lor Q) \lor R$ 在所有情况下真值相同,因此 $P \lor (Q \lor R) \iff (P \lor Q) \lor R$。
\end{proofs}

\begin{proofs}{证明 2. }
    接下来,我们通过语义分析证明该命题。\\
    首先考虑合取形式:
    \[P \land (Q \land R) \iff (P \land Q) \land R\]
    为了证明符号两侧表达式\emph{逻辑等价},需要证明以下两个条件陈述均为\verb|真|:
    \[P \land (Q \land R) \implies (P \land Q) \land R \qquad (P \land Q) \land R \implies P \land (Q \land R)\]
    \begin{itemize}
        \item[($\implies$)] 首先,证明第一个条件陈述。假设 $P \land (Q \land R)$ 为\verb|真|,则 $P$ 为\verb|真|且 $Q \land R$ 为\verb|真|。根据定义,这意味着 $P$、$Q$、$R$ 均为\verb|真|。此时 $P \land Q$ 为\verb|真|且 $R$ 为\verb|真|,因此 $(P \land Q) \land R$ 为\verb|真|。
        \item[($\impliedby$)] 接着,证明第二个条件陈述。假设 $(P \land Q) \land R$ 为\verb|真|,则 $P \land Q$ 为\verb|真|且 $R$ 为\verb|真|。根据定义,这意味着 $P$、$Q$、$R$ 均为\verb|真|。此时 $P$ 为\verb|真|且 $Q \land R$ 为\verb|真|,因此 $P \land (Q \land R)$ 为\verb|真|。
    \end{itemize}
    由上述证明可知二者逻辑等价。\\ \\
    再考虑析取形式:
    \[P \lor (Q \lor R) \iff (P \lor Q) \lor R\]
    为了证明符号两侧表达式\emph{逻辑等价},需要证明以下两个条件陈述均为\verb|真|:
    \[P \lor (Q \lor R) \implies (P \lor Q) \lor R \qquad (P \lor Q) \lor R \implies P \lor (Q \lor R)\]
    \begin{itemize}
        \item[($\implies$)] 首先,证明第一个条件陈述。假设 $P \lor (Q \lor R)$ 为\verb|真|。这意味着 $P$ 为\verb|真|或 $Q \lor R$ 为\verb|真|。分两种情况讨论:
        \begin{enumerate}
            \item 假设 $P$ 为\verb|真|。这意味着 $P \lor Q$ 为\verb|真|。因此,根据定义 $(P \lor Q) \lor R$ 也为\verb|真|。
            \item 假设 $Q \lor R$ 为\verb|真|。这意味着 $Q$ 为\verb|真|或 $R$ 为\verb|真|。同理,分两种情况讨论:
            \begin{enumerate}[label=(\alph*)]
                \item 假设 $Q$ 为\verb|真|。这意味着 $P \lor Q$ 为\verb|真|。因此,根据定义 $(P \lor Q) \lor R$ 也为\verb|真|。
                \item 假设 $R$ 为\verb|真|。这意味着 $(P \lor Q) \lor R$ 为\verb|真|。
            \end{enumerate}
        \end{enumerate} 
        综上,无论何种情况,都有 $(P \lor Q) \lor R$ 为\verb|真|。因此,该条件陈述为\verb|真|。

        \item[($\impliedby$)] 接着,证明第二个条件陈述。假设 $(P \lor Q) \lor R$ 为\verb|真|。这意味着 $P \lor Q$ 为\verb|真|或 $R$ 为\verb|真|。分两种情况讨论:
        \begin{enumerate}
            \item 假设 $P \lor Q$ 为\verb|真|。这意味着 $P$ 为\verb|真|或 $Q$ 为\verb|真|。同理,分两种情况讨论:
            \begin{enumerate}[label=(\alph*)]
                \item 假设 $P$ 为\verb|真|。这意味着 $P \lor (Q \lor R)$ 为\verb|真|。
                \item 假设 $Q$ 为\verb|真|。这意味着 $Q \lor R$ 为\verb|真|。因此,根据定义,$P \lor (Q \lor R)$ 为\verb|真|。
            \end{enumerate}
            \item 假设 $R$ 为\verb|真|。这意味着 $Q \lor R$ 为\verb|真|。因此,根据定义,$P \lor (Q \lor R)$ 为\verb|真|。
        \end{enumerate}   
        综上,无论何种情况,都有 $P \lor (Q \lor R)$ 为\verb|真|。因此,该条件陈述为\verb|真|。 
    \end{itemize}
    上述两个条件陈述均成立,因此二者逻辑等价。
\end{proofs}

仔细理解上述证明,我们通过这些证明达成了什么?我们已经证明了什么?如何证明的?为什么这种证明有效?

在继续比较这两种证明方法之前,先明确证明的结论:我们证明了逻辑连词``$\land$''和``$\lor$''满足结合律。因此,在处理仅涉及同一种连词的陈述时,括号的顺序不再重要。例如,``$P \land (Q \land R)$''与``$(P \land Q) \land R$''具有相同的含义。今后我们将省略括号,直接写做``$P \land Q \land R$''。

\subsubsection*{反思:真值表与语义证明}

我们先来谈谈真值表。由于 $P,Q,R$ 是逻辑陈述,其真值非真即假。真值表的八行穷尽了这三个陈述的所有真值组合。前三列标示 $P,Q,R$ 的真假,后续两列对应命题中的复合子部分,最后两列则代表需要验证的两个命题。通过对比最后两列,可判定二者是否逻辑等价(即``无论 $P,Q,R$ 取何真值,两个命题真值始终相同'')。若两列真值逐行一致,即可证明逻辑等价。

接下来讨论语义证明。这种证明可能显得冗长,但它是严谨的吗?证明过程是否清晰?逻辑是否严密?请重读证明并思考这些问题。需要强调的是,该证明完全正确。当证明析取(``或''命题)时,分情况讨论至关重要;而通过假设前提为\verb|真|来推导结论,正是验证条件命题的有效方法。我们将进一步分析这些技巧,希望此处的示例对后续学习有所助益。

本节后续内容将采用真值表验证此类简单命题——这种方法更为简洁!如果你需要更具说服力的证明,或希望通过自然语言解读逻辑命题从而进行额外练习,可以详细研究前述示例的语义证明。
