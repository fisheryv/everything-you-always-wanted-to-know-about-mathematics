% !TeX root = ../../../book.tex
\subsection{必要条件和充分条件}

数学中偶尔会使用两个术语来表达双向条件陈述的两个方向:\textbf{充分条件}和\textbf{必要条件}。它们与双向条件的``当''和``如果''完全对应。这些术语是由数学家所提问题的自然类型带来的。

\subsubsection*{充分条件:$P$ 当 $Q$}

如果我们发现了一个数学对象的一些有趣的事实或属性(称之为 $P$),我们可能会想,``我们什么时候才能\emph{保证}这样的属性成立?我们是否可以检验某些条件,以便立即给出`肯定'的答案?'' 这就是\textbf{充分条件},能够保证 $P$ 成立的属性。它是``充分''的,因为它``足以''得出 $P$ 的结论;我们不需要任何其他额外信息。

假设我们已经将命题 $Q$ 确定为 $P$ 的充分条件。我们如何逻辑地表达这一点呢?好吧,知道 $Q$ 就足以得出 $P$ 的结论,所以我们可以轻松地将其写为条件陈述:
\begin{center}
    $Q \implies P \qquad$ 意味着 $Q$ 是 $P$ 的充分条件
\end{center}
换句话说,这个条件陈述表达的是:``$P$ 当 $Q$''。

\subsubsection*{必要条件:$P$ 仅当 $Q$}

我们也可能想知道,``我们如何保证 $P$ 为假?我们是否可以检验某些条件从而立即得知这一点?''这就是\textbf{必要条件},即属性 $P$ 成立所必需或必要的属性。这个条件不一定足以得出 $P$ 成立的结论,但为了让 $P$ 有可能成立,这个条件最好也成立。

思考一下这里的逻辑联系。假设我们已知属性 $Q$ 是 $P$ 的必要条件。我们如何符号化地表达 $P$ 和 $Q$ 之间的关系?没错,我们可以使用条件陈述。知道 $P$ 成立就告诉我们 $Q$ 肯定成立;$P$ 必然为真。这可以表示为
\begin{center}
    $P \implies Q \qquad$ 意味着 $Q$ 是 $P$ 的必要条件
\end{center}
换句话说,这个条件陈述表达的是:``$P$ 仅当 $Q$''。

我们也可以从逆否的角度来思考这一点。如果 $Q$ 不成立,则 $P$ 也不成立。即,
\[\neg Q \implies \neg P\]
这是上面条件陈述 $P \implies Q$ 的逆否形式。我们知道这是原陈述的逻辑等价形式。

\subsubsection*{示例}

\begin{example}
    令 $P(x)$ 表示命题 ``$x$ 为能被 $6$ 整除的整数''。对于以下每个条件,确定其是否是 $P(x)$ 成立的\textbf{必要}条件或\textbf{充分}条件(或可能两者都是!)。
    \begin{enumerate}[label=(\arabic*)]
        \item 令 $Q(x)$ 为 ``$x$ 为可被 $3$ 整除的整数''。
            \begin{itemize}
                \item 为了确定 $Q(x)$ 是否是必要条件,我们假设 $P(x)$ 成立。我们也能推导出 $Q(x)$ 成立吗?是的!说一个整数 $x$ 能被 $6$ 整除,意味着它能被 $2$ 和 $3$ 整除。因此,它肯定能被 $3$ 整除,所以 $Q(x)$成立。
                \item 为了确定 $Q(x)$ 是否是充分条件,我们假设 $Q(x)$ 成立。我们也能推导出 $P(x)$ 成立吗?知道 $x$ 是一个能被 $3$ 整除的整数,那么它是否也\emph{一定}能被 $2$ 整除,从而得出它能被 $6$ 整除的结论?我们认为不是!举个反例 $x = 3$;注意 $Q(3)$ 成立,但 $P(3)$ 不成立。
            \end{itemize}
            这说明 $Q(x)$ 只是必要条件,不是充分条件。
        \item 令 $R(x)$ 为 ``$x$ 为可被 $12$ 整除的整数''。\\
            按照与上例类似的推理,我们可以得出结论,$R(x)$ 是 $P(x)$ 的充分条件,但不是必要条件(因为我们可以举出反例 $x = 6$,其中 $P(6)$ 成立,但 $R(6)$ 不成立)。
        \item 令 $S(x)$ 为 ``$x$ 为 $x^2$ 可被 $6$ 整除的整数''。\\
            这个问题留给你来完成……$S(x)$ 是 $P(x)$ 的必要条件吗?是充分条件吗?\\
            请注意我们给定 $x$ 本身是一个整数……
    \end{enumerate}
\end{example}
