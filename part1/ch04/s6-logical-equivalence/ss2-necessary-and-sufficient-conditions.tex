% !TeX root = ../../../book.tex
\subsection{充分条件和必要条件}

数学中常常使用两个术语来表达双向条件陈述的两个方向:\textbf{充分条件}和\textbf{必要条件}。它们与双向条件中的``当''和``如果''完全对应。这些术语源于数学家提出的问题类型。

\subsubsection*{充分条件:$P$ 当 $Q$}

当我们发现某个数学对象具有某种有趣的性质(记为 $P$)时,可能会思考:``在什么条件下能\emph{确保}该性质成立?是否存在可验证的条件,可以直接给出肯定的答案?''这就是\textbf{充分条件}——能够保证 $P$ 成立的性质。它之所以``充分'',是因为仅凭此条件就``足以''推断 $P$ 成立,无需其他额外信息。

若已确定命题 $Q$ 是 $P$ 的充分条件,其逻辑表达为:
\begin{center}
    $Q \implies P \qquad$ 表示 $Q$ 是 $P$ 的充分条件
\end{center}
换句话说,该条件陈述等价于:``$P$ 当 $Q$''。

\subsubsection*{必要条件:$P$ 仅当 $Q$}

我们可能还会探究:``如何确保 $P$ 为假?是否存在可检验的条件,能直接否定 $P$?''这就是\textbf{必要条件}——$P$ 成立所必需的性质。此条件虽不足以直接推出 $P$ 成立,但若 $P$ 要成立,该条件必须满足。

假设 $Q$ 是 $P$ 的必要条件,其逻辑关系可表示为:
\begin{center}
    $P \implies Q \qquad$ 表示 $Q$ 是 $P$ 的必要条件
\end{center}
换句话说,该条件陈述等价于:``$P$ 仅当 $Q$''。

我们也可以从逆否的角度理解:若 $Q$ 不成立,则 $P$ 也不成立,即:
\[\neg Q \implies \neg P\]
此式是 $P \implies Q$ 的逆否命题,二者逻辑等价。

\subsubsection*{示例}

\begin{example}
    设 $P(x)$ 表示命题``整数 $x$ 能被 $6$ 整除''。判断下列各条件是否为 $P(x)$ 的\textbf{充分}条件或\textbf{必要}条件(或\textbf{充要条件})。
    \begin{enumerate}[label=(\arabic*)]
        \item 设 $Q(x)$ 为``整数 $x$ 能被 $3$ 整除''。
            \begin{itemize}
                \item \textbf{必要性检验}:假设 $P(x)$ 成立($x$ 能被 $6$ 整除),则 $x$ 势必可以同时被 $2$ 和 $3$ 整除。因此 $Q(x)$ 成立,说明其为必要条件。
                \item \textbf{充分性检验}:假设 $Q(x)$ 成立($x$ 能被 $3$ 整除),能否推出 $P(x)$?取反例 $x=3$,$Q(3)$ 成立但 $P(3)$ 不成立($3$ 不能被 $6$ 整除),说明其不是充分条件。
            \end{itemize}
            综上,$Q(x)$ 只是必要条件,不是充分条件。
        \item 设 $R(x)$ 为``整数 $x$ 能被 $12$ 整除''。
        
            同理可证:$R(x)$ 是充分条件,但非必要条件(反例 $x=6$,$P(6)$ 成立但 $R(6)$ 不成立)。
        \item 设 $S(x)$ 为``整数 $x$ 满足 $x^2$ 能被 $6$ 整除''。
        
            这个问题留给你来完成……$S(x)$ 是 $P(x)$ 的必要条件吗?是充分条件吗?
            
            请注意,我们给定 $x$ 本身是一个整数……
    \end{enumerate}
\end{example}
