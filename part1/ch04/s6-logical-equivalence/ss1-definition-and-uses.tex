% !TeX root = ../../../book.tex
\subsection{定义与使用}\label{sec:section4.6.1}

以下定义为上一段中描述的逻辑等价概念引入了一个方便的符号:

\begin{definition}
    令 $P$ 和 $Q$ 为数学陈述。我们使用符号 ``$\iff$'' 表示``\dotuline{逻辑等价}于'',或``具有相同的真值''。

    也就是说,当 $P$ 和 $Q$ 始终具有相同的真值时,无论为\verb|真|还为\verb|假|,我们都写为 ``$P \iff Q$''。

    ``$P \iff Q$'' 读作,``$P$ 逻辑等价于 $Q$'' 或 ``$P$ \dotuline{当且仅当} $Q$''。

    这种类型的陈述称为\dotuline{双向条件关系}(或\dotuline{双向蕴含})。
\end{definition}

让我们复用上一节中的真值表,并为 $\iff$ 符号添加一列新列:

\begin{center}
    \begin{tabular}{c|c|c|c|c|c|c}
          $P$      & $Q$      & $\neg P$ &  $\neg P \lor Q$ & $P \implies Q$ & $Q \implies P$ & $P \iff Q$ \\
          \hline
          \verb|T| & \verb|T| & \verb|F| &      \verb|T|    &    \verb|T|    &    \verb|T|    & \verb|T|\\
          \verb|T| & \verb|F| & \verb|F| &      \verb|F|    &    \verb|F|    &    \verb|T|    & \verb|F|\\
          \verb|F| & \verb|T| & \verb|T| &      \verb|T|    &    \verb|T|    &    \verb|F|    & \verb|F|\\
          \verb|F| & \verb|F| & \verb|T| &      \verb|T|    &    \verb|T|    &    \verb|T|    & \verb|T|\\
    \end{tabular}
\end{center}
在 $P \iff Q$ 列中,当(且仅当)$P$ 和 $Q$ 具有相同真值时,该项具有真值 \verb|T|。这种情况发生在第 $1$ 行(两者都为 \verb|T|)和第 $4$ 行(两者都为 \verb|F|)。请注意,$P \iff Q$ 具有真值 \verb|T| 当且仅当
\[(P \implies Q) \land (Q \implies P)\]
是一个真命题。这就是\textbf{逻辑等价}的概念:$P \iff Q$ 意味着 $P \implies Q$ 和 $Q \implies P$ 同时成立。无论 $P$ 的真值如何,$Q$ 都保证具有相同的真值,反之亦然:
\begin{itemize}
    \item 假设 $P$ 为\verb|真|,则 $P \implies Q$ 告诉我们 $Q$ 也必须为\verb|真|。
    \item 假设 $P$ 为\verb|假|,则 $Q \implies P$ 告诉我们 $Q$ 不可能为\verb|真|(因为在这种情况下 $Q \implies P$ 将为\verb|假|),因此 $Q$ 也必须为\verb|假|。
\end{itemize}
无论哪种情况,$P$ 和 $Q$ 都具有相同的真值。

\subsubsection*{示例}

\begin{example}
    回看上面真值表中的第三列和第四列。他们证明了以下逻辑等价关系:
    \[(P \implies Q) \iff (\neg P \lor Q)\]
    无论 $P \implies Q$ 的真值是什么(当然,取决于 $P$ 和 $Q$),它都必然与 $\neg P \lor Q$ 具有相同的真值。我们之前已经提到过这种等价关系,将来我们会经常用到它。
\end{example}

\begin{example}
    请看下面的真值表:
    \begin{center}
        \begin{tabular}{c|c|c|c|c|c}
              $P$      & $Q$      & $\neg P$ &  $\neg Q$  & $P \implies Q$ & $\neg Q \implies \neg P$ \\
              \hline
              \verb|T| & \verb|T| & \verb|F| &  \verb|F|  &    \verb|T|    &    \verb|T|    \\
              \verb|T| & \verb|F| & \verb|F| &  \verb|T|  &    \verb|F|    &    \verb|F|    \\
              \verb|F| & \verb|T| & \verb|T| &  \verb|F|  &    \verb|T|    &    \verb|T|    \\
              \verb|F| & \verb|F| & \verb|T| &  \verb|T|  &    \verb|T|    &    \verb|T|    \\
        \end{tabular}
    \end{center}
    无论 $P$ 和 $Q$ 的真值如何,我们发现 $P \implies Q$ 和 $\neg Q \implies \neg P$ 具有相同的真值。因此,它们是\emph{逻辑等价}的,我们可以写做:
    \[(P \implies Q) \iff (\neg Q \implies \neg P)\]
    这是我们上一节中陈述(但没有证明)的事实:
    \begin{center}
        条件陈述的逆否命题与原陈述逻辑等价。
    \end{center}

    这一事实的另一种证明方法利用了将条件陈述改写为析取的方式。回忆一下上一个示例中提到的逻辑等价关系
    \[(P \implies Q) \iff (\neg P \lor Q)\]
    现在,考虑原始条件陈述的逆否形式:
    \[\neg Q \implies \neg P\]
    将相同的析取形式应用于该陈述会产生以下等价关系:
    \[(\neg Q \implies \neg P) \iff \big(\neg(\neg Q) \lor \neg P\big)\]
    而 $\neg(\neg Q)$ 等价于 $Q$,并且析取的顺序是无关的(即 $P \lor Q$ 和 $Q \lor P$ 具有相同的真值),因此我们得到
    \[(\neg Q \implies \neg P) \iff (\neg P \lor Q) \iff (P \implies Q)\]
    这从另一个角度证明了,条件陈述与其逆否形式具有相同的真值!
\end{example}

\begin{example}
    在本节后面部分,我们将证明以下逻辑等价,无论命题 $P$、$Q$ 和 $R$ 是什么,它们都成立:
    \begin{align*}
        \neg(P \land Q) &\iff \neg P \lor \neg Q \\
        (P \land Q) \land R &\iff P \land (Q \land R) \\
        P \lor (Q \land R) &\iff (P \lor Q) \land (P \lor R) \\
        \neg (P \implies Q) &\iff P \land \neg Q
    \end{align*}
    其中每一个都断言 $\iff$ 符号两边的表达式具有相同的真值。你能弄明白为什么这些说法是正确的吗?你能想到如何证明它们吗?
\end{example}

\subsubsection*{当且仅当}

逻辑等价与短语``当且仅当''具有良好的关系。说 ``$P$ 当且仅当 $Q$'' 意味着我们断言 ``$P$ 当 $Q$'' 且 ``$P$ 仅当 $Q$'' 都成立。其中一个对应于 $P \implies Q$,另一个对应于 $Q \implies P$,因此断言两者都为\verb|真|意味着正如我们所描述的:
\begin{center}
    $P \iff Q$ 与 $(P \implies Q) \land (Q \implies P)$ 含义相同。
\end{center}

那么,哪一个对应哪一个呢?当我们说 ``$P$ 当 $Q$'' 时,这意味着 ``如果 $Q$,则 $P$''。即,
\begin{center}
    $P$ 当 $Q$ 与 $Q \implies P$ 含义相同。
\end{center}

要弄清楚另一个方向有点困难!``$P$ 仅当 $Q$'' 的真正含义是什么?这句话断言,在 $P$ 成立的情况下,$Q$ 也一定成立。也就是说,知道 $P$ 成立意味着我们也立即知道 $Q$ 成立。换句话说,只要 $P$ 为真,我们就必然知道 $Q$ 为真。这相当于 $P \implies Q$ 成立!

另一种思考方式是这样的。说 ``仅当 $Q$ 时 $P$'' 与说 ``不可能 $P$ 成立而 $Q$ 不成立'' 相同。该陈述的逻辑表达式为
\[\neg(P \land \neg Q)\]
在本节后面部分,我们将讲解并证明\textbf{德摩根逻辑定律}。其中一条定律告诉我们如何否定括号内的陈述。(事实上,你可能已经知道这些逻辑定律。如果不知道,你可以先预览一下 \ref{sec:section4.6.5} 和 \ref{sec:section4.6.6} 节。)结论是:
\[\neg P \lor Q\]
正如我们已将发现的,这逻辑等价于 $P \implies Q$!酷。再一次证实 ``P 仅当 Q'' 意味着 $P \implies Q$。

\subsubsection*{在定义中使用 $\iff$}

我们会经常在\textbf{定义}中使用 ``$\iff$'' 符号来指示所定义的术语是定义中所用属性的等效术语。
例如:
\begin{center}
    我们说 $x \in \mathbb{Z}$ 为\textbf{偶数} $\iff \exists k \in \mathbb{Z} \centerdot x = 2k$
\end{center}
也就是说,整数为偶数的概念相当于知道该数字是某整数的两倍。类似地,我们可以定义\textbf{奇数}:
\begin{center}
    我们说 $x \in \mathbb{Z}$ 为\textbf{奇数} $\iff \exists k \in \mathbb{Z} \centerdot x = 2k+1$
\end{center}
请注意,以上是正式定义,并且是保证整数为偶数(或奇数)的唯一方法。我们很快就会使用这些定义来严格证明有关整数和算术的一些事实。每次我们想要断言一个特定的整数(称之为 $x$)是偶数时,我们需要证明存在一个满足 $x = 2k$ 的整数 $k$。也就是说,我们必须通过诉诸定义中给出的逻辑等价来\emph{满足定义}。

\subsubsection*{双向条件陈述:技术上的区别}

我们还可以使用 ``$\iff$'' 符号同时表达两个条件陈述。从技术上讲,这与断言逻辑等价并\emph{不完全}相同,但它传达了类似的思想,因此我们允许以两种方式使用该符号。

逻辑等价涉及一些未定义的命题,它断言两个命题将具有相同的真值,而不管这些命题的真值如何。例如,
\[(P \implies Q) \iff (\neg P \lor Q)\]
是逻辑等价的良好例子。如果不告诉你 $P$ 和 $Q$ 是什么,我们就无法确定 $P \implies Q$ 和 $\neg P \land Q$ 到底是什么意思。然而,我们不需要知道 $P$ 和 $Q$ 具体是什么就可以知道这两个陈述肯定具有相同的真值。

当 ``$\iff$'' 两侧的陈述实际上是正确的数学陈述,没有未定义的命题时,情况则略有不同。例如,考虑以下陈述:
\[\forall x \in \mathbb{R} \centerdot (x > 0) \iff \Big(\frac{1}{x}>0\Big)\]
这是一个逻辑断言,它断言,只要 $x$ 为实数,知道这两个事实之一($x > 0$ 或 $\frac{1}{x} > 0$)成立就必然保证另一个事实成立。也就是说,如果我告诉你我心里有一个实数并且它为正数,你就会得出结论,它的倒数也为正数。反过来,如果我告诉你我心里有一个实数,其倒数为正数,你就会得出结论,这个数本身也为正数。这是\emph{双向的}。(问题:如果我告诉你我心里有一个\emph{负}实数会怎样?你能得出关于其倒数的任何结论吗?为什么能或为什么不能?)

你看出这有什么区别了吗?给定任意 $x \in \mathbb{R}$,陈述 ``$x > 0$'' 肯定为\verb|真|或为\verb|假|。它的真值并非悬而未决。这与上面给出的例子不同,在上面给出的例子中,各个陈述的真值是未知的,但我们仍然可以声明两个陈述的真值必然是相同的。

由于缺乏更好、更广泛的术语来描述此类陈述,我们将它们称为\textbf{双向条件陈述}。这是因为它们实际上代表了两个``方向相反''的条件陈述:
\[\forall x \in \mathbb{R} \centerdot \Bigg[\Bigg((x > 0) \implies \Big(\frac{1}{x}>0\Big)\Bigg) \land \Bigg(\Big(\frac{1}{x}>0\Big) \implies (x > 0)\Bigg)\Bigg]\]
这就是上面陈述所说的:陈述的每一部分都蕴涵着另一部分。

该术语在其他数学著作中不一定是标准术语,但我们想指出这种技术差异,以便你了解它。你可能会在数学逻辑学家或集合论学家面前使用``逻辑等价''这个短语,他们可能会感到困惑或对你使用它的方式感到冒犯。请注意,这并不是一个大问题!由于我们现在是初次学习这些基本思想,因此我们不一定要记住这些概念背后的所有技术细节。此外,在本书的其余部分中,我们可以互换使用``逻辑等价''和``双x向条件''。目前这么做没什么问题,是可以接受的。

使用 ``$\iff$'' 符号的主要目的是断言两个陈述\emph{具有相同的真值}。``逻辑等价''和``双向条件''之间唯一的区别在于其中包含的陈述是否具有任意未定义的命题。从总体上看,这是一个很小的区别,所以我们在这里简单粗暴地将它们合在一起。
