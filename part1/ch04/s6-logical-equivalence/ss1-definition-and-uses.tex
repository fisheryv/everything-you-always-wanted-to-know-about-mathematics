% !TeX root = ../../../book.tex
\subsection{定义与使用}\label{sec:section4.6.1}

以下定义为逻辑等价引入了一个方便的符号:

\begin{definition}
    设 $P$ 和 $Q$ 为数学陈述。我们使用符号``$\iff$''表示``\dotuline{逻辑等价}'',或``具有相同的真值''。

    也就是说,当 $P$ 和 $Q$ 始终具有相同的真值时(无论为\verb|真|还是为\verb|假|),我们记作 ``$P \iff Q$''。

    ``$P \iff Q$''读作``$P$ 逻辑等价于 $Q$''或``$P$ \dotuline{当且仅当} $Q$''。

    此类陈述称为\dotuline{双向条件关系}(或\dotuline{双向蕴含})。
\end{definition}

复用上一节的真值表,为 $\iff$ 添加新列:
\begin{center}
    \begin{tabular}{c|c|c|c|c|c|c}
          $P$      & $Q$      & $\neg P$ &  $\neg P \lor Q$ & $P \implies Q$ & $Q \implies P$ & $P \iff Q$ \\
          \hline
          \verb|T| & \verb|T| & \verb|F| &      \verb|T|    &    \verb|T|    &    \verb|T|    & \verb|T|\\
          \verb|T| & \verb|F| & \verb|F| &      \verb|F|    &    \verb|F|    &    \verb|T|    & \verb|F|\\
          \verb|F| & \verb|T| & \verb|T| &      \verb|T|    &    \verb|T|    &    \verb|F|    & \verb|F|\\
          \verb|F| & \verb|F| & \verb|T| &      \verb|T|    &    \verb|T|    &    \verb|T|    & \verb|T|\\
    \end{tabular}
\end{center}
在 $P \iff Q$ 列中,当且仅当 $P$ 和 $Q$ 真值相同时,该行取值为 \verb|T|(第 $1$ 行二者皆为 \verb|T|,第 $4$ 行二者皆为 \verb|F|)。注意 $P \iff Q$ 为真当且仅当
\[(P \implies Q) \land (Q \implies P)\]
成立。这体现了\textbf{逻辑等价}的本质:$P \iff Q$ 表明 $P \implies Q$ 与 $Q \implies P$ 同时成立。无论 $P$ 取值如何,$Q$ 必与之相同,反之亦然:
\begin{itemize}
    \item 假设 $P$ 为\verb|真|,则 $P \implies Q$ 要求 $Q$ 也必须为\verb|真|。
    \item 假设 $P$ 为\verb|假|,则 $Q \implies P$ 要求 $Q$ 不可能为\verb|真|(否则 $Q \implies P$ 为\verb|假|),因此 $Q$ 也必须为\verb|假|。
\end{itemize}
无论哪种情况,$P$ 和 $Q$ 都具有相同的真值。

\subsubsection*{示例}

\begin{example}
    观察前述真值表第三、第四列,可得如下逻辑等价关系:
    \[(P \implies Q) \iff (\neg P \lor Q)\]
    无论 $P \implies Q$ 的真值是什么(当然,取决于 $P$ 和 $Q$),它都必然与 $\neg P \lor Q$ 具有相同的真值。我们之前已经提及这种等价关系,后续将频繁使用。
\end{example}

\begin{example}
    请看下面的真值表:
    \begin{center}
        \begin{tabular}{c|c|c|c|c|c}
              $P$      & $Q$      & $\neg P$ &  $\neg Q$  & $P \implies Q$ & $\neg Q \implies \neg P$ \\
              \hline
              \verb|T| & \verb|T| & \verb|F| &  \verb|F|  &    \verb|T|    &    \verb|T|    \\
              \verb|T| & \verb|F| & \verb|F| &  \verb|T|  &    \verb|F|    &    \verb|F|    \\
              \verb|F| & \verb|T| & \verb|T| &  \verb|F|  &    \verb|T|    &    \verb|T|    \\
              \verb|F| & \verb|F| & \verb|T| &  \verb|T|  &    \verb|T|    &    \verb|T|    \\
        \end{tabular}
    \end{center}
    可见,无论 $P$ 和 $Q$ 的真值如何,$P \implies Q$ 与 $\neg Q \implies \neg P$ 具有相同的真值。故二者\emph{逻辑等价},可以写作:
    \[(P \implies Q) \iff (\neg Q \implies \neg P)\]
    这正是我们上一节中陈述(但没有证明)的事实:
    \begin{center}
        条件陈述的逆否命题与原陈述逻辑等价。
    \end{center}

    这一事实的另一种证明方法利用了将条件陈述改写为析取形式。回忆一下前例中提到的逻辑等价关系
    \[(P \implies Q) \iff (\neg P \lor Q)\]
    现在,考虑其逆否命题:
    \[\neg Q \implies \neg P\]
    将相同的析取形式应用于该陈述会产生以下等价关系:
    \[(\neg Q \implies \neg P) \iff \big(\neg(\neg Q) \lor \neg P\big)\]
    而 $\neg(\neg Q)$ 等价于 $Q$,并且析取的顺序是无关的(即 $P \lor Q$ 与 $Q \lor P$ 具有相同的真值),因此可得
    \[(\neg Q \implies \neg P) \iff (\neg P \lor Q) \iff (P \implies Q)\]
    这从另一个角度证明,条件陈述与其逆否形式具有相同的真值!
\end{example}

\begin{example}
    本节后续将证明下列逻辑等价对任意命题 $P$、$Q$、$R$ 均成立:
    \begin{align*}
        \neg(P \land Q) &\iff \neg P \lor \neg Q \\
        (P \land Q) \land R &\iff P \land (Q \land R) \\
        P \lor (Q \land R) &\iff (P \lor Q) \land (P \lor R) \\
        \neg (P \implies Q) &\iff P \land \neg Q
    \end{align*}
    以上每个等价式均表明 $\iff$ 两侧的表达式具有相同的真值。你能理解其正确性吗?能否构思证明方法?
\end{example}

\subsubsection*{当且仅当}

逻辑等价与短语``当且仅当''密切相关。说``$P$ 当且仅当 $Q$''意味着我们同时断言``$P$ 当 $Q$''和``$P$ 仅当 $Q$''成立。前一个分句对应 $Q \implies P$,后一个分句对应 $P \implies Q$,因此断言两者均为\verb|真|等价于:
\begin{center}
    $P \iff Q$ 与 $(P \implies Q) \land (Q \implies P)$ 含义相同。
\end{center}

具体而言,当我们说``$P$ 当 $Q$''时,这意味着``若 $Q$ 成立,则 $P$ 成立''。即,
\begin{center}
    $P$ 当 $Q$ 与 $Q \implies P$ 含义相同。
\end{center}

理解另一方向则需要更仔细的分析。``$P$ 仅当 $Q$''断言:若 $P$ 成立,则 $Q$ 必然成立。换言之,$P$ 为真时 $Q$ 必定为真,这等价于 $P \implies Q$。

另一种理解是:``仅当 $Q$ 时 $P$''等同于``$P$ 成立而 $Q$ 不成立的情况不可能发生'',其逻辑表达式为:
\[\neg(P \land \neg Q)\]
后文将讲解并证明\textbf{德摩根定律}(可提前参考 \ref{sec:section4.6.5} 和 \ref{sec:section4.6.6} 节),该定律表明上式等价于:
\[\neg P \lor Q\]
如前所述,此式逻辑等价于 $P \implies Q$。这再次验证``$P$ 仅当 $Q$''意味着 $P \implies Q$。

\subsubsection*{在定义中使用 $\iff$}

我们会经常在\textbf{定义}中使用 ``$\iff$''符号,表示所定义项与所描述性质完全等价。例如:
\begin{center}
    称 $x \in \mathbb{Z}$ 为\textbf{偶数} $\iff \exists k \in \mathbb{Z} \centerdot x = 2k$
\end{center}
也就是说,整数为偶数与它能表示为某个整数的两倍等价。类似地,我们可以定义\textbf{奇数}:
\begin{center}
    称 $x \in \mathbb{Z}$ 为\textbf{奇数} $\iff \exists k \in \mathbb{Z} \centerdot x = 2k+1$
\end{center}
请注意,以上为形式化定义,完整描述了偶数和奇数的本质。后续将用这些定义严格证明整数的性质。每当我们想断言某个整数 $x$ 为偶数,需要证明存在整数 $k$ 满足 $x = 2k$,即通过满足定义中的逻辑等价来验证。

\subsubsection*{双向条件陈述:技术上的区别}

我们还可以用``$\iff$''符号同时表达两个条件陈述。技术上这与断言逻辑等价并\emph{不完全}等同,但两者传达类似的思想,因此我们允许该符号的两种用法。

逻辑等价涉及未定义命题,断言两命题在任何情况下均具有相同的真值。例如:
\[(P \implies Q) \iff (\neg P \lor Q)\]
是逻辑等价的典型例子。若不知 $P$ 和 $Q$ 的具体含义,虽无法确定 $P \implies Q$ 与 $\neg P \lor Q$ 的实质意义,但可知二者真值必然相同。

当``$\iff$''两侧均为不含未定义命题的确定数学陈述时,情况略有不同。例如:
\[\forall x \in \mathbb{R} \centerdot (x > 0) \iff \Big(\frac{1}{x}>0\Big)\]
此断言表明:对于任意实数 $x$,若已知其中一个条件成立($x > 0$ 或 $\frac{1}{x} > 0$),则另一条件必然成立。换言之,若告知某实数为正,可推知其倒数亦为正;反之,若告知某实数的倒数为正,亦可推知该数本身为正。此即\emph{双向性}。(思考:若告知某\emph{负}实数,能否推知其倒数性质?原因何在?)

你看出区别了吗?此处对于任意 $x \in \mathbb{R}$,陈述``$x > 0$''必有确定真值。这与前面给出的例子不同,前述例子中各陈述的真值未知,但我们仍然可以声明两个陈述的真值必然相同。

由于缺乏更广泛的术语,我们称此类陈述为\textbf{双向条件陈述}。其本质是两个``方向相反''的条件陈述:
\[\forall x \in \mathbb{R} \centerdot \Bigg[\Bigg((x > 0) \implies \Big(\frac{1}{x}>0\Big)\Bigg) \land \Bigg(\Big(\frac{1}{x}>0\Big) \implies (x > 0)\Bigg)\Bigg]\]
这就是上面陈述所说的:陈述的每一部分都蕴涵着另一部分。

该术语在其他数学著作中不一定是标准术语,我们在此指出这种技术差异,以便你了解它。若在数理逻辑学家或集合论学家面前使用``逻辑等价''指代此概念,可能会引发困惑。请注意此差异虽然细微,但现阶段学习基础概念时,无需强记技术细节。本书后续内容中,``逻辑等价''与``双向条件''可互换使用,此做法目前可以接受。

``$\iff$''的核心功能是断言两个陈述\emph{真值相同}。``逻辑等价''与``双向条件''之间唯一的区别在于是否包含未定义的命题,此区别甚微,故本书将二者合并处理。
