% !TeX root = ../../../book.tex
\subsection{通过条件陈述证明集合包含}

尽可能使用上一节中采用的方法,以及德摩根逻辑和集合定律;即通过条件陈述和逻辑等价证明集合关系。一般来说,证明集合相等时,需确保所有步骤均基于双向蕴含(``$\iff$'')。在上一节中,我们仅应用逻辑等价的定义和定理,因而保证了证明中所有``$\iff$''箭头的双向性。完成此类证明后,建议重新检查每一行并问自己:``此步骤正确吗?蕴含是双向的吗?''

下面通过一个实际例子演示这一技术。该例子稍复杂,因为给定命题与德摩根逻辑定律不完全相同,因此需要定义命题变量。不过,我们将引用已证明的逻辑定律,并在此基础上建立集合定律。

\begin{proposition}
    设 $A, B, C$ 为全集 $U$ 的任意子集,则:
    \[A \cap (B - C) = (A \cap B) - C\]
\end{proposition}

与德摩根集合定律的证明类似,我们将建立左侧元素与右侧元素间的逻辑等价关系(这等价于同时完成双向包含证明)。为此定义若干命题变量以描述元素在 $A, B, C$ 中的属性,结果将由逻辑法则直接推出。

\begin{proof}
    设 $A, B, C$ 为全集 $U$ 的任意子集。定义以下命题变量:
    \begin{center}
        设 $P(x)$ 为 ``$x \in A$'' \\
        设 $Q(x)$ 为 ``$x \in B$'' \\
        设 $R(x)$ 为 ``$x \in C$'' 
    \end{center}
    设 $x \in U$ 为任意固定元素。基于上述定义,可建立如下逻辑等价链:
    \begin{align*}
        x \in A \cap (B - C) &\iff x \in A \land (x \in B - C) &\quad \cap \text{\ 的定义} \\
        &\iff x \in A \land (x \in B \land x \notin C) &\quad - \text{\ 的定义}\\
        &\iff P(x) \land (Q(x) \land \neg R(x)) &\quad P(x), Q(x), R(x), \notin \text{\ 的定义} \\
        &\iff (P(x) \land Q(x)) \land \neg R(x) &\quad \land \text{\ 分配律} \\
        &\iff (x \in A \land x \in B) \land x \notin C &\quad P(x), Q(x), R(x) \text{\ 的定义}\\
        &\iff (x \in A \cap B \land x \notin C) &\quad \cap \text{\ 的定义} \\
        &\iff x \in (A \cap B) - C &\quad - \text{\ 的定义}
    \end{align*}
    这表明
    \[x \in A \cap (B - C) \iff x \in (A \cap B) - C\]
    对于全集 $U$ 中的任意元素 $x$ 均成立。因此
    \[A \cap (B - C) = (A \cap B) - C\]
\end{proof}

思考为什么我们需要确保所有这些陈述都是正确的\emph{当且仅当}陈述。我们不仅需要证明左侧集合中的任意元素 $x$ 必然属于右侧集合,还需确保不属于右侧集合的元素 $x$ 也必然不属于左侧集合。双向条件陈述意味着``两个方向都成立'',因此我们同时证明了``属于集合''和``不属于集合''这两个主张。

为了说明前文提到的注意事项,请参考以下命题的证明示例,其中 $\iff$ 声明在一个方向上不成立。

\begin{proposition}\label{prop:proposition4.6.11}
    设 $X, Y, Z$ 为全集 $U$ 的任意子集,则以下包含关系成立:
    \[(X \cup Y ) - Z \subseteq X \cup (Y - Z)\]
\end{proposition}

你可能已发现这就是习题 \ref{exc:exercises3.11.17}!该习题要求通过包含论证证明此命题:假设任意 $x \in U$ 属于左侧集合,推导其必然属于右侧集合。下文将用逻辑符号重述(本质上)相同的论证,旨在:
\begin{enumerate}
    \item 帮助我们熟悉此类论证;
    \item 精确识别论证中``反方向''不成立的关键点。
\end{enumerate}
请注意,习题 \ref{exc:exercises3.11.17} 还要求构造反例以说明 $\supseteq$ 方向未必成立。这意味着该方向的逻辑论证存在漏洞,我们将定位其失效点并据此构造所需反例。

\begin{proof}
    设 $X, Y, Z$ 为全集 $U$ 的任意子集。设 $x \in U$ 为任意固定元素,可建立如下逻辑等价链:
    \begin{align*}
        x \in (X \cup Y ) - Z &\iff x \in X \cup Y \land x \notin Z &\quad - \text{\ 的定义}\\
        &\iff (x \in X \lor x \in Y ) \land x \notin Z &\quad \cup \text{\ 的定义} \\
        &\iff (x \in X \land x \notin Z) \lor (x \in Y \land x \notin Z) &\quad \text{德摩根逻辑定律} 
    \end{align*}

    \setlength{\fboxrule}{2pt}
    \setlength\fboxsep{5mm}
    \begin{center}
    \fcolorbox{red}{white}{%
        \parbox{0.85\textwidth}{%
            \textcolor{red}{\textbf{草稿:}}

            进一步推导逻辑等价关系,可将右侧化简为:
            \[x \in X - Z \lor x \in X - Z\]
            从而
            \[x \in (X - Z) \cup (Y - Z)\]
            这证明了另一命题:
            \[(X \cup Y ) - Z = (X - Z) \cup (Y - Z)\]
            然而,我们需要证明的原命题右侧为:
            \[X \cup (Y - Z)\]
            且我们要证明的并不是相等关系,而是证明\emph{包含关系}。因此,后续需要证明:
            \[\left((x \in X \land x \notin Z) \lor (x \in Y \land x \notin Z)\right) \implies x \in X \cup (Y - Z)\]
            为明确此蕴含关系,这里需要做一些临时工作,重写右侧表达式,即可看到为什么上述公式成立:
            \begin{align*}
                x \in X \cup (Y - Z) &\iff x \in X \lor x \in Y - Z &\quad \cup \text{\ 的定义} \\
                &\iff x \in X \lor (x \in Y \land x \notin Z) &\quad - \text{\ 的定义}
            \end{align*}
            此式与先前推导的最后一个逻辑等价类似,但左侧项不同。请思考蕴含关系成立的原因,然后再继续阅读剩下的证明。
        }
    }
    \end{center}
    现在需要证明
    \[\left((x \in X \land x \notin Z) \lor (x \in Y \land x \notin Z)\right) \implies x \in X \cup (Y - Z)\]
    为此,假设左侧表达式为\verb|真|,则有两种情形:
    \begin{enumerate}
        \item 假设 $x \in X \land x \notin Z$ 为\verb|真|,则 $x \in X$,因此 $x \in X \lor x \in Y - Z$ 成立。
        \item 假设 $x \in Y \land x \notin Z$ 为\verb|真|,则 $x \in Y - Z$,因此 $x \in X \lor x \in Y - Z$ 成立。
    \end{enumerate}
    两种情形下均有 $x \in X \lor x \in Y - Z$ 成立,由并集定义可得:
    \[x \in X \cup (Y - Z)\]
    都成立。

    综上,对于每个元素 $x \in U$ 
    \[x \in (X \cup Y ) - Z \implies x \in X \cup (Y - Z)\]
    都成立。因此,根据 $\subseteq$ 的定义,可得:
    \[(X \cup Y ) - Z \subseteq X \cup (Y - Z)\]
\end{proof}

认清我们在哪里以及我们想去哪里,帮助我们完成了这个证明。我们无法仅通过逻辑等价完成证明,因为事实上,给定的集合等式并不总是成立!现在回顾证明,能否识别逻辑等价失效的步骤?能否利用这一点构建反例,反驳这些集合恒等的错误主张?

我们已知以下有效陈述:
\[(x \in X \land x \notin Z) \lor (x \in Y \land x \notin Z)\]
并由此推导出:
\[x \in X \lor (x \in Y \land x \notin Z)\]
根据证明的论证过程,第一个陈述确实蕴涵第二个陈述:若假设第一个陈述成立,则第二个陈述必然成立。两者差异仅在首项,而当一个``$\land$''陈述整体成立时,其任意部分必然成立。

但此推导不可逆。假设第二个陈述成立:若右项 $x \in Y \land x \notin Z$ 有效,则第一个陈述成立。然而,由于这是``$\lor$''陈述,必须考虑左项 $x \in X$ 成立的情形。此时仅凭 $x \in X$ 无法推出 $x \in X \land x \notin Z$ —— 已知``$\land$''成立可推断其各部分,但仅知部分成立不能保证整体成立!

由此可构造反例:只需保证存在元素 $x \in U$ 满足第二个陈述的左项 $x \in X$,却不满足第一个陈述的左项 $x \in X \land x \notin Z$ 即可。换言之,我们需要构造元素 $x \in X \cap Z$。下面的示例实现了此目标:

\begin{example}
    对于任意集合 $X,Y,Z$,存在包含关系
    \[(X \cup Y ) - Z \subseteq X \cup (Y - Z)\]
    但相等关系不一定成立。请参阅命题 \ref{prop:proposition4.6.11} 的证明,了解为什么上述包含关系确实成立。
\end{example}

现在,考虑以下示例。定义:
\[X = \{1\}, \quad Y = \{2\}, \quad Z = \{1, 2\}\]
注意到
\begin{align*}
    (X \cup Y) - Z &= (\{1\} \cup \{2\}) - \{1, 2\} = \{1, 2\} - \{1, 2\} = \varnothing \\
    X \cup (Y - Z) &= \{1\} \cup (\{2\} - \{1, 2\}) = \{1\} \cup \{\varnothing\} = \{1\}
\end{align*}
由于 $\varnothing \subset \{1\}$ 是(真)子集关系,此时有:
\[(X \cup Y ) - Z \ne X \cup (Y - Z)\]
这表明原命题中的等式并非恒成立。

此策略能高效严谨地处理诸多集合证明!我们可以运用已\emph{证明}的逻辑符号与定律,而非依赖``与''和``或''的语言学直觉。正因如此,本节习题聚焦集合理论。建议回顾第 \ref{ch:chapter03} 章重温相关定义。
