% !TeX root = ../../../book.tex
\subsection{习题}

\subsubsection*{温故知新}

以口头或书面的形式简要回答以下问题。这些问题全都基于你刚刚阅读的内容,所以如果忘记了具体的定义、概念或示例,可以回去重读相关部分。确保在继续学习之前能够自信地回答这些问题,这将有助于你的理解和记忆!

\begin{enumerate}[label=(\arabic*)]
    \item $\land$ 和 $\lor$ 有什么区别?
    \item $\land$ 和 $\cap$ 有什么区别?\\
        $\lor$ 和 $\cup$ 有什么区别?
    \item 写出陈述 $P \implies Q$ 的真值表。
    \item 为什么 $P \implies Q$ 和 $\neg P \lor Q$ 是逻辑等价的?
    \item 条件陈述的逆命题是什么?\\
        条件陈述的逆否命题是什么?
    \item 条件陈述的真值与其逆命题是否相关?
\end{enumerate}

\subsubsection*{小试牛刀}

尝试回答以下问题。这些题目要求你实际动笔写下答案,或(对朋友/同学)口头陈述答案。目的是帮助你练习使用新的概念、定义和符号。题目都比较简单,确保能够解决这些问题将对你大有帮助!

\begin{enumerate}[label=(\arabic*)]
    \item 对于以下每个句子,使用逻辑符号重写并确定其为\verb|真|还是\verb|假|。
        \begin{enumerate}[label=(\alph*)]
            \item 每个整数要么严格为正,要么严格为负。
            \item 对于任意给定的实数,都存在一个严格大于它的自然数。
            \item 对于每个实数,如果它是负数,那么它的立方也是负数。
            \item $\mathbb{Z}$ 的一个子集具有以下性质:每当一个数字是该集合的元素时,它的平方也是该集合的元素。
            \item 存在一个既是偶数又是奇数的自然数。
        \end{enumerate}
    \item 将以下每个 $\forall$ 命题重写为条件陈述,并确定它为\verb|真|还是\verb|假|。
        \begin{enumerate}[label=(\alph*)]
            \item $\forall x \in \{y \in \mathbb{N} \mid \exists k \in \mathbb{Z} \centerdot y = 2k\} \centerdot x^2 \in \{y \in \mathbb{N} \mid \exists k \in \mathbb{Z} \centerdot y = 2k\}$
            \item $\forall x \in \{y \in \mathbb{N} \mid \exists k \in \mathbb{Z} \centerdot y = 2k + 1\} \centerdot x^2 \in \{y \in \mathbb{N} \mid \exists k \in \mathbb{Z} \centerdot y = 2k + 1\}$
            \item $\forall t \in \{z \in \mathbb{R} \mid z^2 > 4\} \centerdot t > 2$
        \end{enumerate}
    \item 使用集合构建符将以下每个条件陈述重写为 $\forall$ 声明,并确定其为\verb|真|还是\verb|假|。
        \begin{enumerate}[label=(\alph*)]
            \item $\forall x \in \mathbb{R} \centerdot x > 3 \implies x^2 < 9$
            \item $\forall x \in \mathbb{R} \centerdot x < 3 \implies x^2 < 9$
            \item $\forall t \in \mathbb{R} \centerdot t^2 - 6t + 9 \ge 0 \implies t \ge 3$
        \end{enumerate}
    \item 们定义以下变量命题:
        \begin{align*}
            P(x) &= \frac{1}{2} < x \\
            Q(x) &= x < \frac{3}{2} \\
            R(x) &= x^2 = 4 \\
            S(x) &= x + 1 \in \mathbb{N} 
        \end{align*}
        对于以下每个陈述,确定它为\verb|真|还是\verb|假|。
        \begin{enumerate}[label=(\alph*)]
            \item $\forall x \in \mathbb{N} \centerdot P(x)$
            \item $\forall x \in \mathbb{N} \centerdot Q(x) \implies P(x)$
            \item $\forall x \in \mathbb{Z} \centerdot Q(x) \implies P(x)$
            \item $\exists x \in \mathbb{N} \centerdot \neg S(x) \lor R(x)$
            \item $\exists x \in \mathbb{Z} \centerdot R(x) \land \neg S(x)$
            \item $\forall x \in \mathbb{R} \centerdot R(x) \implies S(x)$
            \item $\exists x \in \mathbb{R} \centerdot P(x) \land S(x)$
            \item $\forall x \in \mathbb{Z} \centerdot R(x) \implies \big(P(x) \lor Q(x)\big)$
        \end{enumerate}
    \item 对于以下每个条件陈述,用逻辑符号重写它,并写出它的逆命题和逆否命题;然后,确定所有三个命题的真值:原命题、逆命题和逆否命题。
        \begin{enumerate}[label=(\alph*)]
            \item 如果一个实数严格介于 $0$ 和 $1$ 之间,那么它的平方也是如此。
            \item 如果一个自然数是偶数,那么它的立方也是偶数。
            \item 每当一个整数是 $10$ 的倍数时,它也是 $5$ 的倍数。
        \end{enumerate}
\end{enumerate}