% !TeX root = ../../../book.tex
\subsection{逻辑与}

说``$P$ 和 $Q$''为\verb|真|,意味着 $P$ 和 $Q$ 均为\verb|真|。若 $P$ 或 $Q$ 中有一个为\verb|假|,则``$P$ 和 $Q$''也为\verb|假|。其定义如下:

\begin{definition}
    在数学陈述间使用符号``$\land$''表示``和''。例如,``$P \land Q$''读作``$P$ 和 $Q$''。

    这称为 $P$ 和 $Q$ 的合取。

    当 $P$ 和 $Q$ 均为真时,``$P \land Q$''的真值为\verb|真|,否则为\verb|假|。
\end{definition}

以下是此定义的一些示例:

\begin{example}
    \begin{align*}
        (1 + 3 = 4) \land (\forall x \in \mathbb{R} \centerdot x^2 \ge 0) \qquad &\text{真} \\
        (1 + 3 = 5) \land (\forall x \in \mathbb{R} \centerdot x^2 \ge 0) \qquad &\text{假} \\
        (1 + 3 = 5) \land (\exists x \in \mathbb{Q} \centerdot x^2 = 2)   \qquad &\text{假}
    \end{align*}
\end{example}

\subsubsection*{符号:括号}

上面示例中的括号常常可以省略。例如,第一行可等价写做:
\[1 + 3 = 4 \land \forall x \in \mathbb{R} \centerdot x^2 \ge 0\]
使用括号能增强可读性。无括号时虽然仍可理解,但需要额外时间划分语句边界。建议在括号能提升清晰度时使用。

\subsubsection*{符号:集合与逻辑}

你可能会注意到逻辑连词``$\land$''与集合运算符``$\cap$''之间的相似性。这并非巧合!

正如 \ref{sec:section4.5.4} 节所述,``$\cap$''的定义可以基于逻辑用``$\land$''来表述。读者可先行尝试,再参阅该节。需要特别注意的是,若 $A$ 和 $B$ 是集合,则 ``$A \land B$''无明确定义,正确形式应该是``$A \cap B$''。
