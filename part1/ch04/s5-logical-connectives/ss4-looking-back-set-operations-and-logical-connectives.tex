% !TeX root = ../../../book.tex
\subsection{回顾:集合运算与逻辑连词}\label{sec:section4.5.4}

回顾 \ref{sec:section3.4} 和 \ref{sec:section3.5} 节,我们定义了子集与集合运算。这些定义当时借助了逻辑思想,但以自然语言表述,依赖于集合直觉和逻辑知识。现在我们可以用量词和连词重新表述它们!

首先回顾\textbf{子集}定义:若满足条件``当 $x \in A$ 时,必有 $x \in B$'',则记作 $A \subseteq B$。注意关键词``当……时'',它同时体现\emph{全称量化}和\emph{条件陈述}。请尝试用量词重写该定义,再对照我们的版本:

\begin{definition}
    设 $A, B, U$ 为集合,其中 $A, B \subseteq U$(即 $U$ 为全集)。称 $A$ 是 $B$ 的\dotuline{子集},记作 $A \subseteq B$,当且仅当
    \[\forall x \in U \centerdot x \in A \implies x \in B \]
\end{definition}

此定义合理对应了``当 $x \in A$ 时,必有 $x \in B$''的原始陈述:它确保如果 $x \in A$ 成立,则 $x \in B$ 必然成立。

接下来回顾集合运算的定义。请尝试用逻辑符号自行重写这些定义,再与我们的版本比较,思考其合理性与等价性。

\begin{definition}
    设 $A, B, U$ 为集合,其中 $A, B \subseteq U$(即 $U$ 为全集)。则
    \begin{align*}
        A \cap B &= \{x \in U \mid x \in A \land x \in B\} \\
        A \cup B &= \{x \in U \mid x \in A \lor x \in B\} \\
        A - B &= \{x \in U \mid x \in A \land \neg (x \in B)\} = \{x\in U \mid x \in A \land x \notin B\} \\
        \overline{A} &= \{x \in U \mid \neg (x \in A)\} = \{x \in U \mid x \notin A\}
    \end{align*}
\end{definition}

我们还可以重新定义集合的划分。这将用到逻辑连词,并涉及索引集及其量化定义——此前所学知识在此汇聚!

\begin{definition} \label{def:definition4.5.11}
    设 $A$ 为集合。$A$ 的\dotuline{划分}是其非空子集构成的集合,满足两两不相交且并集为 $A$。

    也就是说,划分由满足以下条件的索引集 $I$ 和非空集 $S_i$(定义在每一个 $i \in I$ 上)构成:
    \begin{enumerate}[label=(\arabic*)]
        \item $\forall i \in I \centerdot S_i \subseteq A$
        \item $\forall i, j \in I \centerdot i \ne j \implies S_i \cap S_j = \varnothing$
        \item $\displaystyle{\bigcup_{i \in I} S_i = A}$
    \end{enumerate}
\end{definition}

参考定义 \ref{def:definition3.6.9} 的原始表述,能否看出我们如何用逻辑符号表达同一概念?
