% !TeX root = ../../../book.tex
\subsection{回顾:集合运算与逻辑连词}\label{sec:section4.5.4}

回顾一下 \ref{sec:section3.4} 和 \ref{sec:section3.5} 节,我们定义了子集和集合运算。所有这些定义都使用了一些逻辑思想,但当时我们是用自然语言书写的,依靠的是我们的集合直觉和逻辑知识。我们现在可以使用量词和连词重写它们!

首先,回顾一下\textbf{子集}的定义。如果以下条件成立,我们就写做 $A \subseteq B$:每当 $x \in A$ 时,我们也可以说 $x \in B$。注意关键词``每当'',它既表示\emph{全称量化}又表示\emph{条件陈述}。想想如何使用这些概念重写 $A \subseteq B$ 的定义,然后继续阅读我们的版本……

\begin{definition}
    设 $A, B, U$ 为集合,其中 $A, B \subseteq U$(即 $U$ 为全集)。我们说 $A$ 是 $B$ 的\dotuline{子集},并写作 $A \subseteq B$,当且仅当
    \[\forall x \in U \centerdot x \in A \implies x \in B \]
\end{definition}

这是合理的,因为它断言了我们在上一段中写的``每当''陈述:每当 $x \in A$ 时,我们也必然能够得出 $x \in B$ 的结论;``如果 $x \in A$,则 $x \in B$'' 必然成立。

再回顾一下我们给出的集合运算的定义。尝试使用逻辑符号为这些定义编写你自己版本的定义,然后在再阅读我们的版本。想想它们为什么合理,如何表达相同的基本想法。

\begin{definition}
    设 $A, B, U$ 为集合,其中 $A, B \subseteq U$(即 $U$ 为全集)。则
    \begin{align*}
        A \cap B &= \{x \in U \mid x \in A \land x \in B\} \\
        A \cup B &= \{x \in U \mid x \in A \lor x \in B\} \\
        A - B &= \{x \in U \mid x \in A \land \neg (x \in B)\} = \{x\in U \mid x \in A \land x \notin B\} \\
        \overline{A} &= \{x \in U \mid \neg (x \in A)\} = \{x \in U \mid x \notin A\}
    \end{align*}
\end{definition}

我们还可以重新定义集合的划分。这将用到逻辑连词,也会涉及索引集以及如何用量词定义它们。我们所学到的一切都在这里汇集!

\begin{definition} \label{def:definition4.5.11}
    设 $A$ 为集合。 $A$ 的\dotuline{划分}是两两不相交且并集为 $A$ 的集合的集合。

    也就是说,分区由满足以下条件的索引集 $I$ 和非空集 $S_i$(定义在每一个 $i \in I$ 上)构成:
    \begin{enumerate}[label=(\arabic*)]
        \item $\forall i \in I \centerdot S_i \subseteq A$
        \item $\forall i, j \in I \centerdot i \ne j \implies S_i \cap S_j = \varnothing$
        \item $\displaystyle{\bigcup_{i \in I} S_i = A}$
    \end{enumerate}
\end{definition}

回顾一下定义 \ref{def:definition3.6.9},看看我们最初是如何定义划分的。你看到我们在这里如何只使用逻辑符号来表达同样的事情了吗?

\clearpage
