% !TeX root = ../../../book.tex
\subsection{或}

说
\begin{center}
    ``$P$ 或 $Q$'' 为\verb|真|
\end{center}

表示 ``$P$ 为\verb|真|,或 $Q$ 为\verb|真|''。我们需要知道其中一个陈述为\verb|真|才能声明整个陈述的真值为\verb|真|。我们不关心 $P$ 和 $Q$ 是否\emph{都}为\verb|真|,只关心其中\emph{至少一个}为\verb|真|。

这与计算机科学中所谓的 ``异或''(也称为 \verb|XOR|)不同,当 $P$ 和 $Q$ 都为\verb|真| 时,``$P$ \verb|XOR| $Q$'' 为\verb|假|。在数学中,我们使用\textbf{同``或''}。我们只关心其中至少一项陈述是否成立。

\begin{definition}
    我们在两个数学陈述之间使用符号 ``$\lor$'' 来表示 ``或''。例如,我们将 ``$P \lor Q$'' 读作 ``$P$ 或 $Q$''。

    这称为 $P$ 和 $Q$ 的析取。

    当 $P$ 和 $Q$ 中至少一个为\verb|真|时(即使两者都为\verb|真|),``$P \lor Q$'' 的真值为\verb|真|,否则真值为\verb|假|。
\end{definition}

\begin{example}
    \begin{align*}
        (1 + 3 = 4) \lor (\forall x \in \mathbb{R} \centerdot x^2 \ge 0) \qquad &\text{真} \\
        (1 + 3 = 5) \lor (\forall x \in \mathbb{R} \centerdot x^2 \ge 0) \qquad &\text{真} \\
        (1 + 3 = 5) \lor (\exists x \in \mathbb{R} \centerdot x^2 < 0)   \qquad &\text{假}
    \end{align*}
\end{example}

\subsubsection*{符号}

我们在上一小节中提到的有关符号的注释同样也适用于此。首先,使用括号(如上面的示例所示)很有帮助,但在技术上不是必需的。不过,只要有帮助,我们就应当使用它们。

其次,你可能会注意到逻辑连词 ``$\lor$'' 和集合运算符 ``$\cup$'' 之间的相似性。再次强调,这不是巧合!尝试使用 ``$\lor$'' 重写 ``$\cup$'' 的定义,并简要浏览一下第 \ref{sec:section4.5.4} 节。但一般来说,请小心区分这两个符号!如果 $A$ 和 $B$ 为集合,则 ``$A \lor B$'' 没有明确定义;正确含义应该是 ``$A \cup B$''。
