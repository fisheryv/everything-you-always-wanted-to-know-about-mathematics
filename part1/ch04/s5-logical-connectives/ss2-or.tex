% !TeX root = ../../../book.tex
\subsection{逻辑或}

说``$P$ 或 $Q$'' 为\verb|真|,意味着``$P$ 为\verb|真|,或 $Q$ 为\verb|真|''。只要其中一个陈述为\verb|真|,整个陈述就为\verb|真|。我们不关心 $P$ 和 $Q$ 是否均为\verb|真|,只要求\emph{至少一个}为\verb|真|。

这与计算机科学中的``异或''(\verb|XOR|) 不同:当 $P$ 和 $Q$ 均为\verb|真|时,``$P$ \verb|XOR| $Q$''为\verb|假|。数学中采用\textbf{兼或} (inclusive or),只需至少一个陈述成立。

\begin{definition}
    在数学陈述间使用符号``$\lor$''表示``或''。例如,``$P \lor Q$''读作``$P$ 或 $Q$''。

    这称为 $P$ 和 $Q$ 的析取。
    
    当 $P$ 和 $Q$ 中至少一个为\verb|真|时(或者两者都为\verb|真|),``$P \lor Q$''的真值为\verb|真|,否则为\verb|假|。
\end{definition}

\begin{example}
    \begin{align*}
        (1 + 3 = 4) \lor (\forall x \in \mathbb{R} \centerdot x^2 \ge 0) \qquad &\text{真} \\
        (1 + 3 = 5) \lor (\forall x \in \mathbb{R} \centerdot x^2 \ge 0) \qquad &\text{真} \\
        (1 + 3 = 5) \lor (\exists x \in \mathbb{R} \centerdot x^2 < 0)   \qquad &\text{假}
    \end{align*}
\end{example}

\subsubsection*{符号}

关于符号的说明与上一小节类似:括号(如上例所示)虽非必需,但有助于理解,建议合理使用。

逻辑连词``$\lor$''与集合运算符``$\cup$''存在关联,这并非巧合。读者可以尝试用``$\lor$''重写``$\cup$''的定义,并参考第 \ref{sec:section4.5.4} 节。但需严格区分二者:若 $A$ 和 $B$ 是集合,``$A \lor B$''无明确定义,正确形式应该是``$A \cup B$''。
