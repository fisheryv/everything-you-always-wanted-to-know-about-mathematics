% !TeX root = ../../../book.tex
\section{逻辑连词}

为了从简单陈述(即仅由量词和命题构成的基本数学陈述)构建复杂数学陈述,我们可以用特定词语和短语(如``与''、``或''和``蕴涵'')连接多个陈述,从而形成更复杂的结构并表达更深层的含义。这些词语和短语称为\textbf{逻辑连词},它们分别对应特定的数学符号与逻辑含义。尽管这些含义基于语言和思维的直觉显得自然直观,但我们强调:引入数理逻辑及其符号的核心目标之一,正是将这些直觉固化为严格而清晰的形式化概念。

本节中,设 $P$ 和 $Q$ 为任意数学陈述。它们本身可能包含量词、其他连词及数学概念的组合。关键在于,$P$ 和 $Q$ 构成更大陈述的方式独立于其内部结构。例如,已知``$\neg(\forall x \in X \centerdot R(x))$''等价于``$\exists x \in X \centerdot \neg R(x)$'',此等价性成立与 $R(x)$ 的具体内容或复杂度无关。本节延续这一思想:讨论陈述的组合方式时,无需知晓其具体内容。

需特别说明,$P$ 和 $Q$ 可能是\textbf{变量命题}。例如,我们将研究如何连接两个依赖变量 $x$ 的命题 $P(x)$ 和 $Q(x)$。即使此类命题在未指定 $x$ 时无确定真值,本节定义的逻辑连词仍适用于它们。

当需要严谨讨论变量命题时,必须\textbf{量化}其中的变量。因此对于 $P(x)$ 和 $Q(x)$,我们可定义 $P(x) \land Q(x)$($\land$ 表示``逻辑与'',详见后文)。进而可以讨论如下形式的命题:
\[\exists x \in X \centerdot P(x) \land Q(x)\]
该式是一个数学\textbf{陈述}。

本质上,逻辑连词可以直接作用于变量命题,但必须最终对相关变量进行量化,方能使整体表达式成为有真值的\textbf{数学陈述}。

% !TeX root = ../../../book.tex
\subsection{逻辑与}

说``$P$ 和 $Q$''为\verb|真|,意味着 $P$ 和 $Q$ 均为\verb|真|。若 $P$ 或 $Q$ 中有一个为\verb|假|,则``$P$ 和 $Q$''也为\verb|假|。其定义如下:

\begin{definition}
    在数学陈述间使用符号``$\land$''表示``和''。例如,``$P \land Q$''读作``$P$ 和 $Q$''。

    这称为 $P$ 和 $Q$ 的合取。

    当 $P$ 和 $Q$ 均为真时,``$P \land Q$''的真值为\verb|真|,否则为\verb|假|。
\end{definition}

以下是此定义的一些示例:

\begin{example}
    \begin{align*}
        (1 + 3 = 4) \land (\forall x \in \mathbb{R} \centerdot x^2 \ge 0) \qquad &\text{真} \\
        (1 + 3 = 5) \land (\forall x \in \mathbb{R} \centerdot x^2 \ge 0) \qquad &\text{假} \\
        (1 + 3 = 5) \land (\exists x \in \mathbb{Q} \centerdot x^2 = 2)   \qquad &\text{假}
    \end{align*}
\end{example}

\subsubsection*{符号:括号}

上面示例中的括号常常可以省略。例如,第一行可等价写作:
\[1 + 3 = 4 \land \forall x \in \mathbb{R} \centerdot x^2 \ge 0\]
使用括号能增强可读性。无括号时虽然仍可理解,但需要额外时间划分语句边界。建议在括号能提升清晰度时使用。

\subsubsection*{符号:集合与逻辑}

你可能会注意到逻辑连词``$\land$''与集合运算符``$\cap$''之间的相似性。这并非巧合!

正如 \ref{sec:section4.5.4} 节所述,``$\cap$''的定义可以基于逻辑用``$\land$''来表述。读者可先行尝试,再参阅该节。需要特别注意的是,若 $A$ 和 $B$ 是集合,则 ``$A \land B$''无明确定义,正确形式应该是``$A \cap B$''。


% !TeX root = ../../../book.tex
\subsection{逻辑或}

说``$P$ 或 $Q$'' 为\verb|真|,意味着``$P$ 为\verb|真|,或 $Q$ 为\verb|真|''。只要其中一个陈述为\verb|真|,整个陈述就为\verb|真|。我们不关心 $P$ 和 $Q$ 是否均为\verb|真|,只要求\emph{至少一个}为\verb|真|。

这与计算机科学中的``异或''(\verb|XOR|) 不同:当 $P$ 和 $Q$ 均为\verb|真|时,``$P$ \verb|XOR| $Q$''为\verb|假|。数学中采用\textbf{兼或} (inclusive or),只需至少一个陈述成立。

\begin{definition}
    在数学陈述间使用符号``$\lor$''表示``或''。例如,``$P \lor Q$''读作``$P$ 或 $Q$''。

    这称为 $P$ 和 $Q$ 的析取。
    
    当 $P$ 和 $Q$ 中至少一个为\verb|真|时(或者两者都为\verb|真|),``$P \lor Q$''的真值为\verb|真|,否则为\verb|假|。
\end{definition}

\begin{example}
    \begin{align*}
        (1 + 3 = 4) \lor (\forall x \in \mathbb{R} \centerdot x^2 \ge 0) \qquad &\text{真} \\
        (1 + 3 = 5) \lor (\forall x \in \mathbb{R} \centerdot x^2 \ge 0) \qquad &\text{真} \\
        (1 + 3 = 5) \lor (\exists x \in \mathbb{R} \centerdot x^2 < 0)   \qquad &\text{假}
    \end{align*}
\end{example}

\subsubsection*{符号}

关于符号的说明与上一小节类似:括号(如上例所示)虽非必需,但有助于理解,建议合理使用。

逻辑连词``$\lor$''与集合运算符``$\cup$''存在关联,这并非巧合。读者可以尝试用``$\lor$''重写``$\cup$''的定义,并参考第 \ref{sec:section4.5.4} 节。但需严格区分二者:若 $A$ 和 $B$ 是集合,``$A \lor B$''无明确定义,正确形式应该是``$A \cup B$''。


% !TeX root = ../../../book.tex
\subsection{条件陈述}\label{sec:section4.5.3}

这是最难使用的逻辑连词,经常引起问题,因此我们需要格外小心和明确。当 $Q$ 的真值\emph{必然}从 $P$ 的真值推导出来时,我们期望陈述``\textbf{如果} $P$,\textbf{则} $Q$''(或``$P$ 蕴含 $Q$'')为\verb|真|。也就是说,该陈述为\verb|真|当且仅当以下条件成立:

\begin{center}
    每当 $P$ 为\verb|真|时,$Q$ 也为\verb|真|。
\end{center}

\subsubsection*{真值表与定义}

由于这是语义上最难理解的连词,我们引入\textbf{真值表}的概念以便阐明:
\begin{center}
    \begin{tabular}{c|c|c|c|c|c|c}
          $P$      & $Q$      & $\neg P$ & $P \land Q$ &  $P \lor Q$ & $P \implies Q$ & $Q \implies P$\\
          \hline
          \verb|T| & \verb|T| & \verb|F| &   \verb|T|  &  \verb|T|   &    \verb|T|    & \verb|T|\\
          \verb|T| & \verb|F| & \verb|F| &   \verb|F|  &  \verb|T|   &    \verb|F|    & \verb|T|\\
          \verb|F| & \verb|T| & \verb|T| &   \verb|F|  &  \verb|T|   &    \verb|T|    & \verb|F|\\
          \verb|F| & \verb|F| & \verb|T| &   \verb|F|  &  \verb|F|   &    \verb|T|    & \verb|T|\\
    \end{tabular}
\end{center}
你或许在其他数学课程中见过真值表,但即使没有也不必担心。其主要思想是:每一列对应一个特定数学陈述及其真值,每行对应\emph{分配}给原子陈述 $P$ 和 $Q$ 的特定真值。

该真值表有 $4$ 行,因为 $P$ 和 $Q$ 各自可独立取真或假,共有 $4$ 种组合。查看特定行时,我们根据前两列中 $P$ 和 $Q$ 的真值确定其他陈述的真值。

请注意,$P \land Q$ 和 $P \lor Q$ 列严格遵循定义。$P \land Q$ 列仅有一个 \verb|T|,对应 $P$ 和 $Q$ \emph{同时}为\verb|真|的情况;其余情况均为\verb|假|。类似地,$P \lor Q$ 列仅有一个 \verb|F|,对应 $P$ 和 $Q$ \emph{同时}为\verb|假|的情况;其余情况均为\verb|真|。

为什么最后两列的真值如此设定?假设我声称``如果你努力学习,那么你将在这门课程中获得 A''。其中 $P$ 是``你努力学习'',$Q$ 是``你会得到 A''。你何时会指责我是\emph{骗子}?何时会认为我说了真话?显然,如果你努力学习并获得了 A,我的陈述为真。相反,如果你努力学习却未获得 A,我的陈述为假。但如果你未努力学习,无论结果如何,你都\emph{无法宣称我是骗子}!我的声明未涵盖不努力的情况;它假设你会努力学习。因此,我没有说谎,根据排中律,我的陈述为真。

真值表中 $P \implies Q$ 列的第三行和第四行(即 $P$ 为假时)为\verb|真|,这称为\textbf{错误假设}。当``$\implies$''左侧的陈述不成立时,该主张不适用,故不能断言其为\verb|假|。根据排中律,它必然为\verb|真|。

这种情况($P \implies Q$ 列的第三行和第四行为\verb|真|)被称为\textbf{错误假设}。当 ``$\implies$'' 左边的陈述不成立时,我们不在该主张的讨论范围内,因此我们不能断言该主张为\verb|假|。因此,该主张必然为\verb|真|(同样,根据排中间律)。

下面正式定义该符号,并通过更多示例阐释:

\begin{definition}
    在数学陈述间使用符号``$\implies$''表示数学陈述间的``如果……那么''或``蕴涵''关系。例如,``$P \implies Q$'' 读作``如果 $P$,则 $Q$''或``$P$ 蕴含 $Q$''。

    这称为\dotuline{条件语句}。

    假设每当 $P$ 成立时 $Q$ 也成立,``$P \implies Q$''的真值为\verb|真|。

    仅当 $P$ 为\verb|真| 而 $Q$ 为\verb|假| 时,真值才为\verb|假|。

    我们称  $P$ 为条件陈述的\dotuline{假设},$Q$ 为\dotuline{结论}。
\end{definition}

定义中的关键词``每当''解释了\emph{错误假设}的合理性。当 $P$ 为真并且可以推出 $Q$ 为真时,则 $P \implies Q$ 为真。若 $P$ 为假,则不能断言 $P \implies Q$ 为假。仅当存在 $P$ 为真而 $Q$ 为假的情况时(即 $Q$ 不必然从 $P$ 推出),$P \implies Q$ 才为假。

\subsubsection*{示例}

以下示例有助于理解这一概念:
\begin{align*}
    (1 + 3 = 4) \implies (\forall x \in \mathbb{R} \centerdot x^2 \ge 0)  \qquad &\text{真} \\
    (1 + 3 = 5) \implies (\forall x \in \mathbb{R} \centerdot x^2 \ge 0)  \qquad &\text{真} \\
    (1 + 3 = 5) \implies (\text{亚伯拉罕·林肯还活着})  \qquad &\text{真} \\
    (1 + 1 = 2) \implies (0 = 1)  \qquad &\text{假} \\
    (0 = 0) \implies (\exists x \in \mathbb{R} \centerdot x^2 < 0)  \qquad &\text{假} \\
    (\text{毕达哥拉斯定理}) \implies (1 = 1)  \qquad &\text{真} \\
    (0 = 1) \implies (1 = 1)  \qquad &\text{真}
\end{align*}
请注意,第二和第三个示例为\verb|真|,因为其假设``$1 + 3 = 5$''为\verb|假|。无论结论如何,整个条件陈述必然为\verb|真|。``$\forall x \in \mathbb{R} \centerdot x^2 \ge 0$''恰好为\verb|真|或``亚伯拉罕·林肯还活着''恰好为\verb|假|并不重要;错误的假设决定了陈述的真值必为\verb|真|。

此外,倒数第二个例子虽然为\verb|真|,却无法证明毕达哥拉斯定理本身为\verb|真|!这正是第一章中对该定理的错误``证明''所犯的错误。请回顾第 \ref{sec:section1.1.1} 节,尤其是``证明 2''。我们假设毕达哥拉斯定理成立,并从中逻辑推导出为\verb|真|的陈述。这仅表明推理有效,并不证明假设成立!

鉴于该思想的重要性,我们立即展示另一个典型谬误。请注意其逻辑形式与其他错误证明完全一致:

\begin{spoof}
    假设 $1 = 0$。那么,根据 $=$ 的对称性,$0 = 1$ 同样成立。将这两个方程相加可得 $1 = 1$,此为\verb|真|。因此,$0 = 1$ 成立。
\end{spoof}
这里的核心要点如下:
\begin{center}
    条件陈述为\verb|真|本身,并不传递其成分命题真值的\emph{任何}信息。
\end{center}
上述第三和第七个陈述同样印证这一点:两个条件陈述皆为\verb|真|,但我们显然不能得出亚伯拉罕·林肯尚在人世或 $0 = 1$ 的结论。

\subsubsection*{``蕴涵''与``可推导''的区分}

使用``蕴涵''表述``如果……那么……''这样的条件陈述常引发混淆。我们认为这源于``蕴涵''一词隐含的\emph{因果关联}。例如,考虑以下陈述:
\[1 + 3 = 4 \implies 2 + 3 = 5\]
该条件陈述为\verb|真|,我们可能理解为:将假设 $1 + 3 = 4$ 两边同时加 $1$,即可得到结论中的等式。此时假设的真实性似乎直接影响结论的真实性——两者存在\emph{直接}推导关系。但此关系并非必然成立!

回顾上面给出的第一个例子:
\[(1 + 3 = 4) \implies (\forall x \in \mathbb{R} \centerdot x^2 \ge 0)\]
$1+3=4$ 与``任意实数平方非负''有何逻辑关联?实际上两者并无必然联系!无论能否从假设直接推导结论,该条件陈述仍为\verb|真|。唯一决定因素是其成分陈述的真值。

诚然,证明条件陈述时我们往往会尝试直接推导,但请务必记住:这是证明策略的选择,而非条件陈述的本质要求。因此,我们倾向于采用``如果……那么……''而非``蕴涵''来书写条件陈述。虽然其他数学文献可能使用该词,但在学习逻辑陈述与连接词阶段,我们将尽量避免。

\subsubsection*{量化变量:同样重要!}

在数学中,我们经常需要证明涉及变量的条件陈述。例如,在实数集 $\mathbb{R}$ 的背景下,我们可能希望证明:
\[x > 1 \implies x^2 - 1 > 0\]
这本身就是一个\textbf{变量命题},其中符号``$\implies$''的定义适用于此类命题。

若已知 $x > 1$ 且 $x^2 - 1 > 0$,则可断言该条件陈述为\verb|真|。若已知 $x \le 1$,则无论 $x^2 - 1 > 0$ 是否成立,该条件陈述均为\verb|真|——这正是``$\implies$''的定义。

但需注意,上述条件陈述在技术上并非完整的数学命题。由于我们在实数背景下讨论,其完整表述应为:
\[\forall x \in \mathbb{R} \centerdot (x > 1 \implies x^2 - 1 > 0)\]
这才是我们实际要表达的内容。逻辑连词($\land , \lor$ 和 $\implies$)可应用于变量命题;但若要在其他语境中组合语句,则必须对这些变量进行量化。唯有如此,才能确保该语句是具有确定真值的数学命题。

\subsubsection*{用``$\lor$''重写``$\implies$''}

有一个有用且重要的概念值得一提。一方面在于我们稍后会用到它,另一方面在于它有助于理解并应用条件陈述。

考虑条件陈述 $P \implies Q$:若 $P$ 不成立,则无论 $Q$ 的真值如何,该陈述恒为\verb|真|;但若 $P$ 成立,则必须 $Q$ 也成立,整个陈述才为\verb|真|。

由此可得条件陈述成立的两种情形,并可表述为析取形式:要么 $\neg P$ 成立(即前件为假),要么 $Q$ 成立。任一情形均保证 $P \implies Q$ 成立。这一观点可形式化表述为:

\begin{center}
    条件陈述``$P \implies Q$''与陈述``$\neg P \lor Q$''具有相同的真值。
\end{center}

这是\textbf{逻辑等价}的典型范例(下一节将详细讨论)。通过真值表可验证:无论 $P$ 和 $Q$ 的真值如何,二者真值始终保持一致。

\begin{center}
    \begin{tabular}{c|c|c|c|c}
          $P$      & $Q$      & $\neg P$ & $\neg P \lor Q$ & $P \implies Q$ \\
          \hline
          \verb|T| & \verb|T| & \verb|F| &     \verb|T|    &  \verb|T|   \\
          \verb|T| & \verb|F| & \verb|F| &     \verb|F|    &  \verb|F|   \\
          \verb|F| & \verb|T| & \verb|T| &     \verb|T|    &  \verb|T|   \\
          \verb|F| & \verb|F| & \verb|T| &     \verb|T|    &  \verb|T|   \\
    \end{tabular}
\end{center}

\subsubsection*{更多示例}

让我们看更多条件陈述的例子,并判断它们是否正确。这样做有助于你更好地理解 $\implies$ 的工作原理。然后,我们将转向证明\emph{策略},并讨论如何正式且严格地使用逻辑连词和量词来证明此类陈述。

\begin{example}
    我们将从一个``现实世界''的例子开始,以熟悉所涉及的逻辑。假设我们所在的班级只在周一、周三和周五安排正式讲座。你会注意到,我们将使用两个陈述 $P$ 和 $Q$,并考虑它们及其否定形式组合而成的四个条件陈述。
    \begin{itemize}
        \item ``如果今天有讲座,那么今天就是工作日。''
        
        (注:这句话中有一些\emph{隐性量化}。我们实际上是在说``对于周历中的所有自然日 $d$,如果我们在 $d$ 天有讲座,那么 $d$ 就是工作日。''但上面的简洁版本更好地表达了主要思想,因此我们使用它。请记住,这是句子的含义;在下面的讨论中,我们将考虑量化的不同情况。)

        这可以通过定义 $P$ 为``今天有讲座''、$Q$ 为``今天是工作日'',从而将陈述逻辑地写为 $P \implies Q$。

        该陈述为\verb|真|吗?注意,$P$ 和 $Q$ 并未指定具体日期,因此若断言它为真,则必须独立于当前日期成立。也就是说,无论今天是哪一天,$P \implies Q$ 都应成立。让我们通过一些情况来验证:
        \begin{itemize}[label=--]
            \item 假设今天是星期六或星期天。这些天没有讲课,因此该条件陈述为\verb|真|。
            \item 假设今天是星期一、星期三或星期五。今天有讲座且是工作日,因此该条件陈述为\verb|真|。
            \item 假设今天是星期二或星期四。今天通常没有讲座,但即使特殊情况下有讲座,今天仍是工作日,因此该条件陈述依然为\verb|真|。
        \end{itemize}
        在所有可能情况下,该陈述均成立,故 $P \implies Q$ 为\verb|真|。

        你可能会反驳:``为什么要费心处理所有这些情况呢?假设有讲座时能推断出是工作日,不就足够了吗?''是的,这完全正确!这实际上是一条更\emph{直接}的路径。

        这暗示了我们未来证明条件陈述的方法:由于无讲座的情况(\emph{错误假设})不影响结论,我们只需\emph{假设}某天有讲座,并\emph{推断}该天是工作日即可。这就是\textbf{直接证明}条件陈述的策略。

        \item ``如果今天是工作日,那么今天有讲座。''
        
            这在逻辑上可以写作 $Q \implies P$,沿用前述 $P$ 和 $Q$ 的定义。

            该陈述为\verb|真|吗?答案是否定的!例如,学期的第一个星期二无讲座但却是工作日。此时 $Q$ 为\verb|真|而 $P$ 为\verb|假|,故 $Q \implies P$ 为\verb|假|。

        \item ``如果今天不是工作日,那么今天就没有讲座。''
        
            这在逻辑上可以写作 $\neg Q \implies \neg P$,沿用前述 $P$ 和 $Q$ 的定义。

            该陈述为\verb|真|吗?是的!可以直接证明:假设今天不是工作日(即星期六或星期天)。大学通常不会在周末安排讲座,因此 $\neg P$ 成立。这表明 $\neg Q \implies \neg P$ 为\verb|真|。

            (问题:为什么我们不需要考虑今天是工作日的情况?)

        \item ``如果今天没有讲座,那么今天就不是工作日。''
        
            这在逻辑上可以写作 $\neg P \implies \neg Q$,沿用前述 $P$ 和 $Q$ 的定义。

            该陈述为\verb|真|吗?思考一下:假设今天没有讲座,能否推断今天不是工作日?并非如此!例如,星期二可能无讲座但仍是工作日。这提供了一个反例:$\neg P$ 成立时 $\neg Q$ 不成立。

            注意,某些情况下 $\neg P$ 和 $\neg Q$ 同时成立(如星期六无讲座且非工作日)。但\emph{单一实例}不能证明全称陈述成立——必须验证\emph{所有实例}的真实性。
    \end{itemize}
\end{example}

\begin{example}
    让我们用一个更``数学''的例子来做同样的分析。在整个示例中,令 $A$ 和 $B$ 为任意集合。另外,令 $P$ 为``$A \subseteq B$'',令 $Q$ 为``$A - B = \varnothing$''。我们将像前面示例中那样,考虑 $P$ 和 $Q$ 及其否定形式组合而成的所有四种可能的条件陈述。
    \begin{itemize}
        \item $P \implies Q$ 为\verb|真|吗?

            答案是肯定的! 假设 $A$ 和 $B$ 满足 $A \subseteq B$。这意味着 $A$ 的每个元素都属于 $B$,因此不存在属于 $A$ 但不属于 $B$ 的元素。由于 $A - B$ 是那些属于 $A$ 而不属于 $B$ 的元素的集合,故没有这样的元素,从而 $A - B = \varnothing$。

        \item $Q \implies P$ 为\verb|真|吗?

            答案是肯定的!假设 $A - B = \varnothing$。这意味着 $A$ 中所有元素都属于 $B$(仔细思考一下)。换句话说,对于任意元素 $x \in A$,$x$ 不可能不属于 $B$(因为若 $x \in A - B$,则与 $A - B = \varnothing$ 矛盾);因此必然有 $x \in B$。这正是 $A \subseteq B$ 的定义:若 $x \in A$,则 $x \in B$。这就证明了 $Q \implies P$ 成立。

        \item $\neg Q \implies \neg P$ 为\verb|真|吗?

            这个可能较难理解。假设 $\neg Q$ 成立,即 $A - B \ne \varnothing$。也就是说,存在某个元素 $x$ 满足 $x \in A$ 且 $x \notin B$。那么 $A \nsubseteq B$,因为 $\subseteq$ 要求 $A$ 的每个元素都属于 $B$,但我们找到了反例。因此,$\neg Q \implies \neg P$ 为\verb|真|。

        \item $\neg P \implies \neg Q$ 为\verb|真|吗?
            
            同理,$\neg P$ 表示 $A \nsubseteq B$,即存在某个元素 $x \in A$ 且 $x \notin B$(这与前面情况相同)。那么,这说明了什么?考虑集合 $A - B$:由于 $x \in A$ 且 $x \notin B$,我们有 $x \in A - B$,因此 $A - B \ne \varnothing$,即 $\neg Q$ 成立。故 $\neg P \implies \neg Q$ 为\verb|真|。
    \end{itemize}
\end{example}

\subsubsection*{关于 ``$\implies$'' 的观察和事实}

前面我们练习了如何判断条件陈述的真值。从讨论的示例中可以看出,仅知道 $P \implies Q$ 成立,并\textbf{不能}推出 $Q \implies P$ 的真假。在之前的例子中,虽然 $P \implies Q$ 为\verb|真|,但 $Q \implies P$ 在一个例子中为\verb|真|,在另一个例子中为\verb|假|。即使已知 $P \implies Q$ 的真值,我们也无法确定 $Q \implies P$ 的真值。这一点非常重要,我们将在下一小节深入讨论。

现在,我们再补充几点关于``$\implies$''的说明。

\begin{itemize}
    \item 请记住,对于数学陈述 $P$ 和 $Q$,语句``$P \implies Q$''本身也是一个完整的数学陈述,具有确定的真值。该真值取决于 $P$ 和 $Q$(按前述规则定义),但并未直接揭示 $P$ 或 $Q$ 的真假。因此,若仅写下:
    \[\text{Blah blah} \implies \text{Yada yada}\]
    我们无从判断你是否断言``Blah blah''或``Yada yada''的真假。对数学家而言,这仅表示:
    \begin{center}
        条件陈述``Blah blah 蕴涵 Yada yada''为\verb|真|。
    \end{center}
    若要进行推理或演绎,则需通过辅助语句明确表达,例如:
    \begin{center}
        $P \implies Q$ 因为 ……

        且 $P$ 成立,因为 ……

        故 $Q$ 成立。
    \end{center}
    如果你以前学过形式逻辑,或在哲学课上见过此类的论证,会认出这种论证称为\textbf{分离规则}。\label{sec:section4.5.6}
    \item \textbf{错误假设}可推出任意结论这一性质可能显得反直觉。这是排中律的直接结果:当 $P$ 为\verb|假|时,整个蕴涵式不能为\verb|假|,故必为\verb|真|。
    \item 请记住,条件陈述可以等价转化为``或''形式:
        \begin{center}
            条件陈述``$P \implies Q$''与``$\neg P \lor Q$''具有相同的真值。
        \end{center}
\end{itemize}

\subsubsection*{逆命题与逆否命题}

让我们给条件陈述的相关变体命名。这些术语将在后续讨论中频繁使用。

\begin{definition}
    设 $P$ 和 $Q$ 为数学陈述。考虑原命题 $P \implies Q$。

    称 $Q \implies P$ 为原命题的\dotuline{逆命题}。

    称 $\neg Q \implies \neg P$ 为原命题的\dotuline{逆否命题}。
\end{definition}

根据前文分析,我们知道\textbf{逆命题}\emph{不一定}具有与原命题相同的真值。下一节将证明:\textbf{逆否命题}总是与原命题保持\emph{相同}的真值(这就是\textbf{逻辑等价}的概念,后文将详细讨论)。

这些术语之所以重要,是因为逆否命题与原命题的\emph{逻辑等价性}提供了一种有效的证明方法——我们很快就会学习这种方法。

逆命题的独特之处在于其真值与原命题无关:即使原命题为\verb|真|,逆命题也可能为\verb|真|或\verb|假|。因此,当证明 $P \implies Q$ 为\verb|真|时,数学家总会自然地追问:``反之是否成立?''这是面对条件陈述时值得思考的问题(事实上,如果你在一个数学家聚会上,当听到有人谈论``如果……那么……''这样的陈述时,问一句``反之是否成立?''会给大家留下深刻印象)。

逆命题也是日常生活中常见的一类逻辑谬误。假设你与朋友辩论 $A \implies B$,对方反驳:``可 $B$ 不蕴含 $A$!你的观点是错误的!''此时你或许会愤慨:``那又如何?我从未主张 $B \implies A$ 成立。我讨论的是 $A \implies B$,你......''(为避免失礼,我们在此打住。)无论对方观点正确与否,逆命题的真值无法推断原命题的真值。请坚定地指出:``你讨论的是逆命题,这与我的主张逻辑上没有必然关联。''

至此,我们已经定义了所有基础逻辑符号并分析了一些案例,是时候更进一步在证明中应用它们了!在此之前,需要简要介绍一下集合运算并完成若干练习。


% !TeX root = ../../../book.tex
\subsection{回顾:集合运算与逻辑连词}\label{sec:section4.5.4}

回顾 \ref{sec:section3.4} 和 \ref{sec:section3.5} 节,我们定义了子集与集合运算。这些定义当时借助了逻辑思想,但以自然语言表述,依赖于集合直觉和逻辑知识。现在我们可以用量词和连词重新表述它们!

首先回顾\textbf{子集}定义:若满足条件``当 $x \in A$ 时,必有 $x \in B$'',则记作 $A \subseteq B$。注意关键词``当……时'',它同时体现\emph{全称量化}和\emph{条件陈述}。请尝试用量词重写该定义,再对照我们的版本:

\begin{definition}
    设 $A, B, U$ 为集合,其中 $A, B \subseteq U$(即 $U$ 为全集)。称 $A$ 是 $B$ 的\dotuline{子集},记作 $A \subseteq B$,当且仅当
    \[\forall x \in U \centerdot x \in A \implies x \in B \]
\end{definition}

此定义合理对应了``当 $x \in A$ 时,必有 $x \in B$''的原始陈述:它确保如果 $x \in A$ 成立,则 $x \in B$ 必然成立。

接下来回顾集合运算的定义。请尝试用逻辑符号自行重写这些定义,再与我们的版本比较,思考其合理性与等价性。

\begin{definition}
    设 $A, B, U$ 为集合,其中 $A, B \subseteq U$(即 $U$ 为全集)。则
    \begin{align*}
        A \cap B &= \{x \in U \mid x \in A \land x \in B\} \\
        A \cup B &= \{x \in U \mid x \in A \lor x \in B\} \\
        A - B &= \{x \in U \mid x \in A \land \neg (x \in B)\} = \{x\in U \mid x \in A \land x \notin B\} \\
        \overline{A} &= \{x \in U \mid \neg (x \in A)\} = \{x \in U \mid x \notin A\}
    \end{align*}
\end{definition}

我们还可以重新定义集合的划分。这将用到逻辑连词,并涉及索引集及其量化定义——此前所学知识在此汇聚!

\begin{definition} \label{def:definition4.5.11}
    设 $A$ 为集合。$A$ 的\dotuline{划分}是其非空子集构成的集合,满足两两不相交且并集为 $A$。

    也就是说,划分由满足以下条件的索引集 $I$ 和非空集 $S_i$(定义在每一个 $i \in I$ 上)构成:
    \begin{enumerate}[label=(\arabic*)]
        \item $\forall i \in I \centerdot S_i \subseteq A$
        \item $\forall i, j \in I \centerdot i \ne j \implies S_i \cap S_j = \varnothing$
        \item $\displaystyle{\bigcup_{i \in I} S_i = A}$
    \end{enumerate}
\end{definition}

参考定义 \ref{def:definition3.6.9} 的原始表述,能否看出我们如何用逻辑符号表达同一概念?


% !TeX root = ../../../book.tex
\subsection{习题}

\subsubsection*{温故知新}

以口头或书面的形式简要回答以下问题。这些问题全都基于你刚刚阅读的内容,所以如果忘记了具体的定义、概念或示例,可以回去重读相关部分。确保在继续学习之前能够自信地回答这些问题,这将有助于你的理解和记忆!

\begin{enumerate}[label=(\arabic*)]
    \item $\land$ 和 $\lor$ 有什么区别?
    \item $\land$ 和 $\cap$ 有什么区别?\\
        $\lor$ 和 $\cup$ 有什么区别?
    \item 写出陈述 $P \implies Q$ 的真值表。
    \item 为什么 $P \implies Q$ 和 $\neg P \lor Q$ 是逻辑等价的?
    \item 条件陈述的逆命题是什么?\\
        条件陈述的逆否命题是什么?
    \item 条件陈述的真值与其逆命题是否相关?
\end{enumerate}

\subsubsection*{小试牛刀}

尝试回答以下问题。这些题目要求你实际动笔写下答案,或(对朋友/同学)口头陈述答案。目的是帮助你练习使用新的概念、定义和符号。题目都比较简单,确保能够解决这些问题将对你大有帮助!

\begin{enumerate}[label=(\arabic*)]
    \item 对于以下每个句子,使用逻辑符号重写并确定其为\verb|真|还是\verb|假|。
        \begin{enumerate}[label=(\alph*)]
            \item 每个整数要么严格为正,要么严格为负。
            \item 对于任意给定的实数,都存在一个严格大于它的自然数。
            \item 对于每个实数,如果它是负数,那么它的立方也是负数。
            \item $\mathbb{Z}$ 的一个子集具有以下性质:每当一个数字是该集合的元素时,它的平方也是该集合的元素。
            \item 存在一个既是偶数又是奇数的自然数。
        \end{enumerate}
    \item 将以下每个 $\forall$ 命题重写为条件陈述,并确定它为\verb|真|还是\verb|假|。
        \begin{enumerate}[label=(\alph*)]
            \item $\forall x \in \{y \in \mathbb{N} \mid \exists k \in \mathbb{Z} \centerdot y = 2k\} \centerdot x^2 \in \{y \in \mathbb{N} \mid \exists k \in \mathbb{Z} \centerdot y = 2k\}$
            \item $\forall x \in \{y \in \mathbb{N} \mid \exists k \in \mathbb{Z} \centerdot y = 2k + 1\} \centerdot x^2 \in \{y \in \mathbb{N} \mid \exists k \in \mathbb{Z} \centerdot y = 2k + 1\}$
            \item $\forall t \in \{z \in \mathbb{R} \mid z^2 > 4\} \centerdot t > 2$
        \end{enumerate}
    \item 使用集合构建符将以下每个条件陈述重写为 $\forall$ 声明,并确定其为\verb|真|还是\verb|假|。
        \begin{enumerate}[label=(\alph*)]
            \item $\forall x \in \mathbb{R} \centerdot x > 3 \implies x^2 < 9$
            \item $\forall x \in \mathbb{R} \centerdot x < 3 \implies x^2 < 9$
            \item $\forall t \in \mathbb{R} \centerdot t^2 - 6t + 9 \ge 0 \implies t \ge 3$
        \end{enumerate}
    \item 们定义以下变量命题:
        \begin{align*}
            P(x) &= \frac{1}{2} < x \\
            Q(x) &= x < \frac{3}{2} \\
            R(x) &= x^2 = 4 \\
            S(x) &= x + 1 \in \mathbb{N} 
        \end{align*}
        对于以下每个陈述,确定它为\verb|真|还是\verb|假|。
        \begin{enumerate}[label=(\alph*)]
            \item $\forall x \in \mathbb{N} \centerdot P(x)$
            \item $\forall x \in \mathbb{N} \centerdot Q(x) \implies P(x)$
            \item $\forall x \in \mathbb{Z} \centerdot Q(x) \implies P(x)$
            \item $\exists x \in \mathbb{N} \centerdot \neg S(x) \lor R(x)$
            \item $\exists x \in \mathbb{Z} \centerdot R(x) \land \neg S(x)$
            \item $\forall x \in \mathbb{R} \centerdot R(x) \implies S(x)$
            \item $\exists x \in \mathbb{R} \centerdot P(x) \land S(x)$
            \item $\forall x \in \mathbb{Z} \centerdot R(x) \implies \big(P(x) \lor Q(x)\big)$
        \end{enumerate}
    \item 对于以下每个条件陈述,用逻辑符号重写它,并写出它的逆命题和逆否命题;然后,确定所有三个命题的真值:原命题、逆命题和逆否命题。
        \begin{enumerate}[label=(\alph*)]
            \item 如果一个实数严格介于 $0$ 和 $1$ 之间,那么它的平方也是如此。
            \item 如果一个自然数是偶数,那么它的立方也是偶数。
            \item 每当一个整数是 $10$ 的倍数时,它也是 $5$ 的倍数。
        \end{enumerate}
\end{enumerate}