% !TeX root = ../../book.tex
\section{数学陈述}

我们首先要做的是讨论哪些类型的句子可以合理地视为需要证明或证伪的数学真理。完成这一步其实是相当困难的!许多作者倾向于忽略这个主题或仅提供一个简单的定义,而忽略了数学语言和逻辑的许多微妙之处。我们也感到束手无策,因为本书/课程提供的时间和空间不足以正确研究抽象逻辑理论领域。我们鼓励你阅读一些包含相关信息的书籍或网站。就目前的情况而言,我们必须隐藏许多细节。不过,换句话说,数学研究有一个非常深入、丰富和富有成果的领域,这正是我们将在本章以更具启发性的方式讨论的内容。

请记住,我们提到过我们必须假设实数 $\mathbb{R}$ 存在及其常规算术属性。同样,我们会假设数理逻辑的许多结果和概念,这些你通常甚至都没有意识到(直到我们给你指出)。这些细节可以在你以后的数学生涯中更深入地研究。

\subsection{定义}

现在,让我们讨论一下\textbf{数学陈述}的含义。我们希望这个术语能够概括我们可以证明或证伪的``事物''类型。

数学在科学中是独一无二的,因为该领域的结果经过严格\textbf{证明},而不是先做出假设然后通过实验室实验或现实世界观察来``证实''。在数学中,我们假设一组常见的\textbf{公理},然后遵循严格的逻辑推理,从这些公理(以及迄今为止我们已经证明的其他真理)中推导出真理。如果我们遇到虚假信息,我们就必须证明或揭示它确实为假。

基于上述想法,我们来看几个\textbf{数学陈述}或\textbf{命题}可能是什么的例子。(我们甚至已经证明了其中一些!)例如,这句话:
\begin{center}
    \textcolor{olivegreen}{对于任意实数 $x, y \in \mathbb{R}$,不等式 $2xy \le x^2 + y^2$ 成立。}
\end{center}
是一个有效的数学陈述。事实上,它是真的,我们将在稍后的 \ref{sec:section4.9.2} 节中证明这一点。(它有时被称为 AGM 不等式,是\emph{算术几何平均不等式(Arithmetic-Geometric Mean Inequality)}的缩写。)这里需要指出,``成立''一词在数学中经常用于表示``为真''或``是一个真实的陈述''。

下面是另一个数学陈述的例子:
\begin{center}
    \textcolor{olivegreen}{对于任意集合 $S, T, U$,如果 $S \cap T \subseteq U$ 则 $S \subseteq U$ 或 $T \subseteq U$。}
\end{center}

然而,上面这种说法是错误的,如以下反例所示:
\begin{center}
    \textcolor{blue}{设 $S = \{1, 2, 3\}$, $T = \{2, 3, 4\}$, $U = \{2, 3, 5\}$。\\ 很明显 $S \cap T = \{2, 3\} \subseteq U$ 但 $S \nsubseteq U$ 且 $T \nsubseteq U$。}
\end{center}
为什么这个例子证伪了上面的说法?你弄明白了吗?你能解释一下吗?我们将在本章后面更详细地讨论这一点,但我们希望现在我们都认识到这个例子确实实现了这一点。

我们也都认同像这样的句子
\begin{center}
    \textcolor{red}{为什么我们早上 9:00 还要上课?!}
\end{center}
绝对\textcolor{red}{\emph{不是}}数学陈述。这是一个完全有效的句子,但从数学上来讲它没有意义:我们无法\emph{证明}或\emph{证伪}它。

同理,这句话
\begin{center}
    \textcolor{red}{$x^2 - 1 = 0$}
\end{center}
尽管完全由数学符号组成,但它也\textcolor{red}{\emph{不是}}数学陈述。问题是我们无法纯粹从公理和逻辑推论来验证它为真还是为假。该陈述\emph{取决于} $x$,无论该值是什么(即 $x$ 是一个\textbf{变量}),并且在不对其施加额外假设的情况下,我们无法断言该句子为真还是为假。这种类型的句子稍后将被称为\textbf{变量命题}:其真实性取决于句子中的变量。

所有这些观察结果和示例/伪例引出了以下定义:

\begin{definition}
    \dotuline{数学陈述}(或\dotuline{命题}或\dotuline{逻辑陈述})是语法正确的句子(或句子串),由单词/标点符号和数学符号组成,并具有唯一一个真值(真或假)。
\end{definition}

\subsection{示例和伪例}

``语法正确''是指句子中的单词和符号被正确使用和组合并且有意义。这消除了放在一起无意义的符号/单词串,例如:
\begin{center}
    \textcolor{red}{$1+ = 2$, $\text{于儿}^2 = 1$, $\{\{\varnothing\}\} - 7 > 5\pi$, $\text{You am smart}$}
\end{center}
比如,上面的第三个不是数学陈述,因为 $\{\{\varnothing\}\}$ 不是数字,所以我们不知道如何解释从该集合中``减去 $7$''。

而``\emph{唯一}一个真值''说的是该陈述应该要么为\verb|真|要么为\verb|假|,但肯定不能两者兼而有之,或者两者都不是或介于两者之间。这排除了上面``\textcolor{red}{$x^2 - 1 = 0$}''这样的陈述,因为它没有真值。(如果没有声明 $x$ 是什么,我们就无法决定真值是什么。)

\subsubsection*{真值未知}

我们给出的定义中一个奇怪/有趣/复杂的方面是,我们可能不知道给定陈述的真值,即使我们可以确定只有一个这样的值。作为说明,请考虑以下陈述:
\begin{center}
    \textcolor{olivegreen}{任何大于或等于 $4$ 的偶自然数都可以写成两个质数之和。}
\end{center}

这个说法是真是假?如果你能证明或证伪,那么数学界会很乐意看到它!上面的陈述被称为\href{https://baike.baidu.com/item/哥德巴赫猜想/72364}{哥德巴赫猜想},它是数学中一个非常著名的未解难题(我们希望只是暂时的!)。目前还没有人知道这一说法是对是错,但可以肯定的是,\emph{只有其中一个}真值适用。也就是说,这个陈述不可能既是\verb|真|又是\verb|假|,也不可能介于两者之间。要么所有大于或等于 $4$ 的偶自然数都具有该性质,要么至少有一个不具有该性质。即使还不知道两种可能性中哪一种是正确的,我们也可以陈述这种``非此即彼''的性质。因此,这句话实际上满足了我们对\emph{数学陈述}的定义。

(术语说明:一般来说,\textbf{猜想}是某人认为正确但尚未被证明/证伪的断言。)

\subsubsection*{悖论语句}

使句子没有真值的一种方法是创造\textbf{悖论}。考虑下面这句话:
\begin{center}
    \textcolor{red}{这句话是假的。}
\end{center}
很怪异,对吧?这句话本身就断言了它自己的真值。我们来尝试分析一下它的真值:
\begin{itemize}
    \item 假设这句话为\verb|真|。那么,这个句话本身告诉我们,它实际上为\verb|假|。
    \item 假设这句话为\verb|假|。那么同理,这句话告诉我们,它实际上为\verb|真|。
\end{itemize}
这是不可能的!这句话在某种程度上既\verb|真|又\verb|假|,或者两者都不是,或者……无论是什么,都不是个好主意。我们不想在数学中处理这种奇怪现象,所以我们的定义不允许这类语句作为数学陈述。

(\emph{问题}:如果允许这样的句子成为正确的数学陈述,会发生什么?如果你不遵守我们强调的每个句子必须为\verb|真|或为\verb|假|的原则怎么办?想一想!是不是哪里会出问题,或者这只是一个不同的数学宇宙?……)

通常来说,像上面这样的\textbf{自指}句子(即指代自己的句子)是相当奇怪的,并且可能会产生一些我们想要禁止的悖论。

上述自相矛盾的说法有一个变体,通过下面这幅卡通漫画中给出,其中皮诺曹说:``我的鼻子现在要变长!'' 那么它会变长吗?如果他说真话,那么它就会变长,但只有当他说谎时才它的鼻子才会变长!如果他说谎,那么他的鼻子就会变长(根据定义),但他的说法实际上又是真的!哎呀,纠缠不清!
\begin{center}
    \includegraphics[scale=0.4]{figure/pinocchio.png}
\end{center}

这种现象的一个更奇怪的例子是\emph{奎因悖论(Quine's Paradox)}:
\begin{center}
    \textcolor{red}{``Yields falsehood when preceded by its quotation'' \\yields falsehood when preceded by its quotation.}
\end{center}
这个问题留给你自己思考。我只想说,像这样自相矛盾的说法实在是太病态了,以至于我们不必考虑它们。这就是为什么我们的定义禁止了它们。

\subsection{变量命题}

另一类非数学陈述涉及\textbf{未量化变量}。例如,拿这句话来说:
\begin{center}
    ``\textcolor{red}{$x^2 - 1 = 0$}''
\end{center}
这在语法上当然是正确的,我们也能理解它,但它的真值是什么呢?我们不知道!如果 $x = 1$,则该句话为\verb|真|,但如果 $x = 8$,则为\verb|假|,如果 $x = \mathbb{N}$ 或 $x = \text{于儿}$,则该句话没有意义!因此,我们也希望禁止这样的语句。不过,这些类型的语句非常有用且常见。我们将它们称为\textbf{变量命题},因为它们提出的主张\emph{依赖于}某些变量。

对于上面的语句,我们可以将 $P(x)$ 定义为变量命题 ``$x^2 - 1 = 0$''。我们通常会将此声明写为
\begin{center}
    \textcolor{olivegreen}{令 $P(x)$ 为陈述 ``$x^2 - 1 = 0$''。}
\end{center}
通常用大写字母表示变量命题和数学陈述,用小写字母表示其中包含的变量。(但这不是一个要求,只是一种常见的约定。)

现在定义了这个变量命题,我们可以通过将特定值\emph{分配}给表达式中的变量 $x$ 来创建正确的数学陈述。我们可以说 $P(1)$ 为\verb|真|而 $P(0)$ 为\verb|假|。我们还可以对 $P(x)$ 进行\textbf{量化}。例如,下面语句是一个为\verb|真|的数学陈述:
\begin{center}
    存在 $x \in \mathbb{R}$ 使得命题 $P(x)$ 为\verb|真|。
\end{center}
而下面语句是一个为\verb|假|的数学陈述:
\begin{center}
    对于每个 $x \in \mathbb{R}$,命题 $P(x)$ 为\verb|真|。
\end{center}
想想为什么这些陈述具有我们所说的真值。你能明白为什么它们是数学陈述吗?你将如何证明这些说法?

\subsubsection*{定义变量命题}

请注意我们用来定义变量命题的格式,如上面的格式:
\begin{enumerate}[label=(\arabic*)]
    \item 我们给命题起一个字母名称(如 $P$);
    \item 我们表明它对一些变量的依赖性,每个变量都有一个字母名称(如 $x$ 和 $y$);
    \item 我们在实际命题本身周围加上引号;
    \item 我们不包含任何在命题上下文中没有意义的新字母。
\end{enumerate}

这种格式经过精心选择,因为它精确且明确。它为命题中的每个字母赋予含义,并清楚地区分命题中的内容和不包含的内容。

例如,以下是变量命题的\textcolor{red}{\textbf{糟糕}}``定义''。我们会指出为什么它们不好的理由,并给出进行适当修正的建议。
\begin{itemize}
    \item \textcolor{red}{令 $Q(y)$ 为命题``$x < 0$''。}\\
        \textbf{原因}:$x$ 是什么?$y$ 在哪里?我们不知道命题上下文中的 $x$ 是什么,所以这是一个糟糕的定义。\\
        修改为:
        \begin{center}
            \textcolor{olivegreen}{令 $Q(x)$ 为陈述 ``$x<0$''。}
        \end{center}
        就完美了。括号内的变量是后面引号中陈述使用的变量。太棒了。
    \item \textcolor{red}{对于每个 $x \in \mathbb{R}$,令 $P(x)$ 为命题 $x^2 \ge 0$。}\\
        \textbf{原因}:这句话的作者是否想断言无论 $x \in \mathbb{R}$ 是什么 $x^2 \ge 0$? ``对于每个 $x \in \mathbb{R}$'' 这句话是否意味着是命题的一部分?\\
        如果我们将其解释为 $P(x)$ 被定义为 ``$x^2 \ge 0$'',并且这个定义是针对每个 $x \in \mathbb{R}$ 进行的,那么……好吧,这可能是合理的。\\
        然而,如果我们将其解释为 $P(x)$ 被定义为 ``对于每个 $x \in \mathbb{R}$, $x^2 \ge 0$'',那么……嗯,这肯定是不同的。事实上,这甚至不是一个正确定义的命题!命题 $P(x)$ 应取决于输入值 $x$,但不应允许更改或进一步量化命题内的变量!\\
        这个命题最初的写法有两种可能的解释,而且它们非常不同。因此,这是一个糟糕的定义。\\
        如果我们修改为:
        \begin{center}
            \textcolor{olivegreen}{对于每一个 $x \in \mathbb{R}$,定义 $P(x)$ 为陈述 ``$x^2 \ge 0$''。}
        \end{center}
        就会好很多。正如我们下面提到的,从技术上讲,我们不必告诉读者我们想要定义命题的 $x$ 值。不过,也许这只是一些有用的信息,所以写出来并没有什么坏处。
    \item \textcolor{red}{令 $T(x, y)$ = ``$x^2 - 7 = y$''。}\\
        \textbf{原因}:在这种情况下 ``$=$'' 是什么意思?当我们希望比较两个数字并说它们的值相等(或两个集合并说它们的元素相等)时,该符号才适用。对象 $T(x, y)$ 是一个数学陈述,要么为\verb|真|,要么为\verb|假|。因此,它没有数值可以与其他任何东西进行比较。\\
        同理,给定 $x$ 和 $y$ 的值,语句 ``$x2 - 7 = y$'' 要么为\verb|真|,要么为\verb|假|,因此说该方程``等于''其他值是没有意义的。它具有真值,而不是数值。\\
        如果我们改为:
        \begin{center}
            \textcolor{olivegreen}{令 $T(x, y)$ 为 ``$x^2 - 7 = y$''。}
        \end{center}
        那就完美了。
\end{itemize}

我们已经做了足够多的错误示范了,不想把任何不好的想法灌输给你,真的!然而,根据过去的经验,我们知道这些是学生书写命题时常犯的错误(要么是无意的,要么没有意识到为什么他们是错的),所以我们觉得有必要分享出来。

关于变量命题最后有一点说明。定义命题时不必说明变量从何而来。可以稍后在调用命题时或者使用变量的特定值时填写。也就是说,我们可以定义
\begin{center}
    令 $T(x, y)$ 为 ``$x^2 - 7 = y$''。
\end{center}
而无需指定 $x$ 和 $y$ 到底是自然数、整数、复数还是其他类似数字。稍后,我们可以说 $T(3, 2)$ 为\verb|真|,$T(\pi, -1)$ 为\verb|假|,并且 $T(\varnothing, \mathbb{N})$ 没有意义,但在定义 $T(x, y)$ 时,我们不需要以某种方式预期任何这些解释。

\subsection{词序至关重要!}\label{sec:section4.2.4}

我们将在下一节中详细讨论\textbf{量化}变量的概念。现在,我们想考虑一个更加引人注目的数学陈述示例,它说明了语句中词序的重要性。分析如下语句的结构也将是下一节的主要目标。
\begin{center}
    存在一个实数 $y$,使得对于每个实数 $x$, $y = x^3$。
\end{center}
这个句话说了什么?它说的是无论 $x \in \mathbb{R}$ 是什么,我们都可以找到一个数字 $y \in \mathbb{R}$ 使得 $y = x^3$ 为\verb|真|。这是荒唐的!怎么可能有一个数字是所有数字的立方呢?这句话确实是一个数学陈述,但它绝对为\verb|假|。但下面的说法又如何呢?
\begin{center}
    对于每个实数 $x$,存在一个实数 $y$,使得 $y = x^3$。
\end{center}
上面这句话为\verb|真|!你看出两个句子之间的区别了吗?它们包含完全相同的单词和符号,但顺序不同。前一句断言存在某个数字是每个实数的立方(此为\verb|假|),而后一句断言每个实数都有一个立方根,此为\verb|真|。这个例子强调了词序的重要性。

\subsection{习题}

\subsubsection*{温故知新}

以口头或书面的形式简要回答以下问题。这些问题全都基于你刚刚阅读的内容,所以如果忘记了具体的定义、概念或示例,可以回去重读相关部分。确保在继续学习之前能够自信地回答这些问题,这将有助于你的理解和记忆!

\begin{enumerate}[label=(\arabic*)]
    \item 数学陈述的重要定义属性是什么?
    \item 数学陈述和变量命题有什么区别?
    \item 为什么哥德巴赫猜想是一个数学陈述?
    \item 以下定义变量命题的尝试有什么问题?
        \begin{center}
            \textcolor{red}{令 $Q(x, y, z)$ 为 $7x - 5y + z$}
        \end{center}
\end{enumerate}

\subsubsection*{小试牛刀}

尝试回答以下问题。这些题目要求你实际动笔写下答案,或(对朋友/同学)口头陈述答案。目的是帮助你练习使用新的概念、定义和符号。题目都比较简单,确保能够解决这些问题将对你大有帮助!

\begin{enumerate}[label=(\arabic*)]
    \item 对于以下每个语句,判断它是否是数学陈述。如果是,请判断它为\verb|真|还是为\verb|假|。如果不是,请解释原因。
        \begin{enumerate}[label=(\alph*)]
            \item $142857 \cdot 5 = 714285$
            \item 对于每个 $n \in \mathbb{N}$, $\displaystyle{\sum_{k \in [n]}k=\frac{n(n+1)}{2}}$。
            \item 对于任何集合 $A$ 和 $B$,如果 $A \subseteq B$,则 $B \subseteq A$。
            \item 对于任何集合 $A$ 和 $B$,如果 $A \subseteq B$,则 $\mathcal{P}(A) \subseteq \mathcal{P}(B)$。
            \item 数学就是酷。
            \item $1 + 2 = 0$
            \item 对于任意 $x, y \in \mathbb{Z}$,如果 $x \cdot y$ 为偶数,则 $x$ 和 $y$ 都为偶数。
            \item 对于任何 $x, y \in \mathbb{Z}$,如果 $x$ 和 $y$ 都为偶数,则 $x \cdot y$ 也为偶数。
            \item $1+ = 2$
            \item $-5 + \mathbb{Z} \ge \pi$
            \item $x = 7$
            \item 这句话不正确。
        \end{enumerate}
    \item 回顾前三章,找出数学陈述的一些例子和伪例。\\
    你能找到变量命题吗?它们是按照我们在本节中指定的方式编写的吗?你能修改它们使得它们正确书写吗?
    \item 写出一个变量命题的正确定义,当两个输入值具有非负和时,该命题为真。\\
    然后,找到命题为\verb|真|和为\verb|假|时的两个实例。
    \item 令 $S$ 为集合 $\{1, 2, 3, 6, 8, 10\}$。
        \begin{enumerate}[label=(\alph*)]
            \item 编写一个变量命题的正确定义,输入两个变量并判断它们的差的绝对值是否是 $S$ 的元素。然后,分别找到命题为\verb|真|和为\verb|假|时的两个实例。
            \item 编写一个变量命题的正确定义,输入两个集合并确定它们的交集是否是 $S$ 的子集。然后,分别找到命题为\verb|真|和为\verb|假|时的两个实例。
        \end{enumerate}
        (注意:给定任意集合 $X$ 和任意对象 $x$,它必须是 $x \in X$ 或 $x \notin X$。)
    \item 想出另一个数学命题,它为\verb|真|,但当我们交换一些单词的顺序时,它就变为\verb|假|。(请参阅最后一小节中的示例以获得一些灵感。)
    \item 对于以下定义变量命题,请确定它们是否正确。注意:这并不意味着判断它为\verb|真|还是为\verb|假|;相反,我们想知道该声明是否写得好且合理。\\
    如果由于某种原因写得不正确,请解释错误原因并编写一条新陈述来修复该错误。
    \begin{enumerate}[label=(\alph*)]
        \item 令 $P(x)$ 为 ``$x > 1$''。
        \item 令 $Q(x)$ 为命题 ``$x^2 - 1 > 0$''。
        \item 对于每个 $a,b \in \mathbb{R}$,令 $R(a, b)$ 为 ``$a^3 = b$''。
        \item 令 $P(x)$ 为 $x > 1$。
        \item 令 $T(z) = $ ``$z$ 为质数''。
        \item 对于每个 $x \in \mathbb{R}$,令 $Q(x)$ 为命题 ``$x^2 - 1 > 0$''。
        \item 对于每个 $x \in \mathbb{R}$,令 $Q(x)$ 为 ``$x^2 - 1 > 0$''。
        \item 令 $S(a)$ 为 ``$b^2 > 4$''。
        \item 对于每个 $x \in \mathbb{R}$,令 $Q(x)$ 为 $x^2 - 1 > 0$。
    \end{enumerate}
\end{enumerate}