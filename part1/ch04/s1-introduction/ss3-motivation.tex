% !TeX root = ../../../book.tex
\subsection{启下}

数学的核心在于发现真命题,并向他人严谨地解释其真实性。此前我们假设你对逻辑术语和真值概念有一定程度的了解。例如数学归纳原理(PMI,定理 \ref{theorem3.8})的假设:若 $P(k)$ 为真,则 $P(k+1)$ 为真。这究竟意味着什么?$P(k)$ 与 $P(k + 1)$ 之间存在何种逻辑关联?数学命题为真的本质是什么?

本章的目标包含多重维度:首要任务是界定数学中有意义的陈述类型。达成此基础后,我们将学习如何用精确简练的术语表达这些陈述。最终目标是掌握\textbf{证明}这些命题为真(或为假)的核心技术。
