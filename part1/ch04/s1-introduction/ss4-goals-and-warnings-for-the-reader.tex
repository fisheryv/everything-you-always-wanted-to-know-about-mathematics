% !TeX root = ../../../book.tex
\subsection{忠告}

请在学习本章时始终牢记一点:
\begin{center}
    你在学习一门\textbf{语言}。
\end{center}

这些材料中,有些看似艰深,有些略显枯燥,甚至可能两者兼有。但这些都是必经之路。

回想一下你在学校学习外语的经历。起点是什么?想必不是立即创作优美的诗篇。你首先接触的是基础语法、句法结构和核心词汇——学习关键冠词如``the''和``an'',掌握``to be''和``to have''等基本动词的使用规则,认识常见名词如``apple''、``dog''和 ``friend''。从这些基石出发,你开始组合简单短语,随着时间的推移,逐步运用积累的工具构建复杂句式。纵使心中早有精妙语句的雏形,但在新语言的词汇与语法体系完备之前,你仍难以准确表达。

本章的学习路径与此完全相同,只是这里的语言是\textbf{数学}。或许你已对某些绝妙的数学构想成竹在胸,只是尚未找到表达它们的数学语法。有趣的是,我们其实早已开始``谈论''数学!解决各类难题、引入证明技术(如归纳法)、运用数学对象(如集合)。为确保相互理解,我们详尽记录每个步骤,反复阐释细节——这就像在语言不通的异国生存:借助手势比划、捕捉关键词、画图示意、模拟声音。短暂旅行尚可如此应付,但若要在此定居,则需更进一步。


这正是我们当下的处境。要在数学世界定居,就必须系统掌握它的语言。唯有跨越这道门槛,我们方能如母语者般游刃有余。届时便可适度放松,使用数学"俚语"、缩写或惯用表达(如同英语中那些语法未必严谨却约定俗成的短语)。唯有至此,才算真正融会贯通。

在此之前,我们的语言将保持严谨甚至略显迂腐。\emph{若此刻不强迫自己恪守规范,便永远无法真正掌握这门共通的数学语言}。
