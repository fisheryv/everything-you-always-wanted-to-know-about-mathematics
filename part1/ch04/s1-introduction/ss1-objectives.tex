% !TeX root = ../../../book.tex
\subsection{目标}

以下简短内容将向你展示本章如何融入本书的体系。这部分内容会描述我们之前的工作将如何发挥作用,还会激发我们为什么要研究本章出现的主题,并告诉你我们的目标,以及你在阅读时应该记住什么来实现这些目标。现在,我们将通过一系列陈述为你总结本章的主要目标,以及本章结束时你应该获得的技能和知识。以下各节将更详细地重申这些想法,但这里将为你提供一个简短的列表以供将来参考。当学完本章后,请返回此列表,看看你是否理解所有这些目标。你明白为什么我们在这里概述它们很重要吗?你能定义我们使用的所有术语吗?你能应用我们描述的技术吗?

\textbf{学完本章后,你应该能够……}

\begin{itemize}
    \item 使用恰当的符号定义变量命题。
    \item 使用量词及其他恰当的符号定义数学陈述,并识别不是适当数学陈述的句子。
    \item 理解并解释两种类型量词之间的区别以及它们的使用方式。
    \item 定义和理解逻辑连词的含义,并使用它们来定义更复杂的数学语句。
    \item 运用证明技术创建证明数学陈述真实性的正式论证。
    \item 比较和对比不同类型的证明技术,并根据情况了解何时以及如何使用它们。
\end{itemize}
