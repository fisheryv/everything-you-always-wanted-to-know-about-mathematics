% !TeX root = ../../../book.tex
\subsection{目标}

以下内容简要说明本章在本书中的定位。我们将解释前期工作如何为本章研究奠定基础,阐明探讨本章主题的动机,并概述学习目标及注意事项。我们会先总结本章的主要目标,概括你在学完本章后应掌握的技能与知识。后续章节将详细展开这些思想,此处仅提供一个简要列表作为学习指引。完成本章后,请你返回此列表,确认自己是否达成了所有目标。你是否能理解这些目标的重要性?能否清晰地解释相关术语并熟练地应用相关技术?

\textbf{学完本章后,你应该能够……}

\begin{itemize}
    \item 使用恰当的符号定义变量命题。
    \item 运用量词及其他符号精确定义数学陈述,并识别非规范的数学表述。
    \item 理解并解释两类量词的本质区别及其应用场景。
    \item 掌握逻辑连词的含义,并能运用它们构建复合数学命题。
    \item 运用证明技术构建数学命题的严谨论证。
    \item 比较不同证明技术的特性,并能判断其适用场景。
\end{itemize}
