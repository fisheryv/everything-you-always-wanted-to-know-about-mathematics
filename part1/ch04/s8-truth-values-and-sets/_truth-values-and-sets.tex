% !TeX root = ../../../book.tex
\section[真值与集合]{[选学]真值与集合}

集合(及其关系与运算)与逻辑真值(及其关系与连词)之间存在巧妙而实用的对应关系。本节将阐述这种关系并举例说明,供学有余力的读者深入探究。后续内容虽不依赖这些知识,但我们相信,梳理这些概念将助你真正理解逻辑与集合的基础。

设有两个变量命题 $P(x)$ 和 $Q(x)$,它们对全集 $U$ 中的任意元素 $x$ 均有定义(全集 $U$ 取决于命题的具体内容,但此处为普适性讨论无需关注)。对于每个命题,可以定义其\textbf{真值集}——即全集 $U$ 中使命题为\verb|真|的所有 $x$ 构成的集合:
\begin{align*}
    T_P &= \{x \in U \mid P(x) \text{\ 为真}\} \\
    T_Q &= \{x \in U \mid Q(x) \text{\ 为真}\}
\end{align*}

若 $P(x)$ 与 $Q(x)$ 存在逻辑关联。假设:
\[\forall x \in U \centerdot P(x) \implies Q(x)\]
这对\textbf{真值集}意味着什么?该条件命题表明:任何满足 $P(x)$(即 $P(x)$ 为\verb|真|)的 $x$ 必然满足 $Q(x)$。转换为真值集表述为:
\[\forall x \in U \centerdot x \in T_P \implies x \in T_Q\]
此即子集的定义。因此,上述条件命题成立时:
\[T_P \subseteq T_Q\]

我们进一步假设
\[\forall x \in U \centerdot P(x) \iff Q(x)\]
类似地,可以推导 $\iff$ 的``另一个方向''(即 $Q(x) \implies P(x)$ 部分),由双向蕴含关系可推得 $T_Q \subseteq T_P$。根据集合相等的定义,这意味着当双向条件陈述成立时,
\[T_P = T_Q\]

我们还可以组合命题 $P(x)$ 和 $Q(x)$。对于合取式 $P(x) \land Q(x)$,使该合取为\verb|真|的 $x$ 有哪些?我们如何根据定义的真值集来描述这些 $x$ 的实例?想一想,你会发现所有这些 $x$ 的实例的特征都是真值集的交集;我们需要 $P(x)$ 和 $Q(x)$ 同时成立,因此 $x$ 的实例需要在两个真值集中都存在。

类似地,对于合取式 $P(x) \lor Q(x)$。当 $x$ 的实例令\emph{至少一个}命题为\verb|真|时,该 $x$ 的实例令该陈述为\verb|真|。因此,$x$ 必须至少来自其中一个真值集,因此它必然来自真值集的\emph{并集}。

让我们总结一下发现的对应关系:
\begin{align*}
    \forall x \in U & \centerdot \big(P(x) \implies Q(x)\big) \iff \big(T_P \subseteq T_Q \big) \\
    \forall x \in U & \centerdot \big(P(x) \iff Q(x)\big) \iff \big(T_P = T_Q \big) \\
    \forall x \in U & \centerdot \big(P(x) \land Q(x)\big) \iff \big(T_P \cap T_Q \big) \\
    \forall x \in U & \centerdot \big(P(x) \lor Q(x)\big) \iff \big(T_P \cup T_Q \big) 
\end{align*}

请用真值集描述下列命题的特征,并将答案填入空白处:
\begin{align*}
    \forall x \in U \centerdot \big(P(x) \land \neg Q(x)\big) & \iff \underline{\qquad\qquad\qquad} \\
    \big(\exists x \in U \centerdot P(x)\big) & \iff \underline{\qquad\qquad\qquad} \\
    \forall x \in U \centerdot \big(\neg P(x) & \iff \underline{\qquad\qquad\qquad}\big) \\
    \big(\forall x \in U \centerdot \neg P(x)\big) & \iff \underline{\qquad\qquad\qquad}
\end{align*}

(注意:最后两个命题有何不同?)
