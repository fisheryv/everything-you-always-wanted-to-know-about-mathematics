% !TeX root = ../../../book.tex
\subsection{证明 $\forall$ 声明}\label{sec:section4.9.2}

``$\forall$'' 声明一种\emph{全称性声明}。它断言集合中的\emph{所有}元素都具有某些共同的属性。为了证明这种说法,我们需要证明集合中的\emph{每个}元素都具有该属性。为了``一次性''完成证明,我们需要考虑集合中任意固定元素,并证明它具有所需的属性。因为这个对象是任意的,所以我们的论证适用于集合中的每个元素。因为这个对象是固定的,所以我们可以在整个证明中通过名称来引用它。

\subsubsection*{直接证法}

\begin{center}
\noindent \fcolorbox{blue}{white}{%
    \parbox{0.85\textwidth}{%
        \linespread{1.5}\selectfont
        \textcolor{blue}{\textbf{策略:}}\\
        声明:$\forall x \in S \centerdot P(x)$ \\
        \emph{直接证明策略:} \\
        \hspace*{1cm} 设 $y \in S$ 是任意固定元素。\\
        \hspace*{1cm} 证明 $P(y)$ 成立。
    }
}
\end{center}

\newpage

\begin{example}[算术几何平均值不等式的一个实例]

    \begin{center}
        陈述:$\forall x, y \in \mathbb{R} \centerdot xy \le (\frac{x+y}{2})^2$。
    \end{center}
    \begin{center}
        \noindent \fcolorbox{olivegreen}{white}{%
            \parbox{0.85\textwidth}{%
                \linespread{1.5}\selectfont
                \textcolor{olivegreen}{\textbf{实现:}}
                \begin{proof}
                    设 $x, y \in \mathbb{R}$ 是任意固定元素。\\
                    我们知道 $0 \le (x-y)^2$。\\
                    展开整理得 $2xy \le x^2+y^2$。\\
                    两边同时加上 $2xy$,可得 $2xy \le x^2 + 2xy + y^2$。\\
                    因式分解可得 $4xy \le (x+y)^2$。\\
                    两边同时除以 $4$ 可得
                    \[xy \le (\frac{x+y}{2})^2\] 
                \end{proof}
            }
        }
    \end{center}
    这个结论被称为\textbf{算术几何平均值不等式(Arithmetic-Geometric Mean Inequality)},因为它涉及两个实数的算术平均值(Arithmetic Mean)和几何平均值(Geometric Mean)。

    $x$ 和 $y$ 的算术平均值为 $\frac{a+b}{2}$。

    $x$ 和 $y$ 的几何平均值为 $\sqrt{xy}$(请注意,这只适用于 $xy \ge 0$ 时,即 $x$ 和 $y$ 具有相同的符号(无论是正数、负数或是零)。

    算术几何平均值不等式(简称 AGM)断言算术平均值(AM)始终大于等于几何平均值(GM)。一个有用的助记符是将 ``\textbf{AGM}'' 读作 ``算术平均值大于几何平均值(\textbf{A}rithmetic Mean \textbf{G}reater than \textbf{G}eometric Mean)''。

    我们上面证明的是一个更为通用的版本,因为它适用于所有实数 $x$ 和 $y$,而不仅仅是那些具有相同符号的实数。然而,假设 $xy \ge 0$,我们可以简单地两边取平方根,即可得到算术几何平均值不等式的``常规''形式:$\sqrt{xy} \le \frac{x+y}{2}$。
\end{example}

\subsubsection*{间接证法(反证法)}

\begin{center}
    \noindent \fcolorbox{blue}{white}{%
        \parbox{0.85\textwidth}{%
            \linespread{1.5}\selectfont
            \textcolor{blue}{\textbf{策略:}}\\
            声明:$\forall x \in S \centerdot P(x)$ \\
            \emph{间接证明策略:} \\
            \hspace*{1cm} 为了引出矛盾而假设,$\exists y \in S$ 使得 $\neg P(y)$ 成立。 \\
            \hspace*{1cm} 推导得出矛盾。
        }
    }
\end{center}

\begin{example}[$\sqrt{2} $ 为无理数]

    陈述:$ \forall a, b \in \mathbb{Z} \centerdot \frac{a}{b} \ne \sqrt{2}$。

    (注:该声明直接诉诸有理数 $\mathbb{Q}$ 的定义。它说 $\sqrt{2} \notin \mathbb{Q}$ 因为该数无法表示为整数之比。)
\end{example}

\begin{center}
    \noindent \fcolorbox{olivegreen}{white}{%
        \parbox{0.85\textwidth}{%
            \linespread{1.5}\selectfont
            \textcolor{olivegreen}{\textbf{实现:}}
            \begin{proof}
                为了引出矛盾而假设 $\exists a, b \in \mathbb{Z} \centerdot \frac{a}{b} = \sqrt{2}$。

                我们可以假设 $\frac{a}{b}$ 已经化简到最简形式,因此 $a$ 和 $b$ 没有公因数。(如果不是这种情况,我们可分子分母同时以除以公因数并获得最简形式。)

                给定这样的 $a,b \in \mathbb{Z}$。

                (注:我们将在第 \ref{sec:section4.9.8} 节讨论短语``给定这样的$\underline{\qquad}$''。它不仅意味着断言这样的 $a, b \in \mathbb{Z}$ \emph{存在},而且我们想要这些变量具有一些\emph{特定}实例,以便我们可以使用它们来完成余下的证明。)

                这意味着 $\frac{a}{b} = \sqrt{2}$,所以 $\frac{a^2}{b^2} = 2$。

                因此 $2b^2 = a^2$,所以根据定义 $a^2$ 为偶数。

                因为 $a^2$ 为偶数,则 $a$ 也为偶数。

                (注意:这一点需要证明。我们将在第 \ref{sec:section4.9.6} 节证明这一点,你也可以自己尝试证明一下。)

                因此,$\exists k \in \mathbb{Z} \centerdot a = 2k$。给定这样的 $k$,使得 $a^2 = 4k^2$。

                则 $2b^2 = 4k^2$,所以 $b^2 = 2k^2$。

                因此,根据定义,$b^2$ 为偶数。同理,我们可以推导出 $b$ 为偶数。

                这说明 $a$ 和 $b$ 都为偶数,它们有公因数 $2$。

                这与我们假设 $a$ 和 $b$ 没有公因数相矛盾。$\hashx$

                因此,假设必定有误,所以该声明为真。
            \end{proof}
        }
    }
\end{center}