% !TeX root = ../../../book.tex
\subsection{证明 $\forall$ 声明}\label{sec:section4.9.2}

``$\forall$''声明是一种\emph{全称性声明},它断言集合中的\emph{所有}元素都具有某个共同属性。要证明这类命题,需要验证集合中的\emph{每个}元素都满足该属性。为了``一次性''完成证明,我们需要考虑集合中任意固定元素,并证明其满足属性。由于该元素具有任意性,论证过程将适用于集合中的每个元素;同时因其固定性,可在整个证明中通过名称引用它。

\subsubsection*{直接证法}

\begin{center}
\noindent \fcolorbox{blue}{white}{%
    \parbox{0.85\textwidth}{%
        \linespread{1.5}\selectfont
        \textcolor{blue}{\textbf{策略:}}\\
        声明:$\forall x \in S \centerdot P(x)$ \\
        \emph{直接证明策略:} \\
        \hspace*{1cm} 设 $y \in S$ 为任意固定元素。\\
        \hspace*{1cm} 证明 $P(y)$ 成立。
    }
}
\end{center}

\begin{example}[算术几何平均值不等式的一个实例]

    \begin{center}
        陈述:$\forall x, y \in \mathbb{R} \centerdot xy \le (\frac{x+y}{2})^2$。
    \end{center}
    \begin{center}
        \noindent \fcolorbox{olivegreen}{white}{%
            \parbox{0.85\textwidth}{%
                \linespread{1.5}\selectfont
                % \textcolor{olivegreen}{\textbf{实现:}}
                \begin{proof}
                    设 $x, y \in \mathbb{R}$ 为任意固定元素。\\
                    我们知道 $0 \le (x-y)^2$。\\
                    展开整理得 $2xy \le x^2+y^2$。\\
                    两边同时加上 $2xy$,可得 $2xy \le x^2 + 2xy + y^2$。\\
                    因式分解可得 $4xy \le (x+y)^2$。\\
                    两边同时除以 $4$ 即得 $xy \le (\frac{x+y}{2})^2$ 
                \end{proof}
            }
        }
    \end{center}
    此结论称为\textbf{算术几何平均值不等式 (Arithmetic-Geometric Mean Inequality)},因其关联两个实数的算术平均值 (Arithmetic Mean) 和几何平均值 (Geometric Mean)。

    $x$ 和 $y$ 的算术平均值为 $\frac{a+b}{2}$。

    $x$ 和 $y$ 的几何平均值为 $\sqrt{xy}$(请注意,这只适用于 $xy \ge 0$ 时,即 $x$ 和 $y$ 具有相同符号)。

    算术几何平均值不等式(简称 AGM)断言算术平均值 (AM) 恒大于或等于几何平均值 (GM)。可用助记词``\textbf{AGM}''表示``算术平均值大于几何平均值 (\textbf{A}rithmetic Mean \textbf{G}reater than \textbf{G}eometric Mean)''。

    上述证明是一个更为通用的版本,因其适用于所有实数 $x,y$(不限于同号情形)。当 $xy \ge 0$ 时,两边取平方根即可得到``常规''形式:
    \[\sqrt{xy} \le \frac{x+y}{2}\]
\end{example}

\clearpage

\subsubsection*{间接证法(反证法)}

\begin{center}
    \noindent \fcolorbox{blue}{white}{%
        \parbox{0.85\textwidth}{%
            \linespread{1.5}\selectfont
            \textcolor{blue}{\textbf{策略:}}\\
            声明:$\forall x \in S \centerdot P(x)$ \\
            \emph{间接证明策略:} \\
            \hspace*{1cm} 为了引出矛盾而假设,$\exists y \in S$ 使得 $\neg P(y)$ 成立。 \\
            \hspace*{1cm} 推导得出矛盾。
        }
    }
\end{center}

\begin{example}[$\sqrt{2}$ 为无理数]

    \begin{center}
        陈述:$ \forall a, b \in \mathbb{Z} \centerdot \frac{a}{b} \ne \sqrt{2}$。

        (注:该结论直接基于有理数 $\mathbb{Q}$ 的定义,即 $\sqrt{2} \notin \mathbb{Q}$,因为它不能表示为整数之比。)
    \end{center}
    
    \begin{center}
        \noindent \fcolorbox{olivegreen}{white}{%
            \parbox{0.85\textwidth}{%
                \linespread{1.5}\selectfont
                % \textcolor{olivegreen}{\textbf{实现:}}
                \begin{proof}
                    为了引出矛盾而假设 $\exists a, b \in \mathbb{Z} \centerdot \frac{a}{b} = \sqrt{2}$。

                    不妨设 $\frac{a}{b}$ 已化为最简形式,因此 $a$ 和 $b$ 互质。(若非如此,分子分母可同时除以公因数得到最简形式。)

                    给定满足上述条件的 $a,b \in \mathbb{Z}$。

                    (注:我们将在第 \ref{sec:section4.9.8} 节讨论短语``给定满足条件的$\underline{\qquad}$''。它不仅表明此类 $a, b \in \mathbb{Z}$ \emph{存在},更强调使用其\emph{具体}实例,以便完成后续证明。)

                    这意味着 $\frac{a}{b} = \sqrt{2}$,所以 $\frac{a^2}{b^2} = 2$,即 $2b^2 = a^2$。

                    根据定义 $a^2$ 为偶数,因此 $a$ 也为偶数。

                    (注意:这一点需要证明。我们将在第 \ref{sec:section4.9.6} 节证明该性质,你也可以自行尝试证明。)

                    这意味着 $\exists k \in \mathbb{Z} \centerdot a = 2k$。给定这样的 $k$,使得 $a^2 = 4k^2$。

                    则 $2b^2 = 4k^2$,所以 $b^2 = 2k^2$。

                    因此,根据定义,$b^2$ 为偶数。同理可得 $b$ 为偶数。

                    此时 $a$ 和 $b$ 均为偶数,它们有公因数 $2$。

                    这与假设 $a$ 和 $b$ 互质矛盾。$\hashx$

                    因此,假设不成立,原命题成立。
                \end{proof}
            }
        }
    \end{center}
\end{example}
