% !TeX root = ../../../book.tex
\subsection{证明 $\exists$ 声明}\label{sec:section4.9.1}

``$\exists$'' 声明是一种\emph{存在性声明}。它断言某个特定对象是某个集合的元素,并且具有特定的属性。为了证明这种说法,我们需要找到一个这样的对象,并为读者验证
\begin{enumerate}[label=(\arabic*)]
    \item 该对象是正确集合的元素;
    \item 该对象具有正确的属性。
\end{enumerate}

\subsubsection*{直接证法}

\setlength{\fboxrule}{2pt}
\setlength\fboxsep{5mm}
\begin{center}
\noindent \fcolorbox{blue}{white}{%
    \parbox{0.85\textwidth}{%
        \linespread{1.5}\selectfont
        \textcolor{blue}{\textbf{策略:}}\\
        声明:$\exists x \in S \centerdot P(x)$ \\
        \emph{直接证明策略:} \\
        \hspace*{1cm} 定义一个具体的例子,$y = \underline{\qquad\qquad}$。\\
        \hspace*{1cm} 证明 $y \in S$。\\
        \hspace*{1cm} 证明 $P(y)$ 成立。
    }
}
\end{center}

\newpage

\begin{example}[求解线性方程组]
    
    \begin{center}
        陈述:给定 $a, b, c, d, e, f \in \mathbb{R}$,且 $ad - bc \ne 0$。
    \end{center}
    我们声称,存在某个 $x, y \in \mathbb{R}$,使得以下线性方程组成立
    \begin{align}
        ax + by &= e \label{eq:4.1}\\
        cx + dy &= f \label{eq:4.2}
    \end{align}
    将 $S(x, y)$ 定义为 ``$x$ 和 $y$ 同时满足上述两个方程 (\ref{eq:4.1}) 和 (\ref{eq:4.2})''。则我们声称
    \[\exists x, y \in \mathbb{R} \centerdot S(x, y)\]
    首先,我们必须做一些初步工作来构建解决方案。然后,我们才可以撰写证明来定义对象 $x$ 和 $y$ 并说明它们成立的原因。
    \begin{center}
    \noindent \fcolorbox{red}{white}{%
    \parbox{0.85\textwidth}{%
        \linespread{1.5}\selectfont
        \textcolor{red}{\textbf{草稿:}}

        我们需要 $ax + by = e$ 和 $cx + dy = f$ 同时成立,并且我们想知道哪个 $x$ 和 $y$ 可以实现这一点。

        让我们将第一个和第二个方程乘以正确的系数(分别是 $d$ 和 $-b$),这样我们就可以通过将两个方程相加来消掉 $y$ 项:
        \begin{center}
            \begin{tabular}{r@{\,}c@{\,}c@{\,}c@{\,}c@{\,}l}
                    & $adx$ & $+$ & $bdy$  & $=$ & $de$ \\
            $+(-$   & $bcx$ & $-$ & $bdy$  & $=$ & $-bf)$\\
            \hline
                    & $(ad$ & $-$ & $bc)x$ & $=$ & $de - bf$ \\
            \end{tabular}
        \end{center}
        然后除以 $x$ 的系数可得 $x = \frac{de-bf}{ad-bc}$。这是一个有效解,因为 $ad - bc \ne 0$。\\

        同理我们可以消掉 $x$ 项,求得 $y$:
        \begin{center}
            \begin{tabular}{r@{\,}c@{\,}c@{\,}c@{\,}c@{\,}l}
                    & $acx$ & $+$ & $bcy$  & $=$ & $ce$ \\
            $+(-$   & $acx$ & $-$ & $ady$  & $=$ & $-af)$\\
            \hline
                    & $(bc$ & $-$ & $ad)y$ & $=$ & $ce - af$ \\
            \end{tabular}
        \end{center}
        除以 $y$ 的系数可得 $y = \frac{af-ce}{ad-bc}$。
    }
    }
    \end{center}
    这里的主要经验是,我们不需要在下面的证明中展示草稿中的工作!我们不认为读者愿意费力翻阅我们杂乱的笔记,去了解我们是如何得出线性方程组的解的。相反,我们假设读者只关心解是什么以及为什么它是正确的解。这使得证明更加简洁,因此可以更容易、更快速地阅读。

    \begin{center}
        \noindent \fcolorbox{olivegreen}{white}{%
            \parbox{0.85\textwidth}{%
                \linespread{1.5}\selectfont
                \textcolor{olivegreen}{\textbf{实现:}}
                \begin{proof}
                    由于 $ad - bc \ne 0$ (根据假设),我们可以定义
                    \[x = \frac{de - bf}{ad - bc} \qquad \text{和} \qquad y = \frac{af - ce}{ad - bc}\]
                    并知道 $x, y \in \mathbb{R}$。则,
                    \begin{align*}
                        ax + by &= \frac{(ade - abf) + (abf - bce)}{ad - bc} = \frac{ade - bce}{ad - bc} = \frac{e(ad - bc)}{ad - bc} = e \\
                        cx + dy &= \frac{(cde - bcf) + (adf - cde)}{ad - bc} = \frac{adf - bcd}{ad - bc} = \frac{f(ad - bc)}{ad - bc} = f
                    \end{align*}
                    所以 $S(x, y)$ 成立。
                \end{proof}
            }
        }
    \end{center}

    如果你学过线性代数,你会发现 $ad-bc$ 项是矩阵 $\begin{bmatrix}
        a & b \\
        c & d 
        \end{bmatrix}$ 的\textbf{行列式}。$ad - bc \ne 0$ 这个前提条件意味着该系数矩阵\emph{存在逆矩阵},即它是``非奇异''的。在这种情况下,对于任意 $e, f \in \mathbb{R}$ 线性方程组都有解。
\end{example}

\subsubsection*{间接证法(反证法)}

该策略依赖于 $\exists$ 声明的逻辑否定:
\[\neg \big(\exists x \in S \centerdot P(x)\big) \iff \forall x \in S \centerdot \neg P(x)\]

我们先假设否命题成立然后从中推导出一系列矛盾,这意味着否命题为\verb|假|,所以原命题\verb|真|。

\begin{center}
    \noindent \fcolorbox{blue}{white}{%
        \parbox{0.85\textwidth}{%
            \linespread{1.5}\selectfont
            \textcolor{blue}{\textbf{策略:}}\\
            声明:$\exists x \in S \centerdot P(x)$ \\
            \emph{间接证明策略:} \\
            \hspace*{1cm} 为了得到矛盾而假设,对于每个 $y \in S, \neg P(y)$ 成立。 \\
            \hspace*{1cm} 推导得出矛盾。
        }
    }
\end{center}

\begin{example}[抽屉原理的一个实例]\label{ex:example4.9.2}

    \begin{center}
        陈述:假设 $n \in \mathbb{N}$ 且我们有 $n$ 个实数 $a_1, a_2, \dots, a_n \in \mathbb{R}$。\\
        我们声称其中一个数字至少与数列的平均值一样大。即,
        \[\exists B \in [n] \centerdot a_B \ge \frac{1}{n}\sum_{i=1}^{n}a_i\]
    \end{center}

    \begin{center}
        \noindent \fcolorbox{olivegreen}{white}{%
            \parbox{0.85\textwidth}{%
                \linespread{1.5}\selectfont
                \textcolor{olivegreen}{\textbf{实现:}}
                \begin{proof}
                    为了得到矛盾而假设,所有数字都小于数列平均值,即
                    \[\forall i \in [n] \centerdot a_i < \frac{1}{n}\sum_{i=1}^{n}a_i\]
                    定义常数 $S = \sum_{i=1}^{n}a_i$,则 $a_i < \frac{S}{n}$。\\
                    然后我们可以对所有 $a_i$ 求和并发现
                    \[S = \sum_{i=1}^{n}a_i < \sum_{i=1}^{n}\frac{S}{n} = n \cdot \frac{S}{n} = S\]
                    这说明实数 $S$ 严格小于它本身:$S < S$。这里存在矛盾。$\hashx$

                    因此,最初的假设是错误的,原命题成立。
                \end{proof}
            }
        }
    \end{center}

    如前所述,这是\textbf{抽屉原理}的一个实例。当我们学习\textbf{组合学}时,将在 \ref{sec:section8.6} 节再次研究和使用这个原理。
\end{example}