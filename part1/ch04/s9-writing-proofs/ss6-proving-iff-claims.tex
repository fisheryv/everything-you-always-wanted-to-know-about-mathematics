% !TeX root = ../../../book.tex
\subsection{证明 $\iff$}\label{sec:section4.9.6}

回想一下,``$\iff$'' 连词完全是根据 ``$\implies$'' 连词定义的。也就是说,
\[P \iff Q\]
逻辑等价于两个条件陈述:
\[(p \implies Q) \land (Q \implies P)\]
这就产生了一个明显的策略:证明一个条件陈述,然后证明另一个!这里最常见的错误是仅仅证明了其中一个陈述,而不是同时证明两者。永远记住这一点!

\subsubsection*{直接证法}

\begin{center}
    \noindent \fcolorbox{blue}{white}{%
        \parbox{0.85\textwidth}{%
            \linespread{1.5}\selectfont
            \textcolor{blue}{\textbf{策略:}}\\
            声明:$P \iff Q$ \\
            \emph{直接证明策略:} \\
            \hspace*{1cm} 证明 $P \implies Q$ (使用上一小节介绍的任意一种策略)。

            \hspace*{1cm} 证明 $Q \implies P$ (使用上一小节介绍的任意一种策略)。
        }
    }
\end{center}

\newpage

\begin{example}[偶数的平方为偶数]

    陈述:一个整数是偶数当且仅当它的平方是偶数。

    让我们用逻辑符号符号重写这个声明。
    
    设 $E(z)$ 为命题``$z$ 为偶数''。 那么我们声称
    \[\forall z \in \mathbb{Z} \centerdot \Big(E(z) \iff E(z^2)\Big)\]
\end{example}

\begin{center}
    \noindent \fcolorbox{olivegreen}{white}{%
        \parbox{0.85\textwidth}{%
            \linespread{1.5}\selectfont
            \textcolor{olivegreen}{\textbf{实现:}}
            \begin{proof}
                ($\implies$)首先,假设 $z$ 为偶数,因此 $\exists k \in \mathbb{Z} \centerdot z = 2k$。给定这样一个 $k$。由于 $z = 2k$,我们可以将两边平方并得到
                \[z^2=(2k)^2=4k^2=2(2k^2)\]
                
                定义 $l=2k^2$。注意 $l \in \mathbb{Z}$ 且 $z^2=2l$。

                这证明了 $z^2$ 为偶数。

                因此 $E(z) \implies E(z^2)$

                ($\implies$)接着,我们用反证法证明 $E(z^2) \implies E(z)$。

                假设 $z$ 为奇数,因此 $\exists m \in \in \mathbb{Z} \centerdot z = 2m + 1$。给定这样一个 $m$。

                由于 $z = 2m + 1$,我们可以将两边平方并得到
                \[z^2=(2m+1)^2=4m^2+4m+1=2(2m^2+2m)+1\]

                定义 $n=2m^2+2m$。注意 $n \in \mathbb{Z}$ 且 $z^2=2n+1$。

                这证明了 $z^2$ 为奇数。

                因此,$\neg E(z) \implies \neg E(z^2)$;根据逆否策略,可得 $E(z^2) \implies E(z)$。

                综上,
                \[E(z) \iff E(z^2)\]

                由于 $z$ 是任意的,因此上面的证明对所有整数 $z$ 都成立。
            \end{proof}
        }
    }
\end{center}

\subsubsection*{间接证法(反证法)}

\begin{center}
    \noindent \fcolorbox{blue}{white}{%
        \parbox{0.85\textwidth}{%
            \linespread{1.5}\selectfont
            \textcolor{blue}{\textbf{策略:}}\\
            声明:$P \iff Q$ \\
            \emph{间接证明策略:} \\
            \hspace*{1cm} 为了引出矛盾而假设 $\neg (P \implies Q) \lor \neg (Q \implies P)$。

            \hspace*{1cm} 考虑第一种情况,即 $P \land \neg Q$ 成立。找出矛盾点。

            \hspace*{1cm} 考虑第二种情况,即 $Q \land \neg P$ 成立。找出矛盾点。
        }
    }
\end{center}

使用这一策略——尤其是何时使用这一策略——取决于实际的陈述 $P$ 和 $Q$。一般来说,直接证法可能会更好(并非总是如此),但如果你发现自己陷入困境,可以考虑尝试一下原命题的否定形式—— $P \land \neg Q$ 和 $Q \land \neg P$ ——看一下能否帮你解决问题。有时这种策略非常值得一试!

\subsubsection*{中介证法(TFAE)}

由于缺乏更好的术语,我们将这种策略称为\textbf{中介证法}。正如你将看到的那样,这并不完全是直接证法,也不是间接证法。在使用这一策略时,我们不必考虑任何逻辑否定,但我们也没有直接将陈述 $P$ 和 $Q$ 链接起来。

相反,该方法要求我们找到某个\emph{中间}陈述 $R$ 并证明两个双条件陈述:即 $P \iff R$ 和 $R \iff Q$。这会产生以下条件陈述链
\[P \iff R \iff Q\]
它告诉我们所有三个陈述都具有相同的真值。特别是,$P$ 和 $Q$ 必然始终具有相同的真值,因此我们得出 $P \iff Q$ 的结论。

\textbf{TFAE} 是 ``the following are equivalent'' 的缩写,意思是``以下是等效的''。我们选择用这个缩写来命名该策略是因为它是数学中的一个常见短语;它用在提出一系列``相互暗示''的条件/属性的定理中。也就是说,一些定理列出了几个属性并断言它们在逻辑上都是等价的,因此``以下是等价的''。为了证明这样的定理,我们需要一遍又一遍地使用上述策略,并证明这些陈述确实是等价的。这里唯一的区别是我们必须构造出要使用的中间陈述。(但是,无论是谁提出并证明了 TFAE 式的定理,也必须首先给出所有这些陈述!)

\begin{center}
    \noindent \fcolorbox{blue}{white}{%
        \parbox{0.85\textwidth}{%
            \linespread{1.5}\selectfont
            \textcolor{blue}{\textbf{策略:}}\\
            声明:$P \iff Q$ \\
            \emph{中介策略:} \\
            \hspace*{1cm} 定义陈述 $R$。

            \hspace*{1cm} 证明 $P \iff R$(使用上面介绍的任意一种策略)。

            \hspace*{1cm} 证明 $R \iff Q$(使用上面介绍的任意一种策略)。
        }
    }
\end{center}