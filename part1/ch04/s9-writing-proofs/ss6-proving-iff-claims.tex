% !TeX root = ../../../book.tex
\subsection{证明 $\iff$}\label{sec:section4.9.6}

回想一下,连词``$\iff$''完全基于连词``$\implies$''定义。具体而言,
\[P \iff Q\]
逻辑等价于两个条件陈述:
\[(p \implies Q) \land (Q \implies P)\]
这自然导出了一个明显的策略:先证明一个条件陈述,再证明另一个条件陈述!最常见的错误是只证明其中一个而忽略另一个,请务必牢记这一点!

\subsubsection*{直接证法}

\begin{center}
    \noindent \fcolorbox{blue}{white}{%
        \parbox{0.85\textwidth}{%
            \linespread{1.5}\selectfont
            \textcolor{blue}{\textbf{策略:}}\\
            声明:$P \iff Q$ \\
            \emph{直接证明策略:} \\
            \hspace*{1cm} 证明 $P \implies Q$ (使用上一小节介绍的任意一种策略)。

            \hspace*{1cm} 证明 $Q \implies P$ (使用上一小节介绍的任意一种策略)。
        }
    }
\end{center}

\begin{example}[偶数的平方为偶数]

    陈述:一个整数为偶数当且仅当它的平方为偶数。

    用逻辑符号重写如下:
    
    设 $E(z)$ 为命题``$z$ 为偶数''。我们声称
    \[\forall z \in \mathbb{Z} \centerdot \Big(E(z) \iff E(z^2)\Big)\]

    \begin{center}
        \noindent \fcolorbox{olivegreen}{white}{%
            \parbox{0.85\textwidth}{%
                \linespread{1.5}\selectfont
                % \textcolor{olivegreen}{\textbf{实现:}}
                \begin{proof}
                    \begin{itemize}
                        \item ($\implies$) 首先,假设 $z$ 为偶数,则 $\exists k \in \mathbb{Z} \centerdot z = 2k$。给定满足条件的 $k$。由于 $z = 2k$,对两边平方得
                    \[z^2=(2k)^2=4k^2=2(2k^2)\]
                    
                    定义 $l=2k^2$。注意 $l \in \mathbb{Z}$ 且 $z^2=2l$。

                    这证明了 $z^2$ 为偶数。

                    因此 $E(z) \implies E(z^2)$

                    \item ($\implies$) 接着,我们用反证法证明 $E(z^2) \implies E(z)$。
                    
                    假设 $z$ 为奇数,则 $\exists m \in \mathbb{Z} \centerdot z = 2m + 1$。给定满足条件的 $m$。由于 $z = 2m + 1$,对两边平方得
                    \[z^2=(2m+1)^2=4m^2+4m+1=2(2m^2+2m)+1\]

                    定义 $n=2m^2+2m$。注意 $n \in \mathbb{Z}$ 且 $z^2=2n+1$。

                    这证明了 $z^2$ 为奇数。

                    因此,$\neg E(z) \implies \neg E(z^2)$;根据逆否策略,可得 $E(z^2) \implies E(z)$。
                    \end{itemize}

                    综上,
                    \[E(z) \iff E(z^2)\]

                    由于 $z$ 是任意的,因此上述结论对所有整数 $z$ 均成立。
                \end{proof}
            }
        }
    \end{center}
\end{example}

\subsubsection*{间接证法(反证法)}

\begin{center}
    \noindent \fcolorbox{blue}{white}{%
        \parbox{0.85\textwidth}{%
            \linespread{1.5}\selectfont
            \textcolor{blue}{\textbf{策略:}}\\
            声明:$P \iff Q$ \\
            \emph{间接证明策略:} \\
            \hspace*{1cm} 为了引出矛盾而假设 $\neg (P \implies Q) \lor \neg (Q \implies P)$。

            \hspace*{1cm} 考虑第一种情况,即 $P \land \neg Q$ 成立。找出矛盾点。

            \hspace*{1cm} 考虑第二种情况,即 $Q \land \neg P$ 成立。找出矛盾点。
        }
    }
\end{center}

使用这一策略,特别是决定何时使用,取决于具体的陈述 $P$ 和 $Q$。一般来说,直接证法可能更好,但并非总是如此。如果你发现自己陷入困境,可以考虑尝试原命题的否定形式——即 $P \land \neg Q$ 和 $Q \land \neg P$——以查看是否能帮助解决问题。有时这种策略非常值得一试!

\subsubsection*{中介证法 (TFAE)}

由于缺乏更好的术语,我们将这种策略称为\textbf{中介证法}。正如你将看到的,它既非完全的直接证法,也非间接证法。使用这一策略时,我们不必考虑逻辑否定,但也没有直接将陈述 $P$ 和 $Q$ 联系起来。

相反,该方法要求我们找到一个\emph{中间}陈述 $R$,并证明两个双向条件陈述:即 $P \iff R$ 和 $R \iff Q$。这会产生以下条件陈述链:
\[P \iff R \iff Q\]
它表明所有三个陈述具有相同的真值。特别地,$P$ 和 $Q$ 必然始终同真或同假,因此我们得出 $P \iff Q$ 的结论。

\textbf{TFAE} 是``The Following Are Equivalent''的缩写,意为``以下是等价的''。我们选择用此缩写命名策略,是因为它是数学中的常见短语,用于提出一系列``相互蕴涵''的条件或属性的定理中。也就是说,一些定理列出多个属性并断言它们在逻辑上等价,故称``以下是等价的''。为证明此类定理,我们需要反复使用上述策略,证明这些陈述确实等价。唯一的区别是我们必须构造中间陈述(但提出和证明 TFAE 定理的人也必须首先定义所有陈述)。

\begin{center}
    \noindent \fcolorbox{blue}{white}{%
        \parbox{0.85\textwidth}{%
            \linespread{1.5}\selectfont
            \textcolor{blue}{\textbf{策略:}}\\
            声明:$P \iff Q$ \\
            \emph{中介策略:} \\
            \hspace*{1cm} 定义陈述 $R$。

            \hspace*{1cm} 证明 $P \iff R$(使用上面介绍的任意一种策略)。

            \hspace*{1cm} 证明 $R \iff Q$(使用上面介绍的任意一种策略)。
        }
    }
\end{center}

\clearpage