% !TeX root = ../../../book.tex
\subsection{习题}\label{sec:section4.9.9}

\subsubsection*{温故知新}

以口头或书面的形式简要回答以下问题。这些问题全都基于你刚刚阅读的内容,所以如果忘记了具体的定义、概念或示例,可以回去重读相关部分。确保在继续学习之前能够自信地回答这些问题,这将有助于你的理解和记忆!

\begin{enumerate}[label=(\arabic*)]
    \item $\exists$ 声明的直接证法是什么?证明对象存在的重要步骤是什么?
    \item $\implies$ 声明的直接证法是什么?如何通过反证法证明 $\implies$ 声明?这些方法有何不同?
    \item 如何证明 $\iff$ 声明?
    \item 什么是算数几何平均不等式?该缩写从何而来?
    \item 逆否策略适用于哪种类型的声明?为什么它有效?
    \item 什么是反例
    \item ``$\exists a \in A \centerdot P(a)$'' 和 ``$\exists a \in A \centerdot P(a)$, 给定这样一个 $a$'' 有什么区别?
\end{enumerate}

\subsubsection*{小试牛刀}

尝试回答以下问题。这些题目要求你实际动笔写下答案,或(对朋友/同学)口头陈述答案。目的是帮助你练习使用新的概念、定义和符号。题目都比较简单,确保能够解决这些问题将对你大有帮助!

\begin{enumerate}[label=(\arabic*)]
    \item 证明 $\forall x \in \mathbb{R} \centerdot x^2 \ne 1 \implies x \ne 1$。
    \item 证明 $\forall n \in \mathbb{N} \centerdot n \ge 5 \implies 2n^2 > (n+1)^2$ \label{exc:exercises4.9.2}
    \item 用逻辑符号表达下列命题,然后证明。
        \begin{quote}
            存在一个偶自然数,可以用两种不同的方式写成两个质数之和。
        \end{quote}
    \item 证明每个自然数要么小于 $\sqrt{10}$ 要么大于 $3$。即,证明
        \[\forall n \in \mathbb{N} \centerdot n<\sqrt{10} \lor x>3\]
    \item 设 $A,B,C,D$ 为集合。证明,如果 $A \cup B \subset C \cup D$ 且 $C \subset A$ 且 $A \cap B = \varnothing$,则 $B \subset D$。
    \item 定义 $P = \{y \in \mathbb{R} \mid y > 0\}$。证明
        \[\forall \varepsilon \in P \centerdot \forall x, y \in \mathbb{R} \centerdot \exists \delta \in P \centerdot |x - y| < \delta \implies |(3x - 4) - (3y - 4)| < \varepsilon\]
        你还能证明以下声明吗?和上面的声明有什么不同呢?
        \[\forall \varepsilon \in P \centerdot \exists \delta \in P \centerdot \forall x, y \in \mathbb{R} \centerdot  |x - y| < \delta \implies |(3x - 4) - (3y - 4)| < \varepsilon\]
    \item 设 $E(x)$ 为命题 ``$x$ 为偶数''。证明
        \[\forall a, b \in \mathbb{Z} \centerdot E(a) \land E(b) \iff E(a + b) \land E(a \cdot b)\]
    \item 回顾一下 $\sqrt{2}$ 是无理数的证明。修改它以证明 $\sqrt{3}$ 也是无理数。\\
        尝试用相同的思路证明 $\sqrt{6}$ 是无理数。\\
        \textbf{挑战}:你能证明 $\sqrt{p}$ 对于每个质数 $p$ 都是无理数吗?
    \item 证明有无限多个有理数。\\
        \textbf{提示}:用反证法来证明。(假设有有限多个有理数……)
\end{enumerate}