% !TeX root = ../../../book.tex
\subsection{习题}\label{sec:section4.9.9}

\subsubsection*{温故知新}

以口头或书面的形式简要回答以下问题。这些问题全都基于你刚刚阅读的内容,如果忘记了具体定义、概念或示例,可以回顾相关内容。确保在继续学习之前能够自信地作答这些问题,这将有助于你的理解和记忆!

\begin{enumerate}[label=(\arabic*)]
    \item $\exists$ 声明的直接证明方法是什么?证明存在性的关键步骤是什么?
    \item $\implies$ 声明的直接证明方法是什么?如何用反证法证明 $\implies$ 声明?这些方法有何区别?
    \item 如何证明 $\iff$ 声明?
    \item 什么是算术几何平均不等式?其缩写来源是什么?
    \item 逆否策略适用于哪种类型的声明?为什么有效?
    \item 什么是反例?
    \item ``$\exists a \in A \centerdot P(a)$''和``$\exists a \in A \centerdot P(a)$, 并给定这样一个 $a$'' 有什么区别?
\end{enumerate}

\subsubsection*{小试牛刀}

尝试解答以下问题。这些题目需动笔书写或口头阐述答案,旨在帮助你熟练运用新概念、定义及符号。题目难度适中,确保掌握它们将大有裨益!

\begin{enumerate}[label=(\arabic*)]
    \item 证明 $\forall x \in \mathbb{R} \centerdot x^2 \ne 1 \implies x \ne 1$。
    \item 证明 $\forall n \in \mathbb{N} \centerdot n \ge 5 \implies 2n^2 > (n+1)^2$ \label{exc:exercises4.9.2}
    \item 用逻辑符号表达下列命题,然后证明。
        \begin{quote}
            存在一个偶自然数,可以用两种不同方式写成两个质数之和。
        \end{quote}
    \item 证明每个自然数要么小于 $\sqrt{10}$ 要么大于 $3$。即证明
        \[\forall n \in \mathbb{N} \centerdot n<\sqrt{10} \lor x>3\]
    \item 设 $A,B,C,D$ 为集合。证明,若 $A \cup B \subset C \cup D$, $C \subset A$ 且 $A \cap B = \varnothing$,则 $B \subset D$。
    \item 定义 $P = \{y \in \mathbb{R} \mid y > 0\}$。证明
        \[\forall \varepsilon \in P \centerdot \forall x, y \in \mathbb{R} \centerdot \exists \delta \in P \centerdot |x - y| < \delta \implies |(3x - 4) - (3y - 4)| < \varepsilon\]
        你还能证明以下命题吗?它与上述命题有何不同?
        \[\forall \varepsilon \in P \centerdot \exists \delta \in P \centerdot \forall x, y \in \mathbb{R} \centerdot  |x - y| < \delta \implies |(3x - 4) - (3y - 4)| < \varepsilon\]
    \item 设 $E(x)$ 为命题``$x$ 为偶数''。证明
        \[\forall a, b \in \mathbb{Z} \centerdot E(a) \land E(b) \iff E(a + b) \land E(a \cdot b)\]
    \item 回顾 $\sqrt{2}$ 是无理数的证明。修改它以证明 $\sqrt{3}$ 也是无理数。\\
        尝试用相同的思路证明 $\sqrt{6}$ 是无理数。\\
        \textbf{挑战}:你能证明对每个质数 $p$, $\sqrt{p}$ 都是无理数吗?
    \item 证明存在无穷多个有理数\\
        \textbf{提示}:使用反证法(假设仅有有限多个有理数……)。
\end{enumerate}