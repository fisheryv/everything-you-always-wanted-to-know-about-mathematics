% !TeX root = ../../../book.tex
\subsection{在证明中使用假设}\label{sec:section4.9.8}

撰写正式证明的另一个重要方面是,我们有时会获得可用的\textbf{假设}。当我们陈述一个定理时,它通常包含一些\textbf{假设}和一个\textbf{结论}。我们可以暂时将这些假设纳入数学工具包,并利用它们推导出期望的结论。同样地,在此过程中,我们可能发现额外的事实或观察,并保留它们以用于证明结论。在本小节中,我们将探讨在证明中使用假设时可能出现的三个观察结果和问题。

\subsubsection*{``$P \lor Q$''意味着分情况讨论}

如果在证明中的某个时刻,你假设或推导出 $P \lor Q$ 成立,该如何继续?知道析取成立意味着至少一个成分命题——$P$ 或 $Q$——成立。因此,你可以分别考虑这两种情况。例如,证明中可能包含如下部分:
\begin{quote}
    因为 $P \lor Q$,我们有两种情况。

    \qquad 情况 1:假设 $P$ 成立。则……

    \qquad 情况 2:假设 $Q$ 成立。则……
\end{quote}
只要在两种情况下能达成目标,你就可以得出结论。

请注意,无需考虑 $P$ 和 $Q$ 同时成立的情况。一方面,这并非必然发生;另一方面,如果最终仅需其中一个命题来推导出结论,就没有必要同时假设两者。

我们已在某些证明中经常使用分情况讨论。现在,我们确切了解其有效性!当存在潜在的析取命题时,我们采用分情况讨论。

\subsubsection*{``存在'' vs. ``给定''}

这是一个微妙但重要的区别。当你在证明中写下如下的存在性陈述:
\[\exists x \in S \centerdot P(x)\]
你在声明什么?严格来说,你只是断言该陈述为\verb|真|:你宣称\emph{确实}存在某个 $x \in S$ 满足性质 $P(x)$。然而,若你随后直接引用这个 $x$,这是不合法的!因为\emph{存在性断言}本身并未给你提供该断言的一个\emph{具体实例}。可能存在多个满足条件的 $x$,你究竟要讨论所有这样的元素,还是其中某一个?不要让读者凭直觉揣测你的意图!

如果你已知或假设某个存在性陈述(如上所述),并且需要实际引入一个满足该条件的变量,请使用下面这个明确的表达:
\begin{quote}
    ``给定这样一个 $x$。''
\end{quote}
这向读者表明:你不仅承认这样的 $x$ 存在,还将在后续证明中使用它。此后,在论证中字母 $x$ 即代表该具有性质 $P$ 的元素。

若需引入多个满足存在性断言的变量,可以使用类似的短语并稍微变动动词。例如:
\begin{quote}
    …… 因此我们得出 $\exists x, y, z \in \mathbb{Z}$ 使得 $P(x, y, z)$ 成立。给定这样的 $x, y, z$。考察 ……
\end{quote}

\subsubsection*{``$P \implies Q$'' vs. ``$P$, 所以 $Q$''}

此区别与前述例子类似:核心在于陈述命题本身与利用该命题进行推理之间的差异。正如存在性声明不等同于引入具体对象,断言一个条件命题 $P \implies Q$ 也不等同于实际推导出 $Q$。

从技术上讲,在论文中仅写出``$P \implies Q$'',并不能断言 $Q$ 成立。你必须清晰地告知读者:你同时知晓 $P$ 为真,并且正在运用该蕴含关系推导 $Q$。

回顾第 \ref{sec:section4.5.6} 节中关于分离规则的讨论。若需实际推导 $Q$,应写作如下形式:
\begin{quote}
    $P \implies Q$,因为 ……

    且 $P$ 成立,因为 ……

    因此,$Q$ 成立。
\end{quote}