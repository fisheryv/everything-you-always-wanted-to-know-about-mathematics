% !TeX root = ../../../book.tex
\subsection{在证明中使用假设}\label{sec:section4.9.8}

创建和编写正式证明的另一个重要方面是,我们有时会得到要使用的\textbf{假设}。当我们陈述一个定理时,通常会涉及一些\textbf{假设}和一个\textbf{结论}。我们可以暂时将这些假设添加到我们的数学工具包中;我们可以利用它们得出期望的结论。同样地,在此过程中,我们可能会发现一些进一步的事实和观察,并且我们可以保留这些事实和观察,并用它们来证明结论。在这一小节中,我们想指出在证明中使用假设时可能出现的三个观察结果和问题。

\subsubsection*{``$P \lor Q$'' 意味着分情况讨论}

如果在证明中的某个时刻,你假设或推论 $P \lor Q$ 成立,那该如何继续呢?知道这个析取成立意味着至少一个成分陈述——$P$ 或 $Q$——成立。因此,你可以分别考虑这两种情况。例如,证明中可能包含如下部分:
\begin{quote}
    因为 $P \lor Q$,我们有两种情况。

    \qquad 情况 1:假设 $P$ 成立。则……

    \qquad 情况 2:假设 $Q$ 成立。则……
\end{quote}
只要你在这两种情况下都能达到你想要的目标,你就可以得出推论。

请注意,无需考虑 $P$ 和 $Q$ 都成立的情况。其一,这可能不一定会发生。而且,如果你最终仅使用其中一个陈述来推断出您想要的结论,那么就没有必要暂时假设两个陈述。

我们在某些证明中一直使用分情况讨论。现在,我们确切了解了它们为何有效!当存在潜在的析取陈述时,我们使用分情况讨论。

\subsubsection*{``存在'' vs. ``给定''}

这是一个微妙但重要的区别。如果你在证明中写下类似
\[\exists x \in S \centerdot P(x)\]
这样的声明,你声明了什么呢?从技术上讲,你实际上只是声明上面的陈述是一个为\verb|真|的声明;你断言\emph{确实}存在某些 $x \in S$ 具有 $P(x)$ 属性。但是,如果你随后开始引用 $x$ …… 那么这是无效的!在\emph{存在}断言中,你没有引入该断言的\emph{特定实例}。可能存在多个这样的 $x$ 元素。你是想讨论所有这些元素吗?或者只是一个特定元素?不要让读者凭直觉准确地理解你要做什么!

如果你知道或假设某些存在声明(如上面的声明),并且你想实际引入一个满足该存在声明的变量,请使用下面这个美妙的短语:
\begin{quote}
    ``给定这样一个 $x$。''
\end{quote}
这向读者发出信号,表明你不仅在说这样的 $x$ 存在,而且还将在证明中将其发挥作用。对于书面论证的其余部分,你希望字母 $x$ 代表具有该属性的元素。此后,你可以通过名称引用该对象 $x$。

如果你断言多个变量存在并想要引入它们,只需使用类似的短语和稍微不同的动词即可。例如,你可能会写这样的内容:
\begin{quote}
    …… 因此我们推论 $\exists x, y, z \in \mathbb{Z}$ 使得 $P(x, y, z)$ 成立。给定这样的 $x、y、z$。考察 ……
\end{quote}

\subsubsection*{``$P \implies Q$'' vs. ``$P$, 所以 $Q$''}

这种区别与我们上面提到的例子类似。具体来说,编写声明来断言其有效性与编写声明向读者表明你正在从中得出结论之间存在区别。在最后一个例子中,这就是声明某物存在与引入这样一个对象之间的区别。

这里的区别在于,断言条件陈述(如 $P \implies Q$)表示此条件关系存在与使用该陈述来推断 $Q$ 成立。从技术上讲,在论文中仅仅写下``$P \implies Q$'' 并不能断言 $Q$ 成立。你必须让你的读者非常清楚你也知道 $P$ 并且正在使用条件陈述来推导 $Q$。

回顾一下我们在 \ref{sec:section4.5.6} 节中的讨论。在那里,我们描述了这一重要区别,并提到了 ``分离规则'' 的方法。正如我们提到的,如果你想实际推导 $Q$,你应该编写如下内容:
\begin{quote}
    $P \implies Q$ 因为 ……

    另外,$P$ 成立,因为 ……

    因此,$Q$ 成立。
\end{quote}