% !TeX root = ../../../book.tex
\subsection{反驳命题}\label{sec:section4.9.7}

我们已经讨论过(并在许多例子中验证过)如何\textbf{证明}各类数学命题。这很棒!但你可能会问:``如果我想\textbf{反驳}某个命题呢?''对此的回答简明扼要:\emph{二者并无区别}。

反驳一个命题即证明其不成立。根据逻辑否定的定义,这等价于证明该命题的否定形式成立。因此,只需构造出逻辑否定,再运用本节所述的证明策略即可。

为了便于说明,我们考察``$\forall$''命题的反驳。反驳``$\forall$''命题意味着证明``$\exists$''命题,此时\textbf{反例}的概念至关重要。

\subsubsection*{反例}

一般而言,\textbf{反例}是反驳全称量化命题的具体实例。它通过证明存在反例(即``$\exists$''声明),表明原命题不成立——该实例不具备所声称的性质。

\begin{example}
    回顾例 \ref{ex:example4.9.8}。其中,我们定义了集合
    \[S = {x \in \mathbb{R} \mid x > 0}\]

    然后定义了两个变量命题:
    \[P(x) \text{\ 为\ } \frac{x-3}{x+2}>1-\frac{1}{x} \]
    \[Q(x) \text{\ 为\ } \frac{x+3}{x+2}<1+\frac{1}{x} \]

    接着我们证明了
    \[\forall x \in S \centerdot P(x) \implies Q(x)\]

    本例中,我们考虑如下命题
    \[\forall x \in S \centerdot Q(x) \implies S(x)\]

    具体来说,我们将反驳此命题。在开始前,建议你先行探究:尝试证明该命题(尽管已知其为\verb|假|)。注意证明过程中何处失效?为何失效?这些观察能否帮助你构造反例?请先自行探索,再继续阅读。

    \begin{center}
        \noindent \fcolorbox{red}{white}{%
        \parbox{0.85\textwidth}{%
            \linespread{1.5}\selectfont
            \textcolor{red}{\textbf{草稿:}}\\
            首先,我们需要对要反驳的命题进行逻辑否定:
            \[\exists x \in S \centerdot Q(x) \land \neg P(x)\]

            这意味着要找到一个满足以下三个条件的实数 $x$:
            \begin{enumerate}[label=(\arabic*)]
                \item $x > 0$
                \item $\frac{x+3}{x+2}<1+\frac{1}{x}$
                \item $\frac{x-3}{x+2} \le 1-\frac{1}{x}$
            \end{enumerate}

            有多种策略可供选择:如尝试(错误地)证明第一个不等式蕴含第二个不等式以定位失效点,或采用系统化的试值法。

            无论选择何种方法,已知 $x \in \mathbb{R}$ 且 $x > 0$,意味着可以考虑 $x$ 的临界值:趋近于零的极小值或趋向无穷的极大值。

            先尝试``较小''值似乎更简便。取 $x = 1$,可得:
            \begin{enumerate}[label=(\arabic*)]
                \item 成立,因为 $1 > 0$;
                \item 成立,因为 $\frac{4}{3} < 2$;
                \item 成立,因为 $-\frac{2}{3} < 0 \le 0$。
            \end{enumerate}
            酷,问题得证!
        }
        }
    \end{center}

    \begin{proof}
        此处反驳命题 $\forall x \in \mathbb{R} \centerdot Q(x) \implies P(x)$。

        考虑 $x = 1$。显然 $x \in \mathbb{R}$ 且 $x > 0$。

        此时 $Q(1)$ 成立,因为
        \[\frac{1+3}{1+2} = \frac{4}{3} < 2=1+\frac{1}{1}\]

        而 $P(1)$ 不成立,因为
        \[\frac{1-3}{1+2} = -\frac{2}{3} \not{>} 0=1-\frac{1}{1}\]

        因此证得
        \[\exists x \in S \centerdot Q(x) \land \neg P(x)\]

        这反驳了原命题。
    \end{proof}
\end{example}