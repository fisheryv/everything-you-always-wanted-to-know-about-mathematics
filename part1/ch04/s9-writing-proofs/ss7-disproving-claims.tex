% !TeX root = ../../../book.tex
\subsection{反驳主张}\label{sec:section4.9.7}

我们现在已经讨论(并且在许多例子中看到)了如何\textbf{证明}任意类型的数学陈述。棒极了!但你可能会说,``呃……如果我想\textbf{反驳}某个陈述怎么办?''我们对这个问题的回答简短而贴心:\emph{没有区别}。 

反驳一个陈述意味着你想证明它的真值为\verb|假|。根据逻辑否定的定义,这意味着你想要证明陈述的否定为\verb|真|。因此,你可以使用我们在本节中探讨的任意策略得到并写出该逻辑否定并证明该陈述为\verb|真|。

为了便于说明,让我们看个实际例子。具体来说,我们想要反驳 ``$\forall$'' 主张,这意味着我们想要证明 ``$\exists$''主张。这就是\textbf{反例}概念发挥作用的地方。

\subsubsection*{反例}

一般来说,\textbf{反例}是反驳全称量化陈述的实例。因为它可以证明 ``$\exists$'' 声明,并且得到\emph{相反}的结论,因此它表明这个特定的例子不具有所声明的属性。\\

\begin{example}
    回顾例 \ref{ex:example4.9.8}。其中,我们定义了集合
    \[S = {x \in \mathbb{R} \mid x > 0}\]

    然后定义了两个变量命题:
    \begin{align*}
        P(x) \;\text{为}\; \frac{x-3}{x+2}>1-\frac{1}{x} \\
        Q(x) \;\text{为}\; \frac{x+3}{x+2}<1+\frac{1}{x}
    \end{align*}

    接着我们证明
    \[\forall x \in S \centerdot P(x) \implies Q(x)\]

    本例中,我们考虑如下命题
    \[\forall x \in S \centerdot Q(x) \implies S(x)\]
\end{example}

具体来说,我们要反驳它。不过,在我们开始之前,请你自行研究一下该声明。试着证明一下它,尽管我们已经告诉你它为\verb|假|!你是否发现你的``证据''在某个地方失效了?为什么会发生这种情况?你能通过你的观察来帮助你找到该主张的反例吗?看看你能找到什么,然后继续阅读。

\begin{center}
    \noindent \fcolorbox{red}{white}{%
    \parbox{0.85\textwidth}{%
        \linespread{1.5}\selectfont
        \textcolor{red}{\textbf{草稿:}}\\
        首先,我们需要对我们所反驳的主张进行逻辑否定:
        \[\exists x \in S \centerdot Q(x) \land \neg P(x)\]

        这意味着我们需要找到一个满足三个条件的特定实数 $x$:
        \begin{enumerate}[label=(\arabic*)]
            \item 不等式 $x > 0$
            \item 不等式 
                \[\frac{x+3}{x+2}<1+\frac{1}{x}\]
            \item 不等式
                \[\frac{x-3}{x+2} \le 1-\frac{1}{x}\]
        \end{enumerate}

        我们有几个策略可以选择,就像我们上面提到的那样,我们可以尝试(当然是错误的)证明第一个不等式蕴含第二个不等式,并确定它在哪里失效。或者,我们可以使用``有根据的猜测''法``尝试一些值''。

        无论如何,知道 $x \in \mathbb{R}$ 且 $x > 0$ 表明我们可以尝试 $x$ 的``极端''值。这意味着``极小'' $x$(即 $x$ 接近 $0$)或``极大'' $x$(即不断增加的 $x$ 值,直到我们找到一个有效值为止)。

        首先使用一些``较小''值似乎更容易,所以让我们尝试 $x = 1$。我们看到 (1) 成立,因为 $1 > 0$,(2) 也成立,因为 $\frac{4}{3} < 2$,(3) 也成立,因为 $-\frac{2}{3} < 0 \le 0$。酷,就是这样!
    }
    }
\end{center}

\begin{proof}
    这里,我们将反驳 $\forall x \in \mathbb{R} \centerdot Q(x) \implies P(x)$ 这一主张。

    考虑 $x = 1$。注意 $x \in \mathbb{R}$ 并且 $x > 0$。

    另外,请注意 $Q(1)$ 成立,因为
    \[\frac{1+3}{1+2} = \frac{4}{3} < 2=1+\frac{1}{1}\]

    并且,请注意 $P(1)$ 不成立,因为
    \[\frac{1-3}{1+2} = -\frac{2}{3} \not{>} 0=1-\frac{1}{1}\]

    因此,我们证明
    \[\exists x \in S \centerdot Q(x) \land \neg P(x)\]

    这反驳了主张。
\end{proof}