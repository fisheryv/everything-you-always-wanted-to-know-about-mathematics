% !TeX root = ../../../book.tex
\subsection{证明 $\land$ 声明}\label{sec:section4.9.4}

``$\land$''声明断言两个陈述都为\verb|真|。其证明方法直接明了:先证明第一个命题,再证明第二个命题!

以下示例展示此方法的实际应用。由于本例中的 $\land$ 语句位于存在量词 $\exists$ \emph{之后},因此需要构造一个同时满足两个属性的对象。这是首个演示如何组合证明策略来处理量词与连词并存的示例。

\subsubsection*{直接证法}

\begin{center}
    \noindent \fcolorbox{blue}{white}{%
        \parbox{0.85\textwidth}{%
            \linespread{1.5}\selectfont
            \textcolor{blue}{\textbf{策略:}}\\
            声明:$P \land Q$ \\
            \emph{直接证明策略:} \\
            \hspace*{1cm} 证明 $P$ 成立。\\
            \hspace*{1cm} 证明 $Q$ 成立。
        }
    }
\end{center}

\begin{example}[两数中较小数的平方可能更大]
    \begin{center}
        陈述:$ \forall x \in \mathbb{R} \centerdot \exists y \in \mathbb{R} \centerdot (x \ge y \land x^2 < y^2)$。
    \end{center}

    \begin{center}
        \noindent \fcolorbox{red}{white}{%
        \parbox{0.85\textwidth}{%
            \linespread{1.5}\selectfont
            \textcolor{red}{\textbf{草稿:}}\\
            让我们取一个特定的 $x$,比如 $x = 4$。我们需要找到一个小于 $x=4$ 但平方大于 $x^2 = 16$ 的实数。

            关键在于 $y \in \mathbb{R}$,所以我们可以使用负数。在这种情况下,选择一个较大的负数,例如 $y = -5$ 即满足条件。

            让我们再取一个不同的 $x$,比如 $x = -2$。因为 $x$ 为负数,所以只要选择任意更小的数字,比如 $y = -3$ 即可。

            故证明将按 $x>0$ 和 $x \leq 0$ 分两种情况。
        }
        }
    \end{center}

    证明如下:

    \begin{center}
        \noindent \fcolorbox{olivegreen}{white}{%
            \parbox{0.85\textwidth}{%
                \linespread{1.5}\selectfont
                % \textcolor{olivegreen}{\textbf{实现:}}
                \begin{proofs}{证明 1:}
                    设 $x \in \mathbb{R}$ 为任意固定实数。我们分两种情况讨论。
                    
                    \begin{itemize}
                        \item 假设 $x \le 0$。
                        
                            定义 $y = x - 1$。注意,此时 $y \in \mathbb{R}$ 且 $y < x$。

                            由此可得
                            \[y^2 = (x-1)^2 = x^2-2x+1 = x^2-(2x-1)\]
                            因为 $x \le 0$,所以 $2x \le 0$,因此 $2x-1 \le -1$,由此可得
                            \[x^2-(2x-1) \ge x^2+1 > x^2\]
                            因此 $y^2 > x^2$。
                        \item 假设 $x > 0$。
                            定义 $y = -x - 1$。注意,此时 $y \in \mathbb{R}$ 且 $y < 0, x>0$,所以 $y \le x$(实际有 $y < x$)。

                            由此可得
                            \[y^2 = (-x-1)^2 = x^2+2x+1 = x^2+(2x+1)\]
                            因为 $x > 0$,所以 $2x+1 > 0$,由此可得
                            \[x^2+(2x+1) > x^2\]
                            因此 $y^2 > x^2$。
                    \end{itemize}
                    综上,无论哪种情况,均存在 $y \in \mathbb{R}$ 满足 $y \le x$ 且 $x^2 < y^2$。故命题成立。
                \end{proofs}
            }
        }
    \end{center}
\end{example}

我们为何称其为``证明 1''?在初步分析中,我们依据 $x$ 的符号分情况定义 $y$(用 $x$ 表示)。实际上,该证明可以不依赖分情况讨论,这就是``证明 2'',我们希望你来完成它!目标是以通用方式定义 $y$(用 $x$ 表示),并重写证明,使其对任意 $x$ 均成立。(提示:当 $x < 0$ 时 $-x$ 是什么?这是否是已知函数?)

\subsubsection*{间接证法(反证法)}

与其他间接证明方法类似,反证法的核心在于:首先否定待证命题,然后假设该否定成立,并由此推导出矛盾。这表明假设不成立,从而原命题为\verb|真|。

请尝试用此方法证明上面命题。(注:建议先完成前述``证明 2''再进行本练习。)完成后请比较两种证明方法的异同,并思考你更倾向于采用哪种方法。

\begin{center}
    \noindent \fcolorbox{blue}{white}{%
        \parbox{0.85\textwidth}{%
            \linespread{1.5}\selectfont
            \textcolor{blue}{\textbf{策略:}}\\
            声明:$P \land Q$ \\
            \emph{间接证明策略:} \\
            \hspace*{1cm} 为了引出矛盾而假设,$\neg P \lor \neg Q$ 成立。\\
            \hspace*{1cm} 考虑第一种情况,即 $\neg P$ 成立,推导得出矛盾。\\
            \hspace*{1cm} 考虑第二种情况,即 $\neg Q$ 成立,推导得出矛盾。
        }
    }
\end{center}

\clearpage