% !TeX root = ../../../book.tex


\subsection{证明 $\lor$ 声明}\label{sec:section4.9.3}

``$\lor$'' 声明断言两个陈述中至少有一个为\verb|真|。如果碰巧这两个陈述之一显然为\verb|假|,那么就尝试证明另一个为\verb|真|。这就是这里的直接证法;它很简单,因此我们不会提供实现示例。

\subsubsection*{直接证法}

\begin{center}
\noindent \fcolorbox{blue}{white}{%
    \parbox{0.85\textwidth}{%
        \linespread{1.5}\selectfont
        \textcolor{blue}{\textbf{策略:}}\\
        声明:$P \lor Q$ \\
        \emph{直接证明策略:} \\
        \hspace*{1cm} 证明 $P$ 成立,否则证明 $Q$ 成立。
    }
}
\end{center}

当然,直接证法依赖于你能够提前获知哪一个陈述($P$ 还是 $Q$)为\verb|真|。如果你能做到这一点,那么这甚至不是一个真正的``策略''。只需实施适用于 $P$(或 $Q$,视情况而定)的任何策略。

\subsubsection*{间接证法(证明``另一种情况'')}

这种方法比直接证法有趣得多。一般来说,当陈述 $P$ 和 $Q$ 实际上是变量命题,并且对于某些实例 $P$ 为\verb|真|,而对于其他实例 $Q$ 为\verb|真|时,这种证法非常有用。在这种情况下,我们可以直接说:``如果 $P$ 为\verb|真|,那么我们的证明就已经完成了,而不是准确地描述哪些实例满足 $P$,哪些实例满足 $Q$。因此,我们需要担心的是 $P$ 为\verb|假|的情况;对于这些情况,我们需要证明 $Q$ 仍然为\verb|真|。''

\begin{center}
    \noindent \fcolorbox{blue}{white}{%
        \parbox{0.85\textwidth}{%
            \linespread{1.5}\selectfont
            \textcolor{blue}{\textbf{策略:}}\\
            声明:$P \lor Q$ \\
            \emph{间接证明策略 1:} \\
            \hspace*{1cm} 假设 $\neg P$ 成立,证明 $Q$ 成立。
        }
    }
\end{center}

\begin{example}
    当一个实数小于它的平方时,
  
    陈述:假设 $ x \in \mathbb{R}$ 且 $x^2 \ge x$。

    我们声明 $x \ge 1$ 或 $x \le 0$。
\end{example}

\begin{center}
    \noindent \fcolorbox{olivegreen}{white}{%
        \parbox{0.85\textwidth}{%
            \linespread{1.5}\selectfont
            \textcolor{olivegreen}{\textbf{实现:}}
            \begin{proof}
                设 $x \in \mathbb{R}$ 是任意固定的,并假设 $x^2 > x$。
                
                如果 $x \le 0, x^2 \ge x$ 显然成立。于是我们假设另一种情况,即 $x>0$。

                根据条件,$x^ \ge x$,因为 $x > 0$,我们可以不等式两边同时除以 $x$,得 $x \ge 1$。
            \end{proof}
        }
    }
\end{center}

这证明了实数小于(或等于)其平方的必要条件。这个条件(即 $x \ge 1 \lor x \le 0$)也是充分条件吗?请你证明一下!其实很简单,一旦你证出来了,将二者合在一起就可以证明这个双条件陈述:
\[\forall x \in \mathbb{R} \centerdot x^2 \ge x \iff (x \ge 1 \lor x \le 0)\]


\subsubsection*{间接证法(反证法)}

这种证法与上面的间接证法很像,我们都会假设逻辑否定成立,然后推断出一些荒谬的东西。我们通过将其应用于前一个示例的相同声明来说明该策略。

\begin{center}
    \noindent \fcolorbox{blue}{white}{%
        \parbox{0.85\textwidth}{%
            \linespread{1.5}\selectfont
            \textcolor{blue}{\textbf{策略:}}\\
            声明:$\forall x \in S \centerdot P(x)$ \\
            \emph{间接证明策略 2:} \\
            \hspace*{1cm} 为了得到矛盾而假设,$\neg P \land \neg Q$ 成立。推导得出矛盾。
        }
    }
\end{center}

\begin{center}
    \noindent \fcolorbox{olivegreen}{white}{%
        \parbox{0.85\textwidth}{%
            \linespread{1.5}\selectfont
            \textcolor{olivegreen}{\textbf{实现:}}
            \begin{proof}
                设 $x \in \mathbb{R}$ 是任意固定的,并假设 $x^2 > x$。
                
                为了得到矛盾而假设,$0 < x$ 且 $x < 1$。

                因为 $x>0$,我们可以在不等式两边同时乘以 $x$ 并保持符号不变。

                $x<1$ 两边同时乘以 $x$,那么我们得到 $x^2<x$。

                这与我们假设 $x^2 > x$ 矛盾。$\hashx$

                因此我们的假设不成立。该声明为真。
            \end{proof}
        }
    }
\end{center}

这种证法与之前的证法相比如何?我们证明的是完全相同的主张,但证法略有不同。你认为哪个更好?哪个更容易书写?此外,你能回过头用量词和``$\implies$''重写原来的声明吗?完成之后,你明白这两个证明完成了什么吗?尝试一下!