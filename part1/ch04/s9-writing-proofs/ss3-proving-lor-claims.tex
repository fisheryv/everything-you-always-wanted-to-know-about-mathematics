% !TeX root = ../../../book.tex

\subsection{证明 $\lor$ 声明}\label{sec:section4.9.3}

``$\lor$''声明断言两个陈述中至少有一个为\verb|真|。若其中一个陈述明显为\verb|假|,则应尝试证明另一个为\verb|真|。这就是直接证法,因其简单易懂,此处不提供实现示例。

\subsubsection*{直接证法}

\begin{center}
\noindent \fcolorbox{blue}{white}{%
    \parbox{0.85\textwidth}{%
        \linespread{1.5}\selectfont
        \textcolor{blue}{\textbf{策略:}}\\
        声明:$P \lor Q$ \\
        \emph{直接证明策略:} \\
        \hspace*{1cm} 证明 $P$ 成立,或证明 $Q$ 成立。
    }
}
\end{center}

此方法依赖预先判断 $P$ 或 $Q$ 的真假。若能确定其中之一为\verb|真|,则直接采用相应的证明方法即可。

\subsubsection*{间接证法(证明``另一种情况'')}

这种方法比直接证法更为有趣。通常,当陈述 $P$ 和 $Q$ 是变量命题,且对于某些情况 $P$ 为\verb|真|,而对于其他情况 $Q$ 为\verb|真|时,这种证法尤为适用。此时,我们可以直接声明:``如果 $P$ 为\verb|真|,则证明即告完成;因此,我们无需精确描述满足 $P$ 或 $Q$ 的具体实例。我们只需关注 $P$ 为\verb|假|的情况,并证明此时 $Q$ 仍为\verb|真|。''

\begin{center}
    \noindent \fcolorbox{blue}{white}{%
        \parbox{0.85\textwidth}{%
            \linespread{1.5}\selectfont
            \textcolor{blue}{\textbf{策略:}}\\
            声明:$P \lor Q$ \\
            \emph{间接证明策略 1:} \\
            \hspace*{1cm} 假设 $\neg P$ 成立,证明 $Q$ 成立。
        }
    }
\end{center}

\begin{example}
    \begin{center}
        陈述:设 $ x \in \mathbb{R}$ 且 $x^2 \ge x$,则 $x \ge 1 \lor x \le 0$。
    \end{center}

    \begin{center}
        \noindent \fcolorbox{olivegreen}{white}{%
            \parbox{0.85\textwidth}{%
                \linespread{1.5}\selectfont
                % \textcolor{olivegreen}{\textbf{实现:}}
                \begin{proof}
                    设 $x \in \mathbb{R}$ 为任意固定实数,并假设 $x^2 > x$。
                    
                    如果 $x \le 0, x^2 \ge x$ 显然成立。于是考虑 $x>0$ 的情况。

                    根据条件 $x^2 \ge x$ 且 $x > 0$,不等式两边同时除以 $x$,得 $x \ge 1$。
                \end{proof}
            }
        }
    \end{center}

    这证明了实数小于(或等于)其平方的必要条件。该条件(即 $x \ge 1 \lor x \le 0$)是否也是充分条件?请你证明一下!一旦完成,二者结合即可证明双向条件陈述:
    \[\forall x \in \mathbb{R} \centerdot x^2 \ge x \iff (x \ge 1 \lor x \le 0)\]
\end{example}

\subsubsection*{间接证法(反证法)}

这种证法与前述间接证法类似:我们首先假设命题的逻辑否定成立,然后推导出矛盾结果。以下通过前述命题的证明示例说明该策略。

\begin{center}
    \noindent \fcolorbox{blue}{white}{%
        \parbox{0.85\textwidth}{%
            \linespread{1.5}\selectfont
            \textcolor{blue}{\textbf{策略:}}\\
            声明:$\forall x \in S \centerdot P(x)$ \\
            \emph{间接证明策略 2:} \\
            \hspace*{1cm} 为了引出矛盾而假设,$\neg P \land \neg Q$ 成立。\\
            \hspace*{1cm} 推导得出矛盾。
        }
    }
\end{center}

\begin{center}
    \noindent \fcolorbox{olivegreen}{white}{%
        \parbox{0.85\textwidth}{%
            \linespread{1.5}\selectfont
            % \textcolor{olivegreen}{\textbf{实现:}}
            \begin{proof}
                设 $x \in \mathbb{R}$ 为任意固定实数,并假设 $x^2 > x$。
                
                为了引出矛盾而假设,$0 < x$ 且 $x < 1$。

                因为 $x>0$,可在不等式两边同时乘以 $x$ 并保持符号不变。

                对 $x<1$ 两边同时乘以 $x$ 可得 $x^2<x$。

                这与假设 $x^2 > x$ 矛盾。$\hashx$

                因此,假设不成立,原命题成立。
            \end{proof}
        }
    }
\end{center}

此证法与前述证法有何异同?二者证明同一命题,但路径略有差异。你认为哪种更优?哪种更易书写?此外,能否用量词与``$\implies$''重述原命题?完成后,能否理解两个证明的实质?请尝试一下!