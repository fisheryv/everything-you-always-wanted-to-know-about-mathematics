% !TeX root = ../../../book.tex
\subsection{示例和伪例}

``语法正确''是指句子中的单词和符号被正确使用和组合并且有意义。这消除了放在一起无意义的符号/单词串,例如:
\begin{center}
    \textcolor{red}{$1+ = 2$, $\text{于儿}^2 = 1$, $\{\{\varnothing\}\} - 7 > 5\pi$, $\text{You am smart}$}
\end{center}
比如,上面的第三个不是数学陈述,因为 $\{\{\varnothing\}\}$ 不是数字,所以我们不知道如何解释从该集合中``减去 $7$''。

而``\emph{唯一}一个真值''说的是该陈述应该要么为\verb|真|要么为\verb|假|,但肯定不能两者兼而有之,或者两者都不是或介于两者之间。这排除了上面``\textcolor{red}{$x^2 - 1 = 0$}''这样的陈述,因为它没有真值。(如果没有声明 $x$ 是什么,我们就无法决定真值是什么。)

\subsubsection*{真值未知}

我们给出的定义中一个奇怪/有趣/复杂的方面是,我们可能不知道给定陈述的真值,即使我们可以确定只有一个这样的值。作为说明,请考虑以下陈述:
\begin{center}
    \textcolor{olivegreen}{任何大于或等于 $4$ 的偶自然数都可以写成两个质数之和。}
\end{center}

这个说法是真是假?如果你能证明或证伪,那么数学界会很乐意看到它!上面的陈述被称为\href{https://baike.baidu.com/item/哥德巴赫猜想/72364}{哥德巴赫猜想},它是数学中一个非常著名的未解难题(我们希望只是暂时的!)。目前还没有人知道这一说法是对是错,但可以肯定的是,\emph{只有其中一个}真值适用。也就是说,这个陈述不可能既是\verb|真|又是\verb|假|,也不可能介于两者之间。要么所有大于或等于 $4$ 的偶自然数都具有该性质,要么至少有一个不具有该性质。即使还不知道两种可能性中哪一种是正确的,我们也可以陈述这种``非此即彼''的性质。因此,这句话实际上满足了我们对\emph{数学陈述}的定义。

(术语说明:一般来说,\textbf{猜想}是某人认为正确但尚未被证明/证伪的断言。)

\subsubsection*{悖论语句}

使句子没有真值的一种方法是创造\textbf{悖论}。考虑下面这句话:
\begin{center}
    \textcolor{red}{这句话是假的。}
\end{center}
很怪异,对吧?这句话本身就断言了它自己的真值。我们来尝试分析一下它的真值:
\begin{itemize}
    \item 假设这句话为\verb|真|。那么,这个句话本身告诉我们,它实际上为\verb|假|。
    \item 假设这句话为\verb|假|。那么同理,这句话告诉我们,它实际上为\verb|真|。
\end{itemize}
这是不可能的!这句话在某种程度上既\verb|真|又\verb|假|,或者两者都不是,或者……无论是什么,都不是个好主意。我们不想在数学中处理这种奇怪现象,所以我们的定义不允许这类语句作为数学陈述。

(\emph{问题}:如果允许这样的句子成为正确的数学陈述,会发生什么?如果你不遵守我们强调的每个句子必须为\verb|真|或为\verb|假|的原则怎么办?想一想!是不是哪里会出问题,或者这只是一个不同的数学宇宙?……)

通常来说,像上面这样的\textbf{自指}句子(即指代自己的句子)是相当奇怪的,并且可能会产生一些我们想要禁止的悖论。

上述自相矛盾的说法有一个变体,通过下面这幅卡通漫画中给出,其中皮诺曹说:``我的鼻子现在要变长!'' 那么它会变长吗?如果他说真话,那么它就会变长,但只有当他说谎时才它的鼻子才会变长!如果他说谎,那么他的鼻子就会变长(根据定义),但他的说法实际上又是真的!哎呀,纠缠不清!
\begin{center}
    \includegraphics[scale=0.4]{figure/pinocchio.png}
\end{center}

这种现象的一个更奇怪的例子是\emph{奎因悖论(Quine's Paradox)}:
\begin{center}
    \textcolor{red}{``Yields falsehood when preceded by its quotation'' \\yields falsehood when preceded by its quotation.}
\end{center}
这个问题留给你自己思考。我只想说,像这样自相矛盾的说法实在是太病态了,以至于我们不必考虑它们。这就是为什么我们的定义禁止了它们。
