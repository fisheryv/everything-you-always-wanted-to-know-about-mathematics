% !TeX root = ../../../book.tex
\subsection{定义}

现在,让我们探讨\textbf{数学陈述}的含义。这个术语需要能够涵盖所有可被证明或证伪的`事物''类型。

数学在科学中具有独特性,因为其结论需经严格\textbf{证明},而非通过实验室实验或现实观察来``证实''。我们基于一组共同的\textbf{公理},通过严谨的逻辑推理,从这些公理(及已证结论)中推导真理。如果某个命题被证明为假,我们则需揭示其谬误。

以下是几个\textbf{数学陈述}或\textbf{命题}的示例(部分已被证明):
\begin{center}
    \textcolor{olivegreen}{对于任意实数 $x, y \in \mathbb{R}$,不等式 $2xy \le x^2 + y^2$ 成立。}
\end{center}
这是一个真实有效的数学陈述(将在 \ref{sec:section4.9.2} 节证明),亦称\emph{算术几何平均不等式 (Arithmetic-Geometric Mean Inequality)}不等式或 \emph{AGM} 不等式。这里需要指出,``成立''在数学中通常表示``为真''或``是一个真实的陈述''。

下面是另一个数学陈述的例子:
\begin{center}
    \textcolor{olivegreen}{对于任意集合 $S, T, U$,若 $S \cap T \subseteq U$,则 $S \subseteq U$ 或 $T \subseteq U$。}
\end{center}

然而,该命题不成立,反例如下:
\begin{center}
    \textcolor{blue}{令 $S = \{1, 2, 3\}$, $T = \{2, 3, 4\}$, $U = \{2, 3, 5\}$。\\ 显然 $S \cap T = \{2, 3\} \subseteq U$,但 $S \nsubseteq U$ 且 $T \nsubseteq U$。}
\end{center}
此反例证伪了原命题。具体分析将在后续章节展开。

以下句子:
\begin{center}
    \textcolor{red}{为什么我们早上 9:00 还要上课?!}
\end{center}
\textcolor{red}{\emph{不构成}}数学陈述。虽然这是一个有效的句子,但无法被数学\emph{证明}或\emph{证伪}。

类似地:
\begin{center}
    \textcolor{red}{$x^2 - 1 = 0$}
\end{center}
虽由数学符号组成,但仍\textcolor{red}{\emph{不构成}}数学陈述。其真值取\emph{取决于}变量 $x$ 的取值,缺乏额外假设时无法判定其真假。此类含未指定\textbf{变量}的陈述称为\textbf{变量命题}。

综合上述示例/反例,我们可以得到如下定义:

\begin{definition}
    \dotuline{数学陈述}(或称\dotuline{命题}、\dotuline{逻辑陈述})是由词语/标点及数学符号构成的语法正确的语句,具有唯一确定的真值(真或假)。
\end{definition}
