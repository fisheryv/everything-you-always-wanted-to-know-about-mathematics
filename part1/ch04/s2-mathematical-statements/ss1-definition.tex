% !TeX root = ../../../book.tex
\subsection{定义}

现在,让我们讨论一下\textbf{数学陈述}的含义。我们希望这个术语能够概括我们可以证明或证伪的``事物''类型。

数学在科学中是独一无二的,因为该领域的结果经过严格\textbf{证明},而不是先做出假设然后通过实验室实验或现实世界观察来``证实''。在数学中,我们假设一组常见的\textbf{公理},然后遵循严格的逻辑推理,从这些公理(以及迄今为止我们已经证明的其他真理)中推导出真理。如果我们遇到虚假信息,我们就必须证明或揭示它确实为假。

基于上述想法,我们来看几个\textbf{数学陈述}或\textbf{命题}可能是什么的例子。(我们甚至已经证明了其中一些!)例如,这句话:
\begin{center}
    \textcolor{olivegreen}{对于任意实数 $x, y \in \mathbb{R}$,不等式 $2xy \le x^2 + y^2$ 成立。}
\end{center}
是一个有效的数学陈述。事实上,它是真的,我们将在稍后的 \ref{sec:section4.9.2} 节中证明这一点。(它有时被称为 AGM 不等式,是\emph{算术几何平均不等式(Arithmetic-Geometric Mean Inequality)}的缩写。)这里需要指出,``成立''一词在数学中经常用于表示``为真''或``是一个真实的陈述''。

下面是另一个数学陈述的例子:
\begin{center}
    \textcolor{olivegreen}{对于任意集合 $S, T, U$,如果 $S \cap T \subseteq U$ 则 $S \subseteq U$ 或 $T \subseteq U$。}
\end{center}

然而,上面这种说法是错误的,如以下反例所示:
\begin{center}
    \textcolor{blue}{设 $S = \{1, 2, 3\}$, $T = \{2, 3, 4\}$, $U = \{2, 3, 5\}$。\\ 很明显 $S \cap T = \{2, 3\} \subseteq U$ 但 $S \nsubseteq U$ 且 $T \nsubseteq U$。}
\end{center}
为什么这个例子证伪了上面的说法?你弄明白了吗?你能解释一下吗?我们将在本章后面更详细地讨论这一点,但我们希望现在我们都认识到这个例子确实实现了这一点。

我们也都认同像这样的句子
\begin{center}
    \textcolor{red}{为什么我们早上 9:00 还要上课?!}
\end{center}
绝对\textcolor{red}{\emph{不是}}数学陈述。这是一个完全有效的句子,但从数学上来讲它没有意义:我们无法\emph{证明}或\emph{证伪}它。

同理,这句话
\begin{center}
    \textcolor{red}{$x^2 - 1 = 0$}
\end{center}
尽管完全由数学符号组成,但它也\textcolor{red}{\emph{不是}}数学陈述。问题是我们无法纯粹从公理和逻辑推论来验证它为真还是为假。该陈述\emph{取决于} $x$,无论该值是什么(即 $x$ 是一个\textbf{变量}),并且在不对其施加额外假设的情况下,我们无法断言该句子为真还是为假。这种类型的句子稍后将被称为\textbf{变量命题}:其真实性取决于句子中的变量。

所有这些观察结果和示例/伪例引出了以下定义:

\begin{definition}
    \dotuline{数学陈述}(或\dotuline{命题}或\dotuline{逻辑陈述})是语法正确的句子(或句子串),由单词/标点符号和数学符号组成,并具有唯一一个真值(真或假)。
\end{definition}
