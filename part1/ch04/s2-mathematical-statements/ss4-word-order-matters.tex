% !TeX root = ../../../book.tex
\subsection{词序至关重要!}\label{sec:section4.2.4}

我们将在下一节详细讨论\textbf{量化}变量的概念。现在,让我们分析一个更引人注目的数学陈述示例,它凸显了语句中词序的重要性。考察语句结构也将是下一节的核心目标。
\begin{center}
    存在实数 $y$,使得对于每个实数 $x$,都有 $y = x^3$。
\end{center}
这句话宣称:无论取哪个实数 $x \in \mathbb{R}$,总能找到实数 $y \in \mathbb{R}$ 满足 $y = x^3$。这是荒谬的!怎么可能存在一个数字是所有实数的立方?这确实是一个数学陈述,但它显然为\verb|假|。那么下面这句话呢?
\begin{center}
    对于每个实数 $x$,存在实数 $y$,使得 $y = x^3$。
\end{center}
这句话为\verb|真|!你注意到两句话的区别了吗?它们包含完全相同的词语和符号,仅顺序不同。前一句声称存在某个 $y$ 是所有 $x$ 的立方(为\verb|假|),而后一句断言每个 $x$ 都有对应的立方数 $y$(为\verb|真|)。此例深刻揭示了词序的重要性。

\clearpage
