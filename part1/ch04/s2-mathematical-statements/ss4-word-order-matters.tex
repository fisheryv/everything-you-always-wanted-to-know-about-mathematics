% !TeX root = ../../../book.tex
\subsection{词序至关重要!}\label{sec:section4.2.4}

我们将在下一节中详细讨论\textbf{量化}变量的概念。现在,我们想考虑一个更加引人注目的数学陈述示例,它说明了语句中词序的重要性。分析如下语句的结构也将是下一节的主要目标。
\begin{center}
    存在一个实数 $y$,使得对于每个实数 $x$, $y = x^3$。
\end{center}
这个句话说了什么?它说的是无论 $x \in \mathbb{R}$ 是什么,我们都可以找到一个数字 $y \in \mathbb{R}$ 使得 $y = x^3$ 为\verb|真|。这是荒唐的!怎么可能有一个数字是所有数字的立方呢?这句话确实是一个数学陈述,但它绝对为\verb|假|。但下面的说法又如何呢?
\begin{center}
    对于每个实数 $x$,存在一个实数 $y$,使得 $y = x^3$。
\end{center}
上面这句话为\verb|真|!你看出两个句子之间的区别了吗?它们包含完全相同的单词和符号,但顺序不同。前一句断言存在某个数字是每个实数的立方(此为\verb|假|),而后一句断言每个实数都有一个立方根,此为\verb|真|。这个例子强调了词序的重要性。
