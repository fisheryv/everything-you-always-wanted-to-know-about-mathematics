% !TeX root = ../../../book.tex
\subsection{变量命题}

另一类非数学陈述涉及\textbf{未量化变量}。例如:
\begin{center}
    \textcolor{red}{``$x^2 - 1 = 0$''}
\end{center}

该语句在语法上是正确的,我们也能理解它,但其真值无法确定:当 $x = 1$ 时为\verb|真|,当 $x = 8$ 时为\verb|假|;若 $x = \mathbb{N}$ 或 $x = \text{Fisher}$,则该语句无意义。因此这类语句也应被排除。鉴于其广泛用途,我们称之为\textbf{变量命题},因为其主张\emph{依赖于}特定变量。

对于上述语句,可定义变量命题 $P(x)$ 表示 ``$x^2 - 1 = 0$'',通常表述为:
\begin{center}
    \textcolor{olivegreen}{令 $P(x)$ 表示命题``$x^2 - 1 = 0$''。}
\end{center}
惯例上用大写字母表示变量命题,小写字母表示所含变量(非强制要求)。

定义变量命题后,可通过为变量 $x$ \emph{赋值}生成有效的数学陈述:如 $P(1)$ 为真,$P(0)$ 为假。亦可对 $P(x)$ 进行\textbf{量化},例如,下面的语句是一个为\verb|真|的数学陈述:
\begin{center}
    存在 $x \in \mathbb{R}$ 使得命题 $P(x)$ 为\verb|真|。
\end{center}
而下面的语句是一个为\verb|假|的数学陈述:
\begin{center}
    对于每个 $x \in \mathbb{R}$,命题 $P(x)$ 均为\verb|真|。
\end{center}
请思考这些陈述的真值依据及其数学陈述属性,并考虑如何证明。

\subsubsection*{定义变量命题}

定义变量命题需遵循特定格式:
\begin{enumerate}[label=(\arabic*)]
    \item 为命题分配字母名称(如 $P$);
    \item 标明其依赖变量(如 $x$ 或 $y$);
    \item 为实际命题添加引号;
    \item 避免引入命题上下文之外的字母。
\end{enumerate}
此格式经过严谨设计以确保精确性:明确界定命题中各字母的含义,清晰区分命题内容与外部要素。

例如,以下是变量命题的\textcolor{red}{\textbf{不当}}``定义''示例。我们将分析其问题所在,并提出修改建议。
\begin{itemize}
    \item \textcolor{red}{令 $Q(y)$ 为命题``$x < 0$''。}\\
        \textbf{问题}:$x$ 未定义,且 $y$ 未在命题中出现,变量使用混乱。\\
        \textbf{修改为}:
        \begin{center}
            \textcolor{olivegreen}{令 $Q(x)$ 为命题``$x<0$''。}
        \end{center}
        此时括号内变量与命题变量一致,是良好的定义。
    \item \textcolor{red}{对于每个 $x \in \mathbb{R}$,令 $P(x)$ 为命题 $x^2 \ge 0$。}\\
        \textbf{问题}:量词``对于每个''可能被误解为命题组成部分,导致语义歧义。\\
        若将其解释为针对每个 $x \in \mathbb{R}$,逐一定义 $P(x)$ 为``$x^2 \ge 0$'',这或许成立;\\
        但若将其理解为 $P(x)$ 定义为``对于每个 $x \in \mathbb{R}$, $x^2 \ge 0$'',则不构成正确定义的命题——命题内的变量不应被重复量化。\\
        该命题的原始写法有两种可能的解释,而且它们非常不同。因此,这是一个不当的定义。\\
        \textbf{修改为}:
        \begin{center}
            \textcolor{olivegreen}{对于每一个 $x \in \mathbb{R}$,定义 $P(x)$ 为命题``$x^2 \ge 0$''。}
        \end{center}
        明确量词作用范围可消除歧义。虽然变量定义域并非必需,但明确说明并无不妥。
    \item \textcolor{red}{令 $T(x, y)$ = ``$x^2 - 7 = y$''。}\\
        \textbf{问题}:这里的``$=$''是什么意思?当我们希望比较两个数字并说它们的值相等(或比较两个集合并说它们的元素相同)时,该符号才有意义。对象 $T(x, y)$ 是一个数学陈述,要么为\verb|真|,要么为\verb|假|。因此,它没有数值可以与其他任何东西进行比较。\\
        同理,给定 $x$ 和 $y$ 的值,语句 ``$x^2 - 7 = y$'' 要么为\verb|真|,要么为\verb|假|,因此说该方程``等于''其他值是没有意义的。它具有真值,而非数值。\\
        \textbf{修改为}:
        \begin{center}
            \textcolor{olivegreen}{令 $T(x, y)$ 为``$x^2 - 7 = y$''。}
        \end{center}
        就变成有效的变量命题。
\end{itemize}

我们无意过多列举错误示例,但教学经验表明,这些是学生定义命题时的常见疏漏(无论是否意识到其错误性),故特此说明。

最后需要明确一点:定义命题时不必预先说明变量从何而来。例如定义
关于变量命题最后有一点说明。定义命题时不必说明变量从何而来。可以稍后在调用命题时或者使用变量的特定值时填写。也就是说,我们可以定义
\begin{center}
    令 $T(x, y)$ 为``$x^2 - 7 = y$''。
\end{center}
无需指定 $x,y$ 到底是自然数、整数、复数还是其他类型数字。后续可验证 $T(3, 2)$ 为\verb|真|,$T(\pi, -1)$ 为\verb|假|,$T(\varnothing, \mathbb{N})$ 无意义。但在定义 $T(x,y)$ 时,我们无需预设这些具体解释。
