% !TeX root = ../../../book.tex
\subsection{变量命题}

另一类非数学陈述涉及\textbf{未量化变量}。例如,拿这句话来说:
\begin{center}
    ``\textcolor{red}{$x^2 - 1 = 0$}''
\end{center}
这在语法上当然是正确的,我们也能理解它,但它的真值是什么呢?我们不知道!如果 $x = 1$,则该句话为\verb|真|,但如果 $x = 8$,则为\verb|假|,如果 $x = \mathbb{N}$ 或 $x = \text{于儿}$,则该句话没有意义!因此,我们也希望禁止这样的语句。不过,这些类型的语句非常有用且常见。我们将它们称为\textbf{变量命题},因为它们提出的主张\emph{依赖于}某些变量。

对于上面的语句,我们可以将 $P(x)$ 定义为变量命题 ``$x^2 - 1 = 0$''。我们通常会将此声明写为
\begin{center}
    \textcolor{olivegreen}{令 $P(x)$ 为陈述 ``$x^2 - 1 = 0$''。}
\end{center}
通常用大写字母表示变量命题和数学陈述,用小写字母表示其中包含的变量。(但这不是一个要求,只是一种常见的约定。)

现在定义了这个变量命题,我们可以通过将特定值\emph{分配}给表达式中的变量 $x$ 来创建正确的数学陈述。我们可以说 $P(1)$ 为\verb|真|而 $P(0)$ 为\verb|假|。我们还可以对 $P(x)$ 进行\textbf{量化}。例如,下面语句是一个为\verb|真|的数学陈述:
\begin{center}
    存在 $x \in \mathbb{R}$ 使得命题 $P(x)$ 为\verb|真|。
\end{center}
而下面语句是一个为\verb|假|的数学陈述:
\begin{center}
    对于每个 $x \in \mathbb{R}$,命题 $P(x)$ 为\verb|真|。
\end{center}
想想为什么这些陈述具有我们所说的真值。你能明白为什么它们是数学陈述吗?你将如何证明这些说法?

\subsubsection*{定义变量命题}

请注意我们用来定义变量命题的格式,如上面的格式:
\begin{enumerate}[label=(\arabic*)]
    \item 我们给命题起一个字母名称(如 $P$);
    \item 我们表明它对一些变量的依赖性,每个变量都有一个字母名称(如 $x$ 和 $y$);
    \item 我们在实际命题本身周围加上引号;
    \item 我们不包含任何在命题上下文中没有意义的新字母。
\end{enumerate}

这种格式经过精心选择,因为它精确且明确。它为命题中的每个字母赋予含义,并清楚地区分命题中的内容和不包含的内容。

例如,以下是变量命题的\textcolor{red}{\textbf{糟糕}}``定义''。我们会指出为什么它们不好的理由,并给出进行适当修正的建议。
\begin{itemize}
    \item \textcolor{red}{令 $Q(y)$ 为命题``$x < 0$''。}\\
        \textbf{原因}:$x$ 是什么?$y$ 在哪里?我们不知道命题上下文中的 $x$ 是什么,所以这是一个糟糕的定义。\\
        修改为:
        \begin{center}
            \textcolor{olivegreen}{令 $Q(x)$ 为陈述 ``$x<0$''。}
        \end{center}
        就完美了。括号内的变量是后面引号中陈述使用的变量。太棒了。
    \item \textcolor{red}{对于每个 $x \in \mathbb{R}$,令 $P(x)$ 为命题 $x^2 \ge 0$。}\\
        \textbf{原因}:这句话的作者是否想断言无论 $x \in \mathbb{R}$ 是什么 $x^2 \ge 0$? ``对于每个 $x \in \mathbb{R}$'' 这句话是否意味着是命题的一部分?\\
        如果我们将其解释为 $P(x)$ 被定义为 ``$x^2 \ge 0$'',并且这个定义是针对每个 $x \in \mathbb{R}$ 进行的,那么……好吧,这可能是合理的。\\
        然而,如果我们将其解释为 $P(x)$ 被定义为 ``对于每个 $x \in \mathbb{R}$, $x^2 \ge 0$'',那么……嗯,这肯定是不同的。事实上,这甚至不是一个正确定义的命题!命题 $P(x)$ 应取决于输入值 $x$,但不应允许更改或进一步量化命题内的变量!\\
        这个命题最初的写法有两种可能的解释,而且它们非常不同。因此,这是一个糟糕的定义。\\
        如果我们修改为:
        \begin{center}
            \textcolor{olivegreen}{对于每一个 $x \in \mathbb{R}$,定义 $P(x)$ 为陈述 ``$x^2 \ge 0$''。}
        \end{center}
        就会好很多。正如我们下面提到的,从技术上讲,我们不必告诉读者我们想要定义命题的 $x$ 值。不过,也许这只是一些有用的信息,所以写出来并没有什么坏处。
    \item \textcolor{red}{令 $T(x, y)$ = ``$x^2 - 7 = y$''。}\\
        \textbf{原因}:在这种情况下 ``$=$'' 是什么意思?当我们希望比较两个数字并说它们的值相等(或两个集合并说它们的元素相等)时,该符号才适用。对象 $T(x, y)$ 是一个数学陈述,要么为\verb|真|,要么为\verb|假|。因此,它没有数值可以与其他任何东西进行比较。\\
        同理,给定 $x$ 和 $y$ 的值,语句 ``$x2 - 7 = y$'' 要么为\verb|真|,要么为\verb|假|,因此说该方程``等于''其他值是没有意义的。它具有真值,而不是数值。\\
        如果我们改为:
        \begin{center}
            \textcolor{olivegreen}{令 $T(x, y)$ 为 ``$x^2 - 7 = y$''。}
        \end{center}
        那就完美了。
\end{itemize}

我们已经做了足够多的错误示范了,不想把任何不好的想法灌输给你,真的!然而,根据过去的经验,我们知道这些是学生书写命题时常犯的错误(要么是无意的,要么没有意识到为什么他们是错的),所以我们觉得有必要分享出来。

关于变量命题最后有一点说明。定义命题时不必说明变量从何而来。可以稍后在调用命题时或者使用变量的特定值时填写。也就是说,我们可以定义
\begin{center}
    令 $T(x, y)$ 为 ``$x^2 - 7 = y$''。
\end{center}
而无需指定 $x$ 和 $y$ 到底是自然数、整数、复数还是其他类似数字。稍后,我们可以说 $T(3, 2)$ 为\verb|真|,$T(\pi, -1)$ 为\verb|假|,并且 $T(\varnothing, \mathbb{N})$ 没有意义,但在定义 $T(x, y)$ 时,我们不需要以某种方式预期任何这些解释。
