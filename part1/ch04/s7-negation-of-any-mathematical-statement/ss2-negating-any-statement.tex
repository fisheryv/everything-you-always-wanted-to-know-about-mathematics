% !TeX root = ../../../book.tex
\subsection{否定任意陈述}

目前,我们已经讨论了如何否定基本数学陈述:存在量词 $\exists$、全称量词 $\forall$、合取 $\land$、析取 $\lor$ 和蕴含 $\implies$。我们编写的其他任何陈述都是这些基本陈述的组合,因此可以通过反复应用上述技术来否定任意陈述。具体来说,只需从左到右解析陈述并依次否定每个部分:遇到存在量词 $\exists$ 时,将其改为全称量词 $\forall$ 并否定其后属性;遇到析取 $\lor$ 时,否定其前后两部分并将连词改为合取 $\land$;遇到条件陈述时,直接应用前述技术即可。

下面通过示例具体说明上述方法的应用:

\begin{example}
    给出下列陈述的逻辑否定:
    \[\forall x \in \mathbb{R} \centerdot x < 0 \lor x > 0\]

    该陈述表示:``每个实数 $x$ 都满足 $x < 0$ 或 $x > 0$。''

    其逻辑否定为:
    \[\neg (\forall x \in \mathbb{R} \centerdot x < 0 \lor x > 0) \iff \exists x \in \mathbb{R} \centerdot x \ge 0 \land x \le 0\]

    请注意,此处应用了德摩根定律否定析取 $\lor$,并利用 $x \ngtr 0$ 等价于 $x \le 0$ 的性质。

    我们发现这里的否定陈述为\verb|真|,因为存在 $0 \in \mathbb{R}$ 满足 $0 \le 0$ 且 $0 \ge 0$,故原陈述为\verb|假|。
\end{example}

\begin{example}
    给出下列陈述的逻辑否定:
    \[\exists n \in \mathbb{N} \centerdot n \ge 10 \land \sqrt{n} \le 3\]

    该陈述表示:``存在自然数 $n$ 同时满足 $n \ge 10$ 且 $\sqrt{n} \le 3$。''

    其逻辑否定为:
    \[\forall n \in \mathbb{N} \centerdot n < 10 \lor \sqrt{n} > 3\]

    该否定陈述表示:``每个自然数 $n$ 都满足 $n < 10$ 或 $\sqrt{n} > 3$''。
\end{example}

\begin{example}
    给出下列陈述的逻辑否定:
    \[\exists x \in \mathbb{R} \centerdot \forall y \in \mathbb{R} \centerdot x \ge y \implies x^2 \ge y^2\]
    
    该陈述表示:``存在实数 $x$ 使得对于所有实数 $y$,若 $x \ge y$ 则 $x^2 \ge y^2$。''

    其逻辑否定为:
    \[\forall x \in \mathbb{R} \centerdot \exists y \in \mathbb{R} \centerdot x \ge y \land x^2 < y^2\]
    
    能否证明该否定陈述为\verb|真|?请尝试一下!
\end{example}

\begin{example}
    给出下列陈述的逻辑否定:
    \[\forall X \in \mathcal{P}(\mathbb{Z}) \centerdot (\forall x \in X \centerdot x \ge 1) \implies X \subseteq \mathbb{N}\]

    该陈述表示:``对于整数集 $\mathbb{Z}$ 的任意子集 $X$,若 $X$ 的所有元素都满足 $x \ge 1$,则 $X$ 是自然数集 $\mathbb{N}$ 的子集。''

    其逻辑否定为:
    \[\exists X \in \mathcal{P}(\mathbb{Z}) \centerdot (\forall x \in X \centerdot x \ge 1) \land X \nsubseteq \mathbb{N}\]

    该否定陈述表示:``存在子集 $X \subseteq \mathbb{Z}$ 满足所有元素 $x \in X$ 都 $x \ge 1$ 且 $X \not\subseteq \mathbb{N}$。'' 其中 $X \nsubseteq \mathbb{N}$ 可进一步改写为:
     \[X \nsubseteq \mathbb{N} \iff \exists y \in X \centerdot y \notin \mathbb{N}\]

    尽管此步骤并非必须。

    原始陈述与否定陈述中哪一个为\verb|真|?请给出证明。
\end{example}

\setlength{\fboxrule}{0.8pt}
\setlength\fboxsep{5mm}
\begin{center}
\fcolorbox{black}{white}{%
    \parbox{0.85\textwidth}{%
        将上面示例中的陈述与以下陈述进行对比:
        \[\forall X \in \mathcal{P}(\mathbb{Z}) \centerdot \forall x \in X \centerdot (x \ge 1 \implies X \subseteq \mathbb{N})\]
        二者唯一的区别是括号的位置不同,但这完全改变了陈述的含义!\\

        示例中的陈述断言整数集的每个子集。也就是说,无论考虑哪个子集 $X \subseteq \mathbb{Z}$,该陈述都表示:如果集合 $X$ 的所有元素都大于等于 $1$,那么 $X$ 是 $\mathbb{N}$ 的子集。\\

        而框中给出的新陈述含义不同:无论考虑哪个子集 $X \subseteq \mathbb{Z}$,以及无论取 $X$ 中的哪个元素 $x$,该陈述都表示:如果元素 $x$ 大于等于 $1$,那么集合 $X$ 是 $\mathbb{N}$ 的子集。\\

        你看出区别了吗?区别在于``如果''出现的位置:量词的作用域在哪里结束,条件陈述从哪里开始?示例陈述将 $X$ 元素的量化放在条件陈述的``如果''部分之内,而此框中的陈述将量化完全放在条件陈述之前。\\

        此框中的陈述为假,我们鼓励你自己找出原因。
    }
}
\end{center}

\begin{example}
    设 $O(x)$ 为命题``$x$ 为奇数'',$E(x)$ 为命题``$x$ 为偶数''。给出以下陈述的逻辑否定:
    \[\forall x, y \in \mathbb{Z} \centerdot O(x \cdot y) \iff \big(O(x) \land O(y)\big)\]

    该陈述表示:``对于任意两个整数 $x$ 和 $y$,它们的乘积为奇数当且仅当它们都是奇数。''

    在进行逻辑否定之前,请记住 $\iff$ 等价于 $\implies$ 和 $\impliedby$。让我们首先重写该陈述,以便正确地给出否定:
    \[\forall x, y \in \mathbb{Z} \centerdot \Big[O(x \cdot y) \implies \big(O(x) \land O(y)\big)\Big] \land \Big[\big(O(x) \land O(y)\big) \implies O(x \cdot y)\Big]\]

    该陈述的逻辑否定为:
    \[\neg \bigg(\forall x, y \in \mathbb{Z} \centerdot \Big[O(x \cdot y) \implies \big(O(x) \land O(y)\big)\Big] \land \Big[\big(O(x) \land O(y)\big) \implies O(x \cdot y)\Big]\bigg) \]
    \begin{align*}
        \iff \exists x, y \in \mathbb{Z} \centerdot & \neg \Big[O(x \cdot y) \implies \big(O(x) \land O(y)\big)\Big] \lor \neg \Big[\big(O(x) \land O(y)\big) \implies O(x \cdot y)\Big] \\
        \iff \exists x, y \in \mathbb{Z} \centerdot & \Big[O(x \cdot y) \land \neg \big(O(x) \land O(y)\big)\Big] \lor \Big[\big(O(x) \land O(y)\big) \land \neg O(x \cdot y)\Big] \\
        \iff \exists x, y \in \mathbb{Z} \centerdot & \Big[O(x \cdot y) \land \big(E(x) \lor E(y)\big)\Big] \lor \Big[\big(O(x) \land O(y)\big) \land E(x \cdot y)\Big] 
    \end{align*}
    
    也就是说,该否定陈述表示:``存在整数 $x$ 和 $y$,使得它们的乘积为奇数,但 $x$ 和 $y$ 中至少有一个为偶数;或它们的乘积为偶数,但 $x$ 和 $y$ 均为奇数。''

    你能证明哪个陈述为\verb|真|吗?
\end{example}
