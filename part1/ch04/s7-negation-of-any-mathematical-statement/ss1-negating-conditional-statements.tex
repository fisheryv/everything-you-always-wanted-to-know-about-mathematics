% !TeX root = ../../../book.tex
\subsection{否定条件陈述}

考虑 $P \implies Q$ 形式的声明。它表示只要 $P$ 为真,$Q$ 也为真。我们如何否定这样的陈述呢?逻辑否定究竟意味着什么?回想一下我们如何将 ``$\implies$'' 定义为逻辑连词。在哪些情况下我们可以认定条件陈述的声明者为骗子。在这些情况下,我们可以说逻辑否定为\verb|真|。唯一的情况是假设 $P$ 为\verb|真|但结论 $Q$ 为\verb|假|。

为了证明这种等价关系,我们需要用到 $P \implies Q$ 写成 ``$\lor$'' 陈述的方法:
\[(P \implies Q) \iff (\neg P \lor Q)\]
这将有助于我们证明以下主张。

\begin{lemma}
    条件陈述的逻辑否定由下式给出
    \[\neg (P \implies Q) \iff P \land \neg Q\]
\end{lemma}

\begin{proof}
    \begin{align*}
        \neg (P \implies Q) &\iff \neg (\neg P \lor Q) &\quad \text{已经证明的逻辑等价} \\
        &\iff \neg (\neg P) \land \neg Q &\quad \text{德摩根逻辑定律} \\
        &\iff P \land \neg Q &\quad \text{因为} \neg (\neg P) \iff P
    \end{align*}
\end{proof}

这具有直观意义:为了证明条件主张为\verb|假|,我们需要找到假设成立但结论不成立的情况。

尽管存在教坏小孩子的风险,但我们还是要指出一些逻辑上\textbf{不}等价于 $\neg (P \implies Q)$ 的陈述。这些是我们看到学生经常犯的错误。让我们检视一下它们,看看为什么它们实际上不正确的。对于它们中的每一个,请记住,我们希望保证逻辑否定 $\neg (P \implies Q)$ 具有与原始陈述 $P \implies Q$ \emph{完全相反的真值}。在每种情况下,我们都可以看到这种关系不成立。

\begin{itemize}
    \item \textcolor{red}{$\neg P \implies Q$} \\
        该条件陈述与原始陈述 $P \implies Q$ 没有逻辑联系。请记住,陈述 $P \implies Q$ \emph{并未声明}在 $P$ 为\verb|假|的情况下 $Q$ 是否为\verb|真|。(不如``如果下雨,那么我就带伞''这个例子。如果不下雨,谁知道我带的是什么!)为什么我们要期望 $Q$ 像陈述中说的那样,在 $P$ 不成立的情况下\emph{一定为}\verb|真|?
    \item \textcolor{red}{$P \implies \neg Q$} \\
        同理,这个条件陈述与原始陈述没有逻辑联系。还是雨伞的例子。这个陈述说的是``如果下雨,那么我就\textbf{不}带伞。'' 这是否意味着最初的声明是错误的?当然不是!
    \item \textcolor{red}{$P \notimplies Q$} \\
        这种情况就比较微妙了。数学家会将 ``$P \notimplies Q$'' 解读为 ``$P$ \emph{不一定}意味着 $Q$''。也就是说,存在 $P \implies Q$ 为真的情况,也存在 $P \implies Q$ 为假的情况;具体是哪种情况取决于 $P$ 和 $Q$ 各自的陈述是什么。根据具体情况,这个主张有不同的含义,但严格来说,这并不是对原始声明的\textbf{逻辑否定}。\\
        \newline
        特别是,当我们尝试对该陈述进行逻辑否定时,我们会遇到问题。``$P$ 并非不一定意味着 $Q$'' 这句话是什么意思?这是否意味着存在 $P$ 不意味着 $Q$ 的情况,但也存在 $P$ 意味着 $Q$ 的情况?这听起来非常像声明 $P \notimplies Q$ 本身…… \\
        \newline
        出于这些原因,我们希望避免使用 $\notimplies$ 这种表示法。它在数学上确实存在某种意义,但在符号逻辑意义上并没有真正明确的定义。无论如何,它绝对不是 $\implies$ 的逻辑否定。
\end{itemize}
现在我们已经解决了这些常见错误,让我们强调一下 $P \implies Q$ 的正确否定。我们发现记住条件陈述的 ``$\lor$'' 版本非常有帮助;基于 ``$\lor$'' 版本,可以很容易地应用德摩根定律得到该陈述的否定:

\setlength{\fboxrule}{2pt}
\begin{center}
\fcolorbox{olivegreen}{white}{%
    \parbox{0.8\textwidth}{%
        \[\neg (P \implies Q) \iff \neg (\neg P \lor Q) \iff P \land \neg Q\]
    }
}
\end{center}

\subsubsection*{否定 $\iff$}

要否定双向条件陈述,我们只需将其写为两个条件陈述的合取:
\[\neg (P \iff Q) \iff [\neg (P \implies Q) \lor \neg (Q \implies P)] \iff (P \land \neg Q) \lor (Q \land \neg P)\]
如果你熟悉计算机编程,你可能会认出右侧的陈述其实是 \verb|XOR| 运算符!它表示只有一个陈述为\verb|真|,$P$ 或 $Q$ 二者之一,但二者不能\emph{同时}为\verb|真|。
