% !TeX root = ../../../book.tex
\subsection{否定条件陈述}

考虑形如 $P \implies Q$ 的陈述。该陈述表明当 $P$ 为真时,$Q$ 也为真。如何否定这样的陈述?逻辑否定究竟意味着什么?回顾``$\implies$''的定义。在何种情况下可断定该条件陈述为假?唯一的情形是前提 $P$ 为\verb|真|而结论 $Q$ 为\verb|假|。

为了证明这种等价关系,我们需要将 $P \implies Q$ 写成``$\lor$''陈述:
\[(P \implies Q) \iff (\neg P \lor Q)\]
这将有助于证明以下引理。

\begin{lemma}
    条件陈述的逻辑否定由下式给出:
    \[\neg (P \implies Q) \iff P \land \neg Q\]
\end{lemma}

\begin{proof}
    \begin{align*}
        \neg (P \implies Q) &\iff \neg (\neg P \lor Q) &\quad \text{已知逻辑等价} \\
        &\iff \neg (\neg P) \land \neg Q &\quad \text{德摩根逻辑定律} \\
        &\iff P \land \neg Q &\quad \text{因为\ } \neg (\neg P) \iff P
    \end{align*}
\end{proof}

此结论符合直观:要想证明条件命题为\verb|假|,只需找到前提成立而结论不成立的实例。

尽管存在误导风险,但仍需指出几种逻辑上\textbf{不等价}于 $\neg (P \implies Q)$ 的常见错误表述。逻辑否定要求否定式 $\neg (P \implies Q)$ 必须与原陈述 $P \implies Q$ 的真值\emph{完全相反}。让我们逐一分析以下错误,揭示其为何不满足此要求。

\begin{itemize}
    \item \textcolor{red}{$\neg P \implies Q$} \\
        该条件陈述与原陈述 $P \implies Q$ 无逻辑关联。注意 $P \implies Q$ \emph{不承诺} $P$ 为\verb|假|时 $Q$ 的真值。(例如:``如果下雨,那么我就带伞''——未下雨时,带伞与否未知。)因此 $P$ 不成立时 $Q$ \emph{未必}一定为\verb|真|。

    \item \textcolor{red}{$P \implies \neg Q$} \\
        同理,该条件陈述与原陈述亦无逻辑关联。沿用雨伞的例子:``如果下雨,那么我就\textbf{不}带伞''并不能证伪原命题。

    \item \textcolor{red}{$P \notimplies Q$} \\
        此情况比较微妙。数学中``$P \notimplies Q$''表示``$P$ \emph{未必}蕴含 $Q$'',即 $P \implies Q$ 可真可假,其值取决于具体命题。此表述因语境而异义,但严格而言并非\textbf{逻辑否定}。\\
        \newline
        逻辑否定时会产生歧义:``$P$ 并非势必蕴含 $Q$''可能被解读为既存在 $P$ 不蕴含 $Q$ 的情形,也存在蕴含的情形。此表述与 $P \notimplies Q$ 本身非常容易混淆。\\
        \newline
        因此建议避免使用 $\notimplies$ 这种表述。该符号在数学上确实具有某种意义,但在符号逻辑中无精确定义,且绝非 $\implies$ 的逻辑否定。
\end{itemize}

厘清常见错误后,重申 $P \implies Q$ 的正确否定形式。通过对其析取版本 $\neg P \lor Q$ 应用德摩根定律,可以非常容易地推得其否定式:

\setlength{\fboxrule}{2pt}
\begin{center}
\fcolorbox{olivegreen}{white}{%
    \parbox{0.8\textwidth}{%
        \[\neg (P \implies Q) \iff \neg (\neg P \lor Q) \iff P \land \neg Q\]
    }
}
\end{center}

\subsubsection*{否定 $\iff$}

要否定双向条件陈述,可将其转化为两个条件陈述的合取:
\[\neg (P \iff Q) \iff [\neg (P \implies Q) \lor \neg (Q \implies P)] \iff (P \land \neg Q) \lor (Q \land \neg P)\]
熟悉计算机编程的读者可能注意到,右侧表达式正是逻辑运算中的 \verb|XOR|(异或)!它表示 $P$ 或 $Q$ 中有且仅有一个为\verb|真|,二者不能\emph{同时}为\verb|真|。
