% !TeX root = ../../book.tex
\section{模算术}

你可能已经接触过一种自然且常见的等价关系,那就是整数中的\emph{同余}关系。这是一种对``奇偶性''等价关系的直接推广,基于某个特定属性对整数进行分类。在这里,我们将通过定义一些整数的性质来扩展这个概念,并引入多个等价关系。我们还会讨论一些通过这些关系使问题变得更容易证明(或变得可证明!)的有趣结果。

\subsection{定义与示例}

\subsubsection*{整除性}

我们将从一个我们已经多次见过的定义开始。

\begin{definition}
    设 $a,b \in \mathbb{Z}$。我们说 $a$ 整除 $b$ 是指 $b$ 可以被 $a$ 整除,即 $\exists k$ 使得 $b = ak$,或者等价地,$\frac{b}{a} \in \mathbb{Z}$(排除 $a=b=0$ 的情况)。记作 $a \mid b$。
\end{definition}

注意,这个定义表明每个整数都能整除 $0$(例如,$5 \mid 0$),但 $0$ 除了自己之外不能整除任何数(例如,$0 \nmid 5$ 但 $0 \mid 0$)。想一想这是否符合你对``整除''的直觉理解,同时也满足了给定的定义。另外,这里也考虑了负数的情况,因为存在量词意味着\emph{整数} $k \in \mathbb{Z}$。因此,$-2 \mid 4$ 和 $8 | -24$ 也成立。

现在,像 $2 \nmid 5$ 这样的表达告诉我们一些关于整数 $2$ 和 $5$ 之间关系的信息,但并不能涵盖所有情况。我们知道没有整数 $k$ 可以满足 $2k = 5$,但这并没有说明我们到底能多接近。显然,$k = -100$ 是一个糟糕的估计,但 $2 \times 2 = 4$ 和 $2 \times 3 = 6$ 都非常接近 $5$……对于这样的小数字,这似乎很明显,我们可以手工验证,但对于巨大的数字呢?我们知道 $7 \nmid 100000$(为什么?想想质数……),但我们如何解决这个``找到使 $7k$ 最接近 $100000$ 的 $k$''这个问题呢?我们怎么知道是否有一个具体的答案?会不会有两个同样``合理''的答案,就像 $2 \nmid 5$ 一样?

关于上面第二个问题,为了简化,我们希望限制答案,使其只有一个合理的选项。这是为了避免在找到一个答案后还要担心是否有其他答案。因此,我们将采用\emph{价格猜猜猜}\footnote{价格猜猜猜 (The Price Is Right) 是美国历史上风行时间最长的一档电视节目。--- 译者注}的规则:我们寻找\emph{最接近但不超过}目标值的答案。比如在 $2 \nmid 5$ 的例子中,我们认为 $k = 2$ 是最佳估计,因为 $4 < 5$。同理,在 $7 \nmid 100000$ 的例子中,我们认为 $k = 14285$ 是最佳估计,因为 $7 \times 14285 = 99995 < 100000$。(注意,在这种情况下,有一个``更接近''的估计,但它超过了目标值,因此我们不考虑它。)

这就引出了我们如何得到这样的估计。给定 $a, b \in \mathbb{Z}$,我们可以查看 $a$ 的越来越大的倍数,直到超过 $b$;在此之前的那个最大倍数就是最佳估计。估计的``准确性''范围在 $0$ 到 $a - 1$ 之间,当 $a$ 能整除 $b$ 时,准确性为 $0$。(注意:``超过''是指顺序关系 $>$,所以要仔细考虑这在负数中的应用。比如,$2 \nmid -3$ 而 $2 \times -2 = -4$ 被认为是最佳估计,因为 $-4 \le -3$。)以下引理总结了关于逐步查看 $a$ 的倍数直到找到 $b$ 的最佳估计的思路,并声明在我们设定的约束条件下,总会有唯一解。

\begin{lemma}[除法算法]\label{lemma6.5.2}
    设 $a,b \in \mathbb{Z}$。则 $\exists k, r \in \mathbb{Z}$ 使得 $ak + r = b$,其中 $0 \le r \le a - 1$。换句话说,对于任意两个整数,总能找到一个 $a$ 的倍数,使得它与 $a$ 的乘积最接近 $b$ 而不超过 $b$,同时还存在一个唯一的余数。我们把这个 $r$ 称为``$b$ 除以 $a$ 的余数'' 或 ``$b$ 被 $a$ 除的余数''。
\end{lemma}

我们会频繁使用余数的概念。具体来说,我们将比较两个除法的余数,并基于余数定义一种关系。稍后我们会详细介绍这些内容。首先,请你证明这个重要的引理!

\begin{proof}
    留作习题 \ref{exc:exercises6.7.14}。
\end{proof}

之所以称之为除法\emph{算法},是因为它暗示了一种\emph{找到}这些倍数和余数的\emph{过程}。这种方法虽然简单但非常有效,就是\emph{反复应用减法}。也就是说,给定 $a$ 和 $b$,我们可以不断地从 $b$ 中减去 $a$,例如先得到 $b - a$,再得到 $b - 2a$,然后是 $b - 3a$……依此类推,直到剩下一个介于 $0$ 和 $a$ 之间的余数。\\

\begin{example}
    让我们通过一个例子来展示这个过程。假设 $a = 8, b = 62$。我们不断地从 $62$ 中减去 $8$,结果是:
    \[62, 54, 46, 38, 30, 22, 14, 6\]
    我们停在 $6$ 上,因为它满足 $0 \le 6 < a = 8$,这表明 $r = 6$。我们还注意到,我们总共从 $b$ 中减去了 $7$ 次 $a$,因为列表中有八个项,其中第一项是 $b - 0 \cdot a$。因此,我们可以写出:
    \[\underbrace{62}_{b} = \underbrace{7}_{k} \cdot \underbrace{8}_{a} + \underbrace{6}_{r}\]
\end{example}

这里的重点是有一种方法可以找到这个余数,并且这个余数是唯一的。有了这个结果,我们可以用它来定义某些 $\mathbb{Z}$ 上的关系。接下来,我们将展示这些关系都是等价关系,并具体看看它们的\emph{等价类}是多么有用!

\subsubsection*{模 $n$ 同余}

\begin{definition}\label{def:definition6.5.4}
    设 $n \in \mathbb{N}$。我们定义 $\mathbb{Z}$ 上的关系 $R_n$ 为 $(a,b) \in R_n$ 当且仅当 $a$ 和 $b$ 除以 $n$ 时有相同的余数,即
    \[(a,b) \in R_n \iff n \mid a-b\] 
    写法上,我们也将其写成
    \[a \equiv b \mod n\]
    读作``$a$ 与 $b$ \dotuline{模 $n$ 同余}''。(口头上,我们通常将 ``modulo'' 简化为 ``mod''。)
\end{definition}

\begin{remark}
    我们在定义中提到,``$a$ 和 $b$ 除以 $n$ 后有相同的余数''等同于$n \mid a - b$''。为什么会这样呢?这并非定义本身的原因,而是需要一些证明的。稍后你将在习题 \ref{exc:exercises6.7.15} 中进行这个证明。
\end{remark}

\begin{remark}
    在实际应用中(例如解决问题和证明其他结论时),我们会这样使用这个定义:已知 $a ≡\equiv b \mod n$ 意味着我们可以将 $a$ 表示为 $n$ 的倍数加上 $b$。
\end{remark}

我们来看一下为什么这是成立的。假设它们的余数都为 $r$,这意味着存在 $k, \ell \in \mathbb{Z}$ 使得
\[a = kn + r \quad\text{且}\quad b = \ell n + r\]
(它们有相同的余数,但 $n$ 的倍数可能不同。)通过相减来解出 $r$ 这样我们就能得到等式
\[a - kn = b - \ell n\]
然后移项并提取公因式可得
\[a = (k - \ell)n + b\]
瞧!$(k - \ell)n$ 是 $n$ 的倍数,而第二项只有 $b$ 本身。这说明 $a$ 是 $n$ 的倍数加上 $b$。

通常情况下,$b$ 可能并不是 $a$ 除以 $n$ 后的余数;特别是当 $b$ 不满足余数要求 $0 \le r \le a - 1$ 时,就会出现这种情况。

让我们总结一下这个观点,并写下我们将来会用到的定义形式。这是我们在证明或举例中引用\emph{模 $n$ 同余}定义时会用到的陈述:

\setlength{\fboxrule}{2pt}
\begin{center}
\fcolorbox{olivegreen}{white}{%
    \parbox{0.8\textwidth}{%
        \[a \equiv b \mod n \iff \exists m \in \mathbb{Z} \centerdot a=mn+b\]
    }
}
\end{center}

\begin{example}
    让我们通过考察几个较小的 $n$ 值,来看看这些关系的具体表现。

    \begin{itemize}
        \item 设 $n=1$。关系 $R_1$ 会是什么样?这个问题实际上有些无聊,因为任何整数除以 $1$ 的余数都是 $0$,所以每个整数都可以和其他任意整数相关联。也就是说,$\forall x,y \in \mathbb{Z} \centerdot (x, y) \in R_1$。因为这个关系相对不那么有趣,因此数学家们几乎不会讨论``模 $1$''这个话题。
        \item 设 $n=2$。关系 $R_2$ 就是我们之前定义的``奇偶关系''。想想为什么会这样。当我们把任意整数 $a$ 除以 $2$ 时,余数只能是 $0$ 或 $1$。如果 $a$ 和 $b$ 除以 $2$ 的余数都是 $0$,那么它们都是偶数;如果余数都是 $1$,那么它们都是奇数。(回想一下我们在第 \ref{ch:chapter03} 章中的定义,\emph{奇数}和\emph{偶数}是通过\emph{存在}声明来定义的:例如,当且仅当 $\exists k \in \mathbb{Z}$ 使得 $x = 2k$ 时,$x$ 为偶数。这正是除法算法的结果:当且仅当 $x$ 除以 $2$ 的余数为 $0$ 时,$x$ 为偶数,因为我们可以找到一个整数 $k$,使得 $x = 2k$。)\\
        现在,想想同余的另一种表述。如果两个整数都是偶数,那么它们的差也是偶数!也就是说,$a \equiv b \mod 2 \iff a - b \mid 2$;即 $a$ 和 $b$ 都是偶数(或者都是奇数)当且仅当它们的差也是偶数。(注意:我们还没有\emph{证明}这种表述确实等价于余数的定义。我们将在这个例子之后立即进行证明。)
        \item 设 $n=3$。例如,$0 \equiv 9 \mod 3, -1 \equiv 2 \mod 3$ 以及 $4 \equiv 28 \mod 3$。一般来说,只要在行尾加上``$\mod 3$''(或其他数),我们可以连接多个同余语句。当这样做时,整行都按照模 $3$ 处理。例如,以下语句在符号上是有效的,在数学上也是成立的:
        \[-100 \equiv -1 \equiv 8 \equiv 311 \equiv -289 \equiv 41 \mod 3\]
        (虽然我们不确定为什么需要写这样的陈述,但这样做是完全可以的!)
        \item 设 $n=10$。自然数除以 $10$ 的余数就是它的最后一位数字,也就是个位数字!这样我们就可以轻松地比较两个数的模 $10$ 余数。例如,$12 \equiv 32 \equiv 448237402 \mod 10$;而 $37457 \not\equiv 38201 \mod 10$。\\
        但对于\emph{负数}的情况就有所不同了。因为我们定义余数时,是取\emph{不超过}目标值的最大倍数。例如,$-1 \equiv 9 \mod 10$,这是因为 $-1= (-1) \cdot 10 + 9$,而 $9 = (0) \cdot 10 + 9$。它们的余数都是 $9$,需要加到某个 $10$ 的倍数上。请思考以下陈述的具体细节:
        \[-3 \equiv 17 \equiv -33 \equiv 107 \mod 10\]
    \end{itemize}
\end{example}

\subsubsection*{符号}

需要强调的是:在数学中,\textbf{mod} 是一种关系,而不是运算符或函数。在计算机科学和编程中,你可能会看到类似``$5$ mod $3 = 2$''的表达,这表示``$5$ 除以 $3$ 的余数是 $2$''。(在许多编程语言中,这可能表示为 \verb|5 % 3 = 2|。)在这里,我们不会这样写。我们使用 $\mod$ 和 $\equiv$ 符号表示某种\textbf{等价},因为我们讨论的数字不一定\emph{相等}。如果我们表达的等价链在某个自然数 $n$ 下是有意义的,我们会在行末写上``$\mod n$''来指出这一点。在这个意义上,$\mod$ 更像是一个\emph{修饰符},用来表示``这一行的所有陈述仅在除以 $n$ 的余数上有意义''。因此,我们可以写类似这样的表达
\[100 \equiv 97 \equiv 16 \equiv 4 \equiv z \cdot w \equiv 1 \equiv x - y \equiv -2 \equiv -8 \mod 3\]
这表示当考虑 $\mod 3$ 时,所有这些数字和表达式都是等价的。我们并没有断言它们是相等的,也没有断言它们在其他情况下是等价的。行末的 ``$\mod 3$'' 表示``我们只在整数模 $3$ 的范围内讨论。''

(问题:你能找到 $x, y, z, w \in \mathbb{Z}$ 使上面的等式成立吗?)

\subsubsection*{三个重要引理}

在这里,我们将要求你证明两个重要结论:首先,证明模 $n$ 同余可以用\emph{可除性}来等价理解;其次,证明这些关系是等价关系。在阅读本节时,请完成这些对应的练习。如果你已经掌握了这些细节,下一节关于这些关系下的等价类的内容会更容易理解。在这两个证明之后,我们还会展示并证明另一个结果。在讨论等价类之前,最后一个例子是一个有趣的算术问题,用同余可以轻松解决,但如果手工计算就不那么容易了。

\begin{lemma}
    在定义 \ref{def:definition6.5.4} 中,模 $n$ 同余的两种表述确实是等价的。也就是说,对于所有 $a, b \in \mathbb{Z}$ 和所有 $n \in \mathbb{N}$,
    \[a, b \;\text{除以}\; n \;\text{余数相同} \iff n \mid a - b\]
\end{lemma}

\begin{proof}
    见习题 \ref{exc:exercises6.7.15}
\end{proof}

\begin{lemma}\label{lemma6.5.9}
    对于任意 $n \in \mathbb{N}, R_n$ 是 $\mathbb{Z}$ 上的等价关系。
\end{lemma}

\begin{proof}
    见习题 \ref{exc:exercises6.7.16}
\end{proof}

感谢你证明了这些引理!$\smiley{}$ 现在我们知道模 $n$ 同余是一个等价关系(因此我们可以讨论等价类)。我们还了解到,要判断两个整数 (如 $a$ 和 $b$) 是否模 $n$ 同余,只需确定 $a - b$ 是否是 $n$ 的\emph{倍数}即可。这是一种验证同余关系是否成立的有效方法。

下一个引理告诉我们,在``模 n''的情况下进行加法和乘法\textbf{运算},结果仍然正确。如果我们有两个关于整数的等式,并将它们相加,结果仍然是正确的。也就是说,如果 $a + b = c$ 且 $d + e = f$,我们可以得到 $a + b + d + e = c + f$。这个引理说明,同样的原理适用于模 $n$ 的同余关系。同理,我们可以对同余关系进行乘法运算,并且同余关系仍然成立。

虽然这个引理的证明并不复杂,但我们会为你证明它,因为最近我们让你做了太多工作了。

\begin{lemma}[模算术引理 (Modular Arithmetic Lemma, 简称 MAL)]\label{lemma6.5.10}
    设 $n \in \mathbb{N}$。设 $a,b,r,s \in \mathbb{Z}$ 为任意固定整数。假设 $a \equiv r \mod n$ 且 $b \equiv s \mod n$。则
    \begin{align*}
        a + b &\equiv r + s \mod n \\
        a \cdot b &\equiv r \cdot s \mod n
    \end{align*}
\end{lemma}

(这个引理告诉我们,我们只需要处理余数。无论给定的 $a$ 和 $b$ 是什么,我们可以将它们化简为余数 $r$ 和 $s$,然后对这些余数进行计算。因为 $0 \le r, s \le n - 1$,所以它们相对于 $a$ 和 $b$ 来说是较小的,这使得我们在实际操作中可以更快地进行算术运算。以下证明保证了这种方法在所有情况下都有效。)

\begin{proof}
    假设 $a \equiv r \mod n$ 且 $b \equiv s \mod n$。这意味着 $\exists k, \ell \in \mathbb{Z}$ 使得
    \begin{align*}
        a &= kn + r \\
        b &= \ell n + s \\
    \end{align*}
    将上面两个等式相加得
    \[a + b = (kn + r) + (\ell n + s) = (k + \ell)n + (r + s)\]
    因为我们可以将 $a+b$ 表示为 $n$ 的倍数加上余数 $r+s$,所以 $a + b \equiv r + s \mod n$。\\ \\
    将上面两个等式相乘得
    \[a \cdot b =  (kn + r) \cdot (\ell n + s) = k\ell n^2 + (ks + \ell r)n + r \cdot s = n \cdot (k\ell n + ks + \ell r)+r \cdot s\]
    因为我们可以将 $a \cdot b$ 表示为 $n$ 的倍数加上余数 $r \cdot s$,所以 $a \cdot b \equiv r \cdot s \mod n$。
\end{proof}

\begin{remark}
    请注意,我们在此没有提到\textbf{减法}和\textbf{除法},而是只讨论了加法和乘法。这有两方面的原因。首先,减法实际上是``加一个负数''。因此,这个引理表明我们可以通过以下两个步骤来\emph{减去}两个同余:
    \begin{enumerate}[label=(\arabic*)]
        \item 将其中一个同余乘以 $-1$(应用\emph{乘法}引理)
        \item 将结果相加(应用\emph{加法}引理)
    \end{enumerate}
    看到它是如何\emph{同时}利用这两个引理的结果了吗?很巧妙,对吧?

    第二个原因稍微复杂一些。实际上,在模 $n$ 的情况下没有``除法''这种运算。主要原因是我们这里讨论的范围仅限于\emph{整数},而除法可能会产生非整数的\emph{有理数}。例如,我们知道 $4 \equiv 7 \mod 3$,但这是否意味着 $\frac{4}{2} \equiv \frac{7}{2} \mod 3$ 呢?这又意味着什么呢?一个整数(例如 $2$)怎么可能与一个非整数(例如 $\frac{7}{2}$)同余呢?正是因为这个原因,我们在 $\mathbb{Z}$ 模 $n$ 的环境中不讨论\textbf{除法}运算。

    关于这个``除法''问题,其实还有一些更细微的细节。我们会在 \ref{sec:section6.5.3} 节讨论\emph{乘法逆元}时详细说明。现在为了避免引起混淆,我们暂时不讨论这些细节。简单来说,我们将会发展出一种在某些特定情况下类似于模 $n$ 除法的方法。

    同时,为了确保我们只讨论\emph{整数},我们将\emph{仅}涉及加法和乘法。
\end{remark}

\subsubsection*{两个实用例子}

我们还不确定是否已经让你相信模算术的用处。为了确保我们已经证明了同余作为等价关系的概念既有数学趣味又有实用价值,我们将在这里举两个有趣且实用的例子。第一个例子是一个简单的问题,用模算术可以比用``标准''算术更容易解决。第二个例子是一个你可能以前用过但从未考虑过其原理的巧妙技巧。我们会证明它的有效性!\\ \\

\begin{example}
    考虑如下问题:
    \begin{center}
        \parbox{0.8\textwidth}{%
            \textbf{问题:}

            是否\emph{存在}自然数 $k$,使得 $5^k$ 恰好比 $7$ 的倍数多 $1$?

            如果存在,这样的自然数最小是多少?

            你能描述所有满足该条件的自然数吗?
        }
    \end{center}
    我们可以尝试通过代入 $k$ 的值来回答这些问题,看看会发生什么。然而,你很快会注意到,计算大指数会很麻烦,而要确定一个大数是否正好比$7$ 的某个倍数多 $1$ 会更加困难!如果你愿意的话可以继续尝试一下。甚至可以使用计算器。看看你能否解决这个问题!

    不过,我们更倾向于这样做:多次利用模算术引理 (Modular Arithmetic Lemma, MAL)。指数运算只是重复的乘法,因此我们可以反复应用该引理的乘法结论。其核心思想是,我们可以不断乘以 $5$,并在此过程中将所有结果模 $7$。也就是说,我们只需要找到一个比 $7$ 的倍数多 $1$ 的数 --- 即模 $7$ 同余 $1$ 的数 --- 而不需要立即知道该数具体是多少,只需判断它是否满足这一性质。接下来我们将展示此过程。

    我们从 $5^1 \equiv 5 \mod 7$ 开始。将其乘以 $5$ 得
    \[5^2 \equiv 5 \cdot 5 \equiv 25 \equiv 4 \mod 7\]
    我们发现 $25 = 21 + 4$,并且知道 $21$ 是 $7$ 的倍数,从而得出上述结论。(当数字较小时,我们常常可以通过简单观察进行算术运算。也就是说,我们可以直接心算。当然,如果不确定,我们也可以使用除法算法,从 $25$ 中不断减去 $7$,直到剩下余数。)

    接着我们发现
    \[5^3 \equiv 5^2 \cdot 5 \equiv 4 \cdot 5 \equiv 20 \equiv 6 \mod 7\]
    我们发现,通过``观察''可以知道 $20 = 14+6$。请注意,我们现在知道 $5^3$ 模 $7$ 的余数是多少,但并不需要实际计算 $5^3 = 125$ 然后再化简。因为我们在此过程中已经将所有数字都化简到模 $7$ 的余数,所以省去了大量计算。具体来说,我们总是将数字化简到\emph{小于} $7$ 的范围内,因此在任何情况下我们需要处理的最大数也只会在 $20$ 到 $30$ 之间。这真是太方便了!让我们继续看看接下来会得到什么结果:
    \begin{align*}
        5^4 &\equiv 5^3 \cdot 5 \equiv 6 \cdot 5 \equiv 30 \equiv 2 \mod 7 \\ 
        5^5 &\equiv 5^4 \cdot 5 \equiv 2 \cdot 5 \equiv 10 \equiv 3 \mod 7 \\
        5^6 &\equiv 5^5 \cdot 5 \equiv 3 \cdot 5 \equiv 15 \equiv 1 \mod 7 
    \end{align*}
    这正是我们要找的结果!我们已经确定 $5^6$ 比 $7$ 的某个倍数多 $1$。这种方法比直接计算 $5^6 = 15625$ 并找出 $15625 = 7 \cdot 2232 + 1$ 要简单得多,不是吗?

    这已经解决了前两个问题:我们发现存在 $5$ 的幂满足所需性质,并且由于我们是从 $k = 1$ 开始逐步找到它的,因此可以确定这是最小的结果。第三个问题留给你来研究,即描述所有满足该性质的数。你可以继续我们的过程,乘以 $5$ 并进行化简。你是否注意到了某种模式?它是什么?尝试提出一个猜想并加以证明!(我们稍后会回到这个例子……)
\end{example}

\begin{example}\label{ex:example6.5.13}
    考虑数字 $474$。它是 $3$ 的倍数吗?也许你只是将它的各位数字加起来 --- $4 + 7 + 4$ 等于 $15$ --- 并注意到 $15$ 是 $3$ 的倍数,从而得出结论 $474$ 也必然是 $3$ 的倍数。(当然,你也可以通过长除法计算出 $474 = 3 \cdot 158$。)为什么你可以这样做呢?是不是因为你的老师在三年级时告诉了你这个方法,你就记住了?这对我们来说远远不够!$\smiley{}$

    这里,我们将严格\textbf{证明},一个自然数 $x$ 能被 $3$ 整除当且仅当其各位数字之和也能被 $3$ 整除。(在证明过程中,包含了一些用具体例子来详细说明的语句。这些例子是为了帮助你理解我们所写的内容,我们将其放在括号中,以提醒你,仅仅展示一个例子并不是一个正式的证明。例子可以帮助读者更容易理解实际的证明,但仅凭一个例子不足以证明这个普遍适用的结论。)

    \begin{proof}
        设 $x \in \mathbb{N}$ 为任意固定自然数。我们可以用它的十进制展开形式来表示这个数字
        \[x= \sum_{k=0}^{n-1} x_k \cdot 10^k\]
        其中 $n$ 为数字 $x$ 的位数,$x_k$ 为对应 $10^k$ 位的数字,所以 $0 \le x_k \le 9$。(也就是说,$x_k$ 是从右往左读取的 $x$ 的第 $(k + 1)$ 位数。)

        (例如,我们可以将 $47205$ 写做 $47205=4 \cdot 10^4+7 \cdot 10^3+2 \cdot 10^2+0 \cdot 10^1+5 \cdot 10^0$。本例中,$x_0 = 5, x_1=0, x_3=2$。)

        整除技巧声称
        \[x \equiv 0 \mod 3 \iff \sum_{k=1}^{n-1} x_k \equiv 0 \mod 3\]

        为了证明这一点,我们将考虑十进制展开模 $3$。注意,由于 $10=9+1$,因此 $10 \equiv 1 \mod 3$。因此
        \[\forall k \in \mathbb{N} \cup \{0\} \centerdot 10^k \equiv 1^k \equiv 1 \mod 3\]

        (这源于模算术引理和 $1^k = 1$ 对任意 $k$ 都成立的事实。思考一下!)

        这使我们能够在十进制展开中用 $1$ 替换 $10$ 的幂!因此

        \begin{align*}
            x \equiv 0 \mod 3 &\iff \sum_{k=0}^{n-1} x_k \cdot 10^k \equiv 0 \mod 3 &\text{将}\; x \;\text{重写为十进制展开形式}\\
            &\iff \sum_{k=0}^{n-1} x_k \cdot 1^k \equiv 0 \mod 3 &\text{因为}\; 10 \equiv 1 \mod 3 \\
            &\iff \sum_{k=0}^{n-1} x_k \equiv 0 \mod 3
        \end{align*}

        以上完成了该声明的证明。
    \end{proof}

    (注意,$3 \mid 47205$ 是因为 $3 \mid (4 + 7 + 2 + 0 + 5)$,也就是说 $3 \mid 18$。实际上,$15735 \cdot 3 = 47205$)。

    有趣的是,我们实际上在这里证明了一个\textbf{更强的}结论。因为前面的陈述\emph{当且仅当}陈述,所以我们知道更多的信息:如果 $x$ 的各位数字之和不是 $3$ 的倍数,那么 $x$ 也不是 $3$ 的倍数,并且有\emph{相同的余数}。例如,$3 \nmid 122$,因为 $3 \nmid 5$;此外,$5 \equiv 2 \mod 3$,所以我们知道 $122 \equiv 2 \mod 3$。(确实,$122 = 3 \cdot 40 + 2$。)

    我们还可以找到并证明类似的关于 $9$ 和 $11$ 的整除技巧(虽然 $11$ 的技巧稍微复杂一些)。甚至还有一个关于 $7$ 的技巧,但很难用文字表达。这些概念将在本章的习题中进一步探讨。
    
    记住这个结论及其证明。这是一个可以在聚会上炫耀的小技巧。你可以挑战你的朋友:他们真的知道\textbf{为什么}这个技巧有效吗?你却知其然也知其所以然!
\end{example}

\subsection{模 $n$ 等价类}

你已经证明了(参见引理 \ref{lemma6.5.9})模 $n$ 同余确实是集合 $\mathbb{Z}$ 上的等价关系。你还证明了(参见定理 \ref{theorem6.4.10})等价关系的等价类能够\emph{划分}底层集合。结合这两个结果,我们知道模 $n$ 同余可以将 $\mathbb{Z}$ 划分成若干个等价类。那么,我们如何表示这些等价类呢?每个类的代表元素应该如何选择呢?

让我们先从两个更简单的问题开始:
\begin{enumerate}[label=(\arabic*)]
    \item $\mathbb{Z}$ 模 $n$ 有多少个等价类?
    \item 这些等价类有多``大''?
\end{enumerate}

\subsubsection*{有多少等价类?}

要回答问题 (1),我们只需要回忆一下如何定义除以 $n$ 的余数。根据除法算法(参见引理 \ref{lemma6.5.2}),当我们将一个数除以 $n$ 时,余数 $r$ 必须满足 $0 \le r \le n-1$。这意味着余数最多有 $n$ 种可能:$0,1,2 \dots$ 一直到 $n-1$。(即,$r \in [n-1] \cup \{0\}$。)那么,是否确实存在余数为这些数的数呢?当然存在,因为我们可以直接使用这些数本身。例如,当我们将 $n-1$ 除以 $n$ 时,余数就是 $n-1$(因为 $n-1 < n$)。由此可见,$\mathbb{Z}$ 模 $n$ 的等价类恰好有 $n$ 个。

通过相同的观察,我们可以确定这些等价类的\emph{代表元素}。由于 $a \equiv b \mod n$ 表示 $a$ 和 $b$ 除以 $n$ 具有相同的余数,那么我们可以声明这两个数属于由这个\emph{余数}所代表的等价类。这个余数 $r$ 必须满足 $0 \le r \le n-1$,我们写做 $a, b \in [r]_{\mod n}$ 来表示 $a$ 和 $b$ 属于由余数 $r$ 代表的等价类(下标 ``$\mod n$'' 表示余数是 $n$ 除得的结果)。

\subsubsection*{这些类有多大?}

让我们通过一个具体的例子来思考这个问题,例如 $n = 4$。对于一个整数 $z \in \mathbb{Z}$ 属于余数为 $0$ 的等价类,这意味着什么?也就是说,如果我们知道 $z \in [0]_{\mod 4}$,我们可以得出哪些关于 $z$ 的结论?

根据模运算的定义,我们知道这意味着 $z$ 被 $4$ 除后余数为 $0$。也就是说,$z$ 是 $4$ 的\emph{倍数}。在整数集 $\mathbb{Z}$ 中,$4$ 的倍数有多少个呢?答案是无穷多个!例如,$4, 8, 12, 16, \dots ,$ 以及 $0,-4,-8,-12,\dots$。因此,集合 $[0]_{\mod 4}$ 是一个\emph{无限}集合。

那么,$z \in [1]_{\mod 4}$ 又意味着什么呢?余数为 $1$ 意味着 $z$ 可以表示为 $4k + 1$;也就是说,\emph{存在}一个整数 $k$,使得 $z$ 可以这样表示。那 $k$ 可以取什么值呢?实际上,任意 $k \in \mathbb{Z}$ 都会生成一个这样的 $z$,因此我们可以考虑 $k = 0, k = 1, k = 2, \dots$ 以及 $k = -1, k = -2, \dots$ 看看结果如何。我们发现这会生成一个集合
\begin{align*}
    [1]_{\mod 4} &= \{\dots , -7, -3, 1, 5, 9, \dots \} \\
    &= \{z \in \mathbb{Z} \mid \exists k \in \mathbb{Z} \centerdot z = 4k + 1\} \\
    &= \{4k + 1 \mid k \in \mathbb{Z}\}
\end{align*}
请注意,我们一开始用了 ``$\dots$'' 符号来展示我们发现的模式,随后又用集合构建符(以两种不同的方式)重写了这个集合。

这也是一个无限集合。你可以尝试使用其他余数(无论是除以 $4$ 还是其他任意整数 $n$),来发现这些集合都是\emph{无限的}。(此外,我们尚未\emph{正式}定义什么是无限集合,但我们依赖的是共同的直觉。如果你想更好地理解,可以这样想:这个集合是无限的,因为我们可以列出它的所有元素,并找到一个生成所有元素的模式,但这个过程在有限时间内无法结束。)

\subsubsection*{$\mathbb{Z}$ 模 $n$ 的划分}

我们可以根据对等价类的观察,来总结 $\mathbb{Z}$ 模 $n$ 的等价类的标准表示。已知有 $n$ 个等价类,每个等价类包含无穷多个元素。每个等价类对应于整数除以 $n$ 的余数。由于余数必须满足 $0 \le r \le n-1$,我们将集合 $\{0, 1, 2, \dots , n-1\} = [n-1] \cup \{0\}$ 作为标准代表集合。

余数为 $r$ 的等价类包含所有除以 $n$ 余 $r$ 的整数。换句话说,所有 $z \in [r]_{\mod n}$ 的元素都是 $n$ 的某个倍数加上 $r$。也就是说,我们可以通过从 $r$ 开始不断加上或减去 $n$ 来生成等价类的所有元素。这样,同一等价类中的任何两个元素相差 $n$ 的倍数。

\setlength{\fboxrule}{2pt}
\setlength\fboxsep{5mm}
\begin{center}
\noindent \fcolorbox{blue}{white}{%
    \parbox{0.85\textwidth}{%
        \linespread{1.5}\selectfont
        \textcolor{blue}{\textbf{$\mathbb{Z}$ 模 $n$ 等价类:}}\\
        给定 $n \in \mathbb{N}$,恰好存在 $n$ 个等价类
        \[[0]_{\mod n}, [1]_{\mod n}, [2]_{\mod n}, \dots ,[n-1]_{\mod n}\]
        它们的特点是:
        \begin{align*}
            [0]_{\mod n} &=  \{\dots, -2n, -n, 0, n, 2n, \dots \} \\
            &= \{z \in \mathbb{Z} \mid \exists k \in \mathbb{Z} \centerdot z = kn\}\\
            [1]_{\mod n} &=  \{\dots, -2n+1, -n+1, 1, n+1, 2n+1, \dots \} \\
            &= \{z \in \mathbb{Z} \mid \exists k \in \mathbb{Z} \centerdot z = kn+1\}\\
            [2]_{\mod n} &=  \{\dots, -2n+2, -n+2, 2, n+2, 2n+2, \dots \} \\
            &= \{z \in \mathbb{Z} \mid \exists k \in \mathbb{Z} \centerdot z = kn+2\}\\
            &\vdots \\
            [n-1]_{\mod n} &=  \{\dots, -n-1, -1, n-1, 2n-1, 3n-1, \dots \} \\
            &= \{z \in \mathbb{Z} \mid \exists k \in \mathbb{Z} \centerdot z = kn+(n-1)\}\\
            &= \{z \in \mathbb{Z} \mid \exists \ell \in \mathbb{Z} \centerdot z = \ell n-1\}
        \end{align*}
    }
}
\end{center}

以上是我们所有观察结果的全面总结。下面是一些具体 $n$ 值的例子。

\begin{itemize}
    \item 考虑 $n=2$。等价类为
        \begin{align*}
            [0]_{\mod 2} &= \{z \in \mathbb{Z} \mid \exists k \in \mathbb{Z} \centerdot z = 2k\} =  \{\text{偶数}\}\\
            &= \{\dots, -6, -4, -2, 0, 2, 4, 6 \dots \}\\
            [1]_{\mod 2} &= \{z \in \mathbb{Z} \mid \exists k \in \mathbb{Z} \centerdot z = 2k+1\} =  \{\text{奇数}\}\\
            &= \{\dots, -5, -3, -1, 1, 3, 5, 7 \dots \}
        \end{align*}
    \item 考虑 $n=3$。等价类为
        \begin{align*}
            [0]_{\mod 3} &= \{z \in \mathbb{Z} \mid \exists k \in \mathbb{Z} \centerdot z = 3k\} =  \{3 \;\text{ 的倍数}\}\\
            &= \{\dots, -9, -6, -3, 0, 3, 6, 9 \dots \}\\
            [1]_{\mod 3} &= \{z \in \mathbb{Z} \mid \exists k \in \mathbb{Z} \centerdot z = 3k+1\} =  \{3 \;\text{ 的倍数加 }\; 1\}\\
            &= \{\dots, -8, -5, -2, 1, 4, 7, 10 \dots \}\\
            [2]_{\mod 3} &= \{z \in \mathbb{Z} \mid \exists k \in \mathbb{Z} \centerdot z = 3k+2\} =  \{3 \;\text{ 的倍数加 }\; 2\}\\
            &= \{\dots, -7, -4, -1, 2, 5, 8, 11 \dots \}
        \end{align*}
    \item 考虑 $n=4$。等价类为
        \begin{align*}
            [0]_{\mod 4} &= \{z \in \mathbb{Z} \mid \exists k \in \mathbb{Z} \centerdot z = 4k\} =  \{4 \;\text{ 的倍数}\}\\
            &= \{\dots, -12, -8, -4, 0, 4, 8, 12 \dots \}\\
            [1]_{\mod 4} &= \{z \in \mathbb{Z} \mid \exists k \in \mathbb{Z} \centerdot z = 4k+1\} =  \{4 \;\text{ 的倍数加 }\; 1\}\\
            &= \{\dots, -11, -7, -3, 1, 5, 9, 13 \dots \}\\
            [2]_{\mod 4} &= \{z \in \mathbb{Z} \mid \exists k \in \mathbb{Z} \centerdot z = 4k+2\} =  \{4 \;\text{ 的倍数加 }\; 2\}\\
            &= \{\dots, -10, -6, -2, 2, 6, 10, 14 \dots \}\\
            [3]_{\mod 4} &= \{z \in \mathbb{Z} \mid \exists k \in \mathbb{Z} \centerdot z = 4k+3\} =  \{4 \;\text{ 的倍数加 }\; 3\}\\
            &= \{\dots, -9, -5, -1, 3, 7, 11, 15 \dots \}
        \end{align*}
\end{itemize}

\subsubsection*{使用等价类}

为什么这很有用?为什么我们要带你了解整数模特定等价关系集合的构建?

$\mathbb{Z}$ 被这些等价类\textbf{划分}这一点非常重要。因此,每当我们在 $\mathbb{Z}$ 模 $n$ 的背景下进行算术运算时,只需要考虑这些等价类,即余数。我们可以将所有整数简化为 ${0, 1, 2, \dots , n-1}$ 这些数字,因为它们代表了所有整数。这样,我们不需要进行大量大数算术运算再找余数;只需处理这些余数即可。让我们通过几个例子来看看这种划分的实际用处。\\

\begin{example}
    考虑下面的声明:
    \[\forall n \in \mathbb{N} \centerdot 6 \mid n^3 + 5n\]
    我们之前让你通过对 $n$ 进行归纳来证明这个问题!(参见 \ref{sec:section5.7} 节的练习 \ref{exc:exercises5.7.15}) 现在,我们将利用等价类来证明这一点!

    考虑 $\mathbb{Z}$ 模 $6$。因为 $\mathbb{N} \subseteq \mathbb{Z}$,根据除以 $6$ 的余数,我们知道每个 $n \in \mathbb{N}$ 必然落在等价类 $[0]_{\mod 6}, [1]_{\mod 6}, [2]_{\mod 6}, [3]_{\mod 6}, [4]_{\mod 6}, [5]_{\mod 6}$ 中的\textbf{一个}。

    我们可以分别检查每种情况。假设 $n$ 属于某个特定的等价类,这样我们就可以计算出 $n^3+5n$ 属于哪个等价类。在每种情况下,我们通过乘法(以及幂运算,即重复乘法)和加法,应用模算术引理 \ref{lemma6.5.10}。
    \begin{align*}
        n \equiv 0 \mod 6 &\implies n^3 + 5n \equiv 0^3 + 5 \cdot 0 \equiv 0 \mod 6 \\
        n \equiv 1 \mod 6 &\implies n^3 + 5n \equiv 1^3 + 5 \cdot 1 \equiv 6 \equiv 0 \mod 6 \\
        n \equiv 2 \mod 6 &\implies n^3 + 5n \equiv 2^3 + 5 \cdot 2 \equiv 18 \equiv 0 \mod 6 \\
        n \equiv 3 \mod 6 &\implies n^3 + 5n \equiv 3^3 + 5 \cdot 3 \equiv 42 \equiv 0 \mod 6 \\
        n \equiv 4 \mod 6 &\implies n^3 + 5n \equiv 4^3 + 5 \cdot 4 \equiv 84 \equiv 0 \mod 6 \\
        n \equiv 5 \mod 6 &\implies n^3 + 5n \equiv 5^3 + 5 \cdot 5 \equiv 150 \equiv 0 \mod 6 
    \end{align*}
    以上每种情况,我们都得到 $n^3 + 5n$ 是 $6$ 的倍数(因为它除以 $6$ 的余数是 $0$)。这说明无论 $n$ 取什么值,$n^3 + 5n$ 都是 $6$ 的倍数。这证明了对于所有 $n \in \mathbb{N}$,该命题成立,从而无需使用归纳论证!
\end{example}

\begin{example}[二次残差]\label{ex:example6.5.15}

    在这个例子中,我们将研究完全平方数。具体来说,我们将探讨完全平方数在被不同的数字除时会产生哪些余数。这个例子非常有趣,因为你会发现,根据除数的不同,余数会呈现出一些独特的模式,你可能会因此想要进一步探索这些模式(如果是这样,那就太好了!)。此外,这个例子还很有用,因为我们的研究会引导我们得出一些其他结果,这些结果在本文和练习中都有证明。特别是,研究完全平方数在探索\textbf{毕达哥拉斯三元组}时非常有帮助;毕达哥拉斯三元组是指满足 $a^2 + b^2 = c^2$ 的整数三元组 $(a, b, c) \in \mathbb{N}^3$。了解完全平方数的性质可以帮助我们证明这些三元组的一些有趣的事实!

    对于以下情况,我们将固定一个特定的 $n \in \mathbb{N}$,然后研究对于每个 $x \in \mathbb{Z}, x^2$ 模 $n$ 的结果。在了解 $\mathbb{Z}$ 模 $n$ 的划分后,我们可以简化为查看所有 $n$ 个可能的模 $n$ 余数,然后平方取模。这些可能的余数称为\textbf{二次残差}(称之为\emph{二次}是因为我们使用完全平方数,称之为\emph{残差}是因为我们寻找余数)。在每种情况下,我们将总结这些可能的二次残差列表。

    $\mathbf{n=2}$:

    我们知道,只有当底数为偶数时,完全平方数才是偶数;只有当底数为奇数时,完全平方数才是奇数。在第 \ref{ch:chapter04} 章中,我们曾通过讨论双条件陈述、量词和证明技巧来验证这些说法。现在无需重新正式证明这些结论;我们可以通过模运算轻松验证这些结果。\\
    设 $x \in \mathbb{Z}$ 为任意固定整数。
    \begin{itemize}
        \item 首先,假设 $x \equiv 0 \mod 2$ (即 $x$ 为偶数)。则应用模算术引理可得 $x^2 \equiv 0 \mod 2$ (即 $x^2$ 为偶数)。
        \item 其次,假设 $x \equiv 1 \mod 2$ (即 $x$ 为奇数)。则应用模算术引理可得 $x^2 \equiv 1 \mod 2$ (即 $x^2$ 为奇数)。
    \end{itemize}
    因此,$\mathbb{Z}$ 模 $2$ 的划分告诉我们,这是唯一需要考虑的情况。
    \begin{quotation}
        \begin{center}
            \large 模 $2$ 二次残差:$\{0, 1\}$
        \end{center}
    \end{quotation}

    $\mathbf{n=3}$: 

    设 $x \in \mathbb{Z}$ 为任意固定整数。应用模算术引理可得:
    \begin{itemize}
        \item $x \equiv 0 \mod 3 \implies x^2 \equiv 0^2 \equiv 0 \mod 3$
        \item $x \equiv 1 \mod 3 \implies x^2 \equiv 1^2 \equiv 1 \mod 3$
        \item $x \equiv 2 \mod 3 \implies x^2 \equiv 2^2 \equiv 4 \equiv 1 \mod 3$
    \end{itemize}
    \begin{quotation}
        \begin{center}
            \large 模 $3$ 二次残差:$\{0, 1\}$
        \end{center}
    \end{quotation}

    $\mathbf{n=4}$: 

    设 $x \in \mathbb{Z}$ 为任意固定整数。应用模算术引理可得:
    \begin{itemize}
        \item $x \equiv 0 \mod 4 \implies x^2 \equiv 0^2 \equiv 0 \mod 4$
        \item $x \equiv 1 \mod 4 \implies x^2 \equiv 1^2 \equiv 1 \mod 4$
        \item $x \equiv 2 \mod 4 \implies x^2 \equiv 2^2 \equiv 4 \equiv 0 \mod 4$
        \item $x \equiv 3 \mod 4 \implies x^2 \equiv 3^2 \equiv 9 \equiv 1 \mod 4$
    \end{itemize}
    \begin{quotation}
        \begin{center}
            \large 模 $4$ 二次残差:$\{0, 1\}$
        \end{center}
    \end{quotation}

    $\mathbf{n=5}$: 

    设 $x \in \mathbb{Z}$ 为任意固定整数。应用模算术引理可得:
    \begin{itemize}
        \item $x \equiv 0 \mod 5 \implies x^2 \equiv 0^2 \equiv 0 \mod 5$
        \item $x \equiv 1 \mod 5 \implies x^2 \equiv 1^2 \equiv 1 \mod 5$
        \item $x \equiv 2 \mod 5 \implies x^2 \equiv 2^2 \equiv 4 \mod 5$
        \item $x \equiv 3 \mod 5 \implies x^2 \equiv 3^2 \equiv 9 \equiv 4 \mod 5$
        \item $x \equiv 4 \mod 5 \implies x^2 \equiv 4^2 \equiv 16 \equiv 1 \mod 5$
    \end{itemize}
    \begin{quotation}
        \begin{center}
            \large 模 $5$ 二次残差:$\{0, 1, 4\}$
        \end{center}
    \end{quotation}

    $\mathbf{n=6}$: 

    设 $x \in \mathbb{Z}$ 为任意固定整数。应用模算术引理可得:
    \begin{itemize}
        \item $x \equiv 0 \mod 6 \implies x^2 \equiv 0^2 \equiv 0 \mod 6$
        \item $x \equiv 1 \mod 6 \implies x^2 \equiv 1^2 \equiv 1 \mod 6$
        \item $x \equiv 2 \mod 6 \implies x^2 \equiv 2^2 \equiv 4 \mod 6$
        \item $x \equiv 3 \mod 6 \implies x^2 \equiv 3^2 \equiv 9 \equiv 3 \mod 6$
        \item $x \equiv 4 \mod 6 \implies x^2 \equiv 4^2 \equiv 16 \equiv 4 \mod 6$
        \item $x \equiv 5 \mod 6 \implies x^2 \equiv 5^2 \equiv 25 \equiv 1 \mod 6$
    \end{itemize}
    \begin{quotation}
        \begin{center}
            \large 模 $6$ 二次残差:$\{0, 1, 3, 4\}$
        \end{center}
    \end{quotation}

    $\mathbf{n=7}$: 

    设 $x \in \mathbb{Z}$ 为任意固定整数。应用模算术引理可得:
    \begin{itemize}
        \item $x \equiv 0 \mod 7 \implies x^2 \equiv 0^2 \equiv 0 \mod 7$
        \item $x \equiv 1 \mod 7 \implies x^2 \equiv 1^2 \equiv 1 \mod 7$
        \item $x \equiv 2 \mod 7 \implies x^2 \equiv 2^2 \equiv 4 \mod 7$
        \item $x \equiv 3 \mod 7 \implies x^2 \equiv 3^2 \equiv 9 \equiv 2 \mod 7$
        \item $x \equiv 4 \mod 7 \implies x^2 \equiv 4^2 \equiv 16 \equiv 2 \mod 7$
        \item $x \equiv 5 \mod 7 \implies x^2 \equiv 5^2 \equiv 25 \equiv 4 \mod 7$
        \item $x \equiv 6 \mod 7 \implies x^2 \equiv 6^2 \equiv 36 \equiv 1 \mod 7$
    \end{itemize}
    \begin{quotation}
        \begin{center}
            \large 模 $7$ 二次残差:$\{0, 1, 2, 4\}$
        \end{center}
    \end{quotation}

    $\mathbf{n=8}$: 

    设 $x \in \mathbb{Z}$ 为任意固定整数。应用模算术引理可得:
    \begin{itemize}
        \item $x \equiv 0 \mod 8 \implies x^2 \equiv 0^2 \equiv 0 \mod 8$
        \item $x \equiv 1 \mod 8 \implies x^2 \equiv 1^2 \equiv 1 \mod 8$
        \item $x \equiv 2 \mod 8 \implies x^2 \equiv 2^2 \equiv 4 \mod 8$
        \item $x \equiv 3 \mod 8 \implies x^2 \equiv 3^2 \equiv 9 \equiv 1 \mod 8$
        \item $x \equiv 4 \mod 8 \implies x^2 \equiv 4^2 \equiv 16 \equiv 0 \mod 8$
        \item $x \equiv 5 \mod 8 \implies x^2 \equiv 5^2 \equiv 25 \equiv 1 \mod 8$
        \item $x \equiv 6 \mod 8 \implies x^2 \equiv 6^2 \equiv 36 \equiv 4 \mod 8$
        \item $x \equiv 7 \mod 8 \implies x^2 \equiv 7^2 \equiv 49 \equiv 1 \mod 8$
    \end{itemize}
    \begin{quotation}
        \begin{center}
            \large 模 $8$ 二次残差:$\{0, 1, 4\}$
        \end{center}
    \end{quotation}

    我们鼓励你继续研究其他二次残差。你甚至可以尝试编写一个计算机程序来生成这些列表。你发现什么规律了吗?对于给定的 $n \in \mathbb{N}$,模 $n$ 的二次残差有多少个?它们分别是什么?你能否确定某些数字在任何给定的列表中一定会出现或一定不会出现?请尝试探索一下吧!
\end{example}

\begin{example}
    让我们将前一个例子中的思想进行推广,看看在特定情况下\emph{三次残差}的情况如何。
    \begin{quote}
        假设 $x, y, z \in \mathbb{Z}$ 满足 $x^3+y^3=z^3$。\\
        证明值 $\{x, y, z\}$中至少有一个是 $7$ 的倍数。
    \end{quote}
    重申一下我们的目标,我们要证明
    \[x \equiv 0 \mod 7 \lor y \equiv 0 \mod 7 \lor z \equiv 0 \mod 7\]
    为此,让我们来看看模 $7$ 的三次残差有哪些。\\
    设 $x \in \mathbb{Z}$ 为任意固定整数。应用模算术引理可得:
    \begin{itemize}
        \item $x \equiv 0 \mod 7 \implies x^3 \equiv 0^3 \equiv 0 \mod 7$
        \item $x \equiv 1 \mod 7 \implies x^3 \equiv 1^3 \equiv 1 \mod 7$
        \item $x \equiv 2 \mod 7 \implies x^3 \equiv 2^3 \equiv 8 \equiv 1 \mod 7$
        \item $x \equiv 3 \mod 7 \implies x^3 \equiv 3^3 \equiv 9 \cdot 3 \equiv 2 \cdot 3 \equiv 6 \mod 7$
        \item $x \equiv 4 \mod 7 \implies x^3 \equiv 4^3 \equiv 16 \cdot 4 \equiv 2 \cdot 4 \equiv 8 \equiv 1 \mod 7$
        \item $x \equiv 5 \mod 7 \implies x^3 \equiv 5^3 \equiv 25 \cdot 5 \equiv 4 \cdot 5 \equiv 20 \equiv 6 \mod 7$
        \item $x \equiv 6 \mod 7 \implies x^3 \equiv 6^3 \equiv (-1)^3 \equiv -1 \equiv 6 \mod 7$
    \end{itemize}
    (注意,为了简化计算,我们将 $6$ 写成 $-1$ 再模 $7$。)

    我们发现唯一的可能值是 $\{0, 1, 6\}$。

    现在,假设我们有一个方程的解,即我们有 $x, y, z \in \mathbb{Z}$ 使得 $x^3 + y^3 = z^3$。每一项 --- $x^3,y^3,z^3$ 模 $7$ 同余于 $0$ 或 $1$ 或 $6$。让我们来看些例子。
    \begin{itemize}
        \item 假设 $x^3 \equiv 0 \mod 7$。则 $y^3$ 可以与 $0$ 或 $1$ 或 $6$ 模 $7$ 同余,我们只需要让 $x^3$ 落在相同的等价类即可。不管怎样,在这种情况下,我们有 $x^3 \equiv 0 \mod 7$。
        \item 假设 $y^3 \equiv 0 \mod 7$。将上面的论证应用于 $x^3$ 和 $z^3$。不管怎样,在这种情况下,我们有 $y^3 \equiv 0 \mod 7$。
        \item 假设 $x^3 \equiv 1 \mod 7$。\\
            为了引出矛盾而假设 $y^3 \equiv 1 \mod 7$。则 $x^3+y^3 \equiv 1+1 \equiv 2 \mod 7$,但 $2$ 不在模 $7$ 的立方残差中,因此这是不可能的。\\
            然而我们发现 $y^3 \equiv 0 \mod 7$ 是可能的,因为 $x^3+y^3 \equiv 1+0 \equiv 1 \mod 7$。\\
            同时我们发现 $y^3 \equiv 6 \mod 7$ 是可能的,因为 $x^3+y^3 \equiv 1+6 \equiv 7 \equiv 0 \mod 7$。\\
            不管怎样,在这种情况下,我们\emph{至少}有一个立方数 --- 要么是 $y^3$ 要么是 $z^3$ --- 与 $0$ 模 $7$ 同余。
        \item 假设 $y^3 \equiv 1 \mod 7$。将上面的论证应用于 $x^3$ 和 $z^3$。我们发现,不管怎样,至少有一个立方数 与 $0$ 模 $7$ 同余。
        \item 假设 $x^3 \equiv 6 \mod 7$。\\
            为了引出矛盾而假设 $y^3 \equiv 6 \mod 7$。则 $x^3+y^3 \equiv 6+6 \equiv 12 \equiv 5 \mod 7$,但 $5$ 不在模 $7$ 的立方残差中,因此这是不可能的。\\
            然而我们发现 $y^3 \equiv 0 \mod 7$ 是可能的,因为 $x^3+y^3 \equiv 6+0 \equiv 6 \mod 7$。\\
            同时我们发现 $y^3 \equiv 1 \mod 7$ 是可能的,因为 $x^3+y^3 \equiv 6+1 \equiv 7 \equiv 0 \mod 7$。\\
            不管怎样,在这种情况下,我们\emph{至少}有一个立方数 --- 要么是 $y^3$ 要么是 $z^3$ --- 与 $0$ 模 $7$ 同余。
        \item 假设 $y^3 \equiv 6 \mod 7$。将上面的论证应用于 $x^3$ 和 $z^3$。我们发现,不管怎样,至少有一个立方数 与 $0$ 模 $7$ 同余。
    \end{itemize}

    我们现在已经知道,无论是哪种情况,总有\textbf{至少}一个立方数与 $0$ 模 $7$ 同余。具体哪个立方数具有这种性质取决于具体情况(有时可能有多个立方数符合),但总有至少一个。

    这对我们很有帮助,因为我们可以回顾一下立方残差列表,会发现一个有趣的现象:\emph{唯一}一个立方数与 $0$ 模 $7$ 同余的底数本身也与 $0$ 模 $7$ 同余!换句话说,
    \[\forall z \in \mathbb{Z} \centerdot z^3 \equiv 0 \mod 7 \implies z \equiv 0 \mod 7\]
    这意味着,在上述每种情况下,我们至少有一个立方数与 $0$ 模 $7$ 同余,这进一步说明我们至少有一个底数与 $0$ 模 $7$ 同余。通过列出所有可能性并分析一些情况,我们已经证明了这个方程\emph{所有可能解}的一个性质,而无需找到具体的解!
\end{example}

现在,尽管所有工作都已经完成,但我们有个不幸的消息:原方程\emph{唯一}的解是\emph{平凡}解,即 $x = y = z = 0$。就是这样!你可以尝试寻找其他解,但都是徒劳的。这个结果是\textbf{费马大定理}的一个特例,该定理指出,对于方程 $x^k + y^k = z^k$(其中 $k \in \mathbb{N}$),只有当 $k = 1$ 或 $k = 2$ 时,才存在非平凡的整数解(即 $x, y, z \in \mathbb{Z}$);也就是说,当 $k \in \mathbb{N} - \{1, 2\}$ 时,唯一的解是 $x = y = z = 0$。

费马在世时曾提到过这个事实,但他从未发表过证明。他在一个笔记本空白处声称自己有一个简短的证明,但空间不足以写下它,不过我们现在知道这可能并不是真的。费马生活在 1600 年代,但这个定理直到 1990 年代才被证明\footnote{安德鲁·怀尔斯 (Andrew Wiles) 于 1994 年证明了费马大定理。--- 译者注}!而且,这个证明涉及了大量在费马之后逐步发展起来的强大数学工具。

如果我们了解这个定理,就能很轻松地证明这个例子中的陈述了!既然唯一的解是 $x = y = z = 0$,那么显然这些值都是 $7$ 的倍数。然而,这样做既没有趣味,也不能让我们练习模算术和等价类。\\

\begin{example}
    这是另一个涉及立方残差的问题:
    \begin{quote}
        假设 $x, y, z \in \mathbb{Z}$ 满足 $x^3+y^3+z^3=3$。\\
        证明值 $x^3 \equiv y^3 \equiv z^3 \mod 9$。
    \end{quote}

    我们这里讨论的是一个特定的\emph{丢番图方程}。丢番图方程是指那些带有多个变量和整数系数的多项式方程。要解这样的丢番图方程,需要找到一组整数,使得这些整数代入方程后能够成立。在这个问题中,我们要证明方程的任意解都必须使得所有项 --- $x^3, y^3, z^3$ --- 在模 $9$ 同余。

    首先,试着找出该方程的几个解,看看具体的例子。我们提供几个简单的例子帮助你入门:例如 $(x, y, z)$ 可以等于 $(1, 1, 1)$ 或 $(4, 4, -5)$。你发现这些解符合我们要求的性质了吗?你还能找到其他解吗?(这个问题比较难,不用太过在上面投入精力。)

    有趣的是,我们甚至不需要识别所有解的具体形式或找到它们,就可以证明这一结论。我们只需要找出模 $9$ 的立方残差:\\
    设 $x \in \mathbb{Z}$ 为任意固定整数。应用模算术引理可得:
    \begin{itemize}
        \item $x \equiv 0 \mod 9 \implies x^3 \equiv 0^3 \equiv 0 \mod 9$
        \item $x \equiv 1 \mod 9 \implies x^3 \equiv 1^3 \equiv 1 \mod 9$
        \item $x \equiv 2 \mod 9 \implies x^3 \equiv 2^3 \equiv 8 \mod 9$
        \item $x \equiv 3 \mod 9 \implies x^3 \equiv 3^3 \equiv 9 \cdot 3 0 \mod 9$
        \item $x \equiv 4 \mod 9 \implies x^3 \equiv 4^3 \equiv 16 \cdot 4 \equiv (-2) \cdot 4 \equiv -8 \equiv 1 \mod 9$
        \item $x \equiv 5 \mod 9 \implies x^3 \equiv 5^3 \equiv 25 \cdot 5 \equiv (-2) \cdot 5 \equiv -10 \equiv 8 \mod 9$
        \item $x \equiv 6 \mod 9 \implies x^3 \equiv 6^3 \equiv 36 \cdot 6 \equiv 0 \cdot 6 \equiv 0 \mod 9$
        \item $x \equiv 7 \mod 9 \implies x^3 \equiv 7^3 \equiv 49 \cdot 7 \equiv 4 \cdot (-2) \equiv -8 \equiv 1 \mod 9$
        \item $x \equiv 8 \mod 9 \implies x^3 \equiv 8^3 \equiv (-1)^3 \equiv -1 \equiv 8 \mod 9$
    \end{itemize}
    注意,在某些情况下,我们会使用负数来简化计算。这是完全可以的,并且对你大有帮助!例如,与其计算 $4^3 = 64$ 然后模 $9$,我们可以用 $-2$ 来代替 $16$ 以保持数字较小。我们可以随时从任何数中加减 $9$ 的倍数,因此在计算过程中可以这样做,而不是先得到一个大数然后再模 $9$。(当然,$64$ 并不是一个很大的数字,所以这一点似乎不太显著;然而,当你处理更大的数字时,这就非常有用。此外,尽可能将数字简化到个位数,可以减少心算错误的发生!)注意,我们在最右边只看到了三种可能性;模 $9$ 的立方残差为 $\{0, 1, 8\}$。就是这样!

    当然,要使 $x^3 + y^3 + z^3 = 3$ 成立,我们需要 $x^3 + y^3 + z^3 \equiv 3 \mod 9$,因为 $3 \equiv 3 \mod 9$。查看可能的立方残差 --- $0, 1, 8$ --- 我们发现\emph{只有} $1 + 1 + 1$ 等于 $3$。试试其他组合:$0 + 1 + 8 \equiv 9 \equiv 0 \mod 9$ 和 $8 + 8 + 8 \equiv 24 \equiv 6 \mod 9$ 等等。这意味着我们需要 $x^3 \equiv y^3 \equiv z^3 \equiv 1 \mod 9$,才能使 $(x, y, z)$ 成为解。

    在解决这个问题时,我们证明了一个略强的结论。我们不仅知道 $x^3,y^3,z^3$ 必须模 $9$ 同余,它们还必须模 $9$ 同余于 $1$。这比我们原本需要的信息更多了一些。

    现在,事实证明,这个问题还有一个\emph{更强的}结论。实际上,$x \equiv y \equiv z \mod 9$。也就是说,不仅它们的\emph{立方}模 $9$ 同余,它们的\emph{底数}也是模 $9$ 同余的。(注意,这并不意味着底数模 $9$ 同余于 $1$;例如,我们的另一个例子 $(4, 4, -5)$ 就表明情况并非如此。)不幸的是,证明这一点需要涉及很多高等数学,超出了本书的范围。不过,这应该能让你理解,这些``简单''的问题(易描述,小数值,纯整数)实际上需要非常复杂和深奥的数学才能解决。不过,不要把这看作是打击,而是启发:只需一点数学知识,我们就能触及这个问题的表面,这暗示了更为深刻和复杂的基础。

    (如果你感兴趣,这里有一篇论文给出了完整的结论,证明了 $x \equiv y \equiv z \mod 9$ 是必然的:
    \begin{quote}
        \href{http://www.ams.org/journals/mcom/1985-44-169/S0025-5718-1985-0771049-4/S0025-5718-1985-0771049-4.pdf}{http://www.ams.org/journals/mcom/1985-44-169/\\S0025-5718-1985-0771049-4/S0025-5718-1985-0771049-4.pdf}
    \end{quote}
    即便是前两段,你也需要查阅一些定义才能阅读下来。完整阅读下来也需要你学习一些相关的数学知识,可能需要几个月也可能需要几年,具体时间取决于你的兴趣。记住这一点,并在你往后的数学生涯中再回头看看!)
\end{example}

\subsection{乘法逆元}\label{sec:section6.5.3}

我们之前在证明模算数引理(引理 \ref{lemma6.5.10})时提到过,不会在 $\mathbb{Z}$ 模 $n$ 的背景下讨论``除法''。本节中,我们将重新探讨这个想法,并解释为什么(以及如何)在某些特定情况下``除法''是合理的。然而,我们要强调的是,我们实际上在讨论一个更广泛的\textbf{乘法逆元}的概念,而\textbf{不是}真正的``除法''。我们将首先通过几个启发性例子来解释这一点,然后我们将陈述并证明这些特定情况下的具体结果。

\subsubsection*{整体概念}

给定一个特定的数学对象,它的\textbf{乘法逆元}是另一个对象,当我们将这两个对象``相乘''时,结果为 ``$1$''。这里我们加上引号是因为``相乘''和 ``$1$'' 的含义在不同的语境中可能会有很大差异。\\

\begin{example}
    让我们先考虑一个熟悉的例子。假设我们讨论的是实数集 $\mathbb{R}$,且使用通常的乘法运算。现在我们取数字 $2$。它的乘法逆元是什么?也就是说,是否存在另一个实数 $x$ 使得 $2 \cdot x = 1$?如果存在,它是多少?很明显,$x = \frac{1}{2}$ 是符合要求的!可以注意到 $2 \cdot \frac{1}{2} = 1$。出于这个原因,我们可以写出
    \[2^{-1} = \frac{1}{2} \quad \text{在 }\; \mathbb{R} \;\text{范围内}\]
    当我们把方程两边同时除以 $2$ 时,实际上是在把方程的两边都\emph{乘以} $2$ 的\emph{乘法逆元}。
\end{example}

\begin{example}
    现在让我们考虑一个可能不太熟悉的例子。想象一个挂钟,钟面上有均匀分布的 $12$ 个小时刻度标记。我们将考虑旋转挂钟,所以我们声明标准位置,即顶部 $12$ 点的位置为 ``$1$''。也就是说,这是没有进行额外旋转的标准表示法,所以我们称其为\emph{单位元}。实际上,我们的 ``$1$'' 就是指 ``$0 \degree$ 旋转'' 后的挂钟。

    现在,让我们假设将两个旋转``相乘''只是依次进行旋转。例如,我们先将时钟顺时针旋转 $45 \degree$,然后再顺时针旋转 $90 \degree$。在这个例子中,我们实际上是将 ``$45 \degree$ 旋转'' 和 ``$90 \degree$ 旋转'' \emph{乘}在一起,结果得到 ``$135 \degree$ 旋转''。

    建立这些约定的目的是为了明确我们的上下文、对象、``相乘''的含义以及``1''的定义,从而能够识别任意旋转的\emph{乘法逆元}。如果你仔细想一下,就会发现如果我们将 ``$\theta$ 度旋转'' 与 ``$360-\theta$ 度旋转'' 相乘,那么我们实际上是将时钟旋转了 $360 \degree$,并回到了标准位置,这正是我们在这个上下文中的 ``$1$''。这意味着,在我们当前上下文中
    \[(\theta \;\text{度旋转})^-1 = 360-\theta \;\text{度旋转}\]
\end{example}

这两个例子旨在说明,\emph{逆元}的概念是一个普遍的概念,并不局限于数字\emph{除法}这种标准上下文。事实上,当我们讨论\emph{函数的逆}时,也会看到类似的例子。(在这个上下文中,``相乘''指的是函数的复合,``$1$'' 是指恒等函数。虽然具体内容会在下一章详细介绍,但我们现在提到这一点,是为了让已经熟悉这些概念的读者有一个预先的理解。)

\subsubsection*{互质}

你可能已经熟悉以下定义。我们将在后续的结果中用到它,这个结果将说明在 $\mathbb{Z}$ 模 $n$ 的情况下何时存在乘法逆元,因此我们现在想重申这个定义并展示一些例子。

\begin{definition}
    给定 $x,y \in \mathbb{Z}$,我们说 $x$ 和 $y$ \dotuline{互质}当且仅当它们没有除 $1$ 以外的公因子。
\end{definition}

(\textbf{注意}:``互质'' 表示 $x$ 和 $y$ 彼此互质,并不是说 $x$ ``类似质数''或其他意思。)\\

\begin{example}
    例如,$12$ 和 $35$ 互质,因为 $12 = 2^2 \cdot 3$,而 $35 = 5 \cdot 7$,因此它们没有任何公因数。

    通常,写出\emph{质因数分解}是有帮助的,因为我们实际上是想知道两个数是否有共同的质因数(这意味着它们有公因数)。

    举个反例,$12$ 和 $33$ 不互质,因为 $3 \mid 12$ 且 $3 \mid 33$。
\end{example}

\begin{example}
    这个例子陈述的结果在后面会很有用。

    \textbf{声明}:如果 $p$ 是质数且 $a$ 不是 $p$ 的倍数,那么 $p$ 和 $a$ 互质。

    (也就是说,如果 $p$ 是质数且 $p \nmid a$,则 $p$ 和 $a$ 互质。)

    让我们来看看为什么这是正确的!

    \begin{proof}
        设 $p$ 为质数且 $a \in \mathbb{Z}$。假设 $p \nmid a$。

        由于 $p \nmid a$,所以 $a$ 的质因数中不包含 $p$。因为 $p$ 是质数,因此 $a$ 的质因数也不会整除 $p$。这意味着 $a$ 和 $p$ 没有共同的质因数,因此它们互质。
    \end{proof}

    这非常方便!特别是,我们现在知道,只要 $p$ 是质数,那么\textbf{所有}数字 $1, 2, 3, \dots , p-1$ 都与 $p$ 互质。
\end{example}

\subsubsection*{定义与示例}

我们来讨论一下在 $\mathbb{Z}$ 模 $n$ 的情况下,什么是\emph{乘法逆元}。这里的``乘法''指的是常规乘法,但所有结果都要模 $n$。另外,``$1$'' 实际上是对应于 $1$ 的\emph{等价类}。在这种情况下,我们说对于任意 $x \in \mathbb{Z}$,当且仅当 $xy \equiv 1 \mod n$ 时,$x$ 的乘法逆元(记作 $x^{-1}$)等于 $y$。也就是说:
\[\forall x \in \mathbb{Z} \centerdot \forall y \in \mathbb{Z} \centerdot y \equiv x^{-1} \mod n \iff xy \equiv 1 \mod n\]
请注意,这些声明都是在 $\mathbb{Z}$ 模 $n$ 的情况下进行的,因此我们不会写 ``$y = x^{-1}$''。数字 $x$ 表示一个等价类,$x^{-1}$ 也是如此。

让我们来练习一下如何\emph{找到}这些乘法逆元,或者判断它们何时不存在。关键在于以下几点:
\begin{quotation}
    如果 $x \cdot y \equiv 1 \mod n$,则对于所有 $k \in \mathbb{Z}, x \cdot (y+kn) \equiv 1 \mod n$
\end{quotation}
要想理解其中的原因,我们可以将右边表达式中的 $x$ 利用分配律展开:
\[x \cdot (y+kn) \equiv xy+xkn \equiv xy+n(xk) \equiv xy+0 \equiv xy \equiv 1 \mod n\]
也就是说,在展开过程中,给 $y$ 加上 $n$ 的倍数只会得到 $n$ 的倍数,而我们在模 $n$ 时可以``忽略''这些倍数。

由此我们可以得出以下结论:\textbf{如果} $x$ 在模 $n$ 下有一个乘法逆元,\textbf{那么} 
\begin{enumerate}[label=(\alph*)]
    \item 存在\emph{无穷多}个这样的逆元,并且它们都属于同一个模 $n$ 的等价类;
    \item 但是在集合 $\{1, 2, 3, \dots , n-1\}$ 中,我们可以找到\emph{唯一一个}这样的逆元。
\end{enumerate}

这些事实非常有趣且实用。特别是,它告诉我们无需进行复杂的存在性论证来寻找乘法逆元:只需逐一检查每种情况,直到找到一个。如果找不到,就说明不存在。换句话说,我们不必凭直觉臆断或随意猜测,而是有一个更为系统的猜测和检查算法。

让我们通过下面的例子来看看它在实际中的应用。\\

\begin{example}
    在这个例子中,我们会给出一个 $n \in \mathbb{N}$ 和一个 $x \in \mathbb{Z}$,然后寻找一个满足 $y \equiv x^{-1} \mod n$ 的 $y$。如果这样的逆元不存在,我们将说明原因。
    \begin{itemize}
        \item $\mathbf{n=3, x=2}$:\\
            我们只需检查 $y = 1$ 和 $y = 2$。注意到 $2 \cdot 2 \equiv 4 \equiv 1 \mod 3$,所以
            \[2^{-1} \equiv 2 \mod 3\]
        \item $\mathbf{n=4, x=3}$:\\
            我们只需检查 $y = 1, y = 2, y = 3$。注意到 $3 \cdot 3 \equiv 9 \equiv 1 \mod 4$,所以
            \[3^{-1} \equiv 3 \mod 4\]
        \item $\mathbf{n=4, x=2}$:\\
            我们只需检查 $y = 1$ 和 $y = 2$。然而,由于 $x$ 是偶数,所以 $x$ 的任何倍数也都是偶数,而满足 $y \equiv 1 \mod 4$ 的数必须是奇数。因此,$2$ 在模 $4$ 下没有乘法逆元。
        \item $\mathbf{n=10, x=3}$:\\
            我们可以在这里逐一检查所有情况:
            \begin{align*}
                3 \cdot 1 &\equiv 3 \mod 10 \\
                3 \cdot 2 &\equiv 6 \mod 10 \\
                3 \cdot 3 &\equiv 9 \mod 10 \\
                3 \cdot 4 \equiv 12 &\equiv 2 \mod 10 \\
                3 \cdot 5 \equiv 15 &\equiv 5 \mod 10 \\
                3 \cdot 6 \equiv 18 &\equiv 8 \mod 10 \\
                3 \cdot 7 \equiv 21 &\equiv 1 \mod 10 \\
            \end{align*}
            这意味着
            \[3^{-1} \equiv 7 \mod 10\]
            注意,这也同时表明
            \[7^{-1} \equiv 3 \mod 10\]
            因为乘法具有交换律(即顺序不重要)。这一观察使我们得出以下结论:
            \[(a^{-1})^{-1} \equiv a \mod n \quad \text{假设}\; a^{-1} \;\text{存在}\]
        \item $\mathbf{n=15, x=7}$:\\
            如果我们检查 $7$ 的所有倍数,会发现当我们检查到 $13$ 时,我们就成功了:
            \[7 \cdot 13 \equiv 91 \equiv 6 \cdot 15 + 1 ≡ 1 \mod 15\]
            所以
            \[7^{-1} \equiv 13 \mod 15\]
            验证工作留给你来完成。例如在模 $15$ 下 $6$ 没有乘法逆元。 
    \end{itemize}
\end{example}

\subsubsection*{何时存在乘法逆元?}

现在我们已经研究了一些例子,是时候静下心来,描述所有乘法逆元存在的情况。以下引理描述了这些情况。

\begin{lemma}[互质时的乘法逆元引理或 MIRP 引理]
    假设 $n \in \mathbb{M}, a \in \mathbb{Z}$ 且 $n, a$ \dotuline{互质}。考虑同余式 $a \cdot x \equiv 1 \mod n$,那么存在解 $x \in \mathbb{Z}$,使得它满足该同余式。

    事实上,这个同余式有无穷多个解,并且它们都模 $n$ 同余。这意味着在集合 $[n - 1] = \{1, 2, ... , n-1\}$ 中恰好存在一个解。

    我们用 $a^{-1}$ 来表示该同余方程解的等价类,并称其为 $a$ 模 $n$ 的\dotuline{乘法逆元}。

    此外,这是一个充要条件;也就是说,如果 $a$ 和 $n$ 不互质,那么同余方程 $a \cdot x \equiv 1 \mod n$ 在整数范围内无解。
\end{lemma}

这个引理完全说明了乘法逆元何时存在,何时不存在。我们可以用它来判断如下同余式
\[15x \equiv 1 \mod 33\]
在 $x \in \mathbb{Z}$ 下\textbf{误解},因为 $3 \mid 15$ 且 $3 \mid 33$,所以它们不是互质的。同理,我们可以用它来判断如下同余式
\[40x \equiv 1 \mod 51\]
在 $x \in \mathbb{Z}$ 下\textbf{必}有解,因为 $40 = 2^3 \cdot 5$ 和 $51 = 3 \cdot 17$ 互质。(注意,这个引理在帮助我们\emph{找到}解时只提供了一部分信息;它只是保证我们可以在 $\{1, 2, \dots , n-1\}$ 的元素中找到解。)

为了\textbf{证明}这个引理,我们将其分成两部分,因为它是一个双向陈述。我们将为你证明其中一个方向;即当 $a$ 和 $n$ 互质时,$a^{-1}$ 在模 $n$ 下存在。另一个方向(如果 $a$ 和 $n$ 有公因子,那么 $a^{-1}$ 在模 $n$ 下不存在)将会在习题 \ref{ex:exercises6.7.21} 中引导你完成证明。(你可以现在就试试!)在证明过程中,我们需要用到以下有用的引理。

\begin{lemma}[欧几里得引理]\label{lemma6.5.25}
    给定 $a, b, c \in \mathbb{Z}$。假设 $a \mid bc$,并假设 $a$ 和 $b$ 互质。则 $a \mid c$。
\end{lemma}

我们会推迟对这个引理的证明,直到看到 MIRP 引理的证明。我们认为,详细探讨这个引理的证明细节可能会暂时分散我们对本节主要目标的注意力。此外,欧几里得引理的结果本身已经相当可信,我们可以暂时假定其有效性,并在 MIRP 引理的证明中使用它。请看以下几个例子:
\begin{itemize}
    \item 我们知道 $3 \mid 30$,且 $30 = 5 \cdot 6$。由于 $3$ 和 $5$ 互质,我们可以推断 $3 \mid 6$,而这当然是对的。
    \item 假设对于某个整数 $x, 3 \mid 5x$。我们能得出关于 $x$ 的什么结论呢?由于 $3$ 和 $5$ 互质,所以要使 $5x$ 成为 $3$ 的倍数,$x$ 必须``包含''一个 $3$ 的因子。也就是说,$3 \mid x$ 是必要条件。
\end{itemize}
我们意识到这还不够令人满意!我们并不是说要在没有证据的情况下\emph{接受}这个说法;我们只是希望在深入讨论之前稍等片刻。同时,你可以试着自己证明一下!看看你能得出什么结论。

现在,让我们继续向前,证明 MIRP 引理(假设在中间某个步骤会用到欧几里得引理的结果)。

\begin{proof}
    设 $n \in \mathbb{N}, a \in \mathbb{Z}$。假设 $a$ 和 $n$ \textbf{互质}。
    
    我们要证明 $\exists x \in \mathbb{Z} \centerdot ax \equiv 1 \mod n$。

    考虑 $a$ 的前 $n$ 个倍数组成的集合;也就是说,定义集合 $N$ 为
    \begin{align*}
        N &= \{0, a, 2a, 3a, \dots ,(n-1)a\} \\
        &= \{z \in \mathbb{Z} \mid \exists k \in [n - 1] \cup \{0\} \centerdot z = ka\}
    \end{align*}
    请注意,集合 $N$ 中有 $n$ 个元素。

    \textbf{声明}:集合 $N$ 的所有元素在模 $n$ 运算下会产生\emph{不同}余数;也就是说,
    \[\forall i, j \in [n-1] \cup \{0\} \centerdot i \ne j \implies ai \not\equiv aj \mod n\]

    我们来证明这个声明。首先,为了引出矛盾而假设该声明为\verb|假|。

    这意味着 $\exists i, j \in [n-1] \cup \{0\} \centerdot ai \equiv aj \mod n$。假设给定这样的 $i$ 和 $j$。

    通过减法和因式分解,我们可以得到 $ai - aj \equiv a(i-j) \equiv 0 \mod n$。

    这意味着 $n \mid a(i-j)$。已知 $n$ 和 $a$ 互质。根据上面的引理 \ref{lemma6.5.25},我们可以推导出 $n \mid i-j$。

    现在,我们可以推断出 $i = j$。请记住 $i, j \in [n-1] \cup \{0\}$,因此 $0 \le i, j \le n-1$,也就是说 $-(n-1) \le -j, ij \le 0$。

    将这些关于 $i$ 和 $-j$ 的不等式相加后,我们可以发现
    \[-(n-1) + 0 = n-1 \le i + (-j) = i - j \le n - 1 = (n-1) + 0\]
    也就是说,$-(n-1) \le i-j \le n-1$。我们已知 $n \mid i - j$,即 $i-j$ 是 $n$ 的倍数。请注意,在 $-(n-1)$ 和 $+(n-1)$ 之间,$n$ 的倍数\emph{只有} $0$。

    因此 $i-j=0$,即 $i=j$。这就证明了当前的声明。

    我们现在可以确定,$N$ 的元素在模 $n$ 下会产生不同的余数。而且,这些可能的余数是 $\{0, 1, 2, \dots, n-1\} = [n-1] \cup \{0\}$。注意到 $N$ 有 $n$ 个不同的元素,而模 $n$ 下也有 $n$ 个不同的余数(等价类)。这意味着,在集合 $N$ 中,每个模 $n$ 的余数都\emph{恰好出现一次}。

    这表明,$N$ 中\emph{恰好}有一个元素(即一个 $a$ 的倍数)对应于模 $n$ 余数为 $1$。这个 $N$ 中的元素可以表示为 $ax$,其中 $x \in [n-1] \cup \{0\}$。假设给定这样的 $x$。这就是引理中所述同余方程的解。
\end{proof}

真是花了不少功夫,但我们终于到了这一步。既然你已经证明了练习 \ref{ex:6.7.21} 中的论点(确实如此,对吧?$\smiley{}$),我们现在\emph{确切}知道在 $\mathbb{Z}$ 模 $n$ 的情况下,乘法逆元何时存在。我们也知道了一种合理的方法来找到它们:只需检查 $a$ 的前 $n-1$ 个倍数,找出其中模 $n$ 等于 $1$ 的倍数。

既然我们已经完成了这一步,现在让我们回头证明一下欧几里得引理。这一步是必要的,因为重要的 MIRP 引理的证明依赖于这个结果。注意,这个证明中包含一个复杂的\emph{归纳论证}。具体来说,我们有\emph{两个变量} $a$ 和 $b$,我们需要证明某个陈述对所有这样的 $a$ 和 $b$ 都成立。

\begin{proof}
    设 $a, b, c \in \mathbb{Z}$。假设 $a \mid bc$,且 $a$ 和 $b$ 互质。

    我们要证明必然有 $a \mid c$。我们先来证明:

    \textbf{声明}:如果 $a, b \in \mathbb{N}$ 且 $a$ 和 $b$ 互质,则 $\exists x, y \in \mathbb{Z} \centerdot ax+by = 1$。

    基于这个声明,结果将很容易得出。我们将这个声明的证明放在一个框中,方便阅读。在框之后,你会看到我们如何使用这个结果来证明引理的原始陈述。

    (在进行这个证明之前,可以先用一些例子来``说服''自己这个证明为\verb|真|。取两个互质的数,例如 $5$ 和 $11$,或者 $15$ 和 $22$,抑或 $10$ 和 $23$,尝试构造\emph{线性组合}来得到 $1$。然后,取一些有公因子的数,例如 $5$ 和 $10$,或者 $6$ 和 $15$,抑或 $21$ 和 $27$,尝试理解为什么你找不到这样的组合。)

    \begin{tcolorbox}[colback=gray!10,%gray background
        colframe=black,% black frame colour
        width=\textwidth,% Use 5cm total width,
        arc=2mm, auto outer arc,
        title={证明声明},breakable,enhanced jigsaw,
        before upper={\parindent15pt\noindent},	]
            我们将通过对 $a+b$ 进行归纳来证明这一点。在开始之前,先来看几个事实:
            \begin{itemize}
                \item 如果 $a=1$ 或 $a=-1$,则 $b$ 必须为 $0$ 或 $1$ 才能满足它们互质。\\
                    无论哪种情况,我们都可以令 $x=a, y=0$ 从而写出
                    \[ax + by = a^2 + 0 = 1\]
                    同样的论证亦可以应用于 $b=1$ 或 $b=-1$ 的情况(此时 $a$ 必为 $0$ 或 $1$)。
                \item 如果 $b=0$ 且 $|a| \ge 2$(即 $a \ne \pm 1$),则 $a, b$ 有公因子 $a$,因此它们不互质。\\
                    同样的论证亦可以应用于 $a=0$ 的情况。
            \end{itemize}
            综上,我们可以忽略 $a$ 或 $b$ 为 $0$ 的情况。换句话说,我们只考虑 $|a| \ge 1$ 和 $|b| \ge 1$ 的值。\\

            \begin{itemize}
                \item 因为 $a$ 和 $b$ 互质,所以 $-a$ 和 $b$ 也互质(同理,$-a$ 和 $b$ 以及 $a$ 和 $b$ 也都互质)。这是因为取相反数只会改变整数的符号,不会影响其因子。
                \item 如果已知 $\exists x, y \in \mathbb{Z} \centerdot ax + by = 1$,则必有
                    \[(-a)(-x) + (-b)(-y) = ax + by = 1\]
                    因为 $-x,-y \in \mathbb{Z}$,这说明 $-a$ 和 $-b$ 也可以有这样的表示。
            \end{itemize}
            综上,我们只需要考虑 $a$ 和 $b$ 为\emph{正数}的情况。(换句话说,如果 $a$ 或 $b$ 为负数,我们只需取它们的相反数即可。)\\

            结合之前的推论,我们可以推断出只需要考虑 $a,b \in \mathbb{N}$ 的情况。证明这些值的结果,再结合我们之前的观察,就能得出完整的结论。

            现在,我们可以通过对 $a + b$ 应用(强)归纳法来进行证明。由于 $a,b \in \mathbb{N}$,因此 $a + b \ge 2$。我们之前已经考虑了基本情况 $a + b = 2$,但为了完整性,这里再重述一遍。

            给定 $a,b \in \mathbb{N}$,定义 $P(a,b)$ 为陈述
            \[a \;\text{和}\; b \;\text{互质} \implies \exists x, y \in \mathbb{Z} \centerdot ax + by = 1\]

            \textbf{基本情况}:考虑 $P(2)$,即假设 $a,b \in \mathbb{N}$,$a$ 和 $b$ 互质且 $a+b=2$。这意味着 $a=b=1$,我们可以令 $x=1, y=0$ 得出
            \[ax + by = 1 + 0 = 1\]
            因此,$P(2)$ 成立。

            \textbf{归纳假设}:设 $k \in \mathbb{N}$ 为任意固定自然数。假设 $P(2) \land P(3) \land \dots \land P(k)$ 成立。(也就是说,假设每当两个互质数之和等于 $2,3, \dots, k$ 时,我们都能找到它们的一个\emph{线性组合}使其等于 $1$。)

            \textbf{归纳步骤}:我们要证明 $P(k+1)$ 成立。也就是说,设 $a, b \in \mathbb{N}$,且 $a + b = k + 1$,并假设 $a$ 和 $b$ 互质;我们要证明 $\exists x, y \in \mathbb{Z} \centerdot ax + by = 1$。

            首先,根据对称性,我们可以假设 $a \ge b$。(也就是说,给定 $a$ 和 $b$ 的值。无论它们是什么值,我们都可以对它们进行``重命名'',因为其中一个值至少与另一个一样大;我们将较大的那个标记为 $a$。)事实上,由于 $a$ 和 $a$ 不互质(当 $a \ge 2$ 时),我们甚至可以假设 $a > b$。

            现在,我们要利用 $b$ 和 $a - b$ 互质这一事实。要理解为什么会这样,我们需要证明 $b$ 和 $a - b$ 只有 $1$ 这个公约数。

            设 $d$ 为 $b$ 和 $a - b$ 的公因数,即 $d$ 能整除 $b$ 和 $a-b$。这意味着 $d$ 能整除 $b + (a-b)$,即 $d$ 能整除 $a$。我们已经知道 $d$ 能整除 $b$,因此 $d$ 实际上是 $a$ 和 $b$ 的公因数,所以它必然为 $1$。因此,$b$ 和 $a-b$ 互质。

            (我们刚刚证明的是
            \[(d \mid b \land d \mid a - b) \implies d \mid a \land d \mid b\]
            该声明也是一个 $\iff$ 陈述。我们鼓励你思考一下为什么 $\impliedby$ 方向也成立。)

            我们现在有 $b, a-b \in \mathbb{N}$(因为 $b < a$)互质。还要注意,$b + (a-b) = a < a + b = k + 1$,因为 $b \in \mathbb{N}$(所以 $b \ge 1$)。这意味着 $a + b \le k$,因此归纳假设 $P(a + b)$ 适用!

            (注意 $P(a + b)$ 不一定等同于 $P(k)$,因此我们需要使用强归纳法!)

            陈述 $P(a + b)$ --- 即 $P(b + (a - b))$,我们将用到这一点 --- 告诉我们 $b$ 和 $a - b$ 的线性组合可以得到 $1$;也就是说,
            \[\exists u, v \in \mathbb{Z} \centerdot ub + v(a - b) = 1\]
            我们现在要把它转换为 $a$ 和 $b$ 的线性组合,以得到 $1$。为此,我们将重新写这个方程,并重新标记系数:
            \[ub + v(a-b) = 1 \iff \underbrace{v}_{x} a + b \underbrace{(u - v)}_{y} = 1\]
            也就是说,我们现在可以定义 $x = v$ 和 $y = u - v$,这样 $x, y \in \mathbb{Z}$,并且满足 $ax + by = 1$。

            我们现在已经证明了 $P(a + b)$(即 $P(k + 1)$)成立。通过强归纳法,我们推导出 $P(n)$ 对于所有 $n \in \mathbb{N}$ 且 $n \ge 2$ 都成立。
    \end{tcolorbox}	

    为了提醒大家,这个证明的结果是,我们现在知道任意互质的数都可以通过线性组合得到 $1$。

    让我们回到引理的原始陈述。已知 $a, b, c \in \mathbb{N}$,并假设 $a$ 和 $b$ 互质且 $a \mid bc$。

    第一个假设说明存在整数 $x$ 和 $y$,使得 $ax + by = 1$。给定这样的 $x$ 和 $y$。

    第二个假设说明存在整数 $k$,使得 $bc = ak$。给定这样的 $k$。
    
    接下来,我们将第一个假设中的方程乘以 $c$,然后应用第二个假设:
    \[ax + by = 1 \implies acx + (bc)y = 1 \implies acx + (ak)y = c \implies c = a\underbrace{(cx + ky)}_{\ell}\]
    也就是说,通过 $ax + by = 1$,我们可以推导出 $c = a\ell$,其中 $\ell \in \mathbb{Z}$,并且 $\ell$ 是由其他整数定义的。

    根据定义,这意味着 $a \mid c$。这就证明了最初的陈述。
\end{proof}

哇!这个证明包含了很多内容。请你多读几遍,逐行理解并做好笔记。你能明白为什么每个声明都基于我们已知的内容吗?你能看出归纳法是如何应用的吗?虽然我们有两个变量,但我们对其中一个变量进行归纳,这个变量被定义为另外两个变量的和。我们知道这是一个复杂的证明,因此将它放在这里,紧跟在本节更重要的 MIRP 引理之后。

让我们利用这个结果 --- \emph{准确}知道何时存在乘法逆元 --- 来解决一系列问题吧!

\subsubsection*{使用乘法逆元}

这有何用处?虽然这个答案听起来有点调皮,但它确实是正确的:乘法逆元在使用模运算解决同余问题时非常有用。乍一看,似乎我们是为了这些问题而开发了数学工具,但事实并非如此。实际上,正如你将在接下来的例子中看到的那样,在尝试解决这些问题时,你很可能会发明出我们即将应用的这些技术。换句话说,即使你没有学过乘法逆元,也可以尝试解决这些问题,但最终你会重新发现我们已经探讨过的结果。

好了,铺垫就到这里。让我们看几个具体的问题。这些问题的形式都是:``有一个同余方程;找出该方程的所有整数解,或者证明其无解。''\\

\begin{example}\label{ex:example6.5.26}
    找到所有整数 $x, y \in \mathbb{Z}$ 满足
    \[3x - 7y = 11\]
    我们声称有无穷多对 $(x, y) \in \mathbb{Z} \times \mathbb{Z}$ 满足这个方程。此外,我们可以给出所有解的形式,并通过定义这些解的集合来实现。

    通过重写给定方程,我们想找出所有 $x \in \mathbb{Z}$ 使得
    \[3x \equiv 11 \mod 7\]
    假如我们能找到 $x \in \mathbb{Z}$ 所有整数解,我们可以轻松地通过解上面方程得到对应 $y \in \mathbb{Z}$ 的解:$y = \frac{3x-11}{7}$。

    注意到 $3^{-1} \equiv 5 \mod 7$,这是因为 $3 \cdot 5 \equiv 15 \equiv 2 \cdot 7 + 1 \equiv 1 \mod 7$。因此,根据模算术引理,我们可以将同余式两边乘以 $3^{-1}$,从而得到
    \begin{align*}
        \forall x \in \mathbb{Z} \centerdot 3x \equiv 11 \mod 7 &\iff 3^{-1} \cdot 3 \cdot x \equiv 3^{1}
\cdot 11 \mod 7 \\
        &\iff 1 \cdot x \equiv 5 \cdot 4 \mod 7 \\
        &\iff x \equiv 20 \equiv 6 \mod 7
    \end{align*}
    由于我们知道 $3^{-1}$ 表示这个同余式的所有解(即它代表 $3$ 模 $7$ 乘法逆元的等价类),那么我们可以推导出
    \[\forall x \in \mathbb{Z} \centerdot 3x \equiv 11 \mod 7 \iff x \equiv 6 \mod 7 \iff \exists k \in \mathbb{Z} \centerdot x = 7k + 6\]
    这表征了给定方程解中所有可能的 $x \in \mathbb{Z}$ 的值。

    现在,我们用这个方法来确定解中对应的 $y \in \mathbb{Z}$ 的值。假设 $k \in \mathbb{Z}$,且 $x = 7k + 6$,然后我们代入 $x$ 发现
    \[y = \frac{3x-11}{7} = \frac{3(7k + 6)-11}{7} = \frac{21k+7}{7} = 3k+1\]

    现在,我们找到了表示给定方程所有可能解的形式。我们知道,任意 $k \in \mathbb{Z}$ 都会对应一个 $x$,从而对应一个 $y$。此外,由于我们的推导使用了 $\iff$ 陈述,我们可以确定这涵盖了所有的解。

    我们可以将给定方程的解集 $S$ 描述为
    \[S = \{(x, y) \in \mathbb{Z} \times \mathbb{Z} \mid \exists k \in \mathbb{Z} \centerdot (x, y) = (7k + 6, 3k + 1)\}\]
\end{example}

\begin{tcolorbox}[colback=gray!10,
    colframe=black,
    width=\textwidth,
    arc=2mm, auto outer arc,
    title={有趣的事实},breakable,enhanced jigsaw,
    before upper={\parindent15pt\noindent},	]
    在这个例子中,我们解决了一个\textbf{线性丢番图方程},并找出了它的所有解。所谓\emph{线性},指的是变量 $x$ 和 $y$ 是一次的,没有平方或立方项。

    通过我们在这个例子中使用的技术,你可以解决\emph{任意}线性丢番图方程,或者轻松判别它是否有解。事实上,我们还将\emph{证明}一个关于这种方程何时没有解的结论(见贝祖恒等式,定理 \ref{theorem6.5.31})。只要方程有解,这种方法就适用。
    
    在下一个例子中,我们将研究\textbf{二次丢番图方程},其中变量会有平方项(包括 $x^2$ 和 $y^2$)。之后我们将讨论解决这类方程的可能性。
\end{tcolorbox}

\begin{example}
    现在让我们再来看一个例子,这个例子和前一个例子的过程类似(使用乘法逆元简化运算),但还引入了二次残差的概念。

    \textbf{声明}:无整数 $x, y \in \mathbb{Z}$ 满足方程
    \[3x^2-5y^2=1\]

    给定 $x, y \in \mathbb{Z}$。我们要证明 $3x^2-5y^2=1$ 是\emph{不可能的}。

    我们先将给定方程重写为
    \[3x^2 = 5y^2+1\]
    具体来说,这意味着
    \[3x^2 \equiv 1 \mod 5\]
    因为 $5y^2 \equiv 0 \mod 5$。注意到 $3^{-1} \equiv 2 \mod 5$,因为 $3 \cdot 2 = 6 = 5 + 1$。因此,我们可以将等式两边都乘以 $3^{-1}$,从而简化得:
    \[3x^2 \equiv 1 \mod 5 \iff 3^{-1} \cdot 3x^2 \equiv 3^{-1} \cdot 1 \mod 5 \iff x^2 \equiv 2 \mod 5\]
    然而,回顾示例 \ref{ex:example6.5.15},在那里我们研究了\emph{二次残差}。我们发现,模 $5$ 的二次残差集合为 $\{0, 1, 4\}$。也就是说,\emph{不可能}有整数 $x$ 满足 $x^2 \equiv 2 \mod 5$。这就表明,给定方程没有整数解。
\end{example}

\begin{tcolorbox}[colback=gray!10,
    colframe=black,
    width=\textwidth,
    arc=2mm, auto outer arc,
    title={有趣的事实},breakable,enhanced jigsaw,
    before upper={\parindent15pt\noindent},	]
    我们之前提到,我们确切知道何时线性丢番图方程可解,并且知道如何解这些方程。但对于\textbf{二次丢番图方程},我们就没有这么幸运了。要判断一个二次丢番图方程是否有解是非常困难的。即使知道它有解,实际求解也非常复杂。

    事实上,对于这些二次丢番图方程,我们的运气非常糟糕。已知\textbf{没有任何计算机算法可以输入带有一次和二次幂变量的丢番图方程,并判断该方程是否有解}。这一事实甚至不涉及如何解方程,仅仅是判断它是否有解。令人惊讶的是,这个事实是\href{https://en.wikipedia.org/wiki/Hilbert's_tenth_problem}{希尔伯特第十问题}的一种形式。

    请放心,我们在这里提供的例子和练习中的丢番图方程都可以用我们提供的技术进行分析。我们提到的这个事实是针对所有此类方程的一般性声明。
\end{tcolorbox}

\subsubsection*{一点群论知识}

在这一小节中,我们想强调当前主题背后蕴含的一些重要而深刻的数学原理。由于篇幅和时间所限,我们无法全面探讨这些内容。因此,我们将在此简要介绍一些概念和事实,并通过例子来加以说明。

我们想传达的主要思想是,当我们考虑 $\mathbb{Z}$ 模 $p$ 时(其中 $p$ 为\textbf{质数}),会出现一些特殊的现象。在这种情况下,每个小于 $p$ 的数都与 $p$ \emph{互质},因为 $p$ 只有 $1$ 这个因子。这意味着在 $\{1, 2, \dots, p-1\}$ 中的所有数在模 $p$ 下都有乘法逆元。这非常方便,因为除了 $[0]_{\mod p}$ 外,每个同余类都有一个对应的乘法逆元类。

例如,考虑 $p=5$。注意到
\begin{align*}
    1^{-1} \equiv 1 \mod 5\\
    2^{-1} \equiv 3 \mod 5\\
    3^{-1} \equiv 2 \mod 5\\
    4^{-1} \equiv 4 \mod 5
\end{align*}

再比如,考虑 $p=7$。注意到
\begin{align*}
    1^{-1} \equiv 1 \mod 7\\
    2^{-1} \equiv 4 \mod 7\\
    3^{-1} \equiv 5 \mod 7\\
    4^{-1} \equiv 2 \mod 7\\
    5^{-1} \equiv 3 \mod 7\\
    6^{-1} \equiv 6 \mod 7
\end{align*}

请注意,这意味着集合中的所有元素都有一个乘法逆元。

(同时,请注意这些逆元其实是数字 $1$ 到 $p - 1$ 的一种\emph{排列}。这并非巧合!试着证明为什么会这样!试着证明有两个元素是它们自己的逆元,即 $1^{-1} \equiv 1 \mod p$ 和 $(p - 1)^{-1} \equiv p - 1 \mod p$,而其他元素都\emph{不可能}是它们自己的逆元。)

当我们考虑 $\mathbb{Z}$ 模 $n$ 时,如果 $n$ 是合数,情况就不一样了。这种情况下,我们知道 $n$ 可以因式分解;假设 $n = ab$,其中 $a,b \in \mathbb{N}-\{1\}$。那么 $1 < a < n$,但 $a$ 和 $n$ 不互质(它们有公因子 $a$),所以 $a$ 在模 $n$ 下没有乘法逆元。实际上,所有 $n$ 的因数(及其倍数)在模 $n$ 下都没有乘法逆元。

例如,考虑 $p=6$。
\begin{align*}
    1^{-1} \equiv 1 \mod 6\\
    2^{-1} \;\text{不存在} \mod 6\\
    3^{-1} \;\text{不存在} \mod 6\\
    4^{-1} \;\text{不存在} \mod 6\\
    5^{-1} \equiv 5 \mod 6
\end{align*}

由于这种区别,$\mathbb{Z}$ 模 $p$ 的数学``结构''显得格外突出。它具备一些优良的性质,并在某种意义上表现得非常好。虽然这些描述可能比较模糊,但主要思想是:所有元素都有逆元,这使得 $\mathbb{Z}$ 模 $p$ 很特别。事实上,$\mathbb{Z}$ 模 $p$ 构成一种称为\textbf{群}的数学结构。

一般来说,从启发式的角度看,群是一个可以进行``乘法''运算的对象集合,这种乘法运算满足
\begin{enumerate}[label=(\alph*)]
    \item 交换律
    \item 结合律
    \item 所有元素都有逆元
\end{enumerate}
我们已经知道,标准的整数乘法(即使在 $\mathbb{Z}$ 模 $n$ 中,对于任意 $n$)满足交换律和结合律,并且在 $\mathbb{Z}$ 模 $p$ 中(对于质数 $p$)每个元素都有逆元。

如果你对这些概念感兴趣,可以在本章末尾找到一些练习,帮助你理解这些性质。此外,你也可以查阅一些\textbf{抽象代数}、\textbf{近世代数}或\textbf{群论}的入门教材。这些领域中有许多强大而深刻的数学思想,\textbf{群}在许多领域中都有重要的应用!

\subsection{一些有用的定理}

在本节中,我们将探讨一些数论中的定理,这些定理涉及模运算,并且它们本身既有用又有趣。我们将陈述并证明这些定理(有时需要你通过练习给与帮助),然后用例子来展示它们的实际应用。

\subsubsection*{中国剩余定理}

为了引出这个定理,我们先通过一个故事来说明它的用处:

\begin{quote}
    孙武\footnote{在中国,这个故事的主角是韩信。因此这个问题也被称为``韩信点兵''问题。--- 译者注}将军的部队里有许多士兵,战斗结束后,他想快速统计剩余士兵的数量。一个个数显然太费劲了,所以他想用更高效的方法完成点兵。幸运的是,这些士兵训练有素,可以轻松组成等大小的队列。

    孙武将军先命令士兵们排成两排等长的队伍,发现多出一个士兵。

    接着,他又让士兵们排成三个等大小的环形队列,但还是多出一个士兵。

    最后,他命令士兵们排成五个等大小的侧翼队列,这次多出两个士兵。

    此时,他觉得信息已经足够。战斗结束后,他推测这支部队的总人数在 $250$ 到 $300$ 之间。根据这些信息,他能\emph{确切}知道有多少士兵。
    
    你能算出士兵的具体数量吗?这支部队究竟有多少士兵呢?
\end{quote}

请你先试着解决这个问题,看看你能否找到答案。然后再继续阅读我们的解决方案、一个定理陈述以及解决此类问题的方法介绍。

请再次阅读这个故事。设孙武将军军队的士兵数量为 $x$,那么这个故事告诉我们,$x$ 必须满足以下三个同余条件和一个不等式:
\begin{align*}
    x \equiv 1 \mod 2 \\
    x \equiv 1 \mod 3 \\
    x \equiv 2 \mod 5 \\
    250 \le x \le 300
\end{align*}
(你能从故事中看出这些条件的来源吗?)

现在有两个问题需要考虑:
\begin{enumerate}[label=(\arabic*)]
    \item 是否\emph{一定}存在一个 $x$ 满足所有三个同余条件?
    \item 是否存在\emph{多个}满足条件的 $x$ 值?我们能否保证其中一个 $x$ 也满足不等式?
\end{enumerate}

下面陈述的\textbf{中国剩余定理}可以保证:
\begin{enumerate}[label=(\arabic*)]
    \item 同余方程有无穷多个解;
    \item 至少有一个解满足给定的不等式。
\end{enumerate}
不过,在我们陈述并证明这个定理之前,让我们先尝试解决这个初始问题。我们将其分解为几个观察和步骤:
\begin{itemize}
    \item 第一个同余条件要求 $x$ 必须是\textbf{奇数},这样就排除了所有偶数作为潜在解。以下是潜在解列表:
    \[1,\cancel{2}, 3, \cancel{4}, 5, \cancel{6}, 7, \cancel{8}, 9,\cancel{10}, 11,\cancel{12}, 13,\cancel{14}, 15,\cancel{16}, 17,\cancel{18}, 19,\cancel{20}, 21,\cancel{22}, 23, \dots\]
    \item 第二个同余条件要求解必须是 $3$ 的倍数加 $1$,这就排除了模 $3$ 余 $0$ 或 $2$ 的数。以下是潜在解列表:
    \[1,\cancel{2}, \cancel{3}, \cancel{4}, \cancel{5}, \cancel{6}, 7, \cancel{8}, \cancel{9},\cancel{10}, \cancel{11},\cancel{12}, 13,\cancel{14}, \cancel{15},\cancel{16}, \cancel{17},\cancel{18}, 19,\cancel{20}, \cancel{21},\cancel{22}, \cancel{23}, \dots\]
    \item 第三个同余条件要求解必须是 $5$ 的倍数加 $2$,这就排除了模 $5$ 余 $0, 1, 3, 4$ 的数。以下是潜在解列表:
    \[\cancel{1},\cancel{2}, \cancel{3}, \cancel{4}, \cancel{5}, \cancel{6}, \circled{7}, \cancel{8}, \cancel{9},\cancel{10}, \cancel{11},\cancel{12}, \cancel{13},\cancel{14}, \cancel{15},\cancel{16}, \cancel{17},\cancel{18}, \cancel{19},\cancel{20}, \cancel{21},\cancel{22}, \cancel{23}, \dots\]
\end{itemize}
看起来 $7$ 是唯一的解,但我们怎么确定没有其他解呢?我们只检查了前 $23$ 个可能的解……能\emph{确保}没有其他解吗?这个问题就交给你来探究了。试试更大的数字,看看能不能找到其他解。你能猜出其中的规律吗?$7$ 真的是唯一解吗?

现在,我们用更巧妙的方法来解决这些同余问题。具体来说,假设我们有一个解 $x$,它满足所有三个同余式,看看我们能否推导出更多的信息。通过这个推导,我们将揭示所有\emph{可能}解的一个特性。

根据同余的定义,我们知道存在 $k, \ell, m \in \mathbb{Z}$ 使得
\begin{align*}
    x &= 2k + 1 \\
    x &= 3\ell + 1 \\
    x &= 5m + 2 
\end{align*}
给定这样的 $k, \ell, m$。

我们先来看前两个方程,试着将它们合并成一个关于 $x$ 的方程。具体来说,把第一个方程乘以 $3$,第二个方程乘以 $2$,这样就会分别得到 $6k$ 和 $6\ell$ 项。然后,通过相减,我们可以适当地进行因式分解。也就是说,我们首先找到
\begin{align*}
    3x &= 6k + 3 \\
    2x &= 6\ell + 2
\end{align*}
然后
\[(3x - 2x) = (6k + 3) - (6\ell + 2) \implies x = 6(k - \ell) + 1\]
因为 $k, \ell \in \mathbb{Z}$ 已经给定,我们可以定义 $u = k-\ell$,所以 $u \in \mathbb{Z}$。注意这告诉我们此时 $x = 6u+1$,换句话说
\[x \equiv 1 \mod 6\]
现在,我们通过结合前两个同余式得到了这个新的同余式,这并非巧合,因为这个同余式是模 $6$ 的,而 $6 = 2 \times 3$。稍后,当我们引导你证明接下来的定理时,你会明白其中的原理!

接下来,我们尝试将这个新的同余式与上面的第三个同余式结合。我们采用类似的方法:将刚刚推导出的同余式乘以 $5$,将第三个同余式乘以 $6$,这样相减后可以提出一个 $30$ 的因子。(这也解释了为什么新推导出的同余式是模 $30$ 的。)我们得到
\begin{align*}
    5x &= 30u + 5 \\
    6x &= 30m + 12
\end{align*}
然后
\[(6x - 5x) = (30m + 12) - (30u + 5) \implies x = 30(m - u) + 7\]
同理,因为 $u,m$ 已经给定,我们可以定义 $v = m-u$,所以 $v \in \mathbb{Z}$。这告诉我们此时 $x = 30v+7$,换句话说
\[x \equiv 7 \mod 30\]
这个最终的同余式是通过将给定的每个同余式相互结合推导出来的,因此它包含了这三个同余式的所有信息。我们断言,这个同余式现在包含了\textbf{所有的}解!

首先,这个新推导出的同余告诉我们,任何解必须模 $30$ 余 $7$。换句话说,任何除以 $30$ 余数不是 $7$ 的数都不可能是解。本质上,这把我们在上述三种观察中排除潜在解的工作总结成一个陈述。

其次,我们可以解释,事实上,任何模 $30$ 余 $7$ 的数确实是一个解。让我们看看原因。设 $n \in \mathbb{Z}$,并定义 $y = 30n + 7$(即我们选择任意 $y \in \mathbb{Z}$,满足 $y \equiv 7 \mod 30$)。注意 $y$ 满足
\begin{itemize}
    \item 第一个同余式,因为 $y = 30n + 7 = 2(15n + 3) + 1$,所以 $y \equiv 1 \mod 2$。
    \item 第二个同余式,因为 $y = 30n + 7 = 3(10n + 2) + 1$,所以 $y \equiv 1 \mod 3$。
    \item 第三个同余式,因为 $y = 30n + 7 = 5( 6n + 1) + 2$,所以 $y \equiv 2 \mod 6$。
\end{itemize}
至此,我们已经知道:
\begin{enumerate}[label=(\arabic*)]
    \item \emph{任何}解 $x$ 必须满足 $x \equiv 7 \mod 30$;
    \item 任何满足这个条件的 $x$ 实际上\emph{就是}一个解。
\end{enumerate}
这两个陈述共同形成了一个 $\iff$陈述,即
\[x \;\text{是三个同余式的解} \iff x \equiv 7 \mod 30\]
因此\textbf{所有解}的集合 $S$ 为
\[S = \{x \in \mathbb{Z} \mid x \equiv 7 \mod 30\} = \{30n + 7 \mid n \in \mathbb{Z}\}\]

回到最初的问题,我们现在只需要考虑给定的不等式。是否存在一个满足 $x \equiv 7 \mod 30$ 且 $250 \le x \le 300$ 的数 $x$ 呢?是的,确实存在!我们可以从 $7$ 开始,每次加上 $30$ 的倍数,或者从接近 $300$ 的数开始调整,总之用类似的方法就能找到它。无论你怎么做,你会发现 $\mathbf{x = 277}$ 就是我们一直在寻找的解。这就是孙武将军军队里有士兵的数量。

现在,为了进行比较,考虑以下可能来自类似问题的同余方程组:
\begin{align*}
    x &\equiv 3 \mod 4 \\
    x &\equiv 2 \mod 6
\end{align*}
这个同余方程组有解吗?我们之前使用的方法在这里适用吗?如果你尝试使用``划掉不合适的候选者''或``合并同余式''的方法,会发现都\emph{不起作用}。回头看看这个方程组,你会发现这很合理。第一个同余要求 $x$ 比 $4$ 的倍数多 $3$;由于 $4$ 的倍数是偶数,这意味着我们要求 $x$ 是\emph{奇数}。然而,第二个同余要求 $x$ 比 $6$ 的倍数多 $2$;由于 $6$ 的倍数也是偶数,这意味着我们要求 $x$ 是\emph{偶数}。一个解怎么可能同时即是奇数又是偶数呢?!这显然是不可能的。

\textbf{中国剩余定理}告诉我们在什么情况下同余方程组一定有解。它适用于我们之前解决的第一个问题,并且实际上告诉了我们最终的结果:有无穷多个解,并且它们都同余于 $30$。然而,它并没有告诉我们刚刚解决的第二个问题无解。这个定理对某些情况提供了\emph{保证}。当我们遇到这些情况时,可以对解作出有效的判断。然而,当我们遇到\emph{不同}情况时,这个定理并\emph{不能保证}有解。现在让我们看看这个定理的陈述,然后进一步讨论一下,并请你帮助证明它(用两种不同的方法!)。

\begin{theorem}
    假设我们有一个由 $r$ 个不同同余式组成的方程组。具体来说,假设 $r \in \mathbb{N}$,并且我们有 $r$ 个自然数 $n_1, n_2, \dots, n_r$,以及 $r$ 个整数 $a_1, a_2, \dots, a_r$。这个同余方程组可以表示为:
    \begin{align*}
        x &\equiv a_1 \mod n_1 \\
        x &\equiv a_2 \mod n_2 \\
        \vdots \\
        x &\equiv a_r \mod n_r
    \end{align*}
    (换句话说,该方程组要求 $x \in \mathbb{Z}$ 满足 $\forall i \in [r] \centerdot x \equiv a_i \mod n_i$。)

    \dotuline{如果}模数 $n_i$ 是两两互质的,也就是说,任意两个 $n_i$ 之间除了 $1$ 以外没有其他公因数,\dotuline{那么}这个同余方程组一定有解。

    此外,在这种情况下,实际上有无穷多个解,并且这些解模 $N$ 同余。这里的 $N$ 定义为所有模数的乘积:
    \[N = \prod_{i \in [r]}^{} n_i\]
\end{theorem}

请注意,主要结论是``\textbf{如果……那么……}''形式的陈述。还记得我们之前提到的关于这种条件陈述的内容吗?这个定理并没有说明当两个模数不互质时会发生什么。在这种情况下,可能会发生任何事情!我们之前看到的例子中,模数不互质:一个同余式是模 $4$,另一个是模 $6$,而 $4$ 和 $6$ 有一个公因数 $2$。然而,定理并没有说这种情况下无解;我们需要自己去找出答案。如果我们稍微改变一下数字,并提出以下同余关系:
\begin{align*}
    x \equiv 3 \mod 4 \\
    x \equiv 5 \mod 6
\end{align*}
这个方程组就有解。你可以试着解一下。

中国剩余定理的一种证明方法类似于我们之前解决问题的步骤。对于方程组中任意数量的同余关系,我们可以通过逐步将一个同余合并到另一个同余中,最终得到一个模数为所有其他模数乘积的同余关系。那么,如何证明这种方法的有效性呢?这是一个迭代过程……归纳法正好发挥用武之地!确实可以通过对 $r$(即方程组中同余关系的数量)进行归纳来证明中国剩余定理。这种证明在练习 \ref{ex:exercises6.7.26} 中有详细介绍。我们喜欢这种证明,因为它还为解决这类问题提供了一个实际操作的方法。

另一种证明是\textbf{构造性的}。也就是说,它利用定理中的信息,通过组合这些信息来定义一个解 $X$(并且证明了这一点)。这种证明在练习 \ref{exc:exercises6.7.27} 中有详细介绍。我们喜欢这种证明,因为它确实是构造性的;它不是通过论证某个对象存在的原因来证明,而是实际构造出了这个对象。然而,这种方法构造的解并不是通过``排除不合适的候选者''或``合并同余式''找到的相同解。这实际上是一种有点``非自然''的方法,但它确实在不需要进行任何归纳过程的情况下也能有效。为了比较这两种方法,我们鼓励你尝试完成这两个定理的证明。然而,如果我们只能推荐一种方法,我们会建议使用归纳法证明。

\subsubsection*{贝祖恒等式}

这个定理让我们回想起之前关于线性丢番图方程的讨论。在示例 \ref{ex:example6.5.26} 中,我们通过巧妙地应用乘法逆元解决了一个特定的方程。除了展示这种方法外,还有一种简单的方法来验证这种方程是否有解。这个定理准确地描述了二元线性丢番图方程何时有解,称为\textbf{贝祖恒等式},以 $18$ 世纪法国数学家 Étienne Bézout 命名。

在陈述这个定理之前,我们需要先提供一个定义。你可能已经熟悉这个定义了,但它在这个定理中起着至关重要的作用。因此,我们在这里给出这个定义,并提供一些示例来说明。

\begin{definition}[最大公约数]
    给定 $a,b \in \mathbb{Z}$。$a$ 和 $b$ 的\dotuline{最大公约数}用 $\gcd(a, b)$ 表示,定义为能够同时整除 $a$ 和 $b$ 的最大整数。也就是说,
    \[\gcd(a, b) \mid a \land \gcd(a, b) \mid b\]
    且
    \[\forall d \in \mathbb{Z} \centerdot (d \mid a \land d \mid b) \implies d \le \gcd(a, b)\]
\end{definition}

我们假设你对这个概念有一定的了解,或者至少有一些直觉。即将介绍的定理及其证明并不需要你对这一概念有深入的理解。此外,任何涉及这一定义或定理的练习都不会要求你具备很强的计算能力,也不会假设你对这个概念有深入了解。相反,请将此视为你在继续练习吸收数学概念新定义时的一部分,帮助你运用这些抽象概念来证明进一步的事实,并找到例子和反例。这是一项重要的技能!在陈述和证明定理之前,我们先快速浏览几个此概念的实际例子。\\

\begin{example}
    在某些情况下,我们会取两个数字并求出它们的最大公约数。通常,找到最大公约数的合理方法是先找到两个数字的\textbf{质因数分解},然后适当地进行组合。也就是说,$\gcd(a, b)$ 是 $a$ 和 $b$ 共有的质因数的乘积,因此通过考虑这些共同的质因数,我们可以很容易地求出最大公约数。

    在某些情况下,我们会做出一些关于最大公约数的一般性论断,并进行证明(或者可能要求你来证明它!)。这些论断仅依赖于我们上面提供的定义。

    \begin{itemize}
        \item 设 $a=15, b=6$。因为 $a = 3 \cdot 5, b = 2 \cdot 3$,我们发现它们的公因子只有 $3$。因此
            \[\gcd(6, 15) = 3\]
        \item 设 $a=30, b=40$。因为 $a = 2 \cdot 3 \cdot 5, b = 2^3 \cdot 5$,我们发现它们的公因子有 $2$ 和 $5$。因此
            \[\gcd(30, 40) = 10\]
        \item 一般来说,
            \[\gcd(a, b) = \gcd(b, a)\]
            这显然是正确的,因为 $a$ 和 $b$ 的任何公约数也是 $b$ 和 $a$ 的公约数。
        \item 设 $a=77, b=72$。因为 $a = 7 \cdot 11, b = 2^3 \cdot 3^2$,我们发现它们没有公因子。因此
            \[\gcd(77, 72) = 1\]
        \item 设 $a=13$,设 $b \in \mathbb{N}$ 且 $a \nmid b$。因为 $a$ 为质数,且 $b$ 不是 $13$ 的倍数,因此 $b$ 的质因子中没有 $13$。因此
            \[\gcd(13, b) = 1\]
            这意味着 $a$ 和 $b$ \textbf{互质}。这是一个通用事实:
            \[a \;\text{和}\; b \;\text{互质} \iff \gcd(a, b) = 1\]
            此外
            \[\forall a, b \in \mathbb{N} \centerdot a \;\text{为质数} \implies \big(\gcd(a,b)=1 \iff a \nmid b\big)\]
    \end{itemize}
\end{example}

现在,我们觉得已经准备好陈述和证明\textbf{贝祖恒等式}了!

\begin{theorem}[贝祖恒等式]\label{theorem6.5.31}
   给定 $a, b \in \mathbb{Z}$。定义 $L$ 为 $a$ 和 $b$ 的所有线性组合的集合;换句话说,定义
   \[L = \{z \in \mathbb{Z} \mid \exists x, y \in \mathbb{Z} \centerdot ax + by = z\} = \{ax + by \mid x, y \in \mathbb{Z}\}\]
   定义 $M$ 为所有 $\gcd(a, b)$ 的倍数的集合;换句话说,定义
   \[M = \{z \in \mathbb{Z} \mid \exists k \in \mathbb{Z} \centerdot z = k \cdot \gcd(a, b)\} = \{k \cdot \gcd(a, b) \mid k \in \mathbb{Z}\}\]
   则
   \[L = M\]
   换句话说,线性丢番图方程 $ax + by = c$ 有解当且仅当 $c$ 是 $\gcd(a, b)$ 的倍数。
\end{theorem}

这一定理非常实用,它告诉我们线性丢番图方程 $ax + by = c$(给定 $a, b, c \in \mathbb{Z}$)何时有解。只需找到 $\gcd(a, b)$,并确保 $\gcd(a, b) \mid c$。

为了证明这一定理,我们需要证明两个\emph{集合相等}。我们将使用一种叫做\emph{双重包含论证}的方法,这是我们之前多次使用过的策略。我们这里只会证明其中一个包含,另一个包含则留给你作为练习。

\begin{proof}
    给定 $a, b \in \mathbb{Z}$。按照定理的陈述定义集合 $L$ 和 $M$。

    首先,我们证明 $L \subseteq M$。设 $z$ 是 $L$ 中任意固定元素。

    根据 $L$ 的定义,我们知道 $\exists x, y \in \mathbb{Z} \centerdot ax + by = z$。给定这样的 $x$ 和 $y$。

    由于 $\gcd(a, b)$ 能整除 $a$ 和 $b$,我们知道$\exists k, \ell$ 使得 $a = k \cdot \gcd(a, b)$ 且 $b = \ell  \cdot \gcd(a, b)$。给定这样的 $k$ 和 $\ell$。

    我们将 $a$ 和 $b$ 的表达式代入上面的方程:
    \[z = ax + by = k \cdot \gcd(a, b) \cdot x + \ell \cdot \gcd(a, b) \cdot y = \gcd(a, b) \cdot \underbrace{(kx + \ell y)}_{m}\]
    定义 $m = kx + \ell y$。由于 $m \in \mathbb{Z}$,这表明 $z$ 是 $\gcd(a, b)$ 的倍数。

    因此,$z \in M$。这证明了 $L \subseteq M$。\\

    接着,我们证明 $M \subseteq L$……

    留作练习 \ref{exc:exercises6.7.12}。
\end{proof}

通过这一结果的证明,我们可以确定二元线性丢番图方程是否有解。接下来的几个练习将要求你判断这种方程是否有解,只需引用这个结果即可。如果还需要你找到所有解,请使用我们在示例 \ref{ex:example6.5.26} 中展示的方法。

\textbf{挑战性问题}:你认为关于多于两个变量的线性丢番图方程,有哪些可以讨论的内容?比如,考虑
\[6x + 8y + 15z = 10\]
这个方程有解吗?如果有,那有多少个解呢?再比如,考虑
\[3x + 6y + 9z = 2\]
这个方程有解吗?为什么有或为什么没有?

试着陈述并证明一个关于这个问题的结论。你能将这个结论推广到任意数量的变量吗?

\subsection{习题}

\subsubsection*{温故知新}

以口头或书面的形式简要回答以下问题。这些问题全都基于你刚刚阅读的内容,所以如果忘记了具体的定义、概念或示例,可以回去重读相关部分。确保在继续学习之前能够自信地回答这些问题,这将有助于你的理解和记忆!

\begin{enumerate}[label=(\arabic*)]
    \item 为什么在考虑 $\mathbb{Z}$ 模 $n$ 时会将 $\mathbb{Z}$ 划分成几个集合?
    \item $\mathbb{Z}$ 模 $n$ 的等价类是什么?
    \item 如何判断两个整数 $x, y \in \mathbb{Z}$ 是否属于 $\mathbb{Z}$ 模 $n$ 的同一个等价类?
    \item 模算术引理是什么?它为什么如此有用?我们如何利用它以代数的方式处理同余?
    \item 什么是\emph{乘法逆元}?给定 $a \in \mathbb{Z}$ 和 $n \in \mathbb{N}$,在 $\mathbb{Z}$ 模 $n$ 的情况下,如何判定 $a$ 的乘法逆元是否存在?
    \item 当 $p$ 为\emph{质数}时,$\mathbb{Z}$ 模 $p$ 的等价类集合有什么特别之处?
    \item 中国剩余定理是否保证以下同余系统有解?为什么?
    \begin{align*}
        x &\equiv 2 \mod 6 \\
        x &\equiv 5 \mod 9
    \end{align*} 
    你能找到这个同余方程组的解吗?(\textbf{提示}:答案是肯定的!)
\end{enumerate}

\subsubsection*{小试牛刀}

尝试回答以下问题。这些题目要求你实际动笔写下答案,或(对朋友/同学)口头陈述答案。目的是帮助你练习使用新的概念、定义和符号。题目都比较简单,确保能够解决这些问题将对你大有帮助!

\begin{enumerate}[label=(\arabic*)]
    \item 请陈述并证明判断自然数 $x \in \mathbb{N}$ 是否为 $9$ 的倍数的技巧。\\
    (\textbf{提示}:参见示例 \ref{ex:example6.5.13} 中的类似问题。)
    \item 设 $n \in \mathbb{N}, a \in \mathbb{Z}$。证明 $ (n-a)^2 \equiv a^2 \mod n$。
    \item 设 $n \in \mathbb{N} - \{1\}$。证明 $ (n-1)^{-1} \equiv n - 1 \mod n$。
    \item 对于每一对值 $(a, n)$,请找出 $a$ 模 $n$ 的乘法逆元,或者说明它不存在。
    \begin{enumerate}[label=(\alph*)]
        \item $a = 5$ 且 $n = 12$
        \item $a = 7$ 且 $n = 11$
        \item $a = 6$ 且 $n = 27$
        \item $a = 11$ 且 $n = 18$
        \item $a = 70$ 且 $n = 84$
        \item $a = 8$ 且 $n = 17$
    \end{enumerate}
    \item 描述下面方程所有整数解 $x, y \in \mathbb{Z}$ 的特征。
        \[4x - 7y = 18\]
    \item 确定以下同余方程组的所有解:
        \begin{align*}
            x \equiv 3 \mod 5 \\
            x \equiv 4 \mod 7
        \end{align*}
\end{enumerate}