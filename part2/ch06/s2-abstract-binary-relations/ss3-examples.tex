% !TeX root = ../../../book.tex

\subsection{示例}

请再次尝试找出一些符合或不符合我们刚刚定义的四个性质的关系。下面展示一些定义在 $\mathbb{N}$ 上的典型例子,以提供具体参考。你也可以添加一些简单例子,比如定义在 $\mathbb{Z}$ 或 $\mathbb{R}$ 上的关系。\\

\begin{example}
    本例中,所有关系均定义在集合 $\mathbb{N}$ 上。
    \begin{itemize}
        \item 定义 $\mathbb{N}$ 上的关系 $R_1$
        \[(x, y) \in R_1 \iff x \text{\ 整除\ } y\]
        (即 $y$ 能被 $x$ 整除,或 $\exists k \in \mathbb{N}$ 使得 $y=kx$。该定义的严格陈述见定义 \ref{def:definition6.2.15}。)

        则 $R_1$ 具有自反性,因为 $x \mid x$,即 $x=1 \cdot x$。

        \textbf{整除关系具有自反性。}
        \item 定义 $\mathbb{N}$ 上的关系 $R_2$
        \[(x, y) \in R_2 \iff x \text{\ 和\ } y \text{\ 具有相同的奇偶性}\]
        则 $R_2$ 具有对称性,因为若 $x$ 和 $y$ 具有相同的奇偶性,则 $y$ 和 $x$ 也具有相同的奇偶性。

        \textbf{``相同奇偶性''关系具有对称性。}
        \item 定义 $\mathbb{N}$ 上的关系 $R_3$
        \[(x, y) \in R_3 \iff x < y\]
        则 $R_3$ 具有传递性,因为若 $x<y$ 且 $y<z$ 则 $x<y<z$,所以 $x<z$。

        \textbf{``$<$''关系具有传递性。}
        \item 定义 $\mathbb{N}$ 上的关系 $R_4$
        \[(x, y) \in R_4 \iff x \le y\]
        则 $R_4$ 具有反对称性,因为若 $x \le y$ 且 $y \le x$ 则 $x \le y \le x$,所以 $x=y$。

        \textbf{``$\le$''关系具有反对称性。}
    \end{itemize}
\end{example}

\begin{example}
    始终牢记:关系本质是一组有序对,不必通过性质定义。考虑以下例子:\\
    定义集合 $S=\{a,b,c\}$ 上的关系 $R$
    \[R = \{(a, a),(a, c),(b, c),(c, b)\}\]
    该关系
    \begin{itemize}
        \item 不具有自反性:\quad 因为 $(c,c) \notin R$
        \item 不具有对称性:\quad 因为 $(a, c) \in R$ 但 $(c, a) \notin R$
        \item 不具有传递性:\quad 因为 $(a, c) \in R$ 且 $(c, b) \in R$ 但 $(a, b) \notin R$
        \item 不具有反对称性:因为 $(b, c) \in R$ 且 $(c, b) \in R$ 但 $b \ne c$
    \end{itemize}
\end{example}

\begin{example}
    让我们练习使用略微不同的关系符号。注意:$x \;R\; y$ 等价于 $(x, y) \in R$。\\
    在班级同学集合 $S$ 上定义关系 $\bigstar$,对于任意 $x,y \in S$
    \[x \;\bigstar\; y \iff x \text{\ 和\ } y \text{\ 同一个月出生}\]
    我们声称该关系具有自反性、对称性、传递性。原因如下:
    \begin{itemize}
        \item 该关系具有\emph{自反性}:因为每个人当然与其自己同月出生(即 $x \;\bigstar\; x$)。
        \item 该关系具有\emph{对称性}:因为若 $x$ 与 $y$ 同月出生(即 $x \;\bigstar\; y$),则 $y$ 与 $x$(只是顺序不同!)当然也是同月出生(即 $y \;\bigstar\; x$)。
        \item 该关系具有\emph{传递性}:因为若 $x$ 与 $y$ 同月出生,且 $y$ 与 $z$ 同月出生,则 $x$ 与 $z$ 同月出生。我们只是在不断强调``相同''这个词的概念!
    \end{itemize}
    关于\emph{反对称性},这要看具体情况!班上是否有两个人同月出生?如果有,这种关系就不具有反对称性。然而,如果班上每个人都在不同月份出生,那么这种关系就具有反对称性,因为没有人会与别人有此关系,除了他自己!好好琢磨一下……
\end{example}
