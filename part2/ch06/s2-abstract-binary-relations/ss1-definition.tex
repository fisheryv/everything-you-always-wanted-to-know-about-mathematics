% !TeX root = ../../../book.tex

\subsection{定义}

让我们直入主题,开始讨论关系。我们会先给出定义,然后提供一系列例子。

\begin{definition}
    设 $A, B$ 为集合。$A$ 和 $B$ 之间的\dotuline{关系}是由\dotuline{有序对}组成的集合,$R \subseteq A \times B$。对于元素 $a \in A$ 和 $b \in B$,当且仅当 $(a, b) \in R$ 时,我们说 $a$ 和 $b$ 是\dotuline{相关的}。

    集合 $A$ 称为\dotuline{定义域},集合 $B$ 称为\dotuline{值域}。集合 $R$ 称为\dotuline{关系集}。
    
    如果 $A = B$,我们称 $R$ 为 $A$ 上的关系。
\end{definition}

另一种常见的写法是用 $x \;R\; y$ 来表示 $(x,y) \in R$。当我们通过这种方式定义关系时,会坚持使用 $(x, y) \in R$ 来表示其基础集合结构。之后,我们有时会用一些符号来定义关系,比如 $x < y$ 或 $x \;\bigstar\; y$ 等等。

\begin{remark}
    我们这里定义的关系有时也叫\emph{二元关系},因为它涉及两个``输入'';集合 $R$ 是由\emph{有序对}组成的。

    我们可以将这个概念推广到\emph{三元关系}。也就是说,给定集合 $A,B,C$,我们可以定义 $R \subseteq A \times B \times C$ 为三元关系,当且仅当 $(a, b, c) \in R$ 时,$a,b,c$ 是相关的。我们还可以进一步推广到具有 $n$ 个``输入''的关系。不过,在本书中,我们只考虑\emph{二元关系},因此``\emph{关系}''一词将专指\emph{二元关系}。
\end{remark}

\begin{remark}
    关系 $R$ 通常通过确定 $A$ 和 $B$ 元素的某个\emph{性质}来定义(用变量命题 $P(a,b)$ 表示),并设定
    \[(a,b) \in R \iff P(a,b)\]
\end{remark}

\subsubsection*{示例}

\begin{example}
    设 $W=\{\text{英文单词}\}, L=\{\text{英文字母}\}$,定义关系 $R$ 如下:
    \[(w, \ell) \in R \iff w \text{以 } \ell \text{ 开头}\]
    则 $(\text{mathematics},\text{m}) \in R$ 且 $(\text{golf},\text{g}) \in R$,因为这些都是有效的单词,我们已经确定了它们的首字母。再来看一些反例,比如 $(\text{knowledge},\text{n}) \notin R$ 且 $(\text{you},\text{u}) \notin R$。此外 $(\text{zyzyxyqy},\text{z}) \notin R$,因为 $\text{zyzyxyqy} \notin W$。
\end{example}

$A = B$ 的情况经常出现,所以 $R$ 定义了同一集合中元素对之间的关系。下一个例子就是针对这种情况的。\\

\begin{example} \label{ex:example6.2.5}
    设 $A=B=\mathbb{Z}$,定义 $\mathbb{Z}$ 上的关系 $R$ 如下:
    \[(x, y) \in R \iff x \text{ 和 } y \text{ 具有相同的奇偶性}\]
    则 $(2,8) \in R, (-3, 7) \in R, (-99, -99) \in R$,但 $(1,2) \notin R, (0, -3) \notin R$,且 $(\pi, 0) \notin R$(因为 $\pi \notin \mathbb{Z}$)。
\end{example}

\begin{example} \label{ex:example6.2.6}
    定义 $\mathbb{R}$ 上的关系 $L$ 如下:
    \[(x, y) \in L \iff x < y\]
    则 $(-1, \pi) \in L$ 且 $(0, 100) \in L$,但 $(2, 2) \notin L$ 且 $(\pi, -1) \notin L$。
\end{example}

请注意,这里的对都是\emph{有序对}(我们可能会忘记这一点,因为 $A = B = \mathbb{R}$),所以元素的顺序很重要。确实,知道 $(x, y) \in L$ 并不一定意味着 $(y, x) \in L$。在上面这个例子中,这种情况实际上总是错误的!

回想一下,我们有时用 $x \;L\; y$ 来表示 $(x, y) \in L$,所以我们可以说 $-1 \;L\; \pi$ 但 $\pi \;\cancel{L}\; -1$,并且 $2 \;\cancel{L}\; 2$。

\subsubsection*{空关系}

\begin{remark}
    到目前为止,我们看到的例子在某种程度上都是有趣的关系。对于任意 $x,y \in R$,我们可以通过比较来判断 $x$ 是否小于 $y$。换句话说,我们看到的每个例子都是通过这样的方式定义的:对某性质 $P(a, b), (a, b) \in R \iff P(a, b)$ 为真。
\end{remark}

不过,关系不一定要这样定义。举个例子,我们知道对于任何集合 $S$,都有 $\varnothing \subseteq S$。因此,给定两个集合,我们总是可以通过 $\varnothing \subseteq A \times B$ 这一事实来定义一个\emph{平凡关系}。也就是说,\emph{平凡关系}是指没有任何元素相关的关系。这虽然看起来``无趣'',但它仍然符合关系的定义,所以我们也接受这种关系。

\subsubsection*{任何有序对的集合都是一个关系}

\begin{remark}
    给定集合 $A$ 和 $B$,任何子集 $R \subseteq A \times B$ 都定义了一个关系。然而,要找到一个能够描述这种关系的性质可能会非常困难,甚至是不可能的。

    比如 $A=\{1,3,5\}$ 且 $B=\{\bigstar, \heartsuit\}$,则我们可以定义 $A, B$ 上的关系
    \[R=\{(1,\bigstar), (5, \heartsuit)\}\]
    为什么 $1$ 与 $\bigstar$ 有关系?为什么 $3$ 不与任何元素有关系?这谁也说不清楚。这只是一个有序对的集合!从数学角度讲,这完全没有问题。
\end{remark}

\subsubsection*{相等关系}

\begin{example} \label{ex:example6.2.9}
    另一种在任意集合 $X$ 上定义关系的方法是定义相等关系。也就是说,如果 $(x, y) \in R \iff x = y$。需要注意的是,这种定义与集合 $X$ 的具体内容无关,只要它是一个\emph{集合}即可。
\end{example}

\subsubsection*{关系之间的相似之处}

\begin{example}
    假设 $S$ 是你班上学生的集合。定义 $S$ 和 $\mathbb{N}$ 之间的关系 $R_1$,如果 $(s, n) \in R_1$,那么表示学生 $s \in S$ 的年龄是 $n$ 岁。写出这个关系集的一些元素。

    现在,在集合 $S$ 内定义关系 $R_2$,如果 $(s, t) \in R_2$,那么表示学生 $s$ 和 $t$ 的年龄相同(以年为单位)。写出这个关系集的一些元素。

    比较关系 $R_1$ 和 $R_2$,它们是否以某种方式传达了关于集合 $S$ 的相同信息?为什么是或为什么不是?是否可以通过 $R_1$ 确定 $R_2$?反过来是否也可以?仔细思考这些问题,并尝试总结你的想法。我们马上将在下一小节中讨论这些问题,但现在先花点时间自己思考一下吧!
\end{example}

\subsubsection*{关系``编码''信息}

前面的例子旨在说明抽象关系的实际用途,并解释我们为什么要讨论它们(除了我们想要严格定义函数这一目标之外)。从某种意义上说,关系是一种``保存''两个集合或一个集合中元素信息的方法,是比较两个元素并判断它们是否满足某性质的一种手段。而在更广泛的意义上,关系可以提供关于集合元素在特定性质下表征的信息。

例如,在前面的例子中,关系 $R_1$ 告诉了我们更多关于集合 $S$ 元素的信息。确切地说,$R_1$ 告诉我们哪些人年龄相同:我们可以找到像 $(s, n)$ 和 $(t, n)$ 这样的对,它们的第二个值相同。同时,$R_1$ 还告诉我们具体年龄是多少:只需要查看这些对的第二个值即可。而 $R_2$ 则不行。知道 $(s, t) \in R_2$ 只告诉我们学生 $s$ 和学生 $t$ 年龄相同,却没有具体的年龄信息!从这个角度讲,$R_1$ 是一个``更好''的关系,因为它提供了更多的信息。

不过,$R_2$ 也有其优点!例如,它有一个很好的性质:如果 $(x, y) \in R_2$,那么 $(y, x) \in R_2$ 也必然成立。这个性质在 $R_1$ 中显然不成立,因为当 $(s, n) \in R_1$ 时,说 $(n, s) \in R_1$ 是没有意义的,因为顺序不匹配定义域和值域!那么这个性质是否使 $R_2$ 成为一个``更好''的关系呢?嗯,这要看具体情况以及我们想要编码和检索的信息类型。在某些情况下,你可能会选择使用 $R_1$,而在其他情况下,你可能会选择使用 $R_2$。

不过我们这里有点超前了!我们还不能详细解释这些性质的含义及其优缺点。然而,总的来说,我们对这些性质及其在给定集合的所有元素对中何时(或何时不)成立感兴趣。在下一小节中,我们将定义和探索几种常见的抽象关系性质。虽然不能保证任何关系都具备这些性质,但它们在数学和实际应用中已被证明是有趣和有用的。之后,我们将看到更多关系的例子,并讨论如何证明这些性质成立。在这个过程中,我们将培养处理关系的直觉,甚至弄清楚我们首先要证明的那些声明的类型!