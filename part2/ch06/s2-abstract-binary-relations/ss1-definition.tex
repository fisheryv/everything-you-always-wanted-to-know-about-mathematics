% !TeX root = ../../../book.tex

\subsection{定义}

让我们直入主题,讨论关系的概念。首先给出定义,随后提供一系列示例。

\begin{definition}
    设 $A, B$ 为集合。$A$ 和 $B$ 之间的\dotuline{关系}是由\dotuline{有序对}构成的集合 $R \subseteq A \times B$。对于元素 $a \in A$ 和 $b \in B$,当且仅当 $(a, b) \in R$ 时,称 $a$ 与 $b$ \dotuline{相关}。

    集合 $A$ 称为\dotuline{定义域},集合 $B$ 称为\dotuline{值域}。集合 $R$ 称为\dotuline{关系集}。
    
    若 $A = B$,则称 $R$ 为 $A$ 上的关系。
\end{definition}

另一种常见记法是用 $x \;R\; y$ 表示 $(x,y) \in R$。后续讨论中,我们将沿用 $(x, y) \in R$ 强调其集合本质,但也会使用诸如 $x < y$ 或 $x \;\bigstar\; y$ 等符号定义关系。

\begin{remark}
    此处定义的关系又称\emph{二元关系},因其涉及两个``输入'',且 $R$ 由\emph{有序对}组成。

    此概念可推广至\emph{三元关系}。即给定集合 $A,B,C$,定义 $R \subseteq A \times B \times C$,当且仅当 $(a, b, c) \in R$ 时 $a,b,c$ 相关。进一步可推广至 $n$ 元关系。本书仅讨论\emph{二元关系},因此``\emph{关系}''一词特指\emph{二元关系}。
\end{remark}

\begin{remark}
    关系 $R$ 通常通过描述 $A$ 与 $B$ 元素的某个\emph{性质}(以命题 $P(a,b)$ 表示)来定义,并设定
    \[(a,b) \in R \iff P(a,b)\]
\end{remark}

\subsubsection*{示例}

\begin{example}
    设 $W=\{\text{英文单词}\}, L=\{\text{英文字母}\}$,定义关系 $R$ 如下:
    \[(w, \ell) \in R \iff w \text{以\ } \ell \text{\ 开头}\]
    则 $(\text{mathematics},\text{m}) \in R$ 且 $(\text{golf},\text{g}) \in R$,因为这些都是有效的单词,且它们的首字母符合要求。再来看一些反例,比如 $(\text{knowledge},\text{n}) \notin R$ 且 $(\text{you},\text{u}) \notin R$。此外 $(\text{zyzyxyqy},\text{z}) \notin R$,因为 $\text{zyzyxyqy} \notin W$。
\end{example}

$A = B$ 的情况经常出现,此时 $R$ 定义了同一集合中元素对之间的关系。下一个例子就是此类情形。

\begin{example} \label{ex:example6.2.5}
    设 $A=B=\mathbb{Z}$,定义 $\mathbb{Z}$ 上的关系 $R$ 如下:
    \[(x, y) \in R \iff x \text{\ 和\ } y \text{\ 具有相同的奇偶性}\]
    则 $(2,8) \in R, (-3, 7) \in R, (-99, -99) \in R$,但 $(1,2) \notin R, (0, -3) \notin R$(因为奇偶性不同),且 $(\pi, 0) \notin R$(因为 $\pi \notin \mathbb{Z}$)。
\end{example}

\begin{example} \label{ex:example6.2.6}
    定义 $\mathbb{R}$ 上的关系 $L$ 如下:
    \[(x, y) \in L \iff x < y\]
    则 $(-1, \pi) \in L$ 且 $(0, 100) \in L$,但 $(2, 2) \notin L$ 且 $(\pi, -1) \notin L$。
\end{example}

请注意,这里的元素对均为\emph{有序对}(由于 $A = B = \mathbb{R}$,我们容易忽略这一点),因此元素的顺序至关重要。例如 $(x, y) \in L$ 并不一定意味着 $(y, x) \in L$,在上面这个例子中,这种情况永远不成立!

若以 $x \;L\; y$ 表示 $(x, y) \in L$,则可表述为 $-1 \;L\; \pi$ 但 $\pi \;\cancel{L}\; -1$,且 $2 \;\cancel{L}\; 2$。

\subsubsection*{空关系}

\begin{remark}
    到目前为止,我们看到的例子在某种程度上都是有意义的关系。对于任意 $x,y \in R$,我们可以通过比较来判断 $x$ 是否小于 $y$。换言之,前述示例均通过特定性质 $P(a,b)$ 定义:$(a, b) \in R$ 当且仅当 $P(a,b)$ 成立。
\end{remark}

然而,关系未必显式定义。举个例子,我们知道对于任意集合 $S$,都有 $\varnothing \subseteq S$。因此,给定两个集合,我们总是可以通过 $\varnothing \subseteq A \times B$ 这一事实来定义一个\emph{平凡关系}。也就是说,\emph{平凡关系}是指没有任何元素相关的关系。这虽然看起来``无趣'',但它仍然符合关系的定义,所以我们也接受这种关系。

\subsubsection*{任意有序对的集合都是一个关系}

\begin{remark}
    给定集合 $A$ 和 $B$,任意子集 $R \subseteq A \times B$ 都定义了一个关系。然而,要找到描述这种关系的具体性质可能非常困难,甚至是不可能的。

    例如 $A=\{1,3,5\}$ 且 $B=\{\bigstar, \heartsuit\}$,则我们可以定义 $A, B$ 上的关系
    \[R=\{(1,\bigstar), (5, \heartsuit)\}\]
    为什么 $1$ 与 $\bigstar$ 相关?为什么 $3$ 不与任何元素相关?这无法解释清楚。它只是一个有序对的集合!从数学角度看,这完全成立。
\end{remark}

\subsubsection*{相等关系}

\begin{example} \label{ex:example6.2.9}
    另一种在任意集合 $X$ 上定义关系的方法是定义相等关系,即 $(x, y) \in R \iff x = y$。需要注意的是,这种定义与集合 $X$ 的具体元素无关,只要它是一个\emph{集合}即可。
\end{example}

\subsubsection*{关系之间的相似之处}

\begin{example}
    假设 $S$ 是你班上同学的集合。定义 $S$ 与 $\mathbb{N}$ 的关系 $R_1$:若 $(s, n) \in R_1$,则表示学生 $s \in S$ 的年龄为 $n$ 岁。写出该关系的一些元素。

    再定义 $S$ 上的关系 $R_2$:若 $(s, t) \in R_2$,则表示学生 $s$ 和 $t$ 年龄相同(以年计)。写出该关系的一些元素。

    比较 $R_1$ 和 $R_2$,它们是否以某种方式传达了关于 $S$ 的相同信息?为什么?能否通过 $R_1$ 确定 $R_2$?反之是否成立?仔细思考这些问题并尝试总结观点。下一小节将讨论这些内容,但请先独立思考!
\end{example}

\subsubsection*{关系``编码''信息}

前面的例子旨在说明抽象关系的实际用途(除严格定义函数的目标外)。某种意义上,关系是一种``保存''两个集合或单个集合中元素信息的方式,可用于比较元素是否满足特定性质。广义上,关系能提供元素在特定性质下的表征信息。

例如,在前面的例子中,$R_1$ 揭示了更多关于 $S$ 的信息。确切地说,$R_1$ 直接说明哪些人年龄相同:通过找到第二个值相同的对,例如 $(s, n)$ 和 $(t, n)$。与此同时,$R_1$ 还给出了具体年龄:只需查看有序对的第二个值。而 $R_2$ 无法提供具体年龄:$(s, t) \in R_2$ 仅表明 $s$ 和 $t$ 年龄相同,但无年龄数值!因此 $R_1$ 是``更好''的关系,因为它提供了更丰富的信息。

然而 $R_2$ 也有其优势!它具有一个良好的性质:若 $(x, y) \in R_2$,则 $(y, x) \in R_2$ 必然成立。此性质在 $R_1$ 中不成立:$(s, n) \in R_1$ 时,$(n, s) \in R_1$ 无意义,因为顺序与定义域和值域不匹配。这是否使 $R_2$ ``更好''?取决于具体场景和需要编码的信息类型。某些情况下可能选 $R_1$,其他情况下可能选 $R_2$。


此处的讨论有点超前!我们还不能详细解释这些性质的含义及优劣。总体而言,我们关注这些性质在集合所有元素对中何时成立(或不成立)。下一小节将定义并探讨几种常见的抽象关系性质。虽然不能保证任意关系具备这些性质,但它们在数学和实际应用中已被证明是有趣且有用的。后续将展示更多关系实例,讨论如何证明性质成立,并培养处理关系的直觉,进而厘清需要首先证明的声明类型!