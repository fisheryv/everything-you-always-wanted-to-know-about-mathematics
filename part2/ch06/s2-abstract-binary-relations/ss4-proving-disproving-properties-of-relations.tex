% !TeX root = ../../../book.tex

\subsection{证明/证伪关系的性质}

当我们面对一个集合及其上的关系时,我们会立即想知道这些关系是否具有某些性质。通过尝试集合中的一些特定元素,我们可以猜测该关系是否满足某个性质,然后尝试证明或证伪它。这有时相当于``猜测和检验'',但最终,要证明一个性质成立,我们必须证明一个形式为``对于所有……都成立……''的命题(请回顾 \ref{sec:section4.9} 节的证明技巧!)。因此,证明关系性质相当于取一个任意元素(或多个元素)并讨论它们之间的关系。要证伪这样的命题,我们会证明它的逻辑否定形式,即``存在……使得……''(请再次回顾我们的证明技巧!)。因此,证伪一个性质相当于找到一个\emph{反例}。让我们看几个证明或证伪关系性质的例子。在练习中还有更多这种风格的证明例子。

\subsubsection*{$\mathbb{Z}$ 上的``整除''关系}

我们将首先介绍(或者提醒你)一个定义,因为它是我们其中一个例子的基础。这是一个关于一个整数整除另一个整数的正式定义。

\begin{definition}\label{def:definition6.2.15}
    设 $a,b \in \mathbb{Z}$,我们称 \dotuline{$x$ 整除 $y$},写做 \dotuline{$x \mid y$},当且仅当
    \[\exists k \in \mathbb{Z} \centerdot y = kx\]
\end{definition}

\begin{example}\label{ex:example6.2.16}
    定义 $\mathbb{Z}$ 上的关系 $R$
    \[(x, y) \in R \iff x \mid y\]
    让我们判断 $R$ 是否满足关系的四个性质,然后证明或证伪我们所有的主张!

    一般来说,根据所讨论的集合和关系,你可能会通过直觉或者直接``看出''某个性质是否成立。如果是这样,那太好了!如果不是(这是更常见的情况),我们建议发起一个``证明'',假设某个性质成立,并看看否能证明出来。如果你做到了,那么你就证明了这个性质!如果你在某处遇到困难,可能是因为这个性质不成立,而你在证明中遇到困难的地方会给你一些找到反例的启示。这个策略并不总是有效(也许你在证明中遇到困难是因为它实际上很有挑战性),但它可能非常有帮助,所以请记住这一点。在这个例子中我们也会看到这一策略的实际应用。

    另一个策略---事实上更简单的一个策略---就是大声说出或用文字写下所讨论的关系和性质。有时候,仅仅让自己用简单的语言说出某些东西,而不是阅读页面上的抽象符号,会让你的大脑意识到一些有用的信息!我们也会在这里看到这一策略的实际应用。

    \begin{itemize}
        \item 让我们探讨一下 $R$ 是否具有\textbf{自反性}。这具体意味着什么呢?我们不妨大声说出来:任意整数都能被它自己整除。这是肯定的!现在,让我们尝试用证明所需的符号来表达这一点。
        \begin{proof}
            我们声称 $R$ 具有自反性。设 $x \in \mathbb{Z}$ 为任意固定整数。由于 $x = 1 \cdot x$ 且 $1 \in \mathbb{Z}$,所以 $x \mid x$。因此,$(x, x) \in R$。由此可见,$R$ 具有自反性。
        \end{proof}
        你看!通过说出或写下我们的思考,我们意识到了一个事实,这使得我们更容易用数学语言来表达这个陈述。
        \item 让我们探讨一下 $R$ 是否具有\textbf{对称性}。这个性质是通过一个\emph{蕴涵}术语(\emph{条件陈述})来定义的。假设我们有一个任意关联对 $(x, y) \in R$,我们能否必然得到 $(y, x) \in R$ 呢?换句话说:
        \[\text{假设} x \text{能整除} y, \text{我们能否说} y \text{也能整除} x \text{?} \]
        这看起来不太可能!$x \mid y$ 意味着 $y = kx$,其中 $k \in \mathbb{Z}$,但这并不意味着 $x = \frac{1}{k}y$ 表示 $y$ 也能整除 $x$。万一 $\frac{1}{k} \notin \mathbb{Z}$ 怎么办?

        你可能会说:``$\frac{1}{k}$ 只有在 $k = 1$ 或 $k = -1$ 时才是整数,所以就是这样。''但这并不是完整的解释!要反驳一个``对于所有……''的命题,我们需要尽可能提供一个明确的反例。我们不需要全面描述该性质在所有情况下是否成立,只需要一个例子来证明这个性质不成立。这比含糊其辞地说某处可能存在反例要更直接明了。让我们展示一个反例给读者,然后再继续!
        \begin{proof}
            考虑 $2,6 \in \mathbb{Z}$,因为 $6=3 \cdot 2$,所以我们有 $(2,6) \in R$。

            然而,使 $2 = \ell \cdot 6$ 成立需要 $\ell = \frac{1}{3}$,而 $\frac{1}{3} \notin \mathbb{Z}$,因此 $(6,2) \notin R$。

            这证明了 $R$ 不具有\emph{对称性}。
        \end{proof}
        \item 让我们探讨一下 $R$ 是否具有\textbf{传递性}。传递性通常是最难理解的性质之一。这主要是因为它是由包含两个假设的条件陈述定义的,并且涉及到三个变量。

        在这个具体例子中,我们假设 $x \mid y$ 且 $y \mid z$,然后考虑是否必然有 $x \mid z$。试着大声读出来,看看你认为这是否成立。
        
        看起来是成立的,对吧?现在,试着用数学语言写下你的假设和结论。你能看出如何将它们结合起来吗?在继续阅读之前,试着写出你自己的证明。
        \begin{proof}
            设 $x, y, z \in \mathbb{Z}$ 是任意固定的。假设 $(x, y) \in R$ 且 $(y, z) \in R$。这意味着 $x \mid y$ 且 $y \mid z$。所以 $\exists k, \ell \in \mathbb{Z}$ 使得 $y = kx$ 且 $z = \ell y$。给定这样的 $k, \ell$,将第一个等式代入第二个,可得
            \[z = \ell y = \ell (kx) = (k \ell)x\]
            因为 $k \ell \in \mathbb{Z}$,我们证明了 $x \mid z$,所以 $(x, z) \in R$。

            因此,$R$ 具有传递性。
        \end{proof}
        \item 让我们探讨一下 $R$ 是否具有\textbf{反对称性}。这一性质也由包含两个假设的条件陈述定义,因此我们假设有 $x$ 和 $y$,满足 $x \mid y$ 且 $y \mid x$。我们能得出 $x = y$ 吗?这个问题让我们回想起之前证明 $R$ 不具有对称性的过程。记住,我们已经证明了 $x \mid y$ 并不一定意味着 $y \mid x$。实际上,如果稍加思考,就会发现 $x \mid y$ 和 $y \mid x$ 同时为真是不可能的。这究竟是怎么回事呢?请仔细思考,在阅读我们的证明之前尝试自己给出一个证明。
        \begin{proof}
            设 $x, y \in \mathbb{Z}$ 是任意固定的。假设 $(x, y) \in R$ 且 $(y,x) \in R$。

            这意味着 $x \mid y$ 且 $y \mid x$,所以 $\exists k, \ell \in \mathbb{Z}$ 使得 $y = kx$ 且 $x = \ell y$。给定这样的 $k, \ell$,将第一个等式代入第二个,可得
            \[y = kx = k(\ell y) = (k \ell)y\]
            存在如下两种情况:

            \textbf{情况 1}:假设 $y=0$。此时我们无法两边同时除以 $y$。我们反而知道 $x = \ell y = \ell \cdot 0 = 0$,因为 $x=0$,所以该情况下 $x=y$。

            \textbf{情况 2}:假设 $y \ne 0$,此时两边同时除以 $y$ 得 $k \ell = 1$。因为 $k, \ell \in \mathbb{Z}$ 这意味着要么 $k = \ell = 1$ 要么 $k = \ell = -1$。

            如果 $k = \ell = 1$,则 $x = \ell y = y$。

            如果 $k = \ell = -1$, 则 $x = \ell y = -y$。

            因此 $R$ 不具有反对称性。
        \end{proof}
        哦,糟糕!你明白发生了什么吗?在``大多数''情况下,我们确实得出了 $x = y$ 的结论,但实际上还有可能 $y = -x$。比如,当 $y = 3$ 而 $x = -3$ 时,显然 $x \mid y$ 且 $y \mid x$ 但 $x \ne y$。这就是我们需要的反例,尝试完成我们的``证明''。或许你早已预见到这一情况,如果是这样,那真是太棒了!最后,我们通过展示这个反例来完成证明:
        \begin{proof}
            考虑 $x=3, y=-3$, 显然 $x, y \in \mathbb{Z}$。因为 $3 = (-1)(-3)$ 且 $-3 = (-1) \cdot 3$,并且 $1, -1 \in \mathbb{Z}$,所以 $x \mid y$ 且 $y \mid x$。

            然而,显然 $x \ne y$。这证明了 $R$ 不具有反对称性。
        \end{proof}
    \end{itemize}
\end{example}

作为后续问题,思考一下如果我们在集合 $\mathbb{N}$ 上定义这个关系,而不是在 $\mathbb{Z}$ 上,会有什么变化?哪些性质会成立?答案是否会与在 $\mathbb{Z}$ 上有所不同?请仔细思考这些问题,因为此处的答案将引导我们进入下一个小节。

\subsubsection*{构建具有特定性质的关系}

在继续之前,我们再来看一个例子。这个有趣的``游戏''是从集合中选取元素,并构造出满足特定性质的关系 $R$。(注意:$4$ 个性质的组合有 $16$ 种不同的可能。)在习题中,我们会问你类似的问题,所以让我们通过一个例子来演示一下。\\

\begin{example}\label{ex:example6.2.17}
    \textbf{目标}:设 $S$ 为班上同学的集合。定义一个关系 $R$ 
    \begin{enumerate}[label=(\arabic*)]
        \item 不具有自反性
        \item 不具有对称性
        \item 具有传递性
        \item 具有反对称性
    \end{enumerate}

    为了确保 $R$ 不具有自反性,我们必须确保没有任何元素与其自身相关。为了确保 $R$ 不具有对称性,我们必须确保当 $(x, y) \in R$ 时,$(y, x) \notin R$。为了确保 $R$ 具有传递性,我们必须确保当 $(x, y) \in R$ 且 $(y, z) \in R$ 时,($x, z) \in R$。为了确保 $R$ 具有反对称性,我们需要考虑性质定义的逆否命题,这要求任意一对元素最多只能以一种方式相关。最后一个性质可能是最难理解的;它表示对于每个 $x, y \in S$,要么 $x$ 与 $y$ 相关但 $y$ 与 $x$ 不相关,要么 $y$ 与 $x$ 相关但 $x$ 与 $y$ 不相关,或者 $x$ 和 $y$ 根本不相关。也就是说,我们不允许任何 $(x, y) \in R$ 和 $(y, x) \in R$ \emph{同时}成立。(请再次阅读反对称性的定义,并写下其条件陈述的逆否命题,思考为什么这样做有效。)

    现在让我们尝试构造一个满足这些性质的 $R$。性质 (1) 表明我们的定义不能包含``或等于''的形式,而性质 (2) 则要求定义必须以``唯一的方式''关联任意的 $x$ 和 $y$。因此,我们可以猜测,一个类似于 $\mathbb{Z}$ 集合上``小于''关系的\emph{比较}性质可能会奏效。让我们尝试一下,并验证这些性质是否成立。

    我们定义 $S$ 上的关系 $R$
    \[x \;R\; y \iff x \;\text{的年龄(岁)严格小于}\; y\]
    现在,让我们来探讨一下这个关系的性质,并确保它们是符合我们的预期的。在阅读我们的解决方案之前,尝试自己证明或证伪这些性质。另外,尝试在 $S$ 上定义不同的关系(自己创造一个!),看看这些性质有何不同。你能想出另一个具有相同性质的关系吗?

    \begin{itemize}
        \item $R$ \textbf{不具有自反性}。因为任何人 $x \in S$ 都跟他/她自己同岁,因此 $x \;\cancel{R}\; x$。
        \item $R$ \textbf{不具有对称性}。因为如果 $x$ 的年龄严格小于 $y$,则 $y$ 的年龄就严格大于 $x$,因此 $y \;\cancel{R}\; x$。
        \item $R$ \textbf{具有传递性}。因为如果 $x$ 的年龄严格小于 $y$,且 $y$ 的年龄就严格小于 $z$,那么 $x$ 的年龄当然严格小于 $z$。
        \item $R$ \textbf{具有反对称性}。因为对于任意两人 $x, y \in S$,要么其中一人的年龄小于另一人,要么两人同岁,不可能两人同时小于对方的年龄。(本质上,我们通过确保该性质定义中的条件陈述的假设永远不成立来保证反对称性,因此条件陈述本身总是成立。)
    \end{itemize}

    因此该关系 $R$ 满足所有要求的性质。
\end{example}

你可能注意到,我们的论证并不严谨,但这是有原因的。具体来说,我们没有为那些不成立的性质提供明确的反例。如果我们能够找到班上的两个学生,并证明一个比另一个年轻,但反过来却不是这样,那就更好了。但我们并不知道你班上都有谁!这就是为什么我们的论证是``解释某事物存在而不明确指出它''。

我们要指出的是,一般来说,这种形式的关系
\[(x, y) \in R \iff x \;\text{在某种意义上``小于''}\; y\]
通常不具有自反性和对称性,但具有传递性和反对称性。实际上,我们甚至可以将``比……小''替换为``比……大'',这个结论仍然成立。要理解为什么会这样,可以想想在 $\mathbb{N}, \mathbb{Z}$ 或 $\mathbb{R}$ 上的 ``$<$'' 关系,或者这些集合上的 ``$>$'' 关系。再想想人群中的``比……年轻''关系、``比……高''关系,或者``有更多孩子''关系。$\mathbb{Z}$ 上的 ``$\le$'' 关系又如何呢?这与 ``$<$'' 关系有何不同?哪些性质发生了变化?

(这些类型的问题将在下一小节中进一步探讨,我们将研究一种行为类似于 ``$\le$'' 和 ``$\ge$'' 关系的特定类型的关系。它们被称为\textbf{顺序关系}。)
