% !TeX root = ../../../book.tex

\subsection{证明/证伪关系的性质}

当我们研究一个集合及其上的关系时,自然想要探究这些关系是否具有特定性质。通过选取集合中的特定元素进行测试,我们可以初步判断关系是否满足某个性质,进而尝试证明或证伪它。这一过程类似于``猜测和检验'',但最终要严格证明性质成立,必须证明一个形如``对于所有……都成立……''的全称命题(可回顾 \ref{sec:section4.9} 节的证明技巧)。因此,证明关系性质时,必须任意选取元素(或元素组)并分析其关系。要证伪性质,则需证明其否定命题,即``存在……使得……''(请回顾相应的证明技巧)。此时,证伪等价于找到一个\emph{反例}。下面将通过实例展示证明或证伪关系性质的方法,更多同类证明示例可见后续练习。

\subsubsection*{$\mathbb{Z}$ 上的``整除''关系}

首先明确整除关系的定义(或作必要回顾),因为它是后续示例的基础:

\begin{definition}\label{def:definition6.2.15}
    设 $a,b \in \mathbb{Z}$,称 \dotuline{$x$ 整除 $y$},记作 \dotuline{$x \mid y$},当且仅当
    \[\exists k \in \mathbb{Z} \centerdot y = kx\]
\end{definition}

\begin{example}\label{ex:example6.2.16}
    定义 $\mathbb{Z}$ 上的关系 $R$
    \[(x, y) \in R \iff x \mid y\]

    我们将系统判断 $R$ 是否满足关系的四个基本性质,并对每个结论给出严格证明或证伪。

    一般来说,根据具体集合与关系,可通过直觉初步判断性质是否成立。若直觉清晰,则事半功倍;若不确定(更常见的情形),可尝试先假设性质成立并着手证明。若证明成功,则性质得证;若证明受阻,则暗示性质可能不成立——此时证明中的难点往往提示了构造反例的方向。此策略虽然不是绝对有效(证明困难也可能源于复杂性),但极具实践价值。下文将实际应用这一策略。

    另一种思路是尝试用自然语言描述关系与性质。将抽象的符号表达转化为通俗的语句陈述,常能激发新的理解。我们同样会在后续分析中运用这一方法。

    \begin{itemize}
        \item 考虑 $R$ 的\textbf{自反性}。其含义是:任意整数均可被自身整除。这显然成立!现在用数学语言表述如下:
        \begin{proof}
            我们声称 $R$ 具有自反性。设 $x \in \mathbb{Z}$ 为任意固定整数。由于 $x = 1 \cdot x$ 且 $1 \in \mathbb{Z}$,所以 $x \mid x$。因此,$(x, x) \in R$。由此可见,$R$ 具有自反性。
        \end{proof}
        可见,厘清思路能帮助数学表达更严谨。\\

        \item 考虑 $R$ 的\textbf{对称性}。该性质由\emph{条件陈述}定义:若 $(x, y) \in R$,我们能否必然得到 $(y, x) \in R$ 呢?换句话说:
        \[\text{假设\ } x \text{\ 能整除} y, \text{我们能否说\ } y \text{\ 也能整除\ } x \text{?} \]
        这看起来不太可能!$x \mid y$ 意味着 $y = kx$,其中 $k \in \mathbb{Z}$,但这并不意味着 $x = \frac{1}{k}y$ 表示 $y$ 也能整除 $x$。万一 $\frac{1}{k} \notin \mathbb{Z}$ 怎么办?

        你可能会说:``$\frac{1}{k}$ 只有在 $k = 1$ 或 $k = -1$ 时才是整数,所以对称性不成立。''但这并不是完整的解释!要反驳一个``对于所有……''形式的命题,我们需要尽可能提供一个明确的反例。我们不需要全面描述该性质在所有情况下是否成立,只需一个反例来证明该性质不成立就够了。这比含糊其辞地说某处可能存在反例来得更加直接明了。下面明确给出一个反例:
        \begin{proof}
            考虑 $2,6 \in \mathbb{Z}$,因为 $6=3 \cdot 2$,所以我们有 $(2,6) \in R$。

            然而,使 $2 = \ell \cdot 6$ 成立的 $\ell = \frac{1}{3}$,而 $\frac{1}{3} \notin \mathbb{Z}$,因此 $(6,2) \notin R$。

            这证明了 $R$ 不具有\emph{对称性}。
        \end{proof}

        \item 考虑 $R$ 的\textbf{传递性}。传递性通常是最难理解的性质之一,这主要是因为它涉及两个条件和三个变量。

        在这个具体例子中,我们假设 $x \mid y$ 且 $y \mid z$,然后考虑是否必然有 $x \mid z$。
        
        通过朗读和观察发现传递性似乎是成立的。现在,用数学语言写下假设和结论,思考如何将它们结合起来?在继续阅读之前,试着写出你自己的证明。
        \begin{proof}
            设 $x, y, z \in \mathbb{Z}$ 为任意固定整数。假设 $(x, y) \in R$ 且 $(y, z) \in R$。这意味着 $x \mid y$ 且 $y \mid z$。所以 $\exists k, \ell \in \mathbb{Z}$ 使得 $y = kx$ 且 $z = \ell y$。给定这样的 $k, \ell$,将第一个等式代入第二个等式,可得
            \[z = \ell y = \ell (kx) = (k \ell)x\]
            因为 $k \ell \in \mathbb{Z}$,我们证明了 $x \mid z$,故 $(x, z) \in R$。

            因此,$R$ 具有传递性。
        \end{proof}

        \item 考虑 $R$ 的\textbf{反对称性}:这一性质同样涉及两个条件陈述,因此我们假设存在 $x$ 和 $y$ 满足 $x \mid y$ 且 $y \mid x$。此时能得出 $x = y$ 吗?这个问题让我们回想起之前证明 $R$ 不具有对称性的过程。记住,我们已经证明了 $x \mid y$ 并不一定意味着 $y \mid x$。实际上,如果稍加思考,就会发现 $x \mid y$ 和 $y \mid x$ 同时为成立是不可能的。请仔细思考,在阅读我们的证明之前尝试自己给出一个证明。
        \begin{proof}
            设 $x, y \in \mathbb{Z}$ 为任意固定整数。假设 $(x, y) \in R$ 且 $(y,x) \in R$。

            这意味着 $x \mid y$ 且 $y \mid x$,所以 $\exists k, \ell \in \mathbb{Z}$ 使得 $y = kx$ 且 $x = \ell y$。给定这样的 $k, \ell$,将第一个等式代入第二个等式,可得
            \[y = kx = k(\ell y) = (k \ell)y\]
            存在如下两种情况:
            \begin{itemize}
                \item \textbf{情况 1}:假设 $y=0$。此时 $x = \ell y = \ell \cdot 0 = 0$,因为 $x=0$,所以该情况下 $x=y$。
                \item \textbf{情况 2}:假设 $y \ne 0$,此时两边同时除以 $y$ 得 $k \ell = 1$。因为 $k, \ell \in \mathbb{Z}$ 这意味着要么 $k = \ell = 1$ 要么 $k = \ell = -1$。
                \begin{itemize}
                    \item 若 $k = \ell = 1$,\enspace 则 $x = \ell y = y$。
                    \item 若 $k = \ell = -1$,则 $x = \ell y = -y$。
                \end{itemize}
            \end{itemize}
            
            因此 $R$ 不具有反对称性。
        \end{proof}

        回顾上面的证明,在``大多数''情况下,我们确实得出了 $x = y$ 的结论,但关键在于存在 $x = -y$ 的情况。比如,当 $x = 3$ 而 $y = -3$ 时,显然 $x \mid y$ 且 $y \mid x$ 但 $x \ne y$。这就构造出了反例。或许你早已预见到这一情况,如果是这样,那真是太棒了!最后,我们通过展示这个反例来完成证明:
        \begin{proof}
            考虑 $x=3, y=-3$, 显然 $x, y \in \mathbb{Z}$。因为 $3 = (-1)(-3)$ 且 $-3 = (-1) \cdot 3$,并且 $1, -1 \in \mathbb{Z}$,所以 $x \mid y$ 且 $y \mid x$。

            然而,显然 $x \ne y$。这证明了 $R$ 不具有反对称性。
        \end{proof}
    \end{itemize}
\end{example}

作为延伸问题,请思考如果我们在自然数集 $\mathbb{N}$ 而非整数集 $\mathbb{Z}$ 上定义这个关系,会有什么变化?哪些性质成立?答案是否不同?请仔细思考这些问题,这些问题的答案将引导我们进入下一个小节。

\subsubsection*{构建具有特定性质的关系}

在继续之前,我们再来看另一个例子。这个有趣的练习是从集合中选取元素,并构造满足特定性质的关系 $R$。(注意:$4$ 个性质的组合共有 $16$ 种可能。)在习题中,我们将请你回答类似的问题,因此让我们通过一个例子给出演示。\\

\begin{example}\label{ex:example6.2.17}
    \textbf{目标}:设 $S$ 为班上同学的集合。定义一个关系 $R$,使其满足:
    \begin{enumerate}[label=(\arabic*)]
        \item 不具有自反性
        \item 不具有对称性
        \item 具有传递性
        \item 具有反对称性
    \end{enumerate}
    \begin{itemize}
        \item 为了确保 $R$ 不具有自反性,我们必须确保没有任何元素与其自身相关。
        \item 为了确保 $R$ 不具有对称性,我们必须确保当 $(x, y) \in R$ 时,$(y, x) \notin R$。
        \item 为了确保 $R$ 具有传递性,我们必须确保当 $(x, y) \in R$ 且 $(y, z) \in R$ 时,($x, z) \in R$。
        \item 为了确保 $R$ 具有反对称性,我们需要考虑该性质定义的逆否命题,这要求任意一对元素最多只能以一种方式相关。反对称性可能是最难理解的;它表示对于每个 $x, y \in S$,要么 $x$ 与 $y$ 相关但 $y$ 与 $x$ 不相关,要么 $y$ 与 $x$ 相关但 $x$ 与 $y$ 不相关,或者 $x$ 和 $y$ 根本不相关。也就是说,我们不允许任何 $(x, y) \in R$ 和 $(y, x) \in R$ \emph{同时}成立。(请再次阅读反对称性的定义,并写下其条件陈述的逆否命题,思考为什么这样做有效。)
    \end{itemize}
    
    现在尝试构造满足这些性质的 $R$。性质 (1) 要求定义不能包含``等于''形式,性质 (2) 要求关系方向必须``唯一''。因此我们猜测,类似整数集上``小于''关系的比较性质可能适用。让我们验证其可行性。

    定义 $S$ 上的关系 $R$
    \[x \;R\; y \iff x \text{\ 的年龄(岁)严格小于\ } y\]

    现在逐一验证性质是否符合预期。在阅读以下解答之前,建议自行尝试证明或证伪这些性质,或在 $S$ 上定义其他关系(如自创一个),观察性质变化。你能构造另一个满足相同性质的关系吗?

    \begin{itemize}
        \item $R$ \textbf{不具有自反性}。因为任何人 $x \in S$ 都跟自己同岁,因此 $x \;\cancel{R}\; x$。
        \item $R$ \textbf{不具有对称性}。因为如果 $x$ 的年龄严格小于 $y$,则 $y$ 的年龄就严格大于 $x$,因此 $y \;\cancel{R}\; x$。
        \item $R$ \textbf{具有传递性}。因为如果 $x$ 的年龄严格小于 $y$,且 $y$ 的年龄就严格小于 $z$,那么 $x$ 的年龄必然严格小于 $z$。
        \item $R$ \textbf{具有反对称性}。因为对于任意 $x, y \in S$,要么其中一人的年龄小于另一人,要么两人同岁,不可能两人同时小于对方的年龄。(本质上,我们通过确保该性质定义中条件陈述的假设永远不成立来保证反对称性不成立。)
    \end{itemize}

    因此该关系 $R$ 满足所有要求的性质。
\end{example}

你可能注意到上述论证并不严谨(例如未提供具体反例),这是因为我们并不知道班上具体有谁,因此我们只说明事物存在而不指定具体对象。

要指出的是,一般而言,形如
\[(x, y) \in R \iff x \text{\ 在某种意义上``小于''\ } y\]
的关系通常不具有自反性和对称性,但具有传递性和反对称性。实际上,即使将``小于''替换为``大于'',结论仍然成立。例如 $\mathbb{N}, \mathbb{Z}$ 或 $\mathbb{R}$ 上的 ``$<$'' 或 ``$>$'' 关系,或人群中的``比……年轻'',``比……高''或``有更多子女''关系。思考 $\mathbb{Z}$ 上的 ``$\le$'' 关系,它与 ``$<$'' 关系有何不同区别?哪些性质发生了变化?

(下一小节将进一步探讨此类问题,我们将研究一种行为类似于 ``$\le$'' 和 ``$\ge$'' 的关系,称为\textbf{顺序关系}。)
