% !TeX root = ../../../book.tex

\subsection{关系的性质}

我们先来定义几个性质。对于这些性质,每个关系要么满足,要么不满足。建议你逐一阅读每个性质,并尝试构建满足该性质的关系,然后再构建不满足该性质的关系。这将有助于你深入理解性质的本质及关系的运作方式。(也可尝试定义同时具有多种性质的关系!)定义完性质后,我们会提供典型示例,但请先尝试自己构思并分享有趣例子!

\subsubsection*{定义:集合上关系的性质}

这些性质涉及元素对顺序的\emph{颠倒}。给定 $(x, y) \in R$,我们可能会考虑 $(y, x)$;但定义域与值域要求 $(y, x) \in A \times B$。因此需要满足 $A \times B = B \times A$,而这仅当 $A = \varnothing$、$B = \varnothing$ 或 $A = B$ 时成立。(参考第 \ref{ch:chapter03} 章集合论中的证明!)由于 $A = \varnothing$ 和 $B = \varnothing$ 的情形通常不考虑,我们讨论这些性质时假设 $A = B$ 且 $A \ne \varnothing$,即定义在非空集合上的关系。

\begin{definition}
    设 $A$ 为集合,$R$ 为 $A$ 上的关系,即 $R \subseteq A \times A$。
    \begin{itemize}
        \item 若
            \[\forall x \in A \centerdot (x, x) \in R\]
            则称 $R$ 具有\dotuline{自反性}。即每个元素都与其自身相关。
        \item 若
            \[\forall x,y \in A \centerdot (x, y) \in R \implies (y,x) \in R\]
            则称 $R$ 具有\dotuline{对称性}。即比较的顺序无关紧要。
        \item 若
            \[\forall x, y, z \in A \centerdot [(x, y) \in R \land (y, z) \in R] \implies (x, z) \in R\]
            则称 $R$ 具有\dotuline{传递性}。即关系可以通过中间元素传递。
        \item 若
            \[\forall x, y \in A \centerdot [(x, y) \in R \land (y, x) \in R] \implies x = y\]
            则称 $R$ 具有\dotuline{反对称性}。即两个不同元素最多只能有一种关系,或者根本没有关系。为了理解为什么这是等价的陈述,观察上述条件陈述的逆否命题:
            \[\forall x, y \in A \centerdot x \ne y \implies [(x, y) \notin R \lor (y, x) \notin R]\]
    \end{itemize}
\end{definition}

需要注意的是,\emph{反对称性 (anti-symmetric)} 与\emph{非对称性 (asymmetric)}并不相同。通过逻辑结构和量词可验证差异:例如 $\mathbb{R}$ 上的 $\le$ 关系满足反对称性但不满足对称性,请思考其中缘由。

尝试寻找既满足\emph{反对称性}又满足\emph{对称性}的关系(这并不困难,我们之前已经提及过此类基本关系)。