% !TeX root = ../../book.tex
\section{抽象(二元)关系}

\subsection{定义}

让我们直入主题,开始讨论关系。我们会先给出定义,然后提供一系列例子。

\begin{definition}
    设 $A, B$ 为集合。$A$ 和 $B$ 之间的\dotuline{关系}是由\dotuline{有序对}组成的集合,$R \subseteq A \times B$。对于元素 $a \in A$ 和 $b \in B$,当且仅当 $(a, b) \in R$ 时,我们说 $a$ 和 $b$ 是\dotuline{相关的}。

    集合 $A$ 称为\dotuline{定义域},集合 $B$ 称为\dotuline{值域}。集合 $R$ 称为\dotuline{关系集}。
    
    如果 $A = B$,我们称 $R$ 为 $A$ 上的关系。
\end{definition}

另一种常见的写法是用 $x \;R\; y$ 来表示 $(x,y) \in R$。当我们通过这种方式定义关系时,会坚持使用 $(x, y) \in R$ 来表示其基础集合结构。之后,我们有时会用一些符号来定义关系,比如 $x < y$ 或 $x \;\bigstar\; y$ 等等。

\begin{remark}
    我们这里定义的关系有时也叫\emph{二元关系},因为它涉及两个``输入'';集合 $R$ 是由\emph{有序对}组成的。

    我们可以将这个概念推广到\emph{三元关系}。也就是说,给定集合 $A,B,C$,我们可以定义 $R \subseteq A \times B \times C$ 为三元关系,当且仅当 $(a, b, c) \in R$ 时,$a,b,c$ 是相关的。我们还可以进一步推广到具有 $n$ 个``输入''的关系。不过,在本书中,我们只考虑\emph{二元关系},因此``\emph{关系}''一词将专指\emph{二元关系}。
\end{remark}

\begin{remark}
    关系 $R$ 通常通过确定 $A$ 和 $B$ 元素的某个\emph{性质}来定义(用变量命题 $P(a,b)$ 表示),并设定
    \[(a,b) \in R \iff P(a,b)\]
\end{remark}

\subsubsection*{示例}

\begin{example}
    设 $W=\{\text{英文单词}\}, L=\{\text{英文字母}\}$,定义关系 $R$ 如下:
    \[(w, \ell) \in R \iff w \text{以 } \ell \text{ 开头}\]
    则 $(\text{mathematics},\text{m}) \in R$ 且 $(\text{golf},\text{g}) \in R$,因为这些都是有效的单词,我们已经确定了它们的首字母。再来看一些反例,比如 $(\text{knowledge},\text{n}) \notin R$ 且 $(\text{you},\text{u}) \notin R$。此外 $(\text{zyzyxyqy},\text{z}) \notin R$,因为 $\text{zyzyxyqy} \notin W$。
\end{example}

$A = B$ 的情况经常出现,所以 $R$ 定义了同一集合中元素对之间的关系。下一个例子就是针对这种情况的。\\

\begin{example}
    设 $A=B=\mathbb{Z}$,定义 $\mathbb{Z}$ 上的关系 $R$ 如下:
    \[(x, y) \in R \iff x \text{ 和 } y \text{ 具有相同的奇偶性}\]
    则 $(2,8) \in R, (-3, 7) \in R, (-99, -99) \in R$,但 $(1,2) \notin R, (0, -3) \notin R$,且 $(\pi, 0) \notin R$(因为 $\pi \notin \mathbb{Z}$)。
\end{example}

\begin{example}
    定义 $\mathbb{R}$ 上的关系 $L$ 如下:
    \[(x, y) \in L \iff x < y\]
    则 $(-1, \pi) \in L$ 且 $(0, 100) \in L$,但 $(2, 2) \notin L$ 且 $(\pi, -1) \notin L$。
\end{example}

请注意,这里的对都是\emph{有序对}(我们可能会忘记这一点,因为 $A = B = \mathbb{R}$),所以元素的顺序很重要。确实,知道 $(x, y) \in L$ 并不一定意味着 $(y, x) \in L$。在上面这个例子中,这种情况实际上总是错误的!

回想一下,我们有时用 $x \;L\; y$ 来表示 $(x, y) \in L$,所以我们可以说 $-1 \;L\; \pi$ 但 $\pi \;\cancel{L}\; -1$,并且 $2 \;\cancel{L}\; 2$。

\subsubsection*{空关系}

\begin{remark}
    到目前为止,我们看到的例子在某种程度上都是有趣的关系。对于任意 $x,y \in R$,我们可以通过比较来判断 $x$ 是否小于 $y$。换句话说,我们看到的每个例子都是通过这样的方式定义的:对某性质 $P(a, b), (a, b) \in R \iff P(a, b)$ 为真。
\end{remark}

不过,关系不一定要这样定义。举个例子,我们知道对于任何集合 $S$,都有 $\varnothing \subseteq S$。因此,给定两个集合,我们总是可以通过 $\varnothing \subseteq A \times B$ 这一事实来定义一个\emph{平凡关系}。也就是说,\emph{平凡关系}是指没有任何元素相关的关系。这虽然看起来``无趣'',但它仍然符合关系的定义,所以我们也接受这种关系。

\subsubsection*{任何有序对的集合都是一个关系}

\begin{remark}
    给定集合 $A$ 和 $B$,任何子集 $R \subseteq A \times B$ 都定义了一个关系。然而,要找到一个能够描述这种关系的性质可能会非常困难,甚至是不可能的。

    比如 $A=\{1,3,5\}$ 且 $B=\{\bigstar, \heartsuit\}$,则我们可以定义 $A, B$ 上的关系
    \[R=\{(1,\bigstar), (5, \heartsuit)\}\]
    为什么 $1$ 与 $\bigstar$ 有关系?为什么 $3$ 不与任何元素有关系?这谁也说不清楚。这只是一个有序对的集合!从数学角度讲,这完全没有问题。
\end{remark}

\subsubsection*{相等关系}

\begin{example}
    另一种在任意集合 $X$ 上定义关系的方法是定义相等关系。也就是说,如果 $(x, y) \in R \iff x = y$。需要注意的是,这种定义与集合 $X$ 的具体内容无关,只要它是一个\emph{集合}即可。
\end{example}

\subsubsection*{关系之间的相似之处}

\begin{example}
    假设 $S$ 是你班上学生的集合。定义 $S$ 和 $\mathbb{N}$ 之间的关系 $R_1$,如果 $(s, n) \in R_1$,那么表示学生 $s \in S$ 的年龄是 $n$ 岁。写出这个关系集的一些元素。

    现在,在集合 $S$ 内定义关系 $R_2$,如果 $(s, t) \in R_2$,那么表示学生 $s$ 和 $t$ 的年龄相同(以年为单位)。写出这个关系集的一些元素。

    比较关系 $R_1$ 和 $R_2$,它们是否以某种方式传达了关于集合 $S$ 的相同信息?为什么是或为什么不是?是否可以通过 $R_1$ 确定 $R_2$?反过来是否也可以?仔细思考这些问题,并尝试总结你的想法。我们马上将在下一小节中讨论这些问题,但现在先花点时间自己思考一下吧!
\end{example}

\subsubsection*{关系``编码''信息}

前面的例子旨在说明抽象关系的实际用途,并解释我们为什么要讨论它们(除了我们想要严格定义函数这一目标之外)。从某种意义上说,关系是一种``保存''两个集合或一个集合中元素信息的方法,是比较两个元素并判断它们是否满足某性质的一种手段。而在更广泛的意义上,关系可以提供关于集合元素在特定性质下表征的信息。

例如,在前面的例子中,关系 $R_1$ 告诉了我们更多关于集合 $S$ 元素的信息。确切地说,$R_1$ 告诉我们哪些人年龄相同:我们可以找到像 $(s, n)$ 和 $(t, n)$ 这样的对,它们的第二个值相同。同时,$R_1$ 还告诉我们具体年龄是多少:只需要查看这些对的第二个值即可。而 $R_2$ 则不行。知道 $(s, t) \in R_2$ 只告诉我们学生 $s$ 和学生 $t$ 年龄相同,却没有具体的年龄信息!从这个角度讲,$R_1$ 是一个``更好''的关系,因为它提供了更多的信息。

不过,$R_2$ 也有其优点!例如,它有一个很好的性质:如果 $(x, y) \in R_2$,那么 $(y, x) \in R_2$ 也必然成立。这个性质在 $R_1$ 中显然不成立,因为当 $(s, n) \in R_1$ 时,说 $(n, s) \in R_1$ 是没有意义的,因为顺序不匹配定义域和值域!那么这个性质是否使 $R_2$ 成为一个``更好''的关系呢?嗯,这要看具体情况以及我们想要编码和检索的信息类型。在某些情况下,你可能会选择使用 $R_1$,而在其他情况下,你可能会选择使用 $R_2$。

不过我们这里有点超前了!我们还不能详细解释这些性质的含义及其优缺点。然而,总的来说,我们对这些性质及其在给定集合的所有元素对中何时(或何时不)成立感兴趣。在下一小节中,我们将定义和探索几种常见的抽象关系性质。虽然不能保证任何关系都具备这些性质,但它们在数学和实际应用中已被证明是有趣和有用的。之后,我们将看到更多关系的例子,并讨论如何证明这些性质成立。在这个过程中,我们将培养处理关系的直觉,甚至弄清楚我们首先要证明的那些声明的类型!

\subsection{关系的性质}

我们先来定义几个性质。对于这些性质,每个关系要么满足,要么不满足。我们建议你逐一阅读每个性质,并尝试构建一个满足该性质的关系,然后再构建一个不满足该性质的关系。这样可以帮助你更好地理解这些性质的基本原理以及关系的运作方式。(你还可以尝试定义一些同时具有多种性质的关系!)在定义完这些性质后,我们会提供一些典型的例子,或许你自己也能想到类似的例子!不过,真心地,试着自己想几个例子,并分享你想到的有趣例子吧!

\subsubsection*{定义:集合上关系的性质}

这些性质依赖于能够\emph{颠倒}一对元素的顺序。也就是说,给定 $(x, y) \in R$,我们可能会考虑 $(y, x)$;然而,定义域和值域之间的关系要求 $(y, x) \in A \times B$。因此,我们需要 $A \times B = B \times A$,这只有在 $A = \varnothing$ 或 $B = \varnothing$ 或 $A = B$ 时才会发生。(记住我们在第 \ref{ch:chapter03} 章谈论集合时已经证明了这一点!)由于 $A = \varnothing$ 和 $B = \varnothing$ 是不考虑的情况,我们在讨论这些性质时,假设 $A = B$(且 $A \ne \varnothing$),所以我们定义一个非空集合上的关系并比较其元素。

\begin{definition}
    设 $A$ 为集合,设 $R$ 为 $A$ 上的关系,即 $R \subseteq A \times A$。
    \begin{itemize}
        \item 我们称 $R$ 具有\dotuline{自反性},如果
            \[\forall x \in A \centerdot (x, x) \in R\]
            也就是说,每个元素都与其自身相关。
        \item 我们称 $R$ 具有\dotuline{对称性},如果
            \[\forall x,y \in A \centerdot (x, y) \in R \implies (y,x) \in R\]
            也就是说,比较的顺序无关紧要。
        \item 我们称 $R$ 具有\dotuline{传递性},如果
            \[\forall x, y, z \in A \centerdot [(x, y) \in R \land (y, z) \in R] \implies (x, z) \in R\]
            也就是说,关系可以通过一个中间人传递。
        \item 我们称 $R$ 具有\dotuline{反对称性},如果
            \[\forall x, y \in A \centerdot [(x, y) \in R \land (y, x) \in R] \implies x = y\]
            也就是说,两个不同的元素最多只能有一种关系,或者根本没有关系。为了理解为什么这是等价的陈述,让我们看看上述条件陈述的逆否命题:
            \[\forall x, y \in A \centerdot x \ne y \implies [(x, y) \notin R \lor (y, x) \notin R]\]
    \end{itemize}
\end{definition}

需要注意的是,\emph{反对称}(anti-symmetric)与\emph{非对称}(not symmetric)并不相同。要理解这一点,请仔细观察这些性质的逻辑顺序和量词。例如,$\mathbb{R}$ 上的 $\le$ 关系具有反对称性,但不具有对称性。想一想为什么会这样。

实际上,试着找出一个既具有\emph{反对称性}又具有\emph{对称性}的关系。这并不难!我们之前已经提到过一个具有这种性质的基本关系。

\subsection{示例}

请再次尝试找出一些符合和不符合我们刚刚定义的四个性质的关系。我们将在下面展示一些定义在 $\mathbb{N}$ 上的典型例子,以便给你一些具体的参考。你也可以随意添加一些简单的例子,比如定义在 $\mathbb{Z}$ 和 $\mathbb{R}$ 上的关系。

\begin{example}
    在这个例子中,所有关系都定义在集合 $\mathbb{N}$ 上。
    \begin{itemize}
        \item 定义 $\mathbb{N}$ 上的关系 $R_1$
        \[(x, y) \in R_1 \iff x \;\text{整除}\; y\]
        (即 $y$ 能被 $x$ 整除,或 $\exists k \in \mathbb{N}$ 使得 $y=kx$。该定义的严格陈述见定义\ref{def:definition6.2.15}。)

        则 $R_1$ 具有自反性,因为 $x \mid x$,即 $x=1 \cdot x$。

        \textbf{整除关系具有自反性。}
        \item 定义 $\mathbb{N}$ 上的关系 $R_2$
        \[(x, y) \in R_2 \iff x \text{ 和 } y \text{ 具有相同的奇偶性}\]
        则 $R_2$ 具有对称性,因为如果 $x$ 和 $y$ 具有相同的奇偶性,那么 $y$ 和 $x$ 当然也具有相同的奇偶性。

        \textbf{``相同奇偶性''关系具有对称性。}
        \item 定义 $\mathbb{N}$ 上的关系 $R_3$
        \[(x, y) \in R_3 \iff x < y\]
        则 $R_3$ 具有传递性,因为如果 $x<y$ 且 $y<z$ 则 $x<y<z$,所以 $x<z$。

        \textbf{``$<$''关系具有传递性。}
        \item 定义 $\mathbb{N}$ 上的关系 $R_4$
        \[(x, y) \in R_4 \iff x \le y\]
        则 $R_4$ 具有反对称性,因为如果 $x \le y$ 且 $y \le x$ 则 $x \le y \le x$,所以 $x=y$。

        \textbf{``$\le$''关系具有反对称性。}
    \end{itemize}
\end{example}

\begin{example}
    记住,关系其实就是一组有序对。我们不需要用性质来定义它。下面我们来看这样一个例子,并探讨它的性质:

    定义集合 $S=\{a,b,c\}$ 上的关系 $R$
    \[R = \{(a, a),(a, c),(b, c),(c, b)\}\]
    这个关系
    \begin{itemize}
        \item 不具有自反性:因为 $(c,c) \notin R$
        \item 不具有对称性:因为 $(a, c) \in R$ 但 $(c, a) \notin R$
        \item 不具有传递性:因为 $(a, c) \in R$ 且 $(c, b) \in R$ 但 $(a, b) \notin R$
        \item 不具有反对称性:因为 $(b, c) \in R$ 且 $(c, b) \in R$ 但 $b \ne c$
    \end{itemize}
\end{example}

\begin{example}
    让我们练习一下使用略为不同的关系符号。请记住,我们也可以用 $x \;R\; y$ 表示 $(x, y) \in R$。

    在班级同学集合 $S$ 上定义关系 $\bigstar$,对于任意 $x,y \in S$
    \[x \;\bigstar\; y \iff x \;\text{和}\; y \;\text{同一个月出生}\]
    我们声称这个关系具有自反性、对称性、传递性。你知道为什么吗?
    \begin{itemize}
        \item 该关系具有\emph{自反性},因为每个人当然与其自己同一个出生(即 $x \;\bigstar\; x$)。
        \item 该关系具有\emph{对称性},因为如果 $x$ 同学和 $y$ 同学同一个月出生(即 $x \;\bigstar\; y$),那么 $y$ 同学和 $x$ 同学(只是顺序不同!)当然也是同一个月出生(即 $y \;\bigstar\; x$)。
        \item 该关系具有\emph{传递性},因为……你应该明白了吧?我们只是在不断地强调``相同''这个词的概念!
    \end{itemize}
    关于\emph{反对称性},这要看具体情况!班上是否有两个人同一个月出生?如果有,这种关系就不具有反对称性。然而,如果班上每个人都在不同月份出生,那么这种关系就具有反对称性,因为没有人会与别人有关系,除了自己!好好琢磨一下……
\end{example}

\subsection{证明/证伪关系的性质}

当我们面对一个集合及其上的关系时,我们会立即想知道这些关系是否具有某些性质。通过尝试集合中的一些特定元素,我们可以猜测该关系是否满足某个性质,然后尝试证明或证伪它。这有时相当于``猜测和检验'',但最终,要证明一个性质成立,我们必须证明一个形式为``对于所有……都成立……''的命题(请回顾 \ref{sec:section4.9} 节的证明技巧!)。因此,证明关系性质相当于取一个任意元素(或多个元素)并讨论它们之间的关系。要证伪这样的命题,我们会证明它的逻辑否定形式,即``存在……使得……''(请再次回顾我们的证明技巧!)。因此,证伪一个性质相当于找到一个\emph{反例}。让我们看几个证明或证伪关系性质的例子。在练习中还有更多这种风格的证明例子。

\subsubsection*{$\mathbb{Z}$ 上的``整除''关系}

我们将首先介绍(或者提醒你)一个定义,因为它是我们其中一个例子的基础。这是一个关于一个整数整除另一个整数的正式定义。

\begin{definition}\label{def:definition6.2.15}
    设 $a,b \in \mathbb{Z}$,我们称 \dotuline{$x$ 整除 $y$},写做 \dotuline{$x \mid y$},当且仅当
    \[\exists k \in \mathbb{Z} \centerdot y = kx\]
\end{definition}

\begin{example}
    定义 $\mathbb{Z}$ 上的关系 $R$
    \[(x, y) \in R \iff x \mid y\]
    让我们判断 $R$ 是否满足关系的四个性质,然后证明或证伪我们所有的主张!

    一般来说,根据所讨论的集合和关系,你可能会通过直觉或者直接``看出''某个性质是否成立。如果是这样,那太好了!如果不是(这是更常见的情况),我们建议发起一个``证明'',假设某个性质成立,并看看否能证明出来。如果你做到了,那么你就证明了这个性质!如果你在某处遇到困难,可能是因为这个性质不成立,而你在证明中遇到困难的地方会给你一些找到反例的启示。这个策略并不总是有效(也许你在证明中遇到困难是因为它实际上很有挑战性),但它可能非常有帮助,所以请记住这一点。在这个例子中我们也会看到这一策略的实际应用。

    另一个策略---事实上更简单的一个策略---就是大声说出或用文字写下所讨论的关系和性质。有时候,仅仅让自己用简单的语言说出某些东西,而不是阅读页面上的抽象符号,会让你的大脑意识到一些有用的信息!我们也会在这里看到这一策略的实际应用。

    \begin{itemize}
        \item 让我们探讨一下 $R$ 是否具有\textbf{自反性}。这具体意味着什么呢?我们不妨大声说出来:任意整数都能被它自己整除。这是肯定的!现在,让我们尝试用证明所需的符号来表达这一点。
        \begin{proof}
            我们声称 $R$ 具有自反性。设 $x \in \mathbb{Z}$ 为任意固定整数。由于 $x = 1 \cdot x$ 且 $1 \in \mathbb{Z}$,所以 $x \mid x$。因此,$(x, x) \in R$。由此可见,$R$ 具有自反性。
        \end{proof}
        你看!通过说出或写下我们的思考,我们意识到了一个事实,这使得我们更容易用数学语言来表达这个陈述。
        \item 让我们探讨一下 $R$ 是否具有\textbf{对称性}。这个性质是通过一个\emph{蕴涵}术语(\emph{条件陈述})来定义的。假设我们有一个任意关联对 $(x, y) \in R$,我们能否必然得到 $(y, x) \in R$ 呢?换句话说:
        \[\text{假设} x \text{能整除} y, \text{我们能否说} y \text{也能整除} x \text{?} \]
        这看起来不太可能!$x \mid y$ 意味着 $y = kx$,其中 $k \in \mathbb{Z}$,但这并不意味着 $x = \frac{1}{k}y$ 表示 $y$ 也能整除 $x$。万一 $\frac{1}{k} \notin \mathbb{Z}$ 怎么办?

        你可能会说:``$\frac{1}{k}$ 只有在 $k = 1$ 或 $k = -1$ 时才是整数,所以就是这样。''但这并不是完整的解释!要反驳一个``对于所有……''的命题,我们需要尽可能提供一个明确的反例。我们不需要全面描述该性质在所有情况下是否成立,只需要一个例子来证明这个性质不成立。这比含糊其辞地说某处可能存在反例要更直接明了。让我们展示一个反例给读者,然后再继续!
        \begin{proof}
            考虑 $2,6 \in \mathbb{Z}$,因为 $6=3 \cdot 2$,所以我们有 $(2,6) \in R$。

            然而,使 $2 = \ell \cdot 6$ 成立需要 $\ell = \frac{1}{3}$,而 $\frac{1}{3} \notin \mathbb{Z}$,因此 $(6,2) \notin R$。

            这证明了 $R$ 不具有\emph{对称性}。
        \end{proof}
        \item 让我们探讨一下 $R$ 是否具有\textbf{传递性}。传递性通常是最难理解的性质之一。这主要是因为它是由包含两个假设的条件陈述定义的,并且涉及到三个变量。

        在这个具体例子中,我们假设 $x \mid y$ 且 $y \mid z$,然后考虑是否必然有 $x \mid z$。试着大声读出来,看看你认为这是否成立。
        
        看起来是成立的,对吧?现在,试着用数学语言写下你的假设和结论。你能看出如何将它们结合起来吗?在继续阅读之前,试着写出你自己的证明。
        \begin{proof}
            设 $x, y, z \in \mathbb{Z}$ 是任意固定的。假设 $(x, y) \in R$ 且 $(y, z) \in R$。这意味着 $x \mid y$ 且 $y \mid z$。所以 $\exists k, \ell \in \mathbb{Z}$ 使得 $y = kx$ 且 $z = \ell y$。给定这样的 $k, \ell$,将第一个等式代入第二个,可得
            \[z = \ell y = \ell (kx) = (k \ell)x\]
            因为 $k \ell \in \mathbb{Z}$,我们证明了 $x \mid z$,所以 $(x, z) \in R$。

            因此,$R$ 具有传递性。
        \end{proof}
        \item 让我们探讨一下 $R$ 是否具有\textbf{反对称性}。这一性质也由包含两个假设的条件陈述定义,因此我们假设有 $x$ 和 $y$,满足 $x \mid y$ 且 $y \mid x$。我们能得出 $x = y$ 吗?这个问题让我们回想起之前证明 $R$ 不具有对称性的过程。记住,我们已经证明了 $x \mid y$ 并不一定意味着 $y \mid x$。实际上,如果稍加思考,就会发现 $x \mid y$ 和 $y \mid x$ 同时为真是不可能的。这究竟是怎么回事呢?请仔细思考,在阅读我们的证明之前尝试自己给出一个证明。
        \begin{proof}
            设 $x, y \in \mathbb{Z}$ 是任意固定的。假设 $(x, y) \in R$ 且 $(y,x) \in R$。

            这意味着 $x \mid y$ 且 $y \mid x$,所以 $\exists k, \ell \in \mathbb{Z}$ 使得 $y = kx$ 且 $x = \ell y$。给定这样的 $k, \ell$,将第一个等式代入第二个,可得
            \[y = kx = k(\ell y) = (k \ell)y\]
            存在如下两种情况:

            \textbf{情况 1}:假设 $y=0$。此时我们无法两边同时除以 $y$。我们反而知道 $x = \ell y = \ell \cdot 0 = 0$,因为 $x=0$,所以该情况下 $x=y$。

            \textbf{情况 2}:假设 $y \ne 0$,此时两边同时除以 $y$ 得 $k \ell = 1$。因为 $k, \ell \in \mathbb{Z}$ 这意味着要么 $k = \ell = 1$ 要么 $k = \ell = -1$。

            如果 $k = \ell = 1$,则 $x = \ell y = y$。

            如果 $k = \ell = -1$, 则 $x = \ell y = -y$。

            因此 $R$ 不具有反对称性。
        \end{proof}
        哦,糟糕!你明白发生了什么吗?在``大多数''情况下,我们确实得出了 $x = y$ 的结论,但实际上还有可能 $y = -x$。比如,当 $y = 3$ 而 $x = -3$ 时,显然 $x \mid y$ 且 $y \mid x$ 但 $x \ne y$。这就是我们需要的反例,尝试完成我们的``证明''。或许你早已预见到这一情况,如果是这样,那真是太棒了!最后,我们通过展示这个反例来完成证明:
        \begin{proof}
            考虑 $x=3, y=-3$, 显然 $x, y \in \mathbb{Z}$。因为 $3 = (-1)(-3)$ 且 $-3 = (-1) \cdot 3$,并且 $1, -1 \in \mathbb{Z}$,所以 $x \mid y$ 且 $y \mid x$。

            然而,显然 $x \ne y$。这证明了 $R$ 不具有反对称性。
        \end{proof}
    \end{itemize}
\end{example}

作为后续问题,思考一下如果我们在集合 $\mathbb{N}$ 上定义这个关系,而不是在 $\mathbb{Z}$ 上,会有什么变化?哪些性质会成立?答案是否会与在 $\mathbb{Z}$ 上有所不同?请仔细思考这些问题,因为此处的答案将引导我们进入下一个小节。

\subsubsection*{构建具有特定性质的关系}

在继续之前,我们再来看一个例子。这个有趣的``游戏''是从集合中选取元素,并构造出满足特定性质的关系 $R$。(注意:$4$ 个性质的组合有 $16$ 种不同的可能。)在习题中,我们会问你类似的问题,所以让我们通过一个例子来演示一下。

\begin{example}
    \textbf{目标}:设 $S$ 为班上同学的集合。定义一个关系 $R$ 
    \begin{enumerate}[label=(\arabic*)]
        \item 不具有自反性
        \item 不具有对称性
        \item 具有传递性
        \item 具有反对称性
    \end{enumerate}

    为了确保 $R$ 不具有自反性,我们必须确保没有任何元素与其自身相关。为了确保 $R$ 不具有对称性,我们必须确保当 $(x, y) \in R$ 时,$(y, x) \notin R$。为了确保 $R$ 具有传递性,我们必须确保当 $(x, y) \in R$ 且 $(y, z) \in R$ 时,($x, z) \in R$。为了确保 $R$ 具有反对称性,我们需要考虑性质定义的逆否命题,这要求任意一对元素最多只能以一种方式相关。最后一个性质可能是最难理解的;它表示对于每个 $x, y \in S$,要么 $x$ 与 $y$ 相关但 $y$ 与 $x$ 不相关,要么 $y$ 与 $x$ 相关但 $x$ 与 $y$ 不相关,或者 $x$ 和 $y$ 根本不相关。也就是说,我们不允许任何 $(x, y) \in R$ 和 $(y, x) \in R$ \emph{同时}成立。(请再次阅读反对称性的定义,并写下其条件陈述的逆否命题,思考为什么这样做有效。)

    现在让我们尝试构造一个满足这些性质的 $R$。性质 (1) 表明我们的定义不能包含``或等于''的形式,而性质 (2) 则要求定义必须以``唯一的方式''关联任意的 $x$ 和 $y$。因此,我们可以猜测,一个类似于 $\mathbb{Z}$ 集合上``小于''关系的\emph{比较}性质可能会奏效。让我们尝试一下,并验证这些性质是否成立。

    我们定义 $S$ 上的关系 $R$
    \[x \;R\; y \iff x \;\text{的年龄(岁)严格小于}\; y\]
    现在,让我们来探讨一下这个关系的性质,并确保它们是符合我们的预期的。在阅读我们的解决方案之前,尝试自己证明或证伪这些性质。另外,尝试在 $S$ 上定义不同的关系(自己创造一个!),看看这些性质有何不同。你能想出另一个具有相同性质的关系吗?

    \begin{itemize}
        \item $R$ \textbf{不具有自反性}。因为任何人 $x \in S$ 都跟他/她自己同岁,因此 $x \;\cancel{R}\; x$。
        \item $R$ \textbf{不具有对称性}。因为如果 $x$ 的年龄严格小于 $y$,则 $y$ 的年龄就严格大于 $x$,因此 $y \;\cancel{R}\; x$。
        \item $R$ \textbf{具有传递性}。因为如果 $x$ 的年龄严格小于 $y$,且 $y$ 的年龄就严格小于 $z$,那么 $x$ 的年龄当然严格小于 $z$。
        \item $R$ \textbf{具有反对称性}。因为对于任意两人 $x, y \in S$,要么其中一人的年龄小于另一人,要么两人同岁,不可能两人同时小于对方的年龄。(本质上,我们通过确保该性质定义中的条件陈述的假设永远不成立来保证反对称性,因此条件陈述本身总是成立。)
    \end{itemize}

    因此该关系 $R$ 满足所有要求的性质。
\end{example}

你可能注意到,我们的论证并不严谨,但这是有原因的。具体来说,我们没有为那些不成立的性质提供明确的反例。如果我们能够找到班上的两个学生,并证明一个比另一个年轻,但反过来却不是这样,那就更好了。但我们并不知道你班上都有谁!这就是为什么我们的论证是``解释某事物存在而不明确指出它''。

我们要指出的是,一般来说,这种形式的关系
\[(x, y) \in R \iff x \;\text{在某种意义上``小于''}\; y\]
通常不具有自反性和对称性,但具有传递性和反对称性。实际上,我们甚至可以将``比……小''替换为``比……大'',这个结论仍然成立。要理解为什么会这样,可以想想在 $\mathbb{N}, \mathbb{Z}$ 或 $\mathbb{R}$ 上的 ``$<$'' 关系,或者这些集合上的 ``$>$'' 关系。再想想人群中的``比……年轻''关系、``比……高''关系,或者``有更多孩子''关系。$\mathbb{Z}$ 上的 ``$\le$'' 关系又如何呢?这与 ``$<$'' 关系有何不同?哪些性质发生了变化?

(这些类型的问题将在下一小节中进一步探讨,我们将研究一种行为类似于 ``$\le$'' 和 ``$\ge$'' 关系的特定类型的关系。它们被称为\textbf{顺序关系}。)

\subsection{习题}

\subsubsection*{温故知新}

以口头或书面的形式简要回答以下问题。这些问题全都基于你刚刚阅读的内容,所以如果忘记了具体的定义、概念或示例,可以回去重读相关部分。确保在继续学习之前能够自信地回答这些问题,这将有助于你的理解和记忆!

\begin{enumerate}[label=(\arabic*)]
    \item 如何用集合来定义\emph{(二元)关系}?
    \item 假设我们有一个定义在集合 $A$ 和 $B$ 之间的关系 $R$。为了讨论 $R$ 是否具有\textbf{自反性},$A$ 和 $B$ 需要满足什么条件?
    \item 在什么情况下,一个关系具有\textbf{自反性}?请举一个集合和该集合上的自反关系的例子。
    \item 在什么情况下,一个关系具有\textbf{对称性}?请举一个集合和该集合上的对称关系的例子。
    \item 在什么情况下,一个关系具有\textbf{传递性}?请举一个集合和该集合上的传递关系的例子。
    \item 在什么情况下,一个关系具有\textbf{反对称性}?请举一个集合和该集合上的反对称关系的例子。
    \item \emph{非对称}和\emph{反对称}有什么区别?\\
        请举一个既具有对称性又具有反对称性的关系的例子。
\end{enumerate}

\subsubsection*{小试牛刀}

尝试回答以下问题。这些题目要求你实际动笔写下答案,或(对朋友/同学)口头陈述答案。目的是帮助你练习使用新的概念、定义和符号。题目都比较简单,确保能够解决这些问题将对你大有帮助!

\begin{enumerate}[label=(\arabic*)]
    \item 考虑集合 $A = \{1, 2, 3\}$。对于如下每个定义在 $A$ 或 $\mathcal{P}(A)$ 上的关系,判断是否具有
    \begin{enumerate}[i]
        \item 自反性
        \item 对称性
        \item 传递性
        \item 反对称性
    \end{enumerate}

    不需要详细解释,只需回答``\verb|是|''或``\verb|否|''并附上一两句说明即可。

    \begin{enumerate}[label=(\alph*)]
        \item 定义在 $A$ 上的关系 $R_a = \{(1, 1),(1, 2),(2, 1),(2, 2),(3, 3)\}$
        \item 定义在 $A$ 上的关系 $R_b = \{(1, 1),(1, 2),(2, 2),(2, 3),(3, 3)\}$
        \item 定义在 $\mathcal{P}(A)$ 上的关系 $R_c$,$\forall S, T \in \mathcal{P}(A) \centerdot (S, T) \in R_c \iff S \cap T = \varnothing$
        \item 定义在 $\mathcal{P}(A)$ 上的关系 $R_d$,$\forall S, T \in \mathcal{P}(A) \centerdot (S, T) \in R_d \iff S \cap T \ne \varnothing$
        \item 定义在 $\mathcal{P}(A)$ 上的关系 $R_e$,$\forall S, T \in \mathcal{P}(A) \centerdot (S, T) \in R_e \iff S \subseteq T$
    \end{enumerate}
    \item 定义 $\mathbb{Z}$ 上的关系 $\bigstar$ 为
    \[\forall x, y \in \mathbb{Z} \centerdot x \;\bigstar\; y \iff 3 \mid x - y\]
    \begin{enumerate}[label=(\alph*)]
        \item 证明 $\bigstar$ 具有自反性
        \item 证明 $\bigstar$ 具有对称性
        \item 证明 $\bigstar$ 具有传递性
    \end{enumerate}
    (请记住,``$\mid$'' 表示``整除''。请确保使用定义 \ref{def:definition6.2.15} 给出的正式定义。)
    \item 定义 $\mathbb{Z}$ 上的关系 $\sim$ 为
    \[\forall x, y \in \mathbb{Z} \centerdot x \sim y \iff 3 \mid x + 2y\]
    \begin{enumerate}[label=(\alph*)]
        \item 证明 $\sim$ 具有自反性
        \item 证明 $\sim$ 具有对称性
        \item 证明 $\sim$ 具有传递性
    \end{enumerate}
    \item 定义 $\mathbb{R}$ 上的关系 $T$ 为,对于任意 $x, y \in \mathbb{R}$
    \[(x, y) \in T \iff \Big(\frac{y}{x} \in \mathbb{R} \land \frac{y}{x} \ge 0\Big)\]
    \begin{enumerate}[label=(\alph*)]
        \item 找出 $x \in \mathbb{R}$ 使得 $(x,x) \notin T$。这是否意味着 $T$ 不具有自反性?为什么?
        \item 找出 $x,y \in \mathbb{R}$ 使得 $(x,y) \in T$ 且 $(y,x) \in T$。这是否意味着 $T$ 具有对称性?为什么?
        \item 找出 $x,y \in \mathbb{R}$ 使得 $(x,y) \in T$ 但 $(y,x) \notin T$。这是否意味着 $T$ 不具有对称性?为什么?
        \item 判断 $T$ 是否具有传递性,并证明你的声明。
    \end{enumerate}
    \item 定义 $\mathcal{P}(\mathbb{N})$ 上的关系 $\leftrightarrow$ 为,对于任意 $X, Y \in \mathbb{N}$
    \[X \leftrightarrow Y \iff \Big(X \subseteq Y \lor X \cap Y = \varnothing \Big)\]
    证明或证伪该关系上的四个基本性质(即自反性、对称性、传递性和反对称性)。
    \item 以下论证给出了对称性和传递性如何蕴含自反性。这个``证明''存在什么问题?
    \begin{center}
        \noindent
            \parbox{0.85\textwidth}{%
                \linespread{1.5}\selectfont
                设 $A$ 为非空集合。设 $R$ 为 $A$ 上的关系。

                假设 $R$ 具有对称性和传递性。我们将证明 $R$ 也具有自反性。

                设 $x \in A$ 是任意固定的。定义集合 $T$
                \[\{y \in A \mid (x, y) \in R\}\]
                给定 $y \in T$,则 $(x, y) \in R$。

                因为 $R$ 具有对称性,我们可以推导出 $(y,x) \in R$。

                因为 $R$ 具有传递性,根据 $(x, y) \in R$ 且 $(y, x) \in R$,我们可以推导出 $(x,x) \in R$。

                因为 $x$ 是任意的,我们证明了自反性成立。 
            }
    \end{center}
\end{enumerate}