% !TeX root = ../../book.tex
\section{引言}

在我们建立起坚实的数学术语和概念基础之后,你可能会好奇接下来要讨论什么!就像我们之前严格化了数学归纳法的概念一样,在接下来的两章中,我们将进一步探讨一个你可能已经直观理解但还未精确定义的概念:\textbf{函数}。

为此,我们将从\textbf{关系}开始讨论,进而探讨\textbf{等价关系},这将有助于我们研究集合的一些性质。特别地,我们将利用集合 $\mathbb{Z}$ 上的等价关系来陈述并证明许多有趣的整数性质。这将让我们有机会一览\textbf{数论}这个丰富多彩的领域。我们将通过陈述和证明一些有用的定理,并用它们解决一些有趣的问题来初探数论的王国。然后,我们将继续下一章,回到讨论函数的目标上来。

\subsection{目标}

以下简短内容将向你展示本章如何融入本书的体系。我们会解释之前的工作对本章研究的帮助,说明我们为什么要探讨本章的主题,并告诉你我们的目标以及在阅读时需要注意的事项。现在,我们将通过几条陈述总结本章的主要目标。这些陈述概括了你在完成本章后应掌握的技能和知识。接下来的章节会更详细地解释这些思想,这里仅提供一个简要列表供你参考。完成本章后,请返回这个列表,检查你是否理解所有目标。你能看出我们为什么认为这些目标重要吗?你能解释我们使用的术语并应用我们描述的技术吗?

\textbf{学完本章后,你应该能够……}

\begin{itemize}
    \item 定义关系并提供多个例子。
    \item 定义并理解关系的各种属性,并举例说明哪些关系具备或不具备这些属性。
    \item 研究一个已定义的关系,发现并证明它的属性。
    \item 定义等价关系和等价类,并解释为什么它们特别有趣和重要。
    \item 研究一个已定义的等价关系,并对其等价类进行分类。
    \item 利用整数集上的特定等价关系,陈述并证明数论中的有趣结果。
    \item 定义乘法逆元的概念,理解它在模算术中的具体意义,并应用这一概念证明或证伪特定方程的解。
    \item 陈述并理解数论中的各种定理,并应用这些定理解决问题。
\end{itemize}

\subsection{承上}

本章的内容并不像之前的章节那样紧密相关,而是开启了本书的\emph{新篇章}。从现在开始,我们将运用之前学到的所有数学知识,探索其他有趣的领域。之前的学习都是为了这一刻做准备。接下来,我们会提出复杂的命题,并运用证明技巧加以证明。我们会提供定义和定理,并希望你能用这些工具来证明其他命题。可以说,本章是对之前\emph{所有}章节内容的综合应用。我们将充分利用之前积累的知识、术语和经验!

\subsection{启下}

你可能在微积分中处理过函数(比如微分和积分),或者在高中代数课上绘制过函数图像或求解过函数的根,甚至在计算机科学中编写过算法或使用过递归编程。但是,试着\emph{定义}一下什么是函数。你会怎么向从未学过数学的人解释呢?又会如何向一个超智能的外星人解释呢?如果用我们在数学归纳法中那样的严谨程度来解释,你会怎么做?这并不容易,对吧?

为了深入理解\emph{函数}的概念,我们首先要讨论\emph{关系},这是比较集合中元素的一种方法。我们会通过许多例子来了解它们的性质。然后,在下一章中,我们会发现函数其实是关系的一种特殊类型!在讨论关系的过程中,我们会探讨它们的属性,并发现某些属性的组合会形成特定的性质。具体来说,我们会看到\emph{等价关系}会自然地生成集合\emph{划分},反之亦然。这一发现将帮助我们陈述并证明一些关于整数的重要结果。

\subsection{忠告}

本章将继续探讨抽象概念和严谨的数学内容,因此,如果你对这些日益抽象的内容和相关语言感到不适,千万不要因此就认为这些信息``无关紧要''或``百无一用''。这些概念会贯穿整本书,甚至整个数学领域!所以,当你感觉难以集中注意力时,请记住这一点。我们建议你记下学习笔记,以便提醒自己当前的学习内容。当你看到一个定理并多次阅读后终于理解时,可以在书的边缘写下定理的摘要,方便以后查阅。画一些小图形帮助你理解例子或定理的重要部分。遇到定义时,写下一个典型例子和一个反例。读完证明后,记下论证步骤的大纲,这样可以将概念``模块化'',便于更有效地记忆和回忆。如果你对某个定义、定理或证明感到困惑,也要记下来!带着问题去问同学、朋友、助教或教授,看看他们能否帮你解答。最重要的是,请记住,理解消化和融会贯通这些抽象概念和论证需要\emph{时间},通过例子验证自己是否跟上进度依然非常重要。如果你能理解并向别人解释某个概念,那说明你已经掌握得很好了。