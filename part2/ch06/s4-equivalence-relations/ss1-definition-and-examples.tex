% !TeX root = ../../../book.tex

\subsection{定义与示例}

我们稍微转换一下话题,讨论一种满足关系四个标准性质中不同子集的关系。实际上,让我们回到之前提到的一个具体例子:定义集合 $\mathbb{R}$ 上的关系 $R$:
\[(x, y) \in R \iff \lfloor x \rfloor = \lfloor y \rfloor\]
(如果你跳过了选学部分,可以在示例 \ref{ex:example6.3.3} 中找到这个例子。)

注意,这个关系满足如下性质:
\begin{itemize}
    \item \emph{自反性}:因为 $\forall x \in \mathbb{R} \centerdot \lfloor x \rfloor = \lfloor x \rfloor$
    \item \emph{对称性}:因为 $\forall x, y \in \mathbb{R} \centerdot \lfloor x \rfloor = \lfloor y \rfloor \implies \lfloor y \rfloor = \lfloor x \rfloor$
    \item \emph{传递性}:因为 $\forall x, y, z \in \mathbb{R} \centerdot \big(\lfloor x \rfloor = \lfloor y \rfloor \land \lfloor y \rfloor = \lfloor z \rfloor \big) \implies \lfloor x \rfloor = \lfloor z \rfloor$
\end{itemize}
由于这组特定性质带来了许多有趣且有用的结果,因此我们为满足这三条性质的关系赋予一个专门的名字。

\begin{definition}
    设 $A$ 为集合,$R$ 是 $A$ 上的关系。如果 $R$ 具有自反性、对称性和传递性,则我们称 $R$ 为\dotuline{等价关系}。
\end{definition}

我们只需要在集合 $S$ 上的任意关系 $R$ 中验证这三条性质,就能确定 $R$ 是否为等价关系。接下来,我们回顾几个已经见过的关系示例,并根据已有的证明来判断它们是否是等价关系。\\

\begin{example}
    \begin{enumerate}[label=(\arabic*)]
        \item 回顾我们在示例 \ref{ex:example6.2.9} 中定义的任意集合 $X$ 上的相等关系。这是一种等价关系。显然,$(x, x) \in R$,因为 $x=x$。然而,对于任意 $x \ne y$ 的情况,假设 $x \;R\; y$ 不成立,这反而使条件陈述成立。因此,对称性中唯一``相关的情况''是 $x=y$,此时 $y=x$。同样地,在传递性中,如果 $x \ne y$ 或 $y \ne z$,定义条件陈述的假设不成立,所以陈述本身成立;而当 $x = y$ 且 $y = z$ 时,显然 $x = z$。这可能看起来并不是特别有启发性,但知道我们总是可以在任意集合上定义至少一个等价关系还是很有用的。
        \item $\mathbb{Z}$ 上的``整除''关系\textbf{不是}等价关系,因为它不满足对称性。(见示例 \ref{ex:example6.2.16})
        \item (非空)集合的``严格小于''关系\textbf{不是}等价关系,因为它不满足自反性。(见示例 \ref{ex:example6.2.17})
        \item 定义在 $\mathbb{Z}$ 上的关系 $R$
        \[\forall x, y \in \mathbb{Z} \centerdot x \;\bigstar\; y \iff 3 \mid x - y\]
        \textbf{是}等价关系,因为它满足自反性、对称性和传递性。(见 \ref{sec:section6.2.5} 节习题 \ref{exc:exercises6.2.2})这个等价关系示例将在本章后面进行详细讨论和泛化。你甚至可能已经看出它是``模 $3$ 等价''关系!
    \end{enumerate}
\end{example}

本章中的许多习题将会以``确定此定义是否构成等价关系''的形式出现。这些问题类似于我们之前见过的``证明或证伪以下声明''的问题。我们需要设法判断某个给定的关系是否是等价关系;如果是,我们需要证明它,如果不是,我们需要找出哪条性质不成立并提供一个反例。下面让我们通过一个例子来说明这个思路。\\

\begin{example}
    设 $S=\mathbb{N}-\{1\}$ 并定义 $(x, y) \in R \iff x \;\text{与}\; y \;\text{有公因子}$(公因子不包括 $1$,要严格大于 $1$)。我们可以通过尝试证明这个关系的性质来确定它是否是等价关系,并观察论证过程中是否会失败。如果不会失败,那么我们就证明成功了;如果失败了,我们可以利用这些信息来构建一个反例。

    首先,不难发现 $(x, x) \in R$,因为 $x$ 和 $x$ 有公因子 $x$,且根据 $S$ 的定义,我们知道 $x>1$,所以 $R$ 具有自反性。

    其次,不难发现,如果 $(x, y) \in R$,那么 $x$ 和 $y$ 必然存在某公因子 $k>1$。显然交换 $x$ 和 $y$ 不会改变这一事实:$y$ 和 $x$ 必然存在某公因子 $k>1$,即 $(y,x) \in R$。因此 $R$ 具有对称性。

    最后,我们假设 $(x, y) \in R$ 且 $ (y,z) \in R$。这意味着 $x$ 和 $y$ 存在公因子 $k>1$ 且 $y$ 和 $z$ 存在公因子 $\ell>1$。我们可以据此找到 $x$ 和 $z$ 的公因子吗?不一定。我们无法确定 $k$ 和 $\ell$ 存在公因子。比如,如果 $k=2, \ell=3$,我们能否找到 $x, y, z$ 的值满足该公因子,然后验证 $x$ 和 $z$ 是否存在公因子?当然可以。设 $x=2, y=6, z=9$,显然 $(2,6) \in R$ 且 $(6,9) \in R$,但 $(2,9) \notin R$。这个反例证明了该关系不具有传递性。因此它不是等价关系。
\end{example}

我们推荐使用这种方法来判断一个关系是否是等价关系或顺序关系。只需逐一检查每个相关属性 --- 如自反性、对称性、传递性等 --- 并\textbf{尝试证明}它们。如果你成功证明了所有属性,那就说明这个关系是等价关系或顺序关系。如果在证明某个属性时遇到了困难,可以通过分析问题所在,找出该属性失败的原因,并据此构建一个反例。

\subsubsection*{启下}

回想一下我们在本节提到的第一个例子,其中 $x \;R\; y \iff \lfloor x \rfloor = \lfloor y \rfloor$。注意,每个实数都对应一个整数,具体来说是\emph{向下取整}得到的整数。例如,$1.5 \;R\; 1, \pi \;R\; 3$ 和 $-1.5 \;R\; -2$。此外,任意两个对应同一整数的实数也彼此相关。例如,$3.5 \;R\; 3$ 和 $\pi \;R\; 3$,因此 $3.5 \;R\; \pi$。基于这些观察,我们认为可以将满足 $0 \le x < 1$ 的所有实数``打包''成一个``簇'',并用其中一个元素(如 $0$)来表示。同理,我们可以将满足 $1 \le x < 2$ 的所有实数打包成一个簇,用 $1$ 表示。以此类推。我们不一定要选择 $0$ 和 $1$ 作为代表元素,也可以选择 $\frac{1}{2}$ 和 $\frac{3}{2}$。关键在于,同一个簇内的所有实数\emph{彼此相关},我们可以用一个\textbf{代表}元素来表示每个簇。

这一观察将引导我们进入下一节,讨论如何正式描述这些``簇'',即\textbf{等价类}。我们将研究许多例子,并总结出一些一般性质。

在此之前,我们强烈建议你尝试一些我们已经讨论过的例子,寻找这些``簇''和``代表元素''。例如,考虑在 $\mathbb{Z}$ 上定义的关系 $R$:
\[\forall x, y \in \mathbb{Z} \centerdot x \;\bigstar\; y \iff 3 \mid x - y\]
这是一个等价关系。在这种情况下,这些``簇''是什么?你能识别出所有簇吗?每个簇中有多少元素?你能选择代表元素吗?

试着用另一个等价关系做同样的事情,比如``出生在同一个月份''的关系(在你的班级学生集合上)。经过思考,你会发现这是一个等价关系。

一个同样具有启发性的任务是考虑一个\textbf{非}等价关系,并试图弄清楚它\emph{为什么或如何}不具备这种``簇''性质。例如,考虑在 $\mathbb{Z}$ 上的``整除''关系。它在哪些方面不具备这种性质?它是否接近这种性质?

总之,做一些探索吧!这样做将有助于巩固你对关系性质的理解,并使下一节更容易理解。
