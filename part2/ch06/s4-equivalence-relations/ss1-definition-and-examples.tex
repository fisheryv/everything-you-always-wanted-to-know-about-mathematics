% !TeX root = ../../../book.tex

\subsection{定义与示例}

我们稍微转换一下话题,讨论满足关系四大基本性质中部分子集的关系。以先前的一个具体例子为例:定义实数集 $\mathbb{R}$ 上的关系 $R$:
\[(x, y) \in R \iff \lfloor x \rfloor = \lfloor y \rfloor\]

(如果你跳过了选学部分,可以在示例 \ref{ex:example6.3.3} 中找到这个例子。)

注意,这个关系满足如下性质:
\begin{itemize}
    \item \emph{自反性}:因为 $\forall x \qquad \in \mathbb{R} \centerdot \lfloor x \rfloor = \lfloor x \rfloor$
    \item \emph{对称性}:因为 $\forall x, y \quad \in \mathbb{R} \centerdot \lfloor x \rfloor = \lfloor y \rfloor \implies \lfloor y \rfloor = \lfloor x \rfloor$
    \item \emph{传递性}:因为 $\forall x, y, z \in \mathbb{R} \centerdot \big(\lfloor x \rfloor = \lfloor y \rfloor \land \lfloor y \rfloor = \lfloor z \rfloor \big) \implies \lfloor x \rfloor = \lfloor z \rfloor$
\end{itemize}

由于满足这三条性质的关系具有重要价值,我们赋予其专门名称。

\begin{definition}
    设 $A$ 为集合,$R$ 是 $A$ 上的关系。若 $R$ 满足自反性、对称性和传递性,则称 $R$ 为\dotuline{等价关系}。
\end{definition}

要判定集合 $A$ 上的关系 $R$ 是否为等价关系,只需验证这三条性质。下面回顾若干关系示例,通过已有证明判断它们是否是等价关系。

\begin{example}
    \begin{enumerate}[label=(\arabic*)]
        \item 回顾示例 \ref{ex:example6.2.9} 中定义的任意集合 $X$ 上的相等关系。这是一种等价关系。显然,$(x, x) \in R$,因为 $x=x$。然而,对于任意 $x \ne y$ 的情况,假设 $x \;R\; y$ 不成立,这反而使条件陈述成立。因此,对称性中唯一``相关情况''是 $x=y$,此时 $y=x$。同样地,在传递性中,如果 $x \ne y$ 或 $y \ne z$,定义条件陈述的假设不成立,所以陈述本身成立;而当 $x = y$ 且 $y = z$ 时,显然 $x = z$。此例虽简单,但表明任意集合上至少存在一个等价关系。
        \item $\mathbb{Z}$ 上的``整除''关系\textbf{不是}等价关系,因为它不满足对称性。(见示例 \ref{ex:example6.2.16})
        \item (非空)集合的``严格小于''关系\textbf{不是}等价关系,因为它不满足自反性。(见示例 \ref{ex:example6.2.17})
        \item 定义在 $\mathbb{Z}$ 上的关系 $R$
        \[\forall x, y \in \mathbb{Z} \centerdot x \;\bigstar\; y \iff 3 \mid x - y\]
        \textbf{是}等价关系,因为它满足自反性、对称性和传递性。(见 \ref{sec:section6.2.5} 节习题 \ref{exc:exercises6.2.2})此关系称为``模 $3$ 等价''关系,后文将深入讨论其推广形式。
    \end{enumerate}
\end{example}

本章习题常要求``判断给定关系是否为等价关系'',其思路类似于``证明或证伪命题'':若成立则证明三条性质,否则指出失效性质并构造反例。以下通过具体示例说明此方法。

\begin{example}
    设 $S=\mathbb{N}-\{1\}$ 并定义 $(x, y) \in R \iff x \text{\ 与\ } y \text{\ 有公因子}$(公因子不包括 $1$,要严格大于 $1$)。我们可以通过尝试证明这个关系的性质来确定它是否是等价关系。若证明成功,则它是等价关系;若证明失败,则可以利用失败点构造反例。

    首先,不难发现 $(x, x) \in R$ 成立,因为 $x$ 和 $x$ 有公因子 $x$,且根据 $S$ 的定义,我们知道 $x>1$,所以 $R$ 具有自反性。

    其次,若 $(x, y) \in R$,则 $x$ 与 $y$ 必然存在公因子 $k>1$。显然交换 $x$ 和 $y$ 不会改变这一事实:$y$ 与 $x$ 必然存在公因子 $k>1$,即 $(y,x) \in R$。因此 $R$ 具有对称性。

    最后,假设 $(x, y) \in R$ 且 $ (y,z) \in R$。这意味着 $x$ 和 $y$ 存在公因子 $k>1$ 且 $y$ 和 $z$ 存在公因子 $\ell>1$。我们可以据此找到 $x$ 和 $z$ 的公因子吗?不一定。我们无法确定 $k$ 和 $\ell$ 存在公因子。比如,如果 $k=2, \ell=3$,我们可以令 $x=2, y=6, z=9$,显然 $(2,6) \in R$ 且 $(6,9) \in R$,但 $(2,9) \notin R$。这个反例证明了该关系不具有传递性。因此它不是等价关系。
\end{example}

推荐使用此方法判断关系是否为等价关系或序关系:逐一检验自反性、对称性、传递性等性质并\textbf{尝试证明}。若所有性质得证,则该关系成立;若某性质证明失败,则可以通过分析失败原因来构造反例。

\subsubsection*{启下}

回顾本节给出的第一个例子,其中 $x \;R\; y \iff \lfloor x \rfloor = \lfloor y \rfloor$。注意,每个实数对应其\emph{向下取整}得到的整数,例如 $1.5 \;R\; 1$, $\pi \;R\; 3$, $-1.5 \;R\; -2$。此外,所有对应同一整数的实数彼此相关,例如 $3.5 \;R\; \pi$(因为二者均满足 $\lfloor x \rfloor=3$)。由此可将满足 $n \le x < n+1$ ($n \in \mathbb{Z}$) 的实数``划分''成一个``簇'',并用代表元素(如 $n$ 或 $n+\frac{1}{2}$)表示该簇。关键之处在于:同簇元素\emph{彼此相关},且每个簇都可用一个\textbf{代表}元素来表示。

这一观察将引导我们进入下一节,讨论如何正式描述这些``簇'',并引出核心概念\textbf{等价类}。我们将研究若干实例,并总结出一般性质。

在此之前,我们强烈建议你尝试一些我们已经讨论过的例子,寻找这些实例中的``簇''和``代表元素''。例如,考虑在 $\mathbb{Z}$ 上定义的关系 $R$:
\[\forall x, y \in \mathbb{Z} \centerdot x \;\bigstar\; y \iff 3 \mid x - y\]
这是一个等价关系。在这种情况下,``簇''是什么?你能识别出所有簇吗?每个簇中有多少元素?你能选择出代表元素吗?

试着用另一个等价关系做同样的事情,比如班级学生集合上的``出生月份相同''关系。经过思考,你会发现这是一个等价关系。

一个同样具有启发性的任务是考虑一个\textbf{非}等价关系,并试图弄清楚它\emph{为什么或如何}不具备这种``簇''性质。例如,考虑在 $\mathbb{Z}$ 上的``整除''关系。它在哪些方面不具备这种性质?它是否接近这种性质?

总之,尽可能多做一些探索!此类练习将有助于巩固对关系性质的理解,并为后续内容的学习奠定良好基础。
