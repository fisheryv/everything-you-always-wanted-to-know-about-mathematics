% !TeX root = ../../../book.tex
\section{本章习题}

本节习题涵盖本章全部内容,并涉及先前知识点及部分数学假设。我们不要求你解答\textbf{所有}题目,但解决得越多,收获越大!请牢记:真正\emph{掌握}数学必须亲自\emph{实践}。尝试动手解题,仔细阅读并思考题意。撰写证明并与朋友讨论,检验其说服力。持续练习如何清晰、准确、有条理地\emph{书写}思路。完成证明后要反复修改以臻完善。最重要的是,坚持\emph{钻研}数学!

标有 $\blacktriangleright$ 的简答题只需解释或陈述答案,无需严格证明。

特别具有挑战性的问题标记为 $\bigstar$。\\

\begin{exercise}
    $\blacktriangleright$ 考虑集合 $A = \{1, 2, 3, 4\}$。对于下列定义在 $A$ 或 $\mathcal{P}(A)$ 上的关系,判断其是否具有 (i)自反性,(ii)对称性,(iii)传递性, (iv)反对称性。
    \begin{enumerate}[label=(\alph*)]
        \item 定义在 $A$ 上的关系 $R_a= \{ (1, 2),(2, 2),(3, 1),(4, 2),(3, 3) \}$
        \item 定义在 $A$ 上的关系 $R_b= \{  (1, 1),(2, 2),(3, 3),(3, 4),(4, 3),(4, 4) \}$
        \item 定义在 $\mathcal{P}(A)$ 上的关系 $R_c= \forall S, T \in \mathcal{P}(A) \centerdot (S, T) \in R_c \iff S - T \subseteq \{1\}$
        \item 定义在 $\mathcal{P}(A)$ 上的关系 $R_d= \forall S, T \in \mathcal{P}(A) \centerdot (S, T) \in R_d \iff S \cap T \subseteq \{1\}$
    \end{enumerate}
\end{exercise}

\begin{exercise}
    定义 $\mathbb{R}$ 上的关系 $\sim$ 为
    \[\forall a, b \in \mathbb{R} \centerdot a \sim b \iff  \forall x \in \mathbb{R} \centerdot x > 0 \implies ax^2 + bx > 0\]
    对于关系的如下四个性质 —— (i)自反性,(ii)对称性,(iii)传递性, (iv)反对称性 —— 证明 $\sim$ 满足性质或找出反例证明 $\sim$ 不满足性质。
\end{exercise}

\begin{exercise}
    定义 $\mathcal{P}(\mathbb{R})$ 上的关系 $\approx$ 为
    \[\forall X, Y \in \mathcal{P}(\mathbb{R}) \centerdot X \approx Y \iff X - Y \subseteq \mathbb{N}\]
    对于关系的如下四个性质 —— (i)自反性,(ii)对称性,(iii)传递性, (iv)反对称性 —— 证明 $\approx$ 满足性质或找出反例证明 $\approx$ 不满足性质。
\end{exercise}

\begin{exercise}
    定义 $\mathbb{Z} \times \mathbb{N} - \{0\}$ 上的关系 $\#$ 为
    \[\forall (a, b),(c, d) \in \mathbb{Z} \times \mathbb{N} - \{0\} \centerdot (a, b) \# (c, d) \iff ad = bc\]
    \begin{enumerate}[label=(\alph*)]
        \item 证明 $\#$ 是等价关系。
        \item 找出等价类 $[(0, 3)]$ 中的元素,并证明你的结论。
        \item 找出等价类 $[(2, 3)]$ 中的元素,并证明你的结论。
        \item 找出等价类 $[(-2, 2)]$ 中的元素,并证明你的结论。
        \item $\mathbb{Z} \times \mathbb{N} - \{0\} / \#$ 有多少个等价类?
    \end{enumerate}
\end{exercise}

\begin{exercise}
    设 $p$ 为奇质数(即 $p \ne 2$)。证明 $p^2 \equiv 1 \mod 24$。
\end{exercise}

\begin{exercise}
    用欧几里得引理(见引理 \ref{lemma6.5.25})证明自然数的质因数分解是\textbf{唯一的}。\\
    (注意:我们在之前的示例 \ref{ex:example5.4.3} 中证明了质因数分解的\emph{存在性},但尚未证明\emph{唯一性}。)
\end{exercise}

\begin{exercise}
    定义 $\mathbb{R}$ 上的关系 $T$ 为
    \[\forall x, y \in \mathbb{R} \centerdot (x, y) \in T \iff \Big(\frac{y}{x} \in \mathbb{R} \land \frac{y}{x} \ge 0 \Big)\]
    \begin{enumerate}[label=(\alph*)]
        \item 对于任意 $x \in \mathbb{R}$,设集合 $S(x)$ 为
            \[S(x) = \{y \in \mathbb{R} \mid (x, y) \in T\}\]
            写出集合 $S(-1), S(0), S(1)$。
        \item 用 (a) 中的三个集合推导出 $T$ \textbf{不是}等价关系。
        \item 证明 $T$ 不具有自反性和对称性。
        \item $T$ 是否具有传递性?请证明你的结论。
    \end{enumerate}
\end{exercise}

\begin{exercise}
    分析以下错误证明。指出论证中哪里不正确,并给出一个\textbf{反例}说明结论为什么是错误的。
    \begin{quote}
        设 $n \in \mathbb{N}$ 且 $a,b,x \in \mathbb{Z}$。假设 $ax \equiv bx \mod n$。我们声称可以``消去''$x$ 得 $a \equiv b \mod n$。

        因为 $ax \equiv bx \mod n$,根据定义可得 $n \mid ax-bx$,故 $n \mid x(a-b)$,所以 $n \mid a-b$。因此根据定义可得 $a \equiv b \mod n$。
    \end{quote}
\end{exercise}

\begin{exercise}
    考虑如下同余方程组:
    \begin{align*}
        x \equiv 1 \mod 2\\
        x \equiv 2 \mod 6
    \end{align*}
    \textbf{中国剩余定理}能否保证解存在?你能否求出解?
\end{exercise}

\begin{exercise}
    在定义 \ref{def:definition6.5.29} 中,我们将两个整数的\textbf{最大公约数}定义为同时整除这两个整数的\emph{最大}整数。\\
    现在,我们希望你证明下面给出的 $\gcd$ 的定义与我们提供的原定义等价。\\
    \textbf{定义}:给定 $a, b \in \mathbb{Z}$。定义 $G(a,b)$ 为 $a$ 和 $b$ 的公因子,并且 $a$ 和 $b$ 的所有公因子都均整除 $G(a, b)$。即
    \[G(a, b) \mid a \land G(a, b)\]
    且
    \[\forall d \in \mathbb{Z} \centerdot (d \mid a \land d \mid b) \implies d \mid G(a,b)\]
    证明该定义与原定义等价,也就是证明
    \[\forall a, b \in \mathbb{Z} \centerdot \gcd(a, b) = G(a, b)\]
\end{exercise}

\begin{exercise}
    考虑如下(显然)错误的声明:

    \textbf{声明}:$1$ 是 $3$ 的倍数。

    下面给出的错误证明究竟错在哪里?

    \begin{quote}
        要想证明 $1 \equiv 0 \mod 3$。显然
        \begin{align*}
            1 \equiv 4 \mod 3& \implies 2^1 \equiv 2^4 \equiv 2 \equiv 16 \mod 3\\
            & \implies 2 \equiv 1 \mod 3\\
            & \implies 2 - 1 \equiv 1 - 1 \mod 3\\
            & \implies 1 \equiv 0 \mod 3
        \end{align*}
    \end{quote}
\end{exercise}

\begin{exercise}\label{exc:exercises6.7.12}
    通过证明 $M \subseteq L$ 完成贝祖恒等式(定理 \ref{theorem6.5.31})的证明。($M$ 和 $L$ 如定理描述定义。)
\end{exercise}

\begin{exercise}\label{exc:exercises6.7.13}
    在这个问题中,你需要证明定理 \ref{theorem6.4.12} 的逆命题。具体来说,你将证明以下\textbf{定理}:设集合 $S \ne \varnothing$,以及 $S$ 上的等价关系 $R$。则等价类集合 $S/R$ 构成 $S$ 的一个\textbf{划分}。\\
    请注意,符号 $[x]_R$ 表示 $x$ \textbf{对应的等价类},它是 $S$ 中所有与 $x$ 相关的元素的集合。即
    \[[x]_R = \{y \in S \mid (x, y) \in R\}\]
    假设 $R$ 是 $S$ 上的等价关系,因此 $R$ 具有自反性、对称性和传递性。
    \begin{enumerate}[label=(\alph*)]
        \item 对于任意 $x \in S$。证明 $x \in [x]_R$。
        \item 对于任意 $x, y \in S$。假设 $x \ne y$ 且 $(x, y) \in \mathbb{R}$。证明 $[x]_R = [y]_R$。\\
            (\textbf{提示}:使用\emph{两次}传递性。)
        \item 对于任意 $x, y \in S$。假设 $x \ne y$ 且 $(x, y) \notin \mathbb{R}$。证明 $[x]_R \cap [y]_R = \varnothing$。\\
            (\textbf{提示}:用反证法。)
        \item 说明为什么以上结论证明了该\textbf{定理}。
    \end{enumerate}
\end{exercise}

\begin{exercise}\label{exc:exercises6.7.14}
    在这个问题中,你需要证明引理 \ref{lemma6.5.2} 所述的除法算法。具体而言,你将证明
    \[\forall a, b \in \mathbb{Z} \centerdot \exists \not k, r \in \mathbb{Z} \centerdot ak + r = b \land 0 \le r \le a-1\]
    \begin{enumerate}
        \item 设 $M = {\ell \in \mathbb{Z} \mid \ell a \le b}$。证明 $M$ 有、存在\textbf{最大}元素。
        \item 设 $k \in M$ 为最大元素。定义 $r = b-ka$。证明 $0 \le r \le a-1$。
        \item 假设 $K,R \in \mathbb{Z}$ 同时满足 $aK + R = b$ 和 $ 0 \le R \le a-1$。证明 $K=k$ 且 $R=r$,从而证明 $k,r$ 是\emph{唯一的}。
    \end{enumerate}
\end{exercise}

\begin{exercise}\label{exc:exercises6.7.15}
    证明引理 \ref{lemma6.5.8}。即 $n \mid a - b \iff a \equiv b \mod n$
\end{exercise}

\begin{exercise}\label{exc:exercises6.7.16}
    证明引理 \ref{lemma6.5.9}。即证明模 $n$ 同余确实是 $\mathbb{Z}$ 上的等价关系。\\
    (\textbf{提示}:只需证明它具有自反性、对称性和传递性。)
\end{exercise}

\begin{exercise}
    本题要求你证明或证伪有关\textbf{毕达哥拉斯三元组}的陈述。毕达哥拉斯三元组指满足 $x^2 + y^2 = z^2$ 的整数三元组 $(x, y, z) \in \mathbb{Z}^3$。\\
    对于以下每条陈述,判断其是否\emph{必然}成立。若成立,请证明;若不成立,请给出反例。
    \begin{enumerate}[label=(\alph*)]
        \item ${x,y,z}$ 中至少有一个是偶数。
        \item ${x,y,z}$ 中至少有一个是 $3$ 的倍数。
        \item ${x,y,z}$ 中至少有一个是 $4$ 的倍数。
        \item ${x,y,z}$ 中至少有一个是 $5$ 的倍数。
    \end{enumerate}
\end{exercise}

\begin{exercise}
    陈述并证明判断自然数 $x \in \mathbb{N}$ 能否被 $11$ 整除的判定方法。\\
    (\textbf{提示}:参考示例 \ref{ex:example6.5.13} 中的类似问题。)
\end{exercise}

\begin{exercise}\label{exc:exercises6.7.19}
    \indent 很多``较小''质数都模 $4$ 余 $3$,例如 $3, 7, 11, 19, 23, 31 \equiv 3 \mod 4$。本题将证明此类质数有\emph{无穷多}个!

    (你可能会注意到,这里的步骤与证明质数有无穷多个的步骤非常相似!)
    \begin{enumerate}[label=(\alph*)]
        \item 设 $n \in \mathbb{N}$ 且 $n \equiv 3 \mod 4$。证明必然存在质数 $p$ 满足 $p \equiv 3 \mod 4$ 且 $p \mid n$。\\
        (\textbf{提示}:$3 \equiv -1 \mod 4$。)
        \item 为了引出矛盾而假设满足 $p \equiv 3 \mod 4$ 的质数只有\emph{有限多}个。我们定义这些特定质数组成的集合为 $P = \{p_1, p_2, \dots, p_k\}$,其中 $p_k$ 为最大的质数。\\
        定义新数 $N = p1 \cdot p2 \cdot p3 \cdot \dots \cdot p_k$。\\
        解释为什么 $N$ 必然为奇数,以及为什么 $N$ 严格大于集合 $P$ 中的所有质数。
        \item 定义 $M$ 为 $N$ 之后下一个模 $4$ 余 $3$ 的最大数。请解释为什么 $M - N$ 只能是 $2$ 或 $4$。
        \item 请解释为什么在 $M - N = 2$ 或 $M - N = 4$ 的情况下,这意味着 $N$ 的任意质因数都不会是 $M$ 的质因数。\\
        (\textbf{提示}:回想一下 $a \mid b \land a \mid c \implies a \mid (b \pm c)$。)
        \item 利用到目前为止所证明的内容解释为什么 $M$ 必为质数。
        \item 由此得出了什么矛盾?请给出最终结论。
    \end{enumerate}
\end{exercise}

\begin{exercise}
    模仿问题 \ref{exc:exercises6.7.19} 的步骤,证明存在无穷多个质数模 $6$ 余 $5$。
\end{exercise}

\begin{exercise}\label{exc:exercises6.7.21}
    在这个问题中,你将证明 MIRP 引理 \ref{lemma6.5.24} 的第二个结论。具体来说,你将证明以下命题:
    \begin{center}
        给定 $a \in \mathbb{Z}$ 和 $n \in \mathbb{N}$,假设 $a$ 和 $n$ \emph{不}互质。则不存在 $x \in \mathbb{Z}$ 使得 $ax \equiv 1 \mod n$。
    \end{center}
    \begin{enumerate}[label=(\alph*)]
        \item 假设 $a$ 和 $n$ 不互质。这意味着什么?
        \item 为了引出矛盾而假设存在 $x \in \mathbb{Z}$,使得 $ax \equiv 1 \mod n$,给定这样的 $x$。据此写出一个包含 $a,x,n$ 的\emph{方程}(不是\emph{同余式})。
        \item 利用 (a) 给出的信息重写此方程。
        \item 你发现了什么矛盾?
    \end{enumerate}
\end{exercise}

\begin{exercise}\label{exc:exercises6.7.22}
    对于以下每个陈述,判断其为\verb|真|还是为\verb|假|。若为\verb|真|,请证明它;若为\verb|假|,请给出一个反例。
    \begin{enumerate}[label=(\alph*)]
        \item $\forall x, y \in \mathbb{Z} \centerdot (x + y)^2 \equiv x^2 + y^2 \mod 2$
        \item $\forall x, y \in \mathbb{Z} \centerdot (x + y)^3 \equiv x^3 + y^3 \mod 3$
        \item $\forall x, y \in \mathbb{Z} \centerdot (x + y)^4 \equiv x^4 + y^4 \mod 4$
        \item $\forall x, y \in \mathbb{Z} \centerdot (x + y)^5 \equiv x^5 + y^5 \mod 5$
        \item $\forall x, y \in \mathbb{Z} \centerdot (x + y)^6 \equiv x^6 + y^6 \mod 6$
    \end{enumerate}
    \textbf{挑战性问题}:你能否对使下列陈述为\verb|真|的 $n$ 值提出一个猜想?
    \[\forall x, y \in \mathbb{Z} \centerdot (x + y)^n \equiv x^n + y^n \mod n\]
    你能\textbf{证明}这个猜想吗?你能描述使该陈述为\verb|假|的 $n$ 值吗?
\end{exercise}

\clearpage

\begin{exercise}
    判断下列方程是否存在整数解 $x, y \in \mathbb{Z}$
    \[3x^2 - 5y^2 = 2\]
    (\textbf{提示}:用乘法逆元和二次残差。)
\end{exercise}

\begin{exercise}
    证明下列方程无整数解 $x, y \in \mathbb{Z}$
    \[3x^2 - 5y^2 = 15\]
\end{exercise}

\begin{exercise}
    对于以下每个方程,求出所有整数解 $x, y \in \mathbb{Z}$,或者解释为什么不存在整数解。
    \begin{enumerate}[label=(\alph*)]
        \item $\enspace 2x + \enspace 4y = 9$
        \item $18x - 15y = 21$
        \item $\enspace 6x - 15y = 17$
        \item $\enspace 6x - 15y = 33$
    \end{enumerate}
\end{exercise}

\begin{exercise}\label{exc:exercises6.7.26}
    在这个问题中,你将通过\emph{归纳法}证明中国剩余定理(定理 \ref{theorem6.5.28})。然后,应用证明中提出的迭代方法求解一个特定的同余方程组。
    \begin{enumerate}[label=(\alph*)]
        \item 假设有两个同余式
        \begin{align*}
            x \equiv a_1 \mod n_1\\
            x \equiv a_2 \mod n_2
        \end{align*}
        其中 $a_1, a_2$ 互质。使用``模''运算的定义,根据上述同余式列出两个\textbf{方程}。通过代数方法合并方程,推导出\textbf{单一}模 $n_1n_2$ 的同余式。
        \item 利用 $n_1$ 和 $n_2$ 互质的假设,推导出 $n_2 - n_1$ 与 $n_1n_2$ 也互质。\\
        (\textbf{提示}:使用欧几里得引理 \ref{lemma6.5.25}。)
        \item 推导出形如 $X \equiv \underline{\qquad} \mod n_1n_2$ 的单一同余式。
        
            这已经证明了基本情况:两个同余式可以合二为一。
        \item 接下来证明归纳步骤:

            设 $r \in \mathbb{N} - \{1\}$,并且我们有 $r$ 个两两互质的自然数 $n_1, n_2, \dots , n_r \in \mathbb{N}$(即,任意两个数除了 $1$ 以外没有任何公因数);还有 $r$ 个整数 $a_1, a_2, \dots , a_r \in \mathbb{Z}$。

            我们通过对给定的同余数 $r$ 应用归纳法证明
            \[\exists X \in \mathbb{Z} \centerdot \forall i \in [r] \centerdot X \equiv a_i \mod n_i\]
            利用已证结论,将方程组重写为 $r-1$ 个同余式。
        \item 解释为什么这完成了中国剩余定理的归纳证明。
        \item 考虑以下同余方程组:
            \begin{align*}
                x \equiv 2 \mod 3 \\
                x \equiv 2 \mod 5 \\
                x \equiv 4 \mod 7
            \end{align*}
            应用上述证明的迭代方法求解该方程组。
    \end{enumerate}
\end{exercise}

\begin{exercise}\label{exc:exercises6.7.27}
    \indent 在这个问题中,你将通过\emph{构造性}方法证明中国剩余定理(定理 \ref{theorem6.5.28})。然后,应用构造性方法求解一个特定的同余方程组。

    设 $r \in \mathbb{N} - \{1\}$,并且我们有 $r$ 个两两互质的自然数 $n_1, n_2, \dots , n_r \in \mathbb{N}$ (即,任意两个数除了 $1$ 以外没有任何公因数);还有 $r$ 个整数 $a_1, a_2, \dots , a_r \in \mathbb{Z}$。

    我们要证明
    \[\exists X \in \mathbb{Z} \centerdot \forall i \in [r] \centerdot X \equiv a_i \mod n_i\]

    通过构造满足条件的 $X$ 并验证其满足所有同余关系完成证明。

    在整个问题中,我们使用定理陈述中定义的 $N$:
    \[N = \prod_{i \in [r]} n_i\]
    \begin{enumerate}[label=(\alph*)]
        \item 对于每个 $i \in [r]$,定义 $N_i = \frac{N}{n_i}$。解释为什么 $n_i$ 与 $N_i$ 互质。
        \item 引用一个结果,保证(对于每个 $i \in [r]$)存在整数 $y_i$,使得 $y_iN_i \equiv 1 \mod n_i$。
        \item 定义
            \[X = \sum_{j=1}^{r} a_jN_jy_j\]
            现在我们的目标是证明对于所有 $i \in [r], X \equiv a_i \mod n_i$。

            设 $i \in [r]$ 为任意固定元素。证明对于所有 $j \ne i$,$X$ 求和式中的对应项与 $a_i$ 模 $n_i$ 同余;即证明
            \[\forall j \in [r] \centerdot j \ne i \implies a_jN_jy_j \equiv 0 \mod n_i\]
        \item 对同一固定 $i$。证明当 $j = i$ 时,$X$ 求和式中的对应项与 $a_i$ 模 $n_i$ 同余;即证明
            \[a_iN_iy_i \equiv a_i \mod n_i\]
        \item 利用上述结果,解释 $X$ 如何满足\emph{所有} $r$ 个同余关系。
        
        \begin{center}
            额外证明\textbf{中国剩余定理}的第二个结论,即所有解模 $N$ 同余。
        \end{center}

        \item 考虑以下同余方程组
            \begin{align*}
                x \equiv 2 \mod 3 \\
                x \equiv 2 \mod 5 \\
                x \equiv 4 \mod 7
            \end{align*}
            其中 $n_1 = 3, n_2 = 5, n_3 = 7$ 且 $a_1 = 2, a_2 = 2, a_3 = 4$。根据上述步骤中的定义,对于每个 $i \in [3]$,计算 $N$, $N_i$ 以及 $y_i$,并据此求解 $X$。
        \item 利用\textbf{中国剩余定理}的第二个结论,用集合构建符写出该方程组的\emph{所有解},并求出\textbf{最小}自然数解。
    \end{enumerate}
\end{exercise}

\begin{exercise}
    \indent 印度数学家\textbf{婆罗摩笈多 (Brahmagupta)} 于公元七世纪提出以下谜题。(这表明人们已经思考这些问题数千年了!)

    阅读谜题,并根据故事描述建立对应的同余方程组,随后求解。

    (\textbf{提示}:建议使用迭代法,因为根据问题描述,中国剩余定理在此不适用。为什么不适用?能否调整第一步的处理方式使其适用?)

    \begin{quote}
        一妇市归,提篮贮卵。一男过而触之,篮倾卵堕,尽碎矣!

        男子揖曰:``歉甚!容吾复购偿卵。卵数几何?''

        妇人视地,唯见壳黄泥泞,不可计数。遂对曰:``数不能忆,然记其状:二数余一,三数余二,四数余三,五数余四,六数余五,七数适尽。唯忘七之组耳。''

        男子笑曰:``无妨,言已明矣,吾知卵数,如数补偿,并馈糕饵。''言讫,趋市。

        妇人立候,俄顷亦得卵数。

        问:卵数几何?
    \end{quote}
\end{exercise}

\clearpage

\begin{exercise}
    \textbf{挑战性问题}:等价关系研究
    \begin{enumerate}[label=(\alph*)]
        \item 设 $R$ 和 $S$ 为集合 $A$ 上的等价关系。假设 $A/R = A/S$(即每个等价关系下的等价类集合相同)。证明:$R = S$。
        \item 设 $R$ 和 $S$ 为集合 $A$ 上的等价关系。$R \cap S$ 一定是等价关系吗?
        \item 设 $R$ 和 $S$ 为集合 $A$ 上的等价关系。$R \cup S$ 一定是等价关系吗?
        \item 设 $R$ 和 $S$ 为集合 $A$ 上的等价关系。定义关系\emph{复合}为:
            \[S \circ R = \{(x, z) \in A \times A \mid \exists y \in A \centerdot (x, y) \in R \land (y, z) \in S\}\]
            $S \circ R$ 一定是等价关系吗?
        \item 设 $R$ 和 $S$ 是集合 $A$ 上的等价关系。$A/R$ 和 $A/S$ 为 $A$ 的\emph{划分}。
        
            我们称划分 $\mathcal{F}$ \textbf{细分}划分 $\mathcal{G}$ 当且仅当
            \[\forall X \in \mathcal{F} \centerdot \exists Y \in \mathcal{G} \centerdot X \subseteq Y\]
            证明
            \[R \subseteq S \iff A/R \text{\ 细分\ } A/S\]
    \end{enumerate}
\end{exercise}