% !TeX root = ../../../book.tex
\section{模算术}

你可能已经接触过一种自然而常见的等价关系,即整数上的\emph{同余}关系。这是对``奇偶性''等价关系的直接推广,它基于某个特定属性对整数进行分类。在此,我们将通过定义整数的某些性质来扩展这一概念,并引入多个等价类。我们还将探讨一些有趣的结论,这些结论利用同余关系使问题的证明变得更加简便(甚至成为可能!)。

% !TeX root = ../../../book.tex

\subsection{定义与示例}

\subsubsection*{整除性}

我们从熟悉的基本定义开始。

\begin{definition}
    设 $a,b \in \mathbb{Z}$。若存在整数 $k \in \mathbb{Z}$ 使得 $b = ak$(等价地,当 $a \neq 0$ 时 $\frac{b}{a} \in \mathbb{Z}$),则称 $a$ \dotuline{整除} $b$(或 $b$ 可以被 $a$ 整除)。记作 $a \mid b$。
\end{definition}

注意,根据定义,任意非零整数均可整除 $0$(例如 $5 \mid 0$),但 $0$ 仅能整除自身(例如 $0 \nmid 5$ 但 $0 \mid 0$)。请思考这与``整除''的直观理解是否一致。定义中 $k \in \mathbb{Z}$ 的约束允许负数参与运算,因此 $-2 \mid 4$ 和 $8 \mid -24$ 均成立。

诸如 $2 \nmid 5$ 这样的表述揭示了整数间的关系,但并未涵盖所有情况。虽然没有整数 $k$ 满足 $2k=5$,但 $k=2$ 和 $k=3$ 分别给出 $4$ 和 $6$,都非常接近 $5$;而 $k = -100$ 则偏差很远。对于较小数字通常易于验证,但当数字较大时,例如 $7 \nmid 100000$(由质数性质可知),如何找到使 $7k$ 最接近 $100000$ 的 $k$ 呢?答案是否唯一?是否存在多个``合理''选项(如 $2 \nmid 5$ 的情形)?

关于上面第二个问题,为了简化问题,我们需要限定答案,使其只有唯一合理选项。这样才能避免在得到一个答案后还要担心是否还有其他答案。为此,我们将借鉴\emph{价格猜猜猜 (The Price Is Right)}\footnote{《价格猜猜猜》是美国历史上播出时间最长的一档电视节目。—— 译者注}的规则:寻找\emph{最接近但不超过}目标值的答案。比如在 $2 \nmid 5$ 的例子中,我们认为 $k = 2$ 是最佳估计,因为 $4 < 5$。同理,在 $7 \nmid 100000$ 的例子中,我们认为 $k = 14285$ 是最佳估计,因为 $7 \times 14285 = 99995 < 100000$。(注意,在这种情况下,有一个``更接近''的估计,但它超过了目标值,因此我们不予考虑。)

这就引出了另一个问题——如何获得此类估计?对给定 $a,b \in \mathbb{Z}$,可以递增计算 $a$ 的倍数直至超过 $b$,其前一个倍数即为最优解。估计的``准确性''范围在 $0$ 到 $a - 1$ 之间,当 $a$ 能整除 $b$ 时,准确性为 $0$。(注意:``超过''是指顺序关系 $>$,因此负数情形需谨慎处理次序关系。如 $2 \nmid -3$ 时取 $k=-2$,因为 $-4 \le -3$。)下述引理总结了递增计算 $a$ 的倍数直到找到 $b$ 的最佳估计的思路,并阐明在限定条件下,总存在唯一解。

\begin{lemma}[除法算法]\label{lemma6.5.2}
    设 $a,b \in \mathbb{Z}$。$\exists k, r \in \mathbb{Z}$ 使得 $ak + r = b$,其中 $0 \le r \le a - 1$。换句话说,对于任意两个整数,总能找到一个 $a$ 的倍数 $k$,使得 $k$ 与 $a$ 的乘积最接近但不超过 $b$,同时还会得到唯一一个余数 $r$。我们把 $r$ 称为``$b$ 除以 $a$ 的余数''或``$b$ 被 $a$ 除的余数''。
\end{lemma}

余数概念至关重要,我们会频繁使用。后续将基于余数比较定义等价关系。首先请证明此引理!

\begin{proof}
    留作习题 \ref{exc:exercises6.7.14}。
\end{proof}

之所以称之为除法\emph{算法},是因为它提供了一种\emph{求解}倍数和余数的\emph{机制}。这种方法虽然简单但非常有效,简而言之就是\emph{反复应用减法}。也就是说,给定 $a$ 和 $b$,不断地从 $b$ 中减去 $a$,例如先得到 $b - a$,再得到 $b - 2a$,然后是 $b - 3a$……依此类推,直到剩下一个介于 $0$ 到 $a$ 之间的余数。

\begin{example}
    让我们通过一个例子来展示这一过程。假设 $a = 8, b = 62$。我们不断地从 $62$ 中减去 $8$,结果是:
    \[62, 54, 46, 38, 30, 22, 14, 6\]
    最终在 $6$ 上停止,因为它满足 $0 \le 6 < a = 8$,这表明余数 $r = 6$。我们还注意到,我们总共从 $b$ 中减去了 $7$ 次 $a$,因为列表中有 $8$ 项,其中第一项是 $b - 0 \cdot a$。故有:
    \[\underbrace{62}_{b} = \underbrace{7}_{k} \cdot \underbrace{8}_{a} + \underbrace{6}_{r}\]
\end{example}

此过程表明余数存在且唯一。利用该结果,可定义 $\mathbb{Z}$ 上的特定关系。下文将证这些关系均为等价关系,并探讨其等价类的强大应用!

这里的重点是有一种方法可以找到余数,并且余数唯一。利用该结果,我们可以定义 $\mathbb{Z}$ 上的特定关系。下文将证明这些关系均为等价关系,并探讨其\emph{等价类}的强大应用!

\subsubsection*{模 $n$ 同余}

\begin{definition}\label{def:definition6.5.4}
    设 $n \in \mathbb{N}$。定义 $\mathbb{Z}$ 上的关系 $R_n$ 为 $(a,b) \in R_n$ 当且仅当 $a$ 和 $b$ 除以 $n$ 的余数相同,即
    \[(a,b) \in R_n \iff n \mid a-b\] 
    亦可记作
    \[a \equiv b \mod n\]
    读作``$a$ 与 $b$ \dotuline{模 $n$ 同余}''。(口头上通常将 ``modulo'' 简化为 ``mod''。)
\end{definition}

\begin{remark}
    定义中``$a$ 和 $b$ 除以 $n$ 的余数相同''等价于 $n \mid a - b$。此结论需要证明,稍后将在习题 \ref{exc:exercises6.7.15} 中完成论证。
\end{remark}

\begin{remark}
    实际应用(如解题与证明)中,$a \equiv b \mod{n}$ 表明可将 $a$ 表示为 $n$ 的倍数与 $b$ 之和。
\end{remark}

我们来看一下为什么这是成立的。假设二者的余数均为 $r$,这意味着存在 $k, \ell \in \mathbb{Z}$ 使得
\[a = kn + r \quad\text{且}\quad b = \ell n + r\]
(余数相同,但 $n$ 的倍数可能不同。)通过相减消去 $r$ 即可得到等式
\[a - kn = b - \ell n\]
然后移项并提取公因式可得
\[a = (k - \ell)n + b\]
此处 $(k - \ell)n$ 是 $n$ 的倍数,而第二项只包含 $b$ 本身。这说明 $a$ 是 $n$ 的倍数加上 $b$。

通常情况下,$b$ 可能并不是 $a$ 除以 $n$ 后的余数;特别是当 $b$ 不满足余数要求 $0 \le r \le a - 1$ 时,就会出现这种情况。

让我们总结一下此观点,并给出后续证明与示例中常用的\emph{模 $n$ 同余}的等价表述:

\setlength{\fboxrule}{2pt}
\begin{center}
\fcolorbox{olivegreen}{white}{%
    \parbox{0.8\textwidth}{%
        \[a \equiv b \mod n \iff \exists m \in \mathbb{Z} \centerdot a=mn+b\]
    }
}
\end{center}

\begin{example}
    让我们通过考察几个较小的 $n$ 值,来考察这些关系的具体表现。

    \begin{itemize}
        \item 设 $n=1$。关系 $R_1$ 会是什么样?这个问题实际上有些无聊,因为任何整数除以 $1$ 的余数都是 $0$,所以每个整数都可以和其他任意整数相关联。也就是说,$\forall x,y \in \mathbb{Z} \centerdot (x, y) \in R_1$。由于这个关系相对平凡,因此数学家们几乎不会讨论``模 $1$''这个话题。
        
        \item 设 $n=2$。关系 $R_2$ 就是我们之前定义的``奇偶关系''。想想为什么会这样。当我们把任意整数 $a$ 除以 $2$ 时,余数只能是 $0$ 或 $1$。如果 $a$ 和 $b$ 除以 $2$ 的余数都是 $0$,那么它们都是偶数;如果余数都是 $1$,那么它们都是奇数。(回想一下我们在第 \ref{ch:chapter03} 章中的定义,\emph{奇数}和\emph{偶数}是通过\emph{存在}声明来定义的。例如,当且仅当 $\exists k \in \mathbb{Z}$ 使得 $x = 2k$ 时,$x$ 为偶数。这正是除法算法的结果。当且仅当 $x$ 除以 $2$ 的余数为 $0$ 时,$x$ 为偶数,因为我们可以找到一个整数 $k$,使得 $x = 2k$。)\\
        现在,想想同余的另一种表述。如果两个整数都是偶数,那么它们的差也是偶数!也就是说,$a \equiv b \mod 2 \iff a - b \mid 2$;即 $a$ 和 $b$ 都是偶数(或者都是奇数)当且仅当它们的差也是偶数。(注意:我们还没有\emph{证明}这种表述确实等价于余数的定义。我们将在本例之后给出证明。)

        \item 设 $n=3$。例如,$0 \equiv 9 \mod 3, -1 \equiv 2 \mod 3$ 以及 $4 \equiv 28 \mod 3$。一般来说,只要在行尾加上``$\text{mod\ } 3$''(或其他数字),便可以连接多个同余语句。如此连接后,整行都按照模 $3$ 处理。例如,以下语句在符号上是有效的,在数学上也是成立的:
        \[-100 \equiv -1 \equiv 8 \equiv 311 \equiv -289 \equiv 41 \mod 3\]
        (虽然我们不确定为什么需要写出这样的陈述,但这样做是完全可以的!)

        \item 设 $n=10$。自然数除以 $10$ 的余数就是它的最后一位数字,即个位数字!这样我们就可以轻松地比较两个数模 $10$ 的余数。例如,$12 \equiv 32 \equiv 448237402 \mod 10$;而 $37457 \not\equiv 38201 \mod 10$。\\
        但对于\emph{负数}的情况会略有不同。因为我们定义余数时,是取\emph{不超过}目标值的最大倍数。例如,$-1 \equiv 9 \mod 10$,这是因为 $-1= (-1) \cdot 10 + 9$,而 $9 = (0) \cdot 10 + 9$。它们的余数都是 $9$,需要加到某个 $10$ 的倍数上。请思考以下陈述的具体细节:
        \[-3 \equiv 17 \equiv -33 \equiv 107 \mod 10\]
    \end{itemize}
\end{example}

\subsubsection*{符号}

需要强调的是,在数学中,\textbf{mod} 是一种关系,而非运算符或函数。在计算机科学和编程中,你可能会看到类似``\verb|5 mod 3 = 2|''这样的表达,它表示``$5$ 除以 $3$ 的余数是 $2$''。(在许多编程语言中,写作 \verb|5 % 3 = 2|)。但数学中不采用这种写法:我们使用 $\text{mod}$ 和 $\equiv$ 符号表示一种\textbf{等价关系},因为讨论的数字不一定\emph{相等}。当等价关系在某个自然数 $n$ 下成立时,我们在行末标注``$\text{mod\ } n$''以指明这一点。此时,$\text{mod}$ 相当于一个\emph{修饰符},表示``本行所有陈述仅对模 $n$ 的余数成立''。例如:
\[100 \equiv 97 \equiv 16 \equiv 4 \equiv z \cdot w \equiv 1 \equiv x - y \equiv -2 \equiv -8 \mod 3\]
这表示在 $\mod 3$ 下,所有数字和表达式等价。我们既不断言它们绝对相等,也不断言其在其他模数下等价;行末的``$\mod 3$''表示``仅在整数模 $3$ 的范围内讨论''。

(问题:你能找到 $x, y, z, w \in \mathbb{Z}$ 使上述等式成立吗?)

\subsubsection*{三个重要引理}

本节要求你证明两个关键结论:第一,模 $n$ 同余可以用\emph{可除性}等价定义;第二,该关系是等价关系。阅读时请完成对应的练习。掌握这些细节后,下一节关于等价类的内容会更容易理解。完成证明后,我们将展示并证明另一引理,并在讨论等价类之前给出一个同余的典型应用:它能简化手工计算繁琐的算术问题。

\begin{lemma}\label{lemma6.5.8}
    在定义 \ref{def:definition6.5.4} 中,模 $n$ 同余的两种表述等价。即对于所有 $a, b \in \mathbb{Z}$ 和 $n \in \mathbb{N}$,
    \[a, b \text{\ 除以\ } n \text{\ 余数相同\ } \iff n \mid a - b\]
\end{lemma}

\begin{proof}
    见习题 \ref{exc:exercises6.7.15}
\end{proof}

\begin{lemma}\label{lemma6.5.9}
    对于任意 $n \in \mathbb{N}, R_n$ 是 $\mathbb{Z}$ 上的等价关系。
\end{lemma}

\begin{proof}
    见习题 \ref{exc:exercises6.7.16}
\end{proof}

感谢你证明了这些引理!$\smiley{}$ 现在我们知道模 $n$ 同余是等价关系(因此可以讨论等价类),且判断两个整数(如 $a$ 和 $b$)模 $n$ 同余等价于验证 $a - b$ 是否是 $n$ 的\emph{倍数},这是一种高效的判定方法。

下一引理表明:``模 $n$''意义下的加法和乘法\textbf{保持同余性}。若有两个整数等式,如 $a + b = c$ 和 $d + e = f$,相加可得 $a + b + d + e = c + f$。此引理说明该原理对模 $n$ 同余同样成立——同余式可相加或相乘,同余关系仍然成立。

尽管该引理的证明并不复杂,但我们将替你完成,因为本节你已经做了太多工作了。

\begin{lemma}[模算术引理 (Modular Arithmetic Lemma, 简称 MAL)]\label{lemma6.5.10}
    设 $n \in \mathbb{N}$。设 $a,b,r,s \in \mathbb{Z}$ 为任意固定整数。若 $a \equiv r \mod n$ 且 $b \equiv s \mod n$,则
    \begin{align*}
        a + b &\equiv r + s \mod n \\
        a \cdot b &\equiv r \cdot s \mod n
    \end{align*}
\end{lemma}

(该引理表明只需处理余数:无论给定的 $a$ 和 $b$ 是什么,均可简化为余数 $r$ 和 $s$ 后进行运算。由于 $0 \le r, s \le n - 1$,其值比 $a$ 和 $b$ 更小,因此可以加速实际运算。以下证明确保该方法普遍成立。)

\begin{proof}
    假设 $a \equiv r \mod n$ 且 $b \equiv s \mod n$。这意味着 $\exists k, \ell \in \mathbb{Z}$ 使得
    \begin{align*}
        a &= kn + r \\
        b &= \ell n + s
    \end{align*}
    \begin{itemize}
        \item 两式相加得
            \[a + b = (kn + r) + (\ell n + s) = (k + \ell)n + (r + s)\]
            因为可以将 $a+b$ 表示为 $n$ 的倍数加上余数 $r+s$,所以 $a + b \equiv r + s \mod n$。
        \item 两式相乘得
            \[a \cdot b =  (kn + r) \cdot (\ell n + s) = k\ell n^2 + (ks + \ell r)n + r \cdot s = n \cdot (k\ell n + ks + \ell r)+r \cdot s\]
            因为可以将 $a \cdot b$ 表示为 $n$ 的倍数加上余数 $r \cdot s$,所以 $a \cdot b \equiv r \cdot s \mod n$。
    \end{itemize}
\end{proof}

\begin{remark}
    请注意,此处仅讨论加法和乘法,未涉及\textbf{减法}和\textbf{除法},原因有二:

    其一,减法本质是``加负数''。该引理表明,对同余作减法可通过两步实现:
    \begin{enumerate}[label=(\arabic*)]
        \item 将其中一个同余乘以 $-1$(应用\emph{乘法}引理)
        \item 将结果相加(应用\emph{加法}引理)
    \end{enumerate}
    此过程巧妙地结合了两个引理的结果。

    其二较为复杂。实际上,模 $n$ 运算中无``除法''运算。主要原因在于讨论范围仅限于\emph{整数},而除法可能产生非整数的\emph{有理数}。例如 $4 \equiv 7 \mod 3$,但 $\frac{4}{2} \equiv \frac{7}{2} \mod 3$ 无意义——整数(如 $2$)不能与非整数(如 $\frac{7}{2}$)同余。因此在 $\mathbb{Z}$ 模 $n$ 系统中不定义\textbf{除法}。

    ``除法''问题存在更精细的讨论,将在 \ref{sec:section6.5.3} 节讨论\emph{乘法逆元}时详述。为了避免混淆,此处暂不展开。简而言之,后续将发展一种在特定条件下类似模 $n$ 除法的方法。

    综上,为了保持\emph{整数}范畴,本书仅讨论加法与乘法。
\end{remark}

\subsubsection*{两个实用例子}

我们可能尚未完全说服你认识到模算术的妙用。为了展示同余作为等价关系兼具数学趣味与实用价值,这里将展示两个典型示例。第一个例子中,模算术提供了比常规方法更简洁的解法;第二个则是你可能用过却未曾深究的巧妙技巧——我们将证明其有效性。

\begin{example}
    思考以下问题:

    是否\emph{存在}自然数 $k$,使得 $5^k$ 恰好比 $7$ 的倍数多 $1$?如果存在,这样的自然数最小是多少?能否描述所有满足条件的自然数?

    我们可以尝试代入不同的 $k$ 值来寻找答案,但很快会发现:大指数的计算十分繁琐,而验证一个大数是否满足条件更为困难。如果你愿意的话可以自行尝试,甚至借助计算器探索解法。

    不过,我们更倾向反复使用模算术引理 (Modular Arithmetic Lemma, MAL)。由于指数的本质是重复乘法,可以反复运用乘法引理。核心思路是:在连续乘以 $5$ 的过程中,始终保持对 $7$ 取模。我们只需寻找模 $7$ 余 $1$ 的数,无需直接计算其具体值。过程如下:

    从 $5^1 \equiv 5 \mod 7$ 开始。将其乘以 $5$ 得
    \[5^2 \equiv 5 \cdot 5 \equiv 25 \equiv 4 \mod 7\]
    注意到 $25 = 21 + 4$,并且已知 $21$ 是 $7$ 的倍数,从而得出上述结论。(当数字较小时,我们常常可以通过简单观察直接心算。当然,如果不确定,也可以使用除法算法,从 $25$ 中不断减去 $7$,直到剩下余数。)

    继续乘以 $5$ 得
    \[5^3 \equiv 5^2 \cdot 5 \equiv 4 \cdot 5 \equiv 20 \equiv 6 \mod 7\]
    我们发现,通过``观察''可知 $20 = 14+6$。请注意,求 $5^3$ 模 $7$ 的余数并不需要实际计算 $5^3 = 125$ 然后再取模。因为我们在此过程中已经将所有数字都化简到模 $7$ 的余数,所以省去了大量计算。具体来说,我们总是将数字化简到\emph{小于} $7$ 的范围内,因此在任何情况下我们需要处理的最大数字也只会在 $20$ 到 $30$ 之间。这真是太方便了!让我们继续看接下来会得到什么结果:
    \begin{align*}
        5^4 &\equiv 5^3 \cdot 5 \equiv 6 \cdot 5 \equiv 30 \equiv 2 \mod 7 \\ 
        5^5 &\equiv 5^4 \cdot 5 \equiv 2 \cdot 5 \equiv 10 \equiv 3 \mod 7 \\
        5^6 &\equiv 5^5 \cdot 5 \equiv 3 \cdot 5 \equiv 15 \equiv 1 \mod 7 
    \end{align*}
    这正是我们要找的结果!我们已经确定 $5^6$ 比 $7$ 的倍数多 $1$。这种方法比直接计算 $5^6 = 15625$ 并找出 $15625 = 7 \cdot 2232 + 1$ 要简单得多。

    至此解答了前两问:存在满足条件的 $5$ 的幂,且由 $k=1$ 逐步推导可知 $k=6$ 为最小值。第三问留作思考:请继续乘以 $5$ 并观察余数规律。你能否发现模式?提出猜想并尝试证明!(后续将回归此例……)
\end{example}

\begin{example}\label{ex:example6.5.13}
    考虑数字 $474$。它是 $3$ 的倍数吗?只需将各位数字相加:$4 + 7 + 4 = 15$。由于 $15$ 是 $3$ 的倍数,可知 $474$ 也必然是 $3$ 的倍数。(当然,也可通过长除法验证 $474 = 3 \cdot 158$。)但为什么这种方法成立?难道仅仅是因为老师曾在三年级时告诉了你这个方法,你就记住了?这对我们来说远远不够!$\smiley{}$

    这里,我们将严格\textbf{证明}:自然数 $x$ 能被 $3$ 整除当且仅当其各位数字之和能被 $3$ 整除。(证明中括号内的例子仅用于辅助理解。需要注意的是,具体实例不能替代普遍结论的证明。)

    \begin{proof}
        设 $x \in \mathbb{N}$ 为任意固定自然数。其十进制展开形式为
        \[x= \sum_{k=0}^{n-1} x_k \cdot 10^k\]
        其中 $n$ 为数字 $x$ 的位数,$x_k$ 为 $10^k$ 位对应的数字,所以 $0 \le x_k \le 9$。(即 $x_k$ 是从右向左第 $(k+1)$ 位数字。)

        (例如,$47205$ 可写做 $47205=4 \cdot 10^4+7 \cdot 10^3+2 \cdot 10^2+0 \cdot 10^1+5 \cdot 10^0$。本例中,$x_0 = 5, x_1=0, x_3=2$。)

        该整除技巧声称
        \[x \equiv 0 \mod 3 \iff \sum_{k=1}^{n-1} x_k \equiv 0 \mod 3\]

        为了证明这一点,我们将考虑十进制展开模 $3$。注意,由于 $10=9+1$,因此 $10 \equiv 1 \mod 3$。故
        \[\forall k \in \mathbb{N} \cup \{0\} \centerdot 10^k \equiv 1^k \equiv 1 \mod 3\]

        (此结论基于模算术引理和 $1^k = 1$ 对任意 $k$ 恒成立。请思考一下!)

        由此可将十进制展开式中的 $10^k$ 替换为 $1$:

        \begin{align*}
            x \equiv 0 \mod 3 &\iff \sum_{k=0}^{n-1} x_k \cdot 10^k \equiv 0 \mod 3 &\text{将\ } x \text{\ 重写为十进制展开形式}\\
            &\iff \sum_{k=0}^{n-1} x_k \cdot 1^k \equiv 0 \mod 3 &\text{因为\ } 10 \equiv 1 \mod 3 \\
            &\iff \sum_{k=0}^{n-1} x_k \equiv 0 \mod 3
        \end{align*}

        证毕。
    \end{proof}

    (注意,$3 \mid 47205$ 成立是因为 $3 \mid (4 + 7 + 2 + 0 + 5)$,也就是说 $3 \mid 18$。实际上,$15735 \cdot 3 = 47205$)。

    值得注意的是,此证明揭示了\textbf{更强的结论}:由于上述陈述为\emph{当且仅当}陈述,若 $x$ 的各位数字之和不是 $3$ 的倍数,则 $x$ 也不是 $3$ 的倍数,且两者模 $3$ 同余。例如 $3 \nmid 122$,因为 $3 \nmid 5$ 且 $5 \equiv 2 \mod 3$,因此 $122 \equiv 2 \mod 3$。(验证可得 $122 = 3 \cdot 40 + 2$。)

    类似规则对 $9$ 和 $11$ 同样成立($11$ 的规则略微复杂)。甚至存在 $7$ 的整除规则,但表述较为繁琐。本章习题将进一步探讨这些内容。

    请熟记该结论及其证明。这是一个可以在聚会上炫耀的小技巧。你可以向朋友发起挑战:他们真的知道\textbf{为什么}该技巧有效吗?你却能洞悉其本质!
\end{example}


% !TeX root = ../../../book.tex

\subsection{模 $n$ 等价类}

你已经证明了(引理 \ref{lemma6.5.9})模 $n$ 同余是 $\mathbb{Z}$ 上的等价关系,并证明了(定理 \ref{theorem6.4.10})等价关系将集合\emph{划分}为等价类。结合这两点,可知模 $n$ 同余将 $\mathbb{Z}$ 划分为若干等价类。如何表示这些等价类?每个类的代表元素应该如何选择?

我们从两个更简单的问题入手:
\begin{enumerate}[label=(\arabic*)]
    \item $\mathbb{Z}$ 模 $n$ 有多少个等价类?
    \item 这些等价类的``大小''如何?
\end{enumerate}

\subsubsection*{等价类的数量}

要回答问题 (1),回忆除以 $n$ 的余数定义。除法算法(引理 \ref{lemma6.5.2})表明余数 $r$ 满足 $0 \le r \le n-1$,故余数有 $n$ 种可能:$0,1,2,\dots,n-1$(即 $r \in {0,1,\dots,n-1}$)。这些余数均可取到,例如 $n-1$ 除以 $n$ 的余数为 $n-1$(因为 $n-1 < n$)。因此,$\mathbb{Z}$ 模 $n$ 的等价类恰有 $n$ 个。

基于相同的观察,可以确定等价类的\emph{代表元素}。由于 $a \equiv b \mod n$ 表示 $a$ 和 $b$ 除以 $n$ 有相同的余数,故二者属于由该余数 $r$(满足 $0 \le r \le n-1$)代表的等价类,记为 $a, b \in [r]_{\text{\ mod\ } n}$(下标``$\text{mod\ } n$''表明余数基于 $n$)。

\subsubsection*{等价类的大小}

让我们通过一个具体的例子来思考这个问题,例如 $n = 4$。整数 $z \in \mathbb{Z}$ 属于余数 $0$ 的等价类 $[0]{\text{\ mod\ } 4}$ 时,意味着 $z$ 除以 $4$ 余数为 $0$,即 $z$ 是 $4$ 的\emph{倍数}。$\mathbb{Z}$ 中 $4$ 的倍数有无穷多个,如 $0,4,8,12,\dots$ 和 $-4,-8,-12,\dots$,故 $[0]{\text{\ mod\ } 4}$ 是\emph{无限集}。

让我们通过一个具体的例子来思考这个问题,例如 $n = 4$。对于一个整数 $z \in \mathbb{Z}$ 属于余数为 $0$ 的等价类,这意味着什么?也就是说,如果我们知道 $z \in [0]_{\text{\ mod\ } 4}$,我们可以得出哪些关于 $z$ 的结论?

类似地,$z \in [1]{\text{\ mod\ } 4}$ 表示余数为 $1$,即\emph{存在}整数 $k$,使得 $z = 4k + 1$。取 $k = 0,1,2,\dots$ 及 $k = -1,-2,\dots$ 时,可得
\begin{align*}
    [1]_{\text{\ mod\ } 4} &= \{\dots , -7, -3, 1, 5, 9, \dots \} \\
    &= \{z \in \mathbb{Z} \mid \exists k \in \mathbb{Z} \centerdot z = 4k + 1\} \\
    &= \{4k + 1 \mid k \in \mathbb{Z}\}
\end{align*}
请注意,此处先用省略号展示模式,再用两种集合表示法重述。

此集合同样是无限集。对其他余数(模 $4$ 或任意整数 $n$),等价类均为无限集(我们尚未\emph{正式}定义无限集,但直观上可以理解为无限集拥有无穷多元素,我们可以列出其所有元素,并找到一个生成所有元素的模式,但此过程在有限时间内无法结束)。

\subsubsection*{$\mathbb{Z}$ 模 $n$ 的划分}

基于对等价类的观察,我们可以总结出 $\mathbb{Z}$ 模 $n$ 的等价类的标准表示。已知有 $n$ 个等价类,每个等价类包含无穷多个元素。每个等价类对应于整数除以 $n$ 的余数。由于余数必须满足 $0 \le r \le n-1$,我们将集合 $\{0, 1, 2, \dots , n-1\} = [n-1] \cup \{0\}$ 作为标准代表集合。

余数为 $r$ 的等价类包含所有除以 $n$ 余 $r$ 的整数。换句话说,所有 $z \in [r]_{\text{\ mod\ } n}$ 的元素都是 $n$ 的某个倍数加上 $r$。也就是说,我们可以通过从 $r$ 开始不断加上或减去 $n$ 来生成等价类的所有元素。这样,同一等价类中的任意两个元素相差 $n$ 的倍数。

\setlength{\fboxrule}{2pt}
\setlength\fboxsep{5mm}
\begin{center}
\noindent \fcolorbox{blue}{white}{%
    \parbox{0.85\textwidth}{%
        \linespread{1.5}\selectfont
        \textcolor{blue}{\textbf{$\mathbb{Z}$ 模 $n$ 等价类:}}\\
        给定 $n \in \mathbb{N}$,恰好存在 $n$ 个等价类
        \[[0]_{\text{\ mod\ } n}, \; [1]_{\text{\ mod\ } n}, \; [2]_{\text{\ mod\ } n}, \; \dots , \; [n-1]_{\text{\ mod\ } n}\]
        其特点是:
        \begin{align*}
            [0]_{\text{\ mod\ } n} &=  \{\dots, -2n, -n, 0, n, 2n, \dots \} \\
            &= \{z \in \mathbb{Z} \mid \exists k \in \mathbb{Z} \centerdot z = kn\}\\
            [1]_{\text{\ mod\ } n} &=  \{\dots, -2n+1, -n+1, 1, n+1, 2n+1, \dots \} \\
            &= \{z \in \mathbb{Z} \mid \exists k \in \mathbb{Z} \centerdot z = kn+1\}\\
            [2]_{\text{\ mod\ } n} &=  \{\dots, -2n+2, -n+2, 2, n+2, 2n+2, \dots \} \\
            &= \{z \in \mathbb{Z} \mid \exists k \in \mathbb{Z} \centerdot z = kn+2\}\\
            &\vdots \\
            [n-1]_{\text{\ mod\ } n} &=  \{\dots, -n-1, -1, n-1, 2n-1, 3n-1, \dots \} \\
            &= \{z \in \mathbb{Z} \mid \exists k \in \mathbb{Z} \centerdot z = kn+(n-1)\}\\
            &= \{z \in \mathbb{Z} \mid \exists \ell \in \mathbb{Z} \centerdot z = \ell n-1\}
        \end{align*}
    }
}
\end{center}

以上是所有观察结果的总结。下面是一些具体 $n$ 值的例子。

\begin{itemize}
    \item 考虑 $n=2$。等价类为:
        \begin{align*}
            [0]_{\text{\ mod\ } 2} &= \{z \in \mathbb{Z} \mid \exists k \in \mathbb{Z} \centerdot z = 2k\}\\ 
            &=  \{\text{偶数}\}\\
            &= \{\dots, -6, -4, -2, 0, 2, 4, 6 \dots \}\\
            [1]_{\text{\ mod\ } 2} &= \{z \in \mathbb{Z} \mid \exists k \in \mathbb{Z} \centerdot z = 2k+1\}\\ 
            &=  \{\text{奇数}\}\\
            &= \{\dots, -5, -3, -1, 1, 3, 5, 7 \dots \}
        \end{align*}
    \item 考虑 $n=3$。等价类为:
        \begin{align*}
            [0]_{\text{\ mod\ } 3} &= \{z \in \mathbb{Z} \mid \exists k \in \mathbb{Z} \centerdot z = 3k\}\\
            &=  \{3 \text{\ 的倍数}\}\\
            &= \{\dots, -9, -6, -3, 0, 3, 6, 9 \dots \}\\
            [1]_{\text{\ mod\ } 3} &= \{z \in \mathbb{Z} \mid \exists k \in \mathbb{Z} \centerdot z = 3k+1\}\\ 
            &=  \{3 \text{\ 的倍数加\ } 1\}\\
            &= \{\dots, -8, -5, -2, 1, 4, 7, 10 \dots \}\\
            [2]_{\text{\ mod\ } 3} &= \{z \in \mathbb{Z} \mid \exists k \in \mathbb{Z} \centerdot z = 3k+2\}\\ 
            &=  \{3 \text{\ 的倍数加\ } 2\}\\
            &= \{\dots, -7, -4, -1, 2, 5, 8, 11 \dots \}
        \end{align*}
    \item 考虑 $n=4$。等价类为:
        \begin{align*}
            [0]_{\text{\ mod\ } 4} &= \{z \in \mathbb{Z} \mid \exists k \in \mathbb{Z} \centerdot z = 4k\}\\ 
            &=  \{4 \text{\ 的倍数}\}\\
            &= \{\dots, -12, -8, -4, 0, 4, 8, 12 \dots \}\\
            [1]_{\text{\ mod\ } 4} &= \{z \in \mathbb{Z} \mid \exists k \in \mathbb{Z} \centerdot z = 4k+1\}\\
            &=  \{4 \text{\ 的倍数加\ } 1\}\\
            &= \{\dots, -11, -7, -3, 1, 5, 9, 13 \dots \}\\
            [2]_{\text{\ mod\ } 4} &= \{z \in \mathbb{Z} \mid \exists k \in \mathbb{Z} \centerdot z = 4k+2\}\\ 
            &=  \{4 \text{\ 的倍数加\ } 2\}\\
            &= \{\dots, -10, -6, -2, 2, 6, 10, 14 \dots \}\\
            [3]_{\text{\ mod\ } 4} &= \{z \in \mathbb{Z} \mid \exists k \in \mathbb{Z} \centerdot z = 4k+3\}\\ 
            &=  \{4 \text{\ 的倍数加\ } 3\}\\
            &= \{\dots, -9, -5, -1, 3, 7, 11, 15 \dots \}
        \end{align*}
\end{itemize}

\subsubsection*{使用等价类}

为什么这很有用?为什么我们要介绍整数模 $n$ 的等价类的构造?

$\mathbb{Z}$ 被这些等价类划分这一点至关重要。因此,在模 $n$ 的背景下进行算术运算时,只需考虑这些等价类,即余数。我们可以将所有整数简化为数字 $0, 1, 2, \dots, n-1$,因为它们代表了所有整数。这样,我们就无需进行大数运算后再求余数;只需处理这些余数即可。让我们通过具体例子来考察这种划分的实际应用。

\begin{example}
    考虑以下命题:
    \[\forall n \in \mathbb{N} \centerdot 6 \mid n^3 + 5n\]
    我们之前让你通过对 $n$ 进行归纳来证明该命题(参见 \ref{sec:section5.7} 节的练习 \ref{exc:exercises5.7.15})。现在,我们将使用等价类来证明它!

    考虑 $\mathbb{Z}$ 模 $6$。因为 $\mathbb{N} \subseteq \mathbb{Z}$,根据除以 $6$ 的余数,我们知道每个 $n \in \mathbb{N}$ 必然落在等价类 $[0]_{\text{\ mod\ } 6}, \;[1]_{\text{\ mod\ } 6}, \;[2]_{\text{\ mod\ } 6}, \;[3]_{\text{\ mod\ } 6}, \;[4]_{\text{\ mod\ } 6}, \;[5]_{\text{\ mod\ } 6}$ 中的\textbf{一个}。

    我们可以分别检查每种情况。假设 $n$ 属于某个特定的等价类,然后计算 $n^3 + 5n$ 所在的等价类。在每种情况下,通过乘法(包括幂运算,即重复乘法)和加法,应用模算术引理 \ref{lemma6.5.10}。
    \begin{align*}
        n \equiv 0 \mod 6 &\implies n^3 + 5n \equiv 0^3 + 5 \cdot 0 \equiv 0 \mod 6 \\
        n \equiv 1 \mod 6 &\implies n^3 + 5n \equiv 1^3 + 5 \cdot 1 \equiv 6 \equiv 0 \mod 6 \\
        n \equiv 2 \mod 6 &\implies n^3 + 5n \equiv 2^3 + 5 \cdot 2 \equiv 18 \equiv 0 \mod 6 \\
        n \equiv 3 \mod 6 &\implies n^3 + 5n \equiv 3^3 + 5 \cdot 3 \equiv 42 \equiv 0 \mod 6 \\
        n \equiv 4 \mod 6 &\implies n^3 + 5n \equiv 4^3 + 5 \cdot 4 \equiv 84 \equiv 0 \mod 6 \\
        n \equiv 5 \mod 6 &\implies n^3 + 5n \equiv 5^3 + 5 \cdot 5 \equiv 150 \equiv 0 \mod 6 
    \end{align*}
    以上每种情况下,$n^3 + 5n$ 均为 $6$ 的倍数(因为除以 $6$ 的余数为 $0$)。这表明,无论 $n$ 取何值,$n^3 + 5n$ 总是 $6$ 的倍数。这证明了对于所有 $n \in \mathbb{N}$,该命题成立,从而避免了使用归纳论证!
\end{example}

\begin{example}[二次残差 (Quadratic Residues)]\label{ex:example6.5.15}

    本例将研究完全平方数的性质,具体来说,我们探讨完全平方数被不同除数相除时产生的余数规律。这一研究颇具趣味性,因为读者将发现余数会随除数变化呈现独特模式(若你因此产生深入探索的兴趣,那再好不过!)。此外,该研究还能引导出若干重要结论,这些结论在正文和习题中均有证明。特别地,理解完全平方数对探究\textbf{毕达哥拉斯三元组}——即满足 $a^2 + b^2 = c^2$ 的整数三元组 $(a, b, c) \in \mathbb{N}^3$——具有关键作用,其性质可以帮助我们证明关于此类三元组的若干有趣定理。

    对于以下每种情况,固定 $n \in \mathbb{N}$,研究所有 $x \in \mathbb{Z}$ 下 $x^2$ 模 $n$ 的结果。根据 $\mathbb{Z}$ 模 $n$ 的划分,只需考察 $n$ 个可能的模 $n$ 余数,然后平方取模。这些可能的余数称为\textbf{二次残差}(\emph{二次}源于平方运算,\emph{残差}即指余数)。下文将分类总结不同 $n$ 值对应的二次残差集。

    \begin{itemize}
        \item $\mathbf{n=2}$:\\
        已知只有当底数为偶数时,完全平方数才是偶数;只有当底数为奇数时,完全平方数才是奇数。在第 \ref{ch:chapter04} 章中,我们曾通过讨论双向条件陈述、量词和证明技巧来验证这些结论。现在无需重新正式证明这些结论,可以直接运用模运算性质轻松验证这些结论。\\
        设 $x \in \mathbb{Z}$ 为任意固定整数。
        \begin{itemize}
            \item 首先,假设 $x \equiv 0 \mod 2$ (即 $x$ 为偶数)。则应用模算术引理可得 $x^2 \equiv 0 \mod 2$ (即 $x^2$ 为偶数)。
            \item 其次,假设 $x \equiv 1 \mod 2$ (即 $x$ 为奇数)。则应用模算术引理可得 $x^2 \equiv 1 \mod 2$ (即 $x^2$ 为奇数)。
        \end{itemize}
        根据 $\mathbb{Z}$ 模 $2$ 的划分,以上两种情况已经完备。
        \begin{quotation}
            \begin{center}
                \large 模 $2$ 二次残差:$\{0, 1\}$
            \end{center}
        \end{quotation}

        \item $\mathbf{n=3}$:\\
        设 $x \in \mathbb{Z}$ 为任意固定整数。应用模算术引理可得:
        \begin{itemize}
            \item $x \equiv 0 \mod 3 \implies x^2 \equiv 0^2 \equiv 0 \mod 3$
            \item $x \equiv 1 \mod 3 \implies x^2 \equiv 1^2 \equiv 1 \mod 3$
            \item $x \equiv 2 \mod 3 \implies x^2 \equiv 2^2 \equiv 4 \equiv 1 \mod 3$
        \end{itemize}
        \begin{quotation}
            \begin{center}
                \large 模 $3$ 二次残差:$\{0, 1\}$
            \end{center}
        \end{quotation}

        \item $\mathbf{n=4}$:\\
        设 $x \in \mathbb{Z}$ 为任意固定整数。应用模算术引理可得:
        \begin{itemize}
            \item $x \equiv 0 \mod 4 \implies x^2 \equiv 0^2 \equiv 0 \mod 4$
            \item $x \equiv 1 \mod 4 \implies x^2 \equiv 1^2 \equiv 1 \mod 4$
            \item $x \equiv 2 \mod 4 \implies x^2 \equiv 2^2 \equiv 4 \equiv 0 \mod 4$
            \item $x \equiv 3 \mod 4 \implies x^2 \equiv 3^2 \equiv 9 \equiv 1 \mod 4$
        \end{itemize}
        \begin{quotation}
            \begin{center}
                \large 模 $4$ 二次残差:$\{0, 1\}$
            \end{center}
        \end{quotation}

        \item $\mathbf{n=5}$:\\
        设 $x \in \mathbb{Z}$ 为任意固定整数。应用模算术引理可得:
        \begin{itemize}
            \item $x \equiv 0 \mod 5 \implies x^2 \equiv 0^2 \equiv 0 \mod 5$
            \item $x \equiv 1 \mod 5 \implies x^2 \equiv 1^2 \equiv 1 \mod 5$
            \item $x \equiv 2 \mod 5 \implies x^2 \equiv 2^2 \equiv 4 \mod 5$
            \item $x \equiv 3 \mod 5 \implies x^2 \equiv 3^2 \equiv 9 \equiv 4 \mod 5$
            \item $x \equiv 4 \mod 5 \implies x^2 \equiv 4^2 \equiv 16 \equiv 1 \mod 5$
        \end{itemize}
        \begin{quotation}
            \begin{center}
                \large 模 $5$ 二次残差:$\{0, 1, 4\}$
            \end{center}
        \end{quotation}

        \item $\mathbf{n=6}$:\\
        设 $x \in \mathbb{Z}$ 为任意固定整数。应用模算术引理可得:
        \begin{itemize}
            \item $x \equiv 0 \mod 6 \implies x^2 \equiv 0^2 \equiv 0 \mod 6$
            \item $x \equiv 1 \mod 6 \implies x^2 \equiv 1^2 \equiv 1 \mod 6$
            \item $x \equiv 2 \mod 6 \implies x^2 \equiv 2^2 \equiv 4 \mod 6$
            \item $x \equiv 3 \mod 6 \implies x^2 \equiv 3^2 \equiv 9 \equiv 3 \mod 6$
            \item $x \equiv 4 \mod 6 \implies x^2 \equiv 4^2 \equiv 16 \equiv 4 \mod 6$
            \item $x \equiv 5 \mod 6 \implies x^2 \equiv 5^2 \equiv 25 \equiv 1 \mod 6$
        \end{itemize}
        \begin{quotation}
            \begin{center}
                \large 模 $6$ 二次残差:$\{0, 1, 3, 4\}$
            \end{center}
        \end{quotation}

        \item $\mathbf{n=7}$:\\
        设 $x \in \mathbb{Z}$ 为任意固定整数。应用模算术引理可得:
        \begin{itemize}
            \item $x \equiv 0 \mod 7 \implies x^2 \equiv 0^2 \equiv 0 \mod 7$
            \item $x \equiv 1 \mod 7 \implies x^2 \equiv 1^2 \equiv 1 \mod 7$
            \item $x \equiv 2 \mod 7 \implies x^2 \equiv 2^2 \equiv 4 \mod 7$
            \item $x \equiv 3 \mod 7 \implies x^2 \equiv 3^2 \equiv 9 \equiv 2 \mod 7$
            \item $x \equiv 4 \mod 7 \implies x^2 \equiv 4^2 \equiv 16 \equiv 2 \mod 7$
            \item $x \equiv 5 \mod 7 \implies x^2 \equiv 5^2 \equiv 25 \equiv 4 \mod 7$
            \item $x \equiv 6 \mod 7 \implies x^2 \equiv 6^2 \equiv 36 \equiv 1 \mod 7$
        \end{itemize}
        \begin{quotation}
            \begin{center}
                \large 模 $7$ 二次残差:$\{0, 1, 2, 4\}$
            \end{center}
        \end{quotation} 

        \item $\mathbf{n=8}$:\\
        设 $x \in \mathbb{Z}$ 为任意固定整数。应用模算术引理可得:
        \begin{itemize}
            \item $x \equiv 0 \mod 8 \implies x^2 \equiv 0^2 \equiv 0 \mod 8$
            \item $x \equiv 1 \mod 8 \implies x^2 \equiv 1^2 \equiv 1 \mod 8$
            \item $x \equiv 2 \mod 8 \implies x^2 \equiv 2^2 \equiv 4 \mod 8$
            \item $x \equiv 3 \mod 8 \implies x^2 \equiv 3^2 \equiv 9 \equiv 1 \mod 8$
            \item $x \equiv 4 \mod 8 \implies x^2 \equiv 4^2 \equiv 16 \equiv 0 \mod 8$
            \item $x \equiv 5 \mod 8 \implies x^2 \equiv 5^2 \equiv 25 \equiv 1 \mod 8$
            \item $x \equiv 6 \mod 8 \implies x^2 \equiv 6^2 \equiv 36 \equiv 4 \mod 8$
            \item $x \equiv 7 \mod 8 \implies x^2 \equiv 7^2 \equiv 49 \equiv 1 \mod 8$
        \end{itemize}
        \begin{quotation}
            \begin{center}
                \large 模 $8$ 二次残差:$\{0, 1, 4\}$
            \end{center}
        \end{quotation}
    \end{itemize}
    
    我们鼓励你继续探究其他二次残差模式。你甚至可以尝试编写一个计算机程序生成残差列表,并思考以下问题:给定 $n \in \mathbb{N}$,模 $n$ 的二次残差数量如何确定?其具体形式如何?是否存在必然出现或永不出现的残差?期待你的探索发现!
\end{example}

\begin{example}
    让我们推广前例中的思想,考察特定情况下\emph{立方残差 (cubic residues)}的性质。
    \begin{quotation}
        假设 $x, y, z \in \mathbb{Z}$ 满足 $x^3+y^3=z^3$。

        证明 $\{x, y, z\}$ 中至少有一个是 $7$ 的倍数。
    \end{quotation}

    重申一下我们的目标,我们要证明:
    \[(x \equiv 0 \mod 7) \lor (y \equiv 0 \mod 7) \lor (z \equiv 0 \mod 7)\]

    为此,列出模 $7$ 的所有立方残差。设 $x \in \mathbb{Z}$ 为任意固定整数,应用模算术引理可得:
    \begin{itemize}
        \item $x \equiv 0 \mod 7 \implies x^3 \equiv 0^3 \equiv 0 \mod 7$
        \item $x \equiv 1 \mod 7 \implies x^3 \equiv 1^3 \equiv 1 \mod 7$
        \item $x \equiv 2 \mod 7 \implies x^3 \equiv 2^3 \equiv 8 \equiv 1 \mod 7$
        \item $x \equiv 3 \mod 7 \implies x^3 \equiv 3^3 \equiv 9 \cdot 3 \equiv 2 \cdot 3 \equiv 6 \mod 7$
        \item $x \equiv 4 \mod 7 \implies x^3 \equiv 4^3 \equiv 16 \cdot 4 \equiv 2 \cdot 4 \equiv 8 \equiv 1 \mod 7$
        \item $x \equiv 5 \mod 7 \implies x^3 \equiv 5^3 \equiv 25 \cdot 5 \equiv 4 \cdot 5 \equiv 20 \equiv 6 \mod 7$
        \item $x \equiv 6 \mod 7 \implies x^3 \equiv 6^3 \equiv (-1)^3 \equiv -1 \equiv 6 \mod 7$
    \end{itemize}

    (注意,为了简化计算,我们将 $6$ 写成 $-1$,然后再模 $7$。)

    我们发现唯一的可能值是 $\{0, 1, 6\}$。

    现在,假设存在解 $x, y, z \in \mathbb{Z}$ 满足 $x^3 + y^3 = z^3$。此时 $x^3,y^3,z^3$ 模 $7$ 均同余于 $0$, $1$ 或 $6$。考虑以下情形:
    \begin{itemize}
        \item 假设 $x^3 \equiv 0 \mod 7$。则 $y^3$ 可以与 $0$, $1$ 或 $6$ 模 $7$ 同余,此时只需要让 $z^3$ 落在相同的等价类即可。总之,在此情况下,都有 $z^3 \equiv 0 \mod 7$。
        
        \item 假设 $y^3 \equiv 0 \mod 7$。将上面的论证应用于 $x^3$ 和 $z^3$。总之,在此情况下,都有 $y^3 \equiv 0 \mod 7$。
        
        \item 假设 $x^3 \equiv 1 \mod 7$。\\
            为了引出矛盾而假设 $y^3 \equiv 1 \mod 7$。则 $x^3+y^3 \equiv 1+1 \equiv 2 \mod 7$,然而 $2$ 不在模 $7$ 的立方残差中,因此这是不可能的。\\
            然而我们发现 $y^3 \equiv 0 \mod 7$ 是可能的,因为 $x^3+y^3 \equiv 1+0 \equiv 1 \mod 7$。\\
            同时我们发现 $y^3 \equiv 6 \mod 7$ 是可能的,因为 $x^3+y^3 \equiv 1+6 \equiv 7 \equiv 0 \mod 7$。\\
            总之,在此情况下,\emph{至少}有一个立方数 —— 要么是 $y^3$,要么是 $z^3$ —— 与 $0$ 模 $7$ 同余。

        \item 假设 $y^3 \equiv 1 \mod 7$。将上面的论证应用于 $x^3$ 和 $z^3$。在此情况下,至少有一个立方数 与 $0$ 模 $7$ 同余。
        
        \item 假设 $x^3 \equiv 6 \mod 7$。\\
            为了引出矛盾而假设 $y^3 \equiv 6 \mod 7$。则 $x^3+y^3 \equiv 6+6 \equiv 12 \equiv 5 \mod 7$,然而 $5$ 不在模 $7$ 的立方残差中,因此这是不可能的。\\
            然而我们发现 $y^3 \equiv 0 \mod 7$ 是可能的,因为 $x^3+y^3 \equiv 6+0 \equiv 6 \mod 7$。\\
            同时我们发现 $y^3 \equiv 1 \mod 7$ 是可能的,因为 $x^3+y^3 \equiv 6+1 \equiv 7 \equiv 0 \mod 7$。\\
            总之,在此情况下,\emph{至少}有一个立方数 —— 要么是 $y^3$ 要么是 $z^3$ —— 与 $0$ 模 $7$ 同余。

        \item 假设 $y^3 \equiv 6 \mod 7$。将上面的论证应用于 $x^3$ 和 $z^3$。在此情况下,至少有一个立方数 与 $0$ 模 $7$ 同余。
    \end{itemize}

    综上,无论何种情形,\textbf{至少}有一个立方项与 $0$ 模 $7$ 同余。具体哪个立方项具有此性质取决于具体情况(可能有多个立方数符合),但总有至少一个成立。

    这很有用,因为回顾立方残差列表会发现一个关键性质:立方项与 $0$ 模 $7$ 同余当且仅当其底数与 $0$ 模 $7$ 同余。即:
    \[\forall z \in \mathbb{Z} \centerdot z^3 \equiv 0 \mod 7 \implies z \equiv 0 \mod 7\]
    这意味着,在上述每种情形中,至少有一个立方项与 $0$ 模 $7$ 同余,这进一步说明至少有一个底数与 $0$ 模 $7$ 同余。通过穷举所有可能情况,我们证明了该方程\emph{所有可能解}的普适性质,而无需构造具体解!
\end{example}

现在,尽管所有工作都已经完成,但我们有一个不幸的消息:原方程\emph{唯一}的解是\emph{平凡}解,即 $x = y = z = 0$。正是如此!你可以尝试寻找其他解,但终归徒劳。这一结果是\textbf{费马大定理}的一个特例,该定理指出,对于方程 $x^k + y^k = z^k$(其中 $k \in \mathbb{N}$),只有当 $k = 1$ 或 $k = 2$ 时,才存在非平凡整数解(即 $x, y, z \in \mathbb{Z}$);换言之,当 $k \in \mathbb{N} \setminus \{1, 2\}$ 时,唯一的解是 $x = y = z = 0$。

费马生前曾提及这一结论,但从未发表证明。他在笔记本的页边空白处声称自己有一个简洁的证明,只是空间不足而未能写下。然而,我们现在知道这很可能并非事实。费马生活在 17 世纪,但这个定理直到 20 世纪 90 年代才被证明\footnote{安德鲁·怀尔斯 (Andrew Wiles) 于 1994 年证明了费马大定理。—— 译者注}!而且,证明过程用到了大量费马时代之后才逐步发展起来的高深数学工具。

如果我们知晓这个定理,便能轻松证明本例中的结论!既然唯一解是 $x = y = z = 0$,那么显然这些值均为 $7$ 的倍数。然而,这种做法既无趣味,也无法让我们练习模算术和等价类。

\begin{example}
    这是另一个涉及立方残差的问题:
    \begin{quotation}
        假设 $x, y, z \in \mathbb{Z}$ 满足 $x^3+y^3+z^3=3$。

        证明 $x^3 \equiv y^3 \equiv z^3 \mod 9$。
    \end{quotation}

    这里讨论的是一个特定的\emph{丢番图方程 (Diophantine Equation)}。丢番图方程是指含有多个变量且系数为整数的多项式方程。求解这类方程需要找到一组整数解,使方程成立。本例中,我们要证明方程的任意解都必须满足 $x^3, y^3, z^3$ 模 $9$ 同余。

    首先,尝试找出该方程的若干解以观察具体实例。以下提供几个简单例子作为参考:例如 $(x, y, z)$ 可取 $(1, 1, 1)$ 或 $(4, 4, -5)$。这些解是否满足我们要求的性质?你还能找到其他解吗?(此问题较难,不必投入过多精力。)

    有趣的是,我们无需确定所有解的具体形式或实际求出它们,即可证明此结论。只需考察模 $9$ 下的立方残差,设 $x \in \mathbb{Z}$ 为任意固定整数,应用模算术引理可得:
    \begin{itemize}
        \item $x \equiv 0 \mod 9 \implies x^3 \equiv 0^3 \equiv 0 \mod 9$
        \item $x \equiv 1 \mod 9 \implies x^3 \equiv 1^3 \equiv 1 \mod 9$
        \item $x \equiv 2 \mod 9 \implies x^3 \equiv 2^3 \equiv 8 \mod 9$
        \item $x \equiv 3 \mod 9 \implies x^3 \equiv 3^3 \equiv 9 \cdot 3 0 \mod 9$
        \item $x \equiv 4 \mod 9 \implies x^3 \equiv 4^3 \equiv 16 \cdot 4 \equiv (-2) \cdot 4 \equiv -8 \equiv 1 \mod 9$
        \item $x \equiv 5 \mod 9 \implies x^3 \equiv 5^3 \equiv 25 \cdot 5 \equiv (-2) \cdot 5 \equiv -10 \equiv 8 \mod 9$
        \item $x \equiv 6 \mod 9 \implies x^3 \equiv 6^3 \equiv 36 \cdot 6 \equiv 0 \cdot 6 \equiv 0 \mod 9$
        \item $x \equiv 7 \mod 9 \implies x^3 \equiv 7^3 \equiv 49 \cdot 7 \equiv 4 \cdot (-2) \equiv -8 \equiv 1 \mod 9$
        \item $x \equiv 8 \mod 9 \implies x^3 \equiv 8^3 \equiv (-1)^3 \equiv -1 \equiv 8 \mod 9$
    \end{itemize}
    注意,在某些情况下使用负数可简化计算。这是完全可行的,且对你大有裨益!例如,计算 $4^3 = 64$ 再模 $9$ 时,可以 $-2$ 替代 $16$ 以保持数值较小。我们可随时加减 $9$ 的倍数,因此在计算过程中直接处理,而非先得大数再取模。(当然,$64$ 并不算大,因此这一点不够明显;但处理更大数字时,此技巧非常实用。此外,将数字尽量简化至个位数可减少心算错误!)注意,最右侧仅出现三种可能结果:模 $9$ 的立方残差为 $\{0, 1, 8\}$。仅此而已!

    当然,要使 $x^3 + y^3 + z^3 = 3$ 成立,必须满足 $x^3 + y^3 + z^3 \equiv 3 \mod{9}$,因为 $3 \equiv 3 \mod{9}$。观察可能的立方残差——$0, 1, 8$——我们发现\emph{仅有} $1 + 1 + 1$ 模 $9$ 余 $3$。其他组合如 $0 + 1 + 8 \equiv 9 \equiv 0 \mod{9}$ 和 $8 + 8 + 8 \equiv 24 \equiv 6 \mod{9}$ 等均不符合。这意味着解 $(x, y, z)$ 必须满足 $x^3 \equiv y^3 \equiv z^3 \equiv 1 \mod{9}$。

    由此,我们证明了一个稍强的结论:不仅 $x^3, y^3, z^3$ 模 $9$ 同余,它们还必须模 $9$ 余 $1$。这比原要求更进一步。

    事实上,此问题存在一个\emph{更强}的结论:$x \equiv y \equiv z \mod{9}$。换言之,不仅它们的\emph{立方}模 $9$ 同余,其\emph{底数}本身也模 $9$ 同余。(注意,这并非指底数模 $9$ 余 $1$;例如 $(4, 4, -5)$ 就表明情况并非如此。)遗憾的是,证明这一点需要涉及很多高等数学,超出了本书的范围。但这应该能让你理解,这些``简单''的问题(表述简洁、数值较小、纯粹整数)的解决往往需要复杂而深奥的数学工具。不过,请不要将其视为难点,而应视为启发:仅用少量数学知识,我们便能触及问题表层,其下隐藏着更为深刻而复杂的根基。

    如果你感兴趣,可以参考以下论文获取完整结论:
    \begin{center}
        \href{http://www.ams.org/journals/mcom/1985-44-169/S0025-5718-1985-0771049-4/S0025-5718-1985-0771049-4.pdf}{J. W. S. Cassels, ``A Note on the Diophantine Equation'', \\Mathematics of Computation, 44(169): 265-266, 1985.}
    \end{center}
    
    它证明了 $x \equiv y \equiv z \pmod{9}$ 的必要性。然而,即便是阅读前两段,你也需要查阅若干定义。通读全文更是要求学习相关数学知识,耗时可能数月乃至数年,取决于你的兴趣。请牢记这一点,并在未来的数学之旅中重温此问题!
\end{example}


% !TeX root = ../../../book.tex

\subsection{乘法逆元}\label{sec:section6.5.3}

我们之前在证明模算数引理(引理 \ref{lemma6.5.10})时提到过,不会在 $\mathbb{Z}$ 模 $n$ 的背景下讨论``除法''。本节中,我们将重新探讨这个想法,并解释为什么(以及如何)在某些特定情况下``除法''是合理的。然而,我们要强调的是,我们实际上在讨论一个更广泛的\textbf{乘法逆元}的概念,而\textbf{不是}真正的``除法''。我们将首先通过几个启发性例子来解释这一点,然后我们将陈述并证明这些特定情况下的具体结果。

\subsubsection*{整体概念}

给定一个特定的数学对象,它的\textbf{乘法逆元}是另一个对象,当我们将这两个对象``相乘''时,结果为 ``$1$''。这里我们加上引号是因为``相乘''和 ``$1$'' 的含义在不同的语境中可能会有很大差异。\\

\begin{example}
    让我们先考虑一个熟悉的例子。假设我们讨论的是实数集 $\mathbb{R}$,且使用通常的乘法运算。现在我们取数字 $2$。它的乘法逆元是什么?也就是说,是否存在另一个实数 $x$ 使得 $2 \cdot x = 1$?如果存在,它是多少?很明显,$x = \frac{1}{2}$ 是符合要求的!可以注意到 $2 \cdot \frac{1}{2} = 1$。出于这个原因,我们可以写出
    \[2^{-1} = \frac{1}{2} \quad \text{在 }\; \mathbb{R} \;\text{范围内}\]
    当我们把方程两边同时除以 $2$ 时,实际上是在把方程的两边都\emph{乘以} $2$ 的\emph{乘法逆元}。
\end{example}

\begin{example}
    现在让我们考虑一个可能不太熟悉的例子。想象一个挂钟,钟面上有均匀分布的 $12$ 个小时刻度标记。我们将考虑旋转挂钟,所以我们声明标准位置,即顶部 $12$ 点的位置为 ``$1$''。也就是说,这是没有进行额外旋转的标准表示法,所以我们称其为\emph{单位元}。实际上,我们的 ``$1$'' 就是指 ``$0 \degree$ 旋转'' 后的挂钟。

    现在,让我们假设将两个旋转``相乘''只是依次进行旋转。例如,我们先将时钟顺时针旋转 $45 \degree$,然后再顺时针旋转 $90 \degree$。在这个例子中,我们实际上是将 ``$45 \degree$ 旋转'' 和 ``$90 \degree$ 旋转'' \emph{乘}在一起,结果得到 ``$135 \degree$ 旋转''。

    建立这些约定的目的是为了明确我们的上下文、对象、``相乘''的含义以及``1''的定义,从而能够识别任意旋转的\emph{乘法逆元}。如果你仔细想一下,就会发现如果我们将 ``$\theta$ 度旋转'' 与 ``$360-\theta$ 度旋转'' 相乘,那么我们实际上是将时钟旋转了 $360 \degree$,并回到了标准位置,这正是我们在这个上下文中的 ``$1$''。这意味着,在我们当前上下文中
    \[(\theta \;\text{度旋转})^-1 = 360-\theta \;\text{度旋转}\]
\end{example}

这两个例子旨在说明,\emph{逆元}的概念是一个普遍的概念,并不局限于数字\emph{除法}这种标准上下文。事实上,当我们讨论\emph{函数的逆}时,也会看到类似的例子。(在这个上下文中,``相乘''指的是函数的复合,``$1$'' 是指恒等函数。虽然具体内容会在下一章详细介绍,但我们现在提到这一点,是为了让已经熟悉这些概念的读者有一个预先的理解。)

\subsubsection*{互质}

你可能已经熟悉以下定义。我们将在后续的结果中用到它,这个结果将说明在 $\mathbb{Z}$ 模 $n$ 的情况下何时存在乘法逆元,因此我们现在想重申这个定义并展示一些例子。

\begin{definition}
    给定 $x,y \in \mathbb{Z}$,我们说 $x$ 和 $y$ \dotuline{互质}当且仅当它们没有除 $1$ 以外的公因子。
\end{definition}

(\textbf{注意}:``互质'' 表示 $x$ 和 $y$ 彼此互质,并不是说 $x$ ``类似质数''或其他意思。)\\

\begin{example}
    例如,$12$ 和 $35$ 互质,因为 $12 = 2^2 \cdot 3$,而 $35 = 5 \cdot 7$,因此它们没有任何公因数。

    通常,写出\emph{质因数分解}是有帮助的,因为我们实际上是想知道两个数是否有共同的质因数(这意味着它们有公因数)。

    举个反例,$12$ 和 $33$ 不互质,因为 $3 \mid 12$ 且 $3 \mid 33$。
\end{example}

\begin{example}
    这个例子陈述的结果在后面会很有用。

    \textbf{声明}:如果 $p$ 是质数且 $a$ 不是 $p$ 的倍数,那么 $p$ 和 $a$ 互质。

    (也就是说,如果 $p$ 是质数且 $p \nmid a$,则 $p$ 和 $a$ 互质。)

    让我们来看看为什么这是正确的!

    \begin{proof}
        设 $p$ 为质数且 $a \in \mathbb{Z}$。假设 $p \nmid a$。

        由于 $p \nmid a$,所以 $a$ 的质因数中不包含 $p$。因为 $p$ 是质数,因此 $a$ 的质因数也不会整除 $p$。这意味着 $a$ 和 $p$ 没有共同的质因数,因此它们互质。
    \end{proof}

    这非常方便!特别是,我们现在知道,只要 $p$ 是质数,那么\textbf{所有}数字 $1, 2, 3, \dots , p-1$ 都与 $p$ 互质。
\end{example}

\subsubsection*{定义与示例}

我们来讨论一下在 $\mathbb{Z}$ 模 $n$ 的情况下,什么是\emph{乘法逆元}。这里的``乘法''指的是常规乘法,但所有结果都要模 $n$。另外,``$1$'' 实际上是对应于 $1$ 的\emph{等价类}。在这种情况下,我们说对于任意 $x \in \mathbb{Z}$,当且仅当 $xy \equiv 1 \mod n$ 时,$x$ 的乘法逆元(记作 $x^{-1}$)等于 $y$。也就是说:
\[\forall x \in \mathbb{Z} \centerdot \forall y \in \mathbb{Z} \centerdot y \equiv x^{-1} \mod n \iff xy \equiv 1 \mod n\]
请注意,这些声明都是在 $\mathbb{Z}$ 模 $n$ 的情况下进行的,因此我们不会写 ``$y = x^{-1}$''。数字 $x$ 表示一个等价类,$x^{-1}$ 也是如此。

让我们来练习一下如何\emph{找到}这些乘法逆元,或者判断它们何时不存在。关键在于以下几点:
\begin{quotation}
    如果 $x \cdot y \equiv 1 \mod n$,则对于所有 $k \in \mathbb{Z}, x \cdot (y+kn) \equiv 1 \mod n$
\end{quotation}
要想理解其中的原因,我们可以将右边表达式中的 $x$ 利用分配律展开:
\[x \cdot (y+kn) \equiv xy+xkn \equiv xy+n(xk) \equiv xy+0 \equiv xy \equiv 1 \mod n\]
也就是说,在展开过程中,给 $y$ 加上 $n$ 的倍数只会得到 $n$ 的倍数,而我们在模 $n$ 时可以``忽略''这些倍数。

由此我们可以得出以下结论:\textbf{如果} $x$ 在模 $n$ 下有一个乘法逆元,\textbf{那么} 
\begin{enumerate}[label=(\alph*)]
    \item 存在\emph{无穷多}个这样的逆元,并且它们都属于同一个模 $n$ 的等价类;
    \item 但是在集合 $\{1, 2, 3, \dots , n-1\}$ 中,我们可以找到\emph{唯一一个}这样的逆元。
\end{enumerate}

这些事实非常有趣且实用。特别是,它告诉我们无需进行复杂的存在性论证来寻找乘法逆元:只需逐一检查每种情况,直到找到一个。如果找不到,就说明不存在。换句话说,我们不必凭直觉臆断或随意猜测,而是有一个更为系统的猜测和检查算法。

让我们通过下面的例子来看看它在实际中的应用。\\

\begin{example}
    在这个例子中,我们会给出一个 $n \in \mathbb{N}$ 和一个 $x \in \mathbb{Z}$,然后寻找一个满足 $y \equiv x^{-1} \mod n$ 的 $y$。如果这样的逆元不存在,我们将说明原因。
    \begin{itemize}
        \item $\mathbf{n=3, x=2}$:\\
            我们只需检查 $y = 1$ 和 $y = 2$。注意到 $2 \cdot 2 \equiv 4 \equiv 1 \mod 3$,所以
            \[2^{-1} \equiv 2 \mod 3\]
        \item $\mathbf{n=4, x=3}$:\\
            我们只需检查 $y = 1, y = 2, y = 3$。注意到 $3 \cdot 3 \equiv 9 \equiv 1 \mod 4$,所以
            \[3^{-1} \equiv 3 \mod 4\]
        \item $\mathbf{n=4, x=2}$:\\
            我们只需检查 $y = 1$ 和 $y = 2$。然而,由于 $x$ 是偶数,所以 $x$ 的任何倍数也都是偶数,而满足 $y \equiv 1 \mod 4$ 的数必须是奇数。因此,$2$ 在模 $4$ 下没有乘法逆元。
        \item $\mathbf{n=10, x=3}$:\\
            我们可以在这里逐一检查所有情况:
            \begin{align*}
                3 \cdot 1 &\equiv 3 \mod 10 \\
                3 \cdot 2 &\equiv 6 \mod 10 \\
                3 \cdot 3 &\equiv 9 \mod 10 \\
                3 \cdot 4 \equiv 12 &\equiv 2 \mod 10 \\
                3 \cdot 5 \equiv 15 &\equiv 5 \mod 10 \\
                3 \cdot 6 \equiv 18 &\equiv 8 \mod 10 \\
                3 \cdot 7 \equiv 21 &\equiv 1 \mod 10 \\
            \end{align*}
            这意味着
            \[3^{-1} \equiv 7 \mod 10\]
            注意,这也同时表明
            \[7^{-1} \equiv 3 \mod 10\]
            因为乘法具有交换律(即顺序不重要)。这一观察使我们得出以下结论:
            \[(a^{-1})^{-1} \equiv a \mod n \quad \text{假设}\; a^{-1} \;\text{存在}\]
        \item $\mathbf{n=15, x=7}$:\\
            如果我们检查 $7$ 的所有倍数,会发现当我们检查到 $13$ 时,我们就成功了:
            \[7 \cdot 13 \equiv 91 \equiv 6 \cdot 15 + 1 ≡ 1 \mod 15\]
            所以
            \[7^{-1} \equiv 13 \mod 15\]
            验证工作留给你来完成。例如在模 $15$ 下 $6$ 没有乘法逆元。 
    \end{itemize}
\end{example}

\subsubsection*{何时存在乘法逆元?}

现在我们已经研究了一些例子,是时候静下心来,描述所有乘法逆元存在的情况。以下引理描述了这些情况。

\begin{lemma}[互质时的乘法逆元引理或 MIRP 引理]\label{lemma6.5.24}
    假设 $n \in \mathbb{M}, a \in \mathbb{Z}$ 且 $n, a$ \dotuline{互质}。考虑同余式 $a \cdot x \equiv 1 \mod n$,那么存在解 $x \in \mathbb{Z}$,使得它满足该同余式。

    事实上,这个同余式有无穷多个解,并且它们都模 $n$ 同余。这意味着在集合 $[n - 1] = \{1, 2, ... , n-1\}$ 中恰好存在一个解。

    我们用 $a^{-1}$ 来表示该同余方程解的等价类,并称其为 $a$ 模 $n$ 的\dotuline{乘法逆元}。

    此外,这是一个充要条件;也就是说,如果 $a$ 和 $n$ 不互质,那么同余方程 $a \cdot x \equiv 1 \mod n$ 在整数范围内无解。
\end{lemma}

这个引理完全说明了乘法逆元何时存在,何时不存在。我们可以用它来判断如下同余式
\[15x \equiv 1 \mod 33\]
在 $x \in \mathbb{Z}$ 下\textbf{误解},因为 $3 \mid 15$ 且 $3 \mid 33$,所以它们不是互质的。同理,我们可以用它来判断如下同余式
\[40x \equiv 1 \mod 51\]
在 $x \in \mathbb{Z}$ 下\textbf{必}有解,因为 $40 = 2^3 \cdot 5$ 和 $51 = 3 \cdot 17$ 互质。(注意,这个引理在帮助我们\emph{找到}解时只提供了一部分信息;它只是保证我们可以在 $\{1, 2, \dots , n-1\}$ 的元素中找到解。)

为了\textbf{证明}这个引理,我们将其分成两部分,因为它是一个双向陈述。我们将为你证明其中一个方向;即当 $a$ 和 $n$ 互质时,$a^{-1}$ 在模 $n$ 下存在。另一个方向(如果 $a$ 和 $n$ 有公因子,那么 $a^{-1}$ 在模 $n$ 下不存在)将会在习题 \ref{exc:exercises6.7.21} 中引导你完成证明。(你可以现在就试试!)在证明过程中,我们需要用到以下有用的引理。

\begin{lemma}[欧几里得引理]\label{lemma6.5.25}
    给定 $a, b, c \in \mathbb{Z}$。假设 $a \mid bc$,并假设 $a$ 和 $b$ 互质。则 $a \mid c$。
\end{lemma}

我们会推迟对这个引理的证明,直到看到 MIRP 引理的证明。我们认为,详细探讨这个引理的证明细节可能会暂时分散我们对本节主要目标的注意力。此外,欧几里得引理的结果本身已经相当可信,我们可以暂时假定其有效性,并在 MIRP 引理的证明中使用它。请看以下几个例子:
\begin{itemize}
    \item 我们知道 $3 \mid 30$,且 $30 = 5 \cdot 6$。由于 $3$ 和 $5$ 互质,我们可以推断 $3 \mid 6$,而这当然是对的。
    \item 假设对于某个整数 $x, 3 \mid 5x$。我们能得出关于 $x$ 的什么结论呢?由于 $3$ 和 $5$ 互质,所以要使 $5x$ 成为 $3$ 的倍数,$x$ 必须``包含''一个 $3$ 的因子。也就是说,$3 \mid x$ 是必要条件。
\end{itemize}
我们意识到这还不够令人满意!我们并不是说要在没有证据的情况下\emph{接受}这个说法;我们只是希望在深入讨论之前稍等片刻。同时,你可以试着自己证明一下!看看你能得出什么结论。

现在,让我们继续向前,证明 MIRP 引理(假设在中间某个步骤会用到欧几里得引理的结果)。

\begin{proof}
    设 $n \in \mathbb{N}, a \in \mathbb{Z}$。假设 $a$ 和 $n$ \textbf{互质}。
    
    我们要证明 $\exists x \in \mathbb{Z} \centerdot ax \equiv 1 \mod n$。

    考虑 $a$ 的前 $n$ 个倍数组成的集合;也就是说,定义集合 $N$ 为
    \begin{align*}
        N &= \{0, a, 2a, 3a, \dots ,(n-1)a\} \\
        &= \{z \in \mathbb{Z} \mid \exists k \in [n - 1] \cup \{0\} \centerdot z = ka\}
    \end{align*}
    请注意,集合 $N$ 中有 $n$ 个元素。

    \textbf{声明}:集合 $N$ 的所有元素在模 $n$ 运算下会产生\emph{不同}余数;也就是说,
    \[\forall i, j \in [n-1] \cup \{0\} \centerdot i \ne j \implies ai \not\equiv aj \mod n\]

    我们来证明这个声明。首先,为了得到矛盾而假设该声明为\verb|假|。

    这意味着 $\exists i, j \in [n-1] \cup \{0\} \centerdot ai \equiv aj \mod n$。假设给定这样的 $i$ 和 $j$。

    通过减法和因式分解,我们可以得到 $ai - aj \equiv a(i-j) \equiv 0 \mod n$。

    这意味着 $n \mid a(i-j)$。已知 $n$ 和 $a$ 互质。根据上面的引理 \ref{lemma6.5.25},我们可以推导出 $n \mid i-j$。

    现在,我们可以推断出 $i = j$。请记住 $i, j \in [n-1] \cup \{0\}$,因此 $0 \le i, j \le n-1$,也就是说 $-(n-1) \le -j, ij \le 0$。

    将这些关于 $i$ 和 $-j$ 的不等式相加后,我们可以发现
    \[-(n-1) + 0 = n-1 \le i + (-j) = i - j \le n - 1 = (n-1) + 0\]
    也就是说,$-(n-1) \le i-j \le n-1$。我们已知 $n \mid i - j$,即 $i-j$ 是 $n$ 的倍数。请注意,在 $-(n-1)$ 和 $+(n-1)$ 之间,$n$ 的倍数\emph{只有} $0$。

    因此 $i-j=0$,即 $i=j$。这就证明了当前的声明。

    我们现在可以确定,$N$ 的元素在模 $n$ 下会产生不同的余数。而且,这些可能的余数是 $\{0, 1, 2, \dots, n-1\} = [n-1] \cup \{0\}$。注意到 $N$ 有 $n$ 个不同的元素,而模 $n$ 下也有 $n$ 个不同的余数(等价类)。这意味着,在集合 $N$ 中,每个模 $n$ 的余数都\emph{恰好出现一次}。

    这表明,$N$ 中\emph{恰好}有一个元素(即一个 $a$ 的倍数)对应于模 $n$ 余数为 $1$。这个 $N$ 中的元素可以表示为 $ax$,其中 $x \in [n-1] \cup \{0\}$。假设给定这样的 $x$。这就是引理中所述同余方程的解。
\end{proof}

真是花了不少功夫,但我们终于到了这一步。既然你已经证明了练习 \ref{exc:exercises6.7.21} 中的论点(确实如此,对吧?$\smiley{}$),我们现在\emph{确切}知道在 $\mathbb{Z}$ 模 $n$ 的情况下,乘法逆元何时存在。我们也知道了一种合理的方法来找到它们:只需检查 $a$ 的前 $n-1$ 个倍数,找出其中模 $n$ 等于 $1$ 的倍数。

既然我们已经完成了这一步,现在让我们回头证明一下欧几里得引理。这一步是必要的,因为重要的 MIRP 引理的证明依赖于这个结果。注意,这个证明中包含一个复杂的\emph{归纳论证}。具体来说,我们有\emph{两个变量} $a$ 和 $b$,我们需要证明某个陈述对所有这样的 $a$ 和 $b$ 都成立。

\begin{proof}
    设 $a, b, c \in \mathbb{Z}$。假设 $a \mid bc$,且 $a$ 和 $b$ 互质。

    我们要证明必然有 $a \mid c$。我们先来证明:

    \textbf{声明}:如果 $a, b \in \mathbb{N}$ 且 $a$ 和 $b$ 互质,则 $\exists x, y \in \mathbb{Z} \centerdot ax+by = 1$。

    基于这个声明,结果将很容易得出。我们将这个声明的证明放在一个框中,方便阅读。在框之后,你会看到我们如何使用这个结果来证明引理的原始陈述。

    (在进行这个证明之前,可以先用一些例子来``说服''自己这个证明为\verb|真|。取两个互质的数,例如 $5$ 和 $11$,或者 $15$ 和 $22$,抑或 $10$ 和 $23$,尝试构造\emph{线性组合}来得到 $1$。然后,取一些有公因子的数,例如 $5$ 和 $10$,或者 $6$ 和 $15$,抑或 $21$ 和 $27$,尝试理解为什么你找不到这样的组合。)

    \begin{tcolorbox}[colback=gray!10,%gray background
        colframe=black,% black frame colour
        width=\textwidth,% Use 5cm total width,
        arc=2mm, auto outer arc,
        title={证明声明},breakable,enhanced jigsaw,
        before upper={\parindent15pt\noindent},	]
            我们将通过对 $a+b$ 进行归纳来证明这一点。在开始之前,先来看几个事实:
            \begin{itemize}
                \item 如果 $a=1$ 或 $a=-1$,则 $b$ 必须为 $0$ 或 $1$ 才能满足它们互质。\\
                    无论哪种情况,我们都可以令 $x=a, y=0$ 从而写出
                    \[ax + by = a^2 + 0 = 1\]
                    同样的论证亦可以应用于 $b=1$ 或 $b=-1$ 的情况(此时 $a$ 必为 $0$ 或 $1$)。
                \item 如果 $b=0$ 且 $|a| \ge 2$(即 $a \ne \pm 1$),则 $a, b$ 有公因子 $a$,因此它们不互质。\\
                    同样的论证亦可以应用于 $a=0$ 的情况。
            \end{itemize}
            综上,我们可以忽略 $a$ 或 $b$ 为 $0$ 的情况。换句话说,我们只考虑 $|a| \ge 1$ 和 $|b| \ge 1$ 的值。\\

            \begin{itemize}
                \item 因为 $a$ 和 $b$ 互质,所以 $-a$ 和 $b$ 也互质(同理,$-a$ 和 $b$ 以及 $a$ 和 $b$ 也都互质)。这是因为取相反数只会改变整数的符号,不会影响其因子。
                \item 如果已知 $\exists x, y \in \mathbb{Z} \centerdot ax + by = 1$,则必有
                    \[(-a)(-x) + (-b)(-y) = ax + by = 1\]
                    因为 $-x,-y \in \mathbb{Z}$,这说明 $-a$ 和 $-b$ 也可以有这样的表示。
            \end{itemize}
            综上,我们只需要考虑 $a$ 和 $b$ 为\emph{正数}的情况。(换句话说,如果 $a$ 或 $b$ 为负数,我们只需取它们的相反数即可。)\\

            结合之前的推论,我们可以推断出只需要考虑 $a,b \in \mathbb{N}$ 的情况。证明这些值的结果,再结合我们之前的观察,就能得出完整的结论。

            现在,我们可以通过对 $a + b$ 应用(强)归纳法来进行证明。由于 $a,b \in \mathbb{N}$,因此 $a + b \ge 2$。我们之前已经考虑了基本情况 $a + b = 2$,但为了完整性,这里再重述一遍。

            给定 $a,b \in \mathbb{N}$,定义 $P(a,b)$ 为陈述
            \[a \;\text{和}\; b \;\text{互质} \implies \exists x, y \in \mathbb{Z} \centerdot ax + by = 1\]

            \textbf{基本情况}:考虑 $P(2)$,即假设 $a,b \in \mathbb{N}$,$a$ 和 $b$ 互质且 $a+b=2$。这意味着 $a=b=1$,我们可以令 $x=1, y=0$ 得出
            \[ax + by = 1 + 0 = 1\]
            因此,$P(2)$ 成立。

            \textbf{归纳假设}:设 $k \in \mathbb{N}$ 为任意固定自然数。假设 $P(2) \land P(3) \land \dots \land P(k)$ 成立。(也就是说,假设每当两个互质数之和等于 $2,3, \dots, k$ 时,我们都能找到它们的一个\emph{线性组合}使其等于 $1$。)

            \textbf{归纳步骤}:我们要证明 $P(k+1)$ 成立。也就是说,设 $a, b \in \mathbb{N}$,且 $a + b = k + 1$,并假设 $a$ 和 $b$ 互质;我们要证明 $\exists x, y \in \mathbb{Z} \centerdot ax + by = 1$。

            首先,根据对称性,我们可以假设 $a \ge b$。(也就是说,给定 $a$ 和 $b$ 的值。无论它们是什么值,我们都可以对它们进行``重命名'',因为其中一个值至少与另一个一样大;我们将较大的那个标记为 $a$。)事实上,由于 $a$ 和 $a$ 不互质(当 $a \ge 2$ 时),我们甚至可以假设 $a > b$。

            现在,我们要利用 $b$ 和 $a - b$ 互质这一事实。要理解为什么会这样,我们需要证明 $b$ 和 $a - b$ 只有 $1$ 这个公约数。

            设 $d$ 为 $b$ 和 $a - b$ 的公因数,即 $d$ 能整除 $b$ 和 $a-b$。这意味着 $d$ 能整除 $b + (a-b)$,即 $d$ 能整除 $a$。我们已经知道 $d$ 能整除 $b$,因此 $d$ 实际上是 $a$ 和 $b$ 的公因数,所以它必然为 $1$。因此,$b$ 和 $a-b$ 互质。

            (我们刚刚证明的是
            \[(d \mid b \land d \mid a - b) \implies d \mid a \land d \mid b\]
            该声明也是一个 $\iff$ 陈述。我们鼓励你思考一下为什么 $\impliedby$ 方向也成立。)

            我们现在有 $b, a-b \in \mathbb{N}$(因为 $b < a$)互质。还要注意,$b + (a-b) = a < a + b = k + 1$,因为 $b \in \mathbb{N}$(所以 $b \ge 1$)。这意味着 $a + b \le k$,因此归纳假设 $P(a + b)$ 适用!

            (注意 $P(a + b)$ 不一定等同于 $P(k)$,因此我们需要使用强归纳法!)

            陈述 $P(a + b)$ --- 即 $P(b + (a - b))$,我们将用到这一点 --- 告诉我们 $b$ 和 $a - b$ 的线性组合可以得到 $1$;也就是说,
            \[\exists u, v \in \mathbb{Z} \centerdot ub + v(a - b) = 1\]
            我们现在要把它转换为 $a$ 和 $b$ 的线性组合,以得到 $1$。为此,我们将重新写这个方程,并重新标记系数:
            \[ub + v(a-b) = 1 \iff \underbrace{v}_{x} a + b \underbrace{(u - v)}_{y} = 1\]
            也就是说,我们现在可以定义 $x = v$ 和 $y = u - v$,这样 $x, y \in \mathbb{Z}$,并且满足 $ax + by = 1$。

            我们现在已经证明了 $P(a + b)$(即 $P(k + 1)$)成立。通过强归纳法,我们推导出 $P(n)$ 对于所有 $n \in \mathbb{N}$ 且 $n \ge 2$ 都成立。
    \end{tcolorbox}	

    为了提醒大家,这个证明的结果是,我们现在知道任意互质的数都可以通过线性组合得到 $1$。

    让我们回到引理的原始陈述。已知 $a, b, c \in \mathbb{N}$,并假设 $a$ 和 $b$ 互质且 $a \mid bc$。

    第一个假设说明存在整数 $x$ 和 $y$,使得 $ax + by = 1$。给定这样的 $x$ 和 $y$。

    第二个假设说明存在整数 $k$,使得 $bc = ak$。给定这样的 $k$。
    
    接下来,我们将第一个假设中的方程乘以 $c$,然后应用第二个假设:
    \[ax + by = 1 \implies acx + (bc)y = 1 \implies acx + (ak)y = c \implies c = a\underbrace{(cx + ky)}_{\ell}\]
    也就是说,通过 $ax + by = 1$,我们可以推导出 $c = a\ell$,其中 $\ell \in \mathbb{Z}$,并且 $\ell$ 是由其他整数定义的。

    根据定义,这意味着 $a \mid c$。这就证明了最初的陈述。
\end{proof}

哇!这个证明包含了很多内容。请你多读几遍,逐行理解并做好笔记。你能明白为什么每个声明都基于我们已知的内容吗?你能看出归纳法是如何应用的吗?虽然我们有两个变量,但我们对其中一个变量进行归纳,这个变量被定义为另外两个变量的和。我们知道这是一个复杂的证明,因此将它放在这里,紧跟在本节更重要的 MIRP 引理之后。

让我们利用这个结果 --- \emph{准确}知道何时存在乘法逆元 --- 来解决一系列问题吧!

\subsubsection*{使用乘法逆元}

这有何用处?虽然这个答案听起来有点调皮,但它确实是正确的:乘法逆元在使用模运算解决同余问题时非常有用。乍一看,似乎我们是为了这些问题而开发了数学工具,但事实并非如此。实际上,正如你将在接下来的例子中看到的那样,在尝试解决这些问题时,你很可能会发明出我们即将应用的这些技术。换句话说,即使你没有学过乘法逆元,也可以尝试解决这些问题,但最终你会重新发现我们已经探讨过的结果。

好了,铺垫就到这里。让我们看几个具体的问题。这些问题的形式都是:``有一个同余方程;找出该方程的所有整数解,或者证明其无解。''\\

\begin{example}\label{ex:example6.5.26}
    找到所有整数 $x, y \in \mathbb{Z}$ 满足
    \[3x - 7y = 11\]
    我们声称有无穷多对 $(x, y) \in \mathbb{Z} \times \mathbb{Z}$ 满足这个方程。此外,我们可以给出所有解的形式,并通过定义这些解的集合来实现。

    通过重写给定方程,我们想找出所有 $x \in \mathbb{Z}$ 使得
    \[3x \equiv 11 \mod 7\]
    假如我们能找到 $x \in \mathbb{Z}$ 所有整数解,我们可以轻松地通过解上面方程得到对应 $y \in \mathbb{Z}$ 的解:$y = \frac{3x-11}{7}$。

    注意到 $3^{-1} \equiv 5 \mod 7$,这是因为 $3 \cdot 5 \equiv 15 \equiv 2 \cdot 7 + 1 \equiv 1 \mod 7$。因此,根据模算术引理,我们可以将同余式两边乘以 $3^{-1}$,从而得到
    \begin{align*}
        \forall x \in \mathbb{Z} \centerdot 3x \equiv 11 \mod 7 &\iff 3^{-1} \cdot 3 \cdot x \equiv 3^{1}
\cdot 11 \mod 7 \\
        &\iff 1 \cdot x \equiv 5 \cdot 4 \mod 7 \\
        &\iff x \equiv 20 \equiv 6 \mod 7
    \end{align*}
    由于我们知道 $3^{-1}$ 表示这个同余式的所有解(即它代表 $3$ 模 $7$ 乘法逆元的等价类),那么我们可以推导出
    \[\forall x \in \mathbb{Z} \centerdot 3x \equiv 11 \mod 7 \iff x \equiv 6 \mod 7 \iff \exists k \in \mathbb{Z} \centerdot x = 7k + 6\]
    这表征了给定方程解中所有可能的 $x \in \mathbb{Z}$ 的值。

    现在,我们用这个方法来确定解中对应的 $y \in \mathbb{Z}$ 的值。假设 $k \in \mathbb{Z}$,且 $x = 7k + 6$,然后我们代入 $x$ 发现
    \[y = \frac{3x-11}{7} = \frac{3(7k + 6)-11}{7} = \frac{21k+7}{7} = 3k+1\]

    现在,我们找到了表示给定方程所有可能解的形式。我们知道,任意 $k \in \mathbb{Z}$ 都会对应一个 $x$,从而对应一个 $y$。此外,由于我们的推导使用了 $\iff$ 陈述,我们可以确定这涵盖了所有的解。

    我们可以将给定方程的解集 $S$ 描述为
    \[S = \{(x, y) \in \mathbb{Z} \times \mathbb{Z} \mid \exists k \in \mathbb{Z} \centerdot (x, y) = (7k + 6, 3k + 1)\}\]
\end{example}

\begin{tcolorbox}[colback=gray!10,
    colframe=black,
    width=\textwidth,
    arc=2mm, auto outer arc,
    title={有趣的事实},breakable,enhanced jigsaw,
    before upper={\parindent15pt\noindent},	]
    在这个例子中,我们解决了一个\textbf{线性丢番图方程},并找出了它的所有解。所谓\emph{线性},指的是变量 $x$ 和 $y$ 是一次的,没有平方或立方项。

    通过我们在这个例子中使用的技术,你可以解决\emph{任意}线性丢番图方程,或者轻松判别它是否有解。事实上,我们还将\emph{证明}一个关于这种方程何时没有解的结论(见贝祖恒等式,定理 \ref{theorem6.5.31})。只要方程有解,这种方法就适用。
    
    在下一个例子中,我们将研究\textbf{二次丢番图方程},其中变量会有平方项(包括 $x^2$ 和 $y^2$)。之后我们将讨论解决这类方程的可能性。
\end{tcolorbox}

\begin{example}
    现在让我们再来看一个例子,这个例子和前一个例子的过程类似(使用乘法逆元简化运算),但还引入了二次残差的概念。

    \textbf{声明}:无整数 $x, y \in \mathbb{Z}$ 满足方程
    \[3x^2-5y^2=1\]

    给定 $x, y \in \mathbb{Z}$。我们要证明 $3x^2-5y^2=1$ 是\emph{不可能的}。

    我们先将给定方程重写为
    \[3x^2 = 5y^2+1\]
    具体来说,这意味着
    \[3x^2 \equiv 1 \mod 5\]
    因为 $5y^2 \equiv 0 \mod 5$。注意到 $3^{-1} \equiv 2 \mod 5$,因为 $3 \cdot 2 = 6 = 5 + 1$。因此,我们可以将等式两边都乘以 $3^{-1}$,从而简化得:
    \[3x^2 \equiv 1 \mod 5 \iff 3^{-1} \cdot 3x^2 \equiv 3^{-1} \cdot 1 \mod 5 \iff x^2 \equiv 2 \mod 5\]
    然而,回顾示例 \ref{ex:example6.5.15},在那里我们研究了\emph{二次残差}。我们发现,模 $5$ 的二次残差集合为 $\{0, 1, 4\}$。也就是说,\emph{不可能}有整数 $x$ 满足 $x^2 \equiv 2 \mod 5$。这就表明,给定方程没有整数解。
\end{example}

\begin{tcolorbox}[colback=gray!10,
    colframe=black,
    width=\textwidth,
    arc=2mm, auto outer arc,
    title={有趣的事实},breakable,enhanced jigsaw,
    before upper={\parindent15pt\noindent},	]
    我们之前提到,我们确切知道何时线性丢番图方程可解,并且知道如何解这些方程。但对于\textbf{二次丢番图方程},我们就没有这么幸运了。要判断一个二次丢番图方程是否有解是非常困难的。即使知道它有解,实际求解也非常复杂。

    事实上,对于这些二次丢番图方程,我们的运气非常糟糕。已知\textbf{没有任何计算机算法可以输入带有一次和二次幂变量的丢番图方程,并判断该方程是否有解}。这一事实甚至不涉及如何解方程,仅仅是判断它是否有解。令人惊讶的是,这个事实是\href{https://en.wikipedia.org/wiki/Hilbert's_tenth_problem}{希尔伯特第十问题}的一种形式。

    请放心,我们在这里提供的例子和练习中的丢番图方程都可以用我们提供的技术进行分析。我们提到的这个事实是针对所有此类方程的一般性声明。
\end{tcolorbox}

\subsubsection*{一点群论知识}

在这一小节中,我们想强调当前主题背后蕴含的一些重要而深刻的数学原理。由于篇幅和时间所限,我们无法全面探讨这些内容。因此,我们将在此简要介绍一些概念和事实,并通过例子来加以说明。

我们想传达的主要思想是,当我们考虑 $\mathbb{Z}$ 模 $p$ 时(其中 $p$ 为\textbf{质数}),会出现一些特殊的现象。在这种情况下,每个小于 $p$ 的数都与 $p$ \emph{互质},因为 $p$ 只有 $1$ 这个因子。这意味着在 $\{1, 2, \dots, p-1\}$ 中的所有数在模 $p$ 下都有乘法逆元。这非常方便,因为除了 $[0]_{\mod p}$ 外,每个同余类都有一个对应的乘法逆元类。

例如,考虑 $p=5$。注意到
\begin{align*}
    1^{-1} \equiv 1 \mod 5\\
    2^{-1} \equiv 3 \mod 5\\
    3^{-1} \equiv 2 \mod 5\\
    4^{-1} \equiv 4 \mod 5
\end{align*}

再比如,考虑 $p=7$。注意到
\begin{align*}
    1^{-1} \equiv 1 \mod 7\\
    2^{-1} \equiv 4 \mod 7\\
    3^{-1} \equiv 5 \mod 7\\
    4^{-1} \equiv 2 \mod 7\\
    5^{-1} \equiv 3 \mod 7\\
    6^{-1} \equiv 6 \mod 7
\end{align*}

请注意,这意味着集合中的所有元素都有一个乘法逆元。

(同时,请注意这些逆元其实是数字 $1$ 到 $p - 1$ 的一种\emph{排列}。这并非巧合!试着证明为什么会这样!试着证明有两个元素是它们自己的逆元,即 $1^{-1} \equiv 1 \mod p$ 和 $(p - 1)^{-1} \equiv p - 1 \mod p$,而其他元素都\emph{不可能}是它们自己的逆元。)

当我们考虑 $\mathbb{Z}$ 模 $n$ 时,如果 $n$ 是合数,情况就不一样了。这种情况下,我们知道 $n$ 可以因式分解;假设 $n = ab$,其中 $a,b \in \mathbb{N}-\{1\}$。那么 $1 < a < n$,但 $a$ 和 $n$ 不互质(它们有公因子 $a$),所以 $a$ 在模 $n$ 下没有乘法逆元。实际上,所有 $n$ 的因数(及其倍数)在模 $n$ 下都没有乘法逆元。

例如,考虑 $p=6$。
\begin{align*}
    1^{-1} \equiv 1 \mod 6\\
    2^{-1} \;\text{不存在} \mod 6\\
    3^{-1} \;\text{不存在} \mod 6\\
    4^{-1} \;\text{不存在} \mod 6\\
    5^{-1} \equiv 5 \mod 6
\end{align*}

由于这种区别,$\mathbb{Z}$ 模 $p$ 的数学``结构''显得格外突出。它具备一些优良的性质,并在某种意义上表现得非常好。虽然这些描述可能比较模糊,但主要思想是:所有元素都有逆元,这使得 $\mathbb{Z}$ 模 $p$ 很特别。事实上,$\mathbb{Z}$ 模 $p$ 构成一种称为\textbf{群}的数学结构。

一般来说,从启发式的角度看,群是一个可以进行``乘法''运算的对象集合,这种乘法运算满足
\begin{enumerate}[label=(\alph*)]
    \item 交换律
    \item 结合律
    \item 所有元素都有逆元
\end{enumerate}
我们已经知道,标准的整数乘法(即使在 $\mathbb{Z}$ 模 $n$ 中,对于任意 $n$)满足交换律和结合律,并且在 $\mathbb{Z}$ 模 $p$ 中(对于质数 $p$)每个元素都有逆元。

如果你对这些概念感兴趣,可以在本章末尾找到一些练习,帮助你理解这些性质。此外,你也可以查阅一些\textbf{抽象代数}、\textbf{近世代数}或\textbf{群论}的入门教材。这些领域中有许多强大而深刻的数学思想,\textbf{群}在许多领域中都有重要的应用!


% !TeX root = ../../../book.tex

\subsection{一些有用的定理}

在本节中,我们将探讨数论中的一些定理,这些定理涉及模运算,既实用又有趣。我们将陈述并证明这些定理(有时需要你通过练习给予帮助),并用例子展示其实际应用。

\subsubsection*{中国剩余定理}

为了引出这个定理,我们通过一个故事说明其用途:

\begin{quotation}
    孙武\footnote{在中国,这个故事的主角是韩信。因此这个问题也被称为``韩信点兵''问题。—— 译者注}将军的部队有许多士兵。战斗结束后,他想快速统计剩余士兵的数量。逐个清点显然太费时,因此他想用更高效的方法完成点兵。幸运的是,士兵们训练有素,能轻松组成等大小的队列。
    
    孙武将军先命令士兵排成两列等长的队列,结果多出一人。

    接着,他让士兵排成三个等大小的环形队列,但仍多出一人。

    最后,他命令士兵排成五个等大小的侧翼队列,这次多出两人。

    至此,他认为信息已经足够。战斗结束后,他推测部队总人数在 $250$ 到 $300$ 之间。根据这些信息,他能\emph{准确}得知士兵数量。
    
    你能算出具体人数吗?这支部队究竟有多少士兵?
\end{quotation}

请先尝试自行解决此问题,检验自己能否找到答案,然后再继续阅读书中的解决方案、定理陈述及问题解决方法。

回顾这个故事。设士兵数量为 $x$,其满足以下三个同余条件和一个不等式:
\begin{align*}
    x \equiv 1 \mod 2 \\
    x \equiv 1 \mod 3 \\
    x \equiv 2 \mod 5 \\
    250 \le x \le 300
\end{align*}
(你能从故事中找到这些条件的来源吗?)

现在有两个问题需要考虑:
\begin{enumerate}[label=(\arabic*)]
    \item 是否\emph{一定}存在满足所有同余条件的 $x$?
    \item 是否存在\emph{多个}满足条件的 $x$?能否保证其中之一满足不等式?
\end{enumerate}

下面陈述的\textbf{中国剩余定理}可以保证:
\begin{enumerate}[label=(\arabic*)]
    \item 同余方程有无穷多个解;
    \item 至少有一个解满足给定的不等式。
\end{enumerate}
但在陈述并证明该定理前,我们先尝试解决原始问题。我们将其分解为以下步骤:
\begin{itemize}
    \item 第一个同余条件要求 $x$ 必须是\textbf{奇数},这样就排除了所有偶数作为潜在解。以下是潜在解列表:
    \[1,\cancel{2}, 3, \cancel{4}, 5, \cancel{6}, 7, \cancel{8}, 9,\cancel{10}, 11,\cancel{12}, 13,\cancel{14}, 15,\cancel{16}, 17,\cancel{18}, 19,\cancel{20}, 21,\cancel{22}, 23, \dots\]

    \item 第二个同余条件要求解必须是 $3$ 的倍数加 $1$,这就排除了模 $3$ 余 $0$ 或 $2$ 的数。以下是潜在解列表:
    \[1,\cancel{2}, \cancel{3}, \cancel{4}, \cancel{5}, \cancel{6}, 7, \cancel{8}, \cancel{9},\cancel{10}, \cancel{11},\cancel{12}, 13,\cancel{14}, \cancel{15},\cancel{16}, \cancel{17},\cancel{18}, 19,\cancel{20}, \cancel{21},\cancel{22}, \cancel{23}, \dots\]

    \item 第三个同余条件要求解必须是 $5$ 的倍数加 $2$,这就排除了模 $5$ 余 $0, 1, 3, 4$ 的数。以下是潜在解列表:
    \[\cancel{1},\cancel{2}, \cancel{3}, \cancel{4}, \cancel{5}, \cancel{6}, \circled{7}, \cancel{8}, \cancel{9},\cancel{10}, \cancel{11},\cancel{12}, \cancel{13},\cancel{14}, \cancel{15},\cancel{16}, \cancel{17},\cancel{18}, \cancel{19},\cancel{20}, \cancel{21},\cancel{22}, \cancel{23}, \dots\]
\end{itemize}
表面上看 $7$ 是唯一解,但我们仅检查了前 $23$ 个潜在解……能否\emph{确保}无其他解?请自行探究:尝试更大的数值,观察能否找到其他解。你能发现规律吗?$7$ 真的是唯一解?

现在,我们用更巧妙的方法来解决这些同余问题。具体来说,假设我们有一个解 $x$,它满足所有三个同余式,看看我们能否推导出更多的信息。通过此推导,我们将揭示所有\emph{潜在}解的特性。

根据同余的定义,可知存在 $k, \ell, m \in \mathbb{Z}$ 使得
\begin{align*}
    x &= 2k + 1 \\
    x &= 3\ell + 1 \\
    x &= 5m + 2 
\end{align*}
给定这样的 $k, \ell, m$。

先来看前两个方程,试着将它们合并成一个关于 $x$ 的方程。具体来说,把第一个方程乘以 $3$,第二个方程乘以 $2$,这样就会分别得到 $6k$ 和 $6\ell$ 项。
\begin{align*}
    3x &= 6k + 3 \\
    2x &= 6\ell + 2
\end{align*}
然后两式相减,并进行因式分解可得:
\[(3x - 2x) = (6k + 3) - (6\ell + 2) \implies x = 6(k - \ell) + 1\]
因为 $k, \ell \in \mathbb{Z}$ 已经给定,我们可以令 $u = k-\ell$ ($u \in \mathbb{Z}$)。此时 $x = 6u+1$,即
\[x \equiv 1 \mod 6\]
现在,我们通过合并前两个同余式得到了这个新的同余式。这并非巧合,因为这个同余式是模 $6$ 的,而 $6 = 2 \times 3$。稍后,当我们引导你证明接下来的定理时,你会明白其中的原理!

接下来,我们尝试将这个新的同余式与上面的第三个同余式合并。我们采用类似的方法:将刚刚推导出的同余式乘以 $5$,将第三个同余式乘以 $6$,这样相减后可以提出因子 $30$。(这也解释了为什么新推导出的同余式是模 $30$ 的。)
\begin{align*}
    5x &= 30u + 5 \\
    6x &= 30m + 12
\end{align*}
然后两式相减,并进行因式分解可得:
\[(6x - 5x) = (30m + 12) - (30u + 5) \implies x = 30(m - u) + 7\]
同理,因为 $u,m$ 已经给定,我们可以令 $v = m-u$ ($v \in \mathbb{Z}$)。此时 $x = 30v+7$,即
\[x \equiv 7 \mod 30\]
这个最终的同余式是通过将给定的三个同余式合并推导出来的,因此它包含了这三个同余式的所有信息。我们断言,该同余式包含了\textbf{所有的}解!

首先,这个新推导出的同余式表明,任何解必须满足模 $30$ 余 $7$。换句话说,任何除以 $30$ 余数不为 $7$ 的数都不可能是解。这本质上将我们通过上述三种观察排除潜在解的工作总结为一个陈述。

其次,我们可以证明:任何模 $30$ 余 $7$ 的数确实是一个解。设 $n \in \mathbb{Z}$,并定义 $y = 30n + 7$(即任意满足 $y \equiv 7 \mod 30$ 的整数 $y$)。注意 $y$ 满足:
\begin{itemize}
    \item 第一个同余式,因为 $y = 30n + 7 = 2(15n + 3) + 1$,所以 $y \equiv 1 \mod 2$。
    \item 第二个同余式,因为 $y = 30n + 7 = 3(10n + 2) + 1$,所以 $y \equiv 1 \mod 3$。
    \item 第三个同余式,因为 $y = 30n + 7 = 5(\enspace 6n + 1) + 2$,所以 $y \equiv 2 \mod 6$。
\end{itemize}
至此,我们得到:
\begin{enumerate}[label=(\arabic*)]
    \item \emph{任何}解 $x$ 必须满足 $x \equiv 7 \mod 30$;
    \item 任何满足此条件的 $x$ 实际上\emph{就是}一个解。
\end{enumerate}
这两个陈述构成了一个 $\iff$ 陈述,即
\[x \text{\ 是三个同余式的解} \iff x \equiv 7 \mod 30\]
因此\textbf{所有解}的集合 $S$ 为
\[S = \{x \in \mathbb{Z} \mid x \equiv 7 \mod 30\} = \{30n + 7 \mid n \in \mathbb{Z}\}\]

回到原问题,我们只需考虑不等式 $250 \le x \le 300$。是否存在满足 $x \equiv 7 \mod 30$ 且在此范围内的 $x$?答案是肯定的!我们可以从 $7$ 开始累加 $30$ 的倍数,或者从接近 $300$ 的数开始逐步调整。无论采用哪种方法,最终得到 $\mathbf{x = 277}$,这就是孙武将军的士兵数量。

现在,为便于比较,考虑以下同余方程组:
\begin{align*}
    x &\equiv 3 \mod 4 \\
    x &\equiv 2 \mod 6
\end{align*}
该同余方程组有解吗?先前的方法在此是否适用?若尝试``排除候选值''或``合并同余式'',会发现这些方法均\emph{无效}。分析可知:第一个同余式要求 $x$ 比 $4$ 的倍数多 $3$,故 $x$ 为\emph{奇数};第二个同余式要求 $x$ 比 $6$ 的倍数多 $2$,故 $x$ 为\emph{偶数}。一个数如何同时即是奇数又是偶数?这显然不可能。

\textbf{中国剩余定理}阐明了同余方程组有解的条件。它适用于我们解决的第一个问题,并指出存在无穷多个解,且均模 $30$ 同余。然而,该定理并未直接说明第二个问题无解——它仅对特定情况提供\emph{保证}。当条件满足时,我们能有效判定解;但当条件\emph{不满足}时,定理并\emph{不保证}有解。现在,让我们陈述该定理,进一步探讨其内涵,并请你协助完成证明(使用两种不同的方法!)。

\begin{theorem}[中国剩余定理]\label{theorem6.5.28}
    考虑由 $r$ 个同余方程组成的方程组。设 $r \in \mathbb{N}$,给定 $r$ 个自然数 $n_1, n_2, \dots, n_r$ 和 $r$ 个整数 $a_1, a_2, \dots, a_r$,该方程组表示为:
    \begin{align*}
        x &\equiv a_1 \mod n_1 \\
        x &\equiv a_2 \mod n_2 \\
        \vdots \\
        x &\equiv a_r \mod n_r
    \end{align*}
    (换句话说,该方程组要求 $x \in \mathbb{Z}$ 满足 $\forall i \in [r] \centerdot x \equiv a_i \mod n_i$。)

    \dotuline{如果}模数 $n_i$ 两两互质,即任意两个 $n_i$ 之间除了 $1$ 以外没有其他公因数,\dotuline{那么}此同余方程组必有解。

    此时,方程组存在无穷多个解,且所有解模 $N$ 同余,其中 $N$ 为模数的乘积:
    \[N = \prod_{i \in [r]}^{} n_i\]
\end{theorem}

请注意,定理的核心是``\textbf{如果……那么……}''形式的条件陈述。该结论未涉及模数不互质的情形,这种情况下解的存在性不确定。例如,此前讨论的模数为 $4$ 和 $6$ 的方程组($4$ 与 $6$ 有公因数 $2$),定理并没有说这种情况下无解,其解的存在性取决于具体参数。若修改为:
\begin{align*}
    x \equiv 3 \mod 4 \\
    x \equiv 5 \mod 6
\end{align*}
这个方程组就有解。你可以试着解一下。

中国剩余定理的一种证明方法基于逐步合并同余方程的思想。对于含任意数量方程的方程组,可以通过迭代将同余关系归约为一个模数为全体模数乘积的同余方程。此过程的有效性可以通过数学归纳法证明(以方程数量 $r$ 为归纳变量),详见练习 \ref{exc:exercises6.7.26}。该证明的优势在于提供了实际求解的操作方法。

另一种证明是\textbf{构造性}的。也就是说,直接利用定理条件定义解 $X$ 并验证其有效性(见练习 \ref{exc:exercises6.7.27})。相较而言,我们更喜欢构造性证明,因为它不是通过论证某个对象存在的原因来证明,而是实际构造出此对象。然而,此证明方法构造解的方式有点``非自然'',即不是通过``排除候选值''也不是通过``合并同余式''来构造解,其构造解的方式较为间接,但避免了归纳流程。为了比较这两种方法,建议你尝试完成定理的两种证明。如果只能选择一种方法,我们会推荐使用归纳法证明。

\subsubsection*{贝祖恒等式 (Bézout's Identity)}

这一定理让我们回想起之前关于线性丢番图方程的讨论。在示例 \ref{ex:example6.5.26} 中,我们通过巧妙地应用乘法逆元解决了一个特定方程。除了展示该方法外,还有一种更简单的方法可以验证此类方程是否有解。该定理精确描述了二元线性丢番图方程解的存在条件,称为\textbf{贝祖恒等式 (Bézout's Identity)},以 $18$ 世纪法国数学家 Étienne Bézout 命名。

在陈述定理前,需要先给出一个定义。尽管你可能早已熟悉此概念,但它在定理中至关重要,因此我们在此给出正式定义并提供示例说明。

\begin{definition}[最大公约数]\label{def:definition6.5.29}
    给定 $a,b \in \mathbb{Z}$。$a$ 和 $b$ 的\dotuline{最大公约数}记为 $\gcd(a, b)$,定义为能同时整除 $a$ 和 $b$ 的最大整数,即:
    \[\gcd(a, b) \mid a \land \gcd(a, b) \mid b\]
    且满足:
    \[\forall d \in \mathbb{Z} \centerdot (d \mid a \land d \mid b) \implies d \le \gcd(a, b)\]
\end{definition}

我们假设你对此概念已有基本了解或直观认识。后续定理及其证明无需深入理解该定义,相关练习也不要求很强的计算能力或预备知识。请将其视为练习吸收新数学定义的机会,帮助你运用抽象概念证明结论、寻找例证与反例。这是一项关键技能!在陈述定理前,先看几个具体示例。

\begin{example}
    求两个整数的最大公约数时,常用的方法是对其进行\textbf{质因数分解}后组合公有因子。即 $\gcd(a, b)$ 是 $a$ 和 $b$ 公有质因数的乘积。

    以下是一些关于最大公约数的具体实例和一般性结论,这些论断仅依赖于最大公约数的定义。

    \begin{itemize}
        \item 设 $a=15, b=6$。因为 $a = 3 \cdot 5, b = 2 \cdot 3$,我们发现它们的公因子只有 $3$。因此
            \[\gcd(6, 15) = 3\]
        \item 设 $a=30, b=40$。因为 $a = 2 \cdot 3 \cdot 5, b = 2^3 \cdot 5$,我们发现它们的公因子有 $2$ 和 $5$。因此
            \[\gcd(30, 40) = 10\]
        \item 一般来说,
            \[\gcd(a, b) = \gcd(b, a)\]
            这显然是正确的,因为 $a$ 和 $b$ 的任何公约数也是 $b$ 和 $a$ 的公约数。
        \item 设 $a=77, b=72$。因为 $a = 7 \cdot 11, b = 2^3 \cdot 3^2$,我们发现它们没有公因子。因此
            \[\gcd(77, 72) = 1\]
        \item 设 $a=13$,设 $b \in \mathbb{N}$ 且 $a \nmid b$。因为 $a$ 为质数,且 $b$ 不是 $13$ 的倍数,因此 $b$ 的质因子中没有 $13$。因此
            \[\gcd(13, b) = 1\]
            这意味着 $a$ 和 $b$ \textbf{互质}。一般有:
            \[a \text{\ 与\ } b \text{\ 互质} \iff \gcd(a, b) = 1\]
            此外
            \[\forall a, b \in \mathbb{N} \centerdot a \text{\ 为质数} \implies \big(\gcd(a,b)=1 \iff a \nmid b\big)\]
    \end{itemize}
\end{example}

现在,我们已经准备好陈述并证明\textbf{贝祖恒等式}了!

\begin{theorem}[贝祖恒等式]\label{theorem6.5.31}
   给定 $a, b \in \mathbb{Z}$。定义 $L$ 为 $a$ 和 $b$ 所有线性组合的集合;即定义
   \[L = \{z \in \mathbb{Z} \mid \exists x, y \in \mathbb{Z} \centerdot ax + by = z\} = \{ax + by \mid x, y \in \mathbb{Z}\}\]
   定义 $M$ 为 $\gcd(a, b)$ 所有倍数的集合;即定义
   \[M = \{z \in \mathbb{Z} \mid \exists k \in \mathbb{Z} \centerdot z = k \cdot \gcd(a, b)\} = \{k \cdot \gcd(a, b) \mid k \in \mathbb{Z}\}\]
   则
   \[L = M\]
   换言之,线性丢番图方程 $ax + by = c$ 有解当且仅当 $c$ 是 $\gcd(a, b)$ 的倍数。
\end{theorem}

该定理非常实用,它直接指出线性丢番图方程 $ax + by = c$(其中 $a, b, c \in \mathbb{Z}$)有解的条件:计算 $\gcd(a, b)$ 并验证 $\gcd(a, b) \mid c$。

为了证明此定理,需要证两个\emph{集合相等}。我们将采用\emph{双向包含论证}法(此前已多次使用的策略),此处仅证明一个包含关系,另一个包含关系留作练习。

\begin{proof}
    给定 $a, b \in \mathbb{Z}$。按照定理的陈述定义集合 $L$ 和 $M$。
    \begin{itemize}
        \item 首先,证明 $L \subseteq M$。
        
            设 $z$ 是 $L$ 中任意固定元素。

            由 $L$ 的定义,可知 $\exists x, y \in \mathbb{Z} \centerdot ax + by = z$。给定这样的 $x$ 和 $y$。

            因为 $\gcd(a, b)$ 能整除 $a$ 和 $b$,故 $\exists k, \ell$ 使得 $a = k \cdot \gcd(a, b)$ 且 $b = \ell  \cdot \gcd(a, b)$。给定这样的 $k$ 和 $\ell$。

            将 $a$ 和 $b$ 的表达式代入上面的方程:
            \[z = ax + by = k \cdot \gcd(a, b) \cdot x + \ell \cdot \gcd(a, b) \cdot y = \gcd(a, b) \cdot \underbrace{(kx + \ell y)}_{m}\]

            令 $m = kx + \ell y$。由于 $m \in \mathbb{Z}$,这表明 $z$ 是 $\gcd(a, b)$ 的倍数。

            因此,$z \in M$。这证明了 $L \subseteq M$。\\
        \item 接着,证明 $M \subseteq L$。

            留作练习 \ref{exc:exercises6.7.12}。
    \end{itemize}
\end{proof}

通过此定理的证明,我们可以判定二元线性丢番图方程是否有解。后续练习将要求你基于此结果判断方程解的存在性;若需找出所有解,请采用示例 \ref{ex:example6.5.26} 的方法。

\textbf{挑战性问题}:对于多于两个变量的线性丢番图方程,你认为有哪些可讨论的内容?例如,考虑方程
\[6x + 8y + 15z = 10\]

这个方程有解吗?若有,解的个数是多少?再比如,考虑方程
\[3x + 6y + 9z = 2\]

这个方程是否有解?为什么?

尝试陈述并证明此问题的相关结论。你能否将该结论推广至任意多个变量的情形?


% !TeX root = ../../../book.tex

\subsection{习题}

\subsubsection*{温故知新}

以口头或书面的形式简要回答以下问题。这些问题全都基于你刚刚阅读的内容,如果忘记了具体定义、概念或示例,可以回顾相关内容。确保在继续学习之前能够自信地作答这些问题,这将有助于你的理解和记忆!

\begin{enumerate}[label=(\arabic*)]
    \item 在考虑 $\mathbb{Z}$ 模 $n$ 时,为何会将 $\mathbb{Z}$ 划分为若干集合?
    \item $\mathbb{Z}$ 模 $n$ 的等价类是什么?
    \item 如何判断两个整数 $x, y \in \mathbb{Z}$ 是否属于 $\mathbb{Z}$ 模 $n$ 的同一等价类?
    \item 模算术引理是什么?它为什么如此有用?如何利用它以代数的方式处理同余?
    \item 什么是\emph{乘法逆元}?给定 $a \in \mathbb{Z}$ 和 $n \in \mathbb{N}$,在 $\mathbb{Z}$ 模 $n$ 下,如何判定 $a$ 的乘法逆元是否存在?
    \item 当 $p$ 为\emph{质数}时,$\mathbb{Z}$ 模 $p$ 的等价类集合有何特殊性质?
    \item 中国剩余定理是否保证以下同余方程组有解?为什么?
    \begin{align*}
        x &\equiv 2 \mod 6 \\
        x &\equiv 5 \mod 9
    \end{align*} 
    能否求出该方程组的解?(\textbf{提示}:存在解。)
\end{enumerate}

\subsubsection*{小试牛刀}

尝试解答以下问题。这些题目需动笔书写或口头阐述答案,旨在帮助你熟练运用新概念、定义及符号。题目难度适中,确保掌握它们将大有裨益!

\begin{enumerate}[label=(\arabic*)]
    \item 请陈述并证明自然数 $x \in \mathbb{N}$ 能否被 $9$ 整除的判定方法。\\
    (\textbf{提示}:参见示例 \ref{ex:example6.5.13}。)
    \item 设 $n \in \mathbb{N}, a \in \mathbb{Z}$。证明 $ (n-a)^2 \equiv a^2 \mod n$。
    \item 设 $n \in \mathbb{N} - \{1\}$。证明 $ (n-1)^{-1} \equiv n - 1 \mod n$。
    \item 对下列各组 $(a, n)$,求 $a$ 模 $n$ 的乘法逆元,或说明其不存在。
    \begin{enumerate}[label=(\alph*)]
        \item $a = 5 \enspace\quad n = 12$
        \item $a = 7 \enspace\quad n = 11$
        \item $a = 6 \enspace\quad n = 27$
        \item $a = 11 \quad n = 18$
        \item $a = 70 \quad n = 84$
        \item $a = 8 \enspace \quad n = 17$
    \end{enumerate}
    \item 描述下面方程所有整数解 $x, y \in \mathbb{Z}$ 的特征。
        \[4x - 7y = 18\]
    \item 求以下同余方程组的所有解:
        \begin{align*}
            x \equiv 3 \mod 5 \\
            x \equiv 4 \mod 7
        \end{align*}
\end{enumerate}