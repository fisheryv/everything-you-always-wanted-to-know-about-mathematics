% !TeX root = ../../../book.tex

\subsection{定义与示例}

\subsubsection*{整除性}

我们从熟悉的基本定义开始。

\begin{definition}
    设 $a,b \in \mathbb{Z}$。若存在整数 $k \in \mathbb{Z}$ 使得 $b = ak$(等价地,当 $a \neq 0$ 时 $\frac{b}{a} \in \mathbb{Z}$),则称 $a$ \dotuline{整除} $b$(或 $b$ 可以被 $a$ 整除)。记作 $a \mid b$。
\end{definition}

注意,根据定义,任意非零整数均可整除 $0$(例如 $5 \mid 0$),但 $0$ 仅能整除自身(例如 $0 \nmid 5$ 但 $0 \mid 0$)。请思考这与``整除''的直观理解是否一致。定义中 $k \in \mathbb{Z}$ 的约束允许负数参与运算,因此 $-2 \mid 4$ 和 $8 \mid -24$ 均成立。

诸如 $2 \nmid 5$ 这样的表述揭示了整数间的关系,但并未涵盖所有情况。虽然没有整数 $k$ 满足 $2k=5$,但 $k=2$ 和 $k=3$ 分别给出 $4$ 和 $6$,都非常接近 $5$;而 $k = -100$ 则偏差很远。对于较小数字通常易于验证,但当数字较大时,例如 $7 \nmid 100000$(由质数性质可知),如何找到使 $7k$ 最接近 $100000$ 的 $k$ 呢?答案是否唯一?是否存在多个``合理''选项(如 $2 \nmid 5$ 的情形)?

关于上面第二个问题,为了简化问题,我们需要限定答案,使其只有唯一合理选项。这样才能避免在得到一个答案后还要担心是否还有其他答案。为此,我们将借鉴\emph{价格猜猜猜 (The Price Is Right)}\footnote{《价格猜猜猜》是美国历史上播出时间最长的一档电视节目。—— 译者注}的规则:寻找\emph{最接近但不超过}目标值的答案。比如在 $2 \nmid 5$ 的例子中,我们认为 $k = 2$ 是最佳估计,因为 $4 < 5$。同理,在 $7 \nmid 100000$ 的例子中,我们认为 $k = 14285$ 是最佳估计,因为 $7 \times 14285 = 99995 < 100000$。(注意,在这种情况下,有一个``更接近''的估计,但它超过了目标值,因此我们不予考虑。)

这就引出了另一个问题——如何获得此类估计?对给定 $a,b \in \mathbb{Z}$,可以递增计算 $a$ 的倍数直至超过 $b$,其前一个倍数即为最优解。估计的``准确性''范围在 $0$ 到 $a - 1$ 之间,当 $a$ 能整除 $b$ 时,准确性为 $0$。(注意:``超过''是指顺序关系 $>$,因此负数情形需谨慎处理次序关系。如 $2 \nmid -3$ 时取 $k=-2$,因为 $-4 \le -3$。)下述引理总结了递增计算 $a$ 的倍数直到找到 $b$ 的最佳估计的思路,并阐明在限定条件下,总存在唯一解。

\begin{lemma}[除法算法]\label{lemma6.5.2}
    设 $a,b \in \mathbb{Z}$。$\exists k, r \in \mathbb{Z}$ 使得 $ak + r = b$,其中 $0 \le r \le a - 1$。换句话说,对于任意两个整数,总能找到一个 $a$ 的倍数 $k$,使得 $k$ 与 $a$ 的乘积最接近但不超过 $b$,同时还会得到唯一一个余数 $r$。我们把 $r$ 称为``$b$ 除以 $a$ 的余数''或``$b$ 被 $a$ 除的余数''。
\end{lemma}

余数概念至关重要,我们会频繁使用。后续将基于余数比较定义等价关系。首先请证明此引理!

\begin{proof}
    留作习题 \ref{exc:exercises6.7.14}。
\end{proof}

之所以称之为除法\emph{算法},是因为它提供了一种\emph{求解}倍数和余数的\emph{机制}。这种方法虽然简单但非常有效,简而言之就是\emph{反复应用减法}。也就是说,给定 $a$ 和 $b$,不断地从 $b$ 中减去 $a$,例如先得到 $b - a$,再得到 $b - 2a$,然后是 $b - 3a$……依此类推,直到剩下一个介于 $0$ 到 $a$ 之间的余数。

\begin{example}
    让我们通过一个例子来展示这一过程。假设 $a = 8, b = 62$。我们不断地从 $62$ 中减去 $8$,结果是:
    \[62, 54, 46, 38, 30, 22, 14, 6\]
    最终在 $6$ 上停止,因为它满足 $0 \le 6 < a = 8$,这表明余数 $r = 6$。我们还注意到,我们总共从 $b$ 中减去了 $7$ 次 $a$,因为列表中有 $8$ 项,其中第一项是 $b - 0 \cdot a$。故有:
    \[\underbrace{62}_{b} = \underbrace{7}_{k} \cdot \underbrace{8}_{a} + \underbrace{6}_{r}\]
\end{example}

此过程表明余数存在且唯一。利用该结果,可定义 $\mathbb{Z}$ 上的特定关系。下文将证这些关系均为等价关系,并探讨其等价类的强大应用!

这里的重点是有一种方法可以找到余数,并且余数唯一。利用该结果,我们可以定义 $\mathbb{Z}$ 上的特定关系。下文将证明这些关系均为等价关系,并探讨其\emph{等价类}的强大应用!

\subsubsection*{模 $n$ 同余}

\begin{definition}\label{def:definition6.5.4}
    设 $n \in \mathbb{N}$。定义 $\mathbb{Z}$ 上的关系 $R_n$ 为 $(a,b) \in R_n$ 当且仅当 $a$ 和 $b$ 除以 $n$ 的余数相同,即
    \[(a,b) \in R_n \iff n \mid a-b\] 
    亦可记作
    \[a \equiv b \mod n\]
    读作``$a$ 与 $b$ \dotuline{模 $n$ 同余}''。(口头上通常将 ``modulo'' 简化为 ``mod''。)
\end{definition}

\begin{remark}
    定义中``$a$ 和 $b$ 除以 $n$ 的余数相同''等价于 $n \mid a - b$。此结论需要证明,稍后将在习题 \ref{exc:exercises6.7.15} 中完成论证。
\end{remark}

\begin{remark}
    实际应用(如解题与证明)中,$a \equiv b \mod{n}$ 表明可将 $a$ 表示为 $n$ 的倍数与 $b$ 之和。
\end{remark}

我们来看一下为什么这是成立的。假设二者的余数均为 $r$,这意味着存在 $k, \ell \in \mathbb{Z}$ 使得
\[a = kn + r \quad\text{且}\quad b = \ell n + r\]
(余数相同,但 $n$ 的倍数可能不同。)通过相减消去 $r$ 即可得到等式
\[a - kn = b - \ell n\]
然后移项并提取公因式可得
\[a = (k - \ell)n + b\]
此处 $(k - \ell)n$ 是 $n$ 的倍数,而第二项只包含 $b$ 本身。这说明 $a$ 是 $n$ 的倍数加上 $b$。

通常情况下,$b$ 可能并不是 $a$ 除以 $n$ 后的余数;特别是当 $b$ 不满足余数要求 $0 \le r \le a - 1$ 时,就会出现这种情况。

让我们总结一下此观点,并给出后续证明与示例中常用的\emph{模 $n$ 同余}的等价表述:

\setlength{\fboxrule}{2pt}
\begin{center}
\fcolorbox{olivegreen}{white}{%
    \parbox{0.8\textwidth}{%
        \[a \equiv b \mod n \iff \exists m \in \mathbb{Z} \centerdot a=mn+b\]
    }
}
\end{center}

\begin{example}
    让我们通过考察几个较小的 $n$ 值,来考察这些关系的具体表现。

    \begin{itemize}
        \item 设 $n=1$。关系 $R_1$ 会是什么样?这个问题实际上有些无聊,因为任何整数除以 $1$ 的余数都是 $0$,所以每个整数都可以和其他任意整数相关联。也就是说,$\forall x,y \in \mathbb{Z} \centerdot (x, y) \in R_1$。由于这个关系相对平凡,因此数学家们几乎不会讨论``模 $1$''这个话题。
        
        \item 设 $n=2$。关系 $R_2$ 就是我们之前定义的``奇偶关系''。想想为什么会这样。当我们把任意整数 $a$ 除以 $2$ 时,余数只能是 $0$ 或 $1$。如果 $a$ 和 $b$ 除以 $2$ 的余数都是 $0$,那么它们都是偶数;如果余数都是 $1$,那么它们都是奇数。(回想一下我们在第 \ref{ch:chapter03} 章中的定义,\emph{奇数}和\emph{偶数}是通过\emph{存在}声明来定义的。例如,当且仅当 $\exists k \in \mathbb{Z}$ 使得 $x = 2k$ 时,$x$ 为偶数。这正是除法算法的结果。当且仅当 $x$ 除以 $2$ 的余数为 $0$ 时,$x$ 为偶数,因为我们可以找到一个整数 $k$,使得 $x = 2k$。)\\
        现在,想想同余的另一种表述。如果两个整数都是偶数,那么它们的差也是偶数!也就是说,$a \equiv b \mod 2 \iff a - b \mid 2$;即 $a$ 和 $b$ 都是偶数(或者都是奇数)当且仅当它们的差也是偶数。(注意:我们还没有\emph{证明}这种表述确实等价于余数的定义。我们将在本例之后给出证明。)

        \item 设 $n=3$。例如,$0 \equiv 9 \mod 3, -1 \equiv 2 \mod 3$ 以及 $4 \equiv 28 \mod 3$。一般来说,只要在行尾加上``$\text{mod\ } 3$''(或其他数字),便可以连接多个同余语句。如此连接后,整行都按照模 $3$ 处理。例如,以下语句在符号上是有效的,在数学上也是成立的:
        \[-100 \equiv -1 \equiv 8 \equiv 311 \equiv -289 \equiv 41 \mod 3\]
        (虽然我们不确定为什么需要写出这样的陈述,但这样做是完全可以的!)

        \item 设 $n=10$。自然数除以 $10$ 的余数就是它的最后一位数字,即个位数字!这样我们就可以轻松地比较两个数模 $10$ 的余数。例如,$12 \equiv 32 \equiv 448237402 \mod 10$;而 $37457 \not\equiv 38201 \mod 10$。\\
        但对于\emph{负数}的情况会略有不同。因为我们定义余数时,是取\emph{不超过}目标值的最大倍数。例如,$-1 \equiv 9 \mod 10$,这是因为 $-1= (-1) \cdot 10 + 9$,而 $9 = (0) \cdot 10 + 9$。它们的余数都是 $9$,需要加到某个 $10$ 的倍数上。请思考以下陈述的具体细节:
        \[-3 \equiv 17 \equiv -33 \equiv 107 \mod 10\]
    \end{itemize}
\end{example}

\subsubsection*{符号}

需要强调的是,在数学中,\textbf{mod} 是一种关系,而非运算符或函数。在计算机科学和编程中,你可能会看到类似``\verb|5 mod 3 = 2|''这样的表达,它表示``$5$ 除以 $3$ 的余数是 $2$''。(在许多编程语言中,写作 \verb|5 % 3 = 2|)。但数学中不采用这种写法:我们使用 $\text{mod}$ 和 $\equiv$ 符号表示一种\textbf{等价关系},因为讨论的数字不一定\emph{相等}。当等价关系在某个自然数 $n$ 下成立时,我们在行末标注``$\text{mod\ } n$''以指明这一点。此时,$\text{mod}$ 相当于一个\emph{修饰符},表示``本行所有陈述仅对模 $n$ 的余数成立''。例如:
\[100 \equiv 97 \equiv 16 \equiv 4 \equiv z \cdot w \equiv 1 \equiv x - y \equiv -2 \equiv -8 \mod 3\]
这表示在 $\mod 3$ 下,所有数字和表达式等价。我们既不断言它们绝对相等,也不断言其在其他模数下等价;行末的``$\mod 3$''表示``仅在整数模 $3$ 的范围内讨论''。

(问题:你能找到 $x, y, z, w \in \mathbb{Z}$ 使上述等式成立吗?)

\subsubsection*{三个重要引理}

本节要求你证明两个关键结论:第一,模 $n$ 同余可以用\emph{可除性}等价定义;第二,该关系是等价关系。阅读时请完成对应的练习。掌握这些细节后,下一节关于等价类的内容会更容易理解。完成证明后,我们将展示并证明另一引理,并在讨论等价类之前给出一个同余的典型应用:它能简化手工计算繁琐的算术问题。

\begin{lemma}\label{lemma6.5.8}
    在定义 \ref{def:definition6.5.4} 中,模 $n$ 同余的两种表述等价。即对于所有 $a, b \in \mathbb{Z}$ 和 $n \in \mathbb{N}$,
    \[a, b \text{\ 除以\ } n \text{\ 余数相同\ } \iff n \mid a - b\]
\end{lemma}

\begin{proof}
    见习题 \ref{exc:exercises6.7.15}
\end{proof}

\begin{lemma}\label{lemma6.5.9}
    对于任意 $n \in \mathbb{N}, R_n$ 是 $\mathbb{Z}$ 上的等价关系。
\end{lemma}

\begin{proof}
    见习题 \ref{exc:exercises6.7.16}
\end{proof}

感谢你证明了这些引理!$\smiley{}$ 现在我们知道模 $n$ 同余是等价关系(因此可以讨论等价类),且判断两个整数(如 $a$ 和 $b$)模 $n$ 同余等价于验证 $a - b$ 是否是 $n$ 的\emph{倍数},这是一种高效的判定方法。

下一引理表明:``模 $n$''意义下的加法和乘法\textbf{保持同余性}。若有两个整数等式,如 $a + b = c$ 和 $d + e = f$,相加可得 $a + b + d + e = c + f$。此引理说明该原理对模 $n$ 同余同样成立——同余式可相加或相乘,同余关系仍然成立。

尽管该引理的证明并不复杂,但我们将替你完成,因为本节你已经做了太多工作了。

\begin{lemma}[模算术引理 (Modular Arithmetic Lemma, 简称 MAL)]\label{lemma6.5.10}
    设 $n \in \mathbb{N}$。设 $a,b,r,s \in \mathbb{Z}$ 为任意固定整数。若 $a \equiv r \mod n$ 且 $b \equiv s \mod n$,则
    \begin{align*}
        a + b &\equiv r + s \mod n \\
        a \cdot b &\equiv r \cdot s \mod n
    \end{align*}
\end{lemma}

(该引理表明只需处理余数:无论给定的 $a$ 和 $b$ 是什么,均可简化为余数 $r$ 和 $s$ 后进行运算。由于 $0 \le r, s \le n - 1$,其值比 $a$ 和 $b$ 更小,因此可以加速实际运算。以下证明确保该方法普遍成立。)

\begin{proof}
    假设 $a \equiv r \mod n$ 且 $b \equiv s \mod n$。这意味着 $\exists k, \ell \in \mathbb{Z}$ 使得
    \begin{align*}
        a &= kn + r \\
        b &= \ell n + s
    \end{align*}
    \begin{itemize}
        \item 两式相加得
            \[a + b = (kn + r) + (\ell n + s) = (k + \ell)n + (r + s)\]
            因为可以将 $a+b$ 表示为 $n$ 的倍数加上余数 $r+s$,所以 $a + b \equiv r + s \mod n$。
        \item 两式相乘得
            \[a \cdot b =  (kn + r) \cdot (\ell n + s) = k\ell n^2 + (ks + \ell r)n + r \cdot s = n \cdot (k\ell n + ks + \ell r)+r \cdot s\]
            因为可以将 $a \cdot b$ 表示为 $n$ 的倍数加上余数 $r \cdot s$,所以 $a \cdot b \equiv r \cdot s \mod n$。
    \end{itemize}
\end{proof}

\begin{remark}
    请注意,此处仅讨论加法和乘法,未涉及\textbf{减法}和\textbf{除法},原因有二:

    其一,减法本质是``加负数''。该引理表明,对同余作减法可通过两步实现:
    \begin{enumerate}[label=(\arabic*)]
        \item 将其中一个同余乘以 $-1$(应用\emph{乘法}引理)
        \item 将结果相加(应用\emph{加法}引理)
    \end{enumerate}
    此过程巧妙地结合了两个引理的结果。

    其二较为复杂。实际上,模 $n$ 运算中无``除法''运算。主要原因在于讨论范围仅限于\emph{整数},而除法可能产生非整数的\emph{有理数}。例如 $4 \equiv 7 \mod 3$,但 $\frac{4}{2} \equiv \frac{7}{2} \mod 3$ 无意义——整数(如 $2$)不能与非整数(如 $\frac{7}{2}$)同余。因此在 $\mathbb{Z}$ 模 $n$ 系统中不定义\textbf{除法}。

    ``除法''问题存在更精细的讨论,将在 \ref{sec:section6.5.3} 节讨论\emph{乘法逆元}时详述。为了避免混淆,此处暂不展开。简而言之,后续将发展一种在特定条件下类似模 $n$ 除法的方法。

    综上,为了保持\emph{整数}范畴,本书仅讨论加法与乘法。
\end{remark}

\subsubsection*{两个实用例子}

我们可能尚未完全说服你认识到模算术的妙用。为了展示同余作为等价关系兼具数学趣味与实用价值,这里将展示两个典型示例。第一个例子中,模算术提供了比常规方法更简洁的解法;第二个则是你可能用过却未曾深究的巧妙技巧——我们将证明其有效性。

\begin{example}
    思考以下问题:

    是否\emph{存在}自然数 $k$,使得 $5^k$ 恰好比 $7$ 的倍数多 $1$?如果存在,这样的自然数最小是多少?能否描述所有满足条件的自然数?

    我们可以尝试代入不同的 $k$ 值来寻找答案,但很快会发现:大指数的计算十分繁琐,而验证一个大数是否满足条件更为困难。如果你愿意的话可以自行尝试,甚至借助计算器探索解法。

    不过,我们更倾向反复使用模算术引理 (Modular Arithmetic Lemma, MAL)。由于指数的本质是重复乘法,可以反复运用乘法引理。核心思路是:在连续乘以 $5$ 的过程中,始终保持对 $7$ 取模。我们只需寻找模 $7$ 余 $1$ 的数,无需直接计算其具体值。过程如下:

    从 $5^1 \equiv 5 \mod 7$ 开始。将其乘以 $5$ 得
    \[5^2 \equiv 5 \cdot 5 \equiv 25 \equiv 4 \mod 7\]
    注意到 $25 = 21 + 4$,并且已知 $21$ 是 $7$ 的倍数,从而得出上述结论。(当数字较小时,我们常常可以通过简单观察直接心算。当然,如果不确定,也可以使用除法算法,从 $25$ 中不断减去 $7$,直到剩下余数。)

    继续乘以 $5$ 得
    \[5^3 \equiv 5^2 \cdot 5 \equiv 4 \cdot 5 \equiv 20 \equiv 6 \mod 7\]
    我们发现,通过``观察''可知 $20 = 14+6$。请注意,求 $5^3$ 模 $7$ 的余数并不需要实际计算 $5^3 = 125$ 然后再取模。因为我们在此过程中已经将所有数字都化简到模 $7$ 的余数,所以省去了大量计算。具体来说,我们总是将数字化简到\emph{小于} $7$ 的范围内,因此在任何情况下我们需要处理的最大数字也只会在 $20$ 到 $30$ 之间。这真是太方便了!让我们继续看接下来会得到什么结果:
    \begin{align*}
        5^4 &\equiv 5^3 \cdot 5 \equiv 6 \cdot 5 \equiv 30 \equiv 2 \mod 7 \\ 
        5^5 &\equiv 5^4 \cdot 5 \equiv 2 \cdot 5 \equiv 10 \equiv 3 \mod 7 \\
        5^6 &\equiv 5^5 \cdot 5 \equiv 3 \cdot 5 \equiv 15 \equiv 1 \mod 7 
    \end{align*}
    这正是我们要找的结果!我们已经确定 $5^6$ 比 $7$ 的倍数多 $1$。这种方法比直接计算 $5^6 = 15625$ 并找出 $15625 = 7 \cdot 2232 + 1$ 要简单得多。

    至此解答了前两问:存在满足条件的 $5$ 的幂,且由 $k=1$ 逐步推导可知 $k=6$ 为最小值。第三问留作思考:请继续乘以 $5$ 并观察余数规律。你能否发现模式?提出猜想并尝试证明!(后续将回归此例……)
\end{example}

\begin{example}\label{ex:example6.5.13}
    考虑数字 $474$。它是 $3$ 的倍数吗?只需将各位数字相加:$4 + 7 + 4 = 15$。由于 $15$ 是 $3$ 的倍数,可知 $474$ 也必然是 $3$ 的倍数。(当然,也可通过长除法验证 $474 = 3 \cdot 158$。)但为什么这种方法成立?难道仅仅是因为老师曾在三年级时告诉了你这个方法,你就记住了?这对我们来说远远不够!$\smiley{}$

    这里,我们将严格\textbf{证明}:自然数 $x$ 能被 $3$ 整除当且仅当其各位数字之和能被 $3$ 整除。(证明中括号内的例子仅用于辅助理解。需要注意的是,具体实例不能替代普遍结论的证明。)

    \begin{proof}
        设 $x \in \mathbb{N}$ 为任意固定自然数。其十进制展开形式为
        \[x= \sum_{k=0}^{n-1} x_k \cdot 10^k\]
        其中 $n$ 为数字 $x$ 的位数,$x_k$ 为 $10^k$ 位对应的数字,所以 $0 \le x_k \le 9$。(即 $x_k$ 是从右向左第 $(k+1)$ 位数字。)

        (例如,$47205$ 可写做 $47205=4 \cdot 10^4+7 \cdot 10^3+2 \cdot 10^2+0 \cdot 10^1+5 \cdot 10^0$。本例中,$x_0 = 5, x_1=0, x_3=2$。)

        该整除技巧声称
        \[x \equiv 0 \mod 3 \iff \sum_{k=1}^{n-1} x_k \equiv 0 \mod 3\]

        为了证明这一点,我们将考虑十进制展开模 $3$。注意,由于 $10=9+1$,因此 $10 \equiv 1 \mod 3$。故
        \[\forall k \in \mathbb{N} \cup \{0\} \centerdot 10^k \equiv 1^k \equiv 1 \mod 3\]

        (此结论基于模算术引理和 $1^k = 1$ 对任意 $k$ 恒成立。请思考一下!)

        由此可将十进制展开式中的 $10^k$ 替换为 $1$:

        \begin{align*}
            x \equiv 0 \mod 3 &\iff \sum_{k=0}^{n-1} x_k \cdot 10^k \equiv 0 \mod 3 &\text{将\ } x \text{\ 重写为十进制展开形式}\\
            &\iff \sum_{k=0}^{n-1} x_k \cdot 1^k \equiv 0 \mod 3 &\text{因为\ } 10 \equiv 1 \mod 3 \\
            &\iff \sum_{k=0}^{n-1} x_k \equiv 0 \mod 3
        \end{align*}

        证毕。
    \end{proof}

    (注意,$3 \mid 47205$ 成立是因为 $3 \mid (4 + 7 + 2 + 0 + 5)$,也就是说 $3 \mid 18$。实际上,$15735 \cdot 3 = 47205$)。

    值得注意的是,此证明揭示了\textbf{更强的结论}:由于上述陈述为\emph{当且仅当}陈述,若 $x$ 的各位数字之和不是 $3$ 的倍数,则 $x$ 也不是 $3$ 的倍数,且两者模 $3$ 同余。例如 $3 \nmid 122$,因为 $3 \nmid 5$ 且 $5 \equiv 2 \mod 3$,因此 $122 \equiv 2 \mod 3$。(验证可得 $122 = 3 \cdot 40 + 2$。)

    类似规则对 $9$ 和 $11$ 同样成立($11$ 的规则略微复杂)。甚至存在 $7$ 的整除规则,但表述较为繁琐。本章习题将进一步探讨这些内容。

    请熟记该结论及其证明。这是一个可以在聚会上炫耀的小技巧。你可以向朋友发起挑战:他们真的知道\textbf{为什么}该技巧有效吗?你却能洞悉其本质!
\end{example}
