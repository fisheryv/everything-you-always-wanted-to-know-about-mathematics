% !TeX root = ../../../book.tex

\subsection{定义与示例}

\subsubsection*{整除性}

我们将从一个我们已经多次见过的定义开始。

\begin{definition}
    设 $a,b \in \mathbb{Z}$。我们说 $a$ 整除 $b$ 是指 $b$ 可以被 $a$ 整除,即 $\exists k$ 使得 $b = ak$,或者等价地,$\frac{b}{a} \in \mathbb{Z}$(排除 $a=b=0$ 的情况)。记作 $a \mid b$。
\end{definition}

注意,这个定义表明每个整数都能整除 $0$(例如,$5 \mid 0$),但 $0$ 除了自己之外不能整除任何数(例如,$0 \nmid 5$ 但 $0 \mid 0$)。想一想这是否符合你对``整除''的直觉理解,同时也满足了给定的定义。另外,这里也考虑了负数的情况,因为存在量词意味着\emph{整数} $k \in \mathbb{Z}$。因此,$-2 \mid 4$ 和 $8 | -24$ 也成立。

现在,像 $2 \nmid 5$ 这样的表达告诉我们一些关于整数 $2$ 和 $5$ 之间关系的信息,但并不能涵盖所有情况。我们知道没有整数 $k$ 可以满足 $2k = 5$,但这并没有说明我们到底能多接近。显然,$k = -100$ 是一个糟糕的估计,但 $2 \times 2 = 4$ 和 $2 \times 3 = 6$ 都非常接近 $5$……对于这样的小数字,这似乎很明显,我们可以手工验证,但对于巨大的数字呢?我们知道 $7 \nmid 100000$(为什么?想想质数……),但我们如何解决这个``找到使 $7k$ 最接近 $100000$ 的 $k$''这个问题呢?我们怎么知道是否有一个具体的答案?会不会有两个同样``合理''的答案,就像 $2 \nmid 5$ 一样?

关于上面第二个问题,为了简化,我们希望限制答案,使其只有一个合理的选项。这是为了避免在找到一个答案后还要担心是否有其他答案。因此,我们将采用\emph{价格猜猜猜}\footnote{价格猜猜猜 (The Price Is Right) 是美国历史上风行时间最长的一档电视节目。--- 译者注}的规则:我们寻找\emph{最接近但不超过}目标值的答案。比如在 $2 \nmid 5$ 的例子中,我们认为 $k = 2$ 是最佳估计,因为 $4 < 5$。同理,在 $7 \nmid 100000$ 的例子中,我们认为 $k = 14285$ 是最佳估计,因为 $7 \times 14285 = 99995 < 100000$。(注意,在这种情况下,有一个``更接近''的估计,但它超过了目标值,因此我们不考虑它。)

这就引出了我们如何得到这样的估计。给定 $a, b \in \mathbb{Z}$,我们可以查看 $a$ 的越来越大的倍数,直到超过 $b$;在此之前的那个最大倍数就是最佳估计。估计的``准确性''范围在 $0$ 到 $a - 1$ 之间,当 $a$ 能整除 $b$ 时,准确性为 $0$。(注意:``超过''是指顺序关系 $>$,所以要仔细考虑这在负数中的应用。比如,$2 \nmid -3$ 而 $2 \times -2 = -4$ 被认为是最佳估计,因为 $-4 \le -3$。)以下引理总结了关于逐步查看 $a$ 的倍数直到找到 $b$ 的最佳估计的思路,并声明在我们设定的约束条件下,总会有唯一解。

\begin{lemma}[除法算法]\label{lemma6.5.2}
    设 $a,b \in \mathbb{Z}$。则 $\exists k, r \in \mathbb{Z}$ 使得 $ak + r = b$,其中 $0 \le r \le a - 1$。换句话说,对于任意两个整数,总能找到一个 $a$ 的倍数,使得它与 $a$ 的乘积最接近 $b$ 而不超过 $b$,同时还存在一个唯一的余数。我们把这个 $r$ 称为``$b$ 除以 $a$ 的余数'' 或 ``$b$ 被 $a$ 除的余数''。
\end{lemma}

我们会频繁使用余数的概念。具体来说,我们将比较两个除法的余数,并基于余数定义一种关系。稍后我们会详细介绍这些内容。首先,请你证明这个重要的引理!

\begin{proof}
    留作习题 \ref{exc:exercises6.7.14}。
\end{proof}

之所以称之为除法\emph{算法},是因为它暗示了一种\emph{找到}这些倍数和余数的\emph{过程}。这种方法虽然简单但非常有效,就是\emph{反复应用减法}。也就是说,给定 $a$ 和 $b$,我们可以不断地从 $b$ 中减去 $a$,例如先得到 $b - a$,再得到 $b - 2a$,然后是 $b - 3a$……依此类推,直到剩下一个介于 $0$ 和 $a$ 之间的余数。\\

\begin{example}
    让我们通过一个例子来展示这个过程。假设 $a = 8, b = 62$。我们不断地从 $62$ 中减去 $8$,结果是:
    \[62, 54, 46, 38, 30, 22, 14, 6\]
    我们停在 $6$ 上,因为它满足 $0 \le 6 < a = 8$,这表明 $r = 6$。我们还注意到,我们总共从 $b$ 中减去了 $7$ 次 $a$,因为列表中有八个项,其中第一项是 $b - 0 \cdot a$。因此,我们可以写出:
    \[\underbrace{62}_{b} = \underbrace{7}_{k} \cdot \underbrace{8}_{a} + \underbrace{6}_{r}\]
\end{example}

这里的重点是有一种方法可以找到这个余数,并且这个余数是唯一的。有了这个结果,我们可以用它来定义某些 $\mathbb{Z}$ 上的关系。接下来,我们将展示这些关系都是等价关系,并具体看看它们的\emph{等价类}是多么有用!

\subsubsection*{模 $n$ 同余}

\begin{definition}\label{def:definition6.5.4}
    设 $n \in \mathbb{N}$。我们定义 $\mathbb{Z}$ 上的关系 $R_n$ 为 $(a,b) \in R_n$ 当且仅当 $a$ 和 $b$ 除以 $n$ 时有相同的余数,即
    \[(a,b) \in R_n \iff n \mid a-b\] 
    写法上,我们也将其写成
    \[a \equiv b \mod n\]
    读作``$a$ 与 $b$ \dotuline{模 $n$ 同余}''。(口头上,我们通常将 ``modulo'' 简化为 ``mod''。)
\end{definition}

\begin{remark}
    我们在定义中提到,``$a$ 和 $b$ 除以 $n$ 后有相同的余数''等同于$n \mid a - b$''。为什么会这样呢?这并非定义本身的原因,而是需要一些证明的。稍后你将在习题 \ref{exc:exercises6.7.15} 中进行这个证明。
\end{remark}

\begin{remark}
    在实际应用中(例如解决问题和证明其他结论时),我们会这样使用这个定义:已知 $a ≡\equiv b \mod n$ 意味着我们可以将 $a$ 表示为 $n$ 的倍数加上 $b$。
\end{remark}

我们来看一下为什么这是成立的。假设它们的余数都为 $r$,这意味着存在 $k, \ell \in \mathbb{Z}$ 使得
\[a = kn + r \quad\text{且}\quad b = \ell n + r\]
(它们有相同的余数,但 $n$ 的倍数可能不同。)通过相减来解出 $r$ 这样我们就能得到等式
\[a - kn = b - \ell n\]
然后移项并提取公因式可得
\[a = (k - \ell)n + b\]
瞧!$(k - \ell)n$ 是 $n$ 的倍数,而第二项只有 $b$ 本身。这说明 $a$ 是 $n$ 的倍数加上 $b$。

通常情况下,$b$ 可能并不是 $a$ 除以 $n$ 后的余数;特别是当 $b$ 不满足余数要求 $0 \le r \le a - 1$ 时,就会出现这种情况。

让我们总结一下这个观点,并写下我们将来会用到的定义形式。这是我们在证明或举例中引用\emph{模 $n$ 同余}定义时会用到的陈述:

\setlength{\fboxrule}{2pt}
\begin{center}
\fcolorbox{olivegreen}{white}{%
    \parbox{0.8\textwidth}{%
        \[a \equiv b \mod n \iff \exists m \in \mathbb{Z} \centerdot a=mn+b\]
    }
}
\end{center}

\begin{example}
    让我们通过考察几个较小的 $n$ 值,来看看这些关系的具体表现。

    \begin{itemize}
        \item 设 $n=1$。关系 $R_1$ 会是什么样?这个问题实际上有些无聊,因为任何整数除以 $1$ 的余数都是 $0$,所以每个整数都可以和其他任意整数相关联。也就是说,$\forall x,y \in \mathbb{Z} \centerdot (x, y) \in R_1$。因为这个关系相对不那么有趣,因此数学家们几乎不会讨论``模 $1$''这个话题。
        \item 设 $n=2$。关系 $R_2$ 就是我们之前定义的``奇偶关系''。想想为什么会这样。当我们把任意整数 $a$ 除以 $2$ 时,余数只能是 $0$ 或 $1$。如果 $a$ 和 $b$ 除以 $2$ 的余数都是 $0$,那么它们都是偶数;如果余数都是 $1$,那么它们都是奇数。(回想一下我们在第 \ref{ch:chapter03} 章中的定义,\emph{奇数}和\emph{偶数}是通过\emph{存在}声明来定义的:例如,当且仅当 $\exists k \in \mathbb{Z}$ 使得 $x = 2k$ 时,$x$ 为偶数。这正是除法算法的结果:当且仅当 $x$ 除以 $2$ 的余数为 $0$ 时,$x$ 为偶数,因为我们可以找到一个整数 $k$,使得 $x = 2k$。)\\
        现在,想想同余的另一种表述。如果两个整数都是偶数,那么它们的差也是偶数!也就是说,$a \equiv b \mod 2 \iff a - b \mid 2$;即 $a$ 和 $b$ 都是偶数(或者都是奇数)当且仅当它们的差也是偶数。(注意:我们还没有\emph{证明}这种表述确实等价于余数的定义。我们将在这个例子之后立即进行证明。)
        \item 设 $n=3$。例如,$0 \equiv 9 \mod 3, -1 \equiv 2 \mod 3$ 以及 $4 \equiv 28 \mod 3$。一般来说,只要在行尾加上``$\mod 3$''(或其他数),我们可以连接多个同余语句。当这样做时,整行都按照模 $3$ 处理。例如,以下语句在符号上是有效的,在数学上也是成立的:
        \[-100 \equiv -1 \equiv 8 \equiv 311 \equiv -289 \equiv 41 \mod 3\]
        (虽然我们不确定为什么需要写这样的陈述,但这样做是完全可以的!)
        \item 设 $n=10$。自然数除以 $10$ 的余数就是它的最后一位数字,也就是个位数字!这样我们就可以轻松地比较两个数的模 $10$ 余数。例如,$12 \equiv 32 \equiv 448237402 \mod 10$;而 $37457 \not\equiv 38201 \mod 10$。\\
        但对于\emph{负数}的情况就有所不同了。因为我们定义余数时,是取\emph{不超过}目标值的最大倍数。例如,$-1 \equiv 9 \mod 10$,这是因为 $-1= (-1) \cdot 10 + 9$,而 $9 = (0) \cdot 10 + 9$。它们的余数都是 $9$,需要加到某个 $10$ 的倍数上。请思考以下陈述的具体细节:
        \[-3 \equiv 17 \equiv -33 \equiv 107 \mod 10\]
    \end{itemize}
\end{example}

\subsubsection*{符号}

需要强调的是:在数学中,\textbf{mod} 是一种关系,而不是运算符或函数。在计算机科学和编程中,你可能会看到类似``$5$ mod $3 = 2$''的表达,这表示``$5$ 除以 $3$ 的余数是 $2$''。(在许多编程语言中,这可能表示为 \verb|5 % 3 = 2|。)在这里,我们不会这样写。我们使用 $\mod$ 和 $\equiv$ 符号表示某种\textbf{等价},因为我们讨论的数字不一定\emph{相等}。如果我们表达的等价链在某个自然数 $n$ 下是有意义的,我们会在行末写上``$\mod n$''来指出这一点。在这个意义上,$\mod$ 更像是一个\emph{修饰符},用来表示``这一行的所有陈述仅在除以 $n$ 的余数上有意义''。因此,我们可以写类似这样的表达
\[100 \equiv 97 \equiv 16 \equiv 4 \equiv z \cdot w \equiv 1 \equiv x - y \equiv -2 \equiv -8 \mod 3\]
这表示当考虑 $\mod 3$ 时,所有这些数字和表达式都是等价的。我们并没有断言它们是相等的,也没有断言它们在其他情况下是等价的。行末的 ``$\mod 3$'' 表示``我们只在整数模 $3$ 的范围内讨论。''

(问题:你能找到 $x, y, z, w \in \mathbb{Z}$ 使上面的等式成立吗?)

\subsubsection*{三个重要引理}

在这里,我们将要求你证明两个重要结论:首先,证明模 $n$ 同余可以用\emph{可除性}来等价理解;其次,证明这些关系是等价关系。在阅读本节时,请完成这些对应的练习。如果你已经掌握了这些细节,下一节关于这些关系下的等价类的内容会更容易理解。在这两个证明之后,我们还会展示并证明另一个结果。在讨论等价类之前,最后一个例子是一个有趣的算术问题,用同余可以轻松解决,但如果手工计算就不那么容易了。

\begin{lemma}\label{lemma6.5.8}
    在定义 \ref{def:definition6.5.4} 中,模 $n$ 同余的两种表述确实是等价的。也就是说,对于所有 $a, b \in \mathbb{Z}$ 和所有 $n \in \mathbb{N}$,
    \[a, b \;\text{除以}\; n \;\text{余数相同} \iff n \mid a - b\]
\end{lemma}

\begin{proof}
    见习题 \ref{exc:exercises6.7.15}
\end{proof}

\begin{lemma}\label{lemma6.5.9}
    对于任意 $n \in \mathbb{N}, R_n$ 是 $\mathbb{Z}$ 上的等价关系。
\end{lemma}

\begin{proof}
    见习题 \ref{exc:exercises6.7.16}
\end{proof}

感谢你证明了这些引理!$\smiley{}$ 现在我们知道模 $n$ 同余是一个等价关系(因此我们可以讨论等价类)。我们还了解到,要判断两个整数 (如 $a$ 和 $b$) 是否模 $n$ 同余,只需确定 $a - b$ 是否是 $n$ 的\emph{倍数}即可。这是一种验证同余关系是否成立的有效方法。

下一个引理告诉我们,在``模 n''的情况下进行加法和乘法\textbf{运算},结果仍然正确。如果我们有两个关于整数的等式,并将它们相加,结果仍然是正确的。也就是说,如果 $a + b = c$ 且 $d + e = f$,我们可以得到 $a + b + d + e = c + f$。这个引理说明,同样的原理适用于模 $n$ 的同余关系。同理,我们可以对同余关系进行乘法运算,并且同余关系仍然成立。

虽然这个引理的证明并不复杂,但我们会为你证明它,因为最近我们让你做了太多工作了。

\begin{lemma}[模算术引理 (Modular Arithmetic Lemma, 简称 MAL)]\label{lemma6.5.10}
    设 $n \in \mathbb{N}$。设 $a,b,r,s \in \mathbb{Z}$ 为任意固定整数。假设 $a \equiv r \mod n$ 且 $b \equiv s \mod n$。则
    \begin{align*}
        a + b &\equiv r + s \mod n \\
        a \cdot b &\equiv r \cdot s \mod n
    \end{align*}
\end{lemma}

(这个引理告诉我们,我们只需要处理余数。无论给定的 $a$ 和 $b$ 是什么,我们可以将它们化简为余数 $r$ 和 $s$,然后对这些余数进行计算。因为 $0 \le r, s \le n - 1$,所以它们相对于 $a$ 和 $b$ 来说是较小的,这使得我们在实际操作中可以更快地进行算术运算。以下证明保证了这种方法在所有情况下都有效。)

\begin{proof}
    假设 $a \equiv r \mod n$ 且 $b \equiv s \mod n$。这意味着 $\exists k, \ell \in \mathbb{Z}$ 使得
    \begin{align*}
        a &= kn + r \\
        b &= \ell n + s \\
    \end{align*}
    将上面两个等式相加得
    \[a + b = (kn + r) + (\ell n + s) = (k + \ell)n + (r + s)\]
    因为我们可以将 $a+b$ 表示为 $n$ 的倍数加上余数 $r+s$,所以 $a + b \equiv r + s \mod n$。\\ \\
    将上面两个等式相乘得
    \[a \cdot b =  (kn + r) \cdot (\ell n + s) = k\ell n^2 + (ks + \ell r)n + r \cdot s = n \cdot (k\ell n + ks + \ell r)+r \cdot s\]
    因为我们可以将 $a \cdot b$ 表示为 $n$ 的倍数加上余数 $r \cdot s$,所以 $a \cdot b \equiv r \cdot s \mod n$。
\end{proof}

\begin{remark}
    请注意,我们在此没有提到\textbf{减法}和\textbf{除法},而是只讨论了加法和乘法。这有两方面的原因。首先,减法实际上是``加一个负数''。因此,这个引理表明我们可以通过以下两个步骤来\emph{减去}两个同余:
    \begin{enumerate}[label=(\arabic*)]
        \item 将其中一个同余乘以 $-1$(应用\emph{乘法}引理)
        \item 将结果相加(应用\emph{加法}引理)
    \end{enumerate}
    看到它是如何\emph{同时}利用这两个引理的结果了吗?很巧妙,对吧?

    第二个原因稍微复杂一些。实际上,在模 $n$ 的情况下没有``除法''这种运算。主要原因是我们这里讨论的范围仅限于\emph{整数},而除法可能会产生非整数的\emph{有理数}。例如,我们知道 $4 \equiv 7 \mod 3$,但这是否意味着 $\frac{4}{2} \equiv \frac{7}{2} \mod 3$ 呢?这又意味着什么呢?一个整数(例如 $2$)怎么可能与一个非整数(例如 $\frac{7}{2}$)同余呢?正是因为这个原因,我们在 $\mathbb{Z}$ 模 $n$ 的环境中不讨论\textbf{除法}运算。

    关于这个``除法''问题,其实还有一些更细微的细节。我们会在 \ref{sec:section6.5.3} 节讨论\emph{乘法逆元}时详细说明。现在为了避免引起混淆,我们暂时不讨论这些细节。简单来说,我们将会发展出一种在某些特定情况下类似于模 $n$ 除法的方法。

    同时,为了确保我们只讨论\emph{整数},我们将\emph{仅}涉及加法和乘法。
\end{remark}

\subsubsection*{两个实用例子}

我们还不确定是否已经让你相信模算术的用处。为了确保我们已经证明了同余作为等价关系的概念既有数学趣味又有实用价值,我们将在这里举两个有趣且实用的例子。第一个例子是一个简单的问题,用模算术可以比用``标准''算术更容易解决。第二个例子是一个你可能以前用过但从未考虑过其原理的巧妙技巧。我们会证明它的有效性!\\ \\

\begin{example}
    考虑如下问题:
    \begin{center}
        \parbox{0.8\textwidth}{%
            \textbf{问题:}

            是否\emph{存在}自然数 $k$,使得 $5^k$ 恰好比 $7$ 的倍数多 $1$?

            如果存在,这样的自然数最小是多少?

            你能描述所有满足该条件的自然数吗?
        }
    \end{center}
    我们可以尝试通过代入 $k$ 的值来回答这些问题,看看会发生什么。然而,你很快会注意到,计算大指数会很麻烦,而要确定一个大数是否正好比$7$ 的某个倍数多 $1$ 会更加困难!如果你愿意的话可以继续尝试一下。甚至可以使用计算器。看看你能否解决这个问题!

    不过,我们更倾向于这样做:多次利用模算术引理 (Modular Arithmetic Lemma, MAL)。指数运算只是重复的乘法,因此我们可以反复应用该引理的乘法结论。其核心思想是,我们可以不断乘以 $5$,并在此过程中将所有结果模 $7$。也就是说,我们只需要找到一个比 $7$ 的倍数多 $1$ 的数 --- 即模 $7$ 同余 $1$ 的数 --- 而不需要立即知道该数具体是多少,只需判断它是否满足这一性质。接下来我们将展示此过程。

    我们从 $5^1 \equiv 5 \mod 7$ 开始。将其乘以 $5$ 得
    \[5^2 \equiv 5 \cdot 5 \equiv 25 \equiv 4 \mod 7\]
    我们发现 $25 = 21 + 4$,并且知道 $21$ 是 $7$ 的倍数,从而得出上述结论。(当数字较小时,我们常常可以通过简单观察进行算术运算。也就是说,我们可以直接心算。当然,如果不确定,我们也可以使用除法算法,从 $25$ 中不断减去 $7$,直到剩下余数。)

    接着我们发现
    \[5^3 \equiv 5^2 \cdot 5 \equiv 4 \cdot 5 \equiv 20 \equiv 6 \mod 7\]
    我们发现,通过``观察''可以知道 $20 = 14+6$。请注意,我们现在知道 $5^3$ 模 $7$ 的余数是多少,但并不需要实际计算 $5^3 = 125$ 然后再化简。因为我们在此过程中已经将所有数字都化简到模 $7$ 的余数,所以省去了大量计算。具体来说,我们总是将数字化简到\emph{小于} $7$ 的范围内,因此在任何情况下我们需要处理的最大数也只会在 $20$ 到 $30$ 之间。这真是太方便了!让我们继续看看接下来会得到什么结果:
    \begin{align*}
        5^4 &\equiv 5^3 \cdot 5 \equiv 6 \cdot 5 \equiv 30 \equiv 2 \mod 7 \\ 
        5^5 &\equiv 5^4 \cdot 5 \equiv 2 \cdot 5 \equiv 10 \equiv 3 \mod 7 \\
        5^6 &\equiv 5^5 \cdot 5 \equiv 3 \cdot 5 \equiv 15 \equiv 1 \mod 7 
    \end{align*}
    这正是我们要找的结果!我们已经确定 $5^6$ 比 $7$ 的某个倍数多 $1$。这种方法比直接计算 $5^6 = 15625$ 并找出 $15625 = 7 \cdot 2232 + 1$ 要简单得多,不是吗?

    这已经解决了前两个问题:我们发现存在 $5$ 的幂满足所需性质,并且由于我们是从 $k = 1$ 开始逐步找到它的,因此可以确定这是最小的结果。第三个问题留给你来研究,即描述所有满足该性质的数。你可以继续我们的过程,乘以 $5$ 并进行化简。你是否注意到了某种模式?它是什么?尝试提出一个猜想并加以证明!(我们稍后会回到这个例子……)
\end{example}

\begin{example}\label{ex:example6.5.13}
    考虑数字 $474$。它是 $3$ 的倍数吗?也许你只是将它的各位数字加起来 --- $4 + 7 + 4$ 等于 $15$ --- 并注意到 $15$ 是 $3$ 的倍数,从而得出结论 $474$ 也必然是 $3$ 的倍数。(当然,你也可以通过长除法计算出 $474 = 3 \cdot 158$。)为什么你可以这样做呢?是不是因为你的老师在三年级时告诉了你这个方法,你就记住了?这对我们来说远远不够!$\smiley{}$

    这里,我们将严格\textbf{证明},一个自然数 $x$ 能被 $3$ 整除当且仅当其各位数字之和也能被 $3$ 整除。(在证明过程中,包含了一些用具体例子来详细说明的语句。这些例子是为了帮助你理解我们所写的内容,我们将其放在括号中,以提醒你,仅仅展示一个例子并不是一个正式的证明。例子可以帮助读者更容易理解实际的证明,但仅凭一个例子不足以证明这个普遍适用的结论。)

    \begin{proof}
        设 $x \in \mathbb{N}$ 为任意固定自然数。我们可以用它的十进制展开形式来表示这个数字
        \[x= \sum_{k=0}^{n-1} x_k \cdot 10^k\]
        其中 $n$ 为数字 $x$ 的位数,$x_k$ 为对应 $10^k$ 位的数字,所以 $0 \le x_k \le 9$。(也就是说,$x_k$ 是从右往左读取的 $x$ 的第 $(k + 1)$ 位数。)

        (例如,我们可以将 $47205$ 写做 $47205=4 \cdot 10^4+7 \cdot 10^3+2 \cdot 10^2+0 \cdot 10^1+5 \cdot 10^0$。本例中,$x_0 = 5, x_1=0, x_3=2$。)

        整除技巧声称
        \[x \equiv 0 \mod 3 \iff \sum_{k=1}^{n-1} x_k \equiv 0 \mod 3\]

        为了证明这一点,我们将考虑十进制展开模 $3$。注意,由于 $10=9+1$,因此 $10 \equiv 1 \mod 3$。因此
        \[\forall k \in \mathbb{N} \cup \{0\} \centerdot 10^k \equiv 1^k \equiv 1 \mod 3\]

        (这源于模算术引理和 $1^k = 1$ 对任意 $k$ 都成立的事实。思考一下!)

        这使我们能够在十进制展开中用 $1$ 替换 $10$ 的幂!因此

        \begin{align*}
            x \equiv 0 \mod 3 &\iff \sum_{k=0}^{n-1} x_k \cdot 10^k \equiv 0 \mod 3 &\text{将}\; x \;\text{重写为十进制展开形式}\\
            &\iff \sum_{k=0}^{n-1} x_k \cdot 1^k \equiv 0 \mod 3 &\text{因为}\; 10 \equiv 1 \mod 3 \\
            &\iff \sum_{k=0}^{n-1} x_k \equiv 0 \mod 3
        \end{align*}

        以上完成了该声明的证明。
    \end{proof}

    (注意,$3 \mid 47205$ 是因为 $3 \mid (4 + 7 + 2 + 0 + 5)$,也就是说 $3 \mid 18$。实际上,$15735 \cdot 3 = 47205$)。

    有趣的是,我们实际上在这里证明了一个\textbf{更强的}结论。因为前面的陈述\emph{当且仅当}陈述,所以我们知道更多的信息:如果 $x$ 的各位数字之和不是 $3$ 的倍数,那么 $x$ 也不是 $3$ 的倍数,并且有\emph{相同的余数}。例如,$3 \nmid 122$,因为 $3 \nmid 5$;此外,$5 \equiv 2 \mod 3$,所以我们知道 $122 \equiv 2 \mod 3$。(确实,$122 = 3 \cdot 40 + 2$。)

    我们还可以找到并证明类似的关于 $9$ 和 $11$ 的整除技巧(虽然 $11$ 的技巧稍微复杂一些)。甚至还有一个关于 $7$ 的技巧,但很难用文字表达。这些概念将在本章的习题中进一步探讨。
    
    记住这个结论及其证明。这是一个可以在聚会上炫耀的小技巧。你可以挑战你的朋友:他们真的知道\textbf{为什么}这个技巧有效吗?你却知其然也知其所以然!
\end{example}
