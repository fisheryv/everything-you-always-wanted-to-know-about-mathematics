% !TeX root = ../../../book.tex

\subsection{乘法逆元}\label{sec:section6.5.3}

在证明模运算引理(引理 \ref{lemma6.5.10})时曾提及,我们不会在 $\mathbb{Z}$ 模 $n$ 中讨论``除法''。本节将重新探讨该问题,解释为何(及如何)在特定情况下``除法''是可行的。但需要强调的是,我们实际讨论的是更广义的\textbf{乘法逆元}概念,而非真正的``除法''。首先通过启发性示例阐明该概念,再陈述并证明特定情况下的具体结论。

\subsubsection*{整体概念}

给定数学对象,其\textbf{乘法逆元}是另一个对象,二者``相乘''结果为``$1$''。此处引号强调``相乘''与``$1$''的含义随语境而变化。

\begin{example}
    先看一个熟悉的例子,考虑实数集 $\mathbb{R}$ 及其标准乘法。取数字 $2$,它的乘法逆元是什么?也就是说,是否存在另一个实数 $x$ 使得 $2 \cdot x = 1$?如果存在,它是多少?很明显,$x = \frac{1}{2}$ 符合要求!因为 $2 \cdot \frac{1}{2} = 1$。因此可以记作
    \[\text{在\ } \mathbb{R} \text{\ 中} \quad 2^{-1} = \frac{1}{2} \]
    解方程时两边同时除以 $2$,本质是同时\emph{乘以} $2$ 的\emph{乘法逆元}。
\end{example}

\begin{example}
    再看一个可能不太熟悉的例子。想象有一个挂钟,钟面上均匀分布着 $12$ 个小时的刻度标记,以顶部 $12$ 点的位置为标准位置``$1$''。也就是说,这是没有进行额外旋转的标准表示法,故称其为\emph{单位元}。实际上,这里的``$1$''对应``$0 \degree$ 旋转''后的挂钟。

    现在,定义旋转的``乘法''为连续旋转的复合。例如,我们先将时钟顺时针旋转 $45 \degree$,然后再顺时针旋转 $90 \degree$,则 ``$45 \degree$ 旋转''\emph{乘以}``$90 \degree$ 旋转'',结果得 ``$135 \degree$ 旋转''。

    建立这些约定的目的是为了明确我们的上下文、对象、``相乘''的含义以及``1''的定义,从而能够识别任意旋转的\emph{乘法逆元}。在此框架下,不难发现,如果将``$\theta \degree$ 旋转''与``$(360-\theta) \degree$ 旋转''相乘,实际上是将时钟旋转了 $360 \degree$,从而回到标准位置``$1$''。这意味着,$\theta \degree$ 旋转的乘法逆元为
    \[(\theta \degree \text{\ 旋转})^{-1} = (360-\theta)\degree \text{\ 旋转}\]
\end{example}

以上示例表明,\emph{逆元}是一个普适概念,并不局限于数值\emph{除法}这种标准语境。事实上,当我们讨论\emph{函数的逆}时,也会看到类似的例子。(在函数语境中,``相乘''指函数的复合,``$1$''指恒等函数。相关细节将在后续章节展开,此处提及是为了便于已经熟悉此概念的读者提前建立联系。)

\subsubsection*{互质}

你可能已经熟悉以下定义。我们将在后续结论中用到它,该结论描述了在 $\mathbb{Z}$ 模 $n$ 下乘法逆元存在的条件,因此我们在此重申定义并给出若干示例。

\begin{definition}
    给定 $x,y \in \mathbb{Z}$,称 $x$ 与 $y$ \dotuline{互质}当且仅当二者无除 $1$ 以外的公因子。
\end{definition}

(\textbf{注意}:``互质''是指 $x$ 和 $y$ 互为质数关系,并非指 $x$ 具有``类质数''性质。)

\begin{example}
    例如,$12$ 与 $35$ 互质,因为 $12 = 2^2 \cdot 3$,而 $35 = 5 \cdot 7$,二者无公共质因数。

    通常,列出\emph{质因数分解}有助于判断互质,因为互质性取决于两数是否包含公共质因数(这意味着存在非平凡公因数)。

    作为反例,$12$ 和 $33$ 不互质,因为 $3 \mid 12$ 且 $3 \mid 33$。
\end{example}

\begin{example}
    以下结论将在后续讨论中发挥重要作用。

    \textbf{声明}:若 $p$ 为素数且 $a$ 不是 $p$ 的倍数,则 $p$ 与 $a$ 互质。

    (也就是说,若 $p$ 为素数且 $p \nmid a$,则 $p$ 与 $a$ 互质。)

    证明如下:

    \begin{proof}
        设 $p$ 为素数且 $a \in \mathbb{Z}$。假设 $p \nmid a$。

        由 $p \nmid a$ $p$ 不是 $a$ 的质因数。因为 $p$ 为素数,因此 $a$ 的质因数均不能整除 $p$。这意味着 $a$ 和 $p$ 无公共质因数,故二者互质。
    \end{proof}

    这一性质非常实用!特别是,当 $p$ 为素数时,\textbf{所有}整数 $1, 2, 3, \dots , p-1$ 均与 $p$ 互质。
\end{example}

\subsubsection*{定义与示例}

我们讨论在 $\mathbb{Z}$ 模 $n$ 下的\emph{乘法逆元}。这里的``乘法''指常规乘法,但结果需要模 $n$。此外,``$1$''对应于 $1$ 的\emph{等价类}。此时,对于任意 $x \in \mathbb{Z}$,其乘法逆元(记作 $x^{-1}$)是满足 $xy \equiv 1 \mod n$ 的 $y$。即:
\[\forall x \in \mathbb{Z} \centerdot \forall y \in \mathbb{Z} \centerdot y \equiv x^{-1} \mod n \iff xy \equiv 1 \mod n\]
请注意,此定义仅在 $\mathbb{Z}$ 模 $n$ 下成立,因此不会写作``$y = x^{-1}$''。$x$ 表示等价类,$x^{-1}$ 亦是如此。

我们进一步练习如何\emph{寻找}乘法逆元或判断其不存在性。核心在于:
\begin{quotation}
    若 $x \cdot y \equiv 1 \mod n$,则对于所有 $k \in \mathbb{Z}, x \cdot (y+kn) \equiv 1 \mod n$
\end{quotation}
要想理解其中的原因,可以将右侧表达式中的 $x$ 利用分配律展开:
\[x \cdot (y+kn) \equiv xy+xkn \equiv xy+n(xk) \equiv xy+0 \equiv xy \equiv 1 \mod n\]
因为 $xkn$ 是 $n$ 的倍数,模 $n$ 时可以``忽略''。

由此可得:\textbf{若} $x$ 在模 $n$ 下有乘法逆元,\textbf{则}
\begin{enumerate}[label=(\alph*)]
    \item 存在\emph{无穷多}个逆元,且它们都属于同一个模 $n$ 等价类;
    \item 在集合 $\{1, 2, 3, \dots , n-1\}$ 中,存在\emph{唯一}逆元。
\end{enumerate}

这些结论兼具理论价值与实用性。特别是,它表明无需复杂的存在性证明:只需穷举检查集合 $\{1, \dots, n-1\}$ 中的元素。若未找到逆元,则逆元不存在。换言之,我们可以采用系统性的试错法代替直觉猜测。以下示例将展示其具体应用:

\begin{example}
    给定 $n \in \mathbb{N}$ 和 $x \in \mathbb{Z}$,求满足 $y \equiv x^{-1} \mod n$ 的 $y$。若逆元不存在,则说明原因。
    \begin{itemize}
        \item $\mathbf{n=3, x=2}$:\\
            只需检查 $y = 1$ 和 $y = 2$。因为 $2 \cdot 2 \equiv 4 \equiv 1 \mod 3$,所以
            \[2^{-1} \equiv 2 \mod 3\]
        \item $\mathbf{n=4, x=3}$:\\
            只需检查 $y = 1, y = 2, y = 3$。因为 $3 \cdot 3 \equiv 9 \equiv 1 \mod 4$,所以
            \[3^{-1} \equiv 3 \mod 4\]
        \item $\mathbf{n=4, x=2}$:\\
            只需检查 $y = 1$ 和 $y = 2$。然而,由于 $x$ 为偶数,所以 $x$ 的任何倍数也都是偶数,而满足 $y \equiv 1 \mod 4$ 的数必然为奇数。因此,$2$ 在模 $4$ 下没有乘法逆元。
        \item $\mathbf{n=10, x=3}$:\\
            穷举所有情况可得:
            \begin{align*}
                3 \cdot 1 &\equiv 3 \mod 10 \\
                3 \cdot 2 &\equiv 6 \mod 10 \\
                3 \cdot 3 &\equiv 9 \mod 10 \\
                3 \cdot 4 \equiv 12 &\equiv 2 \mod 10 \\
                3 \cdot 5 \equiv 15 &\equiv 5 \mod 10 \\
                3 \cdot 6 \equiv 18 &\equiv 8 \mod 10 \\
                3 \cdot 7 \equiv 21 &\equiv 1 \mod 10
            \end{align*}
            这意味着
            \[3^{-1} \equiv 7 \mod 10\]
            由乘法交换律(顺序无关),同时有
            \[7^{-1} \equiv 3 \mod 10\]
            由此可以得出以下结论:
            \[(a^{-1})^{-1} \equiv a \mod n \quad \text{若\ } a^{-1} \text{\ 存在}\]
        \item $\mathbf{n=15, x=7}$:\\
            检查 $7$ 的所有倍数,会发现当检查到 $13$ 时:
            \[7 \cdot 13 \equiv 91 \equiv 6 \cdot 15 + 1 ≡ 1 \mod 15\]
            所以
            \[7^{-1} \equiv 13 \mod 15\]
            验证工作留给你来完成。例如在模 $15$ 下 $6$ 没有乘法逆元。 
    \end{itemize}
\end{example}

\subsubsection*{何时存在乘法逆元?}

在研究了若干例子后,是时候静下心来,深入探讨所有乘法逆元存在的情形。以下引理完整描述了这些条件。

\begin{lemma}[互质时的乘法逆元引理或 MIRP 引理]\label{lemma6.5.24}
    若 $n \in \mathbb{M}, a \in \mathbb{Z}$ 且 $n, a$ \dotuline{互质},则存在 $x \in \mathbb{Z}$ 满足同余式 $a \cdot x \equiv 1 \mod n$。

    事实上,该同余式有无穷多个解,且所有解都模 $n$ 同余。这意味着在集合 $[n - 1] = \{1, 2, \dots, n-1\}$ 中恰有一个解。

    我们用 $a^{-1}$ 表示该同余方程解的等价类,并称其为 $a$ 模 $n$ 的\dotuline{乘法逆元}。

    此外,这是一个充要条件;也就是说,若 $a$ 与 $n$ 不互质,则同余方程 $a \cdot x \equiv 1 \mod n$ 在整数范围内无解。
\end{lemma}

该引理完整刻画了乘法逆元的存在性。例如,我们可以用它来判定同余式
\[15x \equiv 1 \mod 33\]
在 $x \in \mathbb{Z}$ 下\textbf{无解},因为 $3 \mid 15$ 且 $3 \mid 33$,二者不互质。同理,我们可以用它来判断同余式
\[40x \equiv 1 \mod 51\]
在 $x \in \mathbb{Z}$ 下\textbf{必有解},因为 $40 = 2^3 \cdot 5$ 与 $51 = 3 \cdot 17$ 互质。

(注意,该引理仅保证解存在于 $\{1, 2, \dots , n-1\}$,未提供具体解法。)

为\textbf{证明}此引理(其为双向条件),我们分两部分进行。现证明一个方向,即当 $a$ 与 $n$ 互质时,$a^{-1}$ 模 $n$ 存在。另一方向(若 $a$ 与 $n$ 有公因子,则 $a^{-1}$ 模 $n$ 不存在)将在习题 \ref{exc:exercises6.7.21} 中引导完成。(你可以现在就试试!)证明需用到以下引理。

\begin{lemma}[欧几里得引理]\label{lemma6.5.25}
    给定 $a, b, c \in \mathbb{Z}$。假设 $a \mid bc$,且 $a$ 与 $b$ 互质。则 $a \mid c$。
\end{lemma}

我们将推迟此引理的证明,先完成 MIRP 引理的证明。我们认为,详细探讨欧几里得引理可能会分散本节的焦点,并且欧几里得引理的结果本身直观可信,所以我们暂时假设其成立,并在 MIRP 引理的证明中直接使用它。请看以下几个例子:
\begin{itemize}
    \item 已知 $3 \mid 30$ 且 $30 = 5 \cdot 6$。由于 $3$ 和 $5$ 互质,则可以推断 $3 \mid 6$,而这显然成立。
    \item 假设对于某个整数 $x$ 满足 $3 \mid 5x$。我们能得出关于 $x$ 的什么结论?由于 $3$ 和 $5$ 互质,所以要使 $5x$ 为 $3$ 的倍数,则 $x$ 必须``包含'' $3$ 的因子。也就是说,$3 \mid x$ 是必要条件。
\end{itemize}
上述说明并不严格,我们也无意要求你无条件\emph{接受}该引理;我们只是暂缓详细证明。建议你尝试独立证明!

现在,我们继续证明 MIRP 引理(在证明过程中将用到欧几里得引理的结果)。

\begin{proof}
    设 $n \in \mathbb{N}, a \in \mathbb{Z}$,且 $a$ 与 $n$ \textbf{互质}。我们要证明 $\exists x \in \mathbb{Z} \centerdot ax \equiv 1 \mod n$。

    考虑 $a$ 的前 $n$ 个倍数组成的集合;也就是说,定义集合 $N$ 为
    \begin{align*}
        N &= \{0, a, 2a, 3a, \dots ,(n-1)a\} \\
        &= \{z \in \mathbb{Z} \mid \exists k \in [n - 1] \cup \{0\} \centerdot z = ka\}
    \end{align*}

    请注意,集合 $N$ 中有 $n$ 个元素。

    \textbf{声明}:集合 $N$ 的所有元素在模 $n$ 运算下会产生\emph{不同}余数,即
    \[\forall i, j \in [n-1] \cup \{0\} \centerdot i \ne j \implies ai \not\equiv aj \mod n\]

    首先,为了引出矛盾而假设该声明为\verb|假|。

    这意味着 $\exists i, j \in [n-1] \cup \{0\} \centerdot ai \equiv aj \mod n$。给定这样的 $i$ 和 $j$。

    通过减法和因式分解,可得 $ai - aj \equiv a(i-j) \equiv 0 \mod n$。

    这意味着 $n \mid a(i-j)$。已知 $n$ 与 $a$ 互质。根据欧几里得引理 \ref{lemma6.5.25},可以推导出 $n \mid (i-j)$。

    由于 $i, j \in [n-1] \cup \{0\}$,因此 $0 \le i, j \le n-1$,也就是说 $-(n-1) \le -j, ij \le 0$。

    将这些关于 $i$ 和 $-j$ 的不等式相加后,可得
    \[-(n-1) + 0 = n-1 \le i + (-j) = i - j \le n - 1 = (n-1) + 0\]

    也就是说,$-(n-1) \le i-j \le n-1$。又已知 $n \mid (i - j)$,即 $i-j$ 是 $n$ 的倍数。然而,在 $-(n-1)$ 和 $+(n-1)$ 之间,$n$ 的倍数\emph{只有} $0$。

    因此 $i-j=0$,即 $i=j$。这与假设矛盾。声明得证。

    综上,$N$ 的元素在模 $n$ 下会产生 $n$ 个不同余数。由于余数集合 $\{0, 1, \dots, n-1\} = [n-1] \cup \{0\}$ 恰有 $n$ 个元素,因此每个余数在 $N$ 中\emph{恰好出现一次}。

    这表明,$N$ 中\emph{恰好}有一个元素(即一个 $a$ 的倍数)对应于模 $n$ 余数为 $1$。这个 $N$ 中的元素可以表示为 $ax$,其中 $x \in [n-1] \cup \{0\}$。此 $x$ 就是引理中所述同余方程的解。
\end{proof}

经过这番努力,我们终于完成了证明。既然已经解决了练习 \ref{exc:exercises6.7.21}(确实如此,对吧?$\smiley{}$),我们现已\emph{确切}掌握 $\mathbb{Z}$ 模 $n$ 下乘法逆元的存在条件,并知晓一种可行的解法:检查 $a$ 的前 $n-1$ 个倍数——$ a, 2a, \dots, (n-1)a$——找出其中模 $n$ 余 $1$ 的倍数。

现在转向欧几里得引理的证明。这一步不可或缺,因为 MIRP 引理依赖于该结果。注意,其证明包含一个复杂的\emph{归纳论证}。具体来说,由于存在\emph{两个变量} $a$ 和 $b$,因此需要证明命题对 $a$ 和 $b$ 的所有情形都成立。

\begin{proof}
    设 $a, b, c \in \mathbb{Z}$。假设 $a \mid bc$,且 $a$ 和 $b$ 互质。

    我们要证明必有 $a \mid c$。先证明以下声明:

    \textbf{声明}:若 $a, b \in \mathbb{N}$ 且 $a$ 与 $b$ 互质,则 $\exists x, y \in \mathbb{Z} \centerdot ax+by = 1$。

    基于此声明,结果很容易得出。我们将该声明的证明放在框中,以方便阅读。待证明完成后,你会看到我们如何使用此结论来证明引理的原始陈述。

    (在证明之前,可以先通过几个例子来``说服''自己该命题为\verb|真|。取两个互质的数,例如 $(5,11), (15,22)$ 或 $(10, 23)$,尝试构造\emph{线性组合}来得到 $1$。然后,取两个有公因子的数,例如 $(5,10), (6, 15)$ 或 $(21, 27)$,尝试理解为什么无法构造出满足要求的线性组合。)

    \begin{tcolorbox}[colback=gray!10,%gray background
        colframe=black,% black frame colour
        width=\textwidth,% Use 5cm total width,
        arc=2mm, auto outer arc,
        title={证明声明},breakable,enhanced jigsaw,
        before upper={\parindent15pt\noindent},	]
            下面通过对 $a+b$ 进行归纳来证明这一点。在开始之前,先明确若干事实:
            \begin{itemize}
                \item 若 $a=1$ 或 $a=-1$,则 $b$ 必须为 $0$ 或 $1$ 才能满足二者互质。\\
                    无论哪种情况,我们都可以令 $x=a, y=0$,从而得到
                    \[ax + by = a^2 + 0 = 1\]
                    同样的论证亦适用于 $b=1$ 或 $b=-1$ 的情况(此时 $a$ 必为 $0$ 或 $1$)。
                \item 若 $b=0$ 且 $|a| \ge 2$(即 $a \ne \pm 1$),则 $a, b$ 有公因子 $a$,二者不互质。\\
                    同样的论证亦适用于 $a=0$ 的情况。
            \end{itemize}
            综上,我们可以忽略 $a$ 或 $b$ 为 $0$ 的情况,只需考虑 $|a| \ge 1$ 和 $|b| \ge 1$ 的情形。\\

            \begin{itemize}
                \item 因为 $a$ 与 $b$ 互质,所以 $-a$ 与 $b$ 也互质(同理,$-a$ 与 $b$ 以及 $a$ 与 $b$ 也都互质)。这是因为取相反数只会改变整数的符号,不会影响其因子。
                \item 若已知 $\exists x, y \in \mathbb{Z} \centerdot ax + by = 1$,则必有
                    \[(-a)(-x) + (-b)(-y) = ax + by = 1\]
                    因为 $-x,-y \in \mathbb{Z}$,这说明 $-a$ 和 $-b$ 也可以有这样的表示。
            \end{itemize}
            综上,我们只需要考虑 $a$ 和 $b$ 为\emph{正数}的情况。(换句话说,如果 $a$ 或 $b$ 为负数,我们只需取它们的相反数即可。)\\

            结合之前的推论,我们可以推断出只需要考虑 $a,b \in \mathbb{N}$ 的情况。证明后,再结合上述观察,即可得出完整结论。

            现在,通过对 $a + b$ 应用(强)归纳法来进行证明。由于 $a,b \in \mathbb{N}$,因此 $a + b \ge 2$。我们之前已经考虑了基本情况 $a + b = 2$,但为了完整性,这里再重述一遍。

            给定 $a,b \in \mathbb{N}$,定义 $P(a,b)$ 为陈述
            \[a \text{\ 与\ } b \text{\ 互质} \implies \exists x, y \in \mathbb{Z} \centerdot ax + by = 1\]

            \textbf{基本情况}:考虑 $P(2)$,即假设 $a,b \in \mathbb{N}$,$a$ 与 $b$ 互质且 $a+b=2$。这意味着 $a=b=1$,我们可以令 $x=1, y=0$ 得出 $ax + by = 1 + 0 = 1$。因此,$P(2)$ 成立。

            \textbf{归纳假设}:设 $k \in \mathbb{N}$ 为任意固定自然数。假设 $P(2) \land P(3) \land \dots \land P(k)$ 成立。(也就是说,假设当两个互质数之和等于 $2,3, \dots, k$ 时,我们都能找到它们的\emph{线性组合}使其等于 $1$。)

            \textbf{归纳步骤}:我们要证明 $P(k+1)$ 成立。也就是说,设 $a, b \in \mathbb{N}$,且 $a + b = k + 1$,并假设 $a$ 与 $b$ 互质;我们要证明 $\exists x, y \in \mathbb{Z} \centerdot ax + by = 1$。

            首先,根据对称性,我们可以假设 $a \ge b$。(也就是说,给定 $a$ 和 $b$ 的值。无论它们是什么值,我们都可以对它们进行``重命名'',因为其中一个值至少与另一个一样大;我们将较大的那个标记为 $a$。)事实上,由于 $a$ 与 $a$ 不互质(当 $a \ge 2$ 时),我们甚至可以假设 $a > b$。

            现在,我们要利用 $b$ 与 $a - b$ 互质这一事实。要理解为什么会这样,我们需要证明 $b$ 和 $a - b$ 只有 $1$ 这个公约数。

            设 $d$ 为 $b$ 和 $a - b$ 的公因数,即 $d$ 能整除 $b$ 和 $a-b$。这意味着 $d$ 能整除 $b + (a-b)$,即 $d$ 能整除 $a$。已知 $d$ 能整除 $b$,因此 $d$ 实际上是 $a$ 和 $b$ 的公因数,所以它必然为 $1$。因此,$b$ 与 $a-b$ 互质。

            (我们刚刚证明的是
            \[(d \mid b \land d \mid a - b) \implies d \mid a \land d \mid b\]

            该声明是一个 $\iff$ 陈述。思考一下为什么 $\impliedby$ 方向也成立。)

            我们现在有 $b, a-b \in \mathbb{N}$(因为 $b < a$)互质。还要注意,$b + (a-b) = a < a + b = k + 1$,因为 $b \in \mathbb{N}$(所以 $b \ge 1$)。这意味着 $a + b \le k$,因此归纳假设 $P(a + b)$ 适用!

            (注意 $P(a + b)$ 不一定等同于 $P(k)$,因此我们需要使用强归纳法!)

            陈述 $P(a + b)$ —— 即 $P(b + (a - b))$,我们将用到这一点 —— 告诉我们 $b$ 和 $a - b$ 的线性组合可以得到 $1$;即,
            \[\exists u, v \in \mathbb{Z} \centerdot ub + v(a - b) = 1\]

            接下来将其转换为 $a$ 和 $b$ 的线性组合,从而得到 $1$。为此,我们重新写该方程,并重新标记系数:
            \[ub + v(a-b) = 1 \iff \underbrace{v}_{x} a + b \underbrace{(u - v)}_{y} = 1\]

            即定义 $x = v$ 和 $y = u - v$,这样 $x, y \in \mathbb{Z}$,并且满足 $ax + by = 1$。

            我们已经证明了 $P(a + b)$(即 $P(k + 1)$)成立。通过强归纳法,我们推导出 $P(n)$ 对于所有 $n \in \mathbb{N}$ 且 $n \ge 2$ 都成立。
    \end{tcolorbox}	

    上述论证证明:任意互质整数必有线性组合得 $1$。

    回到原命题。已知 $a, b, c \in \mathbb{N}$,并假设 $a$ 与 $b$ 互质且 $a \mid bc$。

    第一个假设说明存在整数 $x$ 和 $y$,使得 $ax + by = 1$。给定这样的 $x$ 和 $y$。

    第二个假设说明存在整数 $k$,使得 $bc = ak$。给定这样的 $k$。
    
    接下来,我们将第一个假设中的方程乘以 $c$,然后应用第二个假设,可得:
    \[ax + by = 1 \implies acx + (bc)y = 1 \implies acx + (ak)y = c \implies c = a\underbrace{(cx + ky)}_{\ell}\]

    也就是说,通过 $ax + by = 1$,我们可以推导出 $c = a\ell$,其中 $\ell \in \mathbb{Z}$。%并且 $\ell$ 是由其他整数定义的。

    根据定义,这意味着 $a \mid c$。这就证明了原命题。
\end{proof}

哇!这个证明包含了很多内容。建议读者多读几遍,逐行理解并做好笔记。请确保理解每个声明如何基于已知内容,并观察归纳法的应用方式:尽管涉及两个变量,但我们仅对其中一个变量(定义为另两个变量的和)进行归纳。这是一个复杂的证明,因此将其置于本节更重要的 MIRP 引理之后。

现在让我们利用这个结果——\emph{准确}刻画乘法逆元的存在条件——来解决一系列问题。

\subsubsection*{乘法逆元的应用}

乘法逆元有何作用?答案看似简单却至关重要:它在模运算中为求解同余方程提供了关键工具。初看可能以为这些数学工具专为同余问题而设计,实则不然。事实上,正如后续示例所示,尝试解决此类问题时,读者很可能自行推导出我们将要使用的技术。换言之,即使尚未系统学习乘法逆元的概念,通过求解这些问题,最终也会重新发现已有结论。

铺垫至此,现在进入具体问题。这些问题均表现为以下形式:``给定一个同余方程;求得其所有整数解或证明无解。''

\begin{example}\label{ex:example6.5.26}
    求所有整数 $x, y \in \mathbb{Z}$ 满足
    \[3x - 7y = 11\]

    该方程存在无穷多组解 $(x, y) \in \mathbb{Z} \times \mathbb{Z}$。我们将给出解的显式参数化形式,并通过定义解集完整描述。

    通过重写给定方程,将目标转化为求得所有满足下面同余式的 $x \in \mathbb{Z}$:
    \[3x \equiv 11 \mod 7\]

    若能找到所有满足条件的 $x \in \mathbb{Z}$,则对应 $y \in \mathbb{Z}$ 可通过 $y = \frac{3x-11}{7}$ 直接求出。

    注意到 $3^{-1} \equiv 5 \mod 7$(因为 $3 \cdot 5 \equiv 15 \equiv 2 \cdot 7 + 1 \equiv 1 \mod 7$)。根据模算术引理,两边同乘以 $3^{-1}$ 得:
    \begin{align*}
        \forall x \in \mathbb{Z} \centerdot 3x \equiv 11 \mod 7 &\iff 3^{-1} \cdot 3 \cdot x \equiv 3^{1}
\cdot 11 \mod 7 \\
        &\iff 1 \cdot x \equiv 5 \cdot 4 \mod 7 \\
        &\iff x \equiv 20 \equiv 6 \mod 7
    \end{align*}

    由于已知 $3^{-1}$ 表示该同余式的所有解(即它代表 $3$ 模 $7$ 乘法逆元的等价类),我们可以推导出:
    \[\forall x \in \mathbb{Z} \centerdot 3x \equiv 11 \mod 7 \iff x \equiv 6 \mod 7 \iff \exists k \in \mathbb{Z} \centerdot x = 7k + 6\]

    这完整刻画了给定方程解集中 $x \in \mathbb{Z}$ 的所有可能取值。

    现在,进一步确定对应的 $y \in \mathbb{Z}$ 值:设 $k \in \mathbb{Z}$,且 $x = 7k + 6$,代入 $x$ 得

    \[y = \frac{3x-11}{7} = \frac{3(7k + 6)-11}{7} = \frac{21k+7}{7} = 3k+1\]

    至此,我们得到了方程所有解的显式形式:对任意 $k \in \mathbb{Z}$,可以唯一确定 $x$ 和 $y$。由于推导全程使用 $\iff$ 等价关系,此形式已穷尽所有解。

    给定方程的解集 $S$ 可以描述为
    \[S = \{(x, y) \in \mathbb{Z} \times \mathbb{Z} \mid \exists k \in \mathbb{Z} \centerdot (x, y) = (7k + 6, 3k + 1)\}\]
\end{example}

\begin{tcolorbox}[colback=gray!10,
    colframe=black,
    width=\textwidth,
    arc=2mm, auto outer arc,
    title={有趣的事实},breakable,enhanced jigsaw,
    before upper={\parindent15pt\noindent},	]
    在这个例子中,我们解决了一个\textbf{线性丢番图方程},并给出了它的所有解。所谓\emph{线性},是指变量 $x$ 和 $y$ 的次数为一次,不含平方项或立方项。

    通过本例中使用的技术,你可以求解\emph{任意}线性丢番图方程,或判断其是否有解。我们还将\emph{证明}这类方程无解的条件(见贝祖恒等式,定理 \ref{theorem6.5.31})。只要方程有解,该方法即适用。

    在下一个例子中,我们将研究\textbf{二次丢番图方程},其中变量包含平方项(如 $x^2$ 和 $y^2$)。之后将讨论这类方程的求解可能性。
\end{tcolorbox}

\begin{example}
    本例延续前例的思路(利用乘法逆元简化运算),但引入了二次残差的概念。

    \textbf{声明}:不存在整数 $x, y \in \mathbb{Z}$ 满足方程
    \[3x^2-5y^2=1\]

    给定 $x, y \in \mathbb{Z}$。我们要证明 $3x^2-5y^2=1$ \emph{无整数解}。

    首先将给定方程改写为 $3x^2 = 5y^2+1$

    因为 $5y^2 \equiv 0 \mod 5$,这意味着 $3x^2 \equiv 1 \mod 5$

    因为 $3 \cdot 2 = 6 = 5 + 1$,所以 $3^{-1} \equiv 2 \mod 5$。等式两边同乘以 $3^{-1}$,简化得:
    \[3x^2 \equiv 1 \mod 5 \iff 3^{-1} \cdot 3x^2 \equiv 3^{-1} \cdot 1 \mod 5 \iff x^2 \equiv 2 \mod 5\]

    回顾示例 \ref{ex:example6.5.15},在那里我们研究了\emph{二次残差}。我们发现,模 $5$ 的二次残差集合为 $\{0, 1, 4\}$。也就是说,\emph{不存在}整数 $x$ 满足 $x^2 \equiv 2 \mod 5$。这表明原方程无整数解。
\end{example}

\begin{tcolorbox}[colback=gray!10,
    colframe=black,
    width=\textwidth,
    arc=2mm, auto outer arc,
    title={有趣的事实},breakable,enhanced jigsaw,
    before upper={\parindent15pt\noindent},	]
    我们已阐明线性丢番图方程可解的条件及解法。但对\textbf{二次丢番图方程},情况截然不同:判定其是否有解极为困难,即使存在解,实际求解也异常复杂。

    事实上,对于二次丢番图方程,\textbf{不存在任何计算机算法能对含一次项与二次项的方程判定解的存在性}。这里判断解的存在性都不可能,更别说具体求解了。说来你可能不信,该结论正是大名鼎鼎的\href{https://en.wikipedia.org/wiki/Hilbert's_tenth_problem}{希尔伯特第十问题}的一种形式。

    不过请放心,本书提供的所有涉及丢番图方程的例子和练习均可通过书中所述技术进行分析。上述结论是针对所有二次丢番图方程的一般性结论。
\end{tcolorbox}

\subsubsection*{一点群论知识}

在这一小节中,我们将简要介绍当前主题背后蕴含的重要数学原理。由于篇幅和时间有限,我们无法全面展开讨论,因此仅通过概念和实例进行说明。

我们想传达的核心思想是:当考虑 $\mathbb{Z}$ 模 $p$($p$ 为\textbf{质数})时,会出现特殊现象。此时,每个小于 $p$ 的数都与 $p$ \emph{互质},因为 $p$ 只有 $1$ 和其自身作为因子。这意味着集合 $\{1, 2, \dots, p-1\}$ 中所有元素在模 $p$ 意义下均存在乘法逆元。因此,除 $[0]_{\mod p}$ 外,每个同余类都有对应的乘法逆元类。

例如,考虑 $p=5$,易得:
\begin{align*}
    1^{-1} \equiv 1 \mod 5\\
    2^{-1} \equiv 3 \mod 5\\
    3^{-1} \equiv 2 \mod 5\\
    4^{-1} \equiv 4 \mod 5
\end{align*}

再比如,考虑 $p=7$,易得:
\begin{align*}
    1^{-1} \equiv 1 \mod 7\\
    2^{-1} \equiv 4 \mod 7\\
    3^{-1} \equiv 5 \mod 7\\
    4^{-1} \equiv 2 \mod 7\\
    5^{-1} \equiv 3 \mod 7\\
    6^{-1} \equiv 6 \mod 7
\end{align*}

请注意,这表明集合中所有元素均有乘法逆元。

(同时注意,这些逆元构成集合 $\{1, 2, \dots, p-1\}$ 的一个\emph{排列}。这并非巧合!尝试证明其成因,并试着证明存在两个自逆元,即 $1^{-1} \equiv 1 \mod p$ 和 $(p - 1)^{-1} \equiv p - 1 \mod p$,而其余元素均\emph{不可能}为自逆元。)

当考虑 $\mathbb{Z}$ 模 $n$ 时,若 $n$ 为合数,情况则完全不同。此时 $n$ 可以分解为 $n = ab$,其中 $a,b \in \mathbb{N}-\{1\}$)。此时 $1 < a < n$,但 $a$ 与 $n$ 不互质(公因子为 $a$),因此 $a$ 在模 $n$ 下无乘法逆元。事实上,$n$ 的所有因数(及其倍数)在模 $n$ 下均无乘法逆元。

例如,考虑 $p=6$:
\begin{align*}
    1^{-1} \equiv 1 \mod 6\\
    2^{-1} \text{\ 不存在} \mod 6\\
    3^{-1} \text{\ 不存在} \mod 6\\
    4^{-1} \text{\ 不存在} \mod 6\\
    5^{-1} \equiv 5 \mod 6
\end{align*}

由于这种区别,$\mathbb{Z}$ 模 $p$ 的数学``结构''具有显著独特性:其元素均有逆元,这赋予了它良好的性质。这样描述可能比较模糊,但主要思想是:所有元素均有逆元,这使得 $\mathbb{Z}$ 模 $p$ 很特别。严格来说,$\mathbb{Z}$ 模 $p$ 构成一个称为\textbf{群}的数学结构。

直观而言,群是满足以下运算律的对象集合:
\begin{enumerate}[label=(\alph*)]
    \item 交换律
    \item 结合律
    \item 所有元素均有逆元
\end{enumerate}

已知整数乘法(包括任意 $n$ 下的 $\mathbb{Z}$ 模 $n$)满足交换律和结合律,而质数 $p$ 对应的 $\mathbb{Z}$ 模 $p$ 还满足逆元存在性。

本章末尾附有相关练习以供深化理解群的概念。如需进一步探索,建议参考\textbf{抽象代数}、\textbf{近世代数}或\textbf{群论}的入门教材。这些领域蕴含深刻的数学思想,且\textbf{群}在众多学科中具有重要应用!
