% !TeX root = ../../../book.tex

\subsection{一些有用的定理}

在本节中,我们将探讨一些数论中的定理,这些定理涉及模运算,并且它们本身既有用又有趣。我们将陈述并证明这些定理(有时需要你通过练习给与帮助),然后用例子来展示它们的实际应用。

\subsubsection*{中国剩余定理}

为了引出这个定理,我们先通过一个故事来说明它的用处:

\begin{quote}
    孙武\footnote{在中国,这个故事的主角是韩信。因此这个问题也被称为``韩信点兵''问题。--- 译者注}将军的部队里有许多士兵,战斗结束后,他想快速统计剩余士兵的数量。一个个数显然太费劲了,所以他想用更高效的方法完成点兵。幸运的是,这些士兵训练有素,可以轻松组成等大小的队列。

    孙武将军先命令士兵们排成两排等长的队伍,发现多出一个士兵。

    接着,他又让士兵们排成三个等大小的环形队列,但还是多出一个士兵。

    最后,他命令士兵们排成五个等大小的侧翼队列,这次多出两个士兵。

    此时,他觉得信息已经足够。战斗结束后,他推测这支部队的总人数在 $250$ 到 $300$ 之间。根据这些信息,他能\emph{确切}知道有多少士兵。
    
    你能算出士兵的具体数量吗?这支部队究竟有多少士兵呢?
\end{quote}

请你先试着解决这个问题,看看你能否找到答案。然后再继续阅读我们的解决方案、一个定理陈述以及解决此类问题的方法介绍。

请再次阅读这个故事。设孙武将军军队的士兵数量为 $x$,那么这个故事告诉我们,$x$ 必须满足以下三个同余条件和一个不等式:
\begin{align*}
    x \equiv 1 \mod 2 \\
    x \equiv 1 \mod 3 \\
    x \equiv 2 \mod 5 \\
    250 \le x \le 300
\end{align*}
(你能从故事中看出这些条件的来源吗?)

现在有两个问题需要考虑:
\begin{enumerate}[label=(\arabic*)]
    \item 是否\emph{一定}存在一个 $x$ 满足所有三个同余条件?
    \item 是否存在\emph{多个}满足条件的 $x$ 值?我们能否保证其中一个 $x$ 也满足不等式?
\end{enumerate}

下面陈述的\textbf{中国剩余定理}可以保证:
\begin{enumerate}[label=(\arabic*)]
    \item 同余方程有无穷多个解;
    \item 至少有一个解满足给定的不等式。
\end{enumerate}
不过,在我们陈述并证明这个定理之前,让我们先尝试解决这个初始问题。我们将其分解为几个观察和步骤:
\begin{itemize}
    \item 第一个同余条件要求 $x$ 必须是\textbf{奇数},这样就排除了所有偶数作为潜在解。以下是潜在解列表:
    \[1,\cancel{2}, 3, \cancel{4}, 5, \cancel{6}, 7, \cancel{8}, 9,\cancel{10}, 11,\cancel{12}, 13,\cancel{14}, 15,\cancel{16}, 17,\cancel{18}, 19,\cancel{20}, 21,\cancel{22}, 23, \dots\]
    \item 第二个同余条件要求解必须是 $3$ 的倍数加 $1$,这就排除了模 $3$ 余 $0$ 或 $2$ 的数。以下是潜在解列表:
    \[1,\cancel{2}, \cancel{3}, \cancel{4}, \cancel{5}, \cancel{6}, 7, \cancel{8}, \cancel{9},\cancel{10}, \cancel{11},\cancel{12}, 13,\cancel{14}, \cancel{15},\cancel{16}, \cancel{17},\cancel{18}, 19,\cancel{20}, \cancel{21},\cancel{22}, \cancel{23}, \dots\]
    \item 第三个同余条件要求解必须是 $5$ 的倍数加 $2$,这就排除了模 $5$ 余 $0, 1, 3, 4$ 的数。以下是潜在解列表:
    \[\cancel{1},\cancel{2}, \cancel{3}, \cancel{4}, \cancel{5}, \cancel{6}, \circled{7}, \cancel{8}, \cancel{9},\cancel{10}, \cancel{11},\cancel{12}, \cancel{13},\cancel{14}, \cancel{15},\cancel{16}, \cancel{17},\cancel{18}, \cancel{19},\cancel{20}, \cancel{21},\cancel{22}, \cancel{23}, \dots\]
\end{itemize}
看起来 $7$ 是唯一的解,但我们怎么确定没有其他解呢?我们只检查了前 $23$ 个可能的解……能\emph{确保}没有其他解吗?这个问题就交给你来探究了。试试更大的数字,看看能不能找到其他解。你能猜出其中的规律吗?$7$ 真的是唯一解吗?

现在,我们用更巧妙的方法来解决这些同余问题。具体来说,假设我们有一个解 $x$,它满足所有三个同余式,看看我们能否推导出更多的信息。通过这个推导,我们将揭示所有\emph{可能}解的一个特性。

根据同余的定义,我们知道存在 $k, \ell, m \in \mathbb{Z}$ 使得
\begin{align*}
    x &= 2k + 1 \\
    x &= 3\ell + 1 \\
    x &= 5m + 2 
\end{align*}
给定这样的 $k, \ell, m$。

我们先来看前两个方程,试着将它们合并成一个关于 $x$ 的方程。具体来说,把第一个方程乘以 $3$,第二个方程乘以 $2$,这样就会分别得到 $6k$ 和 $6\ell$ 项。然后,通过相减,我们可以适当地进行因式分解。也就是说,我们首先找到
\begin{align*}
    3x &= 6k + 3 \\
    2x &= 6\ell + 2
\end{align*}
然后
\[(3x - 2x) = (6k + 3) - (6\ell + 2) \implies x = 6(k - \ell) + 1\]
因为 $k, \ell \in \mathbb{Z}$ 已经给定,我们可以定义 $u = k-\ell$,所以 $u \in \mathbb{Z}$。注意这告诉我们此时 $x = 6u+1$,换句话说
\[x \equiv 1 \mod 6\]
现在,我们通过结合前两个同余式得到了这个新的同余式,这并非巧合,因为这个同余式是模 $6$ 的,而 $6 = 2 \times 3$。稍后,当我们引导你证明接下来的定理时,你会明白其中的原理!

接下来,我们尝试将这个新的同余式与上面的第三个同余式结合。我们采用类似的方法:将刚刚推导出的同余式乘以 $5$,将第三个同余式乘以 $6$,这样相减后可以提出一个 $30$ 的因子。(这也解释了为什么新推导出的同余式是模 $30$ 的。)我们得到
\begin{align*}
    5x &= 30u + 5 \\
    6x &= 30m + 12
\end{align*}
然后
\[(6x - 5x) = (30m + 12) - (30u + 5) \implies x = 30(m - u) + 7\]
同理,因为 $u,m$ 已经给定,我们可以定义 $v = m-u$,所以 $v \in \mathbb{Z}$。这告诉我们此时 $x = 30v+7$,换句话说
\[x \equiv 7 \mod 30\]
这个最终的同余式是通过将给定的每个同余式相互结合推导出来的,因此它包含了这三个同余式的所有信息。我们断言,这个同余式现在包含了\textbf{所有的}解!

首先,这个新推导出的同余告诉我们,任何解必须模 $30$ 余 $7$。换句话说,任何除以 $30$ 余数不是 $7$ 的数都不可能是解。本质上,这把我们在上述三种观察中排除潜在解的工作总结成一个陈述。

其次,我们可以解释,事实上,任何模 $30$ 余 $7$ 的数确实是一个解。让我们看看原因。设 $n \in \mathbb{Z}$,并定义 $y = 30n + 7$(即我们选择任意 $y \in \mathbb{Z}$,满足 $y \equiv 7 \mod 30$)。注意 $y$ 满足
\begin{itemize}
    \item 第一个同余式,因为 $y = 30n + 7 = 2(15n + 3) + 1$,所以 $y \equiv 1 \mod 2$。
    \item 第二个同余式,因为 $y = 30n + 7 = 3(10n + 2) + 1$,所以 $y \equiv 1 \mod 3$。
    \item 第三个同余式,因为 $y = 30n + 7 = 5( 6n + 1) + 2$,所以 $y \equiv 2 \mod 6$。
\end{itemize}
至此,我们已经知道:
\begin{enumerate}[label=(\arabic*)]
    \item \emph{任何}解 $x$ 必须满足 $x \equiv 7 \mod 30$;
    \item 任何满足这个条件的 $x$ 实际上\emph{就是}一个解。
\end{enumerate}
这两个陈述共同形成了一个 $\iff$陈述,即
\[x \;\text{是三个同余式的解} \iff x \equiv 7 \mod 30\]
因此\textbf{所有解}的集合 $S$ 为
\[S = \{x \in \mathbb{Z} \mid x \equiv 7 \mod 30\} = \{30n + 7 \mid n \in \mathbb{Z}\}\]

回到最初的问题,我们现在只需要考虑给定的不等式。是否存在一个满足 $x \equiv 7 \mod 30$ 且 $250 \le x \le 300$ 的数 $x$ 呢?是的,确实存在!我们可以从 $7$ 开始,每次加上 $30$ 的倍数,或者从接近 $300$ 的数开始调整,总之用类似的方法就能找到它。无论你怎么做,你会发现 $\mathbf{x = 277}$ 就是我们一直在寻找的解。这就是孙武将军军队里有士兵的数量。

现在,为了进行比较,考虑以下可能来自类似问题的同余方程组:
\begin{align*}
    x &\equiv 3 \mod 4 \\
    x &\equiv 2 \mod 6
\end{align*}
这个同余方程组有解吗?我们之前使用的方法在这里适用吗?如果你尝试使用``划掉不合适的候选者''或``合并同余式''的方法,会发现都\emph{不起作用}。回头看看这个方程组,你会发现这很合理。第一个同余要求 $x$ 比 $4$ 的倍数多 $3$;由于 $4$ 的倍数是偶数,这意味着我们要求 $x$ 是\emph{奇数}。然而,第二个同余要求 $x$ 比 $6$ 的倍数多 $2$;由于 $6$ 的倍数也是偶数,这意味着我们要求 $x$ 是\emph{偶数}。一个解怎么可能同时即是奇数又是偶数呢?!这显然是不可能的。

\textbf{中国剩余定理}告诉我们在什么情况下同余方程组一定有解。它适用于我们之前解决的第一个问题,并且实际上告诉了我们最终的结果:有无穷多个解,并且它们都同余于 $30$。然而,它并没有告诉我们刚刚解决的第二个问题无解。这个定理对某些情况提供了\emph{保证}。当我们遇到这些情况时,可以对解作出有效的判断。然而,当我们遇到\emph{不同}情况时,这个定理并\emph{不能保证}有解。现在让我们看看这个定理的陈述,然后进一步讨论一下,并请你帮助证明它(用两种不同的方法!)。

\begin{theorem}\label{theorem6.5.28}
    假设我们有一个由 $r$ 个不同同余式组成的方程组。具体来说,假设 $r \in \mathbb{N}$,并且我们有 $r$ 个自然数 $n_1, n_2, \dots, n_r$,以及 $r$ 个整数 $a_1, a_2, \dots, a_r$。这个同余方程组可以表示为:
    \begin{align*}
        x &\equiv a_1 \mod n_1 \\
        x &\equiv a_2 \mod n_2 \\
        \vdots \\
        x &\equiv a_r \mod n_r
    \end{align*}
    (换句话说,该方程组要求 $x \in \mathbb{Z}$ 满足 $\forall i \in [r] \centerdot x \equiv a_i \mod n_i$。)

    \dotuline{如果}模数 $n_i$ 是两两互质的,也就是说,任意两个 $n_i$ 之间除了 $1$ 以外没有其他公因数,\dotuline{那么}这个同余方程组一定有解。

    此外,在这种情况下,实际上有无穷多个解,并且这些解模 $N$ 同余。这里的 $N$ 定义为所有模数的乘积:
    \[N = \prod_{i \in [r]}^{} n_i\]
\end{theorem}

请注意,主要结论是``\textbf{如果……那么……}''形式的陈述。还记得我们之前提到的关于这种条件陈述的内容吗?这个定理并没有说明当两个模数不互质时会发生什么。在这种情况下,可能会发生任何事情!我们之前看到的例子中,模数不互质:一个同余式是模 $4$,另一个是模 $6$,而 $4$ 和 $6$ 有一个公因数 $2$。然而,定理并没有说这种情况下无解;我们需要自己去找出答案。如果我们稍微改变一下数字,并提出以下同余关系:
\begin{align*}
    x \equiv 3 \mod 4 \\
    x \equiv 5 \mod 6
\end{align*}
这个方程组就有解。你可以试着解一下。

中国剩余定理的一种证明方法类似于我们之前解决问题的步骤。对于方程组中任意数量的同余关系,我们可以通过逐步将一个同余合并到另一个同余中,最终得到一个模数为所有其他模数乘积的同余关系。那么,如何证明这种方法的有效性呢?这是一个迭代过程……归纳法正好发挥用武之地!确实可以通过对 $r$(即方程组中同余关系的数量)进行归纳来证明中国剩余定理。这种证明在练习 \ref{exc:exercises6.7.26} 中有详细介绍。我们喜欢这种证明,因为它还为解决这类问题提供了一个实际操作的方法。

另一种证明是\textbf{构造性的}。也就是说,它利用定理中的信息,通过组合这些信息来定义一个解 $X$(并且证明了这一点)。这种证明在练习 \ref{exc:exercises6.7.27} 中有详细介绍。我们喜欢这种证明,因为它确实是构造性的;它不是通过论证某个对象存在的原因来证明,而是实际构造出了这个对象。然而,这种方法构造的解并不是通过``排除不合适的候选者''或``合并同余式''找到的相同解。这实际上是一种有点``非自然''的方法,但它确实在不需要进行任何归纳过程的情况下也能有效。为了比较这两种方法,我们鼓励你尝试完成这两个定理的证明。然而,如果我们只能推荐一种方法,我们会建议使用归纳法证明。

\subsubsection*{贝祖恒等式}

这个定理让我们回想起之前关于线性丢番图方程的讨论。在示例 \ref{ex:example6.5.26} 中,我们通过巧妙地应用乘法逆元解决了一个特定的方程。除了展示这种方法外,还有一种简单的方法来验证这种方程是否有解。这个定理准确地描述了二元线性丢番图方程何时有解,称为\textbf{贝祖恒等式},以 $18$ 世纪法国数学家 Étienne Bézout 命名。

在陈述这个定理之前,我们需要先提供一个定义。你可能已经熟悉这个定义了,但它在这个定理中起着至关重要的作用。因此,我们在这里给出这个定义,并提供一些示例来说明。

\begin{definition}[最大公约数]\label{def:definition6.5.29}
    给定 $a,b \in \mathbb{Z}$。$a$ 和 $b$ 的\dotuline{最大公约数}用 $\gcd(a, b)$ 表示,定义为能够同时整除 $a$ 和 $b$ 的最大整数。也就是说,
    \[\gcd(a, b) \mid a \land \gcd(a, b) \mid b\]
    且
    \[\forall d \in \mathbb{Z} \centerdot (d \mid a \land d \mid b) \implies d \le \gcd(a, b)\]
\end{definition}

我们假设你对这个概念有一定的了解,或者至少有一些直觉。即将介绍的定理及其证明并不需要你对这一概念有深入的理解。此外,任何涉及这一定义或定理的练习都不会要求你具备很强的计算能力,也不会假设你对这个概念有深入了解。相反,请将此视为你在继续练习吸收数学概念新定义时的一部分,帮助你运用这些抽象概念来证明进一步的事实,并找到例子和反例。这是一项重要的技能!在陈述和证明定理之前,我们先快速浏览几个此概念的实际例子。\\

\begin{example}
    在某些情况下,我们会取两个数字并求出它们的最大公约数。通常,找到最大公约数的合理方法是先找到两个数字的\textbf{质因数分解},然后适当地进行组合。也就是说,$\gcd(a, b)$ 是 $a$ 和 $b$ 共有的质因数的乘积,因此通过考虑这些共同的质因数,我们可以很容易地求出最大公约数。

    在某些情况下,我们会做出一些关于最大公约数的一般性论断,并进行证明(或者可能要求你来证明它!)。这些论断仅依赖于我们上面提供的定义。

    \begin{itemize}
        \item 设 $a=15, b=6$。因为 $a = 3 \cdot 5, b = 2 \cdot 3$,我们发现它们的公因子只有 $3$。因此
            \[\gcd(6, 15) = 3\]
        \item 设 $a=30, b=40$。因为 $a = 2 \cdot 3 \cdot 5, b = 2^3 \cdot 5$,我们发现它们的公因子有 $2$ 和 $5$。因此
            \[\gcd(30, 40) = 10\]
        \item 一般来说,
            \[\gcd(a, b) = \gcd(b, a)\]
            这显然是正确的,因为 $a$ 和 $b$ 的任何公约数也是 $b$ 和 $a$ 的公约数。
        \item 设 $a=77, b=72$。因为 $a = 7 \cdot 11, b = 2^3 \cdot 3^2$,我们发现它们没有公因子。因此
            \[\gcd(77, 72) = 1\]
        \item 设 $a=13$,设 $b \in \mathbb{N}$ 且 $a \nmid b$。因为 $a$ 为质数,且 $b$ 不是 $13$ 的倍数,因此 $b$ 的质因子中没有 $13$。因此
            \[\gcd(13, b) = 1\]
            这意味着 $a$ 和 $b$ \textbf{互质}。这是一个通用事实:
            \[a \;\text{和}\; b \;\text{互质} \iff \gcd(a, b) = 1\]
            此外
            \[\forall a, b \in \mathbb{N} \centerdot a \;\text{为质数} \implies \big(\gcd(a,b)=1 \iff a \nmid b\big)\]
    \end{itemize}
\end{example}

现在,我们觉得已经准备好陈述和证明\textbf{贝祖恒等式}了!

\begin{theorem}[贝祖恒等式]\label{theorem6.5.31}
   给定 $a, b \in \mathbb{Z}$。定义 $L$ 为 $a$ 和 $b$ 的所有线性组合的集合;换句话说,定义
   \[L = \{z \in \mathbb{Z} \mid \exists x, y \in \mathbb{Z} \centerdot ax + by = z\} = \{ax + by \mid x, y \in \mathbb{Z}\}\]
   定义 $M$ 为所有 $\gcd(a, b)$ 的倍数的集合;换句话说,定义
   \[M = \{z \in \mathbb{Z} \mid \exists k \in \mathbb{Z} \centerdot z = k \cdot \gcd(a, b)\} = \{k \cdot \gcd(a, b) \mid k \in \mathbb{Z}\}\]
   则
   \[L = M\]
   换句话说,线性丢番图方程 $ax + by = c$ 有解当且仅当 $c$ 是 $\gcd(a, b)$ 的倍数。
\end{theorem}

这一定理非常实用,它告诉我们线性丢番图方程 $ax + by = c$(给定 $a, b, c \in \mathbb{Z}$)何时有解。只需找到 $\gcd(a, b)$,并确保 $\gcd(a, b) \mid c$。

为了证明这一定理,我们需要证明两个\emph{集合相等}。我们将使用一种叫做\emph{双重包含论证}的方法,这是我们之前多次使用过的策略。我们这里只会证明其中一个包含,另一个包含则留给你作为练习。

\begin{proof}
    给定 $a, b \in \mathbb{Z}$。按照定理的陈述定义集合 $L$ 和 $M$。

    首先,我们证明 $L \subseteq M$。设 $z$ 是 $L$ 中任意固定元素。

    根据 $L$ 的定义,我们知道 $\exists x, y \in \mathbb{Z} \centerdot ax + by = z$。给定这样的 $x$ 和 $y$。

    由于 $\gcd(a, b)$ 能整除 $a$ 和 $b$,我们知道$\exists k, \ell$ 使得 $a = k \cdot \gcd(a, b)$ 且 $b = \ell  \cdot \gcd(a, b)$。给定这样的 $k$ 和 $\ell$。

    我们将 $a$ 和 $b$ 的表达式代入上面的方程:
    \[z = ax + by = k \cdot \gcd(a, b) \cdot x + \ell \cdot \gcd(a, b) \cdot y = \gcd(a, b) \cdot \underbrace{(kx + \ell y)}_{m}\]
    定义 $m = kx + \ell y$。由于 $m \in \mathbb{Z}$,这表明 $z$ 是 $\gcd(a, b)$ 的倍数。

    因此,$z \in M$。这证明了 $L \subseteq M$。\\

    接着,我们证明 $M \subseteq L$……

    留作练习 \ref{exc:exercises6.7.12}。
\end{proof}

通过这一结果的证明,我们可以确定二元线性丢番图方程是否有解。接下来的几个练习将要求你判断这种方程是否有解,只需引用这个结果即可。如果还需要你找到所有解,请使用我们在示例 \ref{ex:example6.5.26} 中展示的方法。

\textbf{挑战性问题}:你认为关于多于两个变量的线性丢番图方程,有哪些可以讨论的内容?比如,考虑
\[6x + 8y + 15z = 10\]
这个方程有解吗?如果有,那有多少个解呢?再比如,考虑
\[3x + 6y + 9z = 2\]
这个方程有解吗?为什么有或为什么没有?

试着陈述并证明一个关于这个问题的结论。你能将这个结论推广到任意数量的变量吗?
