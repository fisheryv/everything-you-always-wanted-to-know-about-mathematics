% !TeX root = ../../../book.tex

\subsection{习题}

\subsubsection*{温故知新}

以口头或书面的形式简要回答以下问题。这些问题全都基于你刚刚阅读的内容,所以如果忘记了具体的定义、概念或示例,可以回去重读相关部分。确保在继续学习之前能够自信地回答这些问题,这将有助于你的理解和记忆!

\begin{enumerate}[label=(\arabic*)]
    \item 为什么在考虑 $\mathbb{Z}$ 模 $n$ 时会将 $\mathbb{Z}$ 划分成几个集合?
    \item $\mathbb{Z}$ 模 $n$ 的等价类是什么?
    \item 如何判断两个整数 $x, y \in \mathbb{Z}$ 是否属于 $\mathbb{Z}$ 模 $n$ 的同一个等价类?
    \item 模算术引理是什么?它为什么如此有用?我们如何利用它以代数的方式处理同余?
    \item 什么是\emph{乘法逆元}?给定 $a \in \mathbb{Z}$ 和 $n \in \mathbb{N}$,在 $\mathbb{Z}$ 模 $n$ 的情况下,如何判定 $a$ 的乘法逆元是否存在?
    \item 当 $p$ 为\emph{质数}时,$\mathbb{Z}$ 模 $p$ 的等价类集合有什么特别之处?
    \item 中国剩余定理是否保证以下同余系统有解?为什么?
    \begin{align*}
        x &\equiv 2 \mod 6 \\
        x &\equiv 5 \mod 9
    \end{align*} 
    你能找到这个同余方程组的解吗?(\textbf{提示}:答案是肯定的!)
\end{enumerate}

\subsubsection*{小试牛刀}

尝试回答以下问题。这些题目要求你实际动笔写下答案,或(对朋友/同学)口头陈述答案。目的是帮助你练习使用新的概念、定义和符号。题目都比较简单,确保能够解决这些问题将对你大有帮助!

\begin{enumerate}[label=(\arabic*)]
    \item 请陈述并证明判断自然数 $x \in \mathbb{N}$ 是否为 $9$ 的倍数的技巧。\\
    (\textbf{提示}:参见示例 \ref{ex:example6.5.13} 中的类似问题。)
    \item 设 $n \in \mathbb{N}, a \in \mathbb{Z}$。证明 $ (n-a)^2 \equiv a^2 \mod n$。
    \item 设 $n \in \mathbb{N} - \{1\}$。证明 $ (n-1)^{-1} \equiv n - 1 \mod n$。
    \item 对于每一对值 $(a, n)$,请找出 $a$ 模 $n$ 的乘法逆元,或者说明它不存在。
    \begin{enumerate}[label=(\alph*)]
        \item $a = 5$ 且 $n = 12$
        \item $a = 7$ 且 $n = 11$
        \item $a = 6$ 且 $n = 27$
        \item $a = 11$ 且 $n = 18$
        \item $a = 70$ 且 $n = 84$
        \item $a = 8$ 且 $n = 17$
    \end{enumerate}
    \item 描述下面方程所有整数解 $x, y \in \mathbb{Z}$ 的特征。
        \[4x - 7y = 18\]
    \item 确定以下同余方程组的所有解:
        \begin{align*}
            x \equiv 3 \mod 5 \\
            x \equiv 4 \mod 7
        \end{align*}
\end{enumerate}