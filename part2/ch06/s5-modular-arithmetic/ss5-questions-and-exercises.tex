% !TeX root = ../../../book.tex

\subsection{习题}

\subsubsection*{温故知新}

以口头或书面的形式简要回答以下问题。这些问题全都基于你刚刚阅读的内容,如果忘记了具体定义、概念或示例,可以回顾相关内容。确保在继续学习之前能够自信地作答这些问题,这将有助于你的理解和记忆!

\begin{enumerate}[label=(\arabic*)]
    \item 在考虑 $\mathbb{Z}$ 模 $n$ 时,为何会将 $\mathbb{Z}$ 划分为若干集合?
    \item $\mathbb{Z}$ 模 $n$ 的等价类是什么?
    \item 如何判断两个整数 $x, y \in \mathbb{Z}$ 是否属于 $\mathbb{Z}$ 模 $n$ 的同一等价类?
    \item 模算术引理是什么?它为什么如此有用?如何利用它以代数的方式处理同余?
    \item 什么是\emph{乘法逆元}?给定 $a \in \mathbb{Z}$ 和 $n \in \mathbb{N}$,在 $\mathbb{Z}$ 模 $n$ 下,如何判定 $a$ 的乘法逆元是否存在?
    \item 当 $p$ 为\emph{质数}时,$\mathbb{Z}$ 模 $p$ 的等价类集合有何特殊性质?
    \item 中国剩余定理是否保证以下同余方程组有解?为什么?
    \begin{align*}
        x &\equiv 2 \mod 6 \\
        x &\equiv 5 \mod 9
    \end{align*} 
    能否求出该方程组的解?(\textbf{提示}:存在解。)
\end{enumerate}

\subsubsection*{小试牛刀}

尝试解答以下问题。这些题目需动笔书写或口头阐述答案,旨在帮助你熟练运用新概念、定义及符号。题目难度适中,确保掌握它们将大有裨益!

\begin{enumerate}[label=(\arabic*)]
    \item 请陈述并证明自然数 $x \in \mathbb{N}$ 能否被 $9$ 整除的判定方法。\\
    (\textbf{提示}:参见示例 \ref{ex:example6.5.13}。)
    \item 设 $n \in \mathbb{N}, a \in \mathbb{Z}$。证明 $ (n-a)^2 \equiv a^2 \mod n$。
    \item 设 $n \in \mathbb{N} - \{1\}$。证明 $ (n-1)^{-1} \equiv n - 1 \mod n$。
    \item 对下列各组 $(a, n)$,求 $a$ 模 $n$ 的乘法逆元,或说明其不存在。
    \begin{enumerate}[label=(\alph*)]
        \item $a = 5 \enspace\quad n = 12$
        \item $a = 7 \enspace\quad n = 11$
        \item $a = 6 \enspace\quad n = 27$
        \item $a = 11 \quad n = 18$
        \item $a = 70 \quad n = 84$
        \item $a = 8 \enspace \quad n = 17$
    \end{enumerate}
    \item 描述下面方程所有整数解 $x, y \in \mathbb{Z}$ 的特征。
        \[4x - 7y = 18\]
    \item 求以下同余方程组的所有解:
        \begin{align*}
            x \equiv 3 \mod 5 \\
            x \equiv 4 \mod 7
        \end{align*}
\end{enumerate}