% !TeX root = ../../../book.tex

\subsection{模 $n$ 等价类}

你已经证明了(引理 \ref{lemma6.5.9})模 $n$ 同余是 $\mathbb{Z}$ 上的等价关系,并证明了(定理 \ref{theorem6.4.10})等价关系将集合\emph{划分}为等价类。结合这两点,可知模 $n$ 同余将 $\mathbb{Z}$ 划分为若干等价类。如何表示这些等价类?每个类的代表元素应该如何选择?

我们从两个更简单的问题入手:
\begin{enumerate}[label=(\arabic*)]
    \item $\mathbb{Z}$ 模 $n$ 有多少个等价类?
    \item 这些等价类的``大小''如何?
\end{enumerate}

\subsubsection*{等价类的数量}

要回答问题 (1),回忆除以 $n$ 的余数定义。除法算法(引理 \ref{lemma6.5.2})表明余数 $r$ 满足 $0 \le r \le n-1$,故余数有 $n$ 种可能:$0,1,2,\dots,n-1$(即 $r \in {0,1,\dots,n-1}$)。这些余数均可取到,例如 $n-1$ 除以 $n$ 的余数为 $n-1$(因为 $n-1 < n$)。因此,$\mathbb{Z}$ 模 $n$ 的等价类恰有 $n$ 个。

基于相同的观察,可以确定等价类的\emph{代表元素}。由于 $a \equiv b \mod n$ 表示 $a$ 和 $b$ 除以 $n$ 有相同的余数,故二者属于由该余数 $r$(满足 $0 \le r \le n-1$)代表的等价类,记为 $a, b \in [r]_{\text{\ mod\ } n}$(下标``$\text{mod\ } n$''表明余数基于 $n$)。

\subsubsection*{等价类的大小}

让我们通过一个具体的例子来思考这个问题,例如 $n = 4$。整数 $z \in \mathbb{Z}$ 属于余数 $0$ 的等价类 $[0]{\text{\ mod\ } 4}$ 时,意味着 $z$ 除以 $4$ 余数为 $0$,即 $z$ 是 $4$ 的\emph{倍数}。$\mathbb{Z}$ 中 $4$ 的倍数有无穷多个,如 $0,4,8,12,\dots$ 和 $-4,-8,-12,\dots$,故 $[0]{\text{\ mod\ } 4}$ 是\emph{无限集}。

让我们通过一个具体的例子来思考这个问题,例如 $n = 4$。对于一个整数 $z \in \mathbb{Z}$ 属于余数为 $0$ 的等价类,这意味着什么?也就是说,如果我们知道 $z \in [0]_{\text{\ mod\ } 4}$,我们可以得出哪些关于 $z$ 的结论?

类似地,$z \in [1]{\text{\ mod\ } 4}$ 表示余数为 $1$,即\emph{存在}整数 $k$,使得 $z = 4k + 1$。取 $k = 0,1,2,\dots$ 及 $k = -1,-2,\dots$ 时,可得
\begin{align*}
    [1]_{\text{\ mod\ } 4} &= \{\dots , -7, -3, 1, 5, 9, \dots \} \\
    &= \{z \in \mathbb{Z} \mid \exists k \in \mathbb{Z} \centerdot z = 4k + 1\} \\
    &= \{4k + 1 \mid k \in \mathbb{Z}\}
\end{align*}
请注意,此处先用省略号展示模式,再用两种集合表示法重述。

此集合同样是无限集。对其他余数(模 $4$ 或任意整数 $n$),等价类均为无限集(我们尚未\emph{正式}定义无限集,但直观上可以理解为无限集拥有无穷多元素,我们可以列出其所有元素,并找到一个生成所有元素的模式,但此过程在有限时间内无法结束)。

\subsubsection*{$\mathbb{Z}$ 模 $n$ 的划分}

基于对等价类的观察,我们可以总结出 $\mathbb{Z}$ 模 $n$ 的等价类的标准表示。已知有 $n$ 个等价类,每个等价类包含无穷多个元素。每个等价类对应于整数除以 $n$ 的余数。由于余数必须满足 $0 \le r \le n-1$,我们将集合 $\{0, 1, 2, \dots , n-1\} = [n-1] \cup \{0\}$ 作为标准代表集合。

余数为 $r$ 的等价类包含所有除以 $n$ 余 $r$ 的整数。换句话说,所有 $z \in [r]_{\text{\ mod\ } n}$ 的元素都是 $n$ 的某个倍数加上 $r$。也就是说,我们可以通过从 $r$ 开始不断加上或减去 $n$ 来生成等价类的所有元素。这样,同一等价类中的任意两个元素相差 $n$ 的倍数。

\setlength{\fboxrule}{2pt}
\setlength\fboxsep{5mm}
\begin{center}
\noindent \fcolorbox{blue}{white}{%
    \parbox{0.85\textwidth}{%
        \linespread{1.5}\selectfont
        \textcolor{blue}{\textbf{$\mathbb{Z}$ 模 $n$ 等价类:}}\\
        给定 $n \in \mathbb{N}$,恰好存在 $n$ 个等价类
        \[[0]_{\text{\ mod\ } n}, \; [1]_{\text{\ mod\ } n}, \; [2]_{\text{\ mod\ } n}, \; \dots , \; [n-1]_{\text{\ mod\ } n}\]
        其特点是:
        \begin{align*}
            [0]_{\text{\ mod\ } n} &=  \{\dots, -2n, -n, 0, n, 2n, \dots \} \\
            &= \{z \in \mathbb{Z} \mid \exists k \in \mathbb{Z} \centerdot z = kn\}\\
            [1]_{\text{\ mod\ } n} &=  \{\dots, -2n+1, -n+1, 1, n+1, 2n+1, \dots \} \\
            &= \{z \in \mathbb{Z} \mid \exists k \in \mathbb{Z} \centerdot z = kn+1\}\\
            [2]_{\text{\ mod\ } n} &=  \{\dots, -2n+2, -n+2, 2, n+2, 2n+2, \dots \} \\
            &= \{z \in \mathbb{Z} \mid \exists k \in \mathbb{Z} \centerdot z = kn+2\}\\
            &\vdots \\
            [n-1]_{\text{\ mod\ } n} &=  \{\dots, -n-1, -1, n-1, 2n-1, 3n-1, \dots \} \\
            &= \{z \in \mathbb{Z} \mid \exists k \in \mathbb{Z} \centerdot z = kn+(n-1)\}\\
            &= \{z \in \mathbb{Z} \mid \exists \ell \in \mathbb{Z} \centerdot z = \ell n-1\}
        \end{align*}
    }
}
\end{center}

以上是所有观察结果的总结。下面是一些具体 $n$ 值的例子。

\begin{itemize}
    \item 考虑 $n=2$。等价类为:
        \begin{align*}
            [0]_{\text{\ mod\ } 2} &= \{z \in \mathbb{Z} \mid \exists k \in \mathbb{Z} \centerdot z = 2k\}\\ 
            &=  \{\text{偶数}\}\\
            &= \{\dots, -6, -4, -2, 0, 2, 4, 6 \dots \}\\
            [1]_{\text{\ mod\ } 2} &= \{z \in \mathbb{Z} \mid \exists k \in \mathbb{Z} \centerdot z = 2k+1\}\\ 
            &=  \{\text{奇数}\}\\
            &= \{\dots, -5, -3, -1, 1, 3, 5, 7 \dots \}
        \end{align*}
    \item 考虑 $n=3$。等价类为:
        \begin{align*}
            [0]_{\text{\ mod\ } 3} &= \{z \in \mathbb{Z} \mid \exists k \in \mathbb{Z} \centerdot z = 3k\}\\
            &=  \{3 \text{\ 的倍数}\}\\
            &= \{\dots, -9, -6, -3, 0, 3, 6, 9 \dots \}\\
            [1]_{\text{\ mod\ } 3} &= \{z \in \mathbb{Z} \mid \exists k \in \mathbb{Z} \centerdot z = 3k+1\}\\ 
            &=  \{3 \text{\ 的倍数加\ } 1\}\\
            &= \{\dots, -8, -5, -2, 1, 4, 7, 10 \dots \}\\
            [2]_{\text{\ mod\ } 3} &= \{z \in \mathbb{Z} \mid \exists k \in \mathbb{Z} \centerdot z = 3k+2\}\\ 
            &=  \{3 \text{\ 的倍数加\ } 2\}\\
            &= \{\dots, -7, -4, -1, 2, 5, 8, 11 \dots \}
        \end{align*}
    \item 考虑 $n=4$。等价类为:
        \begin{align*}
            [0]_{\text{\ mod\ } 4} &= \{z \in \mathbb{Z} \mid \exists k \in \mathbb{Z} \centerdot z = 4k\}\\ 
            &=  \{4 \text{\ 的倍数}\}\\
            &= \{\dots, -12, -8, -4, 0, 4, 8, 12 \dots \}\\
            [1]_{\text{\ mod\ } 4} &= \{z \in \mathbb{Z} \mid \exists k \in \mathbb{Z} \centerdot z = 4k+1\}\\
            &=  \{4 \text{\ 的倍数加\ } 1\}\\
            &= \{\dots, -11, -7, -3, 1, 5, 9, 13 \dots \}\\
            [2]_{\text{\ mod\ } 4} &= \{z \in \mathbb{Z} \mid \exists k \in \mathbb{Z} \centerdot z = 4k+2\}\\ 
            &=  \{4 \text{\ 的倍数加\ } 2\}\\
            &= \{\dots, -10, -6, -2, 2, 6, 10, 14 \dots \}\\
            [3]_{\text{\ mod\ } 4} &= \{z \in \mathbb{Z} \mid \exists k \in \mathbb{Z} \centerdot z = 4k+3\}\\ 
            &=  \{4 \text{\ 的倍数加\ } 3\}\\
            &= \{\dots, -9, -5, -1, 3, 7, 11, 15 \dots \}
        \end{align*}
\end{itemize}

\subsubsection*{使用等价类}

为什么这很有用?为什么我们要介绍整数模 $n$ 的等价类的构造?

$\mathbb{Z}$ 被这些等价类划分这一点至关重要。因此,在模 $n$ 的背景下进行算术运算时,只需考虑这些等价类,即余数。我们可以将所有整数简化为数字 $0, 1, 2, \dots, n-1$,因为它们代表了所有整数。这样,我们就无需进行大数运算后再求余数;只需处理这些余数即可。让我们通过具体例子来考察这种划分的实际应用。

\begin{example}
    考虑以下命题:
    \[\forall n \in \mathbb{N} \centerdot 6 \mid n^3 + 5n\]
    我们之前让你通过对 $n$ 进行归纳来证明该命题(参见 \ref{sec:section5.7} 节的练习 \ref{exc:exercises5.7.15})。现在,我们将使用等价类来证明它!

    考虑 $\mathbb{Z}$ 模 $6$。因为 $\mathbb{N} \subseteq \mathbb{Z}$,根据除以 $6$ 的余数,我们知道每个 $n \in \mathbb{N}$ 必然落在等价类 $[0]_{\text{\ mod\ } 6}, \;[1]_{\text{\ mod\ } 6}, \;[2]_{\text{\ mod\ } 6}, \;[3]_{\text{\ mod\ } 6}, \;[4]_{\text{\ mod\ } 6}, \;[5]_{\text{\ mod\ } 6}$ 中的\textbf{一个}。

    我们可以分别检查每种情况。假设 $n$ 属于某个特定的等价类,然后计算 $n^3 + 5n$ 所在的等价类。在每种情况下,通过乘法(包括幂运算,即重复乘法)和加法,应用模算术引理 \ref{lemma6.5.10}。
    \begin{align*}
        n \equiv 0 \mod 6 &\implies n^3 + 5n \equiv 0^3 + 5 \cdot 0 \equiv 0 \mod 6 \\
        n \equiv 1 \mod 6 &\implies n^3 + 5n \equiv 1^3 + 5 \cdot 1 \equiv 6 \equiv 0 \mod 6 \\
        n \equiv 2 \mod 6 &\implies n^3 + 5n \equiv 2^3 + 5 \cdot 2 \equiv 18 \equiv 0 \mod 6 \\
        n \equiv 3 \mod 6 &\implies n^3 + 5n \equiv 3^3 + 5 \cdot 3 \equiv 42 \equiv 0 \mod 6 \\
        n \equiv 4 \mod 6 &\implies n^3 + 5n \equiv 4^3 + 5 \cdot 4 \equiv 84 \equiv 0 \mod 6 \\
        n \equiv 5 \mod 6 &\implies n^3 + 5n \equiv 5^3 + 5 \cdot 5 \equiv 150 \equiv 0 \mod 6 
    \end{align*}
    以上每种情况下,$n^3 + 5n$ 均为 $6$ 的倍数(因为除以 $6$ 的余数为 $0$)。这表明,无论 $n$ 取何值,$n^3 + 5n$ 总是 $6$ 的倍数。这证明了对于所有 $n \in \mathbb{N}$,该命题成立,从而避免了使用归纳论证!
\end{example}

\begin{example}[二次残差 (Quadratic Residues)]\label{ex:example6.5.15}

    本例将研究完全平方数的性质,具体来说,我们探讨完全平方数被不同除数相除时产生的余数规律。这一研究颇具趣味性,因为读者将发现余数会随除数变化呈现独特模式(若你因此产生深入探索的兴趣,那再好不过!)。此外,该研究还能引导出若干重要结论,这些结论在正文和习题中均有证明。特别地,理解完全平方数对探究\textbf{毕达哥拉斯三元组}——即满足 $a^2 + b^2 = c^2$ 的整数三元组 $(a, b, c) \in \mathbb{N}^3$——具有关键作用,其性质可以帮助我们证明关于此类三元组的若干有趣定理。

    对于以下每种情况,固定 $n \in \mathbb{N}$,研究所有 $x \in \mathbb{Z}$ 下 $x^2$ 模 $n$ 的结果。根据 $\mathbb{Z}$ 模 $n$ 的划分,只需考察 $n$ 个可能的模 $n$ 余数,然后平方取模。这些可能的余数称为\textbf{二次残差}(\emph{二次}源于平方运算,\emph{残差}即指余数)。下文将分类总结不同 $n$ 值对应的二次残差集。

    \begin{itemize}
        \item $\mathbf{n=2}$:\\
        已知只有当底数为偶数时,完全平方数才是偶数;只有当底数为奇数时,完全平方数才是奇数。在第 \ref{ch:chapter04} 章中,我们曾通过讨论双向条件陈述、量词和证明技巧来验证这些结论。现在无需重新正式证明这些结论,可以直接运用模运算性质轻松验证这些结论。\\
        设 $x \in \mathbb{Z}$ 为任意固定整数。
        \begin{itemize}
            \item 首先,假设 $x \equiv 0 \mod 2$ (即 $x$ 为偶数)。则应用模算术引理可得 $x^2 \equiv 0 \mod 2$ (即 $x^2$ 为偶数)。
            \item 其次,假设 $x \equiv 1 \mod 2$ (即 $x$ 为奇数)。则应用模算术引理可得 $x^2 \equiv 1 \mod 2$ (即 $x^2$ 为奇数)。
        \end{itemize}
        根据 $\mathbb{Z}$ 模 $2$ 的划分,以上两种情况已经完备。
        \begin{quotation}
            \begin{center}
                \large 模 $2$ 二次残差:$\{0, 1\}$
            \end{center}
        \end{quotation}

        \item $\mathbf{n=3}$:\\
        设 $x \in \mathbb{Z}$ 为任意固定整数。应用模算术引理可得:
        \begin{itemize}
            \item $x \equiv 0 \mod 3 \implies x^2 \equiv 0^2 \equiv 0 \mod 3$
            \item $x \equiv 1 \mod 3 \implies x^2 \equiv 1^2 \equiv 1 \mod 3$
            \item $x \equiv 2 \mod 3 \implies x^2 \equiv 2^2 \equiv 4 \equiv 1 \mod 3$
        \end{itemize}
        \begin{quotation}
            \begin{center}
                \large 模 $3$ 二次残差:$\{0, 1\}$
            \end{center}
        \end{quotation}

        \item $\mathbf{n=4}$:\\
        设 $x \in \mathbb{Z}$ 为任意固定整数。应用模算术引理可得:
        \begin{itemize}
            \item $x \equiv 0 \mod 4 \implies x^2 \equiv 0^2 \equiv 0 \mod 4$
            \item $x \equiv 1 \mod 4 \implies x^2 \equiv 1^2 \equiv 1 \mod 4$
            \item $x \equiv 2 \mod 4 \implies x^2 \equiv 2^2 \equiv 4 \equiv 0 \mod 4$
            \item $x \equiv 3 \mod 4 \implies x^2 \equiv 3^2 \equiv 9 \equiv 1 \mod 4$
        \end{itemize}
        \begin{quotation}
            \begin{center}
                \large 模 $4$ 二次残差:$\{0, 1\}$
            \end{center}
        \end{quotation}

        \item $\mathbf{n=5}$:\\
        设 $x \in \mathbb{Z}$ 为任意固定整数。应用模算术引理可得:
        \begin{itemize}
            \item $x \equiv 0 \mod 5 \implies x^2 \equiv 0^2 \equiv 0 \mod 5$
            \item $x \equiv 1 \mod 5 \implies x^2 \equiv 1^2 \equiv 1 \mod 5$
            \item $x \equiv 2 \mod 5 \implies x^2 \equiv 2^2 \equiv 4 \mod 5$
            \item $x \equiv 3 \mod 5 \implies x^2 \equiv 3^2 \equiv 9 \equiv 4 \mod 5$
            \item $x \equiv 4 \mod 5 \implies x^2 \equiv 4^2 \equiv 16 \equiv 1 \mod 5$
        \end{itemize}
        \begin{quotation}
            \begin{center}
                \large 模 $5$ 二次残差:$\{0, 1, 4\}$
            \end{center}
        \end{quotation}

        \item $\mathbf{n=6}$:\\
        设 $x \in \mathbb{Z}$ 为任意固定整数。应用模算术引理可得:
        \begin{itemize}
            \item $x \equiv 0 \mod 6 \implies x^2 \equiv 0^2 \equiv 0 \mod 6$
            \item $x \equiv 1 \mod 6 \implies x^2 \equiv 1^2 \equiv 1 \mod 6$
            \item $x \equiv 2 \mod 6 \implies x^2 \equiv 2^2 \equiv 4 \mod 6$
            \item $x \equiv 3 \mod 6 \implies x^2 \equiv 3^2 \equiv 9 \equiv 3 \mod 6$
            \item $x \equiv 4 \mod 6 \implies x^2 \equiv 4^2 \equiv 16 \equiv 4 \mod 6$
            \item $x \equiv 5 \mod 6 \implies x^2 \equiv 5^2 \equiv 25 \equiv 1 \mod 6$
        \end{itemize}
        \begin{quotation}
            \begin{center}
                \large 模 $6$ 二次残差:$\{0, 1, 3, 4\}$
            \end{center}
        \end{quotation}

        \item $\mathbf{n=7}$:\\
        设 $x \in \mathbb{Z}$ 为任意固定整数。应用模算术引理可得:
        \begin{itemize}
            \item $x \equiv 0 \mod 7 \implies x^2 \equiv 0^2 \equiv 0 \mod 7$
            \item $x \equiv 1 \mod 7 \implies x^2 \equiv 1^2 \equiv 1 \mod 7$
            \item $x \equiv 2 \mod 7 \implies x^2 \equiv 2^2 \equiv 4 \mod 7$
            \item $x \equiv 3 \mod 7 \implies x^2 \equiv 3^2 \equiv 9 \equiv 2 \mod 7$
            \item $x \equiv 4 \mod 7 \implies x^2 \equiv 4^2 \equiv 16 \equiv 2 \mod 7$
            \item $x \equiv 5 \mod 7 \implies x^2 \equiv 5^2 \equiv 25 \equiv 4 \mod 7$
            \item $x \equiv 6 \mod 7 \implies x^2 \equiv 6^2 \equiv 36 \equiv 1 \mod 7$
        \end{itemize}
        \begin{quotation}
            \begin{center}
                \large 模 $7$ 二次残差:$\{0, 1, 2, 4\}$
            \end{center}
        \end{quotation} 

        \item $\mathbf{n=8}$:\\
        设 $x \in \mathbb{Z}$ 为任意固定整数。应用模算术引理可得:
        \begin{itemize}
            \item $x \equiv 0 \mod 8 \implies x^2 \equiv 0^2 \equiv 0 \mod 8$
            \item $x \equiv 1 \mod 8 \implies x^2 \equiv 1^2 \equiv 1 \mod 8$
            \item $x \equiv 2 \mod 8 \implies x^2 \equiv 2^2 \equiv 4 \mod 8$
            \item $x \equiv 3 \mod 8 \implies x^2 \equiv 3^2 \equiv 9 \equiv 1 \mod 8$
            \item $x \equiv 4 \mod 8 \implies x^2 \equiv 4^2 \equiv 16 \equiv 0 \mod 8$
            \item $x \equiv 5 \mod 8 \implies x^2 \equiv 5^2 \equiv 25 \equiv 1 \mod 8$
            \item $x \equiv 6 \mod 8 \implies x^2 \equiv 6^2 \equiv 36 \equiv 4 \mod 8$
            \item $x \equiv 7 \mod 8 \implies x^2 \equiv 7^2 \equiv 49 \equiv 1 \mod 8$
        \end{itemize}
        \begin{quotation}
            \begin{center}
                \large 模 $8$ 二次残差:$\{0, 1, 4\}$
            \end{center}
        \end{quotation}
    \end{itemize}
    
    我们鼓励你继续探究其他二次残差模式。你甚至可以尝试编写一个计算机程序生成残差列表,并思考以下问题:给定 $n \in \mathbb{N}$,模 $n$ 的二次残差数量如何确定?其具体形式如何?是否存在必然出现或永不出现的残差?期待你的探索发现!
\end{example}

\begin{example}
    让我们推广前例中的思想,考察特定情况下\emph{立方残差 (cubic residues)}的性质。
    \begin{quotation}
        假设 $x, y, z \in \mathbb{Z}$ 满足 $x^3+y^3=z^3$。

        证明 $\{x, y, z\}$ 中至少有一个是 $7$ 的倍数。
    \end{quotation}

    重申一下我们的目标,我们要证明:
    \[(x \equiv 0 \mod 7) \lor (y \equiv 0 \mod 7) \lor (z \equiv 0 \mod 7)\]

    为此,列出模 $7$ 的所有立方残差。设 $x \in \mathbb{Z}$ 为任意固定整数,应用模算术引理可得:
    \begin{itemize}
        \item $x \equiv 0 \mod 7 \implies x^3 \equiv 0^3 \equiv 0 \mod 7$
        \item $x \equiv 1 \mod 7 \implies x^3 \equiv 1^3 \equiv 1 \mod 7$
        \item $x \equiv 2 \mod 7 \implies x^3 \equiv 2^3 \equiv 8 \equiv 1 \mod 7$
        \item $x \equiv 3 \mod 7 \implies x^3 \equiv 3^3 \equiv 9 \cdot 3 \equiv 2 \cdot 3 \equiv 6 \mod 7$
        \item $x \equiv 4 \mod 7 \implies x^3 \equiv 4^3 \equiv 16 \cdot 4 \equiv 2 \cdot 4 \equiv 8 \equiv 1 \mod 7$
        \item $x \equiv 5 \mod 7 \implies x^3 \equiv 5^3 \equiv 25 \cdot 5 \equiv 4 \cdot 5 \equiv 20 \equiv 6 \mod 7$
        \item $x \equiv 6 \mod 7 \implies x^3 \equiv 6^3 \equiv (-1)^3 \equiv -1 \equiv 6 \mod 7$
    \end{itemize}

    (注意,为了简化计算,我们将 $6$ 写成 $-1$,然后再模 $7$。)

    我们发现唯一的可能值是 $\{0, 1, 6\}$。

    现在,假设存在解 $x, y, z \in \mathbb{Z}$ 满足 $x^3 + y^3 = z^3$。此时 $x^3,y^3,z^3$ 模 $7$ 均同余于 $0$, $1$ 或 $6$。考虑以下情形:
    \begin{itemize}
        \item 假设 $x^3 \equiv 0 \mod 7$。则 $y^3$ 可以与 $0$, $1$ 或 $6$ 模 $7$ 同余,此时只需要让 $z^3$ 落在相同的等价类即可。总之,在此情况下,都有 $z^3 \equiv 0 \mod 7$。
        
        \item 假设 $y^3 \equiv 0 \mod 7$。将上面的论证应用于 $x^3$ 和 $z^3$。总之,在此情况下,都有 $y^3 \equiv 0 \mod 7$。
        
        \item 假设 $x^3 \equiv 1 \mod 7$。\\
            为了引出矛盾而假设 $y^3 \equiv 1 \mod 7$。则 $x^3+y^3 \equiv 1+1 \equiv 2 \mod 7$,然而 $2$ 不在模 $7$ 的立方残差中,因此这是不可能的。\\
            然而我们发现 $y^3 \equiv 0 \mod 7$ 是可能的,因为 $x^3+y^3 \equiv 1+0 \equiv 1 \mod 7$。\\
            同时我们发现 $y^3 \equiv 6 \mod 7$ 是可能的,因为 $x^3+y^3 \equiv 1+6 \equiv 7 \equiv 0 \mod 7$。\\
            总之,在此情况下,\emph{至少}有一个立方数 —— 要么是 $y^3$,要么是 $z^3$ —— 与 $0$ 模 $7$ 同余。

        \item 假设 $y^3 \equiv 1 \mod 7$。将上面的论证应用于 $x^3$ 和 $z^3$。在此情况下,至少有一个立方数 与 $0$ 模 $7$ 同余。
        
        \item 假设 $x^3 \equiv 6 \mod 7$。\\
            为了引出矛盾而假设 $y^3 \equiv 6 \mod 7$。则 $x^3+y^3 \equiv 6+6 \equiv 12 \equiv 5 \mod 7$,然而 $5$ 不在模 $7$ 的立方残差中,因此这是不可能的。\\
            然而我们发现 $y^3 \equiv 0 \mod 7$ 是可能的,因为 $x^3+y^3 \equiv 6+0 \equiv 6 \mod 7$。\\
            同时我们发现 $y^3 \equiv 1 \mod 7$ 是可能的,因为 $x^3+y^3 \equiv 6+1 \equiv 7 \equiv 0 \mod 7$。\\
            总之,在此情况下,\emph{至少}有一个立方数 —— 要么是 $y^3$ 要么是 $z^3$ —— 与 $0$ 模 $7$ 同余。

        \item 假设 $y^3 \equiv 6 \mod 7$。将上面的论证应用于 $x^3$ 和 $z^3$。在此情况下,至少有一个立方数 与 $0$ 模 $7$ 同余。
    \end{itemize}

    综上,无论何种情形,\textbf{至少}有一个立方项与 $0$ 模 $7$ 同余。具体哪个立方项具有此性质取决于具体情况(可能有多个立方数符合),但总有至少一个成立。

    这很有用,因为回顾立方残差列表会发现一个关键性质:立方项与 $0$ 模 $7$ 同余当且仅当其底数与 $0$ 模 $7$ 同余。即:
    \[\forall z \in \mathbb{Z} \centerdot z^3 \equiv 0 \mod 7 \implies z \equiv 0 \mod 7\]
    这意味着,在上述每种情形中,至少有一个立方项与 $0$ 模 $7$ 同余,这进一步说明至少有一个底数与 $0$ 模 $7$ 同余。通过穷举所有可能情况,我们证明了该方程\emph{所有可能解}的普适性质,而无需构造具体解!
\end{example}

现在,尽管所有工作都已经完成,但我们有一个不幸的消息:原方程\emph{唯一}的解是\emph{平凡}解,即 $x = y = z = 0$。正是如此!你可以尝试寻找其他解,但终归徒劳。这一结果是\textbf{费马大定理}的一个特例,该定理指出,对于方程 $x^k + y^k = z^k$(其中 $k \in \mathbb{N}$),只有当 $k = 1$ 或 $k = 2$ 时,才存在非平凡整数解(即 $x, y, z \in \mathbb{Z}$);换言之,当 $k \in \mathbb{N} \setminus \{1, 2\}$ 时,唯一的解是 $x = y = z = 0$。

费马生前曾提及这一结论,但从未发表证明。他在笔记本的页边空白处声称自己有一个简洁的证明,只是空间不足而未能写下。然而,我们现在知道这很可能并非事实。费马生活在 17 世纪,但这个定理直到 20 世纪 90 年代才被证明\footnote{安德鲁·怀尔斯 (Andrew Wiles) 于 1994 年证明了费马大定理。—— 译者注}!而且,证明过程用到了大量费马时代之后才逐步发展起来的高深数学工具。

如果我们知晓这个定理,便能轻松证明本例中的结论!既然唯一解是 $x = y = z = 0$,那么显然这些值均为 $7$ 的倍数。然而,这种做法既无趣味,也无法让我们练习模算术和等价类。

\begin{example}
    这是另一个涉及立方残差的问题:
    \begin{quotation}
        假设 $x, y, z \in \mathbb{Z}$ 满足 $x^3+y^3+z^3=3$。

        证明 $x^3 \equiv y^3 \equiv z^3 \mod 9$。
    \end{quotation}

    这里讨论的是一个特定的\emph{丢番图方程 (Diophantine Equation)}。丢番图方程是指含有多个变量且系数为整数的多项式方程。求解这类方程需要找到一组整数解,使方程成立。本例中,我们要证明方程的任意解都必须满足 $x^3, y^3, z^3$ 模 $9$ 同余。

    首先,尝试找出该方程的若干解以观察具体实例。以下提供几个简单例子作为参考:例如 $(x, y, z)$ 可取 $(1, 1, 1)$ 或 $(4, 4, -5)$。这些解是否满足我们要求的性质?你还能找到其他解吗?(此问题较难,不必投入过多精力。)

    有趣的是,我们无需确定所有解的具体形式或实际求出它们,即可证明此结论。只需考察模 $9$ 下的立方残差,设 $x \in \mathbb{Z}$ 为任意固定整数,应用模算术引理可得:
    \begin{itemize}
        \item $x \equiv 0 \mod 9 \implies x^3 \equiv 0^3 \equiv 0 \mod 9$
        \item $x \equiv 1 \mod 9 \implies x^3 \equiv 1^3 \equiv 1 \mod 9$
        \item $x \equiv 2 \mod 9 \implies x^3 \equiv 2^3 \equiv 8 \mod 9$
        \item $x \equiv 3 \mod 9 \implies x^3 \equiv 3^3 \equiv 9 \cdot 3 0 \mod 9$
        \item $x \equiv 4 \mod 9 \implies x^3 \equiv 4^3 \equiv 16 \cdot 4 \equiv (-2) \cdot 4 \equiv -8 \equiv 1 \mod 9$
        \item $x \equiv 5 \mod 9 \implies x^3 \equiv 5^3 \equiv 25 \cdot 5 \equiv (-2) \cdot 5 \equiv -10 \equiv 8 \mod 9$
        \item $x \equiv 6 \mod 9 \implies x^3 \equiv 6^3 \equiv 36 \cdot 6 \equiv 0 \cdot 6 \equiv 0 \mod 9$
        \item $x \equiv 7 \mod 9 \implies x^3 \equiv 7^3 \equiv 49 \cdot 7 \equiv 4 \cdot (-2) \equiv -8 \equiv 1 \mod 9$
        \item $x \equiv 8 \mod 9 \implies x^3 \equiv 8^3 \equiv (-1)^3 \equiv -1 \equiv 8 \mod 9$
    \end{itemize}
    注意,在某些情况下使用负数可简化计算。这是完全可行的,且对你大有裨益!例如,计算 $4^3 = 64$ 再模 $9$ 时,可以 $-2$ 替代 $16$ 以保持数值较小。我们可随时加减 $9$ 的倍数,因此在计算过程中直接处理,而非先得大数再取模。(当然,$64$ 并不算大,因此这一点不够明显;但处理更大数字时,此技巧非常实用。此外,将数字尽量简化至个位数可减少心算错误!)注意,最右侧仅出现三种可能结果:模 $9$ 的立方残差为 $\{0, 1, 8\}$。仅此而已!

    当然,要使 $x^3 + y^3 + z^3 = 3$ 成立,必须满足 $x^3 + y^3 + z^3 \equiv 3 \mod{9}$,因为 $3 \equiv 3 \mod{9}$。观察可能的立方残差——$0, 1, 8$——我们发现\emph{仅有} $1 + 1 + 1$ 模 $9$ 余 $3$。其他组合如 $0 + 1 + 8 \equiv 9 \equiv 0 \mod{9}$ 和 $8 + 8 + 8 \equiv 24 \equiv 6 \mod{9}$ 等均不符合。这意味着解 $(x, y, z)$ 必须满足 $x^3 \equiv y^3 \equiv z^3 \equiv 1 \mod{9}$。

    由此,我们证明了一个稍强的结论:不仅 $x^3, y^3, z^3$ 模 $9$ 同余,它们还必须模 $9$ 余 $1$。这比原要求更进一步。

    事实上,此问题存在一个\emph{更强}的结论:$x \equiv y \equiv z \mod{9}$。换言之,不仅它们的\emph{立方}模 $9$ 同余,其\emph{底数}本身也模 $9$ 同余。(注意,这并非指底数模 $9$ 余 $1$;例如 $(4, 4, -5)$ 就表明情况并非如此。)遗憾的是,证明这一点需要涉及很多高等数学,超出了本书的范围。但这应该能让你理解,这些``简单''的问题(表述简洁、数值较小、纯粹整数)的解决往往需要复杂而深奥的数学工具。不过,请不要将其视为难点,而应视为启发:仅用少量数学知识,我们便能触及问题表层,其下隐藏着更为深刻而复杂的根基。

    如果你感兴趣,可以参考以下论文获取完整结论:
    \begin{center}
        \href{http://www.ams.org/journals/mcom/1985-44-169/S0025-5718-1985-0771049-4/S0025-5718-1985-0771049-4.pdf}{J. W. S. Cassels, ``A Note on the Diophantine Equation'', \\Mathematics of Computation, 44(169): 265-266, 1985.}
    \end{center}
    
    它证明了 $x \equiv y \equiv z \pmod{9}$ 的必要性。然而,即便是阅读前两段,你也需要查阅若干定义。通读全文更是要求学习相关数学知识,耗时可能数月乃至数年,取决于你的兴趣。请牢记这一点,并在未来的数学之旅中重温此问题!
\end{example}
