% !TeX root = ../../../book.tex

\subsection{启下}

你可能在微积分中处理过函数(如微分与积分),在高中代数课上绘制过函数图像或求根,甚至在计算机科学中编写过递归算法。但尝试\emph{定义}函数本身是另一码事——如何向从未接触数学的人解释?如何向超智能外星人说明?若以数学归纳法般的严谨性表述,该怎么做?这并非易事。

为了深入理解\emph{函数}概念,我们需要先探讨\emph{关系}——一种比较集合元素的方法。通过大量例子了解其性质后,下一章将揭示函数本质是关系的特殊类型!讨论过程中,我们将探究关系属性,发现特定属性组合形成的重要性质。尤其将看到\emph{等价关系}自然生成集合\emph{划分},反之亦然。这一洞见将助力我们陈述并证明关于整数的关键结论。