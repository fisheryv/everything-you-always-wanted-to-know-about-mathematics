% !TeX root = ../../../book.tex

\subsection{启下}

你可能在微积分中处理过函数(比如微分和积分),或者在高中代数课上绘制过函数图像或求解过函数的根,甚至在计算机科学中编写过算法或使用过递归编程。但是,试着\emph{定义}一下什么是函数。你会怎么向从未学过数学的人解释呢?又会如何向一个超智能的外星人解释呢?如果用我们在数学归纳法中那样的严谨程度来解释,你会怎么做?这并不容易,对吧?

为了深入理解\emph{函数}的概念,我们首先要讨论\emph{关系},这是比较集合中元素的一种方法。我们会通过许多例子来了解它们的性质。然后,在下一章中,我们会发现函数其实是关系的一种特殊类型!在讨论关系的过程中,我们会探讨它们的属性,并发现某些属性的组合会形成特定的性质。具体来说,我们会看到\emph{等价关系}会自然地生成集合\emph{划分},反之亦然。这一发现将帮助我们陈述并证明一些关于整数的重要结果。