% !TeX root = ../../../book.tex

\section{引言}

本章将继续探讨函数的第二部分内容。我们将正式给出函数的\emph{定义}。本质上,函数是具有特定性质的\textbf{关系}。这也解释了为何前期需要深入研究关系——不仅仅是因为关系本身具有的趣味性和实用性。定义函数后,我们将通过大量示例与证明,系统分析函数可能具备的各种性质。在此过程中,我们将综合运用先前所学的全部概念,特别是 \ref{sec:section4.9} 节中的证明技巧。

本章后半部分将引入\emph{双射函数} (Bijective Function) 的概念,即集合间元素的一一对应关系,用以讨论集合的``大小''及其比较方法。该主题称为\emph{基数} (Cardinality),它将揭示无限集领域中若干惊人且反直觉的结论。这也为下一章关于有限集及其计数方法的研究奠定基础。

% !TeX root = ../../../book.tex

\subsection{目标}

以下简要说明本章在本书中的定位。我们将阐释先前内容如何发挥作用,阐述研究本章主题的动机,明确学习目标,并提示阅读时的关注重点。我们会先列出本章的核心目标及学成后应掌握的技能,后续章节将详细展开。学完本章后,请返回此处核验:你是否理解所有目标?能否阐述其重要性?能否定义相关术语?能否运用相关技术?

\textbf{学完本章后,你应该能够……}

\begin{itemize}
    \item 定义函数并给出多个示例。
    \item 借助函数的非形式化描述与可视化图示,构建关于函数性质(示例与反例)的形式化论证。
    \item 在函数框架下定义集合的像 (Image) 与原像 (Pre-image),并证明相关运算性质。
    \item 阐述函数的性质特征,运用相关方法判定并证明给定函数是否具备特定性质。
    \item 求函数复合,阐释如何构造新函数,并分析证明复合运算对函数性质的影响。
    \item 描述双射函数 (Bijective Function) 与反函数 (Inverse Function) 的关联,运用其解决问题并证明结论。
    \item 利用双射定义集合的基数 (Cardinality),并证明相关基数定理。
    \item 辨析有限集、可数无限集与不可数无限集之间的区别,并提供各类集合的典型实例。
\end{itemize}

% !TeX root = ../../../book.tex

\subsection{承上}

前一章介绍的重要概念\textbf{关系}将在本章发挥重要作用。我们已经提到,\textbf{函数}实际上是关系的一种特例,这一点将在我们对函数的正式定义中体现。

至于前一章讨论的其他概念,比如等价关系和数论结果,它们在本章中不会直接出现。也就是说,本章中我们探讨的函数及其性质,并不依赖于这些概念。然而,我们会利用这些概念来设计一些有趣的示例和练习。

% !TeX root = ../../../book.tex

\subsection{启下}

正如我们在上一章提到的,你很可能已经对函数的概念和使用方法有了一些直观的理解。这些理解可能来自你之前的数学课程,或者来自计算机编程的经验。我们一直强调,希望能正确、正式地用\emph{数学方式}定义我们所研究的概念,函数也不例外!通过正式定义,我们可以更好地讨论一些你可能以前见过但无法清楚表达的函数性质。此外,函数的某些特定性质还会帮助我们讨论集合的\emph{基数}。请相信我,如果不先探讨函数,我们就无法对这个主题进行深入讨论。


% !TeX root = ../../../book.tex

\subsection{忠告}

我们还要重申上一章中的一些忠告和建议。我们正在继续探索一些抽象的数学领域。本章特别是要将一个你可能在视觉和直觉上熟悉的概念放在更严格的基础之上。我们会尽可能利用大家的直觉,但也无法避免地需要进行抽象思维和问题解决的过程。特别是,我们无法总是将函数与它的图像联系起来,尽管这是我们在数学学习早期常用且有效的方法。此外,在基数的讨论中,我们会遇到一些完全违反直觉的结果。真的!这些奇怪又反直觉的事实需要我们保持开放的心态去理解。