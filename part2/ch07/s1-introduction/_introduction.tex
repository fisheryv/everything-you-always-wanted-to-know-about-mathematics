% !TeX root = ../../../book.tex

\section{引言}

本章将继续探讨函数的第二部分内容。我们将正式给出函数的\emph{定义}。本质上,函数是具有特定性质的\textbf{关系}。这也解释了为何前期需要深入研究关系——不仅仅是因为关系本身具有的趣味性和实用性。定义函数后,我们将通过大量示例与证明,系统分析函数可能具备的各种性质。在此过程中,我们将综合运用先前所学的全部概念,特别是 \ref{sec:section4.9} 节中的证明技巧。

本章后半部分将引入\emph{双射函数} (Bijective Function) 的概念,即集合间元素的一一对应关系,用以讨论集合的``大小''及其比较方法。该主题称为\emph{基数} (Cardinality),它将揭示无限集领域中若干惊人且反直觉的结论。这也为下一章关于有限集及其计数方法的研究奠定基础。

% !TeX root = ../../../book.tex

\subsection{目标}

以下简短内容将向你展示本章如何融入本书的体系。我们会解释之前的工作对本章研究的帮助,说明我们为什么要探讨本章的主题,并告诉你我们的目标以及在阅读时需要注意的事项。现在,我们将通过几条陈述总结本章的主要目标。这些陈述概括了你在完成本章后应掌握的技能和知识。接下来的章节会更详细地解释这些思想,这里仅提供一个简要列表供你参考。完成本章后,请返回这个列表,检查你是否理解所有目标。你能看出我们为什么认为这些目标重要吗?你能解释我们使用的术语并应用我们描述的技术吗?

\textbf{学完本章后,你应该能够……}

\begin{itemize}
    \item 定义函数,并提供多个例子。
    \item 使用函数的非正式描述和可视化图像来构建关于函数(示例和反例)及其属性的正式论证。
    \item 在函数的上下文中定义集合的像 (Image) 和原像 (Pre-image),并证明这些操作的各种属性。
    \item 陈述函数的属性,并应用相关方法来确定和证明给定函数是否具有这些属性。
    \item 找出两个函数的复合,说明如何用它们来创建新函数,并解释和证明复合对所涉及函数属性的影响。
    \item 描述双射函数 (Bijective Function) 和反函数 (Inverse Function) 之间的关系,并用它们来解决问题和证明结论。
    \item 使用双射来定义集合的基数 (Cardinality),并证明关于这些基数的结论。
    \item 说明有限集、可数无限集和不可数无限集之间的区别,并提供每种类型的多个例子。
\end{itemize}

% !TeX root = ../../../book.tex

\subsection{承上}

前一章介绍的重要概念\textbf{关系}将在本章发挥核心作用。我们已经指出\textbf{函数}实际上是一种特殊的关系,这一点将在其正式定义中得以体现。

至于前一章讨论的其他概念(如等价关系及数论相关结论),本章不会直接涉及。换言之,本章对函数及其性质的探讨不依赖于这些概念,但我们将运用它们设计若干启发性示例与练习。

% !TeX root = ../../../book.tex

\subsection{启下}

正如前一章所述,你可能已经通过数学课程或编程实践对函数有了一定直观理解。我们始终强调要以严谨的\emph{数学语言}定义研究对象,函数自然也不例外!形式化定义将帮助我们精准描述那些既往难以言明的函数性质。此外,函数的特定性质还将为后续探讨集合的\emph{基数}奠定基础——请相信我,如果不深入掌握函数理论,便无法真正进入这一主题。


% !TeX root = ../../../book.tex

\subsection{忠告}

我们还要重申上一章中的一些忠告和建议。我们正在继续探索一些抽象的数学领域。本章特别是要将一个你可能在视觉和直觉上熟悉的概念放在更严格的基础之上。我们会尽可能利用大家的直觉,但也无法避免地需要进行抽象思维和问题解决的过程。特别是,我们无法总是将函数与它的图像联系起来,尽管这是我们在数学学习早期常用且有效的方法。此外,在基数的讨论中,我们会遇到一些完全违反直觉的结果。真的!这些奇怪又反直觉的事实需要我们保持开放的心态去理解。