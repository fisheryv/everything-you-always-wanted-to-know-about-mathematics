% !TeX root = ../../../book.tex

\subsection{目标}

以下简短内容将向你展示本章如何融入本书的体系。我们会解释之前的工作对本章研究的帮助,说明我们为什么要探讨本章的主题,并告诉你我们的目标以及在阅读时需要注意的事项。现在,我们将通过几条陈述总结本章的主要目标。这些陈述概括了你在完成本章后应掌握的技能和知识。接下来的章节会更详细地解释这些思想,这里仅提供一个简要列表供你参考。完成本章后,请返回这个列表,检查你是否理解所有目标。你能看出我们为什么认为这些目标重要吗?你能解释我们使用的术语并应用我们描述的技术吗?

\textbf{学完本章后,你应该能够……}

\begin{itemize}
    \item 定义函数,并提供多个例子。
    \item 使用函数的非正式描述和可视化图像来构建关于函数(示例和反例)及其属性的正式论证。
    \item 在函数的上下文中定义集合的像 (Image) 和原像 (Pre-image),并证明这些操作的各种属性。
    \item 陈述函数的属性,并应用相关方法来确定和证明给定函数是否具有这些属性。
    \item 找出两个函数的复合,说明如何用它们来创建新函数,并解释和证明复合对所涉及函数属性的影响。
    \item 描述双射函数 (Bijective Function) 和反函数 (Inverse Function) 之间的关系,并用它们来解决问题和证明结论。
    \item 使用双射来定义集合的基数 (Cardinality),并证明关于这些基数的结论。
    \item 说明有限集、可数无限集和不可数无限集之间的区别,并提供每种类型的多个例子。
\end{itemize}