% !TeX root = ../../../book.tex

\subsection{目标}

以下内容简要说明本章在本书中的定位。我们将解释前期工作如何为本章研究奠定基础,阐明探讨本章主题的动机,并概述学习目标及注意事项。我们会先总结本章的主要目标,概括你在学完本章后应掌握的技能与知识。后续章节将详细展开这些思想,此处仅提供一个简要列表作为学习指引。完成本章后,请你返回此列表,确认自己是否达成了所有目标。你是否能理解这些目标的重要性?能否清晰地解释相关术语并熟练地应用相关技术?

\textbf{学完本章后,你应该能够……}

\begin{itemize}
    \item 定义函数并给出多个示例。
    \item 借助函数的非形式化描述与可视化图示,构建关于函数性质(示例与反例)的形式化论证。
    \item 在函数框架下定义集合的像 (Image) 与原像 (Pre-image),并证明相关运算性质。
    \item 阐述函数的性质特征,运用相关方法判定并证明给定函数是否具备特定性质。
    \item 求函数复合,阐释如何构造新函数,并分析证明复合运算对函数性质的影响。
    \item 描述双射函数 (Bijective Function) 与反函数 (Inverse Function) 的关联,运用其解决问题并证明结论。
    \item 利用双射定义集合的基数 (Cardinality),并证明相关基数定理。
    \item 辨析有限集、可数无限集与不可数无限集之间的区别,并提供各类集合的典型实例。
\end{itemize}