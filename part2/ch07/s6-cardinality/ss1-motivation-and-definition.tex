% !TeX root = ../../../book.tex

\subsection{引言与定义}

我们如此关心双射的一个重要原因是,它们可以帮助我们比较\textbf{集合的大小}。你可能对这个概念有一些直觉。例如,很明显,集合
\[\{1, 2, 3, 4, 5\}\]
有 $5$ 个元素,它是一个\textbf{有限集}。而集合
\[\mathbb{N} = \{1, 2, 3, 4, 5, \dots \}\]
则是\textbf{无限集}。我们知道 $\mathbb{Z}$ 也是一个无限集合,$\mathbb{Q}$ 和 $\mathbb{R}$ 也是如此。那么,这些集合的大小是多少?我们能够比较它们的大小吗?我们如何\emph{在数学上}进行比较?\emph{无限集}到底意味着什么?是否存在``不同的无穷大''?

\subsubsection*{双射``配对''元素}

假设我们面前的桌子上有 $5$ 支笔和 $5$ 本书。但假设我们不知道如何\textbf{数}它们。我们如何验证笔的数量和书的数量是一样的呢?我们不能说``有 $5$ 支笔和 $5$ 本书,且 $5 = 5$'',我们能否在不知道数量的情况下,证明书的集合和笔的集合一样大呢?

这时\emph{双射}就派上用场了。我们可以一对一地将笔和书\emph{配对}。我们可以将它们排在桌子上,并在它们之间画一条线,显示它们之间的对应关系。用集合的语言来说,我们在识别笔的集合和书的集合之间的双射。这一思想如此重要,以至于我们要用一句话来强调它:

\begin{quotation}
    \large{在基数的国度里,双射为王。}
\end{quotation}

想象我们的基数研究是一段穿越基数王国的旅程。在这个王国里,我们向双射王鞠躬,因为他统治一切。只有他能告诉我们两个集合何时具有\emph{相同的基数},无论它们是有限集还是无限集。

此外,我们确实\emph{需要}使用这种集合术语,因为我们将看到一些令人惊讶和反直觉的结果。使用这些正式的定义和概念将使我们更加严谨和精确。我们看到的例子和结果可能会让我们感到震惊(甚至是非常震惊!),但将它们根植于我们已经看到的概念和我们已经证明的定理中,让我们在数学上真正相信这些结果!

\subsubsection*{定义与符号}

首先,我们来定义什么是\textbf{有限集}。

\begin{definition}
    设 $S$ 为任意集合。我们说 $S$ 是\dotuline{有限集} 当且仅当
    \[\exists n \in \mathbb{N} \cup \{0\} \;\text{使得存在双射}\; f : S \to [n]\]
    在这种情况下,我们写做 $|S| = n$ 来表示 $S$ 的\dotuline{大小}为 $n$。
\end{definition}

注意:空集 $S = \varnothing$ 是有限集,因为 $[0] = \varnothing$。这就是为什么我们在定义中说 $n \in \mathbb{N} \cup \{0\}$,而不仅仅是 $n \in \mathbb{N}$。函数 $f : \varnothing \to \varnothing$ 是一个双射,它实际上是一个\emph{空关系}。(请记住,函数是一种关系!)

根据定义,形如 $[n]$ 的集合是有限集。它们是标准的有限集例子,大小为 $|[n]| = n$。因此,要证明一个集合 $S$ 的大小为 $n$,我们需要找到 $S$ 和 $[n]$ 之间的双射。例如,考虑集合 $\{1, 3, 5\}$。显然,它的大小为 $3$。我们可以通过给出双射 $f : \{1, 3, 5\} \to [3]$ 来证明这一点,该双射定义为 $f(1) = 1, f(3) = 2, f(5) = 3$。

思考一下有限集是否可以有两个\emph{不同的}大小,这个问题很有趣。虽然定义上没有排除这种可能,但我们可以\emph{证明}有限集的大小是唯一的。想一想如何证明这一点……我们将在几个基本定义之后给出证明。

\begin{definition}
    设 $S$ 为任意集合。我们说 $S$ 是\dotuline{无限集} 当且仅当 $S$ 不是有限集。

    也就是说,如果 $\forall n \in \mathbb{N} \cup \{0\}$,所有可能的函数 $f : S \to [n]$ 都不是双射,则 $S$ 是无限集。

    当 $S$ 是无限集时,我们用 $|S|$ 表示集合的\dotuline{基数}。
\end{definition}

将无限集定义为``非有限集''可能显得有些奇怪,但它确实反映了这两个概念之间的直观对立。一个集合不可能既是有限集又是无限集,所以我们不妨对其中一个进行定义,并将另一个定义为``其他所有情况''。

此外,我们\textbf{不会}用 $|S| = \infty$ 来表示集合是无限的。正如我们很快会看到的,实际上\textbf{存在许多不同``级别''的无限集}。虽然这听起来可能非常奇怪,但你会理解我们的意思。是的,无限集有不同的``大小'',我们将使用 $|S|$ 来表示集合 $S$ 的\textbf{基数},以便与其他集合进行比较。如果写 $|S| = \infty$,就意味着只有``一个无限'',这是错误的。

接下来,我们将主要区分两种\emph{类型}的无限集,以便展示一些关于我们已经熟悉的集合(如 $\mathbb{N}, \mathbb{Z}, \mathbb{Q}, \mathbb{R}$)的有趣结果。以下定义将解释这两种类型。

\begin{definition}
    设 $S$ 为任意集合。

    我们说 $S$ 为\dotuline{可数无限集}当且仅当存在双射 $f : S \to \mathbb{N}$。

    我们说 $S$ 为\dotuline{不可数无限集}(或\dotuline{不可数的})当且仅当 $S$ 是无限集且每一个函数 $f : S \to \mathbb{N}$ 都不是双射。
\end{definition}

给定一个无限集 $S$,根据 $S$ 的基数 $|S|$ 与 $\mathbb{N}$ 的比较,可以将 $S$ 分为两种情况。我们称之为\textbf{可数无限},因为这解释了我们为什么直观上认为 $\mathbb{N}$ 是无限的。集合 $\mathbb{N}$ 有``很多''元素,多到我们永远数不完;然而,我们可以尝试数这些元素,这本身就说明了一些特别的性质。$\mathbb{N}$ 有第一个元素、第二个元素、第三个元素,依此类推……虽然我们一生中无法命名所有元素,但可以编程一个神奇的、不朽的机器人一个接一个地打印出来。无论我们事先想到多大的自然数,机器人\emph{最终}都会打印出来。

但并非所有无限集都能这样处理。这就是\textbf{不可数无限}的概念。这样的集合是无限的,但没有与 $[n]$ 形式集合的对应关系,而且``如此之大''以至于无法确定``第一个元素''、``第二个元素''和``第三个元素''等等。\textbf{双射} $f : S \to \mathbb{N}$ 可以标记 $S$ 的所有元素,表明它们与自然数配对。如果无法做到这一点,那么这个集合就是不可数无限的。你可能不相信存在这样的集合!别担心,我们会展示一些例子。现在,只需了解\textbf{可数}和\textbf{不可数}无限的区别:区别在于是否存在与 $\mathbb{N}$ 的\textbf{双射}。

\subsubsection*{基数比较}

正如我们之前提到的,当集合 $S$ 是无限集时,我们用 $|S|$ 来\textbf{比较} $S$ 和其他集合的基数。我们不会写成 $|S| = \infty$。相反,我们会写成 $|S| = |T|$,表示 $S$ 和 $T$ 具有\emph{相同的}基数,不管具体是多少。我们还可能写成 $|S| < |P|$,表示集合 $P$ 的基数\emph{严格大于} $S$。下面的定义将告诉我们如何通过函数,特别是不同类型的映射,来比较基数。

\begin{definition}
    设 $S, T$ 为任意集合。

    \begin{itemize}
        \item 我们写 $|S|=|T|$ 当且仅当存在\textbf{双射} $ f : S \to T$。\\
            在这种情况下,我们说 $S$ 和 $T$ 具有\textbf{相同的基数}。
        \item 我们写 $|S|\le|T|$ 当且仅当存在\textbf{单射} $ f : S \to T$。\\
            在这种情况下,我们说 $S$ 的基数\textbf{最多}为 $|T|$。
        \item 我们写 $|S|<|T|$ 当且仅当 $|S|\le|T|$ 且 $|S|\ne|T|$。\\
            在这种情况下,我们说 $S$ \textbf{严格小于} $T$ 的基数。
        \item 我们写 $|S|\ge|T|$ 当且仅当存在\textbf{满射} $ f : S \to T$。\\
            在这种情况下,我们说 $S$ 的基数\textbf{最少}为 $|T|$。
        \item 我们写 $|S|>|T|$ 当且仅当 $|S|\ge|T|$ 且 $|S|\ne|T|$。\\
            在这种情况下,我们说 $S$ \textbf{严格大于} $T$ 的基数。
    \end{itemize}
\end{definition}

我们可以用两种不同的方式来解释这些定义背后的动机:

一般来说,$f : A \to B$ 是单射表示 $|A| \le |B|$,而 $g : A \to B$ 是满射表示 $|A| \ge |B|$。可以通过 $f$ 和 $g$ 的示意图来理解这个定义的含义。$A \to B$ 存在单射意味着我们可以明确地将 $A$ 的元素配对到 $B$ 的元素上,而不会有重叠,但 $B$ 中可能会剩下更多的元素。同样地,从 $A to B$ 存在满射意味着我们可以用 $A$ 的元素覆盖所有的 $B$,但有时可能会有重叠,因此 $A$ 可能有比 $B$ 更多的元素。这两种情况同时存在(即 $A$ 到 $B$ 的双射)意味着 $A$ 和 $B$ 实际上具有相同的基数:我们可以配对它们的所有元素。这是一个直观的解释,用来激发这些定义的动机。请注意,这些解释并不是严格的证明。但现在我们已经做出了这些定义,我们可以用它们来证明和证伪陈述!要比较集合的基数 --- 即使是无限集 --- 我们只需要找到一个具有适当性质的函数。本章其余的工作将在我们穿越基数王国的旅程中非常有帮助。

另一种理解这些定义的方法是,``具有相同基数''是一种``所有集合的集合''上的``等价关系''。我们需要给这些短语加上引号,因为正如 \ref{sec:section3.3.5} 节中关于罗素悖论的详细解释,并不存在``所有集合的集合''。因此,在我们的语境中,讨论这个``集合''上的等价关系在数学上是没有意义的。不过,从某种模糊的角度来看,这就是实际情况:

\begin{itemize}
    \item 给定任意集合 $S$,它当然跟它自身存在双射,即恒等函数 $\id_S : S \to S$。这表明 $|S|=|S|$,即``具有相同基数''关系具有``自反性''。
    \item 假设 $|S| = |T|$,则存在双射 $f:S \to T$。是否也存在双射 $g : T \to S$ 呢?当然存在!我们可以令 $g=f^{-1}$。我们知道 $g$ 也是双射。这表明 $|T| = |S|$,即``具有相同基数''关系具有``对称性''。
    \item 假设 $|S| = |T| = |U|$,则存在双射 $f : S \to T$ 和 $g : T \to U$。是否也存在双射 $h : S \to U$ 呢?当然存在!因为复合 $g \circ h$ 也是双射(这在之前的联系中已经证明了)。这表明 $|S| = |U|$,即``具有相同基数''关系具有``传递性''。
\end{itemize}

再次强调,这并不是\emph{完全准确}的描述,但确实能帮助你理解这些复杂的抽象概念。我们正在建立一种方法,利用函数来比较任意两个集合的基数。所有集合将根据它们的基数被划分为不同的类别。令人惊讶的是,我们即将证明:\emph{基数的种类是去穷多的}。

\subsubsection*{康托尔定理}

以下结论和证明来自 $19$ 世纪中后期德国数学家乔治·康托尔 (Georg Cantor) 的研究。如今,数学家们已经完全接受了这一结论及其影响。然而,在当时这一观点极具争议,以至于一些数学家拒绝相信。随着时间的推移,康托尔的工作和思想推动了形式集合论的发展。

这一结论的证明被称为\textbf{康托尔对角线论证}。稍后我们将使用类似的论证,并解释为什么它像一个``对角线''。现在,我们更关注这个定理的结论。

\begin{theorem}
    设 $S$ 为任意集合。则 $|S| < |\mathcal{P}(S)|$。
\end{theorem}

这表明\textbf{集合的幂集的基数总是严格大于集合本身的基数}。对于有限集合来说,这一点是显而易见的。你已经知道,$[n]$ 的幂集有 $2^n$ 个元素,也就是说 $|\mathcal{P}([n])| = 2^n$。(你将在练习 \ref{exc:exercises7.8.30} 中利用基数的相关结果通过归纳法证明这一点。)我们可以看到,对于每一个 $n \in \mathbb{N}$,都有 $n < 2^n$。然而,这个定理还指出,这种关系对\textbf{无限集}同样适用。哇!这意味着存在一条无限集的链条,每一个集合都比前一个更大。我们可以不断取前一个集合的幂集:
\[|\mathbb{N}| < |\mathcal{P}(\mathbb{N})| < \big|\mathcal{P}\big(\mathcal{P}(\mathbb{N})\big)\big| < \Big|\mathcal{P}\Big(\mathcal{P}\big(\mathcal{P}(\mathbb{N})\big)\Big)\Big| < \dots
\]
接下来我们来证明这个定理。这个证明过程非常简洁巧妙,所以不必担心如何构思这样的论证,只需专注于理解其逻辑流程即可。

\begin{proof}
    设 $S$ 为任意集合。为了引出矛盾而假设 $|S| \ge |\mathcal{P}(S)|$。

    这意味着存在满射函数 $g : S \to \mathcal{P}(S)$。

    定义 $T = \{X \in S \mid X \notin g(X)\}$。(这是合理的,因为对于任意 $X \in S, g(X) \in \mathcal{P}(S)$,即 $g(X) \subseteq S$ 。因此,要么 $X \in g(x)$ 要么 $X \notin g(x)$。)

    根据集合构建符的定义可得 $T \subseteq S$。着意味着 $T \in \mathcal{P}(S)$。

    因为 $g$ 是满射,所以 $\exists Y \in S$ 使得 $g(Y) = T$。给定这样的 $Y$。

    此时,$Y \in T$ 是否成立?我们考虑两种情况:
    \begin{itemize}
        \item 如果 $Y \in T$,则根据 $T$ 的定义可得 $Y \notin g(Y)$。然而,$g(Y) = T$,这意味着 $Y \notin T$。这与假设矛盾。$\hashx$
        \item 如果 $Y \notin T$,则根据 $T$ 的定义可得 $Y \in g(Y)$。然而,$g(Y) = T$,这意味着 $Y \in T$。这与假设矛盾。$\hashx$
    \end{itemize}
    无论是上述哪种情况,都会推导出矛盾。$\hashx$

    因此不存在从 $S$ 到 $\mathcal{P}(S)$ 的满射,即 $|S| < |\mathcal{P}(S)|$。
\end{proof}

回顾 \ref{sec:section7.4.5} 节的练习 \ref{exc:exercises7.4.4}。请注意,我们让你定义一个从 $\mathbb{N}$ 到 $\mathcal{P}(\mathbb{N})$ 的函数,并证明它不是满射。我们不需要知道你定义的函数是什么!因为我们知道这个定理,所以知道你不可能定义一个满射!

\subsubsection*{讨论:公理与定义}

必须承认一点。我们略过了一些关于\emph{定义}和\emph{定理}的细节,定理是需要从基本假设中证明的结论。根据\emph{定义}(至少在我们的语境下),从 $A$ 到 $B$ 的单射和满射(请注意方向)构成了等基数的充分证明,从而保证了一个双射。同样,从 $A$ 到 $B$ 和从 $B$ 到 $A$ 的单射足以保证 $|A| = |B|$,因此,$A$ 和 $B$ 之间必须存在一个双射。

虽然这些说法看起来是正确的,但并\emph{不是显而易见的}。假设我们有一个从 $A$ 到 $B$ 的单射函数和一个从 $B$ 到 $A$ 的单射函数。这是否能保证两个集合之间存在双射?我们希望如此,但这并不能作为证明。实际上,这个结果被称为\textbf{康托尔-施罗德-伯恩斯坦 (Cantor-Schroeder-Bernstein) 定理}:

\begin{theorem}[康托尔-施罗德-伯恩斯坦定理]
    假设 $A,B$ 为任意集合,$f : A \to B$ 和 $g : B \to A$ 为单射。则存在双射 $h : A \to B$。
\end{theorem}

没错,这是一个\emph{定理},而且并不简单!事实上,在该定理的证明中存在一种\emph{构造性}证明:它提供了一种算法,通过使用两个单射 $f : A \to B$ 和 $g : B \to A$ 来构造双射 $h : A \to B$。对于我们的目标 --- 以及时间和空间的限制 --- 没有必要将其单独作为一个定理,更别说还是一个具有构造性证明的定理。只要把单射和满射及其在基数方面的结果当作定义来理解就足够了;这些结果看起来很直观,我们可以接受它们。不过,请注意,我们是基于严格的数学知识来接受这些结果的。如果你对这些细节及其影响感兴趣,可以考虑参加一门关于\textbf{集合论}的课程或阅读一本关于\textbf{集合论}的书。

本质上,真正的问题在于我们预设了\emph{任意}两个集合 $A$ 和 $B$ 的基数可以进行有意义的\emph{比较}。也就是说,对于任意的 $A$ 和 $B$,我们假设可以声明 $|A| \le |B|$ 或 $|B| \le |A|$(或者两者相等)。但我们如何\emph{保证}这种比较总是适用于任意两个集合呢?这并不是一个简单的问题!在本书中,我们假设任何两个集合的基数是可以比较的。然而,在更广泛的数学领域,这需要从更基本的假设中证明出来。

