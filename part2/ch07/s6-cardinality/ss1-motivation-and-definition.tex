% !TeX root = ../../../book.tex

\subsection{引言与定义}

我们如此关心双射的一个重要原因是,它们可以帮助我们比较\textbf{集合的大小}。你可能对此已有直觉认识:例如,集合
\[\{1, 2, 3, 4, 5\}\]
显然包含 $5$ 个元素,它是一个\textbf{有限集}。而集合
\[\mathbb{N} = \{1, 2, 3, 4, 5, \dots \}\]
则是\textbf{无限集}。我们知道 $\mathbb{Z}$, $\mathbb{Q}$ 和 $\mathbb{R}$ 也是都无限集。那么,这些集合的大小究竟是多少?我们能够比较它们的大小吗?我们如何\emph{在数学上}进行比较?\emph{无限集}到底意味着什么?是否存在``不同大小的无穷大''?

\subsubsection*{双射``配对''元素}

假设桌上有 $5$ 支笔和 $5$ 本书,但我们不知道如何计数。如何在不依赖数字的情况下验证笔和书的数量相同?显然不能直接说``两者都是 $5$'',此时\emph{双射}就派上了用场。我们可以将每支笔与每本书\emph{一一配对}。通过排列物品并建立对应关系,本质上是在笔的集合与书的集合之间构造双射。这一思想极其重要:
\begin{center}
    \large{在基数的国度里,双射为王。}
\end{center}

想象我们对基数的研究是一段穿越基数王国的旅程。在此领域中,我们臣服于双射王——因为他统治着一切。唯有他能判定两个集合是否具有\emph{相同的基数},无论是有限集还是无限集。

此外,我们\emph{必须}采用集合论语言,因为后续结论可能出人意料甚至令人震惊!严谨的定义和概念能确保推演的精确性。尽管某些结论可能反直觉,但将它们建立在既有概念和已证定理之上,方能获得在数学上真正令人信服的结论!

\clearpage

\subsubsection*{定义与符号}

首先,我们定义什么是\textbf{有限集}。

\begin{definition}
    设 $S$ 为任意集合。称 $S$ 是\dotuline{有限集} 当且仅当
    \[\exists n \in \mathbb{N} \cup \{0\} \text{\ 使得存在双射\ } f : S \to [n]\]
    在这种情况下,我们用 $|S| = n$ 表示 $S$ 的\dotuline{大小}为 $n$。
\end{definition}

注意:空集 $S = \varnothing$ 是有限集,因为 $[0] = \varnothing$。这就是为什么我们在定义中取 $n \in \mathbb{N} \cup \{0\}$,而不仅仅是 $n \in \mathbb{N}$。函数 $f : \varnothing \to \varnothing$ 是一个双射,它实际上是一个\emph{空关系}。(请记住,函数是一种关系!)

根据定义,形如 $[n]$ 的集合是有限集。它们是标准的有限集实例,大小为 $|[n]| = n$。因此,要证明一个集合 $S$ 的大小为 $n$,需要构造 $S$ 和 $[n]$ 之间的双射。例如,考虑集合 $\{1, 3, 5\}$。显然,它的大小为 $3$。我们可以通过构造双射 $f : \{1, 3, 5\} \to [3]$ 来证明这一点,该双射定义为 $f(1) = 1, f(3) = 2, f(5) = 3$。

思考一下有限集是否可以有两个\emph{不同的}大小?这是一个有趣问题。尽管定义未明确排除这种可能性,但可以\emph{证明}有限集的大小是唯一的。想一想如何证明这一点……我们将在完成基本定义之后给出证明。

\begin{definition}
    设 $S$ 为任意集合。称 $S$ 是\dotuline{无限集} 当且仅当 $S$ 不是有限集。

    也就是说,若 $\forall n \in \mathbb{N} \cup \{0\}$,所有可能的函数 $f : S \to [n]$ 都不是双射,则 $S$ 是无限集。

    当 $S$ 为无限集时,我们用 $|S|$ 表示集合的\dotuline{基数 (Cardinality)}。
\end{definition}

将无限集定义为``非有限集''可能有些奇怪,但它确实反映了这两个概念之间的直观对立。一个集合不可能既是有限集又是无限集,因此通常选择定义其中一个,而将另一个定义为``其他所有情况''。

此外,我们\textbf{不会}用 $|S| = \infty$ 来表示集合是无限的。正如后续将看到的,实际上\textbf{存在许多不同``级别''的无限集}。虽然这听起来可能非常奇特,但你会理解其中含义。没错,无限集也有不同的``大小'',我们将用 $|S|$ 表示集合 $S$ 的\textbf{基数},以便比较集合的大小。若写成 $|S| = \infty$,则暗示只有``一种无限'',这是错误的。

接下来,我们将主要区分两种\emph{类型}的无限集,并展示关于常见集合(如 $\mathbb{N}, \mathbb{Z}, \mathbb{Q}, \mathbb{R}$)的有趣结论。以下定义将解释这两种类型。

\begin{definition}
    设 $S$ 为任意集合。

    称 $S$ 为\dotuline{可数无限集},当且仅当存在双射 $f : S \to \mathbb{N}$。

    称 $S$ 为\dotuline{不可数无限集}(或\dotuline{不可数的}),当且仅当 $S$ 是无限集且每个函数 $f : S \to \mathbb{N}$ 都不是双射。
\end{definition}

给定无限集 $S$,可以根据其基数 $|S|$ 与 $|\mathbb{N}|$ 的关系分为两类。称之为\textbf{可数无限}体现了 $\mathbb{N}$ 的直观特性:$\mathbb{N}$ 的元素``无穷无尽'',但我们可以尝试枚举它们。它具有第一个元素、第二个元素、第三个元素……尽管无法在有限时间内枚举全部,但可以设想一个不朽的神奇机器人逐个输出它们。无论指定哪个自然数,机器人\emph{最终}都会将其打印出来。

但并非所有无限集都能这样处理。这便是\textbf{不可数无限}的概念。这类集合是无限的,但却没有与 $[n]$ 形式集合的双射关系,它``如此之大''以至于无法定义``第一元素''、``第二元素''、``第三个元素''等等。存在\textbf{双射} $f : S \to \mathbb{N}$ 意味着可以标记 $S$ 的所有元素,实现与自然数的配对。若无法做到,则集合不可数。你可能质疑此类集合的存在!不必担心,我们将展示实例。现在只需理解\textbf{可数无限}与\textbf{不可数无限}的区别:关键在于是否存在到 $\mathbb{N}$ 的\textbf{双射}。

\subsubsection*{基数比较}

正如我们之前提到的,当集合 $S$ 为无限集时,我们使用 $|S|$ 来\textbf{比较} $S$ 与其他集合的基数。我们不会写做 $|S| = \infty$,而是写做 $|S| = |T|$ 表示 $S$ 与 $T$ 具有\emph{相同的}基数(无论具体数值如何),或写做 $|S| < |P|$ 表示集合 $P$ 的基数\emph{严格大于} $S$。以下定义通过函数(特别是不同类型的映射)阐述基数比较的方式:

\begin{definition}
    设 $S, T$ 为任意集合。

    \begin{itemize}
        \item 记 $|S|=|T|$,当且仅当存在\textbf{双射} $ f : S \to T$。\\
            此时称 $S$ 和 $T$ 具有\textbf{相同的基数}。
        \item 记 $|S|\le|T|$ 当且仅当存在\textbf{单射} $ f : S \to T$。\\
            此时称 $S$ 的基数\textbf{至多}为 $|T|$。
        \item 记 $|S|<|T|$ 当且仅当 $|S|\le|T|$ 且 $|S|\ne|T|$。\\
            此时称 $S$ 的基数\textbf{严格小于} $T$ 的基数。
        \item 记 $|S|\ge|T|$ 当且仅当存在\textbf{满射} $ f : S \to T$。\\
            此时称 $S$ 的基数\textbf{至少}为 $|T|$。
        \item 记 $|S|>|T|$ 当且仅当 $|S|\ge|T|$ 且 $|S|\ne|T|$。\\
            此时称 $S$ 的基数\textbf{严格大于} $T$ 的基数。
    \end{itemize}
\end{definition}

我们可以用两种不同的方式来解释这些定义背后的内涵:

一般来说,$f : A \to B$ 是单射表示 $|A| \le |B|$,而 $g : A \to B$ 是满射表示 $|A| \ge |B|$。可以通过示意图理解该定义:$A \to B$ 存在单射意味着能将 $A$ 的元素唯一配对到 $B$ 的元素(无重叠),但 $B$ 中可能有剩余元素;$A \to B$ 存在满射则意味着能用 $A$ 的元素覆盖 $B$ 的所有元素(可能有重叠),因此 $A$ 的元素可能更多。当这两种情况同时成立(即存在 $A$ 到 $B$ 的双射)时,表明 $A$ 和 $B$ 具有相同的基数——所有元素都能一一配对。这是一个直观的解释,旨在阐明定义的内涵,并非严格证明。但基于这些定义,我们就能证明或证伪相关命题!要比较集合的基数——即使是无限集——只需找到具有特定性质的函数。本章建立的方法将在探索基数王国的旅程中发挥重要作用。

另一种理解方式是:``具有相同基数''可视为``所有集合的集合''上的``等价关系''。这里需要为这些短语加引号,因为根据 \ref{sec:section3.3.5} 节罗素悖论的阐述,不存在``所有集合的集合''。因此在严格数学意义上,讨论该``集合''上的等价关系没有意义。但从不严谨的直观角度看,其性质如下:

\begin{itemize}
    \item 给定任意集合 $S$,它当然跟其自身存在双射,即恒等函数 $\id_S : S \to S$。这表明 $|S|=|S|$,即``具有相同基数''关系具有``自反性''。
    \item 假设 $|S| = |T|$,则存在双射 $f:S \to T$。是否也存在双射 $g : T \to S$ 呢?当然存在!我们可以令 $g=f^{-1}$,显然 $g$ 也是双射。这表明 $|T| = |S|$,即``具有相同基数''关系具有``对称性''。
    \item 假设 $|S| = |T| = |U|$,则存在双射 $f : S \to T$ 和 $g : T \to U$。是否也存在双射 $h : S \to U$ 呢?当然存在!因为复合 $g \circ h$ 也是双射(此前练习已证明)。这表明 $|S| = |U|$,即``具有相同基数''关系具有``传递性''。
\end{itemize}

再次强调,这并非\emph{完全严谨}的表述,但有助于理解抽象概念。我们正在建立一种通过函数比较任意两个集合基数的方法,所有集合将按其基数归类。令人惊讶的是,我们即将证明:\emph{基数的种类是可数无穷多的}。

\subsubsection*{康托尔定理}

以下结论和证明源于十九世纪中后期德国数学家乔治·康托尔 (Georg Cantor) 的研究。如今,数学家们已普遍接受这一结论及其深远影响。然而在当时,该观点极具争议性,以至于部分数学家拒绝承认。康托尔的工作和思想最终推动了形式集合论的发展。

说明其命此结论的证明被称为\textbf{康托尔对角线论证}。后文将展示类似论证,并解释为什么它以``对角线''命名。眼下,我们更关注定理本身。

\begin{theorem}
    设 $S$ 为任意集合。则 $|S| < |\mathcal{P}(S)|$。
\end{theorem}

该定理表明:\textbf{任意集合幂集的基数总是严格大于其本身的基数}。对有限集而言这是显然的——已知 $[n]$ 的幂集元素数为 $2^n$,即 $|\mathcal{P}([n])| = 2^n$(练习 \ref{exc:exercises7.8.30} 将要求利用基数的相关结论通过归纳法证明此结论)。显然 $n < 2^n$ 对所有 $n \in \mathbb{N}$ 均成立。而定理进一步指出:该关系对\textbf{无限集}同样成立。这意味着存在一条无限集的链条,每一个集合都比前一个更大。通过不断取前一个集合的幂集可得:
\[|\mathbb{N}| < |\mathcal{P}(\mathbb{N})| < \left|\mathcal{P}\left(\mathcal{P}(\mathbb{N})\right)\right| < \left|\mathcal{P}\left(\mathcal{P}\left(\mathcal{P}(\mathbb{N})\right)\right)\right| < \dots
\]
接下来我们来证明该定理。此证明过程简洁而精妙,因此无需纠结论证的构造,只需理解其逻辑脉络。

\begin{proof}
    设 $S$ 为任意集合。为了引出矛盾而假设 $|S| \ge |\mathcal{P}(S)|$。

    这意味着存在满射函数 $g : S \to \mathcal{P}(S)$。

    定义 $T = \{X \in S \mid X \notin g(X)\}$。(这是合理的,因为对于任意 $X \in S, g(X) \in \mathcal{P}(S)$,都有 $g(X) \subseteq S$。因此,要么 $X \in g(x)$ 要么 $X \notin g(x)$。)

    根据集合构建符的定义可得 $T \subseteq S$。这意味着 $T \in \mathcal{P}(S)$。

    因为 $g$ 为满射,所以 $\exists Y \in S$ 使得 $g(Y) = T$。给定这样的 $Y$。

    此时,$Y \in T$ 是否成立?考虑如下两种情形:
    \begin{itemize}
        \item 若 $Y \in T$,则根据 $T$ 的定义可得 $Y \notin g(Y)$。然而,$g(Y) = T$,这意味着 $Y \notin T$。这与假设矛盾。$\hashx$
        \item 若 $Y \notin T$,则根据 $T$ 的定义可得 $Y \in g(Y)$。然而,$g(Y) = T$,这意味着 $Y \in T$。这与假设矛盾。$\hashx$
    \end{itemize}
    
    无论上述哪种情形,都会推出矛盾。$\hashx$

    因此不存在从 $S$ 到 $\mathcal{P}(S)$ 的满射,即 $|S| < |\mathcal{P}(S)|$。
\end{proof}

回顾 \ref{sec:section7.4.5} 节的练习 \ref{exc:exercises7.4.4}。请注意,我们要求你定义一个从 $\mathbb{N}$ 到 $\mathcal{P}(\mathbb{N})$ 的函数,并证明它不是满射。我们无需知道你定义的函数具体是什么!因为我们知道相关定理,因此确定你不可能定义一个满射!

\subsubsection*{讨论:公理与定义}

必须承认一点:我们略过了一些关于\emph{定义}和\emph{定理}的细节。定理是需要从基本假设中证明的结论。根据\emph{定义}(至少在本书的语境下),一个从 $A$ 到 $B$ 的双射(即同时为单射和满射)足以证明 $|A| = |B|$。类似地,若存在从 $A$ 到 $B$ 的单射和从 $B$ 到 $A$ 的单射,也足以保证 $|A| = |B|$,从而 $A$ 和 $B$ 之间必然存在双射。

尽管这些结论看似合理,但并\emph{并非显而易见}。假设我们有一个从 $A$ 到 $B$ 的单射函数和一个从 $B$ 到 $A$ 的单射函数,这能否保证两个集合间存在双射?我们期望如此,但这不能作为证明。实际上,该结论被称为\textbf{康托尔-施罗德-伯恩斯坦 (Cantor-Schroeder-Bernstein) 定理}:

\begin{theorem}[康托尔-施罗德-伯恩斯坦定理]
    设 $A,B$ 为任意集合,$f : A \to B$ 和 $g : B \to A$ 为单射,则存在双射 $h : A \to B$。
\end{theorem}

没错,这是一个\emph{定理},且证明并不简单!事实上,该定理存在一种\emph{构造性}证明:它提供了一种算法,利用两个单射 $f : A \to B$ 和 $g : B \to A$ 来构造双射 $h : A \to B$。鉴于本书的目标以及时间和空间的限制,我们无需单独展开此定理,更不必讨论其构造性证明。将单射和满射在基数比较中的作用视为定义即可;这些结论看起来很直观,我们可以直接接受它们。但需注意,我们是基于严格的数学基础来接受此结论的。若你对细节及其影响感兴趣,可以考虑选修一门关于\textbf{集合论}的课程或阅读一本关于\textbf{集合论}的书。

本质上,核心问题在于我们预设了\emph{任意}两个集合 $A$ 和 $B$ 的基数可进行有意义的\emph{比较},即对于任意 $A$ 和 $B$,我们假设能断言 $|A| \le |B|$ 或 $|B| \le |A|$(或两者相等)。然而,如何\emph{保证}这种比较对所有集合成立?这并非简单问题!本书中,我们假设任何两个集合的基数均可比较。但在更广泛的数学领域,这需要从更基本的假设中证明出来。
