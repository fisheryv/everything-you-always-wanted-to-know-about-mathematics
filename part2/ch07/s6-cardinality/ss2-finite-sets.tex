% !TeX root = ../../../book.tex

\subsection{有限集}

在进入神秘而引人入胜的无限集世界之前,让我们先了解\textbf{有限集}的一些结论。这些结论直观易懂,并能帮助我们练习如何利用函数及其性质证明关于基数的事实。

\subsubsection*{定理}

对于每个结论,我们将陈述定理、命题或引理,然后直接给出证明或通过练习引导你完成证明。

\begin{theorem}\label{theorem7.6.7}
    若 $A,B$ 为互不相交的有限集,则 $|A \cup B| = |A| + |B|$。
\end{theorem}

我们通过几个例子来验证该结论为何成立。你能理解为什么要求集合\emph{互不相交}吗?你能证明这个说法吗?尝试证明此结论时请注意:需要在两个集合间建立\emph{双射}关系……

\begin{proof}
    设 $A$ 和 $B$ 为互不相交的有限集。

    已知 $\exists a, b \in \mathbb{N} \cup \{0\}$,并且存在双射 $f : A \to [a]$ 和 $g : B \to [b]$。(也就是说,我们假设 $|A| = a, |B| = b$)。给定这样的 $a, b, f, g$。

    我们要证明 $|A \cup B| = |A| + |B| = a + b$;也就是说,我们要证明存在双射 $h : A \cup B \to [a + b]$。

    定义函数 $h : A \cup B \to [a + b]$ 为
    \[\forall x \in A \cup B \centerdot h(x) = \begin{cases}
          f(x) & \text{如果\ } x \in A \\
        g(x)+a & \text{如果\ } x \in B
    \end{cases}\]

    请注意,函数 $h$ 是良好定义的,因为 $A \cap B = \varnothing$,所以每个 $x \in A \cup B$ 要么 $x \in A$ 要么 $x \in B$,不可能同时存在两个集合中。此外,对于每个 $x \in A, 1 \le h(x) \le a$;对于每个 $x \in B, a+1 \le h(x) \le a+b$,所以对于 $h$ 定义域中的每个 $x$, $h(x) \in [a+b]$ 均成立。

    定义 $h$ 的反函数 $H : [a + b] \to A \cup B$ 为
    \[\forall y \in [a + b] \centerdot H(y) = \begin{cases}
        f^{-1}(y) & \text{如果\ } 1 \le y \le a \\
        g^{-1}(y-a) & \text{如果\ } a+1 \le y \le a+b
    \end{cases}\]

    若 $H$ 是 $h$ 的反函数,则 $h$ 为双射。

    首先,证明 $H$ 是良好定义的。每个 $y \in [a + b]$ 都满足 $H$ 定义中的唯一一个不等式。由于 $f$ 和 $g$ 为双射,$f^{-1}$ 和 $g^{-1}$ 也是良好定义的函数且为双射。若 $a + 1 \le y \le a + b$,则 $1 \le y - a \le b$,故 $y - a \in [b]$(即 $g^{-1}$ 的定义域)。\\

    接着,证明 $h \circ H = \id_{[a+b]}$。给定 $y \in [a+b]$,则分两种情况讨论:
    \begin{enumerate}[label=(\arabic*)]
        \item 若 $1 \le y \le a$,即假设 $y \in [a]$。则
            \[(h \circ H)(y) = h\big(f^{-1}(y)\big) = f\big(f^{-1}(y)\big)= \id_{[a]}(y) = y\]
            此处使用了 $f^{-1}(y) \in A$ 这一事实。
        \item 若 $a + 1 \le y \le b$。即假设 $y-a \in [b]$。则
            \begin{align*}
                (h \circ H)(y) = h(H(y)) &= h\big(g^{-1}(y-a)\big) = g\big(g^{-1}(y-a)\big)+a \\
                &= \id_{[b]}(y - a) + a = (y - a) + a = y
            \end{align*}
            此处使用了 $g^{-1}(y - a) \in B$ 这一事实。
    \end{enumerate}
    无论上面哪种情况,我们都得到 $(h \circ H)(y) = y$,并且这两种情况不相交且覆盖了所有可能。\\

    最后,证明 $H \circ h = \id_{A \cup B}$。给定 $x \in A \cup B$,则分两种情况讨论:
    \begin{enumerate}[label=(\arabic*)]
        \item 若 $x \in A$,则
            \[(H \circ h)(x) = H(h(x)) = H\big(f(x)\big) = f^{-1}\big(f(x)\big) = \id_A(x) = x\]
           此处使用了 $f(x) \in [a]$ 这一事实。
        \item 若 $x \in B$。则
        \begin{align*}
            (H \circ h)(x) = H(h(x)) &= H\big(g(x)+a\big) = g^{-1}\Big(\big(g(x)+a\big)-a\Big)\\
            &= g^{-1}\big(g(x)\big) = \id_B(x) = x
        \end{align*}
            此处使用了 $g(x) \in [b]$ 这一事实。所以 $a + 1 \le g(x) + a \le a + b$。
    \end{enumerate}
    无论上面哪种情况,我们都得到 $(H \circ h)(x) = x$,并且这两种情况不相交且覆盖了所有可能。\\

    综上,$H = h^{-1}$,也就是说 $h$ 具有反函数。这证明了 $h$ 为双射。

    因此,$|A \cup B| = |[a + b]| = a + b = |A| + |B|$。
\end{proof}

\begin{corollary}
    若 $S,T$ 为有限集且 $S \subseteq T$。则 $|T-S| = |T|-|S|$。
\end{corollary}

\begin{proof}
    定义 $U = T - S$。此时 $U \cap S = \varnothing$。应用上述定理可得
    \[|U| + |S| = |U \cap S| = |T|\]
    两边同时减去 $|S|$ 得 $ |T - S| = |U| = |T| - |S|$。
\end{proof}

你可以利用上述两个结论证明以下推广命题:

\begin{proposition}\label{prop:proposition7.6.9}
    若 $A,B$ 为有限集,则 $|A \cup B| = |A| + |B| - |A \cap B|$。
\end{proposition}

\begin{proof}
    留作 \ref{sec:section7.6.5} 节练习 \ref{exc:exercises7.6.1}。
\end{proof}

上述定理还有另一个推论:

\begin{corollary}\label{corollary7.6.10}
    若 $A_1,B_2,\dots,A_n$ 为有限集且互不相交(这意味着任意两个集合没有交集)。

    则 $|A_1 \cup \dots \cup A_n| = |A_1| + \dots + |A_n|$。
\end{corollary}

\begin{proof}
    留作 \ref{sec:section7.6.5} 节练习 \ref{exc:exercises7.6.2}。
\end{proof}

建议你参参阅本章练习 \ref{exc:exercises7.8.32},其中通过一个证明(使用双变量归纳法)指导如何计算两个有限集的\emph{笛卡尔积}的大小。
