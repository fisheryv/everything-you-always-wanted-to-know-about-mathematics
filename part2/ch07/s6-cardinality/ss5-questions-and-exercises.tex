% !TeX root = ../../../book.tex

\subsection{习题}\label{sec:section7.6.5}

\subsubsection*{温故知新}

以口头或书面的形式简要回答以下问题。这些问题全都基于你刚刚阅读的内容,所以如果忘记了具体的定义、概念或示例,可以回去重读相关部分。确保在继续学习之前能够自信地回答这些问题,这将有助于你的理解和记忆!

\begin{enumerate}[label=(\arabic*)]
    \item 一个集合在什么情况下是\textbf{有限的}?
    \item 有哪两种方法可以表示一个集合是\textbf{无限的}?
    \item \textbf{可数}无限和\textbf{不可数}无限有什么区别?请分别给出两种类型的两个例子。
    \item 给定两个可数无限集 $A$ 和 $B$,我们可以对它们进行哪些集合操作,以\emph{保证}结果仍是一个可数无限集?是否有任何集合操作会导致结果是\emph{有限集}?
    \item $\mathbb{R} \times \mathbb{N}$ 是可数无限还是不可数无限?$\mathbb{R} - \mathbb{N}$ 呢?
\end{enumerate}

\subsubsection*{小试牛刀}

尝试回答以下问题。这些题目要求你实际动笔写下答案,或(对朋友/同学)口头陈述答案。目的是帮助你练习使用新的概念、定义和符号。题目都比较简单,确保能够解决这些问题将对你大有帮助!

\begin{enumerate}[label=(\arabic*)]
    \item 证明命题 \ref{prop:proposition7.6.9}。也就是说证明:如果 $A$ 和 $B$ 为有限集,则
          \[A \cup B| = |A| + |B| - |A \cap B|\] \label{exc:exercises7.6.1}
    \item 证明推论 \ref{corollary7.6.10}。也就是说证明:如果 $A_1, A_2, \dots, A_n$ 为可数集且互不相交,则
          \[|A_1 \cup \dots \cup A_n| = |A_1| + \dots + |A_n|\]\label{exc:exercises7.6.2}
    \item 以下``错误证明''证明了 $\mathbb{R}$ 为可数无限集,请找出其中的错误:
          \begin{quote}
              \begin{spoof}
                  设集合 $S \subset \mathbb{R}$ 为 $S = \{y \in \mathbb{R} \mid 0 \le y < 1\}$。

                  对于每一个 $x \in S$,定义集合 $A_x = \{x + z \mid z \in \mathbb{Z}\}$。

                  (例如 $A_\frac{1}{2} = \{\dots, -\frac{3}{2}, -\frac{1}{2}, \frac{1}{2}, \frac{3}{2}, \dots\}$)

                  因为 $\mathbb{Z}$ 是可数无限集,每个集合 $A_x$ 也都是可数无限集。所以
                  \[\mathbb{R} = \bigcup_{x \in S} A_x\]
                  这是可数无限集的并集,因此 $\mathbb{R}$ 也是可数无限集。
              \end{spoof}
          \end{quote}
          请确保指出每个错误步骤,并解释为什么它是错误的。理想情况下,你应该解释该错误结论为什么是错误的,而不是直接说``$\mathbb{R}$ 是不可数集,因为我们已经证明了这一点''。为什么这个错误步骤是对结论的误用?以及为什么某个步骤的结论是无效的。
    \item 对于一下每种情况,提供一个例子或说明其是不可能的。\\
          例如,如果情况为``有限集 $A$ 和 $B$ 满足 $A \cup B$ 的大小为 $4$'',可能的答案是``考虑 $A=\{1,2\}$ 和 $B=\{3,4\}$''。如果情况为``对于每个 $x \in \mathbb{N}$,无限集 $S_x$ 满足 $\displaystyle\bigcup_{x \in \mathbb{N}} S_x$ 为有限集'',答案将是``不可能''。\\
          无需\emph{证明}你的答案;给出恰当的例子即可。
          \begin{enumerate}[label=(\alph*)]
              \item 不可数无限集 $A$ 和可数无限集 $B$ 满足 $A \cap B$ 为有限集。
              \item 不可数无限集 $C$ 和 $D$ 满足 $C-D$ 为可数无限集。
              \item 不可数无限集 $E$ 和 $F$ 满足 $E-F$ 为不可数无限集。
              \item 对于每个 $x \in \mathbb{N}$,可数无限集 $S_x$ 满足 $\displaystyle\bigcup_{x \in \mathbb{N}} S_x$ 为不可数无限集。
              \item 对于每个 $y \in \mathbb{R}$,可数无限集 $T_y$ 满足 $\displaystyle\bigcup_{y \in \mathbb{R}} T_y$ 为可数无限集。
          \end{enumerate}
    \item 证明引理 \ref{lemma7.6.29}。也就是说,假设 $A$ 为不可数无限集且 $B \subset A$ 为可数无限集;证明 $A-B$ 为不可数无限集。\label
          {exc:exercises7.6.5}
\end{enumerate}