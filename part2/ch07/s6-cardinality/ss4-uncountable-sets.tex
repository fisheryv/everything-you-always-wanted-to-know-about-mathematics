% !TeX root = ../../../book.tex

\subsection{不可数集}

在讨论不可数集之前,我们先证明一个先前提及的结果:集合的可数无限次\emph{笛卡尔积}是不可数的。值得注意的是,这些集合甚至不必是无限集——它们可以都是大小为 $2$ 的有限集。在下一节中,我们将利用这一结论展示一些不可数集的例子,包括一个我们熟知的集合……

\subsubsection*{不可数笛卡尔积}

\begin{theorem}
    双元素集合的可数无限次笛卡尔积是不可数的。即
    \[\{0, 1\}^{\mathbb{N}} = \{0, 1\} \times \{0, 1\} \times \{0, 1\} \times \dots\]
    是不可数无限集。
\end{theorem}

\begin{proof}
    为了引出矛盾而假设 $\{0, 1\}^{\mathbb{N}}$ 是可数无限集。这意味着存在该集合与 $\mathbb{N}$ 之间的双射,从而可以将集合中的每个元素与一个自然数一一对应。因此,该集合中的元素可以枚举为:第一个元素对应自然数 $1$,第二个对应 $2$,依此类推。

    尽管我们不知道这些元素的具体形式,但可以确认这样的对应关系一定存在。我们可以将 $\{0, 1\}^{\mathbb{N}}$ 的所有元素记为 $y_i$,其中每个 $y_i$ 是一个由 $0$ 和 $1$ 组成的无限序列,因此可以表示为:
    \begin{align*}
        1 & \leftrightarrow (a_{1,1} , a_{1,2} , a_{1,3} , a_{1,4} , a_{1,5} , \dots) = y_1 \\
        2 & \leftrightarrow (a_{2,1} , a_{2,2} , a_{2,3} , a_{2,4} , a_{2,5} , \dots) = y_2 \\
        3 & \leftrightarrow (a_{3,1} , a_{3,2} , a_{3,3} , a_{3,4} , a_{3,5} , \dots) = y_3 \\
        4 & \leftrightarrow (a_{4,1} , a_{4,2} , a_{4,3} , a_{4,4} , a_{4,5} , \dots) = y_4 \\
        5 & \leftrightarrow (a_{5,1} , a_{5,2} , a_{5,3} , a_{5,4} , a_{5,5} , \dots) = y_5 \\
          & \vdots
    \end{align*}
    每个值 $a_{i,j}$ 要么是 $0$,要么是 $1$。这里,$i$ 表示对应的自然数(即\emph{垂直}位置),$j$ 表示序列中的坐标(即\emph{水平}位置)。

    由于假设该对应关系是双射,所以上述列表包含了 $\{0, 1\}^{\mathbb{N}}$ 的所有元素。为了完成反证法,我们将构造一个 $\{0, 1\}^{\mathbb{N}}$ 中的元素,并证明它\textbf{不在}列表中。(这正是康托尔对角线论证的一种形式。)

    我们定义对象 $x = (x_1, x_2, x_3, \dots)$ 为
    \[x_i = \begin{cases}
            0 & \text{若}\; a_{i,i} = 1 \\
            1 & \text{若}\; a_{i,i} = 0
        \end{cases}\]
    也就是说,我们沿矩阵的\emph{主对角线}(考虑所有 $a_{i,i}$)构建 $x$,并将每个值\emph{取反},即 $1$ 变为 $0$,$0$ 变为 $1$。

    下图通过一个\emph{具体示例}说明这一过程(虽然它不是证明的必要部分,但有助于理解):
    \begin{align*}
        1 & \leftrightarrow (\circled{1} , 1 , 0 , 0 , 1 , \dots) = y_1                        \\
        2 & \leftrightarrow (1 , \circled{0} , 0 , 0 , 1 , \dots) = y_2                        \\
        3 & \leftrightarrow (0 , 0 , \circled{1} , 1 , 0 , \dots) = y_3                        \\
        4 & \leftrightarrow (1 , 1 , 0 , \circled{1} , 1 , \dots) = y_4                        \\
        5 & \leftrightarrow (0 , 1 , 1 , 1 , \circled{0} , \dots) = y_5                        \\
          & \vdots                                                                             \\
        x & =\big(\circled{0}, \circled{1}, \circled{0}, \circled{0}, \circled{1}, \dots \big)
    \end{align*}
    我们这样做的目的是什么?考虑 $x$ 是否可能出现在上述列表中。
    \begin{itemize}
        \item $x = y_1$ 吗?不相等!因为 $x$ 与 $y_1$ 的第一个坐标不同。(在本例中,$x_1=0$ 而 $y_{1,1} = 1$。)
        \item $x = y_2$ 吗?不相等!因为 $x$ 与 $y_2$ 的第二个坐标不同。(在本例中,$x_2=1$ 而 $y_{2,2} = 0$。)
        \item $x = y_3$ 吗?不相等!因为 $x$ 与 $y_3$ 的第三个坐标不同。(在本例中,$x_3=0$ 而 $y_{3,3} = 1$。)
    \end{itemize}
    一般来讲,对于任意 $i \in \mathbb{N}$,我们可以保证 $x$ 与 $y_i$ 在第 $i$ 个坐标上不同。因此,\textbf{没有}任何 $y_i$ 等于 $x$,即
    \[\big(\forall i \in \mathbb{N} \centerdot x_i \ne y_{i,i}\big) \implies \big(\forall i \in \mathbb{N} \centerdot x \ne yi\big)\]
    然而,根据定义,$x$ 是一个由 $0$ 和 $1$ 组成的无限序列,故 $x \in \{0, 1\}^{\mathbb{N}}$。

    这就产生了矛盾:我们假设可以列出集合中的所有元素,但随后我们用这种顺序构造了一个不在集合中的元素。$\hashx$

    因此,$\{0, 1\}^{\mathbb{N}}$ 为不可数无限集。
\end{proof}

请注意:这是一个非常巧妙的论证,也是我最喜欢的数学证明之一。康托尔真是个天才,他想出了这个证明;更有趣的是,这个证明实际上相当简单且令人难忘。我们相信你不会忘记这个``沿主对角线切换值''的论证。我们甚至可以用八个字总结整个证明,这进一步凸显了它的精彩。

请注意:这是一个非常巧妙的论证。这是我最喜欢的数学证明之一。康托尔真是个天才,他想出了这个证明,更有趣的是,这个证明实际上相当简单且令人难忘。我们相信你不会忘记这个``沿主对角线切换值''的论证。我们甚至可以用八个字总结整个证明,这进一步彰显了它的精彩。

\begin{corollary}
    对于任意至少包含两个元素的集合,其可数无限次笛卡尔积是不可数无限的。
\end{corollary}
(注意:实际上,我们只需乘积中的每个集合非空,且最多有有限个集合只有一个元素。)

\subsubsection*{示例}

你可能会好奇:什么样的集合是不可数无限的呢?我们知道这样的集合吗?当然!以下是一些例子。\\

\begin{example}[所有无限长二进制字符串的集合]

    你可能已经注意到,我们在上面证明中使用的集合——即 $\{0, 1\}^\mathbb{N}$——实际上就是无限长二进制字符串的集合 $S$!$\{0, 1\}^\mathbb{N}$ 中的元素是无限长有序序列,每个位置上的值都是 \verb|0| 或 \verb|1|。$S$ 中的元素也是由 \verb|0| 和 \verb|1| 组成的无限长有序序列,只是不包含括号和逗号。因此,这两者之间存在一个非常自然的双射关系(只需去掉括号和逗号,或者加上它们),所以我们将这两个集合视为相同集合。

    我们在上面的示例 \ref{ex:example7.6.25} 中看到,所有有限长二进制字符串的集合是可数无限集。如上所述,所有无限长二进制字符串的集合是不可数无限集。另一种证明这一事实的方法是找到 $S$ 与 $\mathcal{P}(\mathbb{N})$ 之间的双射,然后应用康托尔定理,该定理指出 $|\mathbb{N}| < |\mathcal{P}(\mathbb{N})|$。(详见练习 \ref{exc:exercises7.8.33}。)
\end{example}

\begin{example}[$\mathbb{R}$ 是不可数无限集]

    这是我们首次接触一个标准的不可数无限集。我们可以利用上述结论来证明这一点。

    直觉上,这个说法是合理的,因为实数轴看起来比自然数集 $\mathbb{N}$ 或整数集 $\mathbb{Z}$ 大得多。但我们也知道,有理数集 $\mathbb{Q}$ 是可数无限集,并且在实数轴上存在无数个有理数。实际上,\emph{在任意两个实数之间}都存在无限多个有理数。
    
    我们将看到,实数集 $\mathbb{R}$ 确实是不可数无限集。此外,我们还将证明 $\mathbb{R}$ 与幂集 $\mathcal{P}(\mathbb{N})$ 具有相同的``无限大小'';也就是说,我们将证明 $|\mathbb{R}| = |\mathcal{P}(\mathbb{N})|$。(请记住,这比仅仅说两个集合都是不可数集包含更多信息;不可数无限集有许多级别,我们暂时不深入讨论,以免头脑爆炸。)

    直观上,要理解 $\mathbb{R}$ 是不可数无限集,可以先将 $\mathbb{R}$ 与集合 $\{0, 1, 2, 3, 4, 5, 6, 7, 8, 9\}^\mathbb{N}$ 关联起来。每个实数都可以用十进制表示,这实际上是一个有序的、可数无限的数字序列。尽管在十进制表示中会出现像 $0.999999 \dots = 1$ 这样的问题,但这些并不影响主要结论。既然我们已经知道,即使像 $\{0, 1\}$ 这样的小集合,在无限次取积时也会生成一个不可数集合,那么对于更大的集合,如 $\{0, 1, \dots, 9\}$,也必定会生成一个不可数集合。这个直观的论证可以帮助你理解并向朋友解释这一结论。(事实上,大多数教科书也是这样解释的。)

    更正式地,我们可以证明 $|\mathbb{R}| = |\mathcal{P}(\mathbb{N})|$。这个更强的结论意味着 $\mathbb{R}$ 是不可数无限集(因为康托尔定理指出 $|\mathbb{N}| < |\mathcal{P}(\mathbb{N})|$)。为此,我们考虑集合
    \[I = \{y \in \mathbb{R} \mid 0 \le y \le 1\}\]
    即区间 $[0, 1] \subseteq \mathbb{R}$。我们将证明
    \[|{0, 1}^\mathbb{N}| = |\mathcal{P}(\mathbb{N})| = |I|\]
    然后再应用一些关于区间与 $\mathbb{R}$ 之间存在双射的结论。

    考虑函数 $f_1 : \{0, 1\}^{\mathbb{N}} \to I$,它接受一个无限长二进制字符串,在字符串前加上一个小数点,并将其解释为\textbf{十进制}表示。

    例如,考虑元素 $(1, 1, 0, 0, 1, 0, \dots)$,其余部分为 $0$。则
    \[f_1(1, 1, 0, 0, 1, 0, \dots) = 0.110010 \dots _\text{DEC} = \frac{1}{10^1}+\frac{1}{10^2}+\frac{1}{10^5} = \frac{11001}{100000}\]
    请注意,这确实是一个函数,因为任何输出都是一个实数(因为它有小数表示;我们刚刚给出了一个例子),并且其值介于 $0$ 和 $1$ 之间,因为我们在最前面加了小数点。此外,函数 $f_1$ 是\textbf{单射};两个不同的无限长二进制字符串在某些位置上必然不同,因此它们表示的小数在某些位上也不同,从而不可能是相同的实数。这表明 $|\{0,1\}^{\mathbb{N}}| \le |I|$。

    考虑函数 $f_2 : \{0, 1\}^{\mathbb{N}} \to I$,它接受一个无限长二进制字符串,在字符串前加上一个小数点,并将其解释为\textbf{二进制}表示。

    例如,考虑上述相同元素。则
    \[f_2(1, 1, 0, 0, 1, 0, \dots) = 0.110010 \dots _\text{BIN} = \frac{1}{2^1}+\frac{1}{2^2}+\frac{1}{2^5} = \frac{25}{32}\]
    请注意,这确实是一个函数,因为任何输出都是一个实数;只需计算分数求和,就会得到一个介于 $0$ 和 $1$ 之间的实数(即使序列是无限的,它也保证收敛)。例如,当输入全为 $0$ 时,输出为 $0$;当输入全为 $1$ 时,输出为 $1$,因为
    \[\frac{1}{2}+\frac{1}{4}+\frac{1}{8}+\dots = \sum_{k \in \mathbb{N}} \frac{1}{2^k} = 1\]
    此外,函数 $f_2$ 是\textbf{满射}。这一事实依赖于一些关于有理数和无理数的外部知识;具体来说,任何无理数都可以通过一系列二进制有理数(即分母为 $2$ 的幂的有理数)来逼近。我们不会详细阐述或证明这些结论,但我们认为通过一些例子,你会逐渐理解其原理。事实上,搜索无理数的二进制展开,你会发现一些有趣的结果。

    因为 $f_2$ 是满射,这表明 $|\{0,1\}^{\mathbb{N}}| \ge |I|$。综上,我们得出结论 $|\{0,1\}^{\mathbb{N}}| = |I|$。我们还知道 $|\mathcal{P}(\mathbb{N})| = |\{0, 1\}^{\mathbb{N}}|$(见练习 \ref{exc:exercises7.8.33}),所以我们知道 $|I| = |\mathcal{P}(\mathbb{N})|$。

    最后一步是证明 $|I| = |\mathbb{R}|$。回看 \ref{sec:section7.5.4} 节中的练习 \ref{exc:exercises7.5.5}。在那里我们找到了集合 $J = \{y \in \mathbb{R} \mid -1 < y < 1\}$ 和 $\mathbb{R}$ 之间的双射。很容易找到集合 $J$ 与集合 $K = \{y \in \mathbb{R} \mid 0 < y < 1\}$ 之间的双射(不妨现在试一试!)。这表明 $|\mathbb{R}| = |J| = |K|$。此外,$K \subseteq I$,它们之间唯一的区别在于元素 $0$ 和 $1$,因此 $|K| = |I|$。最终,我们证明了 $|I| = |\mathbb{R}|$。因此,我们可以得出如下结论:
    \[|\mathbb{R}| = |\mathcal{P}(\mathbb{N})|\]
    我们回顾一下之前提到的两个论证:
    \begin{itemize}
        \item 考虑集合 $\{0, 1, 2, 3, 4, 5, 6, 7, 8, 9\}^\mathbb{N}$
        \item 考虑集合 $\{0, 1\}^\mathbb{N}$
    \end{itemize}
    这两个论证都涉及一些关于十进制展开和二进制展开的知识。似乎没有简单的方法绕过这一点,因此我们希望上述结果仍然是令人信服的。特别是,你可以思考上面讨论中 $f_2$ 是\textbf{满射}但不是\textbf{单射}的结论。你能说服自己接受这个观点吗?你能说服他人吗?
\end{example}

\subsubsection*{定理}

我们首先介绍一个关于不可数集的结论,随后陈述一个关于无限集的最终定理。。

\begin{lemma}\label{lemma7.6.29}
    假设 $A$ 为不可数无限集,$B$ 为可数无限集,且 $B \subseteq A$。则 $A-B$ 为不可数无限集。
\end{lemma}

(注意:我们\emph{不需要}假设 $B \subseteq A$。如果 $B \not\subseteq A$,可以考虑集合 $A$ 和 $B \cap A$。)

\begin{proof}
    留作 \ref{sec:section7.6.5} 节的练习 \ref{exc:exercises7.6.5}。

    (\textbf{提示}:采用反证法……)
\end{proof}

\subsubsection*{识别无限集}

为了定义\textbf{无限集},我们首先定义了\textbf{有限集},然后称任何\emph{不是}有限集的集合为无限集。下面的定理提供了另一种定义\textbf{无限集}的方式:一个集合是无限集,当且仅当存在一个双射将其映射到自身的一个真子集。首先,我们陈述并证明一个有用的引理,该引理将在后续定理的证明中发挥作用。

\begin{lemma}\label{lemma7.6.30}
    设 $A$ 为任意集合。则 $A$ 为无限集 $\iff$ 存在 $B \subset A$ 为可数无限集。
\end{lemma}

\begin{proof}

    $\impliedby$ 方向是显然的。如果 $A$ 包含一个可数无限子集,那么 $A$ 本身必然是无限集。

    $\implies$ 方向更为有趣。假设 $A$ 是无限集。选取一个特定元素 $\bigstar \in A$。我们通过排除 $\bigstar$ 来构造一个可数无限集 $B$,使得 $B \subset A$ 且 $B \ne A$。

    考虑集合 $A_1 = A - \{\bigstar\}$。由于 $A_1$ 仍是无限集,我们可以选择某个元素 $b_1 \in A_1$。

    考虑集合 $A_2 = A_1 - \{b_1\} = A - \{\bigstar, b_1\}$。由于 $A_2$ 仍是无限集,我们可以选择某个元素 $b_2 \in A_2$。

    考虑集合 $A_3 = A_2 - \{b_2\} = A - \{\bigstar, b_1, b_2\}$。由于 $A_3$ 仍是无限集,我们可以选择某个元素 $b_3 \in A_3$。

    这一过程可以无限进行下去。定义 $B = \{b_1, b_2, b_3, \dots\}$。(注意:尽管这里似乎在``取极限'',但这是合理的,因为我们并非通过此方法``保留'' $B$ 的特定性质,而是仅仅\emph{构造}出集合 $B$。)

    需要注意的是,$B$ 是可数无限的,因为存在一个明显的双射将其映射到 $\mathbb{N}$。
\end{proof}

有了这个引理,我们就可以陈述并证明接下来的结论了:

\begin{theorem}
    设 $A$ 为任意集合。则 $A$ 为无限集 $\iff$ 存在 $B \subset A$ 满足函数 $f : A \to B$ 为双射。
\end{theorem}

\begin{proof}
    ($\implies$) 假设 $A$ 为无限集。我们需要找到真子集 $B \subset A$ 和双射 $f : A \to B$。

    由于 $A \ne \varnothing$,取任意元素 $x \in A$。考虑 $B = A - \{x\}$,显然 $B \subset A$。

    接下来证明存在双射 $f : A \to B$。

    根据引理 \ref{lemma7.6.30},我们可以找到一个可数无限的真子集 $C \subset B$。(注意:集合 $A$ 是无限集,因此 $B = A - \{x\}$ 也是无限集,因为移除一个元素不会改变无限性。若对此有疑问,可假设 $B$ 是有限集,则 $B$ 具有某个大小 $m$,那么 $A$ 的大小为 $m+1$,这与 $A$ 无限矛盾。)

    因此 $C$ 是可数无限集,我们可以将其元素列举为 $\{y_1, y_2, y_3, \dots\}$。

    (注意:这里的思路是存在双射 $g : \mathbb{N} \to C$,因此可以定义 $y_1 = g(1), y_2 = g(2)$,依此类推。)

    定义函数 $f : A \to B$ 如下:
    \[\forall y \in A \centerdot f(y) = \begin{cases}
            y       & \text{若对于所有}\; i \in \mathbb{N} \;\text{且}\; y \ne x, y \ne yi \\
            y_1     & \text{若}\; y = x                                               \\
            y_{i+1} & \text{若对于某个}\; i \in \mathbb{N}, y = y_i
        \end{cases}\]

    这是一个双射,因为我们可以构造其反函数 $F : B \to A$:
    \[\forall z \in B \centerdot F(z) = \begin{cases}
            z & \text{若对于所有}\; i \in \mathbb{N}, z \ne y_i \\
            x & \text{若}\; z = y_1                \\
            y_{i-1} &\text{若对于某个}\; i \in \mathbb{N}-\{1\}, z = y_i
        \end{cases}\]
        
    我们留给读者验证 $F = f^{-1}$。(至少画出图像,从直觉上说服自己。)\\

    ($\impliedby$) 这个方向声称,无限集是\emph{唯一}具有这种性质的集合。我们将通过反证法来证明这一点。也就是说,我们将证明任何有限集都\emph{不能}与其真子集建立双射关系。

    假设 $A$ 为有限集,其大小为 $n \in \mathbb{N}$。考虑 $A$ 的任意真子集 $B \subset A$,设 $B$ 的大小为 $m$,则 $m < n$。为了引出矛盾而假设存在双射 $f : A \to B$。

    由于 $B$ 是有限集,存在双射 $g : B \to [m]$。将 $f$ 与 $g$ 复合,得到双射 $h : A \to [m]$。这意味着 $|A| = m$,但已知 $|A| = n$,故 $m = n$,与 $m < n$ 矛盾。$\hashx$

    (注意:我们也可以通过\emph{抽屉原理}来论证这一点。虽然我们尚未介绍抽屉原理,不过很快就会学到。简单来说,当 $n > m$ 时,我们无法得到双射 $p:[n] \to [m]$,因为没有足够的``抽屉''可以容纳 $n$ 个``东西''。)
\end{proof}

在解决问题时,若需要证明某个集合是无限集,与其证明无法找到与任何有限集之间存在双射,不如考虑使用本定理!只要找到一个真子集和一个双射,即可证明该集合是无限集。

\clearpage
