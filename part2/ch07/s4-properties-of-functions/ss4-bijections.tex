% !TeX root = ../../../book.tex

\subsection{双射}

你可能已经猜到了我们在这里要讨论的内容。回想一下我们刚刚研究的两个主要函数性质:\emph{满射性 (Surjectivity)}和\emph{单射性 (Injectivity)}。如果一个函数同时具备这两个性质会怎样呢?也就是说,对于值域中的每个元素,定义域中\emph{至少}有一个对应元素(满射性),并且\emph{最多}有一个这样的元素(单射性)。没错,这意味着每一个输出都有且只有一个对应的输入!这种性质非常重要,它将成为我们接下来讨论\emph{基数}(即集合大小)的基础。让我们先给出定义,然后再通过一些例子来说明。

\subsubsection*{定义}

\begin{definition}
    设 $A, B$ 为集合,$f: A \to B$ 为函数。我们说 $f$ 是\dotuline{双射}函数,当且仅当 $f$ 既是单射的又是满射的。

    同样地,我们可以说 ``$f$ 是双射的''(形容词形式),或者说 ``$f$ 是一个双射''(名词形式)。

    有时我们会说 $f$ 是集合 $A$ 和 $B$ 之间的双射,而不是说``从 $A$ 到 $B$'' 的双射。(为什么这样说将在下一节解释!)
\end{definition}

请注意,从逻辑上将,这个定义是一个\verb|逻辑与|陈述。当前,我们唯一能证明一个函数是双射的方法就是分别证明它是满射和单射。同样地,要证明一个函数不是双射,我们需要证明它不是满射或不是单射。(可能这两个性质都不成立,但只需证明其中一个不成立就足以证明函数不是双射。)我们不再重复这些技术(这些技术在本节之前已经很好地总结了),而是直接指出到目前为止我们看到的一些例子是否是双射。\\

\begin{example}
    \begin{enumerate}[label=(\alph*)]
        \item 设函数 $p:\mathbb{N} \times \mathbb{N} \to \mathbb{N}$ 定义为 $p(a, b) = ab$ \\
            我们已经证明 $p$ 是满射但不是单射,所以它\textbf{不是}双射。
        \item 设函数 $d:\mathbb{N} \times \mathbb{N} \to \mathbb{Z}$ 定义为 $d(a, b) = a-b$ \\
            我们已经证明 $d$ 是满射但不是单射,所以它\textbf{不是}双射。
        \item 设函数 $g : \mathbb{R} - \{-1\} \to \mathbb{R}$ 定义为 
             \[\forall x \in \mathbb{R} - \{-1\} \centerdot g(x) = \frac{x}{1+x}\]
             我们已经证明 $g$ 不是满射。(具体来说,我们证明了 $ 1 \notin Im_g(\mathbb{R} - \{-1\})$。)本节的联系中我们会要求你证明 $g$ 是单射。不管怎样,$g$ 不是双射。

             然而,考虑用相同``规则''定义的函数 $h : \mathbb{R} - \{-1\} \to \mathbb{R}- \{1\}$,即
             \[\forall x \in \mathbb{R} - \{-1\} \centerdot h(x) = \frac{x}{1+x}\]
             在 \ref{sec:section7.3.5} 节的联系中,我们会要求你证明函数 $h$ 满足 $Im_h(\mathbb{R} - \{-1\}) = \mathbb{R}- \{1\}$。这证明了 $h$ 是满射。

             此外,在本节的练习中我们还会要求你证明,通过上面这种方式定义的函数 --- 取一个单射函数,应用相同的``规则'',将值域重新定义为像集 --- 最终能够生成一个双射函数。

             综上,上面的内容证明了 $h$ 是 $\mathbb{R} - \{-1\}$ 和 $\mathbb{R}- \{1\}$ 之间的\textbf{双射}。
    \end{enumerate}
\end{example}

\begin{example}
    让我们来看一个新例子。之所以选择这个例子是为了预览一些即将出现的主要思想。定义 $E \subseteq \mathbb{N}$ 为所有\emph{偶数}的集合;也就是说:
    \[E = \{e \in \mathbb{N} \mid \exists k \in \mathbb{N} \centerdot e = 2k\}\]
    定义函数 $d : \mathbb{N} \to E$ 为 $d(n) = 2n$。我们声称 $d$ 是双射。

    \begin{proof}
        首先,我们来证明 $d$ 是满射。
        
        给定 $e \in E$。根据 $E$ 的定义,$\exists k \in \mathbb{N}$ 使得 $e = 2k$。给定这样的 $k$。

        这告诉我们 $d(k) = 2k = e$。因为 $e$ 是任意的。我们得出结论 $d$ 是满射。\\

        接着,我们来证明 $d$ 是单射。

        设 $m,n \in \mathbb{N}$ 并假设 $d(m) = d(n)$。

        这意味着 $2m=2n$。等式两边同时除以 $2$ 得 $m=n$。因此 $d$ 是单射。

        综上,这证明了 $d$ 是双射。
    \end{proof}
\end{example}

我们将通过提出一些问题来激发未来的思考:你是否觉得 $\mathbb{N}$ 和 $E$ 之间存在\emph{双射}关系有点奇怪,毕竟 $E$ 是 $\mathbb{N}$ 的\emph{真}子集?是否总能在一个集合和它的一个子集之间找到双射关系?我们之前是否见过类似的例子?

\subsubsection*{启下} 

双射 $f : A \to B$ 的核心思想是将 $A$ 和 $B$ 的元素\textbf{一一对应}。这源自满射和单射的定义:每个输出都\emph{有且只有}一个对应的输入。进一步思考这些性质的证明过程。当证明 $f$ 是满射时,我们表明可以通过至少一种方法从值域``返回''到定义域;当证明 $f$ 是单射时,我们表明这种方法是唯一的。实际上,这说明了如何``反转''函数 $f$ 的作用,定义一个从 $B$ 返回到 $A$ 的新函数。这正是我们所说的\emph{反函数}。为了严格定义这种``从值域返回到定义域''的概念,我们需要讨论如何适当地``组合''函数。接下来,我们将给出反函数的精确定义,并将其与双射联系起来。这些内容将在下一节详细介绍。

