% !TeX root = ../../../book.tex

\subsection{双射}

你可能已经猜到本节要讨论的内容。回顾我们刚刚研究的两个核心函数性质:\emph{满射性 (Surjectivity)} 和\emph{单射性 (Injectivity)}。若函数同时具备这两个性质会如何?这意味着值域中每个元素在定义域中\emph{至少}有一个对应元素(满射性),且\emph{至多}有一个对应元素(单射性)。换言之,每个输出都有且仅有一个输入与之对应!这一关键性质将成为讨论\emph{基数}(集合大小)的基础。让我们先给出定义,再通过示例阐明。

\subsubsection*{定义}

\begin{definition}
    设 $A, B$ 为集合,$f: A \to B$ 为函数。称 $f$ 是\dotuline{双射}函数,当且仅当 $f$ 既是单射又是满射。

    亦可表述为``$f$ 是双射的 (Bijective)''或``$f$ 为双射 (Bijection)''。

    有时我们会说``$f$ 是集合 $A$ 和 $B$ 之间的双射'',而非``从 $A$ 到 $B$'' 的双射。(下一节将解释其中原因!)
\end{definition}

从逻辑上讲,该定义是\verb|逻辑与|陈述。目前证明函数双射性的唯一方法是分别验证其满射性与单射性。反之,要证明函数非双射,只需证明其非满射或非单射(即便二者皆不成立,仅需证明其一足矣)。这里不再重复前文已经详细讨论的证明技术,我们直接考察已有案例是否为双射。\\

\begin{example}
    \begin{enumerate}[label=(\alph*)]
        \item 设函数 $p:\mathbb{N} \times \mathbb{N} \to \mathbb{N}$ 定义为 $p(a, b) = ab$ \\
            我们已经证明 $p$ 是满射但不是单射,因此它\textbf{不是}双射。
        \item 设函数 $d:\mathbb{N} \times \mathbb{N} \to \mathbb{Z}$ 定义为 $d(a, b) = a-b$ \\
            我们已经证明 $d$ 是满射但不是单射,因此它\textbf{不是}双射。
        \item 设函数 $g : \mathbb{R} - \{-1\} \to \mathbb{R}$ 定义为 
             \[\forall x \in \mathbb{R} - \{-1\} \centerdot g(x) = \frac{x}{1+x}\]
             我们已经证明 $g$ 不是满射。(具体而言,我们证明了 $ 1 \notin \im_g(\mathbb{R} - \{-1\})$。)本节的练习中,我们将要求你证明 $g$ 是单射。无论如何,$g$ 不是双射。

             然而,考虑用相同``规则''定义的函数 $h : \mathbb{R} - \{-1\} \to \mathbb{R}- \{1\}$,即
             \[\forall x \in \mathbb{R} - \{-1\} \centerdot h(x) = \frac{x}{1+x}\]
             在 \ref{sec:section7.3.5} 节的练习中,我们将要求你证明函数 $h$ 满足 $\im_h(\mathbb{R} - \{-1\}) = \mathbb{R}- \{1\}$。这证明了 $h$ 是满射。

             此外,在本节的练习中,我们还会要求你证明,通过上面这种方式定义的函数 —— 取一个单射函数,应用相同的``规则'',将值域重新定义为像集 —— 最终能够生成一个双射函数。

             综上,上述内容证明了 $h$ 是 $\mathbb{R} - \{-1\}$ 和 $\mathbb{R}- \{1\}$ 之间的\textbf{双射}。
    \end{enumerate}
\end{example}

\begin{example}
    让我们来看一个新例子。之所以选择这个例子是为了预览一些即将出现的核心思想。定义 $E \subseteq \mathbb{N}$ 为所有\emph{偶数}的集合,即:
    \[E = \{e \in \mathbb{N} \mid \exists k \in \mathbb{N} \centerdot e = 2k\}\]
    定义函数 $d : \mathbb{N} \to E$ 为 $d(n) = 2n$。我们断言 $d$ 为双射。

    \begin{proof}
        首先,证明 $d$ 是满射。
        
        给定 $e \in E$。根据 $E$ 的定义,$\exists k \in \mathbb{N}$ 使得 $e = 2k$。给定这样的 $k$。

        此时 $d(k) = 2k = e$。因为 $e$ 是任意的。故可得 $d$ 是满射。\\

        接着,证明 $d$ 是单射。

        设 $m,n \in \mathbb{N}$ 且 $d(m) = d(n)$。

        这意味着 $2m=2n$。等式两边同时除以 $2$ 得 $m=n$。因此 $d$ 是单射。\\

        综上,这证明了 $d$ 为双射。
    \end{proof}
\end{example}

我们将通过提出若干问题来激发思考:你是否觉得 $\mathbb{N}$ 和 $E$ 之间存在\emph{双射}关系有点奇怪,毕竟 $E$ 是 $\mathbb{N}$ 的\emph{真}子集?是否总能在一个集合和它的真子集之间建立双射关系?我们之前是否见过类似的例子?

\subsubsection*{启下} 

双射 $f : A \to B$ 的核心思想是将 $A$ 和 $B$ 的元素\textbf{一一对应}。这源于满射和单射的定义:每个输出都\emph{有且只有}一个输入与之对应。回顾这些性质的证明过程:当证明 $f$ 是满射时,我们表明对于值域中的每个元素,都存在至少一个定义域中的元素映射到它;当证明 $f$ 是单射时,我们表明这样的元素是唯一的。实际上,这揭示了如何``反转''函数 $f$ 的作用,从而定义一个从 $B$ 到 $A$ 的新函数。这正是我们所说的\emph{反函数}。为了严格定义这种``从值域返回到定义域''的映射,我们需要给出反函数的精确定义,并将其与双射关联起来。这些内容将在下一节详细展开。
