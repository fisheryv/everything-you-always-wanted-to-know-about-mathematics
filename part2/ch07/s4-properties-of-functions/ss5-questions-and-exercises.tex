% !TeX root = ../../../book.tex

\subsection{习题}

\subsubsection*{温故知新}

以口头或书面的形式简要回答以下问题。这些问题全都基于你刚刚阅读的内容,所以如果忘记了具体的定义、概念或示例,可以回去重读相关部分。确保在继续学习之前能够自信地回答这些问题,这将有助于你的理解和记忆!

\begin{enumerate}[label=(\arabic*)]
    \item 用\textbf{像}来定义\textbf{满射}。然后,再用量词来定义满射。
    \item 描述证明函数是\textbf{单射}的两种不同方法。
    \item 一个函数可以既是单射又是满射吗?如果可以,举一个例子。
    \item 一个函数可以既不是单射也不是满射吗?如果可以,举一个例子。
    \item 对于以下每一个示意图,判断它们是否是函数;如果是,判断它是否是单射或满射。
\end{enumerate}

\subsubsection*{小试牛刀}

尝试回答以下问题。这些题目要求你实际动笔写下答案,或(对朋友/同学)口头陈述答案。目的是帮助你练习使用新的概念、定义和符号。题目都比较简单,确保能够解决这些问题将对你大有帮助!

\begin{enumerate}[label=(\arabic*)]
    \item 假设 $f : \mathbb{R} \to \mathbb{R}$ 是\emph{单调递增}函数,即
        \[\forall x, y \in \mathbb{R} \centerdot x < y \implies f(x) < f(y)\]
        证明 $f$ 必为\emph{单射}。\\
        然后,通过定义一个递增但不是满射的函数,证明 $f$ \emph{不一定}是满射。
    \item 设函数 $g : \mathbb{R} - \{-1\} \to \mathbb{R}$ 定义为 
        \[\forall x \in \mathbb{R} - \{-1\} \centerdot g(x) = \frac{x}{1+x}\]
        $g$ 是\emph{单射}吗?请证明你的声明。
    \item 给出函数 $f : \mathcal{P}(\mathbb{N}) \to \mathbb{N}$ 为满射的一个例子,并\emph{证明}它确实是满射。\\
        (\textbf{提示}:尤其要注意 $\varnothing \in \mathcal{P}(\mathbb{N})$。你可以回看 \ref{sec:section5.5.2} 获得一些灵感……)
    \item 给出函数 $F : \mathbb{N} \to \mathcal{P}(\mathbb{N})$ 为单射的一个例子,并\emph{证明}它确实是单射。\\
        然后\emph{证明}函数 $F$ \emph{不是}满射。\\
        (注意:我们要求你\emph{在不知道函数定义的情况下},证明它不是满射。我们知道这是正确的!你将在本章后面了解到我们使用的这个小技巧……)
    \item 假设函数 $f : A \to B$ 和 $g : B \to C$ 都是满射。证明 $g \circ f:A \to C$ 也是满射。
    \item 设函数 $f : A \to B$ 为单射。定义函数 $g : A \to Im_f (A)$ 为 $\forall x \in A \centerdot g(x) = f(x)$。证明 $g$ 为双射。
    \item 定义函数 $F:\mathbb{R} \times \mathbb{R} \to \mathbb{R} \times \mathbb{R}$ 为 $F(x, y) = (x+y, 2x-y)$。证明 $F$ 为双射。\\
        (\textbf{提示}:试着在草稿上解一个二元方程组。你可以参考 \ref{sec:section1.3.2} 节中的一些建议。)
    \item 设 $A, B$ 为集合,$g: A \to B$ 为\emph{单射}。\\
        设 $X \subseteq A$。定义函数 $h : X \to B$ 为 $\forall x \in X \centerdot h(x) = g(x)$。(也就是说 $h$ 和 $g$ 定义在相同``规则''下,但 $h$ 的``定义域更小''。)\\
        证明 $h$ 也是单射。
\end{enumerate}