% !TeX root = ../../../book.tex

\subsection{单射函数}

在证明函数满射性时,我们会选取值域中的任意元素,并在定义域中寻找\emph{至少一个}元素与之对应。这种对应关系可能\emph{只有一个}元素,可能有\emph{多个},也可能\emph{不存在}。现在我们将讨论恰好``只有一个''的情形。这里不预设函数满射,而是附加一个条件:任何输出值\emph{至多对应一个}输入值。这意味着可能恰好有一个输入,也可能没有输入,但绝不会出现两个或更多输入。这类函数具有特殊性质,因此我们赋予其特定名称。

\subsubsection*{定义}

\begin{definition}
    设 $A, B$ 为集合,$f: A \to B$ 为函数。称 $f$ 为\dotuline{单射}函数,当且仅当其满足:
    \[\forall a_1, a_2 \in A \centerdot a_1 \ne a_2 \implies f(a_1) \ne f(a_2)\]

    亦可表述为``$f$ 是单射的 (Injective)''或``$f$ 为单射 (Injection)''。

    此性质等价于``不同输入必产生不同输出''。另外,根据逆否命题与原命题逻辑等价,该性质亦可表述为:
    \[\forall a_1, a_2 \in A \centerdot f(a_1) = f(a_2) \implies a_1 = a_2\]
    这意味着``若输出相同,则输入必相同''。
\end{definition}

思考此定义如何体现前述概念:设 $f : A \to B$ 是单射函数,给定 $b \in B$。根据定义,是否\emph{至多}存在一个 $x \in A$ 满足 $f(x) = b$?该定义允许哪些可能性?

\subsubsection*{引言}

我们可以通过一个具体的函数应用来理解这个概念。将函数想象成一台\emph{密码机},用于和朋友收发秘密消息。朋友写下秘密消息,放入编码器后输出加密代码发送给你。当你收到加密代码时,希望它只对应\emph{唯一}的原始消息。假如解码时同时得到\verb|我恨你|\textbf{和}\verb|我爱你|两条信息,你会多么困惑?朋友怎么可能同时发送两条矛盾的信息?若不同输入能产生相同加密输出,这个编码系统显然存在严重缺陷!理想情况下,编码函数应确保\emph{不同}输入永远无法产生\emph{相同}输出——这正是单射定义的核心要求。

\subsubsection*{单射定义的否定}

我们可以借助示意图和单射定义的\textbf{否定}来理解其属性。首先给出单射定义的否定式:
\[\neg \big(\forall a_1, a_2 \in A \centerdot a_1 \ne a_2 \implies f(a_1) \ne f(a_2)\big) \iff \big(\exists a_1, a_2 \in A \centerdot a_1 \ne a_2 \land f(a_1) = f(a_2)\big)\]

(谨记:$P \implies Q$ 的否定为 $P \land \neg Q$!)

一个函数\emph{不是}单射,当且仅当存在两个\emph{不同}的定义域元素映射到\emph{相同}的值域元素。

基于这一点,以下是单射与非单射函数的典型示例:

\begin{multicols}{2}
    \begin{center}
        \begin{tikzpicture}[scale=1]
            \foreach \x in {0,...,4}
                {
                    \node at (4, -\x*0.6)[circle,fill,inner sep=3pt]{};
                }
            \draw (4,-1.2) ellipse (1 and 2);

            \foreach \x in {0,...,3}
                {
                    \node at (0, -\x*0.8)[circle,fill,inner sep=3pt]{};
                }
            \draw (0,-1.2) ellipse (1 and 2);

            \draw[-latex] (0.2,-0.0) -- (3.8,0.0);
            \draw[-latex] (0.2,-0.8) -- (3.8,-2.4);
            \draw[-latex] (0.2,-1.6) -- (3.8,-1.2);
            \draw[-latex] (0.2,-2.4) -- (3.8,-0.6);

            \node[below] at (0, -3.2){$A$};
            \node[below] at (4, -3.2){$B$};
            \node[below] at (2, -4){单射};
            \node[above] at (2, 0.4){$f:A \to B$};
        \end{tikzpicture}
    \end{center}

    \begin{center}
        \begin{tikzpicture}[scale=1]
            \foreach \x in  {0,...,4}
                {
                    \node at (4, -\x*0.6)[circle,fill,inner sep=3pt]{};
                }
            \draw (4,-1.2) ellipse (1 and 2);

            \foreach \x in  {0,...,3}
                {
                    \node at (0, -\x*0.8)[circle,fill,inner sep=3pt]{};
                }
            \draw (0,-1.2) ellipse (1 and 2);

            \draw[-latex] (0.2,-0.0) -- (3.8,0.0);
            \draw[-latex, red] (0.2,-0.8) -- (3.8,-0.55);
            \draw[-latex] (0.2,-1.6) -- (3.8,-2.4);
            \draw[-latex, red] (0.2,-2.4) -- (3.8,-0.65);
            \node[red] at (0, -0.8)[circle,fill,inner sep=3pt]{};
            \node[red] at (0, -2.4)[circle,fill,inner sep=3pt]{};
            \node[red] at (4, -0.6)[circle,fill,inner sep=3pt]{};

            \node[below] at (0, -3.2){$A$};
            \node[below] at (4, -3.2){$B$};
            \node[below] at (2, -4){非单射};
            \node[above] at (2, 0.4){$f:A \to B$};
        \end{tikzpicture}
    \end{center}
\end{multicols}

非单射函数中,不同的定义域元素映射到相同的值域元素,而单射函数则不会出现这种情况。通过否定形式描述属性可能略显奇怪——函数只有在\emph{没有}……的情况下才是单射——但实际上这在数学中十分常见(后文讨论无限集时会看到``……\emph{不是}有限集''这种类似表述)。否定表述便于记忆,且总是可以转化为肯定表述:单射函数对于\emph{任意}给定输出只有 $0$ 或 $1$ 个对应的输入。

\subsubsection*{示例}

我们来讨论如何证明或证伪一个函数是单射。要证明函数\emph{是}单射,可采用定义中的前两种方法:取定义域中任意两个不同元素证明其输出不同,或由相同输出推导其输入相同。反证法也是有效的证明途径。第三种方法则适用于证明函数\emph{不是}单射:只需找到不同输入对应相同输出的反例即可。

下面通过示例展示这些方法的应用,其中部分案例延续自满射的讨论!

\begin{example}
    考虑函数 $p : \mathbb{N} \times \mathbb{N} \to \mathbb{N}$ 定义为 $p(a,b) = ab$。$p$ 是单射吗?

    通过具体验证可知 $p$ 不是单射。选择任意一个具有两种不同因式分解的数字,例如 $12 = 3 \cdot 4 = 2 \cdot 6$。令 $(a, b) = (3, 4)$ 和 $(c, d) = (2, 6)$,即可证明。实际上,只需注意到 $(a, b)$ 是顺序相关的(即有序对),便能更容易地证明这一点。

    \begin{proof}
        该函数不是单射。
        设 $ (a, b) = (1, 2), (c, d) = (2, 1)$。因为 $1 \ne 2$,所以 $ (a, b) \ne (c, d)$。
        然而 $p(a, b) = 1 \cdot 2 = 2$ 且 $p(c, d) = 2 \cdot 1 = 2$。因此 $p(a, b) = p(c, d)$。这证明了 $p$ 不是单射。
    \end{proof}
\end{example}

\begin{example}
    设 $C$ 为美国所有汽车的集合,$S$ 为所有由字母和数字组成且长度不超过 $7$ 的字符串的集合(这些是可能出现在汽车牌照上的\emph{潜在}字符串)。

    设 $f : C \to S$ 定义为输入一辆车,输出其牌照字符串。函数 $f$ 是单射吗?

    答案是否定的:不同州的汽车可能使用相同的牌照字符串\footnote{中国实行一车一牌制,因此不会出现这种状况,故为单射 —— 译者注}。
    
    能否修改函数定义使其成为单射?我们可以试试!令 $S$ 为美国各州集合,$L$ 为所有可能的牌照字符串的集合。定义函数 $g : C \to S \times T$,将输入的汽车映射至其牌照和所属州的有序对。此时 $g$ 为单射,因为同一个州内没有两辆车可以具有相同的车牌。(再次强调,这不是正式证明,只是用一个非数值示例来说明单射的概念。)
\end{example}

\begin{example}
    设函数 $d : \mathbb{N} \times \mathbb{N} \to \mathbb{Z}$ 定义为
    \[\forall (a, b) \in \mathbb{N} \times \mathbb{N} \centerdot d(a, b) = a - b\]

    我们来判断 $d$ 是否为单射,并证明我们的结论。

    事实上 $d$ 不是单射!注意到 $a - b = (a + 1) - (b + 1)$。我们可以据此构造一个反例:

    考虑 $(2, 1) \in \mathbb{N} \times \mathbb{N}$ 和 $(3, 2) \in \mathbb{N} \times \mathbb{N}$。易得 $d(2, 1) = 1$ 且 $d(3, 2) = 1$。由于 $(2, 1) \ne (3, 2)$,但 $d(2, 1) = d(3, 2)$,因此可得 $d$ 不为单射。
\end{example}

\begin{example}
    设函数 $F : \mathcal{P}(\mathbb{N}) \to \mathcal{P}(\mathbb{Z})$ 定义为
    \[\forall X \in \mathcal{P}(\mathbb{N}) \centerdot F(X) = \bigcup_{a \in X} \{a, -a\}\]

    你看出这个函数的作用了吗?(你能解释为什么它是一个\emph{良好定义}的函数吗?)

    通过以下示例帮助你理解:
    \begin{align*}
        F\big(\{1\}\big)       & = \bigcup_{a \in \{1\}} \{a, -a\} = \{-1,1\}                                 \\
        F\big(\{1,3,5\}\big)   & = \bigcup_{a \in \{1,3,5\}} \{a, -a\} = \{-1,1\} \cup \{-3,3\} \cup \{-5,5\} \\
                               & = \{-5,-3,-1,1,3,5\}                                                         \\
        F\big(\varnothing\big) & = \bigcup_{a \in \varnothing} \{a, -a\} = \varnothing                        \\
        F\big(\mathbb{N}\big)  & = \mathbb{Z} - \{0\}
    \end{align*}

    我们断言 $F$ 是单射。在阅读我们的证明之前,请先思考如何证明这一点。特别是,基于单射的正式定义,考虑可能采用的策略。某种策略是否比另一种更有效?

    \begin{proof}
        我们要证明 $F$ 是单射。

        设 $X,Y \in \mathcal{P}(\mathbb{N})$。假设 $X \ne Y$,我们需要证明 $F(X) \ne F(Y)$。

        因为 $X \ne Y$,有两种情况:要么 $X \nsubseteq Y$,要么 $Y \nsubseteq X$。\\

        假设 $X \nsubseteq Y$,这意味着 $\exists n \in X \centerdot n \notin Y$。给定这样的 $n$。

        因为 $n \in \{-n, n\}$ 且 $n \in X$,根据 $F$ 的定义可得 $n \in F(X)$。

        然而,因为 $n \notin Y$,可得 $\forall a \in Y \centerdot n \notin {-a, a}$。由于 $n \notin Y$ 并且 $n \in \mathbb{N}, Y \subseteq \mathbb{N}$,所以 $\forall a \in Y \centerdot n \ne -a \in \mathbb{Z}$。

        因此 $n \notin F(Y)$。这证明了 $F(X) \ne F(Y)$。\\

        同理,对于 $Y \nsubseteq X$,采用与上面完全相同的论证,只需将 $X$ 和 $Y$ 互换一下(即每一步中交换 $X$ 和 $Y$),从而证明 $F(Y) \ne F(X)$。\\

        综上,我们证明了 $\forall X, Y \in \mathcal{P}(\mathbb{N}) \centerdot X \ne Y \implies F(X) \ne F(Y)$。因此 $F$ 是一个单射。
    \end{proof}

    思考一下,如果采用不同的方法来证明此问题会怎样。如果我们从假设 $X, Y \in \mathcal{P}(\mathbb{N})$ 且 $F(X) = F(Y)$ 出发,能否推导出 $X = Y$?
\end{example}