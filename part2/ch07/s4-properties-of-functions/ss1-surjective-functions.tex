% !TeX root = ../../../book.tex

\subsection{满射函数}

你可能会问……如果我们能确定给定函数定义域的像,为什么还要用比这个像更大的值域呢?例如,函数 $f : \mathbb{R} \to \mathbb{R}$ 定义为 $f(x) = x^2$ 是良好定义的函数,但如果把值域改为非负实数集也不会有任何影响。这样做甚至更好,因为函数不会``遗漏''任何值域中的元素!如果你有这样的想法,那么你已经理解了我们接下来的定义,这个定义准确地概括了函数的这一特性:值域和定义域的像是同一个集合这种情况。

\subsubsection*{定义}

\begin{definition}
    设 $A,B$ 为集合,$f:A \to B$ 为函数。我们说 $f$ 是\dotuline{满射}函数,当且仅当 $Im_f (A) = B$。

    同样地,我们可以说 ``$f$ 是满射的''(形容词形式),或者说 ``$f$ 是一个满射''(名词形式)。

    回到像的定义,我们可以用一个量化陈述来等价地描述这个性质:
    \[f \;\text{是满射} \iff \forall b \in B \centerdot \exists a \in A \centerdot f(a) = b\]
    也就是说,$f$ 是满射当且仅当每个输出至少有一个对应的输入。
\end{definition}

想一想为什么这个定义的第二种形式实际上与第一种形式是相同的。性质 $Im_f (A) = B$ 是关于集合的陈述。根据定义,我们已经知道 $Imf (A) \subseteq B$(像中的任何元素都不能``超出''值域),所以这个进一步的性质意味着 $B \subseteq Im_f (A)$。这正是定义的第二种形式所说的:值域的每个元素都满足像元素的定义。

另外,注意定义中没有说我们找到的与 $b$ 对应的 $a$ 必须是唯一的!这个性质所要求的只是,对于每个 $b \in B$,我们可以找到\emph{至少一个} $a \in A$ 满足 $f(a) = b$。可能有多个,也可能只有一个。这无关紧要,只要不是\emph{一个都没有}就行。

在示意图中,\emph{满射}的性质意味着什么?由于值域的每个元素都被函数``命中'',这意味着示意图右侧的每个点都有一个箭头指向它。(记住:这种启发式语言是可以记住的 --- 毕竟我们用它来帮助描述这些概念 --- 但这不构成证明。在你用于证明的任何语句中,应该使用着更严格的陈述,使用数学语言和/或逻辑符号。)为什么我们会关心这样的性质?一般来说,声明一个函数的像究竟是什么可能很困难,我们可能(首先)只能声明值域是什么。实际上证明值域的所有元素都是函数的输出可以提供额外的、有用的信息!

\subsubsection*{满射定义的否定}

通常,我们会先定义一个函数,然后问:这是一个满射吗?如果我们认为一个函数是满射,我们应该通过证明值域和像是同一集合来证实。如果我们认为它不是满射,我们应该通过找到一个反例来证明它不是满射。让我们来看一下满射函数定义的逻辑否定:
\[\neg (\forall b \in B \centerdot \exists a \in A \centerdot f(a) = b) \iff \exists b \in B \centerdot \forall a \in A \centerdot f(a) \ne b\]
也就是说,要证明函数 $f$ 不是满射,我们必须找到一个不在像中的值域元素。这需要一些验算和直觉来识别这样的元素 $b$。接下来,我们必须以某种方式证明没有任何 $a$ 满足 $f(a) = b$。我们可以通过取任意 $a \in A$ 并解释为什么 $f(a) \ne b$ 来直接证明这一点。或者,我们可以通过反证法来证明:假设存在一个 $a \in A$ 使得 $f(a) = b$,然后找出矛盾。这两种方法都是合理的,并且是逻辑等价的。

\subsubsection*{示例}

让我们通过几个例子来看看这些技术是如何应用的。对于其中一些例子,我们可以借助图形直觉或尝试一些测试案例来进行\emph{猜想},但最终我们需要通过\emph{证明}某些逻辑陈述来验证我们的观点。\\

\begin{example}
    考虑函数 $p : \mathbb{N} \times \mathbb{N} \to \mathbb{N}$ 定义为 $p(a,b) = ab$。$p$ 是满射吗?

    答案是肯定的。我们可以设 $a$ 为 $1$,则函数的输出就是 $b$。让我们通过证明来让这个观察更加正式:

    \begin{proof}
        设 $n \in \mathbb{N}$ 为任意固定自然数。定义 $(a, b) = (1, n)$。

        不难发现 $(1, n) \in \mathbb{N} \times \mathbb{N}$ 且 $p(1, n) = 1 \cdot n = n$。

        因为 $n$ 是任意的,这证明了 $p$ 是满射。
    \end{proof}
\end{example}

\begin{example}
    设 $C$ 为美国所有汽车的集合。设 $S$ 为所有由字母和数字组成的长度最多为 $7$ 的字符串的集合(这些是可能出现在汽车牌照上的\emph{潜在}字符串)。

    设 $f : C \to S$ 定义为输入一辆车,输出其牌照字符串。函数 $f$ 是满射吗?

    不,绝对不是!也许你没注意到,\emph{敏感词}是不允许出现在牌照上的!所以,肯定存在许多在美国牌照上绝对不会出现的字符串。(我们就不列举了,你可以自己想一些例子……)

    因为我们展示了 $S$ 中至少有一个元素\emph{不是} $Im_f (C)$ 的元素,因此我们已经证明了 $f$ 不是一个满射。
\end{example}

\begin{example}
    设函数 $d : \mathbb{N} \times \mathbb{N} \to \mathbb{Z}$ 定义为
    \[\forall (a, b) \in \mathbb{N} \times \mathbb{N} \centerdot d(a, b) = a - b\]
    我们来判断 $d$ 是否是满射,并证明我们的结论。
    
    我们可以先尝试一些``较小的''输入值。在下表中,左列表示 $a$,顶行表示 $b$,表格中的值是 $d(a, b) = a - b$:
    \begin{center}
        \begin{tabular}{c|ccccc}
              &  1 &  2 &  3 &  4 &  5 \\
            \hline
            1 &  0 & -1 & -2 & -3 & -4 \\
            2 &  1 &  0 & -1 & -2 & -3 \\
            3 &  2 &  1 &  0 & -1 & -2 \\
            4 &  3 &  2 &  1 &  0 & -1 \\
            5 &  4 &  3 &  2 &  1 &  0 \\
        \end{tabular}
    \end{center}
    看起来所有的整数 $z \in \mathbb{Z}$ 都会出现在这个表格中。然而,它们不会全部出现在某一特定的行或列中。相反,所有非负整数似乎都出现在第一列,而所有非正整数似乎都出现在第一行。让我们根据这些观察来写一个证明。我们将取任意整数 $z \in \mathbb{Z}$,并分两种情况讨论:如果 $z \ge 0$,我们将采取一种方法;如果 $z < 0$,我们将采取另一种方法。只要我们在这两种情况下都成功,我们就证明了 $d$ 是一个满射。

    \begin{proof}
        我们声称 $d$ 是满射。

        设 $z \in \mathbb{Z}$ 是任意固定整数。我们要证明 $\exists (a, b) \in \mathbb{N} \times \mathbb{N} \centerdot d(a, b) = z$。为此,考虑如下两种情况:
        \begin{enumerate}[label=(\arabic*)]
            \item 假设 $z \ge 0$,则定义 $(a, b) = (z + 1, 1)$。\\
                因为 $z \ge 0$,我们知道 $z+1 \ge 1$,因此 $z+1 \in \mathbb{N}$。这确保了 $(z + 1, 1) \in \mathbb{N} \times \mathbb{N}$。\\
                此时 $d(z + 1, 1) = (z + 1) - 1 = z$。
            \item 假设 $z < 0$,则定义 $ (a, b) = (1, -z + 1)$。\\
                因为 $z < 0$,我们知道 $-z > 0$,因此 $-z+1 > 1$,这意味着 $-z+1 \in \mathbb{N}$。这确保了 $(1, -z+1) \in \mathbb{N} \times \mathbb{N}$。\\
                此时 $d(1, -z + 1) = 1 - (-z + 1) = z$。
        \end{enumerate}
        无论哪种情况,我们都可以定义 $(a, b) \in \mathbb{N} \times \mathbb{N} \centerdot d(a, b) = z$。因为 $z \in \mathbb{Z}$ 是任意整数,所以这证明了 $d$ 是一个满射。
    \end{proof}
\end{example}

\begin{example}
    设函数 $g : \mathbb{R} - \{-1\} \to \mathbb{R}$ 定义为 
    \[\forall x \in \mathbb{R} - \{-1\} \centerdot g(x) = \frac{x}{1+x}\]
    (注意,我们为什么从定义域中移除了 $-1$。这是为了确保 $g$ 是\emph{良好定义的}。)

    我们来判断 $g$ 是否是满射,并证明我们的结论。如前所述,我们可以通过做一些验算来确定这一点:我们可以尝试代入一些 $x$ 的值,让 $x$ 非常接近 $-1$ 或变得越来越大,以测试``极端情况''……这些都有助于我们绘制函数图像,或者我们可以使用绘图软件直接画出函数图像:

    \begin{center}
        \begin{tikzpicture}
            \begin{axis}[
                ymin=-1.5,
                ymax=3.5,
                xmin=-5, 
                xmax=5,
                minor tick num=3,
                axis lines*=middle,
                xtick align=inside,
                ytick align=inside
                ]       
                \addplot[blue, line width=1pt, domain=-5:-1.1, samples=40, smooth] {x/(1+x)};
                \addplot[blue, line width=1pt, domain=-0.9: 5, samples=60, smooth] {x/(1+x)};
                \addplot[black, dashed, domain=-5:5, samples=2] {1};
            \end{axis}
        \end{tikzpicture}
    \end{center}

    然而,无论是哪种绘图方式,都不能构成\emph{证明}!它只是帮助我们观察到函数 $g$ \emph{不是}满射。在 $y = 1$ 处似乎有一个水平渐近线。也就是说,函数 $g$ 只会无限接近 $1$,但永远无法``达到'' $1$,。根据我们对\emph{满射}的定义,$g$ 显然不是满射!

    现在尝试证明这一点。你如何证明元素 $-1 \in \mathbb{R}$ \emph{不是}像 $Im_g(\mathbb{R})$ 的元素呢?试试看吧!然后继续阅读我们的证明。

    我们将给出\emph{两个}证明供你比较和对比。它们都证明了同一个结论 --- $g$ 不是满射。其中一个采用反证法,另一个采用直接证法(通过分情况讨论)。你觉得哪个更好?你是否也想到了其中一种证法?哪个更容易理解?对于这些问题,我们没有明确的观点;这两个证明都是有效的!

    \begin{proofs}{证明 1 (直接证法)}
        设 $x \in \mathbb{R} - \{-1\}$ 为任意固定元素。我们要证明 $g(x) \ne 1$。考虑如下两种情况:
        \begin{itemize}
            \item 假设 $x > -1$。\\
                这意味着 $x+1>0$,所以 $\frac{1}{x+1}>0$。我们还知道 $x+1>x$ 对于所有 $x \in \mathbb{R}$ 都成立。不等式两边同时乘以一个正项 $\frac{1}{x+1}$ 得 $1 > \frac{x}(x+1)$,即 $g(x) = \frac{x}(x+1) \ne 1$。
            \item 假设 $x < -1$。\\
                这意味着 $x+1<0$,所以 $\frac{1}{x+1}<0$。我们还知道 $x+1>x$ 对于所有 $x \in \mathbb{R}$ 都成立。不等式两边同时乘以一个负项 $\frac{1}{x+1}$ 得 $1 < \frac{x}(x+1)$,即 $g(x) =  \frac{x}(x+1) \ne 1$。
        \end{itemize}
        无论哪种情况都有 $g(x) \ne 1$。因为上述两种情况覆盖了所有可能的情况,并且 $x \in \mathbb{R} - \{-1\}$ 是任意的,这证明了
        \[1 \notin Im_g(\mathbb{R} - \{-1\})\]
        所以 $g$ 不是满射。
    \end{proofs}

    请注意,证明 1 揭示了图像的一个有趣现象:该函数在 $x = -1$ 的左侧图像位于水平渐近线之上,而在 $x = -1$ 的右侧图像位于渐近线之下。

    \begin{proofs}{证明 2 (反证法)}
        为了引出矛盾而假设 $g$ 是满射。这意味着
        \[\forall y \in \mathbb{R} \centerdot y \in Im_g(\mathbb{R} - \{-1\})\]
        特别地,我们知道 $1 \in Im_g(\mathbb{R} - \{-1\})$,所以 $\exists x \in \mathbb{R} - \{-1\} \centerdot g(x) = 1$。给定这样的 $x$。

        这意味着 $g(x) = \frac{x}{x+1} = 1$。两边同时乘以分母得 $x = x + 1$。两边同时减去 $x$ 得 $0 = 1$,显然这是一个矛盾。$
        \hashx$

        因此,$1 \notin Im_g(\mathbb{R} - \{-1\})$,所以 $g$ 不是满射。
    \end{proofs}

    请注意,证明 2 虽然确实证明了 $g$ 不是满射,但没有提供其他关于函数行为的信息(不像证明 1 那样)。

    接下来,我们来讨论一个与函数密切相关的性质。
\end{example}