% !TeX root = ../../../book.tex

\subsection{满射函数}

你可能会问:既然能确定函数在定义域上的像,为何还要使用比该像更大的值域呢?例如函数 $f : \mathbb{R} \to \mathbb{R}$ 定义为 $f(x) = x^2$ 是良好定义的,但若将值域改为非负实数集也完全成立。这样做反而更优,因为函数不会``遗漏''值域中的任何元素!如果你也有类似的想法,说明你已经理解了接下来的定义,该定义精准刻画了值域与像集相等的情形。

\subsubsection*{定义}

\begin{definition}
    设 $A,B$ 为集合,$f:A \to B$ 为函数。称 $f$ 为\dotuline{满射}函数,当且仅当 $\im_f(A) = B$。

    亦可表述为``$f$ 是满射的 (Surjective)''或``$f$ 为满射 (Surjection)''。

    由像的定义可得等价的量化描述:
    \[f \text{\ 为满射} \iff \forall b \in B \centerdot \exists a \in A \centerdot f(a) = b\]
    即 $f$ 为满射当且仅当每个输出至少有一个对应的输入。
\end{definition}

请思考为什么该定义的两种形式本质相同。性质 $\im_f (A) = B$ 是关于集合的陈述。由定义可知 $\im_f (A) \subseteq B$(像中的任何元素都不能``超出''值域);而 $\im_f (A) = B$ 意味着 $B \subseteq \im_f (A)$,这正是定义的量化形式所表达的——值域中每个元素都属于像集。

请注意,定义仅要求对任意 $b \in B$ 存在\emph{至少一个} $a \in A$ 满足 $f(a) = b$,并未限定 $a$ 的唯一性。原像可能有多个或仅有一个,具体数量无关紧要,只要不是\emph{一个都没有}就行。

在示意图中,\emph{满射}表现为值域中的每个点均有箭头指入(这种直观描述有助于理解概念,但证明时必须采用严格的数学语言)。我们为什么会关心此性质?一般来说,精确刻画函数的像往往是件困难的事情,而通过证明值域中的每个元素都是像元素,可以提供额外关键信息。

\subsubsection*{满射定义的否定}

通常,我们会先定义一个函数,然后询问它是否是满射。如果认为它是满射,我们应通过证明其像等于值域来证实;如果认为它不是满射,我们应通过找到一个反例来证明。现在考察满射定义的逻辑否定:
\[\neg \big(\forall b \in B \centerdot \exists a \in A \centerdot f(a) = b \big) \iff \exists b \in B \centerdot \forall a \in A \centerdot f(a) \ne b\]
也就是说,要证明函数 $f$ 不是满射,我们必须找到一个不在像中的值域元素 $b$。这需要一些演算和直觉来识别这样的元素。接下来,必须证明不存在任何 $a$ 满足 $f(a) = b$。我们可以直接证明:取任意 $a \in A$ 并说明 $f(a) \ne b$;或者使用反证法:假设存在 $a \in A$ 使得 $f(a) = b$,再导出矛盾。这两种方法在逻辑上是等价的。

\subsubsection*{示例}

让我们通过一些例子来展示这些技术的应用。在某些例子中,可以借助几何直觉或测试用例来形成\emph{猜想},但最终需要通过\emph{证明}逻辑陈述来验证观点。\\

\begin{example}
    考虑函数 $p : \mathbb{N} \times \mathbb{N} \to \mathbb{N}$ 定义为 $p(a,b) = ab$。$p$ 是满射吗?

    答案是肯定的。我们可以设 $a$ 为 $1$,则函数的输出就是 $b$。为使这一观察更严谨,我们给出如下证明:

    \begin{proof}
        设 $n \in \mathbb{N}$ 为任意固定自然数。定义 $(a, b) = (1, n)$。

        不难发现 $(1, n) \in \mathbb{N} \times \mathbb{N}$ 且 $p(1, n) = 1 \cdot n = n$。

        由于 $n$ 是任意的,这证明了 $p$ 为满射。
    \end{proof}
\end{example}

\begin{example}
    设 $C$ 为美国所有汽车的集合,$S$ 为所有由字母和数字组成且长度不超过 $7$ 的字符串的集合(这些是可能出现在汽车牌照上的\emph{潜在}字符串)。

    设 $f : C \to S$ 定义为输入一辆车,输出其牌照字符串。函数 $f$ 是满射吗?

    不,绝对不是!也许你没注意到,\emph{敏感词}是不允许出现在牌照上的!因此,必然存在许多在美国牌照上绝对不会出现的字符串。(我们就不列举了,你可以自己想一些例子……)

    由于我们展示了 $S$ 中至少有一个元素\emph{不是} $\im_f (C)$ 的元素,因此我们证明了 $f$ 不是一个满射。
\end{example}

\begin{example}
    设函数 $d : \mathbb{N} \times \mathbb{N} \to \mathbb{Z}$ 定义为
    \[\forall (a, b) \in \mathbb{N} \times \mathbb{N} \centerdot d(a, b) = a - b\]
    我们来判断 $d$ 是否是满射,并证明我们的结论。

    我们可以先尝试一些``较小的''输入值。在下表中,左列表示 $a$,顶行表示 $b$,表格中的值为 $d(a, b) = a - b$:
    \begin{center}
        \begin{tabular}{c|ccccc}
              & 1 & 2  & 3  & 4  & 5  \\
            \hline
            1 & 0 & -1 & -2 & -3 & -4 \\
            2 & 1 & 0  & -1 & -2 & -3 \\
            3 & 2 & 1  & 0  & -1 & -2 \\
            4 & 3 & 2  & 1  & 0  & -1 \\
            5 & 4 & 3  & 2  & 1  & 0  \\
        \end{tabular}
    \end{center}
    似乎所有整数 $z \in \mathbb{Z}$ 都会出现在此表格中。但是,它们不会全都出现在特定行或列中。具体而言,所有非负整数均出现在第一列,而所有非正整数均出现在第一行。现在基于此观察构造证明,取任意整数 $z \in \mathbb{Z}$,分两种情况讨论:若 $z \ge 0$,采用一种方法;若 $z < 0$,采用另一种方法。若两种情况均成立,则 $d$ 是满射。

    \begin{proof}
        我们断言 $d$ 是满射。

        设 $z \in \mathbb{Z}$ 是任意固定整数。我们要证明 $\exists (a, b) \in \mathbb{N} \times \mathbb{N} \centerdot d(a, b) = z$。为此,考虑如下两种情况:
        \begin{enumerate}[label=(\arabic*)]
            \item 若 $z \ge 0$,则定义 $(a, b) = (z + 1, 1)$。\\
                  由 $z \ge 0$ 可知 $z+1 \ge 1$,因此 $z+1 \in \mathbb{N}$。这确保了 $(z + 1, 1) \in \mathbb{N} \times \mathbb{N}$。\\
                  此时 $d(z + 1, 1) = (z + 1) - 1 = z$。
            \item 若 $z < 0$,则定义 $ (a, b) = (1, -z + 1)$。\\
                  由 $z < 0$ 可知 $-z > 0$,因此 $-z+1 > 1$,这意味着 $-z+1 \in \mathbb{N}$。这确保了 $(1, -z+1) \in \mathbb{N} \times \mathbb{N}$。\\
                  此时 $d(1, -z + 1) = 1 - (-z + 1) = z$。
        \end{enumerate}
        无论哪种情况,我们都可以定义 $(a, b) \in \mathbb{N} \times \mathbb{N} \centerdot d(a, b) = z$。由于 $z \in \mathbb{Z}$ 是任意整数,这证明了 $d$ 是满射。
    \end{proof}
\end{example}

\begin{example}
    设函数 $g : \mathbb{R} - \{-1\} \to \mathbb{R}$ 定义为
    \[\forall x \in \mathbb{R} - \{-1\} \centerdot g(x) = \frac{x}{1+x}\]

    (注意,从定义域中移除了 $-1$ 是为了确保 $g$ 是\emph{良好定义的}。)

    我们来判断 $g$ 是否为满射,并证明我们的结论。如前所述,我们可以通过做一些验算来确定这一点:我们可以尝试代入一些 $x$ 的值,让 $x$ 趋近 $-1$ 或趋近无穷,以测试``极端情况''……这些都有助于我们绘制函数图像,或者我们可以使用绘图软件直接画出函数图像:

    \begin{center}
        \begin{tikzpicture}
            \begin{axis}[
                    ymin=-1.5,
                    ymax=3.5,
                    xmin=-5,
                    xmax=5,
                    minor tick num=3,
                    axis lines*=middle,
                    xtick align=inside,
                    ytick align=inside
                ]
                \addplot[blue, line width=1pt, domain=-5:-1.1, samples=40, smooth] {x/(1+x)};
                \addplot[blue, line width=1pt, domain=-0.9: 5, samples=60, smooth] {x/(1+x)};
                \addplot[black, dashed, domain=-5:5, samples=2] {1};
            \end{axis}
        \end{tikzpicture}
    \end{center}

    然而,无论是哪种绘图方式,都不构成\emph{证明}!它只是帮助我们观察到函数 $g$ \emph{不是}满射。在 $y = 1$ 处似乎有一条水平渐近线。也就是说,函数 $g$ 只会无限接近 $1$,但永远无法``达到'' $1$。根据\emph{满射}的定义,$g$ 显然不是满射!

    现在尝试证明这一点。你如何证明元素 $-1 \in \mathbb{R}$ \emph{不是}像 $\im_g(\mathbb{R})$ 的元素呢?请先尝试自行证明,然后继续阅读我们的证明。

    我们将给出\emph{两个}证明供你参考和比较。它们都证明了同一个结论 —— $g$ 不是满射。其中一个采用反证法,另一个采用直接证法(分情况讨论)。你觉得哪个更好?你是否也想到了其中一种证法?哪个更容易理解?对于这些问题,我们没有明确的观点;这两个证明都是有效的!

    \begin{proofs}{证明 1 (直接证法)}
        设 $x \in \mathbb{R} - \{-1\}$ 为任意固定元素。我们要证明 $g(x) \ne 1$。考虑如下两种情况:
        \begin{itemize}
            \item 若 $x > -1$。\\
                  这意味着 $x+1>0$,所以 $\frac{1}{x+1}>0$。已知 $x+1>x$ 对于所有 $x \in \mathbb{R}$ 均成立。不等式两边同时乘以一个正项 $\frac{1}{x+1}$ 得 $1 > \frac{x}{x+1}$,即 $g(x) = \frac{x}{x+1} \ne 1$。
            \item 若 $x < -1$。\\
                  这意味着 $x+1<0$,所以 $\frac{1}{x+1}<0$。已知 $x+1>x$ 对于所有 $x \in \mathbb{R}$ 均成立。不等式两边同时乘以一个负项 $\frac{1}{x+1}$ 得 $1 < \frac{x}{x+1}$,即 $g(x) =  \frac{x}{x+1} \ne 1$。
        \end{itemize}
        无论哪种情况都有 $g(x) \ne 1$。由于上述两种情况覆盖了所有可能的情况,并且 $x \in \mathbb{R} - \{-1\}$ 是任意的,这证明了
        \[1 \notin \im_g(\mathbb{R} - \{-1\})\]
        故 $g$ 不是满射。
    \end{proofs}

    请注意,证明 1 揭示了图像的一个有趣现象:该函数在 $x = -1$ 的左侧图像位于水平渐近线上方,而在 $x = -1$ 的右侧图像位于渐近线下方。

    \begin{proofs}{证明 2 (反证法)}
        为了引出矛盾而假设 $g$ 是满射。这意味着
        \[\forall y \in \mathbb{R} \centerdot y \in \im_g(\mathbb{R} - \{-1\})\]
        特别地,我们知道 $1 \in \im_g(\mathbb{R} - \{-1\})$,所以 $\exists x \in \mathbb{R} - \{-1\} \centerdot g(x) = 1$。给定这样的 $x$。

        这意味着 $g(x) = \frac{x}{x+1} = 1$。两边同时乘以分母得 $x = x + 1$。两边同时减去 $x$ 得 $0 = 1$,显然这是一个矛盾。$\hashx$

        因此,$1 \notin \im_g(\mathbb{R} - \{-1\})$,故 $g$ 不是满射。
    \end{proofs}

    请注意,证明 2 虽然确实证明了 $g$ 不是满射,但并未像证明 1 那样提供任何关于函数行为的信息。
\end{example}

接下来,我们来讨论一个与函数密切相关的性质。
