% !TeX root = ../../../book.tex
\section{本章习题}

本节习题涵盖本章全部内容,并涉及先前知识点及部分数学假设。我们不要求你解答\textbf{所有}题目,但解决得越多,收获越大!请牢记:真正\emph{掌握}数学必须亲自\emph{实践}。尝试动手解题,仔细阅读并思考题意。撰写证明并与朋友讨论,检验其说服力。持续练习如何清晰、准确、有条理地\emph{书写}思路。完成证明后要反复修改以臻完善。最重要的是,坚持\emph{钻研}数学!

标有 $\blacktriangleright$ 的简答题只需解释或陈述答案,无需严格证明。

特别具有挑战性的问题标记为 $\bigstar$。\\

\begin{exercise}
    对于以下每一条``规则'',给定定义域和值域,判断该``规则''是否定义了一个\textbf{良好定义的函数}。如有必要,请用例子解释你的答案。
    \begin{enumerate}[label=(\alph*)]
        \item 设 $a : \mathbb{Z} - \{1\} \to \mathbb{R}$ 定义为 $a(x) = \frac{x^2}{x-1}$。
        \item 设 $b : \mathbb{Q} \to \mathbb{Q}$ 定义为 $b(x = \sqrt{|x|})$。
        \item 设 $c : \mathbb{Z} \to \mathbb{Z}$ 定义为对于每个输入 $x \in \mathbb{Z}$ 输出 $s \in \mathbb{Z}$ 满足 $x \equiv s \mod 3$。
        \item 设 $d : \mathbb{N} \to \mathbb{N}$ 定义为 $d(x) = \lfloor\frac{x}{10}\rfloor$。
        \item 设 $e : \mathcal{P}(\mathbb{N}) \to \mathcal{P}(\mathbb{Z})$ 定义为输入一个自然数集合,输出该集合中最小自然数的所有整数倍的集合。
    \end{enumerate}
\end{exercise}

\begin{exercise}
    考虑集合 $\mathbb{R}^3 = \{(x, y, z) \mid x, y, z \in \mathbb{R}\}$ 和 $\mathbb{R}^2 = \{(a, b) \mid a, b \in \mathbb{R}\}$。\\
    定义函数 $f : \mathbb{R}^3 \to \mathbb{R}^2$ 为 $f(x, y, z) = (xz, yz)$。\\
    $f$ 是单射吗?$f$ 是满射吗?证明你的观点。
\end{exercise}

\begin{exercise}
    定义函数 $f : \mathbb{R} \to \mathbb{R}$ 为 $f(x) = x + 1$。\\
    定义函数 $g : \mathbb{R} \to \mathbb{R}$ 为 $g(x) = x^2 + x$。\\
    求复合函数 $f \circ g$ 和 $g \circ f$ 的表达式。(注意二者不同。)\\
    证明这两个复合函数都不是单射。
\end{exercise}

\begin{exercise}
    设函数 $f : \mathbb{Z} \to \mathbb{Z}$ 为 $f(x) = 2x - 3$。\\
    设函数 $g : \mathbb{Z} \to \mathbb{N}$ 为 $g(z) = |z| + 4$。\\
    $g \circ f$ 的定义域是什么?值域是什么?\\
    写出 $g \circ f$ 的表达式。该函数是单射吗?是满射吗?\\
    $\im_{g \circ f}(\mathbb{Z})$ 是什么?请证明你的观点。
\end{exercise}

\begin{exercise}
    以下每条规则都定义了一个 $\mathbb{N} \times \mathbb{N} \to \mathbb{Z}$ 的函数。对于每个函数,判断其是否为单射、满射或双射,还是都不是,并证明你的结论。
    \begin{enumerate}[label=(\alph*)]
        \item $f_1(a, b) = a - b$
        \item $f_2(a, b) = 2a + 3b$
        \item $f_3(a, b) = a$
        \item $f_4(a, b) = a^2 - b^2$
        \item $f_5(a, b) = 2^a \cdot 3^b$
    \end{enumerate}
\end{exercise}

\begin{exercise}
    在定义域 $\mathbb{N}$ 和值域 $\mathcal{P}(\mathbb{N})$ 上,按照如下性质定义函数 $f_1, f_2, f_3, f_4$,或者解释为什么所需的性质\textbf{无法}满足。
    \begin{itemize}
        \item $f_1$ 为单射但不为满射
        \item $f_2$ 既不是单射也不是满射
        \item $f_3$ 为满射但不为单射
        \item $f_4$ 为双射
    \end{itemize}
\end{exercise}

\begin{exercise}
    定义函数 $f : \mathbb{Z} \times \mathbb{Z} \to \mathbb{Z} \times \mathbb{Z}$ 为
    \[\forall (x, y) \in \mathbb{Z} \times \mathbb{Z} \centerdot f(x, y) = (y + 1, 3 - x)\]
    求 $f$ 的反函数 $F$,并证明 $F$ 确实是 $f$ 的反函数。\\
    这揭示了函数 $f$ 的哪些性质?
\end{exercise}

\begin{exercise}
    定义集合 $S = \{x \in \mathbb{R} \mid 0 < x < 1\}$。定义函数 $g:S \to \mathbb{R}$ 为
    \[g(x) = \frac{2x-1}{2x(1-x)}\]
    证明 $\im_g(S) = \mathbb{R}$。\\
    (\textbf{提示}:需要使用二次方程求根公式。)
\end{exercise}

\begin{exercise}
    设 $f : A \to B$ 和 $g : B \to C$ 为函数。
    \begin{enumerate}[label=(\alph*)]
        \item 若 $f, g$ 为满射,证明 $g \circ f : A \to C$ 也为满射。
        \item 若 $f, g$ 为单射,证明 $g \circ f : A \to C$ 也为单射。
        \item 若 $f, g$ 为双射,证明 $g \circ f : A \to C$ 也为双射。
    \end{enumerate} \label{exc:exercises7.8.9}
\end{exercise}

\begin{exercise}
    假设 $f : A \to B$ 和 $g : B \to C$ 为双射。定义 $h : A \to C$ 为 $h = g \circ f$。\\
    证明 $h$ 可逆,且其反函数为 $h^{-1} = f^{-1} \circ g^{-1}$。\\
    (\textbf{提示}:利用函数复合结合律。)\label{exc:exercises7.8.10}
\end{exercise} 

\begin{exercise}
    设 $f : A \to B$ 和 $g : B \to C$ 为双射,且 $X \subseteq A$。\\
    证明 $\im_{g \circ f} (X) = \im_g(\im_f (X))$。
\end{exercise}

\begin{exercise}
    设 $f : A \to B$ 为双射,所以 $f^{-1} : B \to A$ 为函数。设 $X \subseteq A$。\\
    证明 $\im_f (X) = \pim_{f^{-1}} (X)$。
\end{exercise}

\begin{exercise}
    设 $A, B$ 为集合,$f : A \to B$ 为函数。假设 $X,Y \subseteq A$。
    \begin{enumerate}[label=(\alph*)]
        \item $\im_f (X \cup Y) = \im_f (X) \cup \im_f (Y)$ 是否必然成立?请说明理由并证明。
        \item $\im_f (X \cap Y) = \im_f (X) \cap \im_f (Y)$ 是否必然成立?请说明理由并证明。
    \end{enumerate}
\end{exercise}

\begin{exercise}
    设 $f : A \to B$ 为函数。定义 $B$ 上的关系 $\sim$ 为:对于任意 $x,y \in B$,
    \[x \sim y \iff \pim_f (\{x\}) = \pim_f (\{y\})\]
    请解释为什么 $\sim$ 是等价关系。等价类是什么?\\
    假设 $f$ 为满射,等价类是什么?
\end{exercise}

\begin{exercise}
    设 $f : A \to B$ 为函数。定义 $A$ 上的关系 $\approx$ 为:对于任意 $x,y \in A$,
    \[x \approx y \iff f(x) = f(y)\]
    $\approx$ 是等价关系吗?如果是,请证明并描述等价类;如果不是,请给出反例。\\
    假设 $f$ 为单射。$\approx$ 是等价关系吗?如果是,请证明并描述等价类;如果不是,请给出反例。
\end{exercise}

\begin{exercise}
    设 $f : A \to B$ 为函数,并设 $X,Y \in A$。考虑命题 $\im_f (X) \cap \im_f (Y) \subseteq \im_f (X \cap Y )$。下面对此命题的``错误证明''错在何处?
    \begin{quote}
        \begin{spoof}
            设 $z \in \im_f (X) \cap \im_f (Y)$。

            由于 $z \in \im_f (X)$,这意味着 $\exists a \in X$ 使得 $f(a) = z$。

            由于 $z \in \im_f (Y)$,这意味着 $\exists a \in Y$ 使得 $f(a) = z$。

            因为 $a \in X$ 且 $a \in Y$, 故 $a \in X \cap Y$。

            因由 $f(a) = z$ 可得 $z \in \im_f (X \cap Y)$。
        \end{spoof}
    \end{quote}
    请举出反例证明该命题不成立。
\end{exercise}

\begin{exercise}
    证明或证伪 $\mathcal{P}(\mathbb{N})$ 和 $\mathcal{P}(\mathbb{Z})$ 具有相同的基数。
\end{exercise}

\begin{exercise}
    给定任意 $n \in \mathbb{N}$。考虑集合 $[n] = \{1, 2, 3, \dots , n\}$。\\
    设 $E$ 为 $[n]$ 中具有\textbf{偶数}个元素的子集构成的集合(例如 $\varnothing$ 或 $\{1,4\}$),\\
    设 $O$ 为 $[n]$ 中具有\textbf{奇数}个元素的子集构成的集合(例如 $\{5\}$ 或 $\{1,2,3\}$)。\\
    定义\textbf{双射}函数 $p : E \to O$,并证明它确实为双射。\\
    (\textbf{提示}:尝试一些较小的数值,如 $n = 1, n = 2, n = 3$,再推广至一般情况。)
\end{exercise}

\begin{exercise}
    证明引理 \ref{lemma7.6.16}。\label{exc:exercises7.8.19}\\
    \textbf{提示}:可以参考定理 \ref{theorem7.6.7} 的证明思路:利用 $B$ 的大小来``增加'' $A$ 和 $\mathbb{N}$ 之间的双射数量。
\end{exercise}

\begin{exercise}
    证明推论 \ref{corollary7.6.19}。即假设 $A$ 和 $B$ 为可数无限集,通过应用引理 \ref{lemma7.6.18} 合理选择集合来证明 $A \cup B$ 为可数无限集。 \label{exc:exercises7.8.20}
\end{exercise}

\begin{exercise}
    回顾示例 \ref{ex:example7.6.13},在该示例中我们定义函数 $f : \mathbb{N} \times \mathbb{N} \to \mathbb{N}$ 为
    \[\forall (x, y) \in \mathbb{N} \times \mathbb{N} \centerdot f(x, y) = 2^{x-1}(2y - 1)\]
    证明 $f$ 为\textbf{单射}。\label{exc:exercises7.8.21}
\end{exercise}

\begin{exercise}
    证明推论 \ref{corollary7.6.21}。即假设 $A_1, A_2,\dots,A_n$ 为有限个可数无限集,证明
    \[A_1 \cup A_2 \cup \dots \cup A_n \qquad \text{和} \qquad A_1 \times A_2 \times \dots \times A_n\]
    均为可数无限集。\label{exc:exercises7.8.22}
\end{exercise}

\begin{exercise}
    定义集合 $A$ 为
    \[A = \{(a, b) \in \mathbb{N} \times \mathbb{N} \mid a \le b\}\]
    用两种方法证明 $A$ 为可数无限集。
    \begin{enumerate}[label=(\arabic*)]
        \item 通过将 $A$ 表示为集合的并集并引用相关结论来证明。
        \item 通过构造 $A$ 与某个可数集之间的明确双射来证明。
    \end{enumerate}
\end{exercise}

\begin{exercise}
    定义函数 $g : \mathbb{N} \times \mathbb{N} \to \mathbb{N}$ 为
    \[\forall (x, y) \in \mathbb{N} \times \mathbb{N} \centerdot g(x, y) = (x + y)^2 + x\]
    证明
    \begin{tasks}[label=(\alph*)](2)
        \task $g$ 是单射;
        \task $g$ 不是满射。
    \end{tasks}
\end{exercise}

\begin{exercise}
    设 $A,B,C$ 为集合。设 $f : A \to B, g : B \to C, h : B \to C$ 为函数。
    \begin{enumerate}[label=(\alph*)]
        \item 假设 $g = h$,则 $g \circ f = h \circ f$ 是否必然成立?证明或证伪你的结论。
        \item 假设 $g \circ f = h \circ f$,则 $g = h$ 是否必然成立?证明或证伪你的结论。
    \end{enumerate}
\end{exercise}

\begin{exercise}
    设 $A, B$ 为有限集,且 $|A| = |B| = n$。假设 $f : A \to B$ 为函数。证明
    \[f \text{ 为单射} \iff f \text{ 为满射}\]
\end{exercise}

\begin{exercise}
    考虑如下命题:
    \begin{quote}
        假设 $f : A \to B$ 和 $g : B \to C$ 为函数。若 $g \circ f : A \to C$ 为单射,则 $g$ 也为单射。
    \end{quote}
    下面对此命题的``错误证明''错在何处?
    \begin{quote}
        \begin{spoof}
            假设 $g \circ f$ 为单射。我们要证明 $g$ 也为单射。

            任取 $x, y \in B$。假设 $g(x) = g(y)$。

            我们知道 $\exists a,b \in A$ 满足 $f(a) = x$ 和 $f(b) = y$。

            由于 $g$ 是良好定义的函数,这意味着 $g(f(a)) = g(x)$ 且 $g(f(b)) = g(y)$。

            假设 $g \circ f$ 为单射且 $g(f(a)) = g(f(b))$,这意味着 $a = b$。

            因为 $f$ 是良好定义的函数,所以 $f(a) = f(b)$。

            这意味着 $x = y$。因此 $g$ 为单射。
        \end{spoof}
    \end{quote}
    请给出反例,证明该命题的结论不成立。
\end{exercise}

\begin{exercise}
    设 $a, b \in \mathbb{R}$ 为任意固定实数。假设 $a^2 + b^2 \ne 0$。\\
    定义函数 $f : \mathbb{R} \times \mathbb{R} \to \mathbb{R} \times \mathbb{R}$ 为
    \[\forall (x, y) \in \mathbb{R} \times \mathbb{R} \centerdot f(x, y) = (ax - by, bx + ay)\]
    通过求 $f$ 的反函数证明 $f$ 为双射,并证明该反函数是正确的。
\end{exercise}

\begin{exercise}
    设 $A$ 和 $B$ 为有限集,且 $|A| = |B|$。\\
    假设函数 $f : A \to B$ 为单射。\\
    通过证明 $\im_f (A) = B$ 证明 $f$ 也必然为满射。
\end{exercise}

\begin{exercise}
    给定 $k \in \mathbb{N} - \{1\}$。定义
    \begin{align*}
        S_1 & = \{X \in \mathcal{P}([k]) \mid k \notin X\} \\
        S_2 & = \{X \in \mathcal{P}([k]) \mid k \in X\}
    \end{align*}
    \begin{enumerate}[label=(\alph*)]
        \item 证明 $S_1$ 和 $S_2$ 构成 $\mathcal{P}([k])$ 的一个划分。
        \item 定义双射函数 $f_1 : S_1 \to \mathcal{P}([k-1])$,并证明其确为双射。
        \item 定义双射函数 $f_2 : S_2 \to \mathcal{P}([k-1])$,并证明其确为双射。
        \item 利用前三小问的结论,写一个\textbf{归纳}证明,证明对于所有 $n \in \mathbb{N}, \mathcal{P}([n])$ 有 $2^n$ 个元素。

              注意: 由于 $k \ge 2$ 的限制,将 $n = 1$ 作为基本情况,使用 $n = k \ge 1$ 作为归纳假设,并在归纳步骤中证明 $n = k + 1$ 时命题成立。
    \end{enumerate}\label{exc:exercises7.8.30}
\end{exercise}

\begin{exercise}
    设 $A, B, C, D$ 为集合,且 $A \cap B = C \cap D = \varnothing$。假设 $f : A \to B$ 和 $g : C \to D$ 为双射。\\
    定义分段函数 $h : A \cup B \cup C \cup D$ 为
    \[\forall x \in A \cup B \centerdot h(x) = \begin{cases}
            f(x) & \text{若}\; x \in A \\
            g(x) & \text{若}\; x \in B
        \end{cases}\]
    解释为什么 $h$ 是良好定义的函数,并证明 $h$ 为\textbf{双射}。\label{exc:exercises7.8.31}
\end{exercise}

\begin{exercise}
    在这个问题中,你将证明当集合 $A$ 和 $B$ 为有限集,且 $|A| = a, |B| = b$ 时,$|A \times B| = ab$。这个证明将采用一种叫做``双重归纳法''的方法,对两个变量 $a,b \in \mathbb{N}$ 进行归纳推理。
    \begin{enumerate}[label=(\alph*)]
        \item 证明 $\big|[1] \times [1]\big| = 1$。(这非常非常简单,但却是必不可少的。)
        \item 假设 $n \in \mathbb{N}$ 且 $\big|[1] \times [n]\big| = n$,证明 $\big|[1] \times [n+1]\big| = n+1$。
        \item 解释为什么 (a) 和 (b) 证明了 $\forall n \in \mathbb{N} \centerdot \big|[1] \times [n]\big| = n$。
        \item 假设 $k \in \mathbb{N}$ 并假设 $\forall n \in \mathbb{N} \centerdot \big|[k] \times [n]\big| = kn$。证明 $\forall n \in \mathbb{N} \centerdot \big|[k+1] \times [n]\big| = (k+1)n$。
        \item 解释为什么 (c) 和 (d) 证明了 $\forall k,n \in \mathbb{N} \centerdot \big|[k] \times [n]\big| = kn$。
        \item 解释为什么 (e) 证明了上述结论。
    \end{enumerate}\label{exc:exercises7.8.32}
\end{exercise}

\begin{exercise}
    设 $S$ 为所有无限长二进制字符串的集合(也就是说,$S$ 中的元素是由 $0$ 和 $1$ 组成的无限长字符串)。找出 $S$ 和 $\mathcal{P}(\mathbb{N})$ 之间的双射,并由此证明 $S$ 是不可数无限集。\label{exc:exercises7.8.33}
\end{exercise}

\begin{exercise}
    对于以下每个集合,给出了它们的基数。并通过找到相关集合之间的双射关系和/或引用相关结论证明答案的正确性。\\
    (\textbf{提示}:如果不使用某种形式的归纳论证,你的证明可能会不够严谨……)
    \begin{enumerate}[label=(\alph*)]
        \item $A$ 是所有从 $\mathbb{N}$ 到 $\mathbb{N}$ 函数的集合。证明 $A$ 是不可数无限集。\\
              (\textbf{提示}:比较 $A$ 与所有从 $\mathbb{N}$ 到 $\{1, 2\}$ 的函数集合 $S$。你能解释为什么 $S$ 是不可数无限集吗?这对 $A$ 有什么启示?…… )
        \item $B$ 是所有从 $\mathbb{N}$ 到 $\mathbb{N}$ 且具有如下性质的函数的集合
              \[\forall x \in \mathbb{N} \centerdot f(x + 1) = f(x) + 1 \]
              证明 $B$ 是可数无限集。
        \item $C$ 是所有从 $\mathbb{N}$ 到 $\mathbb{N}$ 且具有如下性质的函数的集合
              \begin{align*}
                  \forall x \in \mathbb{N} \centerdot f(x + 1) & = f(x) + 1 \\
                  f(1)                                         & = 42
              \end{align*}
              证明 $C$ 为有限集,且只有一个元素。
    \end{enumerate}
\end{exercise}

\begin{exercise}
    回顾示例 \ref{ex:example7.6.14},我们通过将该集合描绘成点格并描述了一条覆盖所有点的路径,从而(非正式地)论证了 $\mathbb{N} \times \mathbb{N}$ 是可数无限集。\\
    现在,我们通过定义一个函数 $f : \mathbb{N} \times \mathbb{N} \to \mathbb{N}$(或反过来)来形式化这一论证。该函数实现了我们描述的路径(或类似路径的概念),并且我们需要证明它是一个双射。
\end{exercise}

\begin{exercise}
    证明推论 \ref{corollary7.6.23},即证明可数无限个互不相交的有限集的并集为可数无限集。\label{exc:exercises7.8.36}
\end{exercise}

\begin{exercise}
    考虑定理 \ref{theorem7.6.22},该定理指出可数无限个可数无限集的并集仍为可数无限集。在我们的证明中,只考虑了互不相交的情况。本题中,需要你证明更一般的情况,即集合不一定互不相交。\\
    (\textbf{提示}:请回顾我们在证明中使用的函数。你能否对它们进行调整,从而找到一个从 $\mathbb{N} \times \mathbb{N}$ 到这些集合并集的\emph{满射}?)\label{exc:exercises7.8.37}
\end{exercise}

\begin{exercise}
    设集合 $S$ 为所有无限长二进制字符串的集合。我们之前已经证明 $S$ 为不可数无限集。\\
    设集合 $T \subseteq S$ 为所有仅包含\emph{有限个} $1$ 的无限长二进制字符串的集合。\\
    本题中,你将证明 $T$ 实际上是\emph{可数无限集}。
    \begin{enumerate}[label=(\alph*)]
        \item 考虑所有有序自然数 $k$ 元组的集合 $\mathbb{N}^k$。(即 $\mathbb{N}^1 = \mathbb{N}, \mathbb{N}^2 = \mathbb{N} \times \mathbb{N}$)

              给出一个归纳论证,证明对于每个 $k \in \mathbb{N}, \mathbb{N}^k$ 为可数无限集。

              (提示:该证明应该很简短。你可以引用讲义中关于可数无限集的笛卡尔积的结论。)
        \item 对于每个 $k \in \mathbb{N}$,设 $T_k \subseteq T$ 为所有\emph{恰有} $k$ 个 $1$ 的无限长二进制字符串的集合。

              构造一个从 $T_k$ 到 $\mathbb{N}^k$ 的双射(或至少是单射)函数。解释为什么该函数是良好定义的,且是一个双射(或单射)函数。
        \item 用 (b) 的结论推导出 $T_k$ 为可数无限集。(注意:如果你只找到了单射函数,那么你还需要解释为什么 $T_k$ 不是有限集。)
        \item 将 $T$ 表示为集合的并集,并推导出 $T$ 是可数无限集。\\
              (提示:需要应用讲义中的一个重要定理。)
    \end{enumerate}
    (旁注:思考一下此结论的影响。通过一个简单的双射关系,你可以推导出所有仅包含有限个 $0$ 的无限长二进制字符串的集合也是可数无限集。这意味着,集合 $S$(即所有无限长二进制字符串的集合)之所以是\emph{不可数}无限集,完全依赖于包含无限个 $1$ 和 无限个 $0$ 的字符串的集合。仅该集合就足以令 $S$ 不可数!)
\end{exercise}

\begin{exercise}
    \begin{enumerate}[label=(\alph*)]
        \item 设 $n \in \mathbb{N}$。考虑集合
              \[S = \{f : [n] \to [n] \mid f \;\text{为双射} \}\]
              证明 $S$ \emph{在复合运算下是封闭的};即证明
              \[\forall f, g \in S \centerdot f \circ g \in S\]
              (\textbf{提示}:引用本章习题中的某个问题。)
        \item 考虑集合
              \[T = \big\{ f : \mathbb{N} \to \mathbb{N} \mid f \;\text{为双射且}\; \{i \in \mathbb{N} \mid f(i) \ne i\} \;\text{为有限集}\big\}\]
              证明 $T$ \emph{在复合运算下也是封闭的}。
        \item 证明 $T$ \emph{在求逆运算下是封闭的};即证明
              \[\forall f \in T \centerdot \exists f^{-1} \land f^{-1} \in T\]
        \item 考虑集合
              \[U = \{f : \mathbb{N} \to \mathbb{N} \mid f \;\text{为双射} \}\]
              证明 $U$ 在求逆运算下是封闭的。
        \item 证明
              \[\forall f \in T \centerdot \forall g \in U - T \centerdot \big(f \circ g \notin T \land g \circ f \notin T\big)\]
        \item 找出反例证明 $U-T$ 在复合运算下\textbf{不是}封闭的。
        \item 给定 $A \subseteq \mathbb{N}$ 且 $A$ 为\textbf{有限集}。找到函数 $f, g \in U - T$ 使得
              \[\{i \in \mathbb{N}  \mid (f \circ g)(i) \ne i\} = A\]
        \item $S, T, U$ 的基数分别是多少?如果你的答案是``有限的'',请同时给出具体的大小。如果你的答案是``无限的'',请说明是可数无限还是不可数无限,并通过构造与适当集合之间的双射或引用相关结论来证明你的结论。
    \end{enumerate}
\end{exercise}
