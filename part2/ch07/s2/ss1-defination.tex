% !TeX root = ../../../book.tex

\subsection{定义}

我们将使用以下定义。或许它与你的定义接近,甚至相同,或者只是措辞略有不同。但这个定义完美地概括了我们前面对函数的直观概念(将函数看作一种分配\emph{规则}),并用我们一直以来在发展的集合和逻辑语言来表达。这样做有几个好处:

\begin{enumerate}[label=(\alph*)]
    \item 它为函数提供了严格的基础,使我们能够在数学意义上放心地使用它们;
    \item 它让我们能够讨论函数的性质,并用数学术语和概念来证明这些性质;
    \item 它使我们能够概括函数的概念,并将其应用于比我们熟悉的标准数集更抽象的场景中。
\end{enumerate}
好了,解释到此为止,我们来看定义。\\

\begin{definition}
    设 $A, B$ 为集合。设 $f$ 为 $A, B$ 间的关系,所以 $f \subseteq A \times B$。同时假设 $f$ 具有如下性质:
    \[\forall a \in A \centerdot \exists! b \in B \centerdot (a, b) \in f\]
    (回想一下,``$\exists!$'' 表示``存在唯一的……'',也就是说``有且只有一个……'')

    这样的关系称为 $A$ 到 $B$ 的 \dotuline{函数}。

    我们称 $A$ 为函数的\dotuline{定义域},$B$ 为函数的\dotuline{值域}。

    写做
    \[f:A \to B\]
    表示 $f$ 是 $A$ 到 $B$ 的函数。

    如果 $(a,b) \in f$,则我们写做
    \[f(a) = b\]
    并且知道对于给定的 $a$,$b$ 是唯一满足该属性的元素。
\end{definition}

就是这样!虽然现在把\emph{函数}看作一种\emph{关系} --- 实际上是一种特殊的\emph{集合} --- 可能有点奇怪,但这正是它的本质。用这种方式定义函数让我们能够用集合和关系的语言来讨论它们,同时我们仍然可以使用一些熟悉的符号。对于每一个``输入'' $a$(即\emph{定义域}中的每一个元素),都有\emph{唯一}一个``输出'' $b$(即\emph{值域}中的一个元素)。因此,我们可以写成 $f(a) = b$,并且知道 ``$=$'' 表示真正的相等关系,因为只有这个唯一的 $b$ 满足这种关系。

这种定义的一部分包含了我们之前提到的想法:我们想知道函数会``输出''什么\emph{类型}的对象。这就是通过指定值域实现的。例如,定义函数 $f : \mathbb{R} \to \mathbb{R}$ 为 $f(x) = \sqrt{x}$ 是不合适的,因为定义域中的某些元素(即负数)会使``输出''未定义。(技术上讲,输出会是一个复数,而复数不是值域 $\mathbb{R}$ 的元素;在 $\mathbb{R}$ 的上下文中,我们认为复数是``未定义''的。)当一个函数被正确定义,且定义域和值域已明确,并且相关的对确实属于集合的笛卡尔积时,我们称这个函数是\textbf{良好定义的}。有时,我们会给出两个集合之间的关系,并要求你判断它是否是\emph{良好定义的函数}。实际上,这就是在问这个关系是否符合函数的定义。

\subsubsection*{范围}

\emph{值域}这个词对你来说可能比较陌生。实际上,你可能更习惯用\textbf{范围}来指代函数的\emph{潜在}``输出''集合。我们想在这个上下文中完全避免使用``范围''一词,因为它可能会产生歧义。有些作者和老师用``范围''来表示我们这里所说的``值域'',即函数的\emph{潜在}``输出''集合;而另一些人则用它来表示本书所说的``像'',即函数的\emph{实际}``输出''集合。在 \ref{sec:section7.3} 节中我们会详细定义这个术语。通常,像是值域的子集,但往往可能是\emph{真}子集。当有人使用``范围''这个词时,他们可能指的是这两种解释中的其中一个,而你可能理解的是另一个!为了避免这种混淆,我们将只使用\emph{值域}和\emph{像}这两个词。