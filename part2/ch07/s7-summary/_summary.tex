% !TeX root = ../../../book.tex
\section{总结}

现在,我们已经全面探索了\textbf{函数}及其相关性质!我们发现,函数其实是一种具有特定性质的关系。这个所需的性质就是每个可能的``输入''都有一个对应的``输出''。我们用数学术语形式化了这些概念,包括定义域、值域和像。此外,函数还有一些其他性质,比如单射性和满射性。我们通过许多例子和反例,讨论了如何证明或证伪这些性质,并应用了我们的逻辑证明技巧。

\emph{双射}这一概念特别有用且强大。我们将其与\emph{反函数}的概念联系起来。具体来说,我们证明了一个函数是双射\emph{当且仅当}它有一个反函数!这在我们讨论\textbf{基数}时非常重要,因为``双射为王''。通过``配对元素''的概念,我们理解了一些关于``集合大小''的奇妙和反直觉的结果。

我们将无限集分为可数无限集和不可数无限集。然而,我们还证明了康托尔定理,这一历史性成果表明实际上有无穷多种\emph{基数}!对于我们的讨论来说,区分这两种无限集类型已经足够。我们看到了每种类型的若干例子,并证明了一些从其他集合构建特定基数集合的定理。最终,我们发现这些结果既有趣又具有数学教育意义。不过,从现在开始,我们将只关注\textbf{有限集}。

