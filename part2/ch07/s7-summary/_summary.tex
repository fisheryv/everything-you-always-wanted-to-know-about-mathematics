% !TeX root = ../../../book.tex
\section{总结}

至此,我们已经全面探索了\textbf{函数}及其相关性质!我们认识到,函数本质上是一种具有特定性质的关系。这个所需的性质就是每个可能的``输入''都有一个对应的``输出''。我们使用数学语言将这些概念形式化,包括定义域、值域和像。此外,函数还具有其他重要性质,例如单射性和满射性。通过大量实例与反例,我们讨论了如何证明或证伪这些性质,并运用了之前所学的逻辑推理技巧。

特别值得关注的是,\emph{双射}这一概念既强大又实用。我们将其与\emph{反函数}联系起来,证明了函数是双射\emph{当且仅当}它存在反函数!这一结论在讨论\textbf{基数}时尤为重要,因为``双射为王''。通过``配对元素''的概念,我们得以理解一些关于``集合大小''的奇妙甚至反直觉的结论。

我们将无限集划分为可数无限集和不可数无限集,并借助康托尔定理这一历史性成果,认识到实际上存在无穷多种不同的\emph{基数}!尽管对无限集的完整分类极为复杂,但仅区分上述两种类型便足以支撑我们的讨论。我们考察了各类无限集的例子,证明了若干从已知集合构造特定基数集合的定理。这些结果不仅富有启发性,也具有重要的数学教育意义。不过,从此刻起,我们将把注意力聚焦于\textbf{有限集}。
