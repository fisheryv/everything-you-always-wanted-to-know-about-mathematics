% !TeX root = ../../../book.tex

\section{引言}

我们将继续探讨函数的第二个章节。在本章中,我们将正式\emph{定义}函数。具体来说,函数实际上是一种具有特定性质的\textbf{关系}。这也是我们之前花时间研究关系的原因 --- 除了关系本身的趣味性和实用性之外。定义函数后,我们将通过大量实例和证明来研究函数可能具有的各种特性。在这个过程中,我们将用到之前所有学过的概念,特别是 \ref{sec:section4.9} 节中的证明技巧。

在本章的后半部分,我们将介绍\emph{双射}函数 (Bijective Function) 的概念,即两个集合元素的一一对应关系,用来讨论集合的``大小''及其比较方法。这个主题被称为\emph{基数} (Cardinality),它将揭示一些关于无限集的惊人且反直觉的事实。这也为我们进入下一章奠定基础,在下一章中,我们将集中研究有限集及其计数方法。

% !TeX root = ../../../book.tex
\subsection{目标}

以下简要说明本章在本书中的定位:它将阐释先前内容如何发挥作用,阐述研究本章主题的动机,明确学习目标,并提示阅读时的关注重点。我们先列出本章的核心目标及学成后应掌握的技能,后续章节将详细展开。学完本章后,请返回此处核验:你是否理解所有目标?能否阐述其重要性?能否定义相关术语?能否运用相关技术?

\textbf{学完本章后,你应该能够……}

\begin{itemize}
    \item 明确数学归纳法的定义,并对给定证明方法进行归纳法/非归纳法分类
    \item 根据问题特征判断适用归纳法的场景
    \item 通过类比方式直观描述数学归纳法的运作机制
    \item 辨别不同归纳证明的异同,分析其对应问题的结构特征
\end{itemize}


% !TeX root = ../../../book.tex

\subsection{承上}

前一章介绍的重要概念\textbf{关系}将在本章发挥重要作用。我们已经提到,\textbf{函数}实际上是关系的一种特例,这一点将在我们对函数的正式定义中体现。

至于前一章讨论的其他概念,比如等价关系和数论结果,它们在本章中不会直接出现。也就是说,本章中我们探讨的函数及其性质,并不依赖于这些概念。然而,我们会利用这些概念来设计一些有趣的示例和练习。

% !TeX root = ../../../book.tex
\subsection{启下}

回顾 \ref{sec:section1.4.3} 节的问题,我们证明了前 $n$ 个奇数之和等于 $n^2$。最初通过几何视角观察这一模式:将奇数项排列为逐渐扩大的正方形``角块''。然而,第一种证明方法似乎并未依赖这一观察,而是以\emph{代数}方式运用了关于偶数与奇数之和的既有结论——通过对若干等式进行乘法、减法等操作,最终得到了预期结果。这种方法是否令人满意?它在某种程度上偏离了最初的几何解释,其有效性或许出人意料。(也许存在\emph{不同的}几何解释?读者可尝试探寻。)

第二种方法则是对几何观察的代数建模。我们将求和与正方形面积建立联系,将求和项对应于图形的特定部分。通过在不同问题解释间构建\emph{对应关系},使几何与代数解释互为支撑,共同指向同一结论。这种视觉化优势在于启发了名为\textbf{数学归纳法}的通用证明策略(简称\textbf{归纳法})。(请注意:\emph{归纳法}在电磁学或哲学等领域另有含义,但本书特指\emph{数学归纳法}。)究竟何为归纳法?其运作机制如何?适用范围是什么?如何针对具体问题调整策略?是否存在更有效的变体?本章将解答这些问题。

首先要探讨的,是此前未提及的核心问题:``\emph{为何}采用归纳法?\emph{为何}重视它?''基于 \ref{sec:section1.4.3} 节的问题,数学归纳法看似并非必需,因为其他方法同样可完成证明。这在一定背景下成立,但需强调:\emph{归纳法极具实用价值!}在众多情形中,它是最简洁的证明途径,且作为通用策略可广泛应用于同类问题。此外,适用归纳法的问题需具备特定\text{结构}——即结果的``后续部分''依赖``前序部分''。(``部分''与``依赖''的具体含义取决于上下文。)识别归纳法的适用性并完成证明过程,常能揭示问题的内在结构。即使归纳证明失败,发现``破坏''归纳步骤的具体环节,往往也能提供深刻洞见。

我们将通过若干示例阐明这些观点,再给出数学归纳法的完整\emph{定义}以展示其通用原理。(\text{严格}的形式化定义将延后至后续章节,待集合论、逻辑陈述与蕴涵等基础概念完备后展开。目前给出的定义已足以解决一些有趣的难题,并支撑归纳法作为通用证明策略的讨论。)


% !TeX root = ../../../book.tex

\subsection{忠告}

本章将继续探讨抽象概念和严谨的数学内容,因此,如果你对这些日益抽象的内容和相关语言感到不适,千万不要因此就认为这些信息``无关紧要''或``百无一用''。这些概念会贯穿整本书,甚至整个数学领域!所以,当你感觉难以集中注意力时,请记住这一点。我们建议你记下学习笔记,以便提醒自己当前的学习内容。当你看到一个定理并多次阅读后终于理解时,可以在书的边缘写下定理的摘要,方便以后查阅。画一些小图形帮助你理解例子或定理的重要部分。遇到定义时,写下一个典型例子和一个反例。读完证明后,记下论证步骤的大纲,这样可以将概念``模块化'',便于更有效地记忆和回忆。如果你对某个定义、定理或证明感到困惑,也要记下来!带着问题去问同学、朋友、助教或教授,看看他们能否帮你解答。最重要的是,请记住,理解消化和融会贯通这些抽象概念和论证需要\emph{时间},通过例子验证自己是否跟上进度依然非常重要。如果你能理解并向别人解释某个概念,那说明你已经掌握得很好了。