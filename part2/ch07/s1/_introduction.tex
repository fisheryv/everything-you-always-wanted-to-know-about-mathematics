% !TeX root = ../../../book.tex

\section{引言}

我们将继续探讨函数的第二个章节。在本章中,我们将正式\emph{定义}函数。具体来说,函数实际上是一种具有特定性质的\textbf{关系}。这也是我们之前花时间研究关系的原因 --- 除了关系本身的趣味性和实用性之外。定义函数后,我们将通过大量实例和证明来研究函数可能具有的各种特性。在这个过程中,我们将用到之前所有学过的概念,特别是 \ref{sec:section4.9} 节中的证明技巧。

在本章的后半部分,我们将介绍\emph{双射}函数 (Bijective Function) 的概念,即两个集合元素的一一对应关系,用来讨论集合的``大小''及其比较方法。这个主题被称为\emph{基数} (Cardinality),它将揭示一些关于无限集的惊人且反直觉的事实。这也为我们进入下一章奠定基础,在下一章中,我们将集中研究有限集及其计数方法。

% !TeX root = ../../../book.tex

\subsection{目标}

以下简短内容将向你展示本章如何融入本书的体系。我们会解释之前的工作对本章研究的帮助,说明我们为什么要探讨本章的主题,并告诉你我们的目标以及在阅读时需要注意的事项。现在,我们将通过几条陈述总结本章的主要目标。这些陈述概括了你在完成本章后应掌握的技能和知识。接下来的章节会更详细地解释这些思想,这里仅提供一个简要列表供你参考。完成本章后,请返回这个列表,检查你是否理解所有目标。你能看出我们为什么认为这些目标重要吗?你能解释我们使用的术语并应用我们描述的技术吗?

\textbf{学完本章后,你应该能够……}

\begin{itemize}
    \item 定义函数,并提供多个例子。
    \item 使用函数的非正式描述和可视化图像来构建关于函数(示例和反例)及其属性的正式论证。
    \item 在函数的上下文中定义集合的像 (Image) 和原像 (Pre-image),并证明这些操作的各种属性。
    \item 陈述函数的属性,并应用相关方法来确定和证明给定函数是否具有这些属性。
    \item 找出两个函数的复合,说明如何用它们来创建新函数,并解释和证明复合对所涉及函数属性的影响。
    \item 描述双射函数 (Bijective Function) 和反函数 (Inverse Function) 之间的关系,并用它们来解决问题和证明结论。
    \item 使用双射来定义集合的基数 (Cardinality),并证明关于这些基数的结论。
    \item 说明有限集、可数无限集和不可数无限集之间的区别,并提供每种类型的多个例子。
\end{itemize}

% !TeX root = ../../../book.tex

\subsection{承上}

前一章介绍的重要概念\textbf{关系}将在本章发挥重要作用。我们已经提到,\textbf{函数}实际上是关系的一种特例,这一点将在我们对函数的正式定义中体现。

至于前一章讨论的其他概念,比如等价关系和数论结果,它们在本章中不会直接出现。也就是说,本章中我们探讨的函数及其性质,并不依赖于这些概念。然而,我们会利用这些概念来设计一些有趣的示例和练习。

% !TeX root = ../../../book.tex

\subsection{动机}

正如我们在上一章提到的,你很可能已经对函数的概念和使用方法有了一些直观的理解。这些理解可能来自你之前的数学课程,或者来自计算机编程的经验。我们一直强调,希望能正确、正式地用\emph{数学方式}定义我们所研究的概念,函数也不例外!通过正式定义,我们可以更好地讨论一些你可能以前见过但无法清楚表达的函数性质。此外,函数的某些特定性质还会帮助我们讨论集合的\emph{基数}。请相信我,如果不先探讨函数,我们就无法对这个主题进行深入讨论。


% !TeX root = ../../../book.tex

\subsection{忠告}

我们还要重申上一章中的一些忠告和建议。我们正在继续探索一些抽象的数学领域。本章特别是要将一个你可能在视觉和直觉上熟悉的概念放在更严格的基础之上。我们会尽可能利用大家的直觉,但也无法避免地需要进行抽象思维和问题解决的过程。特别是,我们无法总是将函数与它的图像联系起来,尽管这是我们在数学学习早期常用且有效的方法。此外,在基数的讨论中,我们会遇到一些完全违反直觉的结果。真的!这些奇怪又反直觉的事实需要我们保持开放的心态去理解。