% !TeX root = ../../../book.tex

\subsection{关于原像的证明}

请注意,以下命题是\textbf{相等}关系。将其与命题 \ref{prop:proposition7.3.6} 比较:后者是关于\emph{像}的类似陈述,但仅为集合\emph{包含}关系,这一差异颇具深意。

\begin{proposition}
    设 $A,B$ 为集合,$f:A \to B$ 为函数。若 $X, Y \subseteq B$,则
    \[\pim_f (X \cap Y) = \pim_f (X) \cap \pim_f (Y)\]
\end{proposition}

请注意以下证明是如何直接运用原像的形式定义的。我们将直接给出双向包含证明。练习部分将要求你用 $\cup$ 替换 $\cap$ 来验证相应结论。

\begin{proof}
    设 $x \in \pim_f (X \cap Y)$ 为任意固定元素。

    根据原像的定义,这意味着 $f(x) \in X \cap Y$,即 $f(x) \in X$ 且 $f(x) \in Y$。

    因为 $f(x) \in X$,根据原像的定义,这意味着 $x \in \pim_f (X)$。同理,因为 $f(x) \in Y$,这意味着 $x \in \pim_f (Y)$。

    根据交集的定义,可以推导出 $x \in \pim_f(X) \cap \pim_f(Y)$。

    这证明了 $\pim_f(X \cap Y) \subseteq \pim_f(X) \cap \pim_f(Y)$。\\

    接着,设 $y \in \pim_f(X) \cap \pim_f(Y)$ 为任意固定元素。

    根据交集的定义,这意味着 $y \in \pim_f (X)$ 且 $y \in \pim_f (Y)$。

    因为 $y \in \pim_f (X)$,根据原像的定义,可以推导出 $f(y) \in X$。同理,因为 $y \in \pim_f (Y)$,可以推导出 $f(y) \in Y$。

    根据交集的定义,可得 $f(y) \in X \cap Y$。

    根据原像的定义,可得 $y \in \pim_f(X \cap Y)$。

    这证明了 $ \pim_f(X \cap Y) \supseteq \pim_f(X) \cap \pim_f(Y)$。 \\

    综上,利用双向包含论证,该命题得证。
\end{proof}

读到这里你可能会想:``这种证明方法是怎么想出来的?''其实,这样的结果并没有什么特别的巧思。我们只是直接引用定义,一切便顺理成章。如果你在解决问题时感到困惑,不知从何入手……不妨写下相关定义,尝试将其应用于待证命题,观察结果如何!

\subsubsection*{像与原像的证明}

以下例子涉及本节的两个核心概念。我们将证明一个包含关系,并在练习中要求你证伪另一个包含关系。

\begin{proposition}\label{prop:proposition7.3.12}
    设 $A,B$ 为集合,$f:A \to B$ 为函数。若 $Y \subseteq B$,则
    \[\im_f \left(\pim_f (Y)\right) \subseteq Y\]
\end{proposition}

\begin{proof}
    设 $b \in \im_f \left(\pim_f (Y)\right)$ 为任意固定元素。

    根据像的定义,这意味着 $\exists a \in \pim_f (Y) \centerdot f(a) = b$。给定这样的 $a$。

    因为 $a \in \pim_f (Y)$,根据原像的定义,这意味着 $f(a) \in Y$。

    因为 $b = f(a)$ 且 $f(a) \in Y$,这意味着 $b \in Y$。

    故命题得证。
\end{proof}
