% !TeX root = ../../../book.tex

\subsection{关于原像的证明}

请注意,以下命题是\textbf{相等}关系。将其与命题 \ref{prop:proposition7.3.6} 进行比较,后者是一个关于\emph{像}的类似陈述,但它只是集合包含关系。很有意思,对吧?

\begin{proposition}
    设 $A,B$ 为集合,$f:A \to B$ 为函数。设 $X, Y \subseteq B$,则
    \[PreIm_f (X \cap Y) = PreIm_f (X) \cap PreIm_f (Y)\]
\end{proposition}

注意下面的证明是如何直接运用原像的形式定义的。我们将直接开始证明这两部分。练习部分将要求你用 ``$cup$'' 替换 ``$cap$'' 来验证这个论断。

\begin{proof}
    设 $x \in PreIm_f (X \cap Y)$ 为任意固定元素。

    根据原像的定义,这意味着 $f(x) \in X \cap Y$,即 $f(x) \in X$ 且 $f(x) \in Y$。

    因为 $f(x) \in X$,根据原像的定义,这意味着 $x \in PreIm_f (X)$。同理,因为 $f(x) \in Y$,这意味着 $x \in PreIm_f (Y)$。

    根据交集的定义,我们推导出 $x \in PreIm_f(X) \cap PreIm_f(Y)$。

    这证明了 $PreIm_f(X \cap Y) \subseteq PreIm_f(X) \cap PreIm_f(Y)$。\\

    接着,设 $y \in PreIm_f(X) \cap PreIm_f(Y)$ 为任意固定元素。

    根据交集的定义,这意味着 $y \in PreIm_f (X)$ 且 $y \in PreIm_f (Y)$。

    因为 $y \in PreIm_f (X)$,根据原像的定义,我们可以推导出 $f(y) \in X$。同理,因为 $y \in PreIm_f (Y)$,我们可以推导出 $f(y) \in Y$。

    根据交集的定义,这告诉我们 $f(y) \in X \cap Y$。

    根据原像的定义,可得 $y \in PreIm_f(X \cap Y)$。

    这证明了 $ PreIm_f(X \cap Y) \supseteq PreIm_f(X) \cap PreIm_f(Y)$。 \\

    综上,利用双向包含论证,我们证明了该命题。
\end{proof}

读到这里你可能会想,``这种证明方法是怎么想出来的?''其实,这样的结果并没有什么特别的巧思。我们只是直接引用定义,然后一切就变得顺理成章了。如果你在解决问题时感到困惑,或者不知道从哪里开始……那就写下相关的定义,试着把它们应用到你要证明的陈述上,看看会有什么结果!

\subsubsection*{像和原像的证明}

让我们来看一个涉及本节中两个概念的例子。我们将证明一个包含关系,并在练习中要求你证明另一个包含关系不成立。

\begin{proposition}\label{prop:proposition7.3.12}
    设 $A,B$ 为集合,$f:A \to B$ 为函数。设 $Y \subseteq B$,则
    \[Im_f \big(PreIm_f (Y)\big) \subseteq Y\]
\end{proposition}

\begin{proof}
    设 $b \in Im_f \big(PreIm_f (Y)\big)$ 为任意固定元素。

    根据像的定义,这意味着 $\exists a \in PreIm_f (Y) \centerdot f(a) = b$。给定这样的 $a$。

    因为 $a \in PreIm_f (Y)$,根据原像的定义,这意味着 $f(a) \in Y$。

    因为 $b = f(a)$ 且 $f(a) \in Y$,这意味着 $b \in Y$。

    这就证明了该命题。
\end{proof}
