% !TeX root = ../../../book.tex

\subsection{示例}

让我们使用新定义来考察几个函数及非函数的例子。在分析这些例子时,我们将介绍函数的正确定义与表示方法,并描述如何通过``可视化''更好地理解某些函数。

\subsubsection*{表示法}

定义函数有多种正确方式。以下是定义``实数平方函数''的三种正确方法:
\begin{quotation}
    定义函数 $f \subseteq \mathbb{R} \times \mathbb{R}$ 为 $(x, y) \in f \iff y = x^2$。

    定义函数 $f : \mathbb{R} \to \mathbb{R}$ 为 $f=\left\{(x,x^2) \mid x \in \mathbb{R}\right\}$。

    定义函数 $f : \mathbb{R} \to \mathbb{R}$ 为 $\forall x \in \mathbb{R} \centerdot f(x) = x^2$。
\end{quotation}

思考以上每种定义如何符合前文给出的函数定义。
\begin{itemize}
    \item 第一种直接表明函数是 $\mathbb{R}$ 到 $\mathbb{R}$ 的\emph{关系},其本质是有序对的集合;
    \item 第二种通过集合构建符表示 $f$,避免使用\emph{当且仅当}语句;
    \item 第三种则强调每个输入 $x \in \mathbb{R}$ 都有\emph{唯一}一个``输出''。
\end{itemize}

我们\emph{通常}采用第三种表示法,因为它更容易理解且符合我们对函数的直观认识。特殊情况下也会选用其他表示法——例如需要强调函数底层结构或简化书写时。但定义函数时务必明确四个关键要素:\emph{定义域}、\emph{值域}、\emph{函数名}及\emph{映射规则}或\emph{集合}。

如果你还是不太理解为什么在定义函数时指定\emph{值域}如此重要,可以从计算机编程角度来思考。定义函数时通常需要\emph{声明}输出变量的数据类型(具体取决于编程语言)。以 \verb|Java| 为例:
\begin{verbatim}
    public int PlusOne (int x) {
        return x+1;
    }
\end{verbatim}
上面代码定义了一个函数 (\verb|PlusOne|),输入一个整数 (\verb|x|),加一后输出另一个整数。其中 \verb|x| 前的 \verb|int| 是输入的数据类型(整型),函数名 \verb|PlusOne| 前的 \verb|int| 是输出的数据类型(整型)。

\begin{example}
    考虑一个将自然数转换为其二进制表示的函数,记为 $B$。根据定义,$B(1) = 1, B(2) = 10, B(10) = 1010$。该函数的定义域是什么?值域是什么?能否严格写出其定义\emph{规则},还是用文字描述更为适宜?

    我们可以这样定义此函数:设 $S$ 为所有由 $0$ 和 $1$ 构成的有限二进制字符串的集合,进而定义函数 $B : \mathbb{N} \to S$ 为
    \[B = \{(n, s) \mid n \in \mathbb{N} \;\text{且}\; s \;\text{为}\; n \;\text{的二进制表示}\}\]
\end{example}

\begin{example}
    再次考察``平方函数'':设 $f : \mathbb{R} \to \mathbb{R}$ 定义为 $\forall x \in \mathbb{R}, f(x) = x^2$。此函数与下列函数是否相同?
    \begin{itemize}
        \item 设函数 $g : \mathbb{R} \to \mathbb{C}$ 定义为
            \[\forall x \in \mathbb{R} \centerdot g(x) = x^2\]
        \item 设函数 $h : \mathbb{Z} \to \mathbb{R}$ 定义为
            \[\forall x \in \mathbb{Z} \centerdot h(x) = x^2\]
    \end{itemize}

    函数 $g$ 的值域虽为复数集,但实际上 $\mathbb{R} \subseteq \mathbb{C}$,因此所有有序对 $(x, x^2) \in g$ 仍然满足 $x \in \mathbb{R}$ 且 $x^2 \in \mathbb{R}$。从这个角度看,$f$ 和 $g$ 可视作\emph{同一}函数,可记作 $f = g$。稍后,我们将详细探讨两个函数相等的确切含义。目前,只需说明 $f$ 和 $g$ 对应的底层关系具有相同的实数有序对即可。理论上,函数 $g$ \emph{允许}输出值为复数,但由于定义域和``规则''的设定,这实际上不会发生。

    函数 $h$ 的定义域 $\mathbb{Z} \subset \mathbb{R}$(是 $\mathbb{R}$ 的真子集)。因此,函数 $f$ 中的有许多有序对并不属于函数 $h$。例如,$(\frac{1}{2}, \frac{1}{4}) \in f$ 但 $(\frac{1}{2}, \frac{1}{4}) \notin h$。换句话说,$f(\frac{1}{2}) = \frac{1}{4}$,但 $h(\frac{1}{2})$ 未\emph{良好定义},因为 $\frac{1}{2}$ 不属于 $h$ 的定义域。
\end{example}

\begin{example}
    函数也可以\textbf{分段}定义。例如,考虑定义在 $\mathbb{R}$ 上的\emph{绝对值函数}:

    设函数 $a : \mathbb{R} \to \mathbb{R}$ 定义为
    \[\forall x \in \mathbb{R} \centerdot a(x) = 
    \begin{cases}
         x &\text{如果\ } x \ge 0 \\
        -x &\text{如果\ } x < 0
    \end{cases}\]

    定义域中的每个元素都\emph{恰好}落入某一种情况,因此没有歧义。
\end{example}

\subsubsection*{``良好定义''的函数}

给定定义域、值域和对应``规则''或集合,如何判定其是否构成函数?以下定义提供了明确的判断标准:

\begin{definition}\label{def:definition7.2.5}
    给定定义域 $A$、值域 $B$ 和``规则''$f$,称 $f$ 是\dotuline{良好定义的函数},当且仅当
    \begin{enumerate}[label=(\arabic*)]
        \item 规则对 $A$ 的所有元素均有定义;
        \item 对于任意 $a \in A$,规则都输出集合 $B$ 中的唯一一个元素。
    \end{enumerate}
\end{definition}

让我们用一个例子来阐释此概念。后面我们还会看到一些非函数的例子,并且会再次引用\textbf{良好定义的函数}的定义。

\begin{example}
    设函数 $a : \mathbb{Z} \to \mathbb{N}$ 定义为
    \[\forall z \in \mathbb{Z} \centerdot f(z) = |2z + 1|\]
    我们怎么确定对于任意\emph{整数} $z$,$|2z + 1|$ 均为\emph{自然数}?这并不显而易见,需要进行证明。

    假设 $z \in \mathbb{Z}$ 满足 $z \ge 1$。则 $2z+1 \in \mathbb{Z}$ 且 $|2z+1| = 2z+1 \ge 3$。因此 $f(x) \in \mathbb{N}$。

    假设 $z \in \mathbb{Z}$ 满足 $z \le -1$。则 $2z+1 \in \mathbb{Z}$ 且 $2z+1 \le -1$。故 $f(x) = |2z + 1| \ge 1 \in \mathbb{N}$。因此 $f(x) \in \mathbb{N}$。

    假设 $z = 0$。则 $f(z) = |2 \cdot 0 + 1| = 1$,因此 $f(z) \in \mathbb{N}$。

    无论哪种情况,我们发现定义函数 $f$ 的``规则''确实会生成一个自然数,该数是\emph{值域}中的元素。且只生成唯一一个数。因此,这是一个良好定义的函数。
\end{example}

\begin{example}
    设 $P$ 为世上所有人的集合。函数 $b : P \to \mathbb{N} \cup \{0\}$ 定义为
    \[b = \{(p, n) \mid p \in P \land p \text{\ 有\ } n \text{\ 个姐妹}\}\]
    (请注意,为了练习,我们这里使用了一种强调集合的符号风格。此外,将数学符号和文字杂糅在一起可能看起来有些奇怪,例如 ``$b(p) = p \text{\ 有多少个姐妹}$''。)

    这是一个良好定义的函数吗?我们认为是的。假如你走到某人面前(即某个元素 $p \in P$),问他有多少个姐妹(即 $b(p)$ 的值)。他会告诉你一个非负整数,并且他们不可能给出两个\emph{不同的}数字。

    你可能会指出,在今天这个离婚和再婚普遍的社会中,很多人有\emph{同父异母或同母异父的姐妹} (half-sisters),而 $\frac{1}{2} \notin \mathbb{N} \cup \{0\}$。这个观点有其道理。但是,在``简化假设''下,假设每个人都有\emph{整数数量}的姐妹,这个函数是良好定义的。
\end{example}

\subsubsection*{恒等函数}

设 $S$ 为任意集合。是否\emph{必然}存在一个从 $S$ 到 $S$ 的函数?我们可以想到许多从 $\mathbb{R}$ 到 $\mathbb{R}$ 的函数,但如果 $S$ 是任意集合呢?能否保证存在从 $S$ 到 $S$ 的函数?答案是肯定的!回想讨论关系时曾考虑的类似问题(参见示例 \ref{ex:example6.2.9})。我们知道在集合 $S$ 上总能定义\emph{等价关系}:例如通过 $(x, y) \in R \iff x = y$ 定义 $R$。该关系由所有形如 $(x, x)$ 的有序对组成,其中 $x \in S$。这个关系是否构成函数?只需验证定义性质:每个输入是否对应唯一输出?确实如此!集合中任意元素仅等于其自身。因此 $R$ 确实是一个函数,我们称之为恒等函数。

\begin{definition}[恒等函数]
    设 $S$ 为集合。$S$ 上的\dotuline{恒等函数} $\id : S \to S$ 定义为
    \[\forall x \in S \centerdot \id(x) = x\]
\end{definition}

也就是说,恒等函数的``输出与输入完全相同''。(可以将其想象为一部懒惰的机器,仅原样输出输入内容。)

当涉及\emph{不同}集合的恒等函数时,为避免混淆,使用 $\id_S$ 表示``\textbf{集合 $S$ 上的}恒等函数''。

\subsubsection*{非函数示例}

有时候,在解决问题时,我们可能会提出集合间的``规则''并质疑其是否为函数。那么规则何时不构成函数?回顾\textbf{良好定义的函数}的定义(见定义 \ref{def:definition7.2.5}),可能失败的情形有三种:

\begin{itemize}
    \item 可能存在某个定义域中的元素\textbf{没有对应的``输出''}。
    \item 可能存在某个定义域中的元素\textbf{有多个``输出''}。
    \item 可能存在某个定义域中的元素只有一个``输出'',但该``输出''\textbf{不在值域中}。
\end{itemize}
以下示例说明了这些情况。

\begin{example}
    设函数 $G : \mathbb{N} \times \mathbb{N} \to \mathbb{N}$ 定义为
    \[\forall (a, b) \in \mathbb{N} \times \mathbb{N} \centerdot G(a, b) = a - b\]

    这\textbf{不是}一个良好定义的函数,因为定义域中的多个元素其输出不在值域中。例如,$(5, 10) \in \mathbb{N} \times \mathbb{N}$,此时 $G(5, 10) = -5$,但 $-5 \notin \mathbb{N}$。
\end{example}

\clearpage

\begin{example}
    设 $W$ 为所有英语单词的集合。定义 $A: W \to W$ 为输入单词,输出原单词的异序词(即字母相同但顺序不同的单词)。
    
    这\textbf{不是}一个良好定义的函数。例如,HI 和 FUNCTION 不存在异序词。而像 INTEGRAL 这样的单词存在多个(即非唯一)异序词:TRIANGLE, ALERTING, ALTERING……
\end{example}