% !TeX root = ../../../book.tex

\subsection{习题}\label{sec:section7.5.4}

\subsubsection*{温故知新}

以口头或书面的形式简要回答以下问题。这些问题全都基于你刚刚阅读的内容,如果忘记了具体定义、概念或示例,可以回顾相关内容。确保在继续学习之前能够自信地作答这些问题,这将有助于你的理解和记忆!

\begin{enumerate}[label=(\arabic*)]
    \item 函数复合运算满足结合律吗?(即括号的顺序是否重要?)为什么?
    \item 函数复合运算满足交换律吗?(即运算顺序可否颠倒?)为什么?
    \item 设 $f : A \to B$ 和 $g : B \to A$ 为函数。如何证明 $g = f^{-1}$?
    \item 设 $f : A \to B$ 为双射。其反函数也是双射吗?
\end{enumerate}

\subsubsection*{小试牛刀}

尝试解答以下问题。这些题目需动笔书写或口头阐述答案,旨在帮助你熟练运用新概念、定义及符号。题目难度适中,确保掌握它们将大有裨益!

\begin{enumerate}[label=(\arabic*)]
    \item 设 $O$ 为奇数集,$E$ 为偶数集。构造一个\textbf{双射}函数 $f : O \to E$,并通过求其反函数证明 $f$ 为双射。
    \item 本题中,我们希望你能通过一个例子来说明,寻求函数的反函数时,验证两个复合函数\textbf{都能}得到恒等函数至关重要。\\
    定义集合 $A, B$ 及函数 $f : A \to B, g : B \to A$,使得
    \[\forall x \in A \centerdot g(f(x)) = x\]
    然而
    \[\exists y \in B \centerdot f(g(y)) \ne y\]
    (\textbf{建议}:可以尝试 $A$ 和 $B$ 仅含一到两个元素的例子……或者取 $A=B=\mathbb{N}$。)\label{exc:exercises7.5.2}
    \item 设 $U = \{y \in \mathbb{R} \mid -1 < y < 1\}, I = \{y \in \mathbb{R} \mid -6 < y < 12\}$。\\
        定义函数 $g : U \to I$ 为 $\forall x \in U \centerdot g(x) = 9x + 3$。
        证明 $g$ 是双射,并求 $g^{-1}$。
    \item 定义函数 $f : \mathbb{Z} \to \mathbb{N}$ 为
    \[\forall z \in \mathbb{Z} \centerdot f(z) = \begin{cases}
        -2z + 2 & \text{若\ } z \le 0 \\
        2z-1 & \text{若\ } z > 0
    \end{cases}\]
    证明 $f$ 是双射,并求 $f^-1$。\\
    (\textbf{提示}:反函数需分段定义,证明时请细致处理不同情形。)
    \item \textbf{挑战性问题}:定义 $I = \{y \in \mathbb{R} \mid -1 < y < 1\}$。构造双射函数 $f : I \to \mathbb{R}$ 并证明之。\\
    (\textbf{提示}:无需使用三角函数。可以考虑在表达式中使用 $|x|$……)\label{exc:exercises7.5.5}
\end{enumerate}