% !TeX root = ../../../book.tex

\subsection{习题}\label{sec:section7.5.4}

\subsubsection*{温故知新}

以口头或书面的形式简要回答以下问题。这些问题全都基于你刚刚阅读的内容,所以如果忘记了具体的定义、概念或示例,可以回去重读相关部分。确保在继续学习之前能够自信地回答这些问题,这将有助于你的理解和记忆!

\begin{enumerate}[label=(\arabic*)]
    \item 函数复合运算满足结合律吗?(也就是说,括号的顺序是否重要?)为什么满足或者为什么不满足?
    \item 函数复合运算满足交换律吗?(也就是说,我们可以颠倒顺序进行运算吗?)为什么满足或者为什么不满足?
    \item 假设 $f : A \to B$ 和 $g : B \to A$ 为函数。我们如何证明 $g = f^{-1}$?
    \item 假设 $f : A \to B$ 为双射。它的反函数也是双射吗?
\end{enumerate}

\subsubsection*{小试牛刀}

尝试回答以下问题。这些题目要求你实际动笔写下答案,或(对朋友/同学)口头陈述答案。目的是帮助你练习使用新的概念、定义和符号。题目都比较简单,确保能够解决这些问题将对你大有帮助!

\begin{enumerate}[label=(\arabic*)]
    \item 设 $O$ 为奇数集,设 $E$ 为偶数集。定义\textbf{双射}函数 $f : O \to E$,并通过找到它的反函数证明 $f$ 确实为双射。
    \item 在这个问题中,我们希望你能通过一个例子来展示,当我们尝试找到一个函数的反函数时,验证两个复合函数\textbf{都能}得到恒等函数是多么重要。\\
    定义集合 $A, B$ 以及函数 $f : A \to B, g : B \to A$ 使得
    \[\forall x \in A \centerdot g(f(x)) = x\]
    但
    \[\exists y \in B \centerdot f(g(y)) \ne y\]
    (\textbf{建议}:你可以找一个例子,其中 $A$ 和 $B$ 都只有一到两个元素……或者,你可以找一个例子,其中 $A=B=\mathbb{N}$。)
    \item 设 $U = \{y \in \mathbb{R} \mid -1 < y < 1\}, I = \{y \in \mathbb{R} \mid -6 < y < 12\}$。\\
        设函数 $g : U \to I$ 定义为 $\forall x \in U \centerdot g(x) = 9x + 3$。
        证明 $g$ 是双射,并找到 $g^{-1}$。
    \item 定义函数 $f : \mathbb{Z} \to \mathbb{N}$ 为
    \[\forall z \in \mathbb{Z} \centerdot f(z) = \begin{cases}
        -2z + 2 & \text{如果}\; z \le 0 \\
        2z-1 & \text{如果}\; z > 0
    \end{cases}\]
    证明 $f$ 是双射,并找到 $f^-1$。\\
    (\textbf{提示}:你给出的反函数也是分段定义的。在你的证明中要小心处理可能出现的各种情况。)
    \item \textbf{挑战性问题}:定义 $I = \{y \in \mathbb{R} \mid -1 < y < 1\}$。找到函数 $f : I \to \mathbb{R}$ 是双射函数,并证明之。\\
    (\textbf{提示}:无需使用任何三角函数。可以考虑在表达式中使用 $|x|$……)
\end{enumerate}