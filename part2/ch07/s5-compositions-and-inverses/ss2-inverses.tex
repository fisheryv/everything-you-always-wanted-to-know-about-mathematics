% !TeX root = ../../../book.tex

\subsection{反函数}

\subsubsection*{引言}

此前我们提到,\textbf{双射}函数 $f: A \to B$ 具有一个重要性质:它能在集合 $A$ 和 $B$ 的元素之间建立``一一对应''关系。对于任意 $a \in A$,\emph{有且只有}一个 $b \in B$ 满足 $f(a) = b$,此乃良好定义的函数的要求;同时,由于 $f$ 是双射,$a$ 也是\emph{唯一一个}与 $b$ 关联的元素。基于这种双向唯一性,我们可以``逆转'' $f$。也就是说,给定任意 $b \in B$,存在唯一一个 $a \in A$ 使得 $f(a) = b$,这就是\textbf{反函数}。本节将通过\emph{函数复合}和\emph{恒等函数}形式化这一概念。这也说明双射关系存在于两个集合之间,而非单向;一旦获得一个方向的双射,必然存在反向的双射。

在正式定义之前,我们简要回顾一下\textbf{恒等函数}的定义,它在反函数的定义中扮演至关重要的角色。
\begin{center}
    给定集合 $X$,\textbf{恒等函数} $\id_X: X \to X$ 定义为 $\forall z \in X \centerdot \id_X(z) = z$。
\end{center}

\subsubsection*{定义}

请注意,以下定义未预设函数为双射,仅形式化反函数的含义。反函数与双射的关系将在后续证明。

\begin{definition}
    设 $f:A \to B$ 为函数。若存在函数 $g:B \to A$ 使得 
    \[g \circ f : A \to A \text{\ 满足\ } g \circ f = \id_A\] 
    \[f \circ g : B \to B \text{\ 满足\ } f \circ g = \id_B\]
    则称 $g$ 是 $f$ 的\dotuline{反函数},记作 $g = f^{-1}$。
\end{definition}

(请注意,上述定义中的假设和结论隐含 $B = \im_f(A)$ 以确保 $g$ 是良好定义的函数;同理,$A = \im_g(B)$。)

\subsubsection*{示例}

回顾此前讨论双射时涉及的函数。通过练习,我们已经证明该函数是双射,下面将求其反函数。\\

\begin{example}
    定义函数 $h : \mathbb{R} - \{-1\} \to \mathbb{R}- \{1\}$ 为
    \[\forall x \in \mathbb{R} - \{-1\} \centerdot h(x) = \frac{x}{1+x}\]

    为了找到 $h$ 的候选反函数,通常可以将 $h$ 的``规则''等同于一个新变量,然后求解 $x$。

    这里,我们令 $h(x) = y$。那么我们如何``逆转''这个过程,并用 $y$ 表示 $x$ 呢?我们可以通过以下代数步骤来实现:
    \begin{align*}
        h(x) = y &\iff \frac{x}{1+x} = y \\
        &\iff (1 + x)y = x \\
        &\iff xy + y = x \\
        &\iff y = x(1 - y) \\
        &\iff x = \frac{y}{1-y}
    \end{align*}

    上述推导得到 $h$ 的候选反函数。需注意:此过程并非严格证明。为了严谨表述,我们定义新函数 $H$ 并证明 $H = h^{-1}$。若直接使用 $h^{-1}$ 将导致循环论证,因为反函数存在性尚待证明,不能在证明开始时就假定反函数存在。

    \begin{proof}
        为了简便,我们定义 $S = \mathbb{R} - \{-1\}, T = \mathbb{R}- \{1\}$,则 $h:S \to T$。

        设函数 $H: T \to S$ 定义为 $\forall y \in T \centerdot H(y) = \frac{y}{1-y}$。

        首先,证明 $H$ 是良好定义的函数。对于每个 $y \in T$,都有 $y \ne 1$,所以 $1-y \ne 0$。因此分数 $\frac{y}{1-y}$ 是良好定义的实数。

        此外,还可以推导出 $\frac{y}{1-y} \ne -1$。为了引出矛盾而假设 $\frac{y}{1-y} = -1$,等式两边同时乘以 $1-y$ 得 $y = y-1$,这显然是矛盾的。\\

        接着,证明 $H \circ h = \id_S$。给定 $x \in S$,可得
        \begin{align*}
            (H \circ h)(x) &= H(h(x)) = H\Big(\frac{x}{1+x}\Big) \\
            &=\frac{\frac{x}{1+x}}{1-\frac{x}{1+x}} \cdot \frac{1+x}{1+x} = \frac{x}{(1+x)-x} \\
            &=\frac{x}{1} = x\\
        \end{align*}

        最后,证明 $h \circ H = \id_T$。给定 $y \in T$,可得
        \begin{align*}
            (h \circ H)(y) &= h(H(y)) = h\Big(\frac{y}{1-y}\Big) \\
            &=\frac{\frac{y}{1-y}}{1+\frac{y}{1-y}} \cdot \frac{1-y}{1-y} = \frac{y}{(1+y)-y} \\
            &=\frac{y}{1} = y
        \end{align*}

        综上,根据反函数的定义,$H = h^{-1}$。
    \end{proof}
\end{example}

\subsubsection*{双向检查}

若称函数 $f: A \to B$ 存在反函数 $g: B \to A$,则\textbf{关键}在于必须证明两个复合函数均等于恒等函数;也就是说,必须证明
\[f \circ g = \id_B \quad \text{且} \quad g \circ f = \id_A\]

初学者可能会忽略此验证,或不解其必要性。为了加深理解,\ref{sec:section7.5.4} 节的练习 \ref{exc:exercises7.5.2} 要求构造一个反例,其中``一个方向''的复合可得恒等函数,但``另一方向''的复合却无法得到恒等函数,此时函数并非真正意义上的\emph{反函数}。建议尝试构造多个此种类型的实例,以深刻理解双向验证的重要性。