% !TeX root = ../../../book.tex

\subsection{反函数}

\subsubsection*{引言}

我们之前提到过,\textbf{双射}函数 $f: A \to B$ 有一个很好的性质,即 $f$ 可以将集合 $A$ 和 $B$ 的元素``一一对应''。对于每一个 $a \in A$,\emph{有且只有}一个 $b \in B$ 满足 $f(a) = b$。这是因为 $f$ 是一个良好定义的函数。同时,我们也知道 $a$ 是\emph{唯一一个}以这种方式与 $b$ 关联的元素,因为 $f$ 是一个双射。由于这种双向的唯一对应关系,我们可以``逆转'' $f$。也就是说,给定 $b \in B$,我们可以找到唯一一个 $a \in A$ 使得 $f(a) = b$。这就是\textbf{反函数}。在这里,我们将通过\emph{函数复合}和\emph{恒等函数}来定义它。这也是为什么我们说双射存在于两个集合之间,而不仅仅是从一个集合到另一个集合;一旦我们有了一个方向的双射,我们也知道可以有相反方向的双射。

在介绍正式定义之前,我们先快速回顾一下之前学过的\textbf{恒等函数}的定义。恒等函数在接下来的反函数的定义中扮演着重要角色。
\begin{quotation}
    \textbf{定义}:给定集合 $X$,\textbf{恒等函数} $Id_X: X \to X$ 定义为 $\forall z \in X \centerdot Id_X(z) = z$。
\end{quotation}

\subsubsection*{定义}

请注意,这一定义并没有提到函数是双射。这只是一个关于反函数含义的正式定义。之后,我们需要证明反函数和双射之间的关系。

\begin{definition}
    设 $f:A \to B$ 为函数。假设存在函数 $g:B \to A$ 使得 $g \circ f : A \to A$ 满足 $g \circ f = Id_A$ 且 $f \circ g : B \to B$ 满足 $f \circ g = Id_B$。

    则我们说 $g$ 是 $f$ 的\dotuline{反函数},写做 $g = f^{-1}$。
\end{definition}

(请注意,上述定义中的假设和结论隐含了一些条件。具体来说,为了确保 $g$ 是一个函数,必须满足 $B = Im_f (A)$。同理,$A = Im_g(B)$。)

\subsubsection*{示例}

让我们回顾一下之前在讨论双射时提到的一个函数。在练习中,在你的帮助下,我们已经知道这个函数是一个双射。现在,我们将找到它的反函数。\\

\begin{example}
    定义函数 $h : \mathbb{R} - \{-1\} \to \mathbb{R}- \{1\}$ 为
    \[\forall x \in \mathbb{R} - \{-1\} \centerdot h(x) = \frac{x}{1+x}\]

    为了找到 $h$ 的反函数的候选函数,通常我们可以将 $h$ 的``规则''等同于一个新变量,然后求解 $x$。

    这里,我们令 $h(x) = y$。那么我们如何``逆转''这个过程,并用 $y$ 表示 $x$ 呢?我们可以通过以下代数步骤来实现:

    \begin{align*}
        h(x) = y &\iff \frac{x}{1+x} = y \\
        &\iff (1 + x)y = x \\
        &\iff xy + y = x \\
        &\iff y = x(1 - y) \\
        &\iff x = \frac{y}{1-y}
    \end{align*}

    这段演算为我们找到了 $h$ 的反函数的候选者。注意,这些观察并没有证明任何东西!我们现在需要提出一个声明,并向读者展示所有基本事实。请注意,我们谨慎地定义了一个新函数 $H$,并用它证明了 $H = h^{-1}$。直接定义 $h^-1$ 并使用它是不合理且错误的。我们是要证明 $h$ 有反函数,所以不能在证明开始时就假定它有反函数。

    \begin{proof}
        为了简便,我们定义 $S = \mathbb{R} - \{-1\}, T = \mathbb{R}- \{1\}$,所以 $h:S \to T$。

        设函数 $H: T \to S$ 定义为 $\forall y \in T \centerdot H(y) = \frac{y}{1-y}$。

        首先,我们证明 $H$ 是良好定义的函数。对于每个 $y \in T$,我们知道 $y \ne 1$,所以 $1-y \ne 0$。因此分数 $\frac{y}{1-y}$ 是良好定义的实数。

        此外,我们可以推导出 $\frac{y}{1-y} \ne -1$。为了引出矛盾而假设 $\frac{y}{1-y} = -1$,等式两边同时乘以 $1-y$ 得 $y = y-1$,这显然是矛盾的。

        接着,我们证明 $H \circ h = Id_S$。给定 $x \in S$,可得
        \begin{align*}
            (H \circ h)(x) &= H(h(x)) = H\Big(\frac{x}{1+x}\Big) \\
            &=\frac{\frac{x}{1+x}}{1-\frac{x}{1+x}} \cdot \frac{1+x}{1+x} = \frac{x}{(1+x)-x} \\
            &=\frac{x}{1} = x
        \end{align*}

        最后,我们证明 $h \circ H = Id_T$。给定 $y \in T$,可得
        \begin{align*}
            (h \circ H)(y) &= h(H(y)) = h\Big(\frac{y}{1-y}\Big) \\
            &=\frac{\frac{y}{1-y}}{1+\frac{y}{1-y}} \cdot \frac{1-y}{1-y} = \frac{y}{(1+y)-y} \\
            &=\frac{y}{1} = y
        \end{align*}

        因此,根据反函数的定义,$H = h^{-1}$。
    \end{proof}
\end{example}

\subsubsection*{双向检查}

我们声称函数 $f: A \to B$ 有反函数 $g: B \to A$。\textbf{至关重要}的一点是你需要证明两个复合函数\textbf{都}等于恒等函数;也就是说,你必须证明
\[f \circ g = Id_B \quad \text{且} \quad g \circ f = Id_A\]
有时候你可能会忘记进行这个验证,或者不明白为什么这是必要的。为了帮助你理解这一点,我们在下面的 \ref{sec:section7.5.4} 节中提供了练习 \ref{exc:exercises7.5.2}。它要求你找到一个例子,其中``一个方向''可得恒等函数,但``另一个方向''无法得到恒等函数,所以给出的函数实际上不是\emph{反函数}。如果可以的话,尽量多找几个例子。这样可以更好地强调这个问题的重要性。