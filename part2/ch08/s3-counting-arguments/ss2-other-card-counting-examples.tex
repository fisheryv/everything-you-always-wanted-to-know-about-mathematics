% !TeX root = ../../../book.tex

\subsection{其他计牌示例}

让我们通过一些相关的例子来扩展我们所应用的技术类别。\\

\begin{example}[至少三张 $A$]

    对于这个例子,让我们计算 $5$ 张手牌至少有三张 $A$ 的牌型数量。同理,让我们应用上面使用的方法,思考这种牌型的基本特征。试着自己想几个问题,目的是通过答案确定唯一的牌型,并且给定任何答案,我们可以准确地计算出有多少种方法可以构建出该答案所述的牌型。

    你意识到困难了吗?其中一个问题的答案会\emph{直接影响}其他问题的性质!这表明其中存在某些更深层次的数学问题。一个合理的做法也许是首先明确这个问题,然后再考虑围绕这个问题必须做出什么决定。

    首先,如果手中恰好有 $3$ 张 $A$,那么我们需要确定另外两张牌的特性。这两张牌要么是 (a) 不同点数的,要么是 (b) 相同点数的。因此,这种特殊情况可以分为两种子情况。由此我们可以得到以下步骤:
    \begin{enumerate}
        \item 从 $4$ 张 $A$ 中选择 $3$:有 $\big({4 \atop 3}\big)$ 种方法
              \begin{enumerate}[label=(\alph*)]
                  \item 剩下的两张牌点数不同:
                        \begin{enumerate}[label=(\roman*)]
                            \item 从剩下的 $12$ 个点数中,为剩下的两张牌选择两个点数:有 $\big({12 \atop 2}\big)$ 种方法
                            \item 为点数较小的牌选择一种花色:有 $\big({4 \atop 1}\big)$ 种方法
                            \item 为点数较大的牌选择一种花色:有 $\big({4 \atop 1}\big)$ 种方法
                        \end{enumerate}
                  \item 剩下的两张牌点数相同:
                        \begin{enumerate}[label=(\roman*)]
                            \item 从剩下的 $12$ 个点数中选择一个点数:有 $\big({12 \atop 1}\big)$ 种方法
                            \item 为选出的点数选择两种花色:有 $\big({4 \atop 2}\big)$ 种方法
                        \end{enumerate}
              \end{enumerate}
    \end{enumerate}
    所以,根据乘法原理和加法原理(因为过程中与独立的两种情况),我们得到
    \[\begin{pmatrix}
            4 \\
            3
        \end{pmatrix}\Bigg[\begin{pmatrix}
                12 \\
                2
            \end{pmatrix} \begin{pmatrix}
                4 \\
                1
            \end{pmatrix}^2 + \begin{pmatrix}
                12 \\
                1
            \end{pmatrix} \begin{pmatrix}
                4 \\
                2
            \end{pmatrix}\Bigg]\]
    种\emph{恰好有三张} $A$ 的牌型。\\

    其次,如果手中正好有 $4$ 张 $A$,我们需要确定第五张牌的特性。具体步骤如下
    \begin{enumerate}
        \item 从 $4$ 张 $A$ 中选择 $4$ 张:有 $\big({4 \atop 4}\big)$ 种方法
        \item 从剩下的 $12$ 个点数中,为剩下的一张牌选择一个点数:有 $\big({12 \atop 1}\big)$ 种方法
        \item 为剩下的一张牌选择一种花色:$\big({4 \atop 1}\big)$ 种方法
    \end{enumerate}
    利用乘法原理,我们得到
    \[\begin{pmatrix}
            4 \\
            4
        \end{pmatrix}\begin{pmatrix}
            12 \\
            1
        \end{pmatrix}\begin{pmatrix}
            4 \\
            1
        \end{pmatrix}\]
    种\emph{恰好有四张} $A$ 的牌型。

    现在,我们必须应用加法原理!我们将所需的牌型 --- 至少有三张 $A$ 的牌型 --- 划分成两个子集:一个是恰好有三张 $A$ 的牌型,另一个是恰好有四张 $A$ 的牌型。由于这些子集划分了较大的集合(即每一种至少有三张 $A$ 的牌型要么有三张 $A$,要么有四张 $A$,不会同时有三张和四张 $A$,也不会都没有),我们可以应用加法原理并得出结论,有
    \[\begin{pmatrix}
            4 \\
            3
        \end{pmatrix}\Bigg[\begin{pmatrix}
                12 \\
                2
            \end{pmatrix} \begin{pmatrix}
                4 \\
                1
            \end{pmatrix}^2 + \begin{pmatrix}
                12 \\
                1
            \end{pmatrix} \begin{pmatrix}
                4 \\
                2
            \end{pmatrix}\Bigg] + \begin{pmatrix}
            4 \\
            4
        \end{pmatrix}\begin{pmatrix}
            12 \\
            1
        \end{pmatrix}\begin{pmatrix}
            4 \\
            1
        \end{pmatrix}\]
    种至少有三张 $A$ 的牌型。

    回顾一下,之前提到的加法原理的严格陈述,其中涉及有限集的基数,但在之前的例子中我们没有深入其中的细节。这类组合数学的论证需要一定的判断力和技巧。你能轻松地理解\emph{至少有三张 $A$ 的牌型,要么正好有三张 $A$,要么正好有四张 $A$,而不会同时有三张和四张 $A$,也不会两者都没有吗}?我们并不是说这应该是完全显而易见的,如果你没有立即看到这一点也没关系!我们想表达的是,这类陈述在证明中应该能作为合理的解释。是的,我们可以深入细节,用集合的术语重新表述扑克牌牌型,并将扑克牌游戏完全用集合符号表示。但这样做有什么意义呢?上面楷体字的解释似乎更容易理解。如果有困惑的读者需要更详细的解释,我们当然可以提供,但对于一般读者来说,这样的论证已经足够了。希望这个经验法则 --- 说服一般读者,但在受到进一步质疑时能够进一步解释 --- 能指导你决策在计数论证中需要包含多少细节。这里重要的是,我们解释了为什么我们的选择与牌型集合的划分有关。虽然我们没有严格证明这两个集合是不相交的,但我们提供了理由。

    我们可以用另一种方法来解决这个问题,这种方法不需要考虑非 $A$ 牌的花色。具体来说,我们可以按以下步骤来构建一副至少包含 $3$ 张 $A$ 的扑克手牌:
    \begin{enumerate}
        \item 如果恰有 $3$ 张 $A$:
              \begin{enumerate}[label=(\alph*)]
                  \item 为 $3$ 张 $A$ 选择三种花色:有 $\big({4 \atop 3}\big)$ 种方法
                  \item 从剩余的 $48$ 张非 $A$ 牌中选择 $2$ 张凑齐 $5$ 张牌:有 $\big({48 \atop 2}\big)$ 种方法
              \end{enumerate}
        \item 如果恰有 $4$ 张 $A$:
              \begin{enumerate}[label=(\alph*)]
                  \item 为 $4$ 张 $A$ 选择四种花色:有 $\big({4 \atop 4}\big) = 1$ 种方法
                  \item 从剩余的 $48$ 张非 $A$ 牌中选择 $1$ 张凑齐 $5$ 张牌:有 $\big({48 \atop 1}\big)$ 种方法
              \end{enumerate}
    \end{enumerate}
    因此,根据加法原理(我们根据牌型中有多少张 $A$ 划分成两种情况)以及每种情况下的乘法原理,可得
    \[\begin{pmatrix}
            4 \\
            3
        \end{pmatrix}\begin{pmatrix}
            48 \\
            2
        \end{pmatrix}+\begin{pmatrix}
            4 \\
            4
        \end{pmatrix}\begin{pmatrix}
            48 \\
            1
        \end{pmatrix}\]
    种至少有三张 $A$ 的牌型。

    你会更频繁地看到(并使用)这种方法。前面的论证更类似于之前同花的例子,所以我们先介绍了同花。这种论证稍微简短一些,也更``巧妙'',因此更常用。但等一下,这些答案形式上看起来不一样!我们计算的是同一牌型数量,难道不应该得到相同的最终结果吗?其实结果是相同的,我们建议你做一下必要的代数运算来验证这一点
    \[\begin{pmatrix}
            4 \\
            3
        \end{pmatrix}\begin{pmatrix}
            48 \\
            2
        \end{pmatrix}+\begin{pmatrix}
            4 \\
            4
        \end{pmatrix}\begin{pmatrix}
            48 \\
            1
        \end{pmatrix} = \begin{pmatrix}
            4 \\
            3
        \end{pmatrix}\Bigg[\begin{pmatrix}
                12 \\
                2
            \end{pmatrix} \begin{pmatrix}
                4 \\
                1
            \end{pmatrix}^2 + \begin{pmatrix}
                12 \\
                1
            \end{pmatrix} \begin{pmatrix}
                4 \\
                2
            \end{pmatrix}\Bigg] + \begin{pmatrix}
            4 \\
            4
        \end{pmatrix}\begin{pmatrix}
            12 \\
            1
        \end{pmatrix}\begin{pmatrix}
            4 \\
            1
        \end{pmatrix}\]
    这只需要一分钟,而且很值得一试。
\end{example}

在继续下一个问题之前,我们先来看看这个问题的一个\emph{错误论证}。虽然看错误答案可能有些奇怪,但经验告诉我们,找出错误论证中的\emph{缺陷}是非常有帮助和启发性的。我们当然可以简单地比较两个大整数,然后说:``看,它们不一样!''但这种做法并没有什么启发性。相反,我们希望通过一个组合论证,找出导致逻辑错误或错误计数的步骤。我们强烈推荐这种方法,原因如下。首先,它能让你更好地练习阅读证明和理解他人的论证。这在你学习更多数学知识和阅读其他书籍时会非常有帮助。其次,它能帮助你更好地审视自己的证明。在完成作业后,先搁置一会儿,然后再用清醒的头脑重新阅读。尽量像你从没写过它一样去阅读(我们知道这很难做到!)。它是否有合理?一些当时看似显而易见的步骤现在是否让你摸不着头脑?答案是否正确,你是否被它说服?第三,识别出证明中的错误步骤能巩固你对论证基础原则的理解。通过组合论证并识别错误,将真正帮助你理解并掌握加法原理和乘法原理。相信我们。

你对下面的论证有何看法?记住,这个答案是\emph{不正确的},我们想知道为什么!\\

\begin{example}[找出缺陷]

    \textcolor{red}{
        至少有三张 $A$ 的牌型有多少种?
        \begin{enumerate}[label=(\arabic*)]
            \item 如果恰有 $3$ 张 $A$:
                  \begin{enumerate}[label=(\alph*)]
                      \item 为 $4$ 张 $A$ 中选择 $3$ 张:有 $\big({4 \atop 3}\big)$ 种方法
                      \item 从剩余的 $49$ 张牌中选择 $2$ 张凑齐 $5$ 张牌:有 $\big({49 \atop 2}\big)$ 种方法
                  \end{enumerate}
            \item 如果恰有 $4$ 张 $A$:
                  \begin{enumerate}[label=(\alph*)]
                      \item 为 $4$ 张 $A$ 选择 $4$ 张:有 $\big({4 \atop 4}\big) = 1$ 种方法
                      \item 从剩余的 $48$ 张牌中选择 $1$ 张凑齐 $5$ 张牌:有 $\big({48 \atop 1}\big)$ 种方法
                  \end{enumerate}
        \end{enumerate}
        因此,有
        \[\begin{pmatrix}
                4 \\
                3
            \end{pmatrix}\begin{pmatrix}
                49 \\
                2
            \end{pmatrix}+\begin{pmatrix}
                4 \\
                4
            \end{pmatrix}\begin{pmatrix}
                48 \\
                1
            \end{pmatrix}\]
        种至少有三张 $A$ 的牌型。
    }

    这里有什么问题?你看出任何错误了吗?乘法原理是否被错误地应用了?加法原理是否应用到了不该使用的地方?我们是否多算了?还是少算了?我们是否计算了一些不符合要求的牌型?在继续阅读之前,请思考这些问题。

    我们注意到:这个答案\emph{太大了}。我们\emph{多算了},因为某些牌型在我们的计算中被重复计数了。也就是说,我们要计算的每一种牌型至少被上述步骤包括了一次,但有些牌型可以通过这些步骤以多种方式构建。这些观察让我们确定这里的数字太大了。

    我们是怎么知道的呢?我们建议积极尝试找出可以通过上述步骤以两种不同方式构建的牌型。如果你在阅读证明时能够意识到这一点,你就知道整个证明是有缺陷的。在这种情况下,让我们检查恰好 $4$ 张 $A$ 的牌型;具体来说,让我们看一下 $A \clubsuit \; A\spadesuit \;A\diamondsuit \;A\heartsuit\;2\clubsuit$。我们可以通过以下路径构建此牌型:
    \begin{enumerate}
        \item 选择 $4$ 张 $A$ 中的 $3$ 张:$A \clubsuit \; A\spadesuit \;A\diamondsuit$
        \item 从剩余的 $49$ 张牌中再选择 $2$ 张:$A\heartsuit\;2\clubsuit$
    \end{enumerate}
    或者,我们可以选择这条路径:
    \begin{enumerate}
        \item 选择 $4$ 张 $A$ 中的 $4$ 张:$A \clubsuit \; A\spadesuit \;A\diamondsuit\;A\heartsuit$
        \item 从剩余的 $48$ 张牌中再选择 $1$ 张:$2\clubsuit$
    \end{enumerate}
    你现在看出问题了吗?通过上述过程,这手完全相同的牌至少可以通过两种不同的方式构建。因此,答案是多算了。还有其他方式可以构建这手牌吗?有多少?尝试找出另一牌型被多算的构建方法。我们能否确定每手牌被多算的次数,并以此修正我们的答案呢?这是一个有趣且非常具有挑战性的问题,我们稍后会回到这个问题上来。
\end{example}

\subsubsection*{论证中的潜在错误}

我们现在要重点介绍如何阅读组合证明,并识别其中一些常见的错误:
\begin{itemize}
    \item \textbf{误用乘法原理}\\
          证明在不适用乘法原理的情况下错误地使用了乘法原理。可能是因为每一步的选择数量会随着之前步骤的完成情况而变化,或者是不同的步骤顺序会产生相同的结果。
    \item \textbf{误用加法原理}\\
          证明在不适用加法原理的情况下错误地使用了加法原理。可能是集合的``划分''实际上并不是互不相交的。也可能是这些集合``划分''的并集并没有覆盖整个讨论的集合。
    \item \textbf{多算}\\
          每个所需的对象至少被计算了一次,但有些对象被计算了多次。换句话说,所讨论集合中的一些元素可以通过证明步骤以多种方式被计算。
    \item \textbf{少算}\\
          一些所需的对象没有包括在计算中。换句话说,证明的步骤遗漏了所讨论集合中的一些元素。
    \item \textbf{误算}\\
          一些不相关的对象被包括在计算中。换句话说,证明的步骤计算了一些不属于讨论集合的对象。
\end{itemize}
我们建议你仔细阅读你的书面证明,并尝试找出其中可能存在的缺陷,即使这些缺陷可能并不存在。例如,你可以尝试寻找一个多算的论证,尝试通过不同方法构建某些对象,这样的过程可能会帮助你发现一些你之前未曾注意到的错误。如果你没有发现任何缺陷,那么你可以更加确信你的证明是完全正确的。\\

\begin{example}
    这是一个典型的\textbf{多算}例子。我们将展示它是如何多算的,并通过不同的计数方法来修正它!问题如下:

    $5$ 张手牌中每种花色各一张的牌型有多少?

    以下是一个\textcolor{red}{错误的}论证:

    \begin{quote}\color{red}
        有 $\begin{pmatrix}
                13 \\
                1
            \end{pmatrix}^4 \cdot \begin{pmatrix}
                48 \\
                1
            \end{pmatrix}$ 种这样的牌型。

        我们可以用一个五步过程来构建此牌型。第一步,从 $13$ 个红桃中选择一张;第二步,从 $13$ 个方片中选择一张;第三步,从 $13$ 个黑桃中选择一张;第四步,从 $13$ 个梅花中选择一张。以上每一步都有 $\big({13 \atop 1}\big)$ 种选择方式。

        然后,从剩余的 $48$ 张牌种选择一张凑成 $5$ 张手牌,有 $\big({48 \atop 1}\big)$ 种选择方式。根据乘法原理,即可得到上面的结论。
    \end{quote}


    这有什么问题吗?在继续阅读之前,仔细思考一下。看看上面列出的潜在错误列表;其中有没有适用于这里的?你会如何证明这一点?

    我们认为这是一个\textbf{多算}例子。为了证明这一点,我们将展示一个特定的 $5$ 张牌手牌,这手牌应该只被计数一次,但实际上根据上述论证的程序至少被计数了两次。

    考虑手牌 $A\heartsuit, A\diamondsuit, A\spadesuit, A\clubsuit, K\heartsuit$。注意,这手牌可以通过上述过程以两种方式实现:
    \begin{enumerate}[label=(\arabic*)]
        \item 第一步:选择 $A\heartsuit$。第二步:选择 $A\diamondsuit$。第二步:选择 $A\spadesuit$。第四步:选择 $A\clubsuit$。第五步:选择 $K\heartsuit$。
        \item 第一步:选择 $K\heartsuit$。第二步:选择 $A\diamondsuit$。第二步:选择 $A\spadesuit$。第四步:选择 $A\clubsuit$。第五步:选择 $A\heartsuit$。
    \end{enumerate}
    由于手牌是\emph{无序的},这两种方法会得到\emph{相同的结果}。然而,上述论证会将这两个结果分别计算。因此,这种论证存在多算的问题。

    为了修正这个论证,让我们仔细考虑一下每种花色需要出现的\textbf{次数}。在有 $5$ 张牌的情况下,而花色只有 $4$ 种,这意味着要求每种花色至少出现一次时,必然有三种花色各出现一次,而另一种花色出现两次。换句话说,花色的\emph{分布}必须是 $(1, 1, 1, 2)$。

    为了计数牌型的数量,我们定义如下过程:
    \begin{itemize}
        \item 从四种花色种选择哪种花色出现两次(其余三种花色固定只出现一次。)\\
              有 $\big({4 \atop 1}\big)$ 种方式。
        \item 为选出的花色选择两张牌。\\
              有 $\big({13 \atop 2}\big)$ 种方式。
        \item 剩下的三个花色,每个花色选择一张牌。\\
              有 $\big({13 \atop 1}\big)^3$ 种方式。
    \end{itemize}
    根据乘法原理,我们有
    \[\begin{pmatrix}
            4 \\
            1
        \end{pmatrix}\begin{pmatrix}
            13 \\
            2
        \end{pmatrix}\begin{pmatrix}
            13 \\
            1
        \end{pmatrix}^3 = 685464\]
    种每种花色各一张的牌型。
\end{example}

\begin{example}[最多两张 $A$]

    让我们再来看看一个类似的问题。$5$ 张手牌最多有两张 $A$ 的牌型数量?在继续阅读之前,请花几分钟自己尝试解决这个问题。如果你遇到困难,可以尝试遵循我们在上一个问题中使用的论证方法。这两个问题有哪些相似之处和不同之处?

    以下是我们处理这个问题的方法:
    \begin{enumerate}
        \item 恰有两张 $A$:
              \begin{enumerate}[label=(\alph*)]
                  \item 从 $4$ 张 $A$ 中选择 $2$ 张:有 $\big({4 \atop 2}\big)$ 种方式。
                  \item 从剩余的 $48$ 张非 $A$ 牌中选择 $3$ 张:有 $\big({48 \atop 3}\big)$ 种方式。
              \end{enumerate}
        \item 恰有一张 $A$:
              \begin{enumerate}[label=(\alph*)]
                  \item 从 $4$ 张 $A$ 中选择 $1$ 张:有 $\big({4 \atop 1}\big)$ 种方式。
                  \item 从剩余的 $48$ 张非 $A$ 牌中选择 $4$ 张:有 $\big({48 \atop 4}\big)$ 种方式。
              \end{enumerate}
        \item 没有 $A$:
              \begin{enumerate}[label=(\alph*)]
                  \item 选择 $0$ 张 $A$:有 $\big({4 \atop 0}\big)=1$ 种方式。
                  \item 从剩余的 $48$ 张非 $A$ 牌中选择 $5$ 张:有 $\big({48 \atop 5}\big)$ 种方式。
              \end{enumerate}
    \end{enumerate}
    由于情况 1、2、3 不重合(即一手牌中有特定数量的 $A$),我们可以使用加法原理;此外,我们可以在这三种情况内使用乘法原理,因为我们在每种情况下顺序执行这两个步骤。因此,有
    \[\begin{pmatrix}
            4 \\
            2
        \end{pmatrix}\begin{pmatrix}
            48 \\
            3
        \end{pmatrix}+\begin{pmatrix}
            4 \\
            1
        \end{pmatrix}\begin{pmatrix}
            48 \\
            4
        \end{pmatrix}+\begin{pmatrix}
            4 \\
            0
        \end{pmatrix}\begin{pmatrix}
            48 \\
            5
        \end{pmatrix}\]
    (注意:在表示二项式系数时,通常省略乘法符号 $\cdot$;乘法是隐含的。)

    你有没有考虑没有 $A$ 的情况?这是一个常见的疏漏!你避免了我们在上一个问题中提到的多算问题了吗?我们需要根据手牌中 $A$ 的数量,将手牌集划分为三个不重叠的情况。

    另一种解决该问题的方法是利用我们在前一示例中的成果。也许你已经想到了这种方法。如果是这样,那恭喜你!这种方法的主要思路是将所有手牌分为两种情况:一是至多有 $2$ 个 $A$ 的情况,二是至少有 $3$ 个 $A$ 的情况。设 $S$ 为至多有 $2$ 个 $A$ 的牌型集,$T$ 为至少有 $3$ 个 $A$ 的牌型集,$H$ 为所有牌型的集合。根据我们的解释,$H = S \cup T$ 且 $S \cap T = \varnothing$。因此,可以应用加法原理推导出 $|H| = |S| + |T|$。此外,由于我们需要确定 $|S|$,我们可以写做
    \[|S| = |H| - |T|\]
    因此
    \[|S| = \begin{pmatrix}
            52 \\
            5
        \end{pmatrix}-\Bigg(\begin{pmatrix}
                4 \\
                3
            \end{pmatrix}\begin{pmatrix}
                49 \\
                2
            \end{pmatrix}+\begin{pmatrix}
                4 \\
                4
            \end{pmatrix}\begin{pmatrix}
                48 \\
                1
            \end{pmatrix}\Bigg)\]
    我们可以直接写出这个解,而不需要额外的计算!我们只需要将问题划分为两个集合,并且这两个集合的基数已经是已知的。

    这种策略揭示了一个更深层次的原理。实际上,我们应用了``减法原理''来得到我们想要的答案。这相当于先应用之前提到的加法原理,然后再对表达式进行调整。确实,这是理解这个问题的``正确''方式,因为这符合基本数学原理的应用。然而,在数学证明中,通常会更直接地应用``减法原理''。证明撰写者可能会假设读者已经熟悉加法原理的复杂运作,并直接得出结论,而不会明确指出划分或详细解释如何应用加法原理。例如,一个资深数学家可能会通过写下以下内容来为当前示例提供证明:
    \begin{quotation}
        从所有牌型的集合中,移除那些包含三张或四张 $A$ 的牌型,可得
        \[\begin{pmatrix}
                52 \\
                5
            \end{pmatrix}-\begin{pmatrix}
                4 \\
                3
            \end{pmatrix}\begin{pmatrix}
                49 \\
                2
            \end{pmatrix}-\begin{pmatrix}
                4 \\
                4
            \end{pmatrix}\begin{pmatrix}
                48 \\
                1
            \end{pmatrix}\]
    \end{quotation}
    一位数学家同仁稍加思考后,会接受这个证明。不过,我们猜你可能会想:这是不是\emph{太简短}了?会不会让读者思考得太辛苦?目前,在你学习数学的这个阶段,我们强烈建议(并要求)你在此类证明中提供更详细的细节。我们希望你能应用加法原理,并解释为什么在应用该原理时存在一个划分,然后进行代数操作以得出结论。将来,在这门课程之外,你可以随意使用``减法原理''。但现在,我们希望你能正确掌握基本原理,这就是为什么需要使用加法原理。

    最后还有一个牌型计数问题。它涉及加法原理和乘法原理,需要你仔细思考每个步骤。
\end{example}

\begin{example}[恰好一张 $Q$ 和一张 $\spadesuit$]

    恰好包含一张 $Q$ 和一张 $\spadesuit$ 的牌型有多少种?

    先自己尝试一下。也可以问问朋友,他们遇到过有这样的牌型吗?有没有什么问题能帮你决定接下来要问的问题?你会如何反向思考这些问题并确定一个有效的解决过程?

    这是我们的解决过程。它与你的有何不同?是完全相同还是在某种程度上等价?我们只是以不同的顺序划分了手牌吗?我们得到了相同的最终答案吗?为什么是或者为什么不是?请认真对待,即使我们的步骤或最终答案不同,也不要气馁。花时间思考为什么我们的答案不同,这对你更有启发,而不仅仅是阅读我们的正确答案。
    \begin{enumerate}
        \item 包含 $Q\spadesuit$
              \begin{enumerate}[label=(\alph*)]
                  \item 选择 $Q\spadesuit$:有 $1$ 种方式
                  \item 从剩余 $51$ 张牌中,选择 $4$ 张既不是 $Q$(还剩 $3$ 张)也不是黑桃(还剩 $11$ 张)的牌:有 $\big({51-3-12 \atop 4}\big)=\big({36 \atop 4}\big)$ 种方式
              \end{enumerate}
        \item 不包含 $Q\spadesuit$
              \begin{enumerate}[label=(\alph*)]
                  \item 选择非黑桃 $Q$:有 $\big({3 \atop 1}\big)$ 种方式
                  \item 选择非 $Q$ 黑桃牌:有 $\big({12 \atop 1}\big)$ 种方式
                  \item 从剩余 $50$ 张牌中,选择 $3$ 张既不是 $Q$(还剩 $3$ 张) 也不是黑桃(非 $Q$ 黑桃牌还剩 $11$ 张)的牌:有 $\big({50-3-11 \atop 3}\big)=\big({36 \atop 3}\big)$ 种方式
              \end{enumerate}
    \end{enumerate}
    由于每一步的选择都会产生唯一的结果,因此我们可以应用乘法原理。同时,因为每种牌型要么有 $Q\spadesuit$ 要么没有,所以加法原理也适用。因此,恰好包含一张 $Q$ 和一张 $\spadesuit$ 的牌型数量为
    \[\begin{pmatrix}
            36 \\
            4
        \end{pmatrix}+\begin{pmatrix}
            3 \\
            1
        \end{pmatrix}\begin{pmatrix}
            12 \\
            1
        \end{pmatrix}\begin{pmatrix}
            36 \\
            3
        \end{pmatrix} = 58,905 + 257,040 = 315,945\]

    这个问题比前面的例子更棘手,所以我们鼓励你多读几遍这个证明,直到完全理解为止。实际上,你可以问问你的朋友是否能解决这个问题,然后按照上面证明的步骤试着说服他们接受你的答案。你是否理解得足够透彻,能够向他人解释这些步骤?如果是这样,你就是组合论证的大师了!

    在下一小节中,我们将进一步提升你对组合论证和证明的掌握。在此过程中,我们还将介绍一些标准组合对象,以便我们可以计算除扑克牌型外的其他东西。尽管牌型计数问题很常见也很容易提出,但我们也想讨论一些其他内容!
\end{example}