% !TeX root = ../../../book.tex

\subsection{其他计牌示例}

让我们通过一些相关例子来扩展所学技术的应用范围。\\

\begin{example}[至少三张 $A$]

    本例将计算 $5$ 张手牌中至少有三张 $A$ 的牌型数量。我们沿用之前的方法,先分析这类牌型的基本特征。请尝试提出几个问题,使得通过答案能唯一确定牌型,并且对于任意答案,我们都能精确计算构建该牌型的方法数。

    你是否意识到其中的困难?其中一个问题的答案会\emph{直接影响}其他问题的性质!这反映出背后存在更深层的数学结构。一个合理的解决思路是:先明确问题核心,再考虑围绕它所需做出的决策。

    首先,考虑手牌中恰好有三张 $A$ 的情况。此时需确定另外两张牌的特性,它们要么 (a) 点数不同,要么 (b) 点数相同。因此,这一情形可以进一步分为两种子情况。由此我们可以得到以下步骤:
    \begin{enumerate}
        \item 从 $4$ 张 $A$ 中选择 $3$ 张:有 ${4 \choose 3}$ 种方式
              \begin{enumerate}[label=(\alph*)]
                  \item 剩余两张牌点数不同:
                        \begin{enumerate}[label=(\roman*)]
                            \item 从剩余的 $12$ 个点数中,为剩下的两张牌选择两个不同点数:有 ${12 \choose 2}$ 种方式
                            \item 为较小点数的牌选择花色:有 ${4 \choose 1}$ 种方式
                            \item 为较大点数的牌选择花色:有 ${4 \choose 1}$ 种方式
                        \end{enumerate}
                  \item 剩余两张牌点数相同:
                        \begin{enumerate}[label=(\roman*)]
                            \item 从剩余的 $12$ 个点数中选择一个点数:有 ${12 \choose 1}$ 种方式
                            \item 为选出的点数选择两种花色:有 ${4 \choose 2}$ 种方式
                        \end{enumerate}
              \end{enumerate}
    \end{enumerate}
    所以,根据乘法原理与加法原理(因为过程中涉及两种独立子情况),\emph{恰有三张} $A$ 的牌型数量为:
    \[{4 \choose 3}\left[{12 \choose 2} {4 \choose 1}^2 + {12 \choose 1} {4 \choose 2}\right]\]

    其次,考虑手牌中恰好有四张 $A$ 的情况。此时需确定第五张牌的特性,具体步骤如下:
    \begin{enumerate}
        \item 从 $4$ 张 $A$ 中选择全部 $4$ 张:有 ${4 \choose 4}$ 种方式
        \item 从剩余的 $12$ 个点数中,为剩下的一张牌选择一个点数:有 ${12 \choose 1}$ 种方法
        \item 为剩余的一张牌选择一种花色:有 ${4 \choose 1}$ 种方式
    \end{enumerate}
    由乘法原理,\emph{恰有四张} $A$ 的牌型总数为:
    \[{4 \choose 4}{12 \choose 1}{4 \choose 1}\]

    现在应用加法原理。将``至少三张 $A$''的牌型划分为两个互斥子集:恰好三张 $A$ 与恰好四张 $A$。这两个子集构成了更大集合的划分(即每种牌型仅属于其中一种),故至少有三张 $A$ 的牌型总数为:
    \[{4 \choose 3}\left[{12 \choose 2} {4 \choose 1}^2 + {12 \choose 1} {4 \choose 2}\right] + {4 \choose 4}{12 \choose 1}{4 \choose 1}\]

    回顾之前提到的加法原理的严格陈述,其中涉及有限集的基数,但我们在先前的例子中并未深入细节。这类组合数学的论证需要一定的判断力和技巧。你能轻松理解\emph{至少有三张 $A$ 的牌型,要么正好有三张 $A$,要么正好有四张 $A$,而不会同时有三张和四张 $A$,也不会两者都没有}吗?我们并不是说这应该完全显而易见;如果你没有立即理解,也没关系!我们想强调的是,这类陈述在证明中应当作为合理的解释。当然,我们可以深入细节,用集合术语重新描述扑克牌型,并将整个扑克游戏用集合符号表达。但这样做有何意义呢?上面楷体字的解释似乎更易于理解。如果有困惑的读者需要更详细的说明,我们自然可以提供,但对一般读者而言,这样的论证已经足够了。希望这个经验法则——说服一般读者,但在受到进一步质疑时能够深入解释——能帮助你决定在计数论证中需要包含多少细节。关键在于,我们解释了为什么我们的选择与牌型集合的划分相关。尽管没有严格证明这两个集合不相交,但我们给出了理由。

    我们可以采用另一种方法来解决这个问题,而无需考虑非 $A$ 牌的花色。具体来说,可以按以下步骤构建至少包含 $3$ 张 $A$ 的扑克手牌:
    \begin{enumerate}
        \item 如果恰有 $3$ 张 $A$:
              \begin{enumerate}[label=(\alph*)]
                  \item 为 $3$ 张 $A$ 选择三种花色:有 ${4 \choose 3}$ 种方法
                  \item 从剩余的 $48$ 张非 $A$ 牌中选择 $2$ 张凑齐 $5$ 张牌:有 ${48 \choose 2}$ 种方法
              \end{enumerate}
        \item 如果恰有 $4$ 张 $A$:
              \begin{enumerate}[label=(\alph*)]
                  \item 为 $4$ 张 $A$ 选择四种花色:有 ${4 \choose 4} = 1$ 种方法
                  \item 从剩余的 $48$ 张非 $A$ 牌中选择 $1$ 张凑齐 $5$ 张牌:有 ${48 \choose 1}$ 种方法
              \end{enumerate}
    \end{enumerate}
    因此,根据加法原理(我们根据手牌中 $A$ 的数量划分成两种情况)以及每种情况下的乘法原理,至少有三张 $A$ 的牌型总数为:
    \[{4 \choose 3}{48 \choose 2}+{4 \choose 4}{48 \choose 1}\]

    你会频繁地看到并使用这种方法。前面的论证更类似于之前同花的例子,因此我们先介绍了同花。这种论证更简短、更``巧妙'',因而更常用。但等一下,这些答案在形式上看起来不同!我们计算的是同一牌型数量,难道不应该得到相同的结果吗?实际上,结果是相同的。我们建议你进行必要的代数运算来验证这一点:
    \[{4 \choose 3}{48 \choose 2}+{4 \choose 4}{48 \choose 1} = {4 \choose 3}\left[{12 \choose 2} {4 \choose 1}^2 + {12 \choose 1} {4 \choose 2}\right] + {4 \choose 4}{12 \choose 1}{4 \choose 1}\]
    这只需花一分钟,但很值得一试。
\end{example}

在继续下一个问题之前,我们先来看一个关于该问题的\emph{错误论证}。虽然研究错误答案可能看起来有些奇怪,但经验表明,找出错误论证中的\emph{缺陷}是非常有帮助和启发性的。我们当然可以简单地比较两个大数,然后说:``看,它们不一样!''但这种方法缺乏启发性。相反,我们希望通过组合论证,找出导致逻辑错误或计数错误的步骤。我们强烈推荐这种方法,原因如下:首先,它能帮助你更好地练习阅读证明和理解他人的论证。这在学习更多数学知识和阅读其他书籍时非常有用。其次,它能帮助你更好地审视自己的证明。在完成作业后,先搁置一会儿,再用清醒的头脑重新阅读。尽量像从未写过它一样去阅读(我们知道这很难做到!)。它是否合理?一些当时看似显而易见的步骤现在是否让你感到困惑?答案是否正确?你是否被它说服?第三,识别证明中的错误步骤能巩固你对论证基础原则的理解。通过组合论证并识别错误,将真正帮助你理解并掌握加法原理和乘法原理。请相信我们。

你对下面的论证有何看法?请记住,这个答案是\emph{不正确的},我们想知道为什么!

\begin{example}[找出缺陷]

    \textcolor{red}{
        至少有三张 $A$ 的牌型有多少种?
        \begin{enumerate}[label=(\arabic*)]
            \item 如果恰有 $3$ 张 $A$:
                  \begin{enumerate}[label=(\alph*)]
                      \item 为 $4$ 张 $A$ 中选择 $3$ 张:有 ${4 \choose 3}$ 种方法
                      \item 从剩余的 $49$ 张牌中选择 $2$ 张凑齐 $5$ 张牌:有 ${49 \choose 2}$ 种方法
                  \end{enumerate}
            \item 如果恰有 $4$ 张 $A$:
                  \begin{enumerate}[label=(\alph*)]
                      \item 为 $4$ 张 $A$ 选择 $4$ 张:有 ${4 \choose 4} = 1$ 种方法
                      \item 从剩余的 $48$ 张牌中选择 $1$ 张凑齐 $5$ 张牌:有 ${48 \choose 1}$ 种方法
                  \end{enumerate}
        \end{enumerate}
        因此,至少有三张 $A$ 的牌型总数为:
        \[{4 \choose 3}{49 \choose 2}+{4 \choose 4}{48 \choose 1}\]
    }

    这里有什么问题?你看出任何错误了吗?乘法原理是否被错误地应用了?加法原理是否用在了不该用的地方?我们是否多算了或少算了?我们是否计算了一些不符合要求的牌型?在继续阅读之前,请思考这些问题。

    我们注意到:这个答案\emph{太大了}。我们\emph{多算了},因为某些牌型在计算中被重复计数了。也就是说,每一种符合条件的牌型至少被上述步骤计数了一次,但有些牌型可以通过这些步骤以多种方式构建。这些观察表明,这里的数字被高估了。

    我们注意到:这个答案\emph{太大了}。我们\emph{多算了},因为某些牌型在我们的计算中被重复计数了。也就是说,我们要计算的每一种牌型至少被上述步骤包括了一次,但有些牌型可以通过这些步骤以多种方式构建。这些观察让我们确定这里的数字太大了。

    我们是如何知道的呢?建议你积极尝试找出可以通过上述步骤以两种不同方式构建的牌型。如果你在阅读证明时能够意识到这一点,你就知道整个证明是有缺陷的。在这种情况下,让我们检查恰好有 $4$ 张 $A$ 的牌型;具体来说,考虑 \Ac \As \Ad \Ah \twoc。我们可以通过以下路径构建此牌型:
    \begin{enumerate}
        \item 选择 $4$ 张 $A$ 中的 $3$ 张:\Ac \As \Ad
        \item 从剩余的 $49$ 张牌中再选择 $2$ 张:\Ah \twoc
    \end{enumerate}
    或者,我们可以选择这条路径:
    \begin{enumerate}
        \item 选择 $4$ 张 $A$ 中的 $4$ 张:\Ac \As \Ad \Ah
        \item 从剩余的 $48$ 张牌中再选择 $1$ 张:\twoc
    \end{enumerate}
    你现在看出问题了吗?通过上述过程,这手完全相同的牌至少可以通过两种不同的方式构建。因此,答案多算了。还有其他方式可以构建这手牌吗?有多少?尝试找出另一种被多算的牌型及其构建方法。我们能否确定每手牌被多算的次数,并以此修正我们的答案呢?这是一个有趣且非常具有挑战性的问题,我们稍后会回到这个问题。
\end{example}

\subsubsection*{论证中的潜在错误}

本小节重点介绍如何阅读组合证明,并识别其中常见的错误:
\begin{itemize}
    \item \textbf{误用乘法原理}\\
          证明在不适用乘法原理的情况下错误地使用了乘法原理。可能的原因包括:每一步的选择数量依赖于先前步骤的结果,或者不同的步骤顺序会导致相同的最终结果。
    \item \textbf{误用加法原理}\\
          证明在不适用加法原理的情形下错误地使用了加法原理。可能原因包括:所谓的集合``划分''实际上并非互不相交,或者这些``划分''的并集未能覆盖整个所讨论的集合。
    \item \textbf{多算}\\
          每个所需对象至少被计数一次,但部分对象被重复计数。换言之,所讨论集合中的某些元素可以通过证明步骤以多种方式生成。
    \item \textbf{少算}\\
          部分所需对象未被纳入计算。换言之,证明步骤遗漏了所讨论集合中的某些元素。
    \item \textbf{误算}\\
          计算中包含了不相关的对象。换言之,证明步骤计入了不属于所讨论集合的元素。
\end{itemize}
建议你仔细阅读书面证明,尝试找出潜在缺陷,即使它们可能并不存在。例如,可以尝试寻找多算的论证,或通过不同方法构建对象,这可能有助于发现之前未曾注意到的错误。如果未发现任何缺陷,则可以增强对证明正确性的信心。

\begin{example}\label{ex:example8.3.6}
    以下是一个典型的\textbf{多算}示例。我们将展示其多算的原因,并通过正确的计数方法进行修正。问题如下:

    $5$ 张手牌中每种花色至少出现一次的牌型有多少种?

    以下是一个\textcolor{red}{错误}论证:

    \begin{quote}\color{red}
        存在 ${13 \choose 1}^4 \cdot {48 \choose 1}$ 种此类牌型。

        我们可以通过一个五步过程来构建此牌型:第一步从 $13$ 个红桃中选择一张;第二步从 $13$ 个方片中选择一张;第三步从 $13$ 个黑桃中选择一张;第四步从 $13$ 个梅花中选择一张。以上每一步都有 ${13 \choose 1}$ 种选择方式。

        然后,从剩余的 $48$ 张牌种选择一张凑成 $5$ 张手牌,有 ${48 \choose 1}$ 种选择方式。根据乘法原理,即可得到上面的结论。
    \end{quote}

    问题出在哪儿?在继续阅读之前,请先自行思考。参考上面列出的潜在错误列表,判断此处属于哪种错误?你会如何证明这一点?

    我们认为这是一个\textbf{多算}案例。为了证明这一点,我们通过具体手牌来说明。这手牌应该只被计数一次,但实际上根据上述论证过程至少被计数了两次。

    考虑手牌 \Ah \Ad \As \Ac \Kh。注意,这手牌可以通过上述过程以两种方式构建:
    \begin{enumerate}[label=(\arabic*)]
        \item 第一步选择 \Ah ;第二步选择 \Ad ;第三步选择 \As ;第四步选择 \Ac ;第五步选择 \Kh。
        \item 第一步选择 \Kh ;第二步选择 \Ad ;第三步选择 \As ;第四步选择 \Ac ;第五步选择 \Ah。
    \end{enumerate}
    由于手牌是\emph{无序的},这两种方法会得到\emph{相同的结果}。然而,上述论证会将这两个结果分别计算。因此,论证中存在多算的问题。

    为了修正这一错误论证,让我们仔细思考每种花色需要出现的\textbf{次数}。$5$ 张牌对应 $4$ 种花色,要求每种花色至少出现一次,则必然有三种花色各出现一次,一种花色出现两次。换句话说,花色的\emph{分布}必然是 $(1, 1, 1, 2)$。

    为了计数牌型的数量,我们定义如下过程:
    \begin{itemize}
        \item 从四种花色种选择哪种花色出现两次(其余三种花色固定只出现一次):有 ${4 \choose 1}$ 种方式。
        \item 为选出的花色选择两张牌:有 ${13 \choose 2}$ 种方式。
        \item 剩下的三种花色,每个花色选择一张牌:有 ${13 \choose 1}^3$ 种方式。
    \end{itemize}
    根据乘法原理,每种花色至少出现一次的牌型总数为:
    \[{4 \choose 1}{13 \choose 2}{13 \choose 1}^3 = 685464\]
\end{example}

\begin{example}[最多两张 $A$]

    让我们再来看另一个类似的问题:$5$ 张手牌中最多有两张 $A$ 的牌型有多少种?在继续阅读之前,请花几分钟时间自己尝试解决这个问题。如果遇到困难,可以借鉴我们在上一个问题中使用的方法。这两个问题有哪些相似之处和不同之处?

    以下是我们解决这个问题的步骤:
    \begin{enumerate}
        \item 恰有两张 $A$:
              \begin{enumerate}[label=(\alph*)]
                  \item 从 $4$ 张 $A$ 中选择 $2$ 张:有 ${4 \choose 2}$ 种方式。
                  \item 从剩余的 $48$ 张非 $A$ 牌中选择 $3$ 张:有 ${48 \choose 3}$ 种方式。
              \end{enumerate}
        \item 恰有一张 $A$:
              \begin{enumerate}[label=(\alph*)]
                  \item 从 $4$ 张 $A$ 中选择 $1$ 张:有 ${4 \choose 1}$ 种方式。
                  \item 从剩余的 $48$ 张非 $A$ 牌中选择 $4$ 张:有 ${48 \choose 4}$ 种方式。
              \end{enumerate}
        \item 没有 $A$:
              \begin{enumerate}[label=(\alph*)]
                  \item 选择 $0$ 张 $A$:有 ${4 \choose 0}=1$ 种方式。
                  \item 从剩余的 $48$ 张非 $A$ 牌中选择 $5$ 张:有 ${48 \choose 5}$ 种方式。
              \end{enumerate}
    \end{enumerate}
    由于情况 1、2、3 互不重叠(即一手牌中 $A$ 的数量确定),我们可以使用加法原理;此外,在每种情况下,我们顺序执行两个步骤,因此可以使用乘法原理。于是,总数为:
    \[{4 \choose 2}{48 \choose 3}+{4 \choose 1}{48 \choose 4}+{4 \choose 0}{48 \choose 5}\]
    (注意:在表示二项式系数时,通常省略乘法符号;乘法是隐含的。)

    你是否考虑了没有 $A$ 的情况?这是一个常见的遗漏!你避免了我们在上题中提到的重复计数问题了吗?我们需要根据手牌中 $A$ 的数量,将手牌集合划分为三个互不重叠的情况。

    另一种解决方法是利用前一个例子的结果。也许你已经想到了这种方法。如果是这样,恭喜你!这种方法的核心思想是将所有手牌分为两种情况:一种是至多有两张 $A$,另一种是至少有三张 $A$。设 $S$ 为至多有两张 $A$ 的牌型集合,$T$ 为至少有三张 $A$ 的牌型集合,$H$ 为所有牌型的集合。根据定义,$H = S \cup T$ 且 $S \cap T = \varnothing$。因此,由加法原理可得 $|H| = |S| + |T|$。由于我们需要求 $|S|$,可以写成:
    \[|S| = |H| - |T|\]
    因此
    \[|S| = {52 \choose 5}-\left({4 \choose 3}{49 \choose 2}+{4 \choose 4}{48 \choose 1}\right)\]
    我们可以直接写出这个表达式,无需额外计算!只需将问题划分为两个集合,而这些集合的基数我们已经知道。

    这种策略揭示了一个更深层的原理:实际上,我们应用了``减法原理''来得到答案。这相当于先应用加法原理,再对表达式进行变换。这确实是理解这个问题的``正确''方式,因为它符合基本数学原理的应用。然而,在数学证明中,通常会更直接地使用``减法原理''。证明撰写者可能假设读者已经熟悉加法原理的运用,从而直接得出结论,而不明确说明划分或详细解释加法原理的应用。例如,一位资深数学家可能会这样证明此题:
    \begin{quotation}
        从所有牌型的集合中,移除那些包含三张或四张 $A$ 的牌型,可得:
        \[{52 \choose 5}-{4 \choose 3}{49 \choose 2}-{4 \choose 4}{48 \choose 1}\]
    \end{quotation}
    一位数学家同行稍加思考后,就会接受这个证明。不过,我们猜你可能会想:这是不是\emph{太简短}了?会不会让读者思考得太辛苦?目前,在你学习数学的这个阶段,我们强烈建议(并要求)你在此类证明中提供更详细的细节。我们希望你能应用加法原理,并解释为什么存在一个划分,然后通过代数运算得出结论。将来,在这门课程之外,你可以随意使用``减法原理''。但现在,我们希望你能牢固掌握基本原理,因此需要使用加法原理。

    最后,还有一个牌型计数问题。它涉及加法原理和乘法原理,需要你仔细思考每个步骤。
\end{example}

\begin{example}[恰好一张 $Q$ 和一张 $\spadesuit$]

    恰好包含一张 $Q$ 和一张 $\spadesuit$ 的牌型有多少种?

    先自己尝试一下。你也可以问问朋友,他们是否遇到过这样的牌型?有没有什么问题能帮助你决定接下来该问什么问题?你会如何反向思考这些问题,并确定一个有效的解决过程?

    以下是我们提供的解决过程。它与你的方法有何不同?是完全相同还是在某种程度上等价?我们只是以不同的顺序划分了手牌吗?我们是否得到了相同的最终答案?为什么是或为什么不是?请认真对待,即使我们的步骤或最终答案不同,也不要气馁。花时间思考为什么答案不同,这对你更有启发,而不仅仅是阅读这里的正确答案。
    \begin{enumerate}
        \item 包含 \Qs
              \begin{enumerate}[label=(\alph*)]
                  \item 选择 \Qs:有 $1$ 种方式
                  \item 从剩余 $51$ 张牌中,选择 $4$ 张既不是 $Q$(还剩 $3$ 张)也不是黑桃(还剩 $11$ 张)的牌:有 ${51-3-12 \choose 4}={36 \choose 4}$ 种方式
              \end{enumerate}
        \item 不包含 \Qs
              \begin{enumerate}[label=(\alph*)]
                  \item 选择非黑桃 $Q$:有 ${3 \choose 1}$ 种方式
                  \item 选择非 $Q$ 黑桃牌:有 ${12 \choose 1}$ 种方式
                  \item 从剩余 $50$ 张牌中,选择 $3$ 张既不是 $Q$(还剩 $3$ 张)也不是黑桃(非 $Q$ 黑桃牌还剩 $11$ 张)的牌:有 ${50-3-11 \choose 3}={36 \choose 3}$ 种方式
              \end{enumerate}
    \end{enumerate}
    由于每一步的选择都会产生唯一的结果,因此我们可以应用乘法原理。同时,因为每种牌型要么包含 \Qs 要么不包含,所以加法原理也适用。因此,恰好包含一张 $Q$ 和一张 $\spadesuit$ 的牌型数量为:
    \[{36 \choose 4}+{3 \choose 1}{12 \choose 1}{36 \choose 3} = 58,905 + 257,040 = 315,945\]

    这个问题比前面的例子更加复杂,所以我们鼓励你多读几遍论证过程,直到完全理解为止。实际上,你可以问问朋友是否能够解决这个问题,然后按照上述步骤试着说服他们接受你的答案。你是否理解得足够透彻,能够向他人解释这些步骤?如果是这样,你就是组合论证的大师了!

    在下一小节中,我们将进一步提升你对组合论证和证明的掌握。在此过程中,我们还将介绍一些标准组合对象,以便计算除扑克牌型外的其他内容。尽管牌型计数问题非常常见且易于提出,但我们也希望探讨更多有趣的主题!
\end{example}