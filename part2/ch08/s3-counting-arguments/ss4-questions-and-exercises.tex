% !TeX root = ../../../book.tex

\subsection{习题}

\subsubsection*{温故知新}

以口头或书面的形式简要回答以下问题。这些问题全都基于你刚刚阅读的内容,所以如果忘记了具体的定义、概念或示例,可以回去重读相关部分。确保在继续学习之前能够自信地回答这些问题,这将有助于你的理解和记忆!

\begin{enumerate}[label=(\arabic*)]
    \item 如何判断一个计数方法是否\textbf{少算}了数量?又如何证明它\textbf{多算}了数量?
    \item 解释``从 $[n]$ 中选择 $k$-元组''和``字母表与单词''之间的关系。它们在本质上为何相同?
    \item 假设我们从箱子里有的 $n$ 个球中选择 $k$ 个球。为什么球是否\emph{可区分}很重要?
    \item 为什么从 $(0, 0)$ 到 ($x, y)$ 的格路径数量等于 ${x+y \choose x}$ 或 ${x+y \choose y}$?
\end{enumerate}

\subsubsection*{小试牛刀}

尝试回答以下问题。这些题目要求你实际动笔写下答案,或(对朋友/同学)口头陈述答案。目的是帮助你练习使用新的概念、定义和符号。题目都比较简单,确保能够解决这些问题将对你大有帮助!

\begin{enumerate}[label=(\arabic*)]
    \item 找出由 $5$ 张牌组成的扑克牌型中包含\textbf{两对}牌型的数量,并证明你的结论。
    \item 找出由 $5$ 张牌组成的扑克牌型中包含\textbf{葫芦}牌型的数量,并证明你的结论。
    \item 计算单词 ``\verb|COMBINATORICS|'' 的所有异序词的数量。计算单词 ``\verb|MASSACHUSETTS|''的所有异序词的数量?
    \item 考虑从 $\{1, 2, 3\}$ 中选择每个数字至少出现一次的 $4$-元组的数量。对于以下每个``证明'',通过找出一个在提出的论证中被重复计算的对象来证明它是错误的。
          \begin{enumerate}[label=(\alph*)]
              \item 在 $4$-元组的 $4$ 个位置中选一个位置放 $1$,然后选一个位置放 $2$,再选一个位置放 $3$。最后,从剩下的三个元素中选择一个放在第 $4$ 个位置。
                    \[{4 \choose 1}{3 \choose 1}{2 \choose 1}{3 \choose 1} = 72\]
              \item 从 $4$ 个位置中选择 $3$ 个位置,用元素 $1,2,3$ 填充。然后,对这 $3$ 个位置中的元素进行排列。最后,为剩下的第 $4$ 个位置选择一个数字。
                    \[{4 \choose 3} \cdot 3! \cdot 3 = 72\]
          \end{enumerate}
    \item 在这个问题中,我们将任何由英文字母组成的字符串都视为一个单词,无论它是否在字典中有实际意义。例如,\verb|ZYQFIB| 是一个长度为 $6$ 的有效单词。
          \begin{enumerate}[label=(\alph*)]
              \item 长度为 $2$ 的单词有多少个?\\
                    (用\emph{两种}方式回答该问题:一种用指数形式,另一种用两项之和。)
              \item 长度为 $7$ 的单词中,恰好包含 $3$ 个 \verb|A| 的单词有多少个?
              \item 长度为 $7$ 的单词中,最多包含 $2$ 个元音字母的单词有多少个?(注意:\verb|A, E, I, O, U| 是元音字母,\verb|Y| 不是。)
          \end{enumerate}

          考虑所有长度为 $n$ 的二进制字符串的集合 $S_n$。对于以下每个给定的性质,分别计算 $S_n$ 中有多少元素符合该性质。\\
          (注意:每个性质是独立的,不需要考虑同时满足所有性质的情况。)
          \begin{enumerate}[label=(\alph*)]
              \item 恰有 $3$ 个位置为 $0$。
              \item 最多 $3$ 个位置为 $0$。
              \item 至少 $4$ 个位置为 $0$。\\
                    (注意:用前两问的结论将 $2^n$ 写成二项式系数之和!)
              \item $0$ 比 $1$ 多。
          \end{enumerate}
    \item 设 $n \in \mathbb{N}$。共有多少条格路径可以从 $(0, 0)$ 走到 $(2n, 2n)$?又有多少条格路径会经过 $(n, n)$?
    \item 考虑下面的解释:
          \begin{quote}
              从一副标准扑克牌发出的 $6$ 张牌中,每种花色\emph{至少出现一次}的牌型数量为
              \[{13 \choose 1}{13 \choose 1}{13 \choose 1}{13 \choose 1}{48 \choose 2}\]
              因为我们先从每种花色中各选一张牌,然后再从剩下的 $48$ 张牌中选出两张。
          \end{quote}
          这个计数正确吗?如果你认为\emph{多算}了,请展示一个具体的牌型示例,并说明它是如何被重复计数的。如果你认为\emph{少算}了,请展示一个具体的牌型示例,并说明它是如何没有被计数的。
\end{enumerate}