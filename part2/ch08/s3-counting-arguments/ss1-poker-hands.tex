% !TeX root = ../../../book.tex

\subsection{扑克牌型}

\begin{example}[一对]

    让我们从最简单的情况开始,计算一下一对牌型的数量。我们要强调的是,这里只计算正好是一对的牌型,不包含两对、三条、葫芦和四条。这一思路将在我们接下来的计数过程中逐步展现。(这也提示了为什么计算``高牌''牌型实际上非常复杂,比仅仅选择五张随机牌要难得多!我们如何保证一手牌没有重复的牌,不是顺子,也不是同花?我们将在本节后面讨论这个问题。)

    在这个例子中 --- 以及我们将要解释的每个其他例子中,还有你将完成的每个练习中(你感觉到这很重要吗?) --- 我们将寻找一种方法,通过这种方法我们能构建出具有特定属性的对象(在这种情况下,是一手正好有一对且不是其他牌型的扑克牌)。通过计算每一步的选择数量,并确保每个期望的对象只能通过一种选择组合得到,我们可以应用乘法法则,确定具有期望属性的对象数量。

    这里有一个有用的策略来设计这种方法:假设你的朋友手里拿着你要计算的一个对象,但你看不到它。你会问什么问题来确定他/她手中对象的特定属性?这些问题可以是是/否问题,或者更常见的是关于对象特定属性的询问。在我们的具体情况下,计算一对牌型,我们可能会问以下问题:
    \begin{enumerate}[label=(\arabic*)]
        \item ``一对中的两张牌是什么?''
        \item ``不在一对中的三张牌是什么?''
    \end{enumerate}
    通过这些问题的答案,我们可以完全确定朋友手中的牌。不幸的是,直接这样问,计算这些问题的答案数量太难了。我们应该更具体,并将问题分解成更小的部分。这样,我们可以计算每个问题的答案数量,并在乘法原理中使用这些数字。

    我们如何能更具体一些?我们如何能将第一个问题拆分成多个部分?想象一下我们的朋友可能怎么回答第一个问题。他们可能会说``红心 $A$ 和黑桃 $A$''或者``方块 $7$ 和梅花 $7$''。这表明了第一个问题的关键点:我们需要知道这对牌的\emph{点数}(比如都是 $A$ 吗?还是 $K$?抑或是 $Q$?)以及它们的花色。我们知道一副牌中有 $13$ 个点数和 $4$ 种花色。基于这些信息,我们可以确定如何构造一对牌型并计算可能的选项。
    \begin{enumerate}
        \item 选择一对牌的点数:$13$ 种方式
        \item 为这对牌选择两个花色: $\big({4 \atop 2}\big) = 6$ 种方式
    \end{enumerate}
    注意,这里我们使用了二项式系数 $\big({4 \atop 2}\big)$ 来表示从 $4$ 种花色中选择 $2$ 种花色的方法,因此有 $\big({4 \atop 2}\big)$ 种选择方式。

    这里强调一点:$\big({4 \atop 2}\big)$ 是一个\textbf{数字}。它表示执行某个操作的方法数量,但并不实际执行该操作。也就是说,我们不会说``$\big({4 \atop 2}\big)$ 从 $4$ 种花色中选择 $2$ 种花色''这样的话。毕竟,一个数字怎么可能从一副牌中选择牌呢?

    还要强调一点:在这个例子中,我们写 $\big({4 \atop 2}\big) = 6$ 只是为了说明问题。通常,我们并不希望你去实际计算二项式系数,因为这些计算往往涉及非常大的数字。实际上,$\big({4 \atop 2}\big)$ 比 $6$ 更有说明意义,它表明你在这一步中是从 $4$ 个元素中选择 $2$ 个,而 $6$ 可能代表 $\big({6 \atop 1}\big)$ 或 $2 \cdot \big({3 \atop 2}\big)$ 等等。基于这一点,我们在第一步中最好写成 $\big({13 \atop 1}\big)$。

    现在,我们可以看到,在此步骤中做出的任何选择都会产生一对\emph{唯一}的牌。也就是说,不可能有一对牌是由这个过程的两个不同版本产生的。因此,乘法原理适用,我们可以得出有 $\big({13 \atop 1}\big) \cdot \big({4 \atop 2}\big)$ 种方法选择一对牌。

    如果我们反过来执行这两个步骤呢?我们可以通过询问是哪两个花色,然后再询问它们的点数是什么来识别一对牌吗?(当然,这只有在我们预先知道牌有相同点数的情况下才有效。)在这种情况下,乘法原理会告诉我们有 $\big({4 \atop 2}\big) \cdot \big({13 \atop 1}\big)$ 种对子。嘿,这是同样的数量!实数的乘法交换律(即 $\forall x, y \in \mathbb{R} \centerdot x \cdot y = y \cdot x$)证实了我们的直觉,这些步骤是可逆的。

    我们还未完全构建出一副只包含一对的手牌。还需要再选择三张牌。它们应该具有什么特性呢?除了``它们是什么?''之外,我们还能问朋友什么更具体的问题?我们需要知道这三张牌的点数和花色。它们的花色有没有限制?没有!(因为我们已经有一对了,不可能出现同花。)它们的点数有没有限制?有!我们知道这三张牌的点数都不同,并且没有一个点数与已经选中的那对牌相同。通过这些观察,我们可以反向操作,构建出剩下的牌。
    \begin{enumerate}
        \item 从剩下的 $12$ 个点数中选择 $3$ 个点数(即与对子牌不同的点数):$\big({12 \atop 3}\big)$ 种方式
        \item 将这 $3$ 个点数按升序排列:$1$ 种方式
        \item 为最低点数的牌选择一个花色:$\big({4 \atop 1}\big)$ 种方式
        \item 为中间点数的牌选择一个花色:$\big({4 \atop 1}\big)$ 种方式
        \item 为最高点数的牌选择一个花色:$\big({4 \atop 1}\big)$ 种方式
    \end{enumerate}
    为什么需要步骤 $2$?回顾一下\emph{选择}的定义:它是一个\emph{无序}列表或集合。因此,跳过步骤 $2$ 直接说``为所选牌中的第一张选择一个花色''是不合理的,因为根本没有第一张牌!我们需要对牌进行某种排序,以便单独引用每一张卡片。你可能会想到按照从牌堆中移除它们的顺序来排序。这将把步骤 $1$ 分成 $3$ 个子步骤:
    \begin{enumerate}[label=(\alph*)]
        \item 选择第一张卡片,有 $\big({12 \atop 1}\big)$ 种方式;
        \item 选择第二张卡片,有 $\big({11 \atop 1}\big)$ 种方式;
        \item 选择第三张卡片,有 $\big({10 \atop 1}\big)$ 种方式;
    \end{enumerate}
    将乘法原理应用于此步骤会得出一个与步骤 $1$ 不同的数字:
    \[\begin{pmatrix}
            12 \\
            1
        \end{pmatrix} \cdot \begin{pmatrix}
            11 \\
            1
        \end{pmatrix} \cdot \begin{pmatrix}
            10 \\
            1
        \end{pmatrix} = 12 \cdot 11 \cdot 10 \ne \begin{pmatrix}
            12 \\
            3
        \end{pmatrix} = \frac{12!}{3! \cdot 9!} = \frac{12 \cdot 11 \cdot 10}{6}\]
    这是因为 (a)-(b)-(c) 步骤对这三张牌施加了一个排序,但在牌型中,这个排序实际上并不重要。玩牌时,你不会在意你是以什么顺序拿到的这些牌,只在意它们是什么牌!(然而,请注意,如果我们将排列这三张牌的方法数,即 $3!$,作为分母除以总数,我们会得到相同的结果。这暗示了一个有趣的概念,即``乘法原理''的``逆运算''。我们将在本节末尾讨论这一点。) 这就是为什么我们不能在步骤 $2$ 中提到``第 $1$ 张牌''。相反,我们找到了一种\emph{内在的}排序方式,通过牌本身的特定属性,使我们能够在不应用外部排序的情况下指代特定的牌。

    因为选择 $3$ 张不同点数的牌只能通过这些步骤中的一个选择集来实现,所以乘法原理在这里依然适用。我们可以将选择一对牌视为第一步,将选择另外三张不同点数的牌视为第二步,然后将乘法原理应用于整个过程。最终,我们得出``一对''牌型数量的答案:
    \[\begin{pmatrix}
            13 \\
            1
        \end{pmatrix} \cdot \begin{pmatrix}
            4 \\
            2
        \end{pmatrix} \cdot \begin{pmatrix}
            12 \\
            3
        \end{pmatrix} \cdot \begin{pmatrix}
            4 \\
            1
        \end{pmatrix}^3\]
    注意,我们已经将上一步中的三个数字合并成一个三次方的形式。以这种形式作为数值答案是\emph{完全可以接受的},比直接写 $1,098,240$ 要好得多。如果你在作业中出现了``笔误''或计算错误,我们该如何识别并评价这些错误呢?$\smiley{}$

    我们之前确实提到了乘法交换律和以不同顺序执行步骤的想法。然而,我想你会认同,尽管
    \[\begin{pmatrix}
            4 \\
            1
        \end{pmatrix}^3 \cdot \begin{pmatrix}
            13 \\
            1
        \end{pmatrix} \cdot \begin{pmatrix}
            4 \\
            2
        \end{pmatrix} \cdot \begin{pmatrix}
            12 \\
            3
        \end{pmatrix}\]
    表示的是同一过程,但解释起来显得过于复杂,而且没有必要。

    在上一小节中,我们特意进行了详细解释。但我们并不期望你写得那么多。我们只是正式介绍了一种应用上一节中提到的计数规则和公式的方法,同时也提到了一些启发式规则和解决问题的策略。现在,让我们以更简洁的形式给出这个问题的典型解决方案。这也是我们期望你写的解决方案:

    \begin{questions}{问}:$5$ 张扑克能组成多少``一对''牌型?\end{questions}

    \begin{proofs}{答}:一共有
        \[\begin{pmatrix}
                13 \\
                1
            \end{pmatrix} \cdot \begin{pmatrix}
                4 \\
                2
            \end{pmatrix} \cdot \begin{pmatrix}
                12 \\
                3
            \end{pmatrix} \cdot \begin{pmatrix}
                4 \\
                1
            \end{pmatrix}^3\]
        种。为了说明这一点,我们将定义一个四步过程并应用 ROP。这个过程的结果是得到一手``一对''牌型的手牌:
        \begin{enumerate}[label=(\arabic*)]
            \item 选择一个点数来构成一对。\\
                  有 $\big({13 \atop 1}\big)$ 种方式。
            \item 选择步骤 (1) 中选定牌的两种花色。\\
                  有 $\big({4 \atop 2}\big)$ 种方式。
            \item 选择另外三张牌的点数。\\
                  有 $\big({12 \atop 3}\big)$ 种方式。
            \item 为步骤 (3) 中选择的每一个点数,选择一种花色。\\
                  每个点数有 $\big({4 \atop 1}\big)$ 种花色选择,一共三次,因此总共有 $\big({4 \atop 1}\big)^3$ 种方式。
        \end{enumerate}
        应用 ROP,即可得到上面的答案。
    \end{proofs}

    这段解释是否清晰合理?注意它比我们之前的解释要简短得多。这是可以接受的!我们在这里的书面例子中会继续写出一些细节(帮助你理解如何解决这些问题,然后再写出来),但你的书面解决方案可以稍微简洁一些,只要它们包含问题解决方案的所有关键要素即可。注意,我们指出了 ROP 的使用,引用了它,并识别了过程中所有的步骤;对于每一步,我们说明了有多少种方法可以完成该步骤。恰好这些步骤都很简单,每种方法的数量在每种情况下都很清楚。通常情况下,我们可能会期待更详细的描述。例如,我们会考虑写出步骤 (3) 有 $\big({12 \atop 3}\big)$ 种方法,因为我们不允许重新选择步骤 (1) 中选择的点数。然而,我们觉得已经描述地很清楚了,所以省略了这一点。这是一个判断问题,不过,我们建议(像往常一样)忘掉你的证明,然后重新阅读它们,就像你从没写过它们一样。如果你记不住,或者不完全确定某些事情为什么是正确的,考虑在那里增加一些额外的描述。

    在开启另一个例子之前,让我们给出该问题的另一种解决方案!

    \begin{questions}{问}:$5$ 张扑克能组成多少``一对''牌型?\end{questions}

    \begin{proofs}{答}:一共有
        \[\begin{pmatrix}
                13 \\
                4
            \end{pmatrix} \cdot \begin{pmatrix}
                4 \\
                1
            \end{pmatrix} \cdot \begin{pmatrix}
                4 \\
                2
            \end{pmatrix} \cdot \begin{pmatrix}
                4 \\
                1
            \end{pmatrix}^3\]
        种。我们将定义一个六步过程并应用 ROP。其主要思想是,通过选择所有出现的四个点数并确定哪个点数重复了两次(其余的点数只出现一次),可以得到``一对''牌型。
        \begin{enumerate}[label=(\arabic*)]
            \item 选择牌型中的 $4$ 个点数。\\
                  有 $\big({13 \atop 4}\big)$ 种方式。
            \item 在步骤 (1) 选出的点数中,选择一个作为对子的点数。\\
                  有 $\big({4 \atop 1}\big)$ 种方式。
            \item 为步骤 (2) 选择的点数,选择两种花色。\\
                  有 $\big({4 \atop 2}\big)$ 种方式。
            \item 为步骤 (2) \emph{未}选择的三个点数中最小的点数,选择一种花色。\\
                  有 $\big({4 \atop 1}\big)$ 种方式。
            \item 为步骤 (2) \emph{未}选择的三个点数中中间的点数,选择一种花色。\\
                  有 $\big({4 \atop 1}\big)$ 种方式。
            \item 为步骤 (2) \emph{未}选择的三个点数中最大的点数,选择一种花色。\\
                  有 $\big({4 \atop 1}\big)$ 种方式。
        \end{enumerate}
        根据 ROP 并化简 $\big({4 \atop 1}\big)\big({4 \atop 1}\big)\big({4 \atop 1}\big)=\big({4 \atop 1}\big)^3$,我们证明了上面答案是正确的。
    \end{proofs}

    这是不是很有趣?我们留给你自己去验证这个数字
    \[\begin{pmatrix}
            13 \\
            4
        \end{pmatrix} \cdot \begin{pmatrix}
            4 \\
            1
        \end{pmatrix} \cdot \begin{pmatrix}
            4 \\
            2
        \end{pmatrix} \cdot \begin{pmatrix}
            4 \\
            1
        \end{pmatrix}^3 =  1098240 = \begin{pmatrix}
            13 \\
            1
        \end{pmatrix} \cdot \begin{pmatrix}
            4 \\
            2
        \end{pmatrix} \cdot \begin{pmatrix}
            12 \\
            3
        \end{pmatrix} \cdot \begin{pmatrix}
            4 \\
            1
        \end{pmatrix}^3\]
    是否正确。不过,即使不计算中间的那个数字,我们也可以确定左右两边的表达式是完全相等的,因为它们都在计算同一个东西:一对牌型的数量。这是我们正在介绍的``双法计数''这一概念的另一个例子。
\end{example}

\begin{example}[同花]

    让我们直接解决另一个问题:计算同花牌型的数量。同花牌型有两个特点:所有 $5$ 张牌的花色相同,且点数不同。因此,同花可以通过两步生成:
    \begin{enumerate}[label=(\arabic*)]
        \item 为所有 $5$ 张牌选一种花色。\\
              有 $\big({4 \atop 1}\big)$ 种方式。
        \item 为选出的花色选择 $5$ 个点数。\\
              有 $\big({13 \atop 5}\big)$ 种方式。
    \end{enumerate}
    由于每一个同花牌型都由这两个步骤唯一确定,因此我们可以应用 ROP 得出结论,有
    \[\begin{pmatrix}
            4 \\
            1
        \end{pmatrix} \cdot \begin{pmatrix}
            13 \\
            5
        \end{pmatrix}=5148\]
    种同花牌型。

    这个例子中给出的证明(除了最后的数字 $5148$,这里仅仅是为了与``一对''牌型数量进行比较,后者要大得多),是完全正确和严谨的,并且会获得满分。可以将其作为使用乘法原理进行简单计数的模板。
\end{example}

\begin{example}[顺子]

    顺子牌型由``起始牌'',即手牌中最小的那张牌的点数决定。如果我告诉你我有一个以 $7$ 开头的 $5$ 张顺子,你会立刻知道我有 $7\;8\;9\;T\;J$ 这样的顺子。因为顺子可以是 $A\;2\;3\;4\;5$,也可以是 $2\;3\;4\;5\;6$,一直到 $T\;J\;Q\;K\;A$(注意:顺子中没有 $Q\;K\;A\;2\;3$ 这样的``循环''情况),这意味着顺子的\emph{最小牌}有 $10$ 种可能。因此,有十种类型的顺子,确定了顺子的类型之后,我们只需要分配花色,确保它们不全是同一种花色(否则就会变成同花顺)。

    我们说一共有
    \[\begin{pmatrix}
            10 \\
            1
        \end{pmatrix}\Bigg[\begin{pmatrix}
                4 \\
                1
            \end{pmatrix}^5 - \begin{pmatrix}
                4 \\
                1
            \end{pmatrix}\Bigg]=10 \cdot (4^5-4)\]
    种顺子牌型。

    \begin{proof}
        我们将通过一个两步过程来描述 $5$ 张顺子牌型:
        \begin{enumerate}[label=(\arabic*)]
            \item 选择 $10$ 个点数中的一个作为顺子的\emph{最低}点数。选项为 $A,2,3,4,5,6,7,8,9,T$,所以这一步有 $10$ 种方法。\\
                  注意:选出了最低点数也就\emph{确定}了其余 $4$ 张牌的点数,因为 $5$ 张牌的点数必须是连续的,而我们已经知道最低点数是什么。
            \item 给 $5$ 张牌分配花色,使它们不是同一花色。\\
                  假设 $X$ 是所有可能的花色分配的集合,所以这一步有 $|X|$ 种方法。
        \end{enumerate}

        我们现在通过创建集合划分来得到 $|X|$。设 $Y$ 为所有 $5$ 张牌花色都相同的分配集合。注意,集合 $X$ 和 $Y$ 形成了 $U$ 的划分,$U$ 是所有 $5$ 张牌花色分配的集合。(也就是说,任意 $5$ 张牌的花色分配要么选择相同的花色,要么不选择相同的花色。) 因此,根据 ROS,我们有 $|U| = |X| + |Y|$。

        我们可以通过 $5$ 步过程得到 $|U|$,在步骤 $i$ 中,我们为手中第 $i$ 大的牌选择 $4$ 种花色中的一种。每一步有 $4$ 个选项,我们有 $|U| = 4^5$。

        $|Y|$ 可以理解为选择 $4$ 种花色中的一种分配给所有 $5$ 张牌,因此 $|Y| = 4$。

        综上,我们可以整理上面的等式,得到
        \[|X| = |U| - |Y| = 4^5 - 4\]

        由于 $|X|$ 是上述步骤 (2) 中可能的方法数,根据 ROP,我们已经证明了该声明。
    \end{proof}

    \textbf{注意}:在这个证明中,我们展示了所有相关步骤,证明了有 $10 \cdot 4 = 40$ 种可能的\emph{同花顺}(同一花色的顺子),其中只有 $1 \cdot 4 = 4$ 种\emph{皇家同花顺}(同一花色的 $T\;J\;Q\;K\;A$)。试着自己写出这些论证过程吧!
\end{example}