% !TeX root = ../../../book.tex

\subsection{扑克牌型}

\begin{example}[一对]

    让我们从最简单的情况开始,计算一对牌型的数量。需要强调的是,这里只计算恰好是一对的牌型,不包括两对、三条、葫芦和四条。这一思路将在后续的计数过程中逐步展开。(这也揭示了为什么计算``高牌''牌型实际上非常复杂,它比单纯选择五张随机牌要困难得多!我们如何确保一手牌没有对子、不是顺子、也不是同花?我们将在本节后面讨论这一问题。)

    在这个例子中——以及我们将要讨论的其他例子和练习中——我们将寻找一种方法来构建具有特定属性的对象(即恰好有一对且不构成其他牌型的扑克牌手牌)。通过计算每一步的选择数,并确保每个目标对象仅对应一种选择组合,我们就能应用乘法原理来确定具有所需属性的对象数量。

    这里有一个有用的策略:假设你的朋友手里拿着一个你要计数的对象,但你无法看到它。你会提出哪些问题来确定该对象的特定属性?这些问题可以是是否问题,或者更常见的是关于对象具体属性的问题。在计算一对牌型的具体情况时,我们可能会问:
    \begin{enumerate}[label=(\arabic*)]
        \item ``一对中的两张牌是什么?''
        \item ``不在一对中的三张牌是什么?''
    \end{enumerate}
    通过这些问题的答案,我们可以完全确定朋友手中的牌。然而,直接这样提问,计算答案的数量会非常困难。因此,我们应该将问题具体化并分解为更小的部分。这样,我们就能计算每个子问题的选择数,并在乘法原理中应用这些数字。

    我们怎样才能更具体一些?如何将第一个问题拆分成多个部分?想象一下,我们的朋友可能会如何回答第一个问题。他们可能会说``红心 $A$ 和黑桃 $A$''或者``方片 $7$ 和梅花 $7$''。这揭示了第一个问题的关键:我们需要知道这对牌的\emph{点数}(例如,都是 $A$ 吗?还是 $K$?或者是 $Q$?)以及它们的花色。我们知道一副牌中有 $13$ 种点数和 $4$ 种花色。基于这些信息,我们可以确定如何构造一对牌并计算可能的选项。
    \begin{enumerate}
        \item 选择一对牌的点数:有 $13$ 种方式
        \item 为这对牌选择两个花色:有 ${4 \choose 2} = 6$ 种方式
    \end{enumerate}
    注意,这里我们使用了二项式系数 ${4 \choose 2}$ 来表示从 $4$ 种花色中选择 $2$ 种花色的方法数,因此有 ${4 \choose 2}$ 种选择方式。

    这里需要强调一点:${4 \choose 2}$ 是一个\textbf{数字},它表示执行某个操作的方法数量,但并不实际执行该操作。也就是说,我们不会说``${4 \choose 2}$ 从 $4$ 种花色中选择 $2$ 种花色''这样的话,毕竟一个数字怎么可能从一副牌中选择牌呢?

    还要强调一点:在这个例子中,我们写 ${4 \choose 2} = 6$ 只是为了说明问题。通常,我们并不希望实际计算二项式系数,因为这些计算往往涉及非常大的数字。实际上,${4 \choose 2}$ 比 $6$ 更有说明性,因为它表明你是从 $4$ 个元素中选择 $2$ 个,而 $6$ 可能代表 ${6 \choose 1}$ 或 $2 \cdot {3 \choose 2}$ 等。基于这一点,我们在第一步中最好写成 ${13 \choose 1}$。

    现在,我们可以看到,在此步骤中做出的任何选择都会产生\emph{唯一}的一对牌。也就是说,不可能有一对牌是由这个过程的两个不同版本产生的。因此,乘法原理适用,我们可以得出有 ${13 \choose 1} \cdot {4 \choose 2}$ 种方法选择一对牌。

    如果我们反过来执行这两个步骤呢?我们可以先询问是哪两个花色,再询问它们的点数来构造一对牌吗?(当然,这只有在我们预先知道牌的点数相同的情况下才有效。)在这种情况下,乘法原理告诉我们有 ${4 \choose 2} \cdot {13 \choose 1}$ 种对子。看,数量是一样的!实数的乘法交换律(即 $\forall x, y \in \mathbb{R}, x \cdot y = y \cdot x$)证实了我们的直觉,这些步骤是可逆的。

    我们还未完全构建出一副只包含一对的牌型。还需要再选择三张牌。它们应该具有什么特性呢?除了``它们是什么?''之外,我们还能问朋友什么更具体的问题?我们需要知道这三张牌的点数和花色。它们的花色有没有限制?没有!(因为我们已经有一对了,不可能出现同花。)它们的点数有没有限制?有!我们知道这三张牌的点数都不同,并且没有一个点数与已经选中的那对牌相同。通过这些观察,我们可以反向操作,构建出剩下的牌。
    \begin{enumerate}
        \item 从剩下的 $12$ 个点数中选择 $3$ 个点数(即与对子牌不同的点数):有 ${12 \choose 3}$ 种方式
        \item 将这 $3$ 个点数按升序排列:有 $1$ 种方式
        \item 为最低点数的牌选择一个花色:有 ${4 \choose 1}$ 种方式
        \item 为中间点数的牌选择一个花色:有 ${4 \choose 1}$ 种方式
        \item 为最高点数的牌选择一个花色:有 ${4 \choose 1}$ 种方式
    \end{enumerate}
    为什么需要步骤 $2$?回顾一下\emph{选择}的定义:它是一个\emph{无序}列表或集合。因此,跳过步骤 $2$ 直接说``为所选牌中的第一张选择一个花色''是不合理的,因为根本没有第一张牌!我们需要对牌进行某种排序,以便单独引用每一张牌。你可能会想到按照从牌堆中移除它们的顺序来排序。这将把步骤 $1$ 分成 $3$ 个子步骤:
    \begin{enumerate}[label=(\alph*)]
        \item 选择第一张卡片:有 ${12 \choose 1}$ 种方式;
        \item 选择第二张卡片:有 ${11 \choose 1}$ 种方式;
        \item 选择第三张卡片:有 ${10 \choose 1}$ 种方式;
    \end{enumerate}
    将乘法原理应用于此步骤会得出一个与步骤 $1$ 不同的数字:
    
    \[{12 \choose 1} \cdot {11 \choose 1} \cdot {10 \choose 1} = 12 \cdot 11 \cdot 10 \ne {12 \choose 3} = \frac{12!}{3! \cdot 9!} = \frac{12 \cdot 11 \cdot 10}{6}\]

    这是因为 (a)-(b)-(c) 步骤对这三张牌施加了一个排序,但在牌型中,排序并不重要。玩牌时,你不会在意拿牌的顺序,只在意它们是什么牌!(然而,请注意,如果我们将排列这三张牌的方法数,即 $3!$,作为分母除以总数,我们会得到相同的结果。这暗示了一个有趣的概念,即``乘法原理''的``逆运算''。我们将在本节末尾讨论这一点。)这就是为什么我们不能在步骤 $2$ 中提到``第 $1$ 张牌''。相反,我们找到了一种\emph{内在的}排序方式,通过牌本身的特定属性(如点数大小),使我们能够在不应用外部排序的情况下指代特定的牌。

    因为选择 $3$ 张不同点数的牌只能通过这些步骤中的一个选择集合来实现,所以乘法原理在这里依然适用。我们可以将选择一对牌视为第一步,将选择另外三张不同点数的牌视为第二步,然后将乘法原理应用于整个过程。最终,我们得出``一对''牌型数量的答案:
    
    \[{13 \choose 1} \cdot {4 \choose 2} \cdot {12 \choose 3} \cdot {4 \choose 1}^3\]

    注意,我们已经将上一步中的三个 ${4 \choose 1}$ 合并成 ${4 \choose 1}^3$ 的形式。以这种形式作为数值答案是\emph{完全可以接受的},比直接写 $1,098,240$ 要好得多。如果你在作业中出现了``笔误''或计算错误,我们该如何识别并修正这些错误呢?$\smiley{}$

    我们之前确实提到了乘法交换律和以不同顺序执行步骤的想法。然而,我想你会认同,尽管
    \[{4 \choose 1}^3 \cdot {13 \choose 1} \cdot {4 \choose 2} \cdot {12 \choose 3}\]
    表示的是同一过程,但解释起来显得过于复杂,而且没有必要。

    在上一小节中,我们进行了详细解释,但并不期望你写那么多内容。我们只是正式介绍了一种应用上一节提到的计数规则和公式的方法,同时也提及了一些启发式规则和问题解决策略。现在,我们以更简洁的形式给出这个问题的典型解决方案,这也是我们期望你写的解决方案:

    \begin{questions}{问}:$5$ 张扑克能组成多少``一对''牌型?\end{questions}

    \begin{proofs}{答}:``一对''牌型的数量为:
        \[{13 \choose 1} \cdot {4 \choose 2} \cdot {12 \choose 3} \cdot {4 \choose 1}^3\]
        为了说明这一点,我们定义一个四步过程并应用 ROP。这个过程的结果是得到一手``一对''牌型的手牌:
        \begin{enumerate}[label=(\arabic*)]
            \item 选择构成一对的点数:有 ${13 \choose 1}$ 种方式。
            \item 选择步骤 (1) 中选定点数的两种花色:有 ${4 \choose 2}$ 种方式。
            \item 选择另外三张牌的点数:有 ${12 \choose 3}$ 种方式。
            \item 为步骤 (3) 中选择的每个点数选择一种花色。\\
                  每个点数有 ${4 \choose 1}$ 种花色选择,共三个点数,因此总共有 ${4 \choose 1}^3$ 种方式。
        \end{enumerate}
        应用 ROP,即可得到上面的答案。
    \end{proofs}

    这段解释是否清晰合理?注意它比我们之前的解释更简短。这是可以接受的!在书面例子中,我们会继续提供一些细节(以帮助你理解如何解决这些问题),但你的书面解决方案可以稍微简洁一些,只要包含所有关键要素即可。注意,我们指出了 ROP 的使用,引用了它,并识别了过程中的所有步骤;对于每一步,我们说明了完成该步骤的方法数。这些步骤都很简单,方法数在每种情况下都很清楚。通常情况下,我们可能会期待更详细的描述。例如,我们本可以说明步骤 (3) 有 ${12 \choose 3}$ 种方法,因为我们不允许重新选择步骤 (1) 中选择的点数。然而,我们认为已经描述得很清楚了,所以省略了这一点。这是一个判断题。不过,我们建议(像往常一样)忘记你的证明,然后重新阅读它们,就像你从未写过一样。如果你记不住或不完全确定某些事情为什么正确,考虑在那里增加一些额外描述。

    在开始另一个例子之前,我们先给出这个问题的另一种解决方案!

    \begin{questions}{问}:$5$ 张扑克能组成多少``一对''牌型?\end{questions}

    \begin{proofs}{答}:``一对''牌型的数量为:
        \[{13 \choose 4} \cdot {4 \choose 1} \cdot {4 \choose 2} \cdot {4 \choose 1}^3\]
        我们将通过一个六步过程来应用 ROP。其主要思想是:通过选择所有出现的四个点数,并确定哪个点数重复两次(其余点数只出现一次),就可以得到``一对''牌型。
        \begin{enumerate}[label=(\arabic*)]
            \item 选择牌型中的 $4$ 个点数:有 ${13 \choose 4}$ 种方式。
            \item 在步骤 (1) 选出的点数中,选择一个作为对子的点数:有 ${4 \choose 1}$ 种方式。
            \item 为步骤 (2) 选择的点数,选择两种花色:有 ${4 \choose 2}$ 种方式。
            \item 为步骤 (2) \emph{未}选择的三个点数中最小的点数,选择一种花色:有 ${4 \choose 1}$ 种方式。
            \item 为步骤 (2) \emph{未}选择的三个点数中中间的点数,选择一种花色:有 ${4 \choose 1}$ 种方式。
            \item 为步骤 (2) \emph{未}选择的三个点数中最大的点数,选择一种花色:有 ${4 \choose 1}$ 种方式。
        \end{enumerate}
        根据 ROP 并化简 ${4 \choose 1}{4 \choose 1}{4 \choose 1}={4 \choose 1}^3$,我们证明了上述答案的正确性。
    \end{proofs}

    这是不是很有趣?我们留给读者自己验证这个数字
    \[{13 \choose 4} \cdot {4 \choose 1} \cdot {4 \choose 2} \cdot {4 \choose 1}^3 =  1098240 = {13 \choose 1} \cdot {4 \choose 2} \cdot {12 \choose 3} \cdot {4 \choose 1}^3\]
    是否正确。不过,即使不计算中间值,我们也可以确定左右两边的表达式是相等的,因为它们都在计算同一个东西:一对牌型的数量。这是我们正在介绍的``双法计数''这一概念的另一个例子。
\end{example}

\begin{example}[同花]

    让我们直接解决另一个问题:计算同花牌型的数量。同花牌型有两个特点:所有 $5$ 张牌的花色相同,且点数互不相同。因此,同花可以通过两步生成:
    \begin{enumerate}[label=(\arabic*)]
        \item 为所有 $5$ 张牌选一种花色:有 ${4 \choose 1}$ 种方式。
        \item 为选出的花色选择 $5$ 个不同点数:${13 \choose 5}$ 种方式。
    \end{enumerate}
    由于每一个同花牌型都由这两个步骤唯一确定,因此我们可以应用 ROP 得出结论,同花牌型的数量为:
    \[{4 \choose 1} \cdot {13 \choose 5}=5148\]

    这个例子中给出的证明(除了最后的数字 $5148$,这里仅仅是为了与``一对''牌型数量进行比较,后者要大得多),是完全正确和严谨的,并且可以得到满分。可以将其作为使用乘法原理进行简单计数的模板。
\end{example}

\begin{example}[顺子]

    顺子牌型由``起始牌'',即手牌中最小的那张牌的点数决定。如果我告诉你我有一个以 $7$ 开头的 $5$ 张顺子,你会立刻知道我有 $7\;8\;9\;10\;J$ 这样的顺子。因为顺子可以是 $A\;2\;3\;4\;5$,也可以是 $2\;3\;4\;5\;6$,一直到 $10\;J\;Q\;K\;A$(注意:顺子中没有 $Q\;K\;A\;2\;3$ 这样的``循环''情况),这意味着顺子的\emph{最小牌}有 $10$ 种可能。因此,有十种类型的顺子,确定了顺子的类型之后,我们只需要分配花色,确保它们不全是同一种花色(否则就会变成同花顺)。

    我们说顺子牌型的数量为:
    \[{10 \choose 1}\Bigg[{4 \choose 1}^5 - {4 \choose 1}\Bigg]=10 \cdot (4^5-4)\]

    \begin{proof}
        我们将通过一个两步过程来描述 $5$ 张顺子牌型:
        \begin{enumerate}[label=(\arabic*)]
            \item 选择 $10$ 个点数中的一个作为顺子的\emph{最小}点数。选项为 $A,2,3,4,5,6,7,8,9,T$,所以这一步有 $10$ 种方法。\\
                  注意:选出了最小点数也就\emph{确定}了其余 $4$ 张牌的点数,因为 $5$ 张牌的点数必须是连续的,而我们已经知道最小点数是什么。
            \item 给 $5$ 张牌分配花色,使它们不是同一花色。\\
                  假设 $X$ 是所有可能的花色分配的集合,所以这一步有 $|X|$ 种方法。
        \end{enumerate}

        我们现在通过创建集合划分来得到 $|X|$。设 $Y$ 为所有 $5$ 张牌花色都相同的分配集合。注意,集合 $X$ 和 $Y$ 构成 $U$ 的一个划分,其中 $U$ 是所有 $5$ 张牌花色分配的集合。(也就是说,任意 $5$ 张牌的花色分配要么所有牌花色相同,要么不全相同。)因此,根据 ROS,我们有 $|U| = |X| + |Y|$。

        我们可以通过一个 $5$ 步过程得到 $|U|$:在步骤 $i$ 中,我们为手中第 $i$ 张牌选择 $4$ 种花色中的一种。每一步有 $4$ 个选项,因此 $|U| = 4^5$。

        $|Y|$ 可以理解为选择 $4$ 种花色中的一种分配给所有 $5$ 张牌,因此 $|Y| = 4$。

        综上,我们可以整理上面的等式,得到
        \[|X| = |U| - |Y| = 4^5 - 4\]

        由于 $|X|$ 是上述步骤 (2) 中可能的方法数,根据 ROP,我们已经证明了该声明。
    \end{proof}

     \textbf{注意}:在这个证明中,我们展示了所有相关步骤,证明了有 $10 \cdot 4 = 40$ 种可能的\emph{同花顺}(同一花色的顺子),其中只有 $1 \cdot 4 = 4$ 种\emph{皇家同花顺}(同一花色的 $10\;J\;Q\;K\;A$)。尝试自己写出这些论证过程吧!
\end{example}