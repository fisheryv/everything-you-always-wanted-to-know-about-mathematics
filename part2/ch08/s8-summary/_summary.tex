% !TeX root = ../../../book.tex
\section{总结}

我们现在已经开发了几种基本的计数技术,并将它们进一步发展成更高级的技术。我们首先简单讨论了加法原理和乘法原理,这些规则基于前一章关于有限集基数的结果。通过这些规则,我们可以开发出一系列基本计数对象,并描述如何对它们进行计数。其中包括非常有用的\textbf{二项式系数}。我们亲自推导了二项式系数公式,并实施了一种计数策略。然后,我们将这些原理应用于大量示例,以便练习处理计数论证的细微差别:有时涉及多种情况,有时我们必须巧妙地应用乘法原理,有时我们需要注意避免多算或少算。基于此,我们讨论了如何提出一个假设论证并\emph{展示}其错误。

\emph{双法计数}证明技术非常重要,在许多数学领域有着广泛应用。我们看了一些具有指导意义的例子,这些例子本身就是有用的定理,并在练习中提出了许多此类问题,以便你有足够的练习。我们使用双法计数技术后来证明了一些进一步的结果和技术,包括二项式定理和重复选择公式。

我们还简要讨论了一些更高级的计数技术,如抽屉原理和容斥原理。这些技术被认为是更高级的技术,其原因在于很难看出何时以及如何应用它们。通过一些说明性示例,我们希望能帮助你更好地理解这些技术的应用,以便在解决问题时知道何时使用它们。