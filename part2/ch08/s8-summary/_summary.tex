% !TeX root = ../../../book.tex
\section{总结}

我们已经介绍了几种基本的计数技术,并将它们发展成更高级的技术。首先,我们简要讨论了加法原理和乘法原理,这些规则基于前一章关于有限集基数的结论。通过这些规则,我们引入了一系列基本计数对象,并解释了如何对它们进行计数,其中包括非常有用的\textbf{二项式系数}。我们推导了二项式系数的公式,并应用了一种计数策略。随后,我们通过大量示例来应用这些原理,以练习处理计数论证中的细微差别:例如,有时涉及多种情况,有时需要巧妙运用乘法原理,有时则需要注意避免重复计数或遗漏。在此基础上,我们讨论了如何提出一个假设性论证并\emph{展示}其错误。

\emph{双法计数}是一种非常重要的证明技术,在众多数学领域中有广泛应用。我们介绍了一些具有指导意义的例子,这些例子本身也是有用的定理,并在练习中设置了大量相关问题,以帮助你获得充分的练习。之后,我们运用双法计数技术证明了一些二级结论和方法,包括二项式定理和重复选择公式。

我们还简要讨论了一些更高级的计数技术,例如抽屉原理和容斥原理。这些技术被视为高级方法,原因在于很难判断何时以及如何应用它们。通过一些示例,我们希望能够帮助你更好地理解这些技术的使用场景,从而在解决问题时能够准确判断何时应用它们。