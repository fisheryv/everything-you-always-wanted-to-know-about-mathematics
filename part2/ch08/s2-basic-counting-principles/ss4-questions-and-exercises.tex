% !TeX root = ../../../book.tex

\subsection{习题}\label{sec:section8.2.4}

\subsubsection*{温故知新}

以口头或书面的形式简要回答以下问题。这些问题全都基于你刚刚阅读的内容,所以如果忘记了具体的定义、概念或示例,可以回去重读相关部分。确保在继续学习之前能够自信地回答这些问题,这将有助于你的理解和记忆!

\begin{enumerate}[label=(\arabic*)]
    \item \textbf{选择}与\textbf{排列}有何区别?
    \item 如何利用选择和排列来定义\textbf{全排列}?
    \item $\displaystyle {10 \choose 15}$ 表示什么?\\
    \item 全排列与\textbf{双射}之间有何关联?
\end{enumerate}

\subsubsection*{小试牛刀}

尝试回答以下问题。这些题目要求你实际动笔写下答案,或(对朋友/同学)口头陈述答案。目的是帮助你练习使用新的概念、定义和符号。题目都比较简单,确保能够解决这些问题将对你大有帮助!

\begin{enumerate}[label=(\arabic*)]
    \item 通过代数方法验证 $\displaystyle {n \choose k} = {n \choose {n-k}}$。\\
    \item 证明命题 \ref{prop:proposition8.2.16}。即证明
        \[{n \choose k} = \frac{n!}{k!(n-k)!}\]\label{exc:exercises8.2.2}
    \item 证明命题 \ref{prop:proposition8.2.18}。即证明 $[n]$ 中存在 $\displaystyle \frac{n!}{(n-k)!}$ 种可能的 $k$-排列。\\ \label{exc:exercises8.2.3}
    \item 证明命题 \ref{prop:proposition8.2.21}。即证明 $[n]$ 中存在 $n^k$ 种可能的可重复 $k$-排列。\label{exc:exercises8.2.4}
\end{enumerate}