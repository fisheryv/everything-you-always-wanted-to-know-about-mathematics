% !TeX root = ../../../book.tex

\subsection{加法原理}

回顾我们在上一章中证明的定理 \ref{theorem7.6.7}。这个定理指出,当两个有限集不相交(即没有共同元素)时,它们的并集大小等于各自大小之和。这一结论对于有限集来说是直观的,我们通过双射证明了它。这个结论是组合数学中第一个基本且有用的原理。请注意,这一结论牢固地建立在集合论的原理之上。

\subsubsection*{划分}

我们首先回顾一下定义 \ref{def:definition3.6.9},这是我们在讨论集合时引入的。

\begin{definition}
    设 $A$ 为集合。$A$ 的\dotuline{划分}为互不相交且并集为 $A$ 的集合所组成的集合。

    也就是说,划分由满足以下条件的索引集 $I$ 和非空集 $S_i$(定义在每个 $i \in I$ 上)构成:
    \begin{enumerate}[label=(\arabic*)]
        \item 对于所有 $i \in I, S_i \subseteq A$。
        \item 对于所有 $i, j \in I \;\text{且}\; i \ne j$,我们有 $S_i \cap S_j = \varnothing$。
        \item $\displaystyle \bigcup_{i \in I} S_i = A$
    \end{enumerate}
\end{definition}
本质上,划分是一种将集合分成不重叠的小集合的方法。让我们先看几个例子。\\

\begin{example}
    设 $A$ 为当前房间里人员的集合。设 $I = \{1, 2\}$,并设 $S_1$ 为左撇子的集合,$S_2$ 为右撇子的集合。那么 $S = \{S_1, S_2\}$ 是 $A$ 的一个划分。注意,``$\{S_1, S_2\}$ 划分 $A$'' 是正确的,而 ``$S_1, S_2$ 划分 $A$'' 是不正确的。那么在这种情况下,$S_1$ 和 $S_2$ 的含义是什么呢?实际上,我们的意思是这两个集合作为一个整体,形成了 $A$ 的一个划分。这就是为什么我们必须记住要把集合 $S$ 的元素写在括号里。

    为了严谨起见,我们应该\emph{证明}为什么 $S$ 是 $A$ 的一个划分。为此,我们指出 $S_1 \cap S_2 = \varnothing$,因为这里的每个人要么是左撇子,要么是右撇子,但不能两者都是。(假设这里没有``特殊情况'',比如双手灵活的人或没有手的人。如果有这样的情况存在,则将他们包含在集合 $S_3$ 中,并将其包含在我们的划分集合 $S$ 中。)我们还指出 $S_1 \cup S_2 = A$,因为房间里的每个人必然是左撇子或右撇子,因此不存在 $x \in A$ 满足 $x \notin S_1$ 且 $x \notin S_2$。这证明了为什么 $S$ 是一个划分。

    如果我们想通过他们名字的首字母来划分这个房间里人员的集合呢?试着像前面的例子那样使用数学符号来定义这个划分。
\end{example}

\begin{example}
    现在,让我们来看一个无限分割。考虑集合 $A = \mathbb{N}$,索引集 $I = \mathbb{N}$。对于每个 $i \in \mathbb{N}$,定义集合
    \[S_i = \{2i - 1, 2i\}\]
    集合 $S = \{S_i \mid i \in \mathbb{N}\}$ 是 $\mathbb{N}$ 的划分吗?我们认为答案是肯定的。让我们来探讨其中的原因。我们可以写出前面几个集合,看一下是什么样的(事实上,这通常是一个绝佳的起始策略:只需写出前几个情况,看看会发生什么):
    \begin{align*}
        S_1 & = \{1, 2\} \\
        S_2 & = \{3, 4\} \\
        S_3 & = \{5, 6\} \\
        &\vdots
    \end{align*}
    目前为止看上去是 $\mathbb{N}$ 的分割。让我们证明确实如此!

    首先,我们要证明集合 $S_i$ \emph{互不相交}(即任意两个集合都没有共同元素)。我们使用反证法来证明这一点。为了引出矛盾而假设 $\exists i, j \in \mathbb{N}$ 且 $i \ne j$,使得 $S_i \cap S_j \ne \varnothing$。这意味着 $S_i$ 中至少有一个元素也是 $S_j$ 的元素;在这种情况下,我们发现有四种可能的情况:
    \begin{enumerate}
        \item $2i - 1 = 2j - 1$
        \item $2i - 1 = 2j$
        \item $2i = 2j - 1$
        \item $2i = 2j$
    \end{enumerate}
    第一和第四种种情况通过简单的代数处理即可得到 $i = j$,这与我们假设 $i \ne j$ 相矛盾。第二和第三种情况与它们自身相矛盾,因为它们都会令一个偶数等于一个奇数。无论哪种情况,我们都得到一个矛盾。因此 $\forall i, j \in \mathbb{N}$ 且 $i \ne j$ 只可能是 $S_i \cap S_j = \varnothing$。

    接着,我们来证明所有 $S_i$ 的并集为 $\mathbb{N}$。也就是说我们要证明
    \[\bigcup_{i \in \mathbb{N}} S_i = \mathbb{N}\]
    请记住,左侧的集合包含所有满足 $\exists i \in I$ 使得 $x \in S_i$ 的元素 $x$。(思考一下为什么这是合理的,即使 $I$ 是无限集。这意味着并集包含了至少属于一个 $S_i$ 的所有元素。)注意,对于每个 $i \in \mathbb{N}$,元素 $2i - 1$ 和 $2i$ 都是 $S_i$ 中的自然数。因此
    \[\mathbb{N} \supseteq \bigcup_{i \in \mathbb{N}} S_i\]
    然后,我们证明反方向包含。设 $n \in \mathbb{N}$。我们需要考虑两种情况:
    \begin{enumerate}[label=(\arabic*)]
        \item 如果 $n$ 为偶数,则 $\exists k \in \mathbb{N}$ 使得 $n = 2k$。因此 $n \in S_k$。
        \item 如果 $n$ 为奇数,则 $\exists \ell \in \mathbb{N}$ 使得 $n = 2\ell-1$。因此 $n \in S_\ell$。
    \end{enumerate}
    无论哪种情况,我们都证明 $\displaystyle n \in \bigcup_{i \in \mathbb{N}} S_i$。

    因此,$S$ 是 $\mathbb{N}$ 的划分,并且还是一个无限划分。
\end{example}

现在我们已经看过了有限划分和无限划分的例子。

(\textbf{挑战性问题}:你能找到一种 $\mathbb{N}$ 的无限划分,使得划分的组成集合也都是无限集吗?)

\subsubsection*{陈述}

在本章的其余部分,我们只讨论有限集的有限划分。需要注意的是,加法原理仅适用于这种情况。

\begin{proposition}
    设 $A$ 为有限集,设 $n \in \mathbb{N}$,并设 $S = \{S_i \mid i \in [n]\}$ 为 $A$ 的有限划分。\dotuline{加法原理} 说的是
    \[|A| = \sum_{i \in [n]} |S_i|\]
\end{proposition}

加法原理告诉我们,一个集合的大小可以通过将其划分为若干小集合,并将这些小集合的大小相加来确定。请注意,这正是我们在上一章讨论有限集时提到的推论 \ref{corollary7.6.10}!我们在 \ref{sec:section7.6.5} 节的练习 \ref{exc:exercises7.6.2} 中要求你用归纳法证明过这个结论。现在,我们将继续看看一些例子。

\subsubsection*{示例}

\begin{example}
    在一所奇特的大学中,每个学生每年必须参加一项校队运动。参加多于一项运动会占用太多时间,而完全不参加运动会使他们变得懒惰,所以每个人都只能参加以下现代体育运动之一:高尔夫、板球、羽毛球和国际象棋。体育部门发布了今年各运动队的统计数据:
    \begin{itemize}
        \item 高尔夫:$12$ 人
        \item 板球: $16$ 人
        \item 羽毛球:$23$ 人
        \item 国际象棋:$33$ 人
    \end{itemize}
    那么这所大学有多少学生?

    这是一个超级简单的例子,因为我们已经确保大学体育项目的规则构成学生集的一个划分。(与``大学提供的体育项目集是学生集的一个划分''这句话相比,两者都是正确的。)因此,我们可以通过相加来求出 $S$ 的基数,即所有学生的总数:
    \[|S| = 12 + 18 + 23 + 33 = 86\]
    这真是一个又小又奇怪的大学,最好还是不去为妙。
\end{example}

当我们将加法原理与其他计数原理结合起来时,会有更有趣的应用案例。目前,加法原理是一个简单的概念,用于指导我们如何计数可以分解成不相交部分的集合。通常,使用加法原理最难的部分在于确定应用于哪个划分,并且需要一些创造性思考。

接下来的计数原理同样有用,甚至可能更有用,但它的定义和证明要复杂一些。