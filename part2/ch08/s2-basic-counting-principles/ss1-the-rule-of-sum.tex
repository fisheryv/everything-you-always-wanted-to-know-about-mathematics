% !TeX root = ../../../book.tex

\subsection{加法原理}

回顾我们在上一章中证明的定理 \ref{theorem7.6.7}。该定理指出,对于两个不相交的有限集(即它们没有共同元素),它们的并集的大小等于各自大小之和。这一结论对于有限集而言是直观的,我们通过双射对其进行了证明。它是组合数学中第一个基本且实用的原理。请注意,这一结论完全建立在集合论的基本原理之上。

\subsubsection*{划分}

我们首先回顾一下定义 \ref{def:definition3.6.9},这是在讨论集合时引入的概念。

\begin{definition}
    设 $A$ 为集合。$A$ 的\dotuline{划分}为互不相交且并集为 $A$ 的集合所构成的集合。

    也就是说,划分由满足以下条件的索引集 $I$ 和非空集 $S_i$(定义在每个 $i \in I$ 上)构成:
    \begin{enumerate}[label=(\arabic*)]
        \item $\forall i \in I, S_i \subseteq A$。
        \item $\forall i, j \in I$ 且 $i \ne j$,有 $S_i \cap S_j = \varnothing$。
        \item $\displaystyle \bigcup_{i \in I} S_i = A$
    \end{enumerate}
\end{definition}
本质上,划分是将一个集合分割为若干互不重叠的子集的方法。下面我们通过几个例子来说明。

\begin{example}
    设 $A$ 表示当前房间里所有人的集合。取索引集 $I = \{1, 2\}$,并定义 $S_1$ 为左撇子的人的集合,$S_2$ 为右撇子的人的集合。那么 $S = \{S_1, S_2\}$ 构成 $A$ 的一个划分。注意,正确的表述是``$\{S_1, S_2\}$ 划分 $A$'',而不是``$S_1, S_2$ 划分 $A$''。那么 $S_1$ 和 $S_2$ 在这里具体指什么呢?实际上,我们是指这两个集合作为一个整体,共同构成了 $A$ 的一个划分。这就是为什么我们必须用花括号将集合 $S$ 的元素括起来。

    为了严谨起见,我们应当\emph{证明} $S$ 是 $A$ 的一个划分。首先,注意到 $S_1 \cap S_2 = \varnothing$,因为每个人要么是左撇子,要么是右撇子,不能两者都是。(假设不存在``特殊情况'',例如双手灵活的人或没有手的人。如果存在这类情况,应将他们归入集合 $S_3$,并将 $S_3$ 纳入划分 $S$ 中。)其次,有 $S_1 \cup S_2 = A$,因为房间里的每个人必然是左撇子或右撇子,因此不存在 $x \in A$ 满足 $x \notin S_1$ 且 $x \notin S_2$。这证明了 $S$ 是一个划分。

    如果我们想根据名字的首字母来划分房间里的人的集合,该如何用数学符号定义这个划分?请尝试仿照前面的例子进行描述。
\end{example}

\begin{example}
    现在,让我们来看一个无限划分的例子。考虑集合 $A = \mathbb{N}$,索引集 $I = \mathbb{N}$。对于每个 $i \in \mathbb{N}$,定义集合
    \[S_i = \{2i - 1, 2i\}\]
    集合 $S = \{S_i \mid i \in \mathbb{N}\}$ 是 $\mathbb{N}$ 的划分吗?我们认为答案是肯定的。让我们来探讨其中的原因。我们可以先写出前面几个集合,观察其结构(事实上,这通常是一个绝佳的起始策略:只需写出前几种情况,看看会发生什么):
    \begin{align*}
        S_1 & = \{1, 2\} \\
        S_2 & = \{3, 4\} \\
        S_3 & = \{5, 6\} \\
        &\vdots
    \end{align*}
    从目前的情况看,它似乎是 $\mathbb{N}$ 的一个划分。接下来我们严格证明这一点。

    首先,证明集合 $S_i$ \emph{互不相交}(即任意两个集合没有共同元素)。我们使用反证法来证明这一点。为了引出矛盾而假设 $\exists i, j \in \mathbb{N}$ 且 $i \ne j$,使得 $S_i \cap S_j \ne \varnothing$。这意味着 $S_i$ 中至少有一个元素也是 $S_j$ 的元素;此时可能出现四种情况:
    \begin{enumerate}
        \item $2i - 1 = 2j - 1$
        \item $2i - 1 = 2j$
        \item $\quad \enspace \: 2i = 2j - 1$
        \item $\quad \enspace \: 2i = 2j$
    \end{enumerate}
    第一种和第四种情况通过简单的代数运算可得 $i = j$,与假设 $i \ne j$ 矛盾。第二种和第三种情况自相矛盾,因为它们分别要求一个偶数等于一个奇数。无论哪种情况,均导致矛盾。因此 $\forall i, j \in \mathbb{N}$ 且 $i \ne j$,必有 $S_i \cap S_j = \varnothing$。

    接下来,证明所有 $S_i$ 的并集等于 $\mathbb{N}$,也就是说我们要证明
    \[\bigcup_{i \in \mathbb{N}} S_i = \mathbb{N}\]
    注意,左侧的集合包含所有满足 $\exists i \in I$ 使得 $x \in S_i$ 的元素 $x$。(思考一下为什么这是合理的,即使 $I$ 是无限集,这也是合理的,因为并集包含至少属于某个 $S_i$ 的所有元素。)显然,对于每个 $i \in \mathbb{N}$,元素 $2i - 1$ 和 $2i$ 都是 $S_i$ 中的自然数,因此
    \[\mathbb{N} \supseteq \bigcup_{i \in \mathbb{N}} S_i\]
    然后,证明反向包含。设 $n \in \mathbb{N}$。分两种情况讨论:
    \begin{enumerate}[label=(\arabic*)]
        \item 若 $n$ 为偶数,则 $\exists k \in \mathbb{N}$ 使得 $n = 2k$。因此 $n \in S_k$。
        \item 若 $n$ 为奇数,则 $\exists \ell \in \mathbb{N}$ 使得 $n = 2\ell-1$。因此 $n \in S_\ell$。
    \end{enumerate}
    无论哪种情况,均有 $\displaystyle n \in \bigcup_{i \in \mathbb{N}} S_i$。

    因此,$S$ 是 $\mathbb{N}$ 的一个划分,且是一个无限划分。
\end{example}

现在我们已经看过了有限划分和无限划分的例子。

(\textbf{挑战性问题}:你能找到一种 $\mathbb{N}$ 的无限划分,使得划分的每个组成部分也都是无限集吗?)

\subsubsection*{陈述}

在本章的其余部分,我们只讨论有限集的有限划分。需要注意的是,加法原理仅适用于这种情况。

\begin{proposition}
    设 $A$ 为有限集,设 $n \in \mathbb{N}$,且 $S = \{S_i \mid i \in [n]\}$ 为 $A$ 的有限划分。\dotuline{加法原理}指出:
    \[|A| = \sum_{i \in [n]} |S_i|\]
\end{proposition}

加法原理表明,一个集合的大小可以通过将其划分为若干互不相交的子集,并将这些子集的大小相加得到。请注意,这正是上一章讨论有限集时提到的推论 \ref{corollary7.6.10}!我们在 \ref{sec:section7.6.5} 节的练习 \ref{exc:exercises7.6.2} 中要求你用归纳法证明了这个结论。接下来,我们通过一些例子进一步说明。

\subsubsection*{示例}

\begin{example}
    在一所奇特的大学里,每名学生每年必须参加一项校队运动。参加多于一项运动可能占用太多时间,而完全不参加运动又会导致懒惰,因此每名学生只能选择以下现代体育运动之一:高尔夫、板球、羽毛球或国际象棋。体育部公布了今年各运动队的统计数据:
    \begin{itemize}
        \item \quad 高尔夫:$12$ 人
        \item \qquad 板球:$16$ 人
        \item \quad 羽毛球:$23$ 人
        \item 国际象棋:$33$ 人
    \end{itemize}
    那么这所大学共有多少学生?

    这是一个超级简单的例子,因为大学体育项目的规则天然构成了学生集合的一个划分。(事实上,``大学提供的体育项目集合是学生集合的一个划分''这一说法同样成立。)因此,我们可以直接相加求出学生集合 $S$ 的基数,即所有学生的总数:
    \[|S| = 12 + 18 + 23 + 33 = 86\]
    这真是一所又小又古怪的大学,最好还是不去为妙。
\end{example}

当加法原理与其他计数原理结合时,会产生更有趣的应用案例。目前来看,加法原理是一个简单而基础的概念,用于指导我们如何对可分解为不相交部分的集合进行计数。通常,应用加法原理最困难的部分在于识别合适的划分,这需要一定的创造性思考。

下一个计数原理同样实用,甚至可能更为强大,但它的定义和证明会稍微复杂一些。