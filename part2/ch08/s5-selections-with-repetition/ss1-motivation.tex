% !TeX root = ../../../book.tex

\subsection{引言}

当我们推导\emph{排列}和\emph{选择}数量的公式时,我们会特意指出是否允许\textbf{重复}对象。当时,我们没有推导带有重复对象的选择数量公式。具体来说,我们需要先掌握一种叫做``双法计数''的技术,然后再解决这个问题。现在,我们已经准备好了!\\

\begin{example}
    假设我的厨房灶台上有一盒水果,里面有很多苹果、香蕉和桃子。假设每种水果至少有 $10$ 个。我伸手进去,拿出 $5$ 个水果,准备上学期间吃。我可能拿到多少种不同的组合?

    在这个例子中,我们假设任何两个苹果对我来说是\emph{无法区分的}。它们没有颜色异常或大小差异。根据这个假设,这个问题涉及从 $3$ \textbf{种}水果中\textbf{选择} $5$ 个。这里的结果是\emph{无序的}(所以这是一个选择问题,不是排列问题),并且我们允许重复选择任何一种水果(即我可以选择多个香蕉)。

    例如,我们可以选择 $4$ 个苹果和 $1$ 个桃子,或者 $5$ 个香蕉,或者 $1$ 个苹果、$2$ 个香蕉和 $2$ 个桃子。

    这代表了这类问题的最一般形式。给定若干种\textbf{类型},如何从这些类型中选择一定数量的总物体?在找到这类问题的公式之前,让我们再看一个例子。事实上,这就是我们将用来推导公式的解释!
\end{example}

\begin{example}
    假设我们有 $n$ 个相同的金币需要分配给 $k$ 个不同的海盗。我们有多少种分配方式呢?

    可以先试着用较小的 $n$ 和 $k$ 来计算一下。实际上,你可以拿一些硬币找朋友来实操一下这个问题。如果你有 $5$ 个金币和 $3$ 个朋友,你能用多少种方式分配这些金币呢?
    
    要记住,海盗是\emph{不同的}。例如,给红胡子 $2$ 个金币,给黑胡子 $10$ 个金币与给红胡子 $10$ 个金币,给黑胡子 $2$ 个金币是不同的结果,我们应该分别计算这些情况。
    
    还要记住,金币是\emph{相同的}。这意味着我们如何分配这些金币,或者按什么顺序分配,都不重要,重要的是最终的分配结果。例如,给红胡子 $5$ 个金币,然后给黑胡子 $5$ 个,再给红胡子 $2$ 个金币,这与直接给红胡子 $7$ 个金币,给黑胡子 $5$ 个金币是一样的结果。我们应该将这些情况视为相同的结果。
\end{example}

试着用这些例子来找出一个解决这些问题的公式。你能把它推广到任意的 $n$ 和 $k$ 吗?你能证明你的结论吗?试试看!然后,阅读下一节看看我们的公式和证明。