% !TeX root = ../../../book.tex

\subsection{引言}

在推导\emph{排列}和\emph{选择}数量的公式时,我们通常会明确指出是否允许对象\textbf{重复}。当时,我们并未推导允许重复对象的选择数量公式。具体来说,我们需要先掌握一种称为``双法计数''的技术,然后再来解决这个问题。现在,我们已经准备好了!

\begin{example}
    假设厨房灶台上有一盒水果,其中包含苹果、香蕉和桃子若干。假设每种水果至少有 $10$ 个。我伸手进去,取出 $5$ 个水果,作为上学期间的食物。我可能得到多少种不同的组合?

    在这个例子中,我们假设任何两个苹果都是\emph{无法区分的}。它们没有颜色或大小的差异。根据这个假设,这个问题涉及从 $3$ \textbf{种}水果中\textbf{选择} $5$ 个。这里的结果是\emph{无序的}(因此这是一个选择问题,而非排列问题),并且我们允许重复选择同一种水果(例如,我可以选择多个香蕉)。

    例如,我们可以选择 $4$ 个苹果和 $1$ 个桃子,或 $5$ 个香蕉,或 $1$ 个苹果、$2$ 个香蕉和 $2$ 个桃子。

    这代表了这类问题的最一般形式。给定若干种\textbf{类型},如何从这些类型中选择一定数量的物品?在推导这类问题的公式之前,让我们再考虑一个例子。实际上,我们将用这个例子来推导公式!
\end{example}

\begin{example}
    假设有 $n$ 个相同的金币需要分配给 $k$ 个不同的海盗,那么共有多少种分配方式呢?

    我们可以先尝试用较小的 $n$ 和 $k$ 来计算。例如,你可以用硬币和朋友来模拟这个问题:如果你有 $5$ 个金币和 $3$ 个朋友,你能找出多少种不同的分配方式?
    
    需要注意的是,海盗是\emph{不同的}。例如,给红胡子 $2$ 个金币、给黑胡子 $1$ 个金币,与给红胡子 $1$ 个金币、给黑胡子 $2$ 个金币是不同的分配方式,应当分别计算。

    另外,金币是\emph{相同的}。这意味着分配金币的顺序或过程并不重要,重要的是每个海盗最终获得的金币数量。例如,先给红胡子 $1$ 个金币,再给黑胡子 $1$ 个金币,然后又给红胡子 $1$ 个金币,实际上相当于红胡子最终得到 $2$ 个金币、黑胡子得到 $1$ 个金币。因此,无论分配顺序如何,只要最终各海盗所得金币数量相同,就应视为同一种分配方式。
\end{example}

请尝试通过这些例子推导出解决此类问题的公式。你能将结论推广到任意的 $n$ 和 $k$ 吗?能否证明你的结论?动手试一试吧!然后,阅读下一节来对照我们的公式和证明。