% !TeX root = ../../../book.tex

\subsection{公式}

我们将通过海盗分金币的例子来推导重复选择的公式。首先,解释为什么这与重复选择类似:

假设在这个场景中,我们是\emph{黄金分配者}。我们坐在一张桌子旁,周围坐着 $k$ 个海盗,旁边有一个装满 $n$ 个金币的袋子。我们可以选择如何一个一个地分配这些金币。当我们选择将一个金币分给某个编号为 $i$ 的海盗($i \in [k]$)时,我们实际上是在 $k$ 个不同海盗中做选择。最终,要分配 $n$ 枚金币,我们需要进行 $n$ 次选择。因此,我们总共从 $k$ 种类型中选择 $n$ 个对象,并且允许重复选择。

\subsubsection*{推导}

想象一下,桌子上排着 $n$ 枚金币。为了在 $k$ 个海盗之间分配这些金币,我们需要将金币分成 $k$ 堆。然后,海盗 $1$ 拿走最左边的一堆,海盗 $2$ 拿走下一堆,依此类推。

通过``分割线''论证,我们可以方便地计算出完成这个过程的方法数!要将 $n$ 枚金币分配给 $k$ 个不同的海盗,需要用 $k - 1$ 个分隔线将 $n$ 枚金币分开。为什么只需要 $k - 1$ 个分隔线呢?想象一下,我们只需要 $1$ 个分隔线就可以把一排硬币分成两堆。同理,$2$ 个分隔线可以将硬币分成 $3$ 堆。一般来说,$k - 1$ 个分隔线,可以将金币分成 $k$ 堆,不需要在最右边再放一个分隔线来表示第 $k$ 个海盗的那堆。\\

\begin{example}
    例如,当 $k=3, n=7$ 时,我们可能会得到下面的分配方案:
    \[\circ \mid \circ \circ \circ \circ \mid \circ \circ\]
    这种情况下,海盗 $1$ 获得 $1$ 枚金币,海盗 $2$ 获得 $4$ 枚金币,海盗 $3$ 获得 $2$ 枚金币。

    需要注意的是,这种分配方案与下面的分配方案是\emph{不同的}:
    \[\circ \circ \circ \circ \mid \circ \circ \mid \circ\]
    这种情况下,海盗 $1$ 获得 $4$ 枚金币,海盗 $2$ 获得 $2$ 枚金币,海盗 $3$ 获得 $1$ 枚金币。

    我们也可以让某些海盗获得 $0$ 枚金币:
    \[\circ \circ \circ \circ \circ \mid \circ \circ \mid\]
    这种情况下,海盗 $1$ 获得 $5$ 枚金币,海盗 $2$ 获得 $2$ 枚金币,海盗 $3$ 获得 $0$ 枚金币。
\end{example}

这些观察结果告诉了我们什么呢?实际上,这意味着任何金币的分配方式都可以对应一个长度为 $n + k - 1$ 的字符串,其中包含 $n$ 个金币和 $k - 1$ 个分隔线。这两个对象(金币分配和分隔线位置)的集合之间存在\emph{双射}的关系。给定一个金币分配方案,我们可以构建出相应的分隔线排列。(例如,如果我们知道海盗 $1$ 得到 $5$ 枚金币,海盗 $2$ 得到 $2$ 枚金币,海盗 $3$ 得到 $0$ 枚金币,我们就可以构建出前面例子中的分隔线排列。)同理,给定一个分隔线排列,我们也可以从中得出对应的金币分配方案。

这种双射关系告诉我们,要计算金币分配方案的数量,只需要计算可能的分隔线排列数量。我们可以很容易地计算出这些排列的数量!一个分隔线排列就是一个长度为 $n + k - 1$ 的字符串,其中包含 $k - 1$ 个分隔线。这是因为我们需要 $n$ 个金币和 $k - 1$ 个分隔线,总共需要 $n + k - 1$ 个位置。因此,根据选择的定义,有
\[\begin{pmatrix}n+k-1 \\k-1\end{pmatrix}\]
种方法来构建这样的排列!

(你可能也听说过``星条''论证法。它只是这个问题的另一种常见解释,其中金币被替换成了星星,分隔线被替换成了长条。)

由于这些排列方式与金币分配方案之间存在双射关系,我们可以得出结论:$\big({n+k-1 \atop k-1}\big)$ 就是分配金币的方法数!

我们已经知道一般情况下 $\big({n \atop k}\big) = \big({n \atop n-k}\big)$,因此可以应用到这里,推导出金币分配方案的数量也等于
\[\begin{pmatrix}n+k-1 \\n\end{pmatrix}\]
实际上,我们在推导过程中已经看到了这个结论。我们需要构造一个长度为 $n+k-1$ 的字符串,其中 $k-1$ 个位置是分隔线(其余位置是金币)。换句话说,我们需要一个长度为 $n+k-1$ 的字符串,其中 $n$ 个位置是金币(其余位置是分隔线)。
