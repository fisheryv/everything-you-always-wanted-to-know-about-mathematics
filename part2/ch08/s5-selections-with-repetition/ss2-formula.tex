% !TeX root = ../../../book.tex

\subsection{公式}

我们通过海盗分金币的例子来推导重复选择的公式。首先,解释为什么这个例子与重复选择类似:

假设在这个场景中,我们是\emph{金币分配者}。我们坐在一张桌子旁,周围有 $k$ 个海盗,旁边放着一个装满 $n$ 枚金币的袋子。我们可以逐个分配这些金币。每次选择将一枚金币分给编号为 $i$ 的海盗(其中 $i \in [k]$),就相当于从 $k$ 个不同海盗中做出一次选择。分配 $n$ 枚金币需要进行 $n$ 次选择,因此我们实际上是从 $k$ 种类型中重复选择 $n$ 个对象。

\subsubsection*{推导}

想象一下,桌面上排列着 $n$ 枚金币。为了将这些金币分配给 $k$ 个海盗,我们需要把金币分成 $k$ 堆。然后,海盗 $1$ 取走最左边的一堆,海盗 $2$ 取走下一堆,依此类推。

通过``分割线''论证,可以方便地计算完成分配的方法数!要将 $n$ 枚金币分配给 $k$ 个不同的海盗,我们需要用 $k - 1$ 个分隔线将金币分开。为什么只需要 $k - 1$ 个分隔线?例如,仅用 $1$ 个分隔线就能将一排金币分成两堆。类似地,$2$ 个分隔线可以将金币分成 $3$ 堆。一般地,$k - 1$ 个分隔线可以将金币分成 $k$ 堆,无需在最右侧额外添加分隔线来表示第 $k$ 个海盗的那一堆。

\begin{example}
    例如,当 $k=3, n=7$ 时,可能得到如下分配方案:
    \[\circ \mid \circ \circ \circ \circ \mid \circ \circ\]

    此时,海盗 $1$ 获得 $1$ 枚金币,海盗 $2$ 获得 $4$ 枚金币,海盗 $3$ 获得 $2$ 枚金币。

    需要注意的是,这种分配方案与以下方案是\emph{不同的}:
    \[\circ \circ \circ \circ \mid \circ \circ \mid \circ\]

    在这种情况下,海盗 $1$ 获得 $4$ 枚金币,海盗 $2$ 获得 $2$ 枚金币,海盗 $3$ 获得 $1$ 枚金币。

    某些海盗也可能获得 $0$ 枚金币:
    \[\circ \circ \circ \circ \circ \mid \circ \circ \mid\]
    
    此时,海盗 $1$ 获得 $5$ 枚金币,海盗 $2$ 获得 $2$ 枚金币,海盗 $3$ 获得 $0$ 枚金币。
\end{example}

这些观察结果揭示了什么?实际上,这意味着每一种金币分配方案都可以对应一个长度为 $n + k - 1$ 的字符串,其中包含 $n$ 个金币和 $k - 1$ 个分隔线。在金币分配方案与分隔线排列的集合之间,存在一种\emph{双射}关系。给定一个金币分配方案,我们可以构造出对应的分隔线排列。(例如,如果已知海盗 $1$ 获得 $5$ 枚金币,海盗 $2$ 获得 $2$ 枚金币,海盗 $3$ 获得 $0$ 枚金币,我们就可以构建出如前例所示的分隔线排列。)反之,给定一个分隔线排列,我们也能还原出相应的金币分配方案。

这种双射关系表明,计算金币分配方案的数量等价于计算可能的分隔线排列数量。而排列的数量很容易计算!一个分隔线排列本质上是一个长度为 $n + k - 1$ 的字符串,其中包含 $k - 1$ 个分隔线。因为我们需要放置 $n$ 个金币和 $k - 1$ 个分隔线,总共占用 $n + k - 1$ 个位置。因此,根据选择的定义,共有
\[{n+k-1 \choose k-1}\]
种方式构造这样的排列!

(你可能也听说过``星条''论证法。它只是同一问题的另一种常见表述,其中金币被替换为星星,分隔线被替换为长条。)

由于排列方式与金币分配方案之间存在双射关系,我们可以得出结论:${n+k-1 \choose k-1}$ 即为分配金币的总方法数!

我们已知 ${n \choose k} = {n \choose n-k}$,将其应用于上式,可以推导出金币分配方案的数量也等于
\[{n+k-1 \choose n}\]
实际上,这一结论在推导过程中已有所体现。我们构造长度为 $n+k-1$ 的字符串时,既可以选择 $k-1$ 个位置放置分隔线(其余放金币),也可以选择 $n$ 个位置放置金币(其余放分隔线)。
