% !TeX root = ../../../book.tex

\subsection{等价形式}

在使用这个新公式解决问题之前,让我们先看看基本\emph{重复选择}问题的一些\textbf{等价形式}。当你遇到涉及这些概念或形式的问题时,都可以考虑使用我们刚刚推导出的公式来解决。

\subsubsection*{海盗分金币}

这是我们推导公式时使用的原始公式,所以在这种情况下肯定适用。通常,我们只需要知道海盗的数量和金币的数量。

\begin{tcolorbox}[colback=blue!10,
    colframe=blue,
    width=\textwidth,
    arc=2mm, auto outer arc,
    breakable,enhanced jigsaw,
    before upper={\parindent15pt\noindent},	]
    将 $n$ 枚相同的金币分配给 $k$ 个不同的海盗有 ${n+k-1 \choose k-1}$ 种方法。
\end{tcolorbox}

在我们的推导中,有一个隐含的假设,那就是海盗可能会得到 $0$ 枚金币,请注意这一点。有些问题可能会要求你考虑其他分配\emph{条件}。例如,如果每个海盗必须\emph{至少}获得 $1$ 枚金币,又会怎样呢?

\subsubsection*{整数求和}

我们可以将``海盗分金币''问题重新表述如下。

定义 $x_i \in \mathbb{N} \cup \{0\}$ 为海盗 $i$ 分配到的金币数量。问题的条件要求
\[\forall i \in [k] \centerdot x_i \in \mathbb{N} \cup \{0\}\]
并且
\[\sum_{i=1}^{k}x_i = x_1 + x_2 + x_3 + \dots + x_k = n\]

如果我们关注这种方程\textbf{解}的数量会怎样呢?这实际上(以\emph{双射}的方式)正是解决``海盗分金币''问题的方法。只要有一个方程的解,我们就能给每个海盗 $i$ 分配恰好 $x_i$ 个金币。这就提供了一种新的问题描述方法:

\begin{tcolorbox}[colback=blue!10,
    colframe=blue,
    width=\textwidth,
    arc=2mm, auto outer arc,
    breakable,enhanced jigsaw,
    before upper={\parindent15pt\noindent},	]
    方程 $x_1 + x_2 + x_3 + \dots + x_k = n$ 的解中,满足条件 $\forall i \in [k] \centerdot x_i \in \mathbb{N} \cup \{0\}$ 的解的数量为 ${n+k-1 \choose k-1}$。
\end{tcolorbox}

\subsubsection*{球与桶}

假设我们有 $n$ 个相同的球,现在需要把它们放入 $k$ 个不同的桶中(这些桶是可区分的,我们可以将它们标记为 $1$ 到 $k$)。那么,有多少种方法可以完成这个任务呢?这实际上可以很容易地与之前的公式联系起来。设 $x_i \in \mathbb{N} \cup \{0\}$ 表示最终进入第 $i$ 个桶的球的数量。这样,我们可以应用与之前问题相同的条件。

\begin{tcolorbox}[colback=blue!10,
    colframe=blue,
    width=\textwidth,
    arc=2mm, auto outer arc,
    breakable,enhanced jigsaw,
    before upper={\parindent15pt\noindent},	]
    将 $n$ 个相同的球分配到 $k$ 个不同的桶的方法数量为 ${n+k-1 \choose k-1}$。
\end{tcolorbox}

\subsubsection*{不可区分骰子}

考虑掷 $n$ 个相同的骰子。会有多少种可能的结果呢?这与掷不同颜色的可区分骰子不同。在这里,每次掷骰子的结果是一个\textbf{无序的}数字列表。

例如,如果我们掷 $3$ 个不可区分的 $6$ 面骰子,结果可能是无序的 $(1, 3, 3)$。想象一下,你的朋友在另一个房间掷了 $3$ 个骰子,然后回来告诉你结果。如果他说``我掷出了一个 $1$ 和两个 $3$'',你不会知道每个数字分别是哪个骰子掷出来的。(相比之下,如果他说``第一个骰子是 $1$,第二个和第三个骰子是 $3$'',那你就知道了具体的顺序。)这里我们关心的是有多少种不同的无序结果。

我们可以将这种情况与``球与桶''问题联系起来:将所有骰子掷出后,根据显示的数字将它们放入编号为 $1$ 到 $6$ 的桶中。换句话说,要描述这个过程的结果,我们只需要知道有多少个骰子显示了 $1$,有多少个显示了 $2$,等等,而不需要知道具体哪个骰子显示了哪个数字。

\begin{tcolorbox}[colback=blue!10,
    colframe=blue,
    width=\textwidth,
    arc=2mm, auto outer arc,
    breakable,enhanced jigsaw,
    before upper={\parindent15pt\noindent},	]
    掷 $n$ 个不可区分的 $k$ 面骰子的结果数量为 ${n+k-1 \choose k-1}$。
\end{tcolorbox}