% !TeX root = ../../../book.tex

\subsection{等价形式}

在应用这个新公式解决问题之前,我们先来探讨基本\emph{重复选择}问题的一些\textbf{等价形式}。当你遇到涉及这些概念或形式的问题时,可以考虑使用我们刚刚推导出的公式。

\subsubsection*{海盗分金币}

这是我们推导公式时使用的原始场景,因此公式在此处直接适用。通常,我们只需知道海盗的数量和金币的数量。

\begin{tcolorbox}[colback=blue!10,
    colframe=blue,
    width=\textwidth,
    arc=2mm, auto outer arc,
    breakable,enhanced jigsaw,
    before upper={\parindent15pt\noindent}]
    将 $n$ 枚相同的金币分配给 $k$ 个不同的海盗,共有 ${n+k-1 \choose k-1}$ 种方法。
\end{tcolorbox}

推导过程中有一个隐含假设:海盗可能分得 $0$ 枚金币,这一点需要注意。某些问题可能附加其他分配\emph{条件},例如要求每个海盗必须\emph{至少}获得 $1$ 枚金币。

\subsubsection*{整数求和}

我们可以将``海盗分金币''问题重新表述如下。

定义 $x_i \in \mathbb{N} \cup \{0\}$ 为海盗 $i$ 分配到的金币数量。问题的条件要求
\[\forall i \in [k] \centerdot x_i \in \mathbb{N} \cup \{0\} \quad \text{且} \quad \sum_{i=1}^{k}x_i = x_1 + x_2 + x_3 + \dots + x_k = n\]

如果我们关注该方程\textbf{解}的数量会怎样呢?实际上,通过\emph{双射}的方式,此问题对应于``海盗分金币''问题的方法。每个方程的解都对应一种分配方式,即给海盗 $i$ 分配恰好 $x_i$ 枚金币。因此,我们可以这样描述问题:

\begin{tcolorbox}[colback=blue!10,
    colframe=blue,
    width=\textwidth,
    arc=2mm, auto outer arc,
    breakable,enhanced jigsaw,
    before upper={\parindent15pt\noindent}]
    方程 $x_1 + x_2 + x_3 + \dots + x_k = n$ 在条件 $\forall i \in [k] \centerdot x_i \in \mathbb{N} \cup \{0\}$ 下的解的数量为 ${n+k-1 \choose k-1}$。
\end{tcolorbox}

\subsubsection*{球与桶}

假设有 $n$ 个相同的球,要放入 $k$ 个不同的桶中(桶是可区分的,例如标记为 $1$ 到 $k$)。那么,有多少种分配方法?这可以轻松关联到之前的公式。设 $x_i \in \mathbb{N} \cup \{0\}$ 表示第 $i$ 个桶中球的数量,这样我们就能应用相同的条件。

\begin{tcolorbox}[colback=blue!10,
    colframe=blue,
    width=\textwidth,
    arc=2mm, auto outer arc,
    breakable,enhanced jigsaw,
    before upper={\parindent15pt\noindent}]
    将 $n$ 个相同的球分配到 $k$ 个不同的桶的方法数量为 ${n+k-1 \choose k-1}$。
\end{tcolorbox}

\subsubsection*{不可区分骰子}

考虑掷 $n$ 个相同的骰子,可能产生多少种结果?这与掷不同颜色的可区分骰子不同,因为这里的结果是一个\textbf{无序}数列。

例如,掷 $3$ 个不可区分的 $6$ 面骰子,可能得到无序列表 $(1, 3, 3)$。假设你的朋友在另一个房间掷了 $3$ 个骰子后告诉你:``我掷出了一个 $1$ 和两个 $3$。''你无法知道具体哪个骰子显示哪个数字。相比之下,如果他说``第一个骰子是 $1$,第二个和第三个是 $3$'',你就知道了具体顺序。这里我们关心的是不同无序结果的数量。

我们可以将这种情况与``球与桶''问题联系起来:将骰子根据显示的数字放入编号为 $1$ 到 $6$ 的桶中。换句话说,描述结果只需知道有多少个骰子显示了 $1$,多少个显示了 $2$……而不需知道具体哪个骰子显示了哪个数字。

\begin{tcolorbox}[colback=blue!10,
    colframe=blue,
    width=\textwidth,
    arc=2mm, auto outer arc,
    breakable,enhanced jigsaw,
    before upper={\parindent15pt\noindent}]
    掷 $n$ 个不可区分的 $k$ 面骰子的结果数量为 ${n+k-1 \choose k-1}$。
\end{tcolorbox}