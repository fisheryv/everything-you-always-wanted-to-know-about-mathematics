% !TeX root = ../../../book.tex

\subsection{示例}

让我们使用这个新推导出的公式来解决一些问题吧!在这个过程中,我们将探讨这个基本结论的几种不同表述。

\begin{example}
      假设我们有 $20$ 枚金币要分给 $3$ 个海盗。这些海盗分别是红胡子船长 (Khair ad Din,一位奥斯曼人)、黑胡子船长 (Edward Teach,一位英国人) 和基德船长 (Captain Kidd,一位苏格兰人)。

      现在我们来计算在某种特定条件下,有多少种金币分配方法:
      \begin{enumerate}[label=(\arabic*)]
            \item 总共有多少种方法?\\
                  这就像从 $3$ 种类型中选择 $20$ 个对象,并且允许重复选择。每当我们选择一个海盗,就意味着我们给了他一枚金币。\\
                  根据上述选择公式,有
                  \[{20+3-1 \choose 20}={22 \choose 20}=\frac{22 \cdot 21}{2}=231\]
                  种分配方法。
            \item 如何确保每个海盗至少得到两枚金币?\\
                  我们可以先给每个海盗两枚金币,这样总共发出了 $6$ 枚金币。接下来,我们还有 $20 - 6 = 14$ 枚金币要分配给所有 $3$ 个海盗,因此有
                  \[{14+3-1 \choose 14}={16 \choose 14}=\frac{16 \cdot 15}{2}=120\]
                  种分配方法。想想为什么这样做是有效的。实际上,我们是在重新定义``得到 $0$ 枚金币''的含义。与其从 $20$ 枚金币开始并担心是否每个海盗都能至少分得两枚金币,不如直接确保这个条件成立,然后再分配剩下的金币。
            \item 如何确保红胡子船长和黑胡子船长至少得到 $2$ 枚金币,而基德船长至少得到 $6$ 枚金币?
                  跟上面的方法一样,我们首先给红胡子船长和黑胡子船长 每人分配 $2$ 枚金币,再给基德船长分配 $6$ 枚金币。这样,我们还剩下 $20 - 2 \cdot 2 - 6 = 10$ 枚金币需要分配给三位海盗,因此有
                  \[{10+3-1 \choose 10}={12 \choose 10}=\frac{12 \cdot 11}{2}=66\]
                  种分配方法。
            \item 有多少种方法可以确保红胡子船长和黑胡子船长至少得到 $2$ 枚金币,而基德船长最多得到 $2$ 枚金币?\\
                  有两种方法可以解决这个问题。
                  \begin{enumerate}[label=(\roman*)]
                        \item 我们可以根据基德船长获得 $0$ 枚、$1$ 枚或 $2$ 枚金币来划分不同的情况。在每种情况下,我们会先给 红胡子船长和黑胡子船长各 $2$ 枚金币,然后再给基德船长相应数量($0$ 或 $1$ 或 $2$)的金币。这样一来,我们就剩下 $16$ 或 $15$ 或 $14$ 枚金币需要在前两个海盗之间分配,因此总共有
                              \[{16+2-1 \choose 16}+{15+2-1 \choose 15}+{14+2-1 \choose 14}=17+16+15=48\]
                              种分配方法。
                        \item 我们首先考虑\emph{所有}能确保红胡子船长和黑胡子船长每人至少得到 $2$ 枚金币的方法,然后再\emph{去除}基德船长得到太多金币(即至少 $3$ 枚金币)的情况。

                              如果我们给红胡子船长和黑胡子船长每人 $2$ 枚金币,那么还剩下 $16$ 枚金币可以分配给这 $3$ 个海盗,因此有
                              \[{16+3-1 \choose 16}={18 \choose 16}=\frac{18 \cdot 17}{2}=153\]
                              种分配方法。\\

                              然后,如果我们给红胡子船长和黑胡子船长每人 $2$ 枚金币,给基德船长 $3$ 枚金币,再将剩下的 $13$ 枚金币分配给这 $3$ 个海盗,这一步共有
                              \[{13+3-1 \choose 13}={15 \choose 13}=\frac{15 \cdot 14}{2}=105\]
                              种分配方法。\\

                              我们需要从之前的计数中去除这些情况。因此,总共有
                              \[{18 \choose 16}-{15 \choose 13}=153 - 105 = 48\]
                              种方法可以确保红胡子船长和黑胡子船长每人至少得到 $2$ 枚金币,而基德船长最多得到 $2$ 枚金币。
                  \end{enumerate}
                  (看,我们用两种方法得到了相同的答案!)
      \end{enumerate}
\end{example}

\begin{example}
      考虑如下方程:
      \[x_1 + x_2 + x_3 + x_4 + x_5 = 25\]
      我们来找出这个方程的解的数量,其中每个变量 $x_i$ 都是非负整数。我们会施加一些条件,并计算满足这些条件的解的数量。
      \begin{enumerate}[label=(\arabic*)]
            \item 总共有多少组解?\\
                  利用我们推导出的公式可得
                  \[{25+(5-1) \choose 5-1}={29 \choose 4}=23751\]
            \item 有多少解满足 $x_1 \ge 4$?
                  这就像要求红胡子船长至少分得 $4$ 枚金币一样。我们将预先分配 $4$ 个``计数''给变量 $x_1$,然后再把剩下的 $21$ 个``计数''分配给所有五个变量。

                  更正式地,我们定义 $y1 = x_1 - 4$,这样条件只要求 $y_1 \ge 0$。因此,我们要解的方程变为
                  \begin{align*}
                        x_1 + x_2 + x_3 + x_4 + x_5 = 25 & \iff (x_1-4) + x_2 + x_3 + x_4 + x_5 = 21 \\
                                                         & \iff y_1 + x_2 + x_3 + x_4 + x_5 = 21
                  \end{align*}
                  应用公式可得
                  \[{21+(5-1) \choose 5-1}={25 \choose 4}=12650\]
            \item 有多少解满足 $x_1, x_2 \ge 5$ 且 $x_3, x_4, x_5 \ge 2$?
                  采用与上一问相同的方法,我们发现只需解方程
                  \[y_1 + y_2 + y_3 + y_4 + y_5 = 9\]
                  其中,对于 $i=1,2, y_i = x_i-5$,对于 $i=3,4,5, y_i = x_i-2$。等式右边变为 $25 - 5 - 5 - 2 - 2 - 2 = 9$。根据公式可得
                  \[{9 +(5-1) \choose 5-1} = {13 \choose 4}=715\]
            \item 有多少解满足 $x_2 \le 5$?
                  我们可以用两种方法来解决这个问题。

                  第一种方法:我们取解的总数量(本例第一问已经得到),然后去除不符合期望条件的解的数量。也就是说,$x_2 \ge 6$ 的解的数量,即
                  \[{(25-6)+(5-1) \choose 5-1}={23 \choose 4}\]
                  从总数中减去它,得到
                  \[{29 \choose 4}-{23 \choose 4}\] \\

                  第二种方法:我们可以将这个问题写成一个求和公式,找到满足 $x_2 = \ell$ 的解的数量,其中 $0 le \ell le 5$:
                  \begin{align*}
                        \sum_{\ell=0}^{5} {25+(4-1)-\ell \choose 4-1} & = {28 \choose 3}+{27 \choose 3}+{26 \choose 3}+{25 \choose 3}+{24 \choose 3}+{23 \choose 3} \\
                                                                      & = \sum_{k=0}^{28}{k \choose 3}-\sum_{k=0}^{22}{k \choose 3}                                 \\
                                                                      & = {29 \choose 4}-{23 \choose 4}
                  \end{align*}
                  跟第一种方法的结果完美对上。
      \end{enumerate}
\end{example}