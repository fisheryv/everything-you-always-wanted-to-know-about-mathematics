% !TeX root = ../../../book.tex

\subsection{示例}

现在,让我们运用这个新推导的公式来解决一些问题。通过这个过程,我们将探讨该基本结论的几种不同表述。\\

\begin{example}
      假设有 $20$ 枚金币需要分给 $3$ 位海盗,他们分别是红胡子船长(Khair ad Din,一个奥斯曼人)、黑胡子船长(Edward Teach,一个英国人)和基德船长(Captain Kidd,一个苏格兰人)。

      接下来,我们计算在特定条件下金币分配的方案数:
      \begin{enumerate}[label=(\arabic*)]
            \item 总共有多少种分配方案?\\
                  这相当于从 $3$ 种类型中重复选取 $20$ 个对象。每选取一位海盗,就相当于分配给他一枚金币。
                  根据前述公式,分配方案数为
                  \[{20+3-1 \choose 20}={22 \choose 20}=\frac{22 \cdot 21}{2}=231\] 
            \item 如何确保每位海盗至少获得两枚金币?\\
                  我们可以先分配给每位海盗两枚金币,共发出 $6$ 枚金币。剩余 $20 - 6 = 14$ 枚金币需要分配给 $3$ 位海盗,因此分配方案数为
                  \[{14+3-1 \choose 14}={16 \choose 14}=\frac{16 \cdot 15}{2}=120\]
                  想想为什么这种做法是合理的?实际上,我们重新定义了``得到 $0$ 枚金币''的含义。与其从 $20$ 枚金币开始,并时刻担心是否满足每位海盗都至少分得两枚金币,不如通过预先分配来满足最低要求,再处理剩余金币的分配,从而避免了直接处理约束条件的复杂性。
            \item 如何确保红胡子船长和黑胡子船长各获得至少 $2$ 枚金币,而基德船长获得至少 $6$ 枚金币?\\
                  类似地,先分配给红胡子船长和黑胡子船长各 $2$ 枚金币,基德船长 $6$ 枚金币。剩余金币为 $20 - 2 \times 2 - 6 = 10$ 枚,需要分配给三位海盗,因此分配方案数为
                  \[{10+3-1 \choose 10}={12 \choose 10}=\frac{12 \cdot 11}{2}=66\]
            \item 有多少种方案可以确保红胡子船长和黑胡子船长各获得至少 $2$ 枚金币,而基德船长最多获得 $2$ 枚金币?\\
                  这个问题有两种解法。
                  \begin{enumerate}[label=(\roman*)]
                        \item 第一种方法:按基德船长获得 $0$ 枚、$1$ 枚或 $2$ 枚金币的情况分类讨论。在每种情况下,先分配给红胡子船长和黑胡子船长各 $2$ 枚金币,再分配给基德船长相应数量的金币。剩余金币数分别为 $16$、$15$ 或 $14$ 枚,需要在前两位海盗之间分配。因此总分配方案数为
                              \[{16+2-1 \choose 16}+{15+2-1 \choose 15}+{14+2-1 \choose 14}=17+16+15=48\]

                        \item 第二种方法:先计算\emph{所有}满足红胡子船长和黑胡子船长各获得至少 $2$ 枚金币的方法,再\emph{减去}基德船长获得至少 $3$ 枚金币的情况。

                              首先,分配给红胡子船长和黑胡子船长各 $2$ 枚金币后,剩余 $16$ 枚金币可任意分配给三位海盗,分配方案数为
                              \[{16+3-1 \choose 16}={18 \choose 16}=\frac{18 \cdot 17}{2}=153\]

                              然后,若基德船长获得至少 $3$ 枚金币,则先分配给他 $3$ 枚金币,同时红胡子船长和黑胡子船长仍各得 $2$ 枚金币。剩余 $13$ 枚金币可以任意分配给三位海盗,分配方案数为
                              \[{13+3-1 \choose 13}={15 \choose 13}=\frac{15 \cdot 14}{2}=105\]

                              用前者减去后者,得到满足条件的分配方案数为
                              \[{18 \choose 16}-{15 \choose 13}=153 - 105 = 48\]
                  \end{enumerate}

                  (可见,两种方法得到了相同的结果!)
      \end{enumerate}
\end{example}

\begin{example}
      考虑如下方程:
      \[x_1 + x_2 + x_3 + x_4 + x_5 = 25\]
      我们来计算该方程非负整数解的数量,并探讨在附加不同条件下的情况。
      \begin{enumerate}[label=(\arabic*)]
            \item 总共有多少组解?\\
                  利用我们推导出的公式可得
                  \[{25+(5-1) \choose 5-1}={29 \choose 4}=23751\] 
            \item 有多少组解满足 $x_1 \ge 4$?\\
                  这类似于要求红胡子船长至少获得 $4$ 枚金币。我们预先分配 $4$ 个``计数''给变量 $x_1$,再将剩余的 $21$ 个``计数''分配给所有五个变量。

                  具体地,定义 $y1 = x_1 - 4$,则条件变为 $y_1 \ge 0$。因此,原方程转化为
                  \begin{align*}
                        x_1 + x_2 + x_3 + x_4 + x_5 = 25 & \iff (x_1-4) + x_2 + x_3 + x_4 + x_5 = 21 \\
                                                         & \iff y_1 + x_2 + x_3 + x_4 + x_5 = 21
                  \end{align*}
                  应用公式可得
                  \[{21+(5-1) \choose 5-1}={25 \choose 4}=12650\] 
            \item 有多少组解满足 $x_1, x_2 \ge 5$ 且 $x_3, x_4, x_5 \ge 2$?\\
                  采用与上一问相同的方法,对于 $i=1,2$,定义 $y_i = x_i - 5$;对于 $i=3,4,5$,定义 $y_i = x_i - 2$。等式右边变为 $25 - 5 - 5 - 2 - 2 - 2 = 9$,因此,原方程转化为
                  \[y_1 + y_2 + y_3 + y_4 + y_5 = 9\]
                  根据公式可得
                  \[{9 +(5-1) \choose 5-1} = {13 \choose 4}=715\] 
            \item 有多少组解满足 $x_2 \le 5$?\\
                  我们可以用两种方法来求解。
                  \begin{enumerate}[label=(\roman*)]
                        \item 第一种方法:从解的总数(本例第一问已得)中减去不满足条件的解的数量。其中不满足条件的解的数量,即满足 $x_2 \ge 6$ 的解的数量为
                        \[{(25-6)+(5-1) \choose 5-1}={23 \choose 4}\]
                        相减得
                        \[{29 \choose 4}-{23 \choose 4}\] 

                        \item 第二种方法:将问题改写为求和公式,找到满足 $x_2 = \ell$ 的解的数量,其中 $0 \le \ell \le 5$
                        \begin{align*}
                              \sum_{\ell=0}^{5} {25+(4-1)-\ell \choose 4-1} & = {28 \choose 3}+{27 \choose 3}+{26 \choose 3}+{25 \choose 3}+{24 \choose 3}+{23 \choose 3} \\
                                                                        & = \sum_{k=0}^{28}{k \choose 3}-\sum_{k=0}^{22}{k \choose 3}                                 \\
                                                                        & = {29 \choose 4}-{23 \choose 4}
                        \end{align*}
                  \end{enumerate}
                  结果与第一种方法完全一致。
      \end{enumerate}
\end{example}
\clearpage