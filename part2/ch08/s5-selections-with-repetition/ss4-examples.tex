% !TeX root = ../../../book.tex

\subsection{示例}

让我们使用这个新推导出的公式来解决一些问题吧!在这个过程中,我们将探讨这个基本结论的几种不同表述。\\

\begin{example}
    假设我们有 $20$ 枚金币要分给 $3$ 个海盗。这些海盗分别是红胡子船长 (Khair ad Din,一位奥斯曼人)、黑胡子船长 (Edward Teach,一位英国人) 和基德船长 (Captain Kidd,一位苏格兰人)。

    现在我们来计算在某种特定条件下,有多少种金币分配方法:
    \begin{enumerate}[label=(\arabic*)]
        \item 总共有多少种方法?\\
              这就像从 $3$ 种类型中选择 $20$ 个对象,并且允许重复选择。每当我们选择一个海盗,就意味着我们给了他一枚金币。\\
              根据上述选择公式,有
              \[\begin{pmatrix}20+3-1\\20\end{pmatrix}=\begin{pmatrix}22\\20\end{pmatrix}=\frac{22 \cdot 21}{2}=231\]
              种分配方法。
        \item 如何确保每个海盗至少得到两枚金币?\\
              我们可以先给每个海盗两枚金币,这样总共发出了 $6$ 枚金币。接下来,我们还有 $20 - 6 = 14$ 枚金币要分配给所有 $3$ 个海盗,因此有
              \[\begin{pmatrix}14+3-1\\14\end{pmatrix}=\begin{pmatrix}16\\14\end{pmatrix}=\frac{16 \cdot 15}{2}=120\]
              种分配方法。想想为什么这样做是有效的。实际上,我们是在重新定义``得到 $0$ 枚金币''的含义。与其从 $20$ 枚金币开始并担心是否每个海盗都能至少分得两枚金币,不如直接确保这个条件成立,然后再分配剩下的金币。
        \item 如何确保红胡子船长和黑胡子船长至少得到 $2$ 枚金币,而基德船长至少得到 $6$ 枚金币?
              跟上面的方法一样,我们首先给红胡子船长和黑胡子船长 每人分配 $2$ 枚金币,再给基德船长分配 $6$ 枚金币。这样,我们还剩下 $20 - 2 \cdot 2 - 6 = 10$ 枚金币需要分配给三位海盗,因此有
              \[\begin{pmatrix}10+3-1\\10\end{pmatrix}=\begin{pmatrix}12\\10\end{pmatrix}=\frac{12 \cdot 11}{2}=66\]
              种分配方法。
        \item 有多少种方法可以确保红胡子船长和黑胡子船长至少得到 $2$ 枚金币,而基德船长最多得到 $2$ 枚金币?\\
              有两种方法可以解决这个问题。
              \begin{enumerate}[label=(\roman*)]
                  \item 我们可以根据基德船长获得 $0$ 枚、$1$ 枚或 $2$ 枚金币来划分不同的情况。在每种情况下,我们会先给 红胡子船长和黑胡子船长各 $2$ 枚金币,然后再给基德船长相应数量($0$ 或 $1$ 或 $2$)的金币。这样一来,我们就剩下 $16$ 或 $15$ 或 $14$ 枚金币需要在前两个海盗之间分配,因此总共有
                        \[\begin{pmatrix}16+2-1\\16\end{pmatrix}+\begin{pmatrix}15+2-1\\15\end{pmatrix}+\begin{pmatrix}14+2-1\\14\end{pmatrix}=17+16+15=48\]
                        种分配方法。
                  \item 我们首先考虑\emph{所有}能确保红胡子船长和黑胡子船长每人至少得到 $2$ 枚金币的方法,然后再\emph{去除}基德船长得到太多金币(即至少 $3$ 枚金币)的情况。

                        如果我们给红胡子船长和黑胡子船长每人 $2$ 枚金币,那么还剩下 $16$ 枚金币可以分配给这 $3$ 个海盗,因此有
                        \[\begin{pmatrix}16+3-1\\16\end{pmatrix}=\begin{pmatrix}18\\16\end{pmatrix}=\frac{18 \cdot 17}{2}=153\]
                        种分配方法。\\

                        然后,如果我们给红胡子船长和黑胡子船长每人 $2$ 枚金币,给基德船长 $3$ 枚金币,再将剩下的 $13$ 枚金币分配给这 $3$ 个海盗,这一步共有
                        \[\begin{pmatrix}13+3-1\\13\end{pmatrix}=\begin{pmatrix}15\\13\end{pmatrix}=\frac{15 \cdot 14}{2}=105\]
                        种分配方法。\\

                        我们需要从之前的计数中去除这些情况。因此,总共有
                        \[\begin{pmatrix}18\\16\end{pmatrix}-\begin{pmatrix}15\\13\end{pmatrix}=153 - 105 = 48\]
                        种方法可以确保红胡子船长和黑胡子船长每人至少得到 $2$ 枚金币,而基德船长最多得到 $2$ 枚金币。
              \end{enumerate}
              (看,我们用两种方法得到了相同的答案!)
    \end{enumerate}
\end{example}

\begin{example}
    考虑如下方程:
    \[x_1 + x_2 + x_3 + x_4 + x_5 = 25\]
    我们来找出这个方程的解的数量,其中每个变量 $x_i$ 都是非负整数。我们会施加一些条件,并计算满足这些条件的解的数量。
    \begin{enumerate}[label=(\arabic*)]
        \item 总共有多少组解?\\
              利用我们推导出的公式可得
              \[\begin{pmatrix}25+(5-1)\\5-1\end{pmatrix}=\begin{pmatrix}29\\4\end{pmatrix}=23751\]
        \item 有多少解满足 $x_1 \ge 4$?
              这就像要求红胡子船长至少分得 $4$ 枚金币一样。我们将预先分配 $4$ 个``计数''给变量 $x_1$,然后再把剩下的 $21$ 个``计数''分配给所有五个变量。

              更正式地,我们定义 $y1 = x_1 - 4$,这样条件只要求 $y_1 \ge 0$。因此,我们要解的方程变为
              \begin{align*}
                  x_1 + x_2 + x_3 + x_4 + x_5 = 25 & \iff (x_1-4) + x_2 + x_3 + x_4 + x_5 = 21 \\
                                                   & \iff y_1 + x_2 + x_3 + x_4 + x_5 = 21
              \end{align*}
              应用公式可得
              \[\begin{pmatrix}21+(5-1)\\5-1\end{pmatrix}=\begin{pmatrix}25\\4\end{pmatrix}=12650\]
        \item 有多少解满足 $x_1, x_2 \ge 5$ 且 $x_3, x_4, x_5 \ge 2$?
              采用与上一问相同的方法,我们发现只需解方程
              \[y_1 + y_2 + y_3 + y_4 + y_5 = 9\]
              其中,对于 $i=1,2, y_i = x_i-5$,对于 $i=3,4,5, y_i = x_i-2$。等式右边变为 $25 - 5 - 5 - 2 - 2 - 2 = 9$。根据公式可得
              \[\begin{pmatrix}9 +(5-1)\\5-1\end{pmatrix} = \begin{pmatrix}13\\4\end{pmatrix}=715\]
        \item 有多少解满足 $x_2 \le 5$?
              我们可以用两种方法来解决这个问题。

              第一种方法:我们取解的总数量(本例第一问已经得到),然后去除不符合期望条件的解的数量。也就是说,$x_2 \ge 6$ 的解的数量,即
              \[\begin{pmatrix}(25-6)+(5-1)\\5-1\end{pmatrix}=\begin{pmatrix}23\\4\end{pmatrix}\]
              从总数中减去它,得到
              \[\begin{pmatrix}29\\4\end{pmatrix}-\begin{pmatrix}23\\4\end{pmatrix}\] \\

              第二种方法:我们可以将这个问题写成一个求和公式,找到满足 $x_2 = \ell$ 的解的数量,其中 $0 le \ell le 5$:
              \begin{align*}
                  \sum_{\ell=0}^{5} \begin{pmatrix}25+(4-1)-\ell\\4-1\end{pmatrix} & = \begin{pmatrix}28\\3\end{pmatrix}+\begin{pmatrix}27\\3\end{pmatrix}+\begin{pmatrix}26\\3\end{pmatrix}+\begin{pmatrix}25\\3\end{pmatrix}+\begin{pmatrix}24\\3\end{pmatrix}+\begin{pmatrix}23\\3\end{pmatrix} \\
                                                                                   & = \sum_{k=0}^{28}\begin{pmatrix}k\\3\end{pmatrix}-\sum_{k=0}^{22}\begin{pmatrix}k\\3\end{pmatrix}                                                                                                             \\
                                                                                   & = \begin{pmatrix}29\\4\end{pmatrix}-\begin{pmatrix}23\\4\end{pmatrix}
              \end{align*}
              跟第一种方法的结果完美对上。
    \end{enumerate}
\end{example}