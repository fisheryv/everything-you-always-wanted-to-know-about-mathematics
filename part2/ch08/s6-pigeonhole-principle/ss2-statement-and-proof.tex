% !TeX root = ../../../book.tex

\subsection{陈述与证明}

抽屉原理有两个版本,我们将分别陈述并证明它们。第一个版本在组合问题中较为常用。

\begin{theorem}{抽屉原理}
    \begin{enumerate}[label=(\arabic*)]
        \item 设集合 $S$ 的大小为 $|S| = n$,将 $S$ 划分为 $k$ 个互不相交且并集为 $S$ 的子集。若 $k < n$,则划分中至少有一个子集的元素数量大于 $1$。实际上,该子集至少包含 $\lceil \frac{n}{k} \rceil$ 个元素。\\
              (也就是说,如果我们将 $n$ 个物体分成 $k$ 堆,则必定有一堆至少包含 $\lceil \frac{n}{k} \rceil$ 个物体。)
        \item 若 $x_1, x_2, \dots , x_n$ 为实数,且满足 $\sum_{i=1}^{n} x_i \ge z$,则至少存在一个 $i$ 使得 $x_i \ge \frac{z}{n}$。\\
              (也就是说,如果我们有 $n$ 个实数,则至少有一个数不小于它们的平均值。)
    \end{enumerate}
\end{theorem}

\begin{proof}
    设 $k < n$,且集合 $S$ 被划分为子集 $S_1, \dots, S_k$。为了引出矛盾而假设对于每个 $i$,有 $|S_i| < \frac{n}{k}$。由于这些子集构成 $S$ 的一个划分,因此
    \[n = |S| = \sum_{i=1}^{k}|S_i| < \sum_{i=1}^{k} \frac{n}{k}=n\]

    这就得到了 $n < n$,显然矛盾。$\hashx$\\

    为了引出矛盾而假设所有数字 $x_i$ 都满足 $x_i < \frac{z}{n}$。则
    \[z = \sum_{i=1}^{n}x_i<\sum_{i=1}^{n}\frac{z}{n}=n \cdot \frac{z}{n}=z\]
    
    这就得到了 $z < z$,显然矛盾。$\hashx$
\end{proof}

值得注意的是,这些证明非常相似!从代数角度看,它们几乎是一模一样的。事实上,它们表达了相同的基本思想。