% !TeX root = ../../../book.tex

\subsection{习题}

\subsubsection*{温故知新}

以口头或书面的形式简要回答以下问题。这些问题全都基于你刚刚阅读的内容,所以如果忘记了具体的定义、概念或示例,可以回去重读相关部分。确保在继续学习之前能够自信地回答这些问题,这将有助于你的理解和记忆!

\begin{enumerate}[label=(\arabic*)]
    \item 抽屉原理的两个版本是什么?
    \item 我们使用了哪种证明方法来证明抽屉原理?
\end{enumerate}

\subsubsection*{小试牛刀}

尝试回答以下问题。这些题目要求你实际动笔写下答案,或(对朋友/同学)口头陈述答案。目的是帮助你练习使用新的概念、定义和符号。题目都比较简单,确保能够解决这些问题将对你大有帮助!

\begin{enumerate}[label=(\arabic*)]
    \item 假设数学系有 $5$ 位教授。每年从中选出 $2$ 位教授来讲数学概念课程。该系可以在不重复选择相同两位教授的情况下持续多少年?通过展示这样一个长度的序列来证明这是最优的,并且利用抽屉原理证明任何更长的序列必然会重复使用某一对教授。
    \item 设 $n \in \mathbb{N}$,考虑集合 $[2n]$。假设我们有一个大小为 $|S|=n+1$ 的集合 $S \subseteq [2n]$。证明在 $S$ 中必定存在两个元素 $x,y \in S$ 是\emph{互质的}。
    \item 假设我们有一个边长为 $1$ 公里的正方形公园。我们想在公园里建一个高尔夫球场,但我们只能设置 $5$ 个洞。特别地,出于安全原因,我们需要考虑球洞(地面上的洞)之间的距离。
    
    证明无论我们如何放置 $5$ 个球洞,必定存在两个球洞之间的距离 $d$ 满足 $d \le \frac{\sqrt{2}}{2}$ 公里。

    (注意:球洞可以设置在公园的边界上。)
    
    接下来,证明这个界是\emph{最优的};也就是说,展示一种在公园地面上设置 $5$ 个球洞的方法(同样允许设置在边界上),使得任意两个洞之间的距离大于或等于 $d \le \frac{\sqrt{2}}{2}$ 公里。
\end{enumerate}