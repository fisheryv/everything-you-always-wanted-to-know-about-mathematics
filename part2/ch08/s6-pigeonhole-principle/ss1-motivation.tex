% !TeX root = ../../../book.tex

\subsection{引言}

这是我们之前提到的一个结论。实际上,在我们初次学习证明技巧时,就已经证明了它的一个特殊\emph{版本}(参见示例 \ref{ex:example4.9.2})。其基本思想可以概括如下:
\begin{center}
    如果我们要将过多的``东西''放进过少的``盒子''里,\\那么至少有一个盒子里会装下很多``东西''。
\end{center}
这个解释虽然相当不正式,但有助于理解它的用处。

抽屉原理 (Pigeonhole Principle) 非常有用。例如,当我们将一组对象划分到若干类别时,如果对象的数量超过类别的数量,那么可以保证至少有一个类别包含至少两个对象。

\begin{example}
    以下是一个典型应用:
    \begin{center}
        \textbf{在任意三人中,至少有两人性别相同。}
    \end{center}

    注意,这里并未说明\emph{哪种}性别至少出现两次,而只是保证至少有一种性别出现至少两次。为了验证这一点,可以列举所有可能的性别组合(其中 \verb|M| 代表男性,\verb|F| 代表女性):\verb|MMM, MMF, MFF, FFF|。在每一种情况下,至少有一种性别出现两次或更多次。\\
    上述陈述的一个逻辑等价形式是:
    \begin{center}
        \textbf{如果我们抛 $3$ 枚硬币,至少有两枚会显示相同的面。}
    \end{center}

    另一个类似的陈述是:
    \begin{center}
        \textbf{如果我们掷 $7$ 个骰子,至少有两个骰子会显示相同的点数。}
    \end{center}

    你是否理解其中的一般模式?以下是另一个版本,我们将用它过渡到下一节,下一节中我们将陈述并证明广义抽屉原理。
    \begin{center}
        \textbf{如果我们将 $n + 1$ 张纸放入 $n$ 个不同的抽屉,那么至少有一个抽屉中会被放入至少 $2$ 张纸。}
    \end{center}
    
    顺便一提,上述陈述正是``抽屉''一词的来源:有些人也称之为``鸽笼原理''。这里的``鸽笼 (Pigeonhole)'' 是一个术语,指老式写字台上的小开口,类似于信箱的投递口,而不是真的要将鸽子这种温顺的生物塞进狭窄的笼子里!
\end{example}