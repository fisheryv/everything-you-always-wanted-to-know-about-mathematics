% !TeX root = ../../../book.tex

\subsection{引言}

这是我们之前提到过的一个结论。其实,早在我们初次学习证明技巧时,就已经证明了它的一个特定\emph{版本}(参见示例 \ref{ex:example4.9.2})。基本思想是这样的:
\begin{quote}
    如果我们有太多的``东西''要放进太少的``盒子''里,那么某个盒子里肯定会装很多``东西''。
\end{quote}
这个解释虽然非常地不正式,但可以帮助你理解它的用处。

抽屉原理 (Pigeonhole Principle) 很有用,例如,当我们将一组对象分到若干类别时。如果我们知道对象的数量和类别的数量,那么可以保证至少有一个类别包含一定数量的对象。

\begin{example}
    这里有一个应用该原理的典型例子:
    \begin{quotation}
        \textbf{在任意三个人中,至少有两个人的性别是相同的。}
    \end{quotation}
    注意,这并没有说明\emph{哪种}性别至少出现两次。它只是保证了这种类别的\emph{存在}。为了验证这个事实,你可以列举所有可能的情况(其中 \verb|M| 代表男性,\verb|F| 代表女性):\verb|MMM, MMF, MFF, FFF|。在每种情况下,至少有一种性别出现两次或更多。\\
    下面是上述陈述的逻辑等价版本:
    \begin{quotation}
        如果我们抛 $3$ 枚硬币,至少两枚硬币会出现相同的面。
    \end{quotation}
    上面的陈述还有一个类似的陈述:
    \begin{quotation}
        如果我们掷 $7$ 个骰子,至少两个骰子会出现相同的点字。
    \end{quotation}
    你开始理解其一般思路了吗?这里是这些声明的另一个版本,并过渡到下一部分,我们将在其中陈述并证明一个广义版本。
    \begin{quotation}
        如果我们将 $n + 1$ 张纸塞入 $n$ 个不同的抽屉,最终某个抽屉一定会塞入至少 $2$ 张纸。
    \end{quotation}
    顺带一提,上面的陈述正是``抽屉''一词的来源:有些人也称之为``鸽笼原理'',这里的鸽笼 (Pigeonhole) 是一个抽屉术语,你会在某些老式写字台上见到,像信箱的出入口一样。而不是真的要将鸽子这种温柔的生物塞进狭窄的笼子里!
\end{example}