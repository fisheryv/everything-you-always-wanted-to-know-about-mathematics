% !TeX root = ../../../book.tex

\subsection{示例}

让我们直接开始,看看如何在组合数学问题中应用抽屉原理。我们会通过一些练习示例来展示它的具体应用。通常来说,使用抽屉原理最难的部分是确定什么是``抽屉''。

\begin{example}
    $8$ 个人中,一定会有两个人的生日在一星期的同一天。类似地,$13$ 个人中,必然会有两个人的生日在同一个月。

    对于第一个论断,我们可以将``抽屉''比作一周的 $7$ 天。如果我们将 $8$ 个人按他们今年生日是星期几进行分类,就会发现有 $8$ 个人分到 $7$ 天中。因此,某一天至少会有 $\frac{8}{7}$ 个人。因为我们处理的是整数,这意味着至少有一天会有 $2$ 个人过生日。

    对于第二个论断,可以用类似的推理。我们只需将一年的 $12$ 个月作为我们的``抽屉''。
\end{example}

\begin{example}
    在纽约,至少有 $8$ 个人的头发数量是完全一样的。

    这基于以下几个事实。首先,科学家估计人类头上的头发数量在 $10$ 万到 $15$ 万之间。为了保守起见,我们将这个范围扩大到 $0$ 到$100$ 万。其次,纽约市大约有 $800$ 万人。通过将我们的``抽屉''定义为 $0$ 到 $100$ 万这个范围(基于每个人头上的头发数量),我们得出上述结论。

    (事实上,这个论证可能多此一举。我敢说,我们只需在城市里走一走,很快就能找到 $8$ 个秃头的人!)
\end{example}

\begin{example}
    回顾 \ref{sec:section1.4.4} 节,我们探讨了如何在一群人中找到一组互为好友的问题。在解决这个问题时,我们实际上应用了抽屉原理!我们将 $5$ 个元素任意分成 $2$ 类,由此推断出至少有一类包含 $3$ 个元素。
\end{example}

\begin{example}
    假设有 $n$ 名高尔夫球手($n \ge 2$)参加一场循环赛。他们会进行多少场比赛?在这些比赛结束后,是否必然会有两名球手的胜负记录完全相同?如果不是,有没有什么条件可以保证这一点?

    通过计算,我们可以得出会进行 $\frac{n(n-1)}{2}$ 场比赛。(为什么?你能补充细节吗?试试看!)但是,我们不能保证两个人的战绩是相同的。例如,假设 $n = 3$,球员 $1$ 输给了其他两人,球员 $2$ 赢了球员 $1$ 但输给了球员 $3$,而球员 $3$ 赢了其他两人。这样分别产生了 \verb|0-2|、\verb|1-1| 和 \verb|2-0| 的记录,我们可以看到没有任何两人的记录是相同的。

    现在,如果我们施加一个条件,即没有人是全胜的,那么我们可以保证有两名球员的记录是相同的。每个球员会进行 $n-1$ 场比赛(每个球员与除自己外的所有人比赛一次)。由于没有人是全胜的,所以没有人会有 $n-1$ 胜。因此,每个球员可能的胜场数是 $0,1,2, \dots, n-2$。这里有 $n-1$ 种可能。根据抽屉原理,在 $n$ 名球员中,必然有两名球员的胜场数是一样的。
\end{example}

\begin{example}
    \textbf{命题}:``在任意 $m$ 个不同自然数组成的集合中,至少有两个数的和或差是 $10$ 的倍数。''

    找出使该命题成立的最小 $m$ 值。

    通过尝试一些较小的数值,我们发现当 $m \le 6$ 时,该命题不成立。即使 $m = 6$,我们可以构造集合 $\{1, 2, 3, 4, 5, 10\}$。注意,该集合中的数字中没有两个数的和或差是 $10$ 的倍数。(注意:不能重复选择数字,比如 $5 - 5 = 0$ 或 $5 + 5 = 10$,以得到 $10$ 的倍数。)

    那么 $m = 7$ 可能是我们要找的数字吗?让我们试着证明一下!

    假设我们有任意 $7$ 个自然数组成的集合。我们将它们按最后一位数字分配到抽屉中(即,将每个数字 $n$ 依据满足 $x \equiv n \mod 10$ 的最小正数 $x$ 放入对应抽屉),具体如下:
    \[\{1, 9\}, \{2, 8\}, \{3, 7\}, \{4, 6\}, \{5\}, \{0\}\]
    也就是说,我们有 $6$ 个抽屉。

    由于我们有 $7$ 个数字,所以必定有一个抽屉里至少有两个数字。这意味着这些数字的最后一位数字之和是 $10$ 的倍数(例如 $2$ 和 $8$ 或 $5$ 和 $5$),或者这些数字有相同的最后一位数字,因此它们的差是 $10$ 的倍数。无论哪种情况,我们都有一个和或差是 $10$ 的倍数。
\end{example}