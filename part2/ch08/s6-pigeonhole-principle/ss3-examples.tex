% !TeX root = ../../../book.tex

\subsection{示例}

让我们直接开始,看看如何在组合数学问题中应用抽屉原理。我们将通过一些具体示例来展示它的应用。通常来说,使用抽屉原理最困难的部分在于确定什么是``抽屉''。

\begin{example}
    $8$ 个人中,一定会有两个人的生日在一星期的同一天。类似地,$13$ 个人中,必然会有两个人的生日在同一个月。

    对于第一个论断,我们可以将``抽屉''比作一周的 $7$ 天。如果我们将 $8$ 个人按照他们今年生日是星期几进行分类,就会发现有 $8$ 个人被分配到 $7$ 天中。因此,至少有一天会有 $\frac{8}{7}$ 个人。由于我们处理的是整数,这意味着至少有一天会有 $2$ 个人过生日。

    对于第二个论断,可以采用类似的推理。我们只需将一年的 $12$ 个月作为我们的``抽屉''。
\end{example}

\begin{example}
    在纽约,至少有 $8$ 个人的头发数量是完全一样的。

    这基于以下几个事实。首先,科学家估计人类头上的头发数量在 $10$ 万到 $15$ 万之间。为了保守起见,我们将这个范围扩大到 $0$ 到 $100$ 万。其次,纽约市大约有 $800$ 万人。通过将``抽屉''定义为头发数量从 $0$ 到 $100$ 万这个范围,我们得出上述结论。

    (事实上,这个论证可能有些多余。我敢说,我们只需在城市里走一走,很快就能找到 $8$ 个秃头的人!)
\end{example}

\begin{example}
    回顾 \ref{sec:section1.4.4} 节,我们探讨了如何在一群人中找到一组互为好友的问题。在解决这个问题时,我们实际上应用了抽屉原理!我们将 $5$ 个元素任意分成 $2$ 类,由此推断出至少有一类包含 $3$ 个元素。
\end{example}

\begin{example}
    假设有 $n$ 名高尔夫球手($n \ge 2$)参加一场循环赛。他们会进行多少场比赛?比赛结束后,是否必然有两名球手的胜负记录完全相同?如果不是,有没有什么条件可以保证这一点?

    通过计算,我们可以得出比赛总数为 $\frac{n(n-1)}{2}$ 场。(为什么?你能补充细节吗?试试看!)但是,我们不能保证有两名球手的记录相同。例如,当 $n = 3$ 时,球员 $1$ 输给其他两人,球员 $2$ 赢球员 $1$ 但输给球员 $3$,球员 $3$ 赢其他两人。这样,记录分别为 \verb|0-2|、\verb|1-1| 和 \verb|2-0|,没有任何两人的记录相同。

    现在,如果我们施加一个条件,即没有人全胜,那么可以保证有两名球员的记录相同。每名球员进行 $n-1$ 场比赛。由于没有人全胜,所以没有人取得 $n-1$ 胜。因此,每名球员可能的胜场数为 $0, 1, 2, \dots, n-2$,共有 $n-1$ 种可能。根据抽屉原理,在 $n$ 名球员中,必然有两名球员的胜场数相同。
\end{example}

\begin{example}
    \textbf{命题}:``在任意 $m$ 个不同自然数中,至少有两个数的和或差是 $10$ 的倍数。''

    找出使该命题成立的最小 $m$ 值。

    通过尝试一些较小的数值,我们发现当 $m \le 6$ 时,命题不成立。即使 $m = 6$,我们可以构造集合 $\{1, 2, 3, 4, 5, 10\}$。注意,该集合中任意两个数的和或差都不是 $10$ 的倍数。(注意:不能重复选择数字,例如通过 $5 - 5 = 0$ 或 $5 + 5 = 10$ 来得到 $10$ 的倍数。)

    那么 $m = 7$ 是否就是我们要找的最小值呢?让我们试着证明一下!

    假设我们有任意 $7$ 个自然数组成的集合。我们将它们按最后一位数字分配到抽屉中(即,将每个数字 $n$ 依据满足 $x \equiv n \mod{10}$ 的最小正数 $x$ 放入对应抽屉),具体如下:
    \[\{1, 9\}, \{2, 8\}, \{3, 7\}, \{4, 6\}, \{5\}, \{0\}\]
    也就是说,共有 $6$ 个抽屉。

    由于我们有 $7$ 个数字,根据抽屉原理,必定至少有一个抽屉包含至少两个数字。这意味着要么这两个数字的最后一位数字之和为 $10$(例如 $2$ 和 $8$),要么它们最后一位数字相同(例如 $5$ 和 $5$),因此它们的和或差是 $10$ 的倍数。
\end{example}