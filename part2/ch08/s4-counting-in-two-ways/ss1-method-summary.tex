% !TeX root = ../../../book.tex

\subsection{方法概述}

\subsubsection*{为何有效}

让我们提升一个层次,讨论一种叫做``双法计数''的证明技巧。首先,我们来探讨它\emph{为何}有效以及\emph{如何}应用。接着,我们会通过几个例子进一步说明。在上一节的结尾,我们已经初步探讨了其原理,这里我们将再次阐述这些观点。``双法计数''这个名称非常贴切,因为它直接说明了这种策略!任何使用这种技巧的证明,都会先确定一个有限元素的集合,并提供\emph{两种}计数这些元素的方法。通过加法和乘法原理,以及其他组合数学的结论,这两种方法会对同一个数(即该有限集的\emph{基数})得出不同的代数表达式。

一个好的证明会明确指出要计数的有限集及其元素的两种不同计数方法,然后通过等式将这两种代数表达式关联。因此,通过这种方法证明的结果通常会涉及二项式系数、求和以及其他代数表达式的\emph{恒等式}或\emph{方程}。证明的关键在于通过计数论证清晰地解释这些表达式,而不是单纯地进行代数简化。

看看我们刚刚证明的结论:是的,我们可以直接验证 ${n \choose k}= \frac{n!}{k!(n-k)!} = {n \choose n-k}$,但那有什么乐趣呢?这根本不能算作证明,无论如何解释其含义,都没有提供任何关于结论为何成立的见解。此外,随着我们研究越来越多这类挑战性问题,代数验证变得相当困难,有时几乎是不可能的!

\subsubsection*{如何使用}

我们将在本节稍后展示几个例子(包括反例),但首先我们要介绍``双法计数''的概要。这将为我们提供一个标准,用来衡量未来这种风格的证明;我们可以通过阅读这些证明,确保它们遵循结构、清晰度和正确性的要点。我们将遵守这些标准,并希望你也能这样做。我们还会向你介绍一些在``双法计数''证明中使用的标准组合对象,并且在展示例子时,我们会指出何时该考虑使用特定的对象集进行计数。

下面给出的是一个\emph{好的``双法计数''证明}的基本结构!

\begin{enumerate}
    \item 陈述要证明的结论。(注意:记得对表达式中的所有变量进行量化!)
    \item 定义集合 $S$,表示需要计数的对象。
    \item 用一种组合论证方法计算 $S$ 中元素的数量,并将结果记为 $|S|$。
    \item 用另一种组合论证方法计算 $S$ 中元素的数量,并将结果记为 $|S|$。
    \item 最后得出结论,由于两种方法得到的结果都等于 $|S|$,因此它们必然相等。
\end{enumerate}

就是这样!正如我们所言,这个证明技术的名字就是证明技巧本身,所以很容易记忆。然而,多年来我们阅读了许多这样的证明,发现了一些常见错误。我们在这里列出了这些常见错误。想一想,为什么做这些错误会导致证明变得``糟糕''?每个错误未能满足好的证明的哪个属性?是正确性?还是清晰性?抑或是简洁性?

\subsubsection*{常见错误}

\begin{itemize}
    \item 忘记定义需要计数的对象集合。
    \item 定义了对象集合,但用两种方法计数了其他集合。
    \item 计数了对象集合,但随后用另一种方法计数了别的对象集合。
    \item 在结论中未能将两个表达式建立恒等关系。
\end{itemize}
在该证明技术之外,其他错误也可能出现在实际的组合证明中,这些错误也要留意!
