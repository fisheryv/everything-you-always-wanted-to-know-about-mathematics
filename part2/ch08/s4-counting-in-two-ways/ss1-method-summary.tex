% !TeX root = ../../../book.tex

\subsection{方法概述}

\subsubsection*{为何有效}

让我们从更高的视角来探讨一种名为``双法计数''的证明技巧。首先,我们将分析它\emph{为何}有效以及\emph{如何}应用;接着,通过若干实例进一步阐释这一方法。在上一节末尾,我们已经初步涉及其基本原理,这里将对这些观点作进一步展开。``双法计数''这一名称直指策略核心:任何采用此方法的证明都会先确定一个有限元素集合,并给出\emph{两种}不同的计数方式。基于加法原理、乘法原理以及其他组合数学工具,这两种方式会对同一数值(即该有限集的\emph{基数})推导出不同的代数表达式。

一个优秀的证明会明确指定待计数的有限集及其元素的两种计数方法,进而通过等式将两种代数表达式联系起来。因此,这类证明所得结果往往涉及二项式系数、求和符号以及其他代数表达式的\emph{恒等式}或\emph{方程}。证明的关键在于借助计数论证清晰阐释这些表达式的含义,而非仅仅依赖代数化简。

回顾我们此前证明的结论:诚然,我们可以直接验证 ${n \choose k}= \frac{n!}{k!(n-k)!} = {n \choose n-k}$,但那有什么乐趣呢?那根本算不上真正的证明——无论怎样解释其形式,都未能揭示结论背后的内在原理。更何况,随着我们面对的问题日益复杂,代数验证往往会变得异常繁琐,甚至几乎无法完成!

\subsubsection*{如何使用}

我们将在本节后续部分展示若干示例(包括反例),但首先要介绍``双法计数''的基本框架。这将为我们提供一个标准,用以评估此类证明的优劣;我们可以通过阅读这些证明,确保它们满足结构清晰、表述明确和推理正确的要求。我们将遵循这些标准,并希望读者也能做到这一点。同时,我们将介绍一些在``双法计数''证明中常用的组合对象,并在示例中说明何时应考虑使用特定的对象集合进行计数。

以下是一个\emph{良好的``双法计数''证明}应具备的基本结构:

\begin{enumerate}
    \item 明确陈述待证明的结论。(注意:务必对表达式中的所有变量进行量化!)
    \item 定义集合 $S$,表示所要计数的对象。
    \item 通过一种组合论证方法计算 $S$ 中元素的数量,记为 $|S|$。
    \item 通过另一种组合论证方法计算 $S$ 中元素的数量,同样记为 $|S|$。
    \item 得出结论:由于两种方法所得结果均等于 $|S|$,因此这两个表达式必然相等。
\end{enumerate}

以上就是全部要点!正如其名,这一证明技巧的核心在于计数方法本身,因此易于记忆。然而,在多年阅读此类证明的过程中,我们发现了一些常见错误。下面列出这些错误,请思考它们为何会导致证明变得``糟糕''?每个错误违背了良好证明的哪个特性?是正确性、清晰性,还是简洁性?

\subsubsection*{常见错误}

\begin{itemize}
    \item 未定义所要计数的对象集合。
    \item 定义了对象集合,但在计数时使用了其他集合。
    \item 对计数对象集合进行计数后,转而用另一种方法计数其他集合。
    \item 在结论中未将两个表达式建立恒等关系。
\end{itemize}
除了上述错误外,在实际的组合证明中也可能出现其他问题,这些错误也要留意!
