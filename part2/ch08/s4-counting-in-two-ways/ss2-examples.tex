% !TeX root = ../../../book.tex

\subsection{示例}

让我们仔细看几个例子。这将帮助你理解如何应用``双法计数''技术,并为你提供一些典型示例供你参考和复习,此外还介绍了一些可用于未来问题的基本组合数学结论。在每个例子中,我们不仅证明所讨论的结论,还解释如何推导证明、构建论证时的思考过程,以及你如何自己尝试解决类似问题。双法计数证明的一个优点是,在证明结束时,我们通常能简洁地总结主要思想。我们将为每个证明总结核心思想,并鼓励你在完成类似证明后也进行同样的总结。这样,证明的思想更容易记忆,且只需一句话就能重构整个证明。\\

\begin{proposition}{帕斯卡恒等式 (Pascal's Identity)}
    对于任意 $n,k \in \mathbb{N}$,
    \[{n \choose k}={n-1 \choose k}+{n-1 \choose k-1}\]
\end{proposition}

\begin{questions}{证明策略:}
    看到像 ${n \choose k}$ 这样的二项式系数,表明我们可能想要计算 $[n]$ 中特定大小的子集。公式左边很容易理解(计算所有大小为 $k$ 的子集),但右边该如何解释?看到两项相加,表明这里存在某种划分。我们需要找到 $[n]$ 中大小为 $k$ 的子集的某个特性,使得一部分子集具有该特性,而另一部分没有。注意到两项的唯一区别在于``底部系数'',我们可以据此找到一个有效的划分……在继续阅读之前,尝试自己思考一下!
\end{questions}

\begin{proof}
    设 $S = \{T \subseteq [n] \mid |T| = k\}$。根据 $k$-选择的定义,我们知道 $|T| = {n \choose k}$。

    定义以下两个集合:
    \begin{align*}
        A & = \{T \subseteq [n] \mid |T| = k ∧ 1 \in T\}    \\
        B & = \{T \subseteq [n] \mid |T| = k ∧ 1 \notin T\}
    \end{align*}

    显然,$A \cap B = \varnothing$,因为对于任意集合 $T$,$1 \in T$ 和 $1 \notin T$ 不可能同时成立。并且 $S = A \cup B$,因为对于任意集合 $T \in S$,要么 $1 \in T$ 要么 $1 \notin T$。因此 $\{A,B\}$ 构成 $S$ 的一个划分,由此可知 $|S| = |A| + |B|$。

    为了计算 $|A|$,我们可以通过一个两步过程来构造任意元素 $T \in A$:
    \begin{enumerate}[label=(\arabic*)]
        \item 将元素 $1$ 加入 $T$;
        \item 从剩余的 $n-1$ 个元素中选择 $k-1$ 个元素加入 $T$。
    \end{enumerate}

    由乘法原理可得:
    \[|A| = 1 \cdot {n-1 \choose k-1} = {n-1 \choose k-1}\]

    同理,为了计算 $|B|$,我们可以通过一个两步过程来构造任意元素 $T \in B$:
    \begin{enumerate}[label=(\arabic*)]
        \item 从 $T$ 中删除元素 $1$;
        \item 从剩余的 $n-1$ 个元素中选择 $k$ 个元素构成 $T$。
    \end{enumerate}

    由乘法原理可得:
    \[|B| = {n-1 \choose k}\]

    根据加法原理,我们有:
    \[|S|=|A|+|B| = {n-1 \choose k-1} + {n-1 \choose k}\]

    将 $|S|$ 的两个表达式用等号连结,我们得出
    \[{n \choose k}={n-1 \choose k}+{n-1 \choose k-1}\]
\end{proof}

\begin{questions}{证明总结:}
    通过将 $[n]$ 中所有 $k$ 元子集划分为包含或不包含某一特定元素(例如 $1$),从而完成对子集的计数。
\end{questions}

\begin{questions}{问题:}
    若改用元素 $n$ 而非元素 $1$ 来构造划分,证明过程是否会有所不同?答案是否定的。关键在于选取任意一个特定元素,并基于该元素定义划分集合 $A$ 和 $B$,证明结构保持不变。
\end{questions}

\begin{questions}{历史注记:}
    此命题以法国数学家布莱兹·帕斯卡 (Blaise Pascal) 命名。你可能听说过帕斯卡三角 (Pascal's Triangle),该三角由自然数排列而成,构造方法如下:首行和第二行均为 $1$,三角的左右边界均为 $1$,内部每个数等于其上方两数之和。根据刚才证明的公式,你能猜出三角中的数字吗?正是二项式系数!第 $n$ 行(从 $0$ 开始计)包含形如 $\binom{n}{k}$ 的数值,其中 $k$ 从左至右递增。帕斯卡三角具有许多有趣的性质,我们将在后续示例与练习中进一步探索。
    \begin{center}
        \begin{tabular}{ccccccccccc}
              &     &     &     &      & $1$ &      &     &     &     &     \\
              &     &     &     & $1$  &     & $1$  &     &     &     &     \\
              &     &     & $1$ &      & $2$ &      & $1$ &     &     &     \\
              &     & $1$ &     & $3$  &     & $3$  &     & $1$ &     &     \\
              & $1$ &     & $4$ &      & $6$ &      & $4$ &     & $1$ &     \\
          $1$ &     & $5$ &     & $10$ &     & $10$ &     & $5$ &     & $1$ \\
        \end{tabular}
    \end{center}
\end{questions}

\begin{proposition}{主席恒等式 (Chairperson Identity)}
    对于任意 $n,k \in \mathbb{N}$,
    \[k{n \choose k}=n{n-1 \choose k-1}\]
\end{proposition}

\begin{questions}{证明策略:}
    这个等式的两边都是两项的\emph{乘积},因此我们需要寻找两个两步过程来构造相同的元素集合。左边的项可以表示为 ${n \choose k} \cdot k$,由于乘法满足交换律,这等价于从 $[n]$ 中选择 $k$ 个元素,然后进行下一步操作。如果将 $k$ 写成 ${k \choose 1}$,第二步就变得清晰了:从已选出的 $k$ 个元素中再选择一个元素。

    这引入了一种描述子集及其特定元素的新策略:\emph{委员会 (Committee)} 和\emph{领导者 (Leader)}。这种策略在组合证明中非常流行,因为它减少了技术性数学语言和符号的使用,使关键思想更易于理解。我们将在下面的证明中展示如何运用这种策略,并与使用更多数学语言的证明进行对比。在继续阅读之前,试试看你能不能预见我们如何描述等式右边的过程……
\end{questions}

\begin{proof}
    给定 $n,k \in \mathbb{N}$。设 $S$ 为从 $n$ 个人中选出 $k$ 人组成委员会,并指定其中一人为主席的集合。一种构造集合 $S$ 中元素的方法是:先选出 $k$ 人组成委员会,然后再从中选出一人作为主席。根据乘法原理,我们可以得出
    \[|S| = {n \choose k} \cdot {k \choose 1} = k{n \choose k}\]

    另一种构造集合 $S$ 中元素的方法是:先从所有 $n$ 个人中选出主席,然后从剩余的 $n-1$ 人中选出 $k-1$ 人组成委员会。根据乘法原理,我们可以得出
    \[|S| = {n \choose 1} \cdot {n-1 \choose k-1} = n{n-1 \choose k-1}\]

    将 $|S|$ 的两个表达式用等号连结,我们得到
    \[k{n \choose k}=n{n-1 \choose k-1}\]
\end{proof}

\begin{questions}{证明总结:}
    通过先选择委员会成员再选择主席,或者先选择主席再选择其余委员会成员,计算从 $n$ 个人中选出 $k$ 人委员会的方法数。
\end{questions}

\begin{questions}{注释:}
    如果我们尝试用纯粹的数学语言(即集合论)来描述这个证明,会怎样呢?在集合论中,``主席''具体对应什么?从 $n$ 个人中选出一个大小为 $k$ 的委员会可以表示为集合 $T \subseteq [n]$,其中 $|T| = k$。但如何区分该集合中 $k$ 个成员分别担任主席的情况?一个合理的方法是定义一个有序对,其中第一个元素是委员会成员的集合,第二个元素指定主席。基于这一策略,我们定义集合:
    \[\hat{S} = \{(T, x) \mid T \subseteq [n] \land |T| = k \land x \in T\}\]
    这个集合 $\hat{S}$ 与上述证明中定义的 $S$ 相同,因为它包含了所有包含一个主席的 $k$ 人委员会。然而,在计数 $\hat{S}$ 的元素时,我们可能仍然会使用委员会和主席的口语描述!(你可以尝试不使用这些描述来计数 $\hat{S}$ 的元素。)这种描述方式更自然,也更容易理解。总之,没有必要严格地使用集合论语言描述这些委员会;但指出这种可能性是重要的。这表明我们在上述证明中的描述是足够严格的,它们基于数学概念,但用其他术语表达时更易于理解和遵循。

    本节的练习中介绍了几个涉及委员会和子委员会的双法计数证明的例子。我们在这里再举一个例子作为练习。\\
\end{questions}

\begin{proposition}{所有规模的委员会}
    \[\sum_{k=0}^{n} {n \choose k}=2^n\]
\end{proposition}

\begin{questions}{证明策略:}
    右侧的表达式可能有多种含义,但它似乎涉及一个 $n$ 步过程,每一步都有两种选择。稍后我们将详细探讨这个表达式。左侧表示一个划分,因为它是多个项之和。求和中的每一项 ${n \choose k}$ 表示从 $n$ 个人中选出 $k$ 个人组成委员会的方法数。当 $k$ 从 $0$ 变化到 $n$ 时,我们涵盖了所有可能的委员会规模。这表明我们在计算从 $n$ 个人中选出所有可能的委员会。既然我们知道右侧在计算什么,我们可以为此构建一个证明……在阅读我们的证明之前,试着自己推导一下!
\end{questions}

\begin{proof}
    设 $n \in \mathbb{N}$。并设 $S$ 为从 $n$ 个人中选出的所有规模的委员会的集合。$S$ 中的元素是规模介于 $0$ 到 $n$(含端点)人的委员会。对于每个 $k \in [n] \cup \{0\}$,设 $S_k$ 为 $k$ 人委员会的集合。则集合 $\{S_k \mid k \in [n] \cup \{0\}\}$ 构成 $S$ 的一个划分。因此,根据加法原理,我们得出
    \[|S| = \sum_{k=0}^{n} |S_k| = \sum_{k=0}^{n} {n \choose k} \]
    其中 $|S_k| = {n \choose k}$,因为 $S_k$ 是 $[n]$ 中所有 $k$ 元子集的集合。

    我们还可以通过另一种方式计算集合 $S$ 的元素数量:考虑一个包含 $n$ 个人的集合,并为他们从 $1$ 到 $n$ 编号(例如,可以给每个人分发一件印有唯一编号的 T 恤)。为了构建一个委员会,我们按数字顺序排列所有人,并沿着队伍前进,对每个人说``Yes''或``No'',以表示他们是否属于正在构建的委员会。每一个由 $n$ 个``Yes''和``No''组成的序列都会唯一对应一个委员会。由于这是一个 $n$ 步过程,每一步有两种选择,根据乘法原理,我们有 $2^n$ 种方式完成这个过程,因此 $|S| = 2^n$。通过将这两个 $|S|$ 的表达式等同起来,我们得出结论
    \[\sum_{k=0}^{n} {n \choose k}=2^n\]
\end{proof}

\begin{questions}{证明总结:}
    通过基于规模的划分来计算 $[n]$ 的所有子集。(注意:虽然这个总结使用集合术语表述,但我们认为用委员会术语来书写和理解证明会更直观。)
\end{questions}

你可能会觉得这个证明有些冗长,特别是因为我们已经通过归纳法证明了 $|\mathcal{P}([n])| = 2^n$。既然我们在考虑所有规模的委员会,实际上就是说``令 $S = \mathcal{P}([n])$'',然后用两种方法计算 $|S|$。然而,当我们使用委员会术语书写证明时,如果不解释这些表述为何等价,就不能直接切换到讨论 $[n]$ 的子集。作为练习,尝试不使用委员会术语,完全用集合符号重写这个证明。你更喜欢哪种方法?

\subsubsection*{求和恒等式}

下面的组合恒等式非常有用,将在本章后续的证明和练习中反复出现,因此我们提前在此给出结论。此外,我们将展示\emph{两种不同的}双法计数证明。该问题甚至还有第三种证明方法,留作练习由你来完成。这两种证明方法属于标准的组合证明技巧,我们鼓励你阅读并理解它们之间的联系。你可能会问,为什么要对同一结论给出两种证明?(``一个证明难道不够吗?'')通过理解这些证明的结构及其等价性,你将对这些证明技术有更深入的认识,并能更灵活地运用它们。请相信我们!此外,我们还会将这些技术与前面问题中使用的委员会方法进行比较,探讨这三种方法之间的关联。\\

\begin{theorem}{求和恒等式}\label{theorem8.4.5}
    设 $n,k \in \mathbb{N}$,则
    \[\sum_{i=0}^{n} {i \choose k} = {n+1 \choose k+1}\]
\end{theorem}

\begin{questions}{证明策略 1:}
    右侧是一个二项式系数,表明我们在讨论 $[n+1]$ 中大小为 $k+1$ 的子集。左侧的求和提示我们根据某种性质将这些子集划分成若干类。由于求和中的二项式系数底部都是 $k$ 而非 $k+1$,这意味着索引 $i$ 可能表示子集中某个特定元素(例如最大元素)的取值。在继续阅读之前,请尝试自行写出这一划分的具体细节……
\end{questions}

\begin{proofs}{证明 1.}
    设 $n,k \in \mathbb{N}$。定义
    \[S = \{T \in [n + 1] \mid |T| = k + 1\}\]
    根据 $[n+1]$ 的 $k+1$-选择定义,我们知道 $|S| = {n+1 \choose k+1}$。

    接着,对于每一个 $i \in [n] \cup \{0\}$,定义集合
    \[S_i = \{T \in S \mid i + 1 \in T \land (\forall j \in T \centerdot j \le i + 1)\}\]
    即 $S_i$ 是由 $[n + 1]$ 中所有大小为 $k + 1$ 且\emph{最大索引}元素为 $i + 1$ 的子集构成的集合。我们断言 $\{S_i \mid i \in [n] \cup \{0\}\}$ 构成 $S$ 的一个划分。

    首先,我们注意到 $S_i \cap S_j = \varnothing$ ,当且仅当 $i \ne j$。这是因为 $T \in S_i$ 意味着 $i + 1 \in T$;具体来说,若 $i > j$,则任意 $U \in S_j$ 的最大索引元素是 $j + 1$,而 $j + 1$ 小于 $i + 1$;若 $i < j$,则任意 $U ∈ S_j$ 包含 $j + 1$,但 $j + 1 \notin T$。

    其次,我们注意到每个 $T \in S$ 在 $1$ 到 $n + 1$ 之间必有一个最大索引元素,因此属于某个 $S_i$ 集合。为了更好地理解这一部分的证明,我们在下面提供了一个示例,展示了 $n = 4, k = 2$ 的情况。请注意,有几个集合是空集。通常,对于每个 $i \in [k - 1] \cup \{0\}, S_i=\varnothing$,这很合理,因为对于所有这些 $i$ 值,${i \choose k}=0$。

    接下来,我们必须找出对于每个 $i \in [n] \cup {0}, |S_i|$ 的值。为了构建元素 $T \in S_i$,我们可以定义一个两步过程:
    \begin{enumerate}[label=(\arabic*)]
        \item 将元素 $i + 1$ 加入 $T$;
        \item 从 $i$ 个较小的索引元素中,选择 $k$ 个加入 $T$。
    \end{enumerate}
    根据乘法原理和选择的定义,共有 ${i \choose k}$ 种方式。

    因此,根据加法原理,我们得出
    \[|S| = \sum_{i=0}^{n} |S_i| = \sum_{i=0}^{n} {i \choose k}\]

    将两个关于 $|S|$ 的表达式建立相等关系,我们得到
    \[\sum_{i=0}^{n} {i \choose k} = {n+1 \choose k+1}\]
\end{proofs}

\begin{tcolorbox}[colback=gray!10,
    colframe=black,
    width=\textwidth,
    arc=2mm, auto outer arc,
    title={$n = 4, k = 2$ 的示例},breakable,enhanced jigsaw,
    before upper={\parindent15pt\noindent},	]
    \begin{align*}
        S =  \big\{&\{1, 2, 3\}, \{1, 2, 4\}, \{1, 2, 5\}, \{1, 3, 4\}, \{1, 3, 5\},       \\
                   &\{1, 4, 5\}, \{2, 3, 4\}, \{2, 3, 5\}, \{2, 4, 5\}, \{3, 4, 5\}\big\}  \\
        S_1 = \enspace\; & \varnothing                                                     \\
        S_2 = \enspace\; & \varnothing                                                     \\
        S_3 = \big\{&\{1, 2, 3\} \big\}                                                    \\
        S_4 = \big\{&\{1, 2, 4\}, \{1, 3, 4\}, \{2, 3, 4\} \big\}                          \\
        S_5 = \big\{&\{1, 2, 5\}, \{1, 3, 5\}, \{1, 4, 5\}, \{2, 3, 5\}, \{2, 4, 5\}, \{3, 4, 5\} \big\}
    \end{align*}
\end{tcolorbox}

\begin{questions}{证明 1 总结:}
    通过根据子集中最大索引元素进行划分,计算 $[n+1]$ 中具有 $(k+1)$ 个元素的子集数量。
\end{questions}

这一证明策略源于我们最初的观察:二项式系数 $\binom{n+1}{k+1}$ 表示从 $[n+1]$ 中选取子集的数量。然而,我们还可以通过另一种常见的计数方法——二进制元组,来解释该系数。这将引导我们从不同角度思考左侧的求和式。现在,让我们深入探讨这一证明!

\begin{proofs}{证明 2.}
    设 $n,k \in \mathbb{N}$,并设 $S$ 为所有包含 $k+1$ 个 \verb|1| 的二进制 $(n+1)$-元组的集合。即 $S \subset \{0,1\}^{n+1}$,且每个 $T \in S$ 包含 $k+1$ 个 \verb|1| 和 $(n + 1) - (k + 1) = n - k$ 个 \verb|0|。

    我们可以通过观察直接确定 $|S|$ 的值。构造 $S$ 的一个元素相当于从 $n+1$ 个位置中选择 $k+1$ 个填入 \verb|1|(其余位置填入 \verb|0|)。因此,$|S| = \binom{n+1}{k+1}$。

    接下来,我们根据\emph{最右侧} \verb|1| 出现的位置对 $S$ 进行划分。具体来说,对于每个 $i \in [n+1]$,令 $S_i$ 为 $S$ 中最右侧  \verb|1| 出现在第 $i$ 个位置(从左到右计数)的所有元组构成的子集。(参见证明下方的示例,其中 $n$ 和 $k$ 已赋予具体值。)为计算 $S_i$ 的元素数量,我们在第 $i$ 个位置固定一个 \verb|1|,然后从前 $i-1$ 个位置中选择 $k$ 个位置填入 \verb|1|,其余位置填入 \verb|0|。根据乘法原理,$|S_i| = \binom{i-1}{k}$。

    现在验证 $\{S_i \mid i \in [n+1]\}$ 构成 $S$ 的一个划分。首先,当 $i \ne j$ 时,$S_i \cap S_j = \varnothing$。若 $i < j$,则 $S_i$ 中任何元组的第 $j$ 个位置为 \verb|0|,而 $S_j$ 中任何元组的第 $j$ 个位置为 \verb|1|,因此没有元组同时属于两者;类似地,若 $j < i$,则第 $i$ 个位置在 $S_i$ 中为 \verb|1|,在 $S_j$ 中为 \verb|0|,同样无交集。其次,$S$ 中每个元组均有一个最右侧的 \verb|1|,其位置在 $1$ 至 $n+1$ 之间,因此每个元组均属于某个 $S_i$。

    因此,根据加法原理可得
    \[|S| = \sum_{j=1}^{n+1} |S_i| = \sum_{j=1}^{n+1} {j-1 \choose k}\]

    通过重新定义求和索引为 $i = j - 1$,我们可以将上面表达式写成
    \[|S| = \sum_{i=0}^{n} {i \choose k}\]

    将两个关于 $|S|$ 的表达式建立相等关系,即可证明要证明的结论。
\end{proofs}

\begin{tcolorbox}[colback=gray!10,
    colframe=black,
    width=\textwidth,
    arc=2mm, auto outer arc,
    title={$n = 4, k = 2$ 的示例},breakable,enhanced jigsaw,
    before upper={\parindent15pt\noindent},	]
    \begin{align*}
        S =   \big\{&\{11100\}, \{11010\}, \{11001\}, \{10110\}, \{10101\},                  \\
                    &\{10011\}, \{01110\}, \{01101\}, \{01011\}, \{00111\}\big\}             \\
        S_1 = \enspace\; & \varnothing                                                       \\
        S_2 = \enspace\; & \varnothing                                                       \\
        S_3 = \big\{&\{11100\} \big\}                                                        \\
        S_4 = \big\{&\{11010\}, \{10110\}, \{01110\} \big\}                                  \\
        S_5 = \big\{&\{11001\}, \{10101\}, \{10011\}, \{01101\}, \{01011\}, \{00111\} \big\} 
    \end{align*}
\end{tcolorbox}
\clearpage
\begin{questions}{证明 2 总结:}
    基于最右侧的 \verb|1| 出现的位置进行划分,计算恰好有 $k + 1$ 个 \verb|1| 的二进制 $(n+1)$-元组的数量。
\end{questions}

通过这段内容,我们希望你能更好地理解双法计数论证,并学会如何通过观察等式的形式来构思这样的论证。这一主题需要多加练习,建议你尝试完成本节末尾的练习题。若需进一步帮助,我们推荐阅读下一节。下一节将介绍一些启发式方法,用于分析双计数问题,并帮助你选择合适的集合 $S$ 进行证明。这些方法基于本章前面介绍的标准计数对象及其对应公式。

在继续之前,我们想展示最后一个双法计数证明,因为我们认为它极具启发性,既聪明又优雅。我们并不期望你能独立提出这样的论证,特别是因为它并不完全符合我们迄今为止所描述的``双法计数''证明,但我们认为它值得仔细阅读和欣赏,因此请务必细心阅读。\\

\begin{proposition}{高斯配对求和}
    对于任意 $n \in \mathbb{N}$,
    \[\sum_{k=1}^{n} k = \frac{n(n+1)}{2}\]
\end{proposition}

\begin{proof}
    首先,通过观察易得 $\frac{n(n+1)}{2} = {n+1 \choose 2}$。

    现在,考虑一个由 $n + 1$ 行组成的规则三角形点阵,其中第 $k$ 行有 $k$ 个点。左边的求和表示点阵前 $n$ 行的``面积'',即前 $n$ 行的总点数。

    接下来,我们建立这些点与 $(n+1)$ 行中点对之间的双射关系。对于任意一对点,通过向上绘制指向内的对角线,可以在上方行中找到唯一一个点;反之,对于点阵中的任意一点,通过向下绘制指向外的对角线,可以在底行中找到唯一一对点。因此,$(n+1)$ 行中点对的数量等于 $\sum_{k=1}^{n} k$。
\end{proof}