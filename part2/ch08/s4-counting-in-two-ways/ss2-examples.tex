% !TeX root = ../../../book.tex

\subsection{示例}

让我们详细地看几个例子。这将帮助你了解如何应用``双法计数''技术,并为你提供一些典型的例子供你参考和重读,同时还为你提供了一些可以应用于未来问题的基本组合数学结论。在每个例子中,我们不仅尝试证明所讨论的结论,还解释我们如何得出证明,我们在构建论证时的思维过程,以及你如何尝试自己解决类似的问题。双法计数证明的一个优点在于,在这样的证明结束时,我们通常可以简洁地总结证明的主要思想。我们将为我们展示的每个证明总结核心思想,并鼓励你在写出任何这样的证明后也尝试做同样的总结。这使得证明的思想更容易记住,并且只需一句话就可以重构整个证明。

\begin{proposition}{帕斯卡恒等式 (Pascal's Identity)}
    对于任意 $n,k \in \mathbb{N}$,
    \[\begin{pmatrix}n\\k\end{pmatrix}=\begin{pmatrix}n-1\\k\end{pmatrix}+\begin{pmatrix}n-1\\k-1\end{pmatrix}\]
\end{proposition}

\begin{questions}{证明策略:}
    看到像 $\big({n \atop k}\big)$ 这样的二项式系数,说明我们可能想要计算 $[n]$ 中具有特定大小的子集。这个公式的左边很好理解(计算所有大小为 $k$ 的子集),但右边该怎么理解呢?看到两项相加,说明这里存在某种划分。我们需要找到 $[n]$ 中大小为 $k$ 的子集的某个特性,使得一些子集具有这个特性,而另一些子集没有。注意到两个项的唯一区别在于``底部系数'',我们可以据此找到一个有效的划分……在继续阅读之前,看看你能否自己想出来!
\end{questions}

\begin{proof}
    设 $S = \{T \subseteq [n] \mid |T| = k\}$。根据 $k$-选择的定义,我们知道 $|T| = \big({n \atop k}\big)$。

    接着定义集合
    \begin{align*}
        A & = \{T \subseteq [n] \mid |T| = k ∧ 1 \in T\}    \\
        B & = \{T \subseteq [n] \mid |T| = k ∧ 1 \notin T\}
    \end{align*}
    显然,$A \cap B = \varnothing$,因为对于任意集合 $T$,$1 \in T$ 和 $1 \notin T$ 不可能同时成立。并且 $S = A \cup B$,因为对于任意集合 $T$,要么 $1 \in T$ 要么 $1 \notin T$。因此 $\{A,B\}$ 是 $S$ 的一个划分。由此可知 $|S| = |A| + |B|$。

    要得到 $|A|$,我们可以通过一个两步过程来构造元素 $T \in A$:
    \begin{enumerate}[label=(\arabic*)]
        \item 将元素 $1$ 加入 $T$;
        \item 从剩下 $n-1$ 个元素中选择 $k-1$ 个组成集合的 $k$ 个元素。
    \end{enumerate}
    根据乘法原理,我们得出
    \[|A| = 1 \cdot \begin{pmatrix}n-1\\k-1\end{pmatrix} = \begin{pmatrix}n-1\\k-1\end{pmatrix}\]

    同理,要得到 $B$,我们可以通过一个两步过程来构造元素 $T \in B$:
    \begin{enumerate}[label=(\arabic*)]
        \item 从 $T$ 中删除元素 $1$;
        \item 从剩下 $n-1$ 个元素中选择 $k$ 个元素。
    \end{enumerate}
    根据乘法原理,我们得出
    \[|B| = \begin{pmatrix}n-1\\k\end{pmatrix}\]

    根据加法原理,我们得出
    \[|S|=|A|+|B| = \begin{pmatrix}n-1\\k-1\end{pmatrix} + \begin{pmatrix}n-1\\k\end{pmatrix}\]

    将 $|S|$ 的两个表达式用等号连结,我们得出
    \[\begin{pmatrix}n\\k\end{pmatrix}=\begin{pmatrix}n-1\\k\end{pmatrix}+\begin{pmatrix}n-1\\k-1\end{pmatrix}\]
\end{proof}

\begin{questions}{证明总结:}
    通过将 $k$ 个元素的子集划分为是否包含某个特定元素(例如 $1$),来计数集合 $[n]$ 中所有 $k$ 个元素的子集。
\end{questions}

\begin{questions}{问题:}
    如果我们使用元素 $n$ 而不是元素 $1$ 来构建我们的划分,会有什么不同吗?证明在结构上会发生变化吗?答案是不会的!关键在于我们识别了一个特定元素,并基于这个元素定义了划分集 $A$ 和 $B$。
\end{questions}

\begin{questions}{历史注记:}
    这个命题以法国数学家布莱兹·帕斯卡 (Blaise Pascal) 命名。你可能听说过帕斯卡三角 (Pascal's Triangle),它是由一系列自然数构成的。帕斯卡三角的构造方法是:先在第一行和第二行都写上 $1$,然后在边界上也都写 $1$,接着用上面两个数字的和填充其他位置。根据我们刚刚证明的这个命题,你是否能猜到三角形中的数字是什么呢?没错,它们正是二项式系数!第 $n$ 行包含所有形式为 $\big({n \atop k}\big)$ 的数字,其中系数 $k$ 从左到右递增。帕斯卡三角有许多有趣的性质,我们将在接下来的示例和练习中深入探讨。

    \begin{center}
        \Large\begin{tabular}{ccccccccccc}
              &   &   &   &    & 1 &    &   &   &   &   \\
              &   &   &   & 1  &   & 1  &   &   &   &   \\
              &   &   & 1 &    & 2 &    & 1 &   &   &   \\
              &   & 1 &   & 3  &   & 3  &   & 1 &   &   \\
              & 1 &   & 4 &    & 6 &    & 4 &   & 1 &   \\
            1 &   & 5 &   & 10 &   & 10 &   & 5 &   & 1 \\
        \end{tabular}
    \end{center}
\end{questions}

\begin{proposition}{Chairperson 恒等式 (Chairperson Identity)}
    对于任意 $n,k \in \mathbb{N}$,
    \[k\begin{pmatrix}n\\k\end{pmatrix}=n\begin{pmatrix}n-1\\k-1\end{pmatrix}\]
\end{proposition}

\begin{questions}{证明策略:}
    这个等式两边都是两项的\emph{乘积},因此我们需要寻找两个两步过程来构建相同的元素集合。左边的项可以看作 $\big({n \atop k}\big) \cdot k$,因为乘法满足交换律。这表示从 $[n]$ 中选择 $k$ 个元素,然后……做其他的事情。如果我们将 $k$ 写成 $\big({k \atop 1}\big)$,第二步就清晰了:我们从第一步选出的 $k$ 个元素中再选择一个元素。

    这引入了一种新的策略来描述子集及其中的特定元素:\emph{委员会 (Committee)}和\emph{领导者 (Leader)}。这种策略在组合证明中非常流行,因为它减少了技术性数学语言和符号的使用,使得关键思想更易于理解。我们将在下面的证明中展示如何使用这种策略,然后将其与使用更多数学语言的证明进行比较。在继续阅读之前,看看你是否能预见我们会如何描述等式右边的内容……
\end{questions}

\begin{proof}
    给定 $n,k \in \mathbb{N}$。并设 $S$ 为从 $n$ 个人中选出 $k$ 人组成委员会的集合,其中包括一名指定的主席。一种构造集合 $S$ 中的元素的方法是:先选出 $k$ 人组成委员会,然后再从中选出一人作为主席。根据乘法原理,我们可以得出
    \[|S| = \begin{pmatrix}n\\k\end{pmatrix} \cdot \begin{pmatrix}k\\1\end{pmatrix} = k\begin{pmatrix}n\\k\end{pmatrix}\]

    另一种构造集合 $S$ 中的元素的方法是:首先从所有 $n$ 个人中选出委员会主席,然后从剩下的 $n-1$ 人中选出 $k-1$ 人来组成委员会。根据乘法原理,我们可以得出
    \[|S| = \begin{pmatrix}n\\1\end{pmatrix} \cdot \begin{pmatrix}n-1\\k-1\end{pmatrix} = n\begin{pmatrix}n-1\\k-1\end{pmatrix}\]

    将 $|S|$ 的两个表达式用等号连结,我们得出
    \[k\begin{pmatrix}n\\k\end{pmatrix}=n\begin{pmatrix}n-1\\k-1\end{pmatrix}\]
\end{proof}

\begin{questions}{证明总结:}
    通过先选择委员会成员再选择主席,或者先选择主席再选择其余委员会成员,计算从 $n$ 个人中选出 $k$ 人委员会的方法数。
\end{questions}

\begin{questions}{注释:}
    如果我们试图用纯粹的数学语言(即集合)来描述这个证明会怎样呢?在集合论中,``主席''到底对应什么?从 $n$ 个人中选出一个大小为 $k$ 的委员会可以表示为集合 $T \subseteq [n]$,其中 $|T| = k$,但我们如何区分这个集合中 $k$ 种不同成员作为主席的方式呢?一个合理的方法是定义一个有序对,第一个元素是委员会成员的集合,第二个元素特指主席。基于这种策略,我们将定义集合
    \[\hat{S} = \{(T, x) \mid T \subseteq [n] \land |T| = k \land x \in T\}\]
    这个集合 $\hat{S}$ 与我们在上面证明中定义的 $S$ 是相同的,因为它包含了所有拥有一个主席的 $k$-人委员会的方法。然而,在描述如何计数 $\hat{S}$ 的元素时,我们可能会发现自己还是会用到委员会和主席的口语描述!(你可以试试,不用这些描述来计数 $\hat{S}$ 的元素。)这样做更自然,也更容易理解。总之,没有必要严格地写出这些委员会集合的集合论描述;但指出我们可以这样做是重要的。这验证了我们在上述证明中的描述确实足够严格,它们基于数学概念,但用其他术语描述时更容易理解和遵循。

    本节的练习中探讨了几个涉及委员会和子委员会的双重计数证明的例子。我们将在这里再举一个例子,作为练习。
\end{questions}

\begin{proposition}{所有大小的委员会}
    \[\sum_{k=0}^{n} \begin{pmatrix}n\\k\end{pmatrix}=2^n\]
\end{proposition}

\begin{questions}{证明策略:}
    右边的表达式可能有多种含义,但它似乎涉及一个 $n$-步过程,每一步都有 $2$ 种选择。稍后我们再详细探讨这个表达式。左边表示一个划分,因为它是多个项之和。求和中的每一项 $\big({n \atop k}\big)$ 代表从 $n$ 个人中选出 $k$ 个人组成委员会的方法数。当 $k$ 从 $0$ 到 $n$ 变化时,我们考虑了所有可能的委员会大小。这说明我们在计算从 $n$ 个人中选出所有可能的委员会。既然我们知道右边在计算什么,我们可以为此构建一个证明……在阅读我们的证明之前,试着自己推导一下!
\end{questions}

\begin{proof}
    设 $n \in \mathbb{N}$。并设 $S$ 为从 $n$ 个人中选出的所有大小的委员会的集合。$S$ 中的元素是大小介于 $0$ (含)到 $n$ (含)人的委员会。对于每个 $k \in [n] \cup \{0\}$,设 $S_k$ 为 $k$-人委员会的集合。则集合 $\{S_k \mid k \in [n] \cup \{0\}\}$ 是 $S$ 的一个划分。因此,根据加法原理,我们得出
    \[|S| = \sum_{k=0}^{n} |S_k| = \sum_{k=0}^{n} \begin{pmatrix}n\\k\end{pmatrix} \]
    其中 $|S_k| = \big({n \atop k}\big)$,因为是 $[n]$ 中所有 $k$-选择的集合。

    我们还可以这样计算集合 $S$ 中元素的数量:取一个包含 $n$ 个人的集合,并给他们从 $1$ 到 $n$ 编号(例如,可以给每个人发一件印有他们唯一编号的 T 恤)。为了构建一个委员会,我们按数字顺序排好所有人,并沿着队伍前进,对每个人说``Yes''或``No'',表示他们是否属于我们正在创建的委员会。每一个包含 $n$ 个``Yes''和``No''的分配序列都会生成一个唯一的委员会。由于这是一个 $n$-步过程,每一步有两个选择,根据乘法原理,我们有 $2^n$ 种完成这个过程的方法,所以 $|S| = 2^n$。通过将这两个 $|S|$ 的表达式建立相等关系,我们得出结论
    \[\sum_{k=0}^{n} \begin{pmatrix}n\\k\end{pmatrix}=2^n\]
\end{proof}

\begin{questions}{证明总结:}
    通过基于大小的划分来计算 $[n]$ 的所有子集。(注意:虽然这个总结是用集合术语写的,但我们认为用委员会术语来书写和理解证明会更容易。)
\end{questions}

你可能会觉得这个证明有点冗长,特别是我们已经通过归纳法证明了 $|\mathcal{P}([n])| = 2^n$。既然我们在考虑所有大小的委员会,实际上就是说``令 $S = \mathcal{P}([n])$'',然后用两种方法来计算 $|S|$。然而,当我们用委员会术语来书写证明时,如果不解释一下这些表述为何等价,就不能直接切换到讨论 $[n]$ 的子集。作为练习,试着不用委员会术语,完全用集合符号来重写这个证明。你更喜欢哪种方法?

\subsubsection*{求和恒等式}

接下来的组合恒等式非常有用,会在本章的后续证明和练习中反复出现,所以我们先在这里给出结论。此外,我们将展示\emph{两种不同的}双法计数证明。这个问题甚至还有第三种证明方法,留作练习有你来完成证明。这两种证明方法属于标准的计数对象,我们鼓励你阅读并理解它们之间的关系。你可能会问,为什么要展示两个相同事实的证明?(``一个证明不就够了吗?'')通过理解这些证明的结构及其等价性,你将对这些证明技术有更深入的理解,并能更好地应用它们。相信我们!另外,我们还会将这些技术与前面问题中使用的委员会方法进行比较,探讨这三种方法之间的关系。

\begin{theorem}{求和恒等式}\label{theorem8.4.5}
    设 $n,k \in \mathbb{N}$,则
    \[\sum_{i=0}^{n} \begin{pmatrix}i\\k\end{pmatrix} = \begin{pmatrix}n+1\\k+1\end{pmatrix}\]
\end{theorem}

\begin{questions}{证明策略 1:}
    右侧是一个二项式项,表明我们正在讨论 $[n+1]$ 中大小为 $k+1$ 的子集。左侧的求和表示我们正在根据某种属性对所有这些子集进行划分。由于求和中的二项式底部系数都是 $k$,而不是 $k+1$,这意味着索引 $i$ 以某种方式表示某个特定元素被包含在子集中。在继续阅读之前,请尝试写出这个划分的详细信息……
\end{questions}

\begin{proofs}{证明 1.}
    设 $n,k \in \mathbb{N}$。并定义
    \[S = \{T \in [n + 1] \mid |T| = k + 1\}\]
    根据 $[n+1]$ 的 $k+1$-选择定义,我们知道 $|S| = \big({n+1 \atop k+1}\big)$。

    接着,对于每一个 $i \in [n] \cup \{0\}$,定义集合
    \[S_i = \{T \in S \mid i + 1 \in T \land (\forall j \in T \centerdot j \le i + 1)\}\]
    也就是说,$S_i$ 是由 $[n + 1]$ 中所有大小为 $k + 1$ 且\emph{最大索引}元素为 $i + 1$ 的子集组成的集合。我们认为 $\{S_i \mid i \in [n] \cup \{0\}\}$ 构成了 $S$ 的一个划分。

    首先,我们注意到 $S_i \cap S_j = \varnothing$ ,当且仅当 $i \ne j$。这是因为 $T ∈ S_i$ 意味着 $i + 1 \in T$;进一步来说,如果 $i > j$,那么任意 $U \in S_j$ 的最大索引元素是 $j + 1$,而 $j + 1$ 小于 $i + 1$;如果 $i < j$,那么任意 $U ∈ S_j$ 包含 $j + 1$,但 $j + 1 \notin T$。

    其次,我们注意到每个 $T \in S$ 在 $1$ 到 $n + 1$ 之间都有一个最大索引元素,因此属于某个 $S_i$ 集合。为了更好地理解这一部分的证明,我们在下面提供了一个示例,展示了 $n = 4, k = 2$ 的情况。请注意,有几个集合是空集。通常,对于每个 $i \in [k - 1] \cup \{0\}, S_i=\varnothing$,这很合理,因为对于所有这些 $i$ 值,$\big({i \atop k}\big)=0$。

    接下来,我们必须找出对于每个 $i \in [n] \cup {0}, |S_i|$ 的值。为了构建元素 $T \in S_i$,我们可以定义一个两步过程:
    \begin{enumerate}[label=(\arabic*)]
        \item 包含元素 $i + 1 \in T$;
        \item 从 i 个较小的索引元素中,选择 $k$ 个。
    \end{enumerate}
    根据乘法原理和选择的定义,有 $\big({i \atop k}\big)$ 种方式。

    因此,根据加法原理,我们得出
    \[|S| = \sum_{i=0}^{n} |S_i| = \sum_{i=0}^{n} \begin{pmatrix}i\\k\end{pmatrix}\]
    将两个关于 $|S|$ 的表达式建立相等关系,我们得到
    \[\sum_{i=0}^{n} \begin{pmatrix}i\\k\end{pmatrix} = \begin{pmatrix}n+1\\k+1\end{pmatrix}\]
\end{proofs}

$n = 4, k = 2$ 的示例

\begin{align*}
    S =   & \Big\{\{1, 2, 3\}, \{1, 2, 4\}, \{1, 2, 5\}, \{1, 3, 4\}, \{1, 3, 5\},                    \\
          & \{1, 4, 5\}, \{2, 3, 4\}, \{2, 3, 5\}, \{2, 4, 5\}, \{3, 4, 5\}\Big\}                     \\
    S_1 = & \varnothing                                                                               \\
    S_2 = & \varnothing                                                                               \\
    S_3 = & \Big\{\{1, 2, 3\} \Big\}                                                                  \\
    S_4 = & \Big\{\{1, 2, 4\}, \{1, 3, 4\}, \{2, 3, 4\} \Big\}                                        \\
    S_5 = & \Big\{\{1, 2, 5\}, \{1, 3, 5\}, \{1, 4, 5\}, \{2, 3, 5\}, \{2, 4, 5\}, \{3, 4, 5\} \Big\}
\end{align*}

\begin{questions}{证明 1 总结:}
    通过按子集中最大索引元素进行划分,计算 $[n+1]$ 中具有 $(k+1)$ 个元素的子集的数量。
\end{questions}

这个证明策略源于我们最初的观察,即 $\big({i+1 \atop k+1}\big)$ 这样的二项式系数表示从 $[n + 1]$ 中选择子集。然而,我们还可以通过另一种常见的计数方法 --- 二进制元组,来解释这个系数。这将引导我们以不同的方式思考左侧的求和。现在,让我们深入探讨这个证明吧!

\begin{proofs}{证明 2.}
    设 $n,k \in \mathbb{N}$,并设 $S$ 为所有包含 $k+1$ 个 \verb|1| 的二进制 $(n+1)$-元组的集合。也就是说 $S \subset \{0,1\}^{n+1}$,且每个 $T \in S$ 包含 $k+1$ 个 \verb|1| 和 $(n + 1) - (k + 1) = n - k$ 个 \verb|0|。

    我们可以通过观察来直接确定 $|S|$。构造 $S$ 的一个元素相当于从 $n + 1$ 个空位中选择 $k + 1$ 个位置填上 \verb|1|(其余位置填 \verb|0|)。因此,$|S| = \big({n+1 \atop k+1}\big)$。

    接下来,我们可以通过\emph{最右侧}的 \verb|1| 出现的位置来划分 $S$。具体来说,对于 $i \in [n + 1]$,令 $S_i$ 为 $S$ 的子集,包含最右侧的 \verb|1| 出现在位置 $i$ 的所有元组(从左到右读取)。(参见证明下方的示例,其中 $n$ 和 $k$ 赋予了具体值)。要计算 $S_i$ 的元素数量,我们在位置 $i$ 放置一个 \verb|1|,然后从左边的 $i - 1$ 个位置中选择 $k$ 个位置填上 \verb|1|,其余位置填 \verb|0|。根据乘法原理,$|S_i| = \big({i-1 \atop k}\big)$。

    现在,我们验证 ${S_i \mid i \in [n + 1]}$ 是否构成 $S$ 的一个划分。首先,注意到当 $i \ne j$ 时,$S_i \cap S_j = \varnothing$;如果 $i < j$,则 $S_i$ 的任何元素的第 $j$ 个位置都为 \verb|0|,而 $S_j$ 的任何元素的第 $j$ 个位置都为 \verb|1|,因此任何 $(n + 1)$-元组都不能同时属于这两个集合。类似地,如果 $j < i$,则第 $i$ 个位置要么为 \verb|1|(对于 $S_i$ 的元素),要么为 \verb|0|(对于 $S_j$ 的元素)。其次,注意到 $S$ 的任何元素都有一个最右边的 \verb|1|,并且必须出现在 $1$ 到 $n+1$ 之间的某个位置,因此 $S$ 的每个元素都属于某个 $S_i$ 集合。

    因此,根据加法原理
    \[|S| = \sum_{j=1}^{n+1} |S_i| = \sum_{j=1}^{n+1} \begin{pmatrix}j-1\\k\end{pmatrix}\]
    通过重新定义求和索引为 $i = j - 1$,我们可以将上面表达式写成
    \[|S| = \sum_{i=0}^{n} \begin{pmatrix}i\\k\end{pmatrix}\]
    将两个关于 $|S|$ 的表达式建立相等关系,即可证明要证明的结论。
\end{proofs}

$n = 4, k = 2$ 的示例

\begin{align*}
    S =   & \Big\{\{11100\}, \{11010\}, \{11001\}, \{10110\}, \{10101\},                  \\
          & \{10011\}, \{01110\}, \{01101|\}, \{01011\}, \{00111\}\Big\}                  \\
    S_1 = & \varnothing                                                                   \\
    S_2 = & \varnothing                                                                   \\
    S_3 = & \Big\{\{11100\} \Big\}                                                        \\
    S_4 = & \Big\{\{11010\}, \{10110\}, \{01110\} \Big\}                                  \\
    S_5 = & \Big\{\{11001\}, \{10101\}, \{10011\}, \{01101\}, \{01011\}, \{00111\} \Big\} \\
\end{align*}

\begin{questions}{证明 2 总结:}
    基于最右侧的 \verb|1| 出现的位置进行划分,计算恰好有 $k + 1$ 个 \verb|1| 的二进制 $(n+1)$-元组的数量。
\end{questions}

我们希望通过这段内容,你能更好地了解双法计数论证,并知道如何通过观察等式的形式来提出这样的论证。这一主题需要一定的练习,请你尝试完成本节末尾的练习题。如果你需要更多帮助,我们建议你阅读下一节。下一节将描述一些启发式方法,用于观察双法计数问题,并提出一个``恰当''的集合 $S$ 进行证明。这些方法基于本章前面介绍的标准计数对象及其对应的公式。

在继续之前,我们想展示最后一个双法计数证明,因为我们认为它极具启发性,不仅聪明而且优雅。我们并不期望你能提出这样的论证,特别是因为它并不完全符合我们迄今为止所描述的``双法计数''证明,但我们认为它值得阅读和惊叹,所以请务必仔细阅读。

\begin{proposition}{高斯配对求和}
    对于任意 $n \in \mathbb{N}$,
    \[\sum_{k=1}^{n} k = \frac{n(n+1)}{2}\]
\end{proposition}

\begin{proof}
    首先,通过观察易得 $\frac{n(n+1)}{2} = \big({n+1 \atop 2}\big)$。

    现在,考虑一个由 $n + 1$ 行组成的规则三角形点阵,第 $k$ 行有 $k$ 个点。左边的求和表示点阵前 $n$ 行的``面积'',也就是前 $n$ 行的总点数。

    接下来,我们将这些点与 $(n+1)$ 行中的点对建立双射关系。对于任意一对点,从点阵中向上绘制指向内的对角线,可以找到上面行中的唯一一个点。反之,对于点阵中的任意一点,从点阵中向下绘制指向外的对角线,可以找到底行中的唯一一对点。因此,$(n + 1)$ 行中点对的数量是 $\sum_{k=1}^n k$。
\end{proof}