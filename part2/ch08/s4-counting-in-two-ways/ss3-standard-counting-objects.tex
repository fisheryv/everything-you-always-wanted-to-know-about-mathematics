% !TeX root = ../../../book.tex

\subsection{标准计数对象}

在上一节中,我们已经介绍了几种常见的组合对象。然而,在``双法计数''证明中,真正的挑战在于确定所要计数的具体对象!这类练习通常以这样的形式出现:``这是一个恒等式,请用双法计数来证明它。''这样的表述并没有指明具体的计数对象,而只是提示我们需要通过计数来验证等式。在本节中,我们将提供一个实用指南,帮助大家``解开''组合恒等式,并构建相应的双法计数证明。这些思路基于我们的经验以及组合数学中一些常用的标准论证方法。

\subsubsection*{二项式系数的多种解释}

这些对象及其对应的计数公式已在上一节中讨论过。如果有不熟悉的地方,建议回顾前一节的内容。这里我们重点强调如何\emph{识别}一个计数对象是否与给定的计数问题相关。以``主席恒等式''为例:
\[k{n \choose k} = n{n-1 \choose k-1}\]

假设我们尚未证明该等式。该恒等式仅包含二项式系数的乘积(注意我们总可以将 $k$ 写作 ${k \choose 1}$),这意味着我们可以尝试计数那些能够用简单二项式系数描述的对象。最自然的选择是 $[n]$ 的子集;或者,我们也可以考虑从一群人中选出特定规模的委员会,或是含有 $k$ 个 \verb|1| 的二进制 $n$-元组。这三种选择中的任意一种都有助于我们轻松描述等式中的各项,并将它们相互关联。接下来,我们需要选择一种最易于理解且能清晰解释各项含义的表述方式。

如果选择委员会的解释方式,我们可以沿用之前的证明思路。如果选择 $[n]$ 的子集,则需要设计一个合理的两步过程来描述等式两边的乘积项。在选出大小为 $k$ 的子集后,右边的 ${n \choose 1}$ 项可能表示我们先选出一个``特殊''元素,再填充子集的其余部分。然而,当我们讨论 $[n]$ 的子集时,就不能再使用``委员会主席''这样的术语(这正是我们认为委员会解释更为直观的原因:只需为成员编号,便可自然地使用这些术语)。一种常见的替代说法是``圈出''一个元素以标记其特殊性。也就是说,等式两边都在计数大小为 $k$ 的 $[n]$ 子集,且其中有一个元素被特别标记。左边表示先选子集再标记特殊元素;右边表示先标记特殊元素并确保其属于子集,再补充子集的其余部分。还有其他直观的方式可以解释这一论证,但需要注意的是,除非从一开始就采用委员会框架,否则不宜使用相关术语。(思考题:如何在二进制 $n$-元组的背景下完成这一证明?提示:考虑允许某个``特殊''位置由固定符号填充,而非仅限于 \verb|0| 或 \verb|1|。)

在数学中,我们经常遇到二项式系数相乘的情形,且往往``顶部系数''相同。例如,在双法计数证明中,考虑如何描述如下项(假设这只是等式的一边,另一边在此处不重要):
\[{n \choose k}{n \choose \ell}\]

这类乘积有两种合理的解释方式,具体选择取决于等式的另一边或其他相关项。下面我们将介绍这两种解释,并通过上下文帮助你判断选用哪一种。

假设在委员会的解释框架下,每一项代表从 $n$ 个人中选出一个特定规模的委员会(大小为 $k$ 或 $\ell$)。一种解释是从同一组 $n$ 个人中选出两个委员会。例如,某系有 $n$ 名教授,需要选出 $k$ 名教授监督预算,同时选出 $\ell$ 名教授监督课程,且允许同一教授兼任两个委员会的职务。另一种解释是从不同人群但规模均为 $n$ 的群体中选出两个委员会。例如,一个班级有 $n$ 名男生和 $n$ 名女生,从中选出 $k$ 名男生和 $\ell$ 名女生组成一个俱乐部。两种解释都是合理的,但具体选择需要根据问题的上下文决定。

在委员会类型的论证中,子委员会 (subcommittee) 是一个有用的概念。由于从 $n$ 个人中选出 $k$ 人组成的委员会本身就是一个子集,所以子委员会实际上表示该子集的一个子集。因此,如果在等式中出现如下表达式:
\[{a \choose b}{b \choose c}\]

我们可能会将其解释为:从 $a$ 个人中选出 $b$ 人组成一个委员会,再从这个委员会中选出 $c$ 人组成一个子委员会。这可以类比为成立一个俱乐部并选举其官员,或组建一支运动队并确定首发阵容等场景。

\subsubsection*{指数与过程}

除了二项式系数,组合恒等式中还经常出现指数项,例如 $n^3$, $2^n$, $n^{k-1}$ 等。通常,这些项的解释取决于恒等式中其他项的上下文。我们在此介绍几种标准、常见且易于理解的解释方法。有趣的是,解释有时会取决于底数和指数的大小关系!

考虑诸如这样的项
\[{n \choose k}2^k\]

假设我们根据恒等式的其余部分,将问题解释为``委员会''问题,并声明二项式系数 ${n \choose k}$ 代表从 $n$ 名学生中选出 $k$ 人组成一个委员会。那么 $2^k$ 项代表什么呢?记住,这个项可能来源于一个 $k$-步过程,每一步都有两种选择。由于 $k$ 是委员会的大小,我们可以将其描述为对每个成员进行一个二选一的决策过程。例如,可以为每个成员分配一顶红帽子或蓝帽子;或者选择是否授予每个成员一颗金星;或者让每个成员选择成为共和党人或民主党人。你可以自由发挥创造力!当然,所选择的解释必须与恒等式的其余部分相符,因此有时一种解释比另一种更易理解。记住这一点,如果发现难以表达想法,要主动回头调整解释。

再来考虑诸如这样的项
\[{n \choose k}2^n\]

假设我们仍将问题解释为``委员会''问题。这种情况与之前有何不同?在这里,指数与二项式系数的``顶项''相同。因此,委员会的选择不一定与后续的 $n$-步过程相关。这可能意味着从一个有 $n$ 个学生的班级中选出 $k$ 名班干部,然后将每个学生(包括班干部)分配到 $A$ 组或 $B$ 组,且分配过程不区分班干部身份。如果我们不采用``委员会''解释,这个项可能描述一个二进制 $n$-元组,其中恰好有 $k$ 个 \verb|1|,且某些 \verb|0| 和 \verb|1| 被圈出。具体采用哪种解释取决于问题的背景以及你对这些解释的熟悉程度。

考虑当数字稍作修改时,如何解释这些项。例如
\[{n \choose k}4^n\]

可以解释为:选出一个 $k$ 人委员会,然后每个成员戴一顶红色、蓝色、绿色或黄色的帽子。又如
\[{n \choose k}5^n\]

可以解释为:一个二进制 $n$-元组,其中恰好有 $k$ 个 \verb|1|,且每个 \verb|0| 和 \verb|1| 周围有 $1$ 到 $5$ 个圈。

接下来,让我们探讨一些底数为变量而指数为定值的表达式。例如,考虑如下的项
\[{n \choose k}2^k\]

在这里,我们从 $n$ 个对象中选择 $k$ 个对象,并进行一个两步过程,每步有 $k$ 种选择。也就是说,选择这 $k$ 个对象会影响后续两步过程的结果。如果我们使用``委员会''解释,可以将其视为先选出一个 $k$ 人委员会,然后从中选出两名官员(例如发言人和财务官),委员会中的任何成员都可以担任这两个职位,甚至一人兼任两职。

如果我们使用``元组''解释,可以将其描述为一个二进制 $n$-元组,其中恰好有 $k$ 个 \verb|1|,且一个 \verb|1| 被圈出,另一个 \verb|1| 被框出(允许同一个 \verb|1| 既被圈也被框)。我们还可以使用``字母表''进行解释,从 $n$ 个字母中选择 $k$ 个字母,然后用这 $k$ 个字母组成两个字母的单词。思考这三种解释为何有效及其相互关系。尝试用这些解释重写一个证明,并考虑当项为 $k^3$ 或 $k^4$ 时,解释会有何不同。

最后,考虑诸如这样的项
\[{n \choose k}n^3\]

在``委员会''的语境下。由于指数项的底数与二项式系数的顶项相同,因此 ${n \choose k}$ 项所代表的委员会与随后的三步过程之间不一定存在联系。因此,我们可以解释为选出一个 $k$ 人委员会,然后分配红色、蓝色和绿色丝带各一条,其中一个人可能获得多条丝带,且任何人(无论是否在委员会中)都可能获得一条或多条丝带。用``二进制 $n$-元组''解释该项的任务留给你。试试看吧!

\subsubsection*{求和即划分}

组合恒等式中常常会出现\emph{求和}。在双法计数证明中,处理这一点可能稍显复杂,因为求和表示多个项的总和。不过,最重要的规则是:求和总是代表一个\emph{划分}。特别是,这个划分揭示了所有子集的大小。为了在双法计数证明中阐明这一点,我们需要描述以下三个性质:
\begin{itemize}
    \item 划分的子集是什么。
    \item 为什么求和的索引\emph{限制}在上下文中是合理的。
    \item 对于任意索引,为什么对应集合的大小是求和中的项。
\end{itemize}
我们将通过一个例子来说明这些。

\begin{example}[支持/反对委员会恒等式]
    \[{n \choose k}2^{n-k} = \sum_{i=k}^{n}{n \choose i}{k \choose i}\]
    \begin{questions}{直觉:}
        考虑从 $n$ 个人中选出一个由 $k$ 人组成的委员会,并确定非委员会成员是否支持委员会的决定。另一种方式是,先选择至少 $k$ 个可能进入委员会或支持委员会的人,其余人则反对委员会。然后,从这些支持者中选出 $k$ 人实际进入委员会,其余支持者则仅支持但不进入委员会。(注意:在这些步骤中,详细说明每一步至关重要,不要假设读者能自动理解。)
    \end{questions}

    \begin{proof}
        假设有 $n$ 个人。定义集合 $S$ 为从这 $n$ 个人中选择 $k$ 个人组成委员会的所有可能方式,而每个非委员会成员明确表示\textbf{支持}或\textbf{反对}委员会。

        首先,我们可以通过多个步骤计算 $|S|$:
        \begin{itemize}
            \item 从 $n$ 个人中选择 $k$ 个人组成委员会:有 ${n \choose k}$ 种方法。
            \item 对于其余 $n-k$ 人,每个人决定\textbf{支持}或\textbf{反对}委员会。这个过程有 $n-k$ 步,每步两个选择。所以根据乘法原理:有 $2^{n-k}$ 种方法。
        \end{itemize}

        根据乘法原理,可得 $|S| = {n \choose k} \cdot 2^{n-k}$。

        其次,我们可以根据支持者的人数划分 $S$。根据定义,\textbf{支持}委员会的非委员会成员人数可以从 $0$ 到 $n-k$。因此,委员会成员和支持者的总人数从 $k$ 到 $n$(含端点)。

        对于每个满足 $k \le i \le n$ 的 $i$,令 $S_i \subseteq S$ 表示那些有 $k$ 个委员会成员和 $i - k$ 个支持者的集合。(注意 $0 \le i - k \le n - k$,这与之前的范围一致。)

        请注意,$\{S_i \mid k \le i \le n\}$ 是 $S$ 的一个划分。因为 $S$ 中的每个元素可以由支持委员会的总人数(包括委员会成员和支持者)唯一确定,这个人数是 $i$,其中 $k \le i \le n$。

        现在,对于每个 $i$,我们可以通过多步骤过程计算 $|S_i|$:
        \begin{itemize}
            \item 从所有 $n$ 个人中选择 $i$ 个人作为潜在的委员会候选人:有 ${n \choose i}$ 种方法。
            \item 指定其余 $n - i$ 个人明确反对委员会。\\
                  (这一步是确定的,只有一种方法,但为了完整描述 $S$ 中的元素,我们需要明确指出。)
            \item 从选出的 $i$ 个人中选出 $k$ 人作为委员会成员:有 ${i \choose k}$ 种方法。
            \item 将剩余的 $i - k$ 个人指定为委员会的支持者(非委员会成员)。\\
                  (再次强调,这一步只有一种方法,但需要明确指出这些成员的态度和状态,以便完整描述结果。)
        \end{itemize}

        根据乘法原理,可得 $|S_i| = {n \choose i}{i \choose k}$。

        根据加法原理,可得 $|S| = \sum_{i=k}^{n} |S_i| = \sum_{i=k}^{n}{n \choose i}{i \choose k}$。

        由于我们通过两种方法计算了 $|S|$,因此它们相等,这就证明了恒等式。
    \end{proof}
\end{example}

请注意,在证明中,明确划分之后,我们完成了以下几件事:首先,解释了为什么这是一个划分;其次,说明了划分如何与求和索引关联;然后,解释了求和上下限如何覆盖所有可能的情况;最后,对于每个 $i$,解释了为什么 $|S_i|$ 对应求和中的项。