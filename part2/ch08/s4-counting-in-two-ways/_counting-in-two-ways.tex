% !TeX root = ../../../book.tex
\section{双法计数}\label{sec:section8.4}

如果你是直接阅读这一节,我们建议你先阅读一下前一节的最后一个例子,因为它为``双法计数''提供了一个完美的介绍和例子。在那个例子中,我们用\emph{两种}不同的方法计算了到某个特定点的格路径数,从而得出我们找到的两个表达式必然相等的结论。具体来说,我们得出 $\big({x+y \atop x}\big)=\big({x+y \atop y}\big)$ 的结论。基于那个例子,我们将在这里概述一种通用策略,并将其应用于几个例子。在此过程中,我们不仅会练习这种技术,还会证明一些有用的组合结论,这些结果可以应用于其他问题。

让我们先展示前一节示例的\emph{另一种}证明方法。这种方法完全不涉及格路径,并且更容易记忆和理解。

\begin{proposition}
    设 $n,k \in \mathbb{N} \cup \{0\}$。则 $\begin{pmatrix}n\\k\end{pmatrix} = \begin{pmatrix}n\\n-k\end{pmatrix}$。
\end{proposition}

\begin{proof}
    设 $S$ 为 $[n]$ 的子集,大小为 $k$,即
    \[S = \{T \subseteq [n] \mid |T| = k\}\]
    根据 $k$-选择的定义,$|S|=\big({n \atop k}\big)$,因为构建一个集合 $T \subseteq [n]$ 且 $|T| = k$,实际上就是从 $n$ 个元素中选择 $k$ 个元素。

    等价地,我们可以通过选择\emph{不包含}在 $T$ 中的 $n - k$ 个元素来构造集合 $T \subseteq [n]$,其中 $|T| = k$;这意味着剩下的 $n - (n - k) = k$ 个元素属于 $T$。这样的方法有 $\big({n \atop n-k}\big)$ 种。由于每个这样的集合 $T$ 都可以通过这种方式构造,我们已经证明 $|S| = \big({n \atop n-k}\big)$。

    因此,$\big({n \atop k}\big) = \big({n \atop n-k}\big)$。
\end{proof}

我们认为这一事实的证明更容易记住,因为我们可以用一句话来概括整个证明:

\begin{quotation}
    ``通过确定要包含的元素或要舍弃的元素来计算 $[n]$ 的 $k$-元素子集。''
\end{quotation}

这是我们要记住的核心思想;通过这一思想,我们可以重构整个证明。逐句``死记''证明没有意义;相反,记住证明的\emph{核心思想},然后再补充细节会更有帮助。

% !TeX root = ../../../book.tex

\subsection{方法概述}

\subsubsection*{为何有效}

让我们提升一个层次,讨论一种叫做``双法计数''的证明技巧。首先,我们来探讨它\emph{为何}有效以及\emph{如何}应用。接着,我们会通过几个例子进一步说明。在上一节的结尾,我们已经初步探讨了其原理,这里我们将再次阐述这些观点。``双法计数''这个名称非常贴切,因为它直接说明了这种策略!任何使用这种技巧的证明,都会先确定一个有限元素的集合,并提供\emph{两种}计数这些元素的方法。通过加法和乘法原理,以及其他组合数学的结论,这两种方法会对同一个数(即该有限集的\emph{基数})得出不同的代数表达式。

一个好的证明会明确指出要计数的有限集及其元素的两种不同计数方法,然后通过等式将这两种代数表达式关联。因此,通过这种方法证明的结果通常会涉及二项式系数、求和以及其他代数表达式的\emph{恒等式}或\emph{方程}。证明的关键在于通过计数论证清晰地解释这些表达式,而不是单纯地进行代数简化。

看看我们刚刚证明的结论:是的,我们可以直接验证 ${n \choose k}= \frac{n!}{k!(n-k)!} = {n \choose n-k}$,但那有什么乐趣呢?这根本不能算作证明,无论如何解释其含义,都没有提供任何关于结论为何成立的见解。此外,随着我们研究越来越多这类挑战性问题,代数验证变得相当困难,有时几乎是不可能的!

\subsubsection*{如何使用}

我们将在本节稍后展示几个例子(包括反例),但首先我们要介绍``双法计数''的概要。这将为我们提供一个标准,用来衡量未来这种风格的证明;我们可以通过阅读这些证明,确保它们遵循结构、清晰度和正确性的要点。我们将遵守这些标准,并希望你也能这样做。我们还会向你介绍一些在``双法计数''证明中使用的标准组合对象,并且在展示例子时,我们会指出何时该考虑使用特定的对象集进行计数。

下面给出的是一个\emph{好的``双法计数''证明}的基本结构!

\begin{enumerate}
    \item 陈述要证明的结论。(注意:记得对表达式中的所有变量进行量化!)
    \item 定义集合 $S$,表示需要计数的对象。
    \item 用一种组合论证方法计算 $S$ 中元素的数量,并将结果记为 $|S|$。
    \item 用另一种组合论证方法计算 $S$ 中元素的数量,并将结果记为 $|S|$。
    \item 最后得出结论,由于两种方法得到的结果都等于 $|S|$,因此它们必然相等。
\end{enumerate}

就是这样!正如我们所言,这个证明技术的名字就是证明技巧本身,所以很容易记忆。然而,多年来我们阅读了许多这样的证明,发现了一些常见错误。我们在这里列出了这些常见错误。想一想,为什么做这些错误会导致证明变得``糟糕''?每个错误未能满足好的证明的哪个属性?是正确性?还是清晰性?抑或是简洁性?

\subsubsection*{常见错误}

\begin{itemize}
    \item 忘记定义需要计数的对象集合。
    \item 定义了对象集合,但用两种方法计数了其他集合。
    \item 计数了对象集合,但随后用另一种方法计数了别的对象集合。
    \item 在结论中未能将两个表达式建立恒等关系。
\end{itemize}
在该证明技术之外,其他错误也可能出现在实际的组合证明中,这些错误也要留意!


% !TeX root = ../../../book.tex

\subsection{示例}

让我们详细地看几个例子。这将帮助你了解如何应用``双法计数''技术,并为你提供一些典型的例子供你参考和重读,同时还为你提供了一些可以应用于未来问题的基本组合数学结论。在每个例子中,我们不仅尝试证明所讨论的结论,还解释我们如何得出证明,我们在构建论证时的思维过程,以及你如何尝试自己解决类似的问题。双法计数证明的一个优点在于,在这样的证明结束时,我们通常可以简洁地总结证明的主要思想。我们将为我们展示的每个证明总结核心思想,并鼓励你在写出任何这样的证明后也尝试做同样的总结。这使得证明的思想更容易记住,并且只需一句话就可以重构整个证明。

\begin{proposition}{帕斯卡恒等式 (Pascal's Identity)}
    对于任意 $n,k \in \mathbb{N}$,
    \[{n \choose k}={n-1 \choose k}+{n-1 \choose k-1}\]
\end{proposition}

\begin{questions}{证明策略:}
    看到像 ${n \choose k}$ 这样的二项式系数,说明我们可能想要计算 $[n]$ 中具有特定大小的子集。这个公式的左边很好理解(计算所有大小为 $k$ 的子集),但右边该怎么理解呢?看到两项相加,说明这里存在某种划分。我们需要找到 $[n]$ 中大小为 $k$ 的子集的某个特性,使得一些子集具有这个特性,而另一些子集没有。注意到两个项的唯一区别在于``底部系数'',我们可以据此找到一个有效的划分……在继续阅读之前,看看你能否自己想出来!
\end{questions}

\begin{proof}
    设 $S = \{T \subseteq [n] \mid |T| = k\}$。根据 $k$-选择的定义,我们知道 $|T| = {n \choose k}$。

    接着定义集合
    \begin{align*}
        A & = \{T \subseteq [n] \mid |T| = k ∧ 1 \in T\}    \\
        B & = \{T \subseteq [n] \mid |T| = k ∧ 1 \notin T\}
    \end{align*}
    显然,$A \cap B = \varnothing$,因为对于任意集合 $T$,$1 \in T$ 和 $1 \notin T$ 不可能同时成立。并且 $S = A \cup B$,因为对于任意集合 $T$,要么 $1 \in T$ 要么 $1 \notin T$。因此 $\{A,B\}$ 是 $S$ 的一个划分。由此可知 $|S| = |A| + |B|$。

    要得到 $|A|$,我们可以通过一个两步过程来构造元素 $T \in A$:
    \begin{enumerate}[label=(\arabic*)]
        \item 将元素 $1$ 加入 $T$;
        \item 从剩下 $n-1$ 个元素中选择 $k-1$ 个组成集合的 $k$ 个元素。
    \end{enumerate}
    根据乘法原理,我们得出
    \[|A| = 1 \cdot {n-1 \choose k-1} = {n-1 \choose k-1}\]

    同理,要得到 $B$,我们可以通过一个两步过程来构造元素 $T \in B$:
    \begin{enumerate}[label=(\arabic*)]
        \item 从 $T$ 中删除元素 $1$;
        \item 从剩下 $n-1$ 个元素中选择 $k$ 个元素。
    \end{enumerate}
    根据乘法原理,我们得出
    \[|B| = {n-1 \choose k}\]

    根据加法原理,我们得出
    \[|S|=|A|+|B| = {n-1 \choose k-1} + {n-1 \choose k}\]

    将 $|S|$ 的两个表达式用等号连结,我们得出
    \[{n \choose k}={n-1 \choose k}+{n-1 \choose k-1}\]
\end{proof}

\begin{questions}{证明总结:}
    通过将 $k$ 个元素的子集划分为是否包含某个特定元素(例如 $1$),来计数集合 $[n]$ 中所有 $k$ 个元素的子集。
\end{questions}

\begin{questions}{问题:}
    如果我们使用元素 $n$ 而不是元素 $1$ 来构建我们的划分,会有什么不同吗?证明在结构上会发生变化吗?答案是不会的!关键在于我们识别了一个特定元素,并基于这个元素定义了划分集 $A$ 和 $B$。
\end{questions}

\begin{questions}{历史注记:}
    这个命题以法国数学家布莱兹·帕斯卡 (Blaise Pascal) 命名。你可能听说过帕斯卡三角 (Pascal's Triangle),它是由一系列自然数构成的。帕斯卡三角的构造方法是:先在第一行和第二行都写上 $1$,然后在边界上也都写 $1$,接着用上面两个数字的和填充其他位置。根据我们刚刚证明的这个命题,你是否能猜到三角形中的数字是什么呢?没错,它们正是二项式系数!第 $n$ 行包含所有形式为 ${n \choose k}$ 的数字,其中系数 $k$ 从左到右递增。帕斯卡三角有许多有趣的性质,我们将在接下来的示例和练习中深入探讨。

    \begin{center}
        \Large\begin{tabular}{ccccccccccc}
              &   &   &   &    & 1 &    &   &   &   &   \\
              &   &   &   & 1  &   & 1  &   &   &   &   \\
              &   &   & 1 &    & 2 &    & 1 &   &   &   \\
              &   & 1 &   & 3  &   & 3  &   & 1 &   &   \\
              & 1 &   & 4 &    & 6 &    & 4 &   & 1 &   \\
            1 &   & 5 &   & 10 &   & 10 &   & 5 &   & 1 \\
        \end{tabular}
    \end{center}
\end{questions}

\begin{proposition}{Chairperson 恒等式 (Chairperson Identity)}
    对于任意 $n,k \in \mathbb{N}$,
    \[k{n \choose k}=n{n-1 \choose k-1}\]
\end{proposition}

\begin{questions}{证明策略:}
    这个等式两边都是两项的\emph{乘积},因此我们需要寻找两个两步过程来构建相同的元素集合。左边的项可以看作 ${n \choose k} \cdot k$,因为乘法满足交换律。这表示从 $[n]$ 中选择 $k$ 个元素,然后……做其他的事情。如果我们将 $k$ 写成 ${k \choose 1}$,第二步就清晰了:我们从第一步选出的 $k$ 个元素中再选择一个元素。

    这引入了一种新的策略来描述子集及其中的特定元素:\emph{委员会 (Committee)}和\emph{领导者 (Leader)}。这种策略在组合证明中非常流行,因为它减少了技术性数学语言和符号的使用,使得关键思想更易于理解。我们将在下面的证明中展示如何使用这种策略,然后将其与使用更多数学语言的证明进行比较。在继续阅读之前,看看你是否能预见我们会如何描述等式右边的内容……
\end{questions}

\begin{proof}
    给定 $n,k \in \mathbb{N}$。并设 $S$ 为从 $n$ 个人中选出 $k$ 人组成委员会的集合,其中包括一名指定的主席。一种构造集合 $S$ 中的元素的方法是:先选出 $k$ 人组成委员会,然后再从中选出一人作为主席。根据乘法原理,我们可以得出
    \[|S| = {n \choose k} \cdot {k \choose 1} = k{n \choose k}\]

    另一种构造集合 $S$ 中的元素的方法是:首先从所有 $n$ 个人中选出委员会主席,然后从剩下的 $n-1$ 人中选出 $k-1$ 人来组成委员会。根据乘法原理,我们可以得出
    \[|S| = {n \choose 1} \cdot {n-1 \choose k-1} = n{n-1 \choose k-1}\]

    将 $|S|$ 的两个表达式用等号连结,我们得出
    \[k{n \choose k}=n{n-1 \choose k-1}\]
\end{proof}

\begin{questions}{证明总结:}
    通过先选择委员会成员再选择主席,或者先选择主席再选择其余委员会成员,计算从 $n$ 个人中选出 $k$ 人委员会的方法数。
\end{questions}

\begin{questions}{注释:}
    如果我们试图用纯粹的数学语言(即集合)来描述这个证明会怎样呢?在集合论中,``主席''到底对应什么?从 $n$ 个人中选出一个大小为 $k$ 的委员会可以表示为集合 $T \subseteq [n]$,其中 $|T| = k$,但我们如何区分这个集合中 $k$ 种不同成员作为主席的方式呢?一个合理的方法是定义一个有序对,第一个元素是委员会成员的集合,第二个元素特指主席。基于这种策略,我们将定义集合
    \[\hat{S} = \{(T, x) \mid T \subseteq [n] \land |T| = k \land x \in T\}\]
    这个集合 $\hat{S}$ 与我们在上面证明中定义的 $S$ 是相同的,因为它包含了所有拥有一个主席的 $k$-人委员会的方法。然而,在描述如何计数 $\hat{S}$ 的元素时,我们可能会发现自己还是会用到委员会和主席的口语描述!(你可以试试,不用这些描述来计数 $\hat{S}$ 的元素。)这样做更自然,也更容易理解。总之,没有必要严格地写出这些委员会集合的集合论描述;但指出我们可以这样做是重要的。这验证了我们在上述证明中的描述确实足够严格,它们基于数学概念,但用其他术语描述时更容易理解和遵循。

    本节的练习中探讨了几个涉及委员会和子委员会的双重计数证明的例子。我们将在这里再举一个例子,作为练习。
\end{questions}

\begin{proposition}{所有大小的委员会}
    \[\sum_{k=0}^{n} {n \choose k}=2^n\]
\end{proposition}

\begin{questions}{证明策略:}
    右边的表达式可能有多种含义,但它似乎涉及一个 $n$-步过程,每一步都有 $2$ 种选择。稍后我们再详细探讨这个表达式。左边表示一个划分,因为它是多个项之和。求和中的每一项 ${n \choose k}$ 代表从 $n$ 个人中选出 $k$ 个人组成委员会的方法数。当 $k$ 从 $0$ 到 $n$ 变化时,我们考虑了所有可能的委员会大小。这说明我们在计算从 $n$ 个人中选出所有可能的委员会。既然我们知道右边在计算什么,我们可以为此构建一个证明……在阅读我们的证明之前,试着自己推导一下!
\end{questions}

\begin{proof}
    设 $n \in \mathbb{N}$。并设 $S$ 为从 $n$ 个人中选出的所有大小的委员会的集合。$S$ 中的元素是大小介于 $0$ (含)到 $n$ (含)人的委员会。对于每个 $k \in [n] \cup \{0\}$,设 $S_k$ 为 $k$-人委员会的集合。则集合 $\{S_k \mid k \in [n] \cup \{0\}\}$ 是 $S$ 的一个划分。因此,根据加法原理,我们得出
    \[|S| = \sum_{k=0}^{n} |S_k| = \sum_{k=0}^{n} {n \choose k} \]
    其中 $|S_k| = {n \choose k}$,因为是 $[n]$ 中所有 $k$-选择的集合。

    我们还可以这样计算集合 $S$ 中元素的数量:取一个包含 $n$ 个人的集合,并给他们从 $1$ 到 $n$ 编号(例如,可以给每个人发一件印有他们唯一编号的 T 恤)。为了构建一个委员会,我们按数字顺序排好所有人,并沿着队伍前进,对每个人说``Yes''或``No'',表示他们是否属于我们正在创建的委员会。每一个包含 $n$ 个``Yes''和``No''的分配序列都会生成一个唯一的委员会。由于这是一个 $n$-步过程,每一步有两个选择,根据乘法原理,我们有 $2^n$ 种完成这个过程的方法,所以 $|S| = 2^n$。通过将这两个 $|S|$ 的表达式建立相等关系,我们得出结论
    \[\sum_{k=0}^{n} {n \choose k}=2^n\]
\end{proof}

\begin{questions}{证明总结:}
    通过基于大小的划分来计算 $[n]$ 的所有子集。(注意:虽然这个总结是用集合术语写的,但我们认为用委员会术语来书写和理解证明会更容易。)
\end{questions}

你可能会觉得这个证明有点冗长,特别是我们已经通过归纳法证明了 $|\mathcal{P}([n])| = 2^n$。既然我们在考虑所有大小的委员会,实际上就是说``令 $S = \mathcal{P}([n])$'',然后用两种方法来计算 $|S|$。然而,当我们用委员会术语来书写证明时,如果不解释一下这些表述为何等价,就不能直接切换到讨论 $[n]$ 的子集。作为练习,试着不用委员会术语,完全用集合符号来重写这个证明。你更喜欢哪种方法?

\subsubsection*{求和恒等式}

接下来的组合恒等式非常有用,会在本章的后续证明和练习中反复出现,所以我们先在这里给出结论。此外,我们将展示\emph{两种不同的}双法计数证明。这个问题甚至还有第三种证明方法,留作练习有你来完成证明。这两种证明方法属于标准的计数对象,我们鼓励你阅读并理解它们之间的关系。你可能会问,为什么要展示两个相同事实的证明?(``一个证明不就够了吗?'')通过理解这些证明的结构及其等价性,你将对这些证明技术有更深入的理解,并能更好地应用它们。相信我们!另外,我们还会将这些技术与前面问题中使用的委员会方法进行比较,探讨这三种方法之间的关系。

\begin{theorem}{求和恒等式}\label{theorem8.4.5}
    设 $n,k \in \mathbb{N}$,则
    \[\sum_{i=0}^{n} {i \choose k} = {n+1 \choose k+1}\]
\end{theorem}

\begin{questions}{证明策略 1:}
    右侧是一个二项式项,表明我们正在讨论 $[n+1]$ 中大小为 $k+1$ 的子集。左侧的求和表示我们正在根据某种属性对所有这些子集进行划分。由于求和中的二项式底部系数都是 $k$,而不是 $k+1$,这意味着索引 $i$ 以某种方式表示某个特定元素被包含在子集中。在继续阅读之前,请尝试写出这个划分的详细信息……
\end{questions}

\begin{proofs}{证明 1.}
    设 $n,k \in \mathbb{N}$。并定义
    \[S = \{T \in [n + 1] \mid |T| = k + 1\}\]
    根据 $[n+1]$ 的 $k+1$-选择定义,我们知道 $|S| = {n+1 \choose k+1}$。

    接着,对于每一个 $i \in [n] \cup \{0\}$,定义集合
    \[S_i = \{T \in S \mid i + 1 \in T \land (\forall j \in T \centerdot j \le i + 1)\}\]
    也就是说,$S_i$ 是由 $[n + 1]$ 中所有大小为 $k + 1$ 且\emph{最大索引}元素为 $i + 1$ 的子集组成的集合。我们认为 $\{S_i \mid i \in [n] \cup \{0\}\}$ 构成了 $S$ 的一个划分。

    首先,我们注意到 $S_i \cap S_j = \varnothing$ ,当且仅当 $i \ne j$。这是因为 $T ∈ S_i$ 意味着 $i + 1 \in T$;进一步来说,如果 $i > j$,那么任意 $U \in S_j$ 的最大索引元素是 $j + 1$,而 $j + 1$ 小于 $i + 1$;如果 $i < j$,那么任意 $U ∈ S_j$ 包含 $j + 1$,但 $j + 1 \notin T$。

    其次,我们注意到每个 $T \in S$ 在 $1$ 到 $n + 1$ 之间都有一个最大索引元素,因此属于某个 $S_i$ 集合。为了更好地理解这一部分的证明,我们在下面提供了一个示例,展示了 $n = 4, k = 2$ 的情况。请注意,有几个集合是空集。通常,对于每个 $i \in [k - 1] \cup \{0\}, S_i=\varnothing$,这很合理,因为对于所有这些 $i$ 值,${i \choose k}=0$。

    接下来,我们必须找出对于每个 $i \in [n] \cup {0}, |S_i|$ 的值。为了构建元素 $T \in S_i$,我们可以定义一个两步过程:
    \begin{enumerate}[label=(\arabic*)]
        \item 包含元素 $i + 1 \in T$;
        \item 从 i 个较小的索引元素中,选择 $k$ 个。
    \end{enumerate}
    根据乘法原理和选择的定义,有 ${i \choose k}$ 种方式。

    因此,根据加法原理,我们得出
    \[|S| = \sum_{i=0}^{n} |S_i| = \sum_{i=0}^{n} {i \choose k}\]
    将两个关于 $|S|$ 的表达式建立相等关系,我们得到
    \[\sum_{i=0}^{n} {i \choose k} = {n+1 \choose k+1}\]
\end{proofs}

$n = 4, k = 2$ 的示例

\begin{align*}
    S =   & \Big\{\{1, 2, 3\}, \{1, 2, 4\}, \{1, 2, 5\}, \{1, 3, 4\}, \{1, 3, 5\},                    \\
          & \{1, 4, 5\}, \{2, 3, 4\}, \{2, 3, 5\}, \{2, 4, 5\}, \{3, 4, 5\}\Big\}                     \\
    S_1 = & \varnothing                                                                               \\
    S_2 = & \varnothing                                                                               \\
    S_3 = & \Big\{\{1, 2, 3\} \Big\}                                                                  \\
    S_4 = & \Big\{\{1, 2, 4\}, \{1, 3, 4\}, \{2, 3, 4\} \Big\}                                        \\
    S_5 = & \Big\{\{1, 2, 5\}, \{1, 3, 5\}, \{1, 4, 5\}, \{2, 3, 5\}, \{2, 4, 5\}, \{3, 4, 5\} \Big\}
\end{align*}

\begin{questions}{证明 1 总结:}
    通过按子集中最大索引元素进行划分,计算 $[n+1]$ 中具有 $(k+1)$ 个元素的子集的数量。
\end{questions}

这个证明策略源于我们最初的观察,即 ${i+1 \choose k+1}$ 这样的二项式系数表示从 $[n + 1]$ 中选择子集。然而,我们还可以通过另一种常见的计数方法 --- 二进制元组,来解释这个系数。这将引导我们以不同的方式思考左侧的求和。现在,让我们深入探讨这个证明吧!

\begin{proofs}{证明 2.}
    设 $n,k \in \mathbb{N}$,并设 $S$ 为所有包含 $k+1$ 个 \verb|1| 的二进制 $(n+1)$-元组的集合。也就是说 $S \subset \{0,1\}^{n+1}$,且每个 $T \in S$ 包含 $k+1$ 个 \verb|1| 和 $(n + 1) - (k + 1) = n - k$ 个 \verb|0|。

    我们可以通过观察来直接确定 $|S|$。构造 $S$ 的一个元素相当于从 $n + 1$ 个空位中选择 $k + 1$ 个位置填上 \verb|1|(其余位置填 \verb|0|)。因此,$|S| = {n+1 \choose k+1}$。

    接下来,我们可以通过\emph{最右侧}的 \verb|1| 出现的位置来划分 $S$。具体来说,对于 $i \in [n + 1]$,令 $S_i$ 为 $S$ 的子集,包含最右侧的 \verb|1| 出现在位置 $i$ 的所有元组(从左到右读取)。(参见证明下方的示例,其中 $n$ 和 $k$ 赋予了具体值)。要计算 $S_i$ 的元素数量,我们在位置 $i$ 放置一个 \verb|1|,然后从左边的 $i - 1$ 个位置中选择 $k$ 个位置填上 \verb|1|,其余位置填 \verb|0|。根据乘法原理,$|S_i| = {i-1 \choose k}$。

    现在,我们验证 ${S_i \mid i \in [n + 1]}$ 是否构成 $S$ 的一个划分。首先,注意到当 $i \ne j$ 时,$S_i \cap S_j = \varnothing$;如果 $i < j$,则 $S_i$ 的任何元素的第 $j$ 个位置都为 \verb|0|,而 $S_j$ 的任何元素的第 $j$ 个位置都为 \verb|1|,因此任何 $(n + 1)$-元组都不能同时属于这两个集合。类似地,如果 $j < i$,则第 $i$ 个位置要么为 \verb|1|(对于 $S_i$ 的元素),要么为 \verb|0|(对于 $S_j$ 的元素)。其次,注意到 $S$ 的任何元素都有一个最右边的 \verb|1|,并且必须出现在 $1$ 到 $n+1$ 之间的某个位置,因此 $S$ 的每个元素都属于某个 $S_i$ 集合。

    因此,根据加法原理
    \[|S| = \sum_{j=1}^{n+1} |S_i| = \sum_{j=1}^{n+1} {j-1 \choose k}\]
    通过重新定义求和索引为 $i = j - 1$,我们可以将上面表达式写成
    \[|S| = \sum_{i=0}^{n} {i \choose k}\]
    将两个关于 $|S|$ 的表达式建立相等关系,即可证明要证明的结论。
\end{proofs}

$n = 4, k = 2$ 的示例

\begin{align*}
    S =   & \Big\{\{11100\}, \{11010\}, \{11001\}, \{10110\}, \{10101\},                  \\
          & \{10011\}, \{01110\}, \{01101|\}, \{01011\}, \{00111\}\Big\}                  \\
    S_1 = & \varnothing                                                                   \\
    S_2 = & \varnothing                                                                   \\
    S_3 = & \Big\{\{11100\} \Big\}                                                        \\
    S_4 = & \Big\{\{11010\}, \{10110\}, \{01110\} \Big\}                                  \\
    S_5 = & \Big\{\{11001\}, \{10101\}, \{10011\}, \{01101\}, \{01011\}, \{00111\} \Big\} \\
\end{align*}

\begin{questions}{证明 2 总结:}
    基于最右侧的 \verb|1| 出现的位置进行划分,计算恰好有 $k + 1$ 个 \verb|1| 的二进制 $(n+1)$-元组的数量。
\end{questions}

我们希望通过这段内容,你能更好地了解双法计数论证,并知道如何通过观察等式的形式来提出这样的论证。这一主题需要一定的练习,请你尝试完成本节末尾的练习题。如果你需要更多帮助,我们建议你阅读下一节。下一节将描述一些启发式方法,用于观察双法计数问题,并提出一个``恰当''的集合 $S$ 进行证明。这些方法基于本章前面介绍的标准计数对象及其对应的公式。

在继续之前,我们想展示最后一个双法计数证明,因为我们认为它极具启发性,不仅聪明而且优雅。我们并不期望你能提出这样的论证,特别是因为它并不完全符合我们迄今为止所描述的``双法计数''证明,但我们认为它值得阅读和惊叹,所以请务必仔细阅读。

\begin{proposition}{高斯配对求和}
    对于任意 $n \in \mathbb{N}$,
    \[\sum_{k=1}^{n} k = \frac{n(n+1)}{2}\]
\end{proposition}

\begin{proof}
    首先,通过观察易得 $\frac{n(n+1)}{2} = {n+1 \choose 2}$。

    现在,考虑一个由 $n + 1$ 行组成的规则三角形点阵,第 $k$ 行有 $k$ 个点。左边的求和表示点阵前 $n$ 行的``面积'',也就是前 $n$ 行的总点数。

    接下来,我们将这些点与 $(n+1)$ 行中的点对建立双射关系。对于任意一对点,从点阵中向上绘制指向内的对角线,可以找到上面行中的唯一一个点。反之,对于点阵中的任意一点,从点阵中向下绘制指向外的对角线,可以找到底行中的唯一一对点。因此,$(n + 1)$ 行中点对的数量是 $\sum_{k=1}^n k$。
\end{proof}

% !TeX root = ../../../book.tex

\subsection{标准计数对象}

我们在前一节已经讨论了几种标准的组合对象。然而,在``双法计数''证明中,难点在于确定要计数的对象!这些练习通常是这样提出的:``这是一个恒等式;请用双法计数来证明它。''这种说法并没有告诉你具体要计数什么,只是告诉你需要计数某些东西!在这个简短的章节中,我们将尽力提供一个实用的指南,帮助你``解开''组合恒等式,并构建一个双法计数证明。这些思路基于我们的经验以及一些组合学家常用的标准论证。

\subsubsection*{二项式系数的多种解释}

这些对象和相应的计数公式在前一节已经讨论过了。如果有不熟悉的部分,建议你重读上一节。这里我们要强调的是如何\emph{识别}某个计数对象是否与计数问题相关。例如,回想一下``Chairperson 恒等式'':
\[k\begin{pmatrix}n\\k\end{pmatrix} = n\begin{pmatrix}n-1\\k-1\end{pmatrix}\]
假装我们还没有证明它。这个恒等式只包含二项式系数的乘积(记住我们总是可以将 $k$ 写成 $\big({k \atop 1}\big)$ ),这意味着我们应该尝试计数一些可以通过简单的二项式系数描述的东西。最自然的选择是 $[n]$ 的子集;或者,我们可以考虑从一群人中选择一定大小的委员会,或者是包含 $k$ 个 \verb|1| 的二进制 $n$-元组。这三种选择中的任何一种都可以让我们轻松描述表达式中的各个项,并将它们联系起来。接下来,我们需要选择一种我们最熟悉的解释方式,即最容易解释所有项的方法。

如果我们选择使用选人组建委员会的方法,那么我们可以遵循之前证明中的论证。如果我们选择 $[n]$ 的子集,那么我们需要设计一个合理的两步过程来描述恒等式两边各项的乘积。在选择大小为 $k$ 的子集之后,右边的 $\big({n \atop 1}\big)$ 项可能表示我们先选出一个``特殊''元素,然后再填充子集的其他元素。然而,当我们讨论的是 $[n]$ 的子集时,就不能再使用``委员会主席''这样的术语了。(这就是为什么我们认为委员会的解释更合理且易于使用。只需为每个人编号,然后我们就可以放心地使用这些术语。)通常的解释可能是``圈出''一个元素,表明它是特殊元素。也就是说,方程两边都在计数大小为 $k$ 的 $[n]$ 的子集,其中一个元素被圈出。在左边,我们先选择子集然后圈出这个特殊元素;在右边,我们先圈出这个特殊元素并将其包括在子集中,然后再填充子集的其余元素。还有其他简单易懂的方法来解释这个论证,但我们想强调的是,除非从一开始就选择这种设置,否则``委员会''的术语不适用。(挑战性问题:你会如何在二进制 $n$-元组的背景下处理这个证明?提示:考虑允许某个``特殊''位置由某个固定符号填充,而不是选择 \verb|0| 或 \verb|1|。)

在数学中,我们经常会遇到二项式系数相乘的情况,并且往往``顶部系数''是相同的。例如,在双法计数证明的背景下,考虑如何描述如下项。(假设这只是等式的一边;另一边在这里不重要。)
\[\begin{pmatrix}n\\k\end{pmatrix}\begin{pmatrix}n\\\ell\end{pmatrix}\]
描述这种类型的乘积有两种合理的方法,选择哪一种取决于等式的另一边或其他相关项。我们将在这里介绍两种解释,并通过上下文来帮助你确定使用哪一种。

假设在委员会的背景下,每一项代表从 $n$ 个人中选择一个特定大小的委员会 ( $k$ 或 $\ell$) 。一种解释是从同一组 $n$ 个人中选择两个委员会。比如,一个部门有 $n$ 名教授,我们需要选择 $k$ 名教授来监督预算,再选择 $\ell$ 名教授来监督课程,而且教授们可能同时在两个委员会中任职。另一种解释是从不同的人群中选择两个委员会,但每组人群的大小都是 $n$ 。比如,一个班级有 $n$ 名男生和 $n$ 名女生,我们想从中选择 $k$ 名男生和 $\ell$ 名女生组成一个俱乐部。两种解释都是``正确的'',但具体选择哪种取决于问题的具体情况。

委员会类型的论证中,一个有用的概念是小委员会 (subcommittee) 的概念。由于从 $n$ 个人中选择 $k$ 个人组成的委员会已经代表了一个子集,所以小委员会实际上代表了一个子集的子集。因此,如果我们在一个等式中发现诸如
\[\begin{pmatrix}a\\b\end{pmatrix}\begin{pmatrix}b\\c\end{pmatrix}\]
这样的表达,我们可能会选择将其解释为从总人数为 $a$ 的人群中选择 $b$ 个人组成一个委员会,然后从这 $b$ 个人中选择 $c$ 个人组成一个小委员会。这可以描述为成立一个俱乐部并选出其官员,或组建一支运动队并确定首发阵容,或其他诸如此类的事情。

\subsubsection*{指数与过程}

除了二项式系数,在组合恒等式中经常出现的还有指数项,例如 $n^3, 2^n, n^{k-1}$ 等等。通常,这些项的解释会根据恒等式中其他项的上下文来确定。我们在这里介绍几种标准的、常见的和容易理解的解释方法。有趣的是,解释有时会取决于底数和指数哪个数更大!

考虑诸如这样的项
\[\begin{pmatrix}n\\k\end{pmatrix}2^k\]
假设我们根据恒等式的其余部分,将问题解释为``委员会''问题,并声明二项式系数 $\big({n \atop 1}\big)$ 代表从 $n$ 名学生中选出 $k$ 人组成一个委员会。那么 $2^k$ 项代表什么呢?记住,这个项可能来自一个 $k$-步过程,每一步都有两种选择。因为 $k$ 是委员会的大小,我们可以简单地将其描述为对每个成员的 $2$-步决策过程。例如,我们可以给每个成员分配一顶红帽子或蓝帽子;或者选择是否给每个成员一颗金星;或者让每个成员选择成为共和党人或民主党人。你可以自由发挥创造力!当然,所选择的解释必须与恒等式的其余部分相符,因此有时一种解释比另一种更容易理解。记住这一点,如果发现难以表达你的想法,要主动回过头修改你的解释。

再来考虑诸如这样的项
\[\begin{pmatrix}n\\k\end{pmatrix}2^n\]
假设我们还是将问题解释为``委员会''问题。这种情况与之前有什么不同呢?在这里,指数与二项式系数的``顶项''相同。因此,委员会的选择不一定与后续的 $n$-步过程相关。这可能意味着从一个有 $n$ 个学生的班级中选出 $k$ 个班干部,然后将每个学生分配到 $A$ 组或 $B$ 组(不考虑班干部的分配)。如果我们不采用``委员会''解释,这个项可能描述的是一个二进制 $n$-元组,其中恰好有 $k$ 个 \verb|1|,并且某些 \verb|0| 和 \verb|1| 被圈了起来。具体用哪种方法解释取决于问题的具体情况,以及你对这些解释的熟悉程度。

考虑当数字发生些许修改后如何解释这些项。例如
\[\begin{pmatrix}n\\k\end{pmatrix}4^n\]
可以解释为一个具有 $k$ 个成员的委员会,每个成员戴着红色、蓝色、绿色或黄色的帽子。例如
\[\begin{pmatrix}n\\k\end{pmatrix}5^n\]
可以解释为一个二进制 $n$-元组,其中恰好有 $k$ 个 \verb|1|,且每个 \verb|0| 和 \verb|1| 周围都有 $1$ 到 $5$ 个圈。

接下来,让我们探讨一些底数为变量而指数为定值的表达式。例如,考虑诸如这样的项
\[\begin{pmatrix}n\\k\end{pmatrix}2^k\]
在这里,我们从 $n$ 个对象中选择 $k$ 个对象,并进行一个每步有 $k$ 个选择的两步过程。也就是说,选择这 $k$ 个对象会影响后续的两步过程的结果。如果我们使用``委员会''进行解释,可以将其看作是先选出一个具有 $k$ 个人的委员会,然后从中选出两名官员 --- 比如一个发言人和一个财务官 --- 委员会中的任何人都可以担任这两个职位,甚至可以同时担任两个职位。

如果我们使用``元组''进行解释,可以将其描述为选择一个具有 $k$ 个 \verb|1| 的二进制 $n$-元组,其中一个 \verb|1| 被圈起来,另一个 \verb|1| 被框起来(也可能同一个 \verb|1| 既被圈也被框)。我们还可以使用``字母表''进行解释,从 $n$ 个字母中选择 $k$ 个字母,然后用这 $k$ 个字母组成两个字母的单词。思考这三种解释为什么都有效及其相互关系。尝试用这些解释来重写其中一个证明,并考虑当 $k^3$ 或 $k^4$ 时这些解释会有什么不同。

最后,考虑诸如这样的项
\[\begin{pmatrix}n\\k\end{pmatrix}n^3\]
在``委员会''的上下文下。由于指数项的底数与二项式系数的顶项相同,因此 $\big({n \atop k}\big)$ 项所代表的委员会与随后的三步过程之间不一定存在联系。因此,我们可以解释为选择一个 $k$ 人委员会,然后分配一条红色、一条蓝色和一条绿色丝带,其中一个人可能会收到多条丝带,任何人(无论是否在委员会内)都可能收到一条或多条丝带。用``二进制 $n$-元组''来解释该项留给你来完成。试试看吧!

\subsubsection*{求和即划分}

组合恒等式中常常会出现\emph{求和}。在双法计数证明中,处理这一点会稍微有些复杂,因为求和代表了一次多个项。不过,最重要的规则是:求和总是代表一个\emph{划分}。特别是,这个划分告诉我们所有划分集的基数。为了在双法计数证明中解释这一点,我们需要描述以下三个性质:
\begin{itemize}
    \item 划分集是什么。
    \item 为什么求和的索引\emph{限制}在上下文中是合理的。
    \item 对于任意索引,为什么对应集合的大小是求和中的项。
\end{itemize}
我们将通过一个例子来说明这些。\\

\begin{example}[支持/反对委员会恒等式]
    \[\begin{pmatrix}n\\k\end{pmatrix}2^{n-k} = \sum_{i=k}^{n}\begin{pmatrix}n\\i\end{pmatrix}\begin{pmatrix}k\\i\end{pmatrix}\]

    \begin{questions}{直觉:}
        从 $n$ 个人中选出一个由 $k$ 人组成的委员会。然后,确定那些没有进入委员会的人是否支持委员会的决定。我们也可以先选择至少 $k$ 个将会在委员会内或支持委员会的人,并将其他人设定为退出和反对委员会的人。然后,从这些人中再选出 $k$ 个人实际进入委员会,并将其余人设定为支持委员会的决定。(注意:在这些步骤中,详细说明我们将执行的每一步是非常重要的。不要假设读者会理所当然地理解这些步骤。)
    \end{questions}

    \begin{proof}
        假设有 $n$ 个人。设 $S$ 为从这 $n$ 个人中选择 $k$ 个人组成委员会的所有可能方式的集合,而每个不在委员会中的人都对委员会持有明确\textbf{支持}或\textbf{反对}态度。

        首先,我们可以通过多个步骤来得到 $|S|$:
        \begin{itemize}
            \item 从 $n$ 个人中选择 $k$ 个人组成一个委员会:有 $\big({n \atop k}\big)$ 种方法。
            \item 对于其余 $n-k$ 人,让他们每个人决定是\verb|支持|还是\verb|反对|委员会。这个过程有 $n-k$ 步,每步两个选择。所以根据乘法原理:有 $2^{n-k}$ 种方法。
        \end{itemize}
        根据乘法原理,我们得到 $|S| = \big({n \atop k}\big) \cdot 2^{n-k}$。

        其次,我们可以通过划分不同人数的支持者来确定 $|S|$ 的大小。根据 $S$ 的定义,从没有到所有 $n-k$ 个非委员会成员都\verb|支持|委员会。这样一来,委员会成员和他们的支持者总人数可以在 $k$ (含)到 $k + (n - k) = n$ (含)之间。

        对于每个满足 $k \ge i \ge n$ 条件的 $i$,令 $S_i \subseteq S$ 表示包含 $k$ 个委员会成员和 $i - k$ 个支持者的集合。 (注意 $0 \ge i - k \ge n - k$,这与之前提到的限制一致。)

        请注意,$\{S_i \mid k \ge i \ge n\}$ 是 $S$ 的一个划分。这是因为,$S$ 中的每个元素都可以通过其支持委员会的成员数量来表征,并且这个数量是一个特定值。现在,我们可以通过多步骤过程来找到每个 $i$ 对应的 $|S_i|$:
        \begin{itemize}
            \item 从所有 $n$ 个人种选择 $i$ 个人,其中包含潜在的委员会候选人:有 $\big({n \atop i}\big)$ 种方法。
            \item 指定其他 $n - i$ 个人明确反对我们组建的委员会。\\
                  (这一步是确定的,因此只有一种方法,但我们需要指出这一点,以便完整描述作其是 $S$ 中的元素这一结果。)
            \item 对于第一步选出的 $i$ 个人,从中选出 $k$ 个作为委员会成员:有 $\big({i \atop k}\big)$ 种方法。
            \item 将上一步中未被选中的 $i - k$ 个人指定为委员会的支持者,但他们不是委员会成员。\\
                  (再次强调,这里只有一种方法,但对结果进行完整描述需要我们明确指出这些成员的态度和状态。)
        \end{itemize}
        根据乘法原理,我们得到 $|S_i| = \big({n \atop i}\big)\big({i \atop k}\big)$。\\
        根据加法原理,我们得到 $|S| = \sum_{i=k}^{n} |S_i| = \sum_{i=k}^{n}\big({n \atop i}\big)\big({i \atop k}\big)$。\\
        由于我们通过两种方法得到了 $|S|$,我们可以将它们建立起相等关系。这证明了这一结论。
    \end{proof}
\end{example}

请注意,在我们的证明中,在明确划分之后,我们做了几件事。首先,我们解释了为什么这是一个划分。接着,我们说明了它如何与求和中的索引相关联。然后,我们解释了求和的上下限是如何对应这个划分并涵盖所有可能情况的。最后,对于任意 $i$,我们解释了为什么 $|S_i|$ 是求和中的相应项。

% !TeX root = ../../../book.tex

\subsection{二项式定理}\label{sec:section8.4.4}


% !TeX root = ../../../book.tex

\subsection{习题}

\subsubsection*{温故知新}

以口头或书面的形式简要回答以下问题。这些问题全都基于你刚刚阅读的内容,所以如果忘记了具体的定义、概念或示例,可以回去重读相关部分。确保在继续学习之前能够自信地回答这些问题,这将有助于你的理解和记忆!

\begin{enumerate}[label=(\arabic*)]
    \item 什么是\textbf{双法计数}论证的基本方法?
    \item 为本节中的每个示例证明写一个简短的\emph{证明摘要}。
    \item 当给出的恒等式中存在一个求和时,我们在后续的双法计数论证中必须讨论什么?
    \item 我们用哪些不同的方法证明了求和恒等式?它们在本质上有何相同之处?
\end{enumerate}

\subsubsection*{小试牛刀}

尝试回答以下问题。这些题目要求你实际动笔写下答案,或(对朋友/同学)口头陈述答案。目的是帮助你练习使用新的概念、定义和符号。题目都比较简单,确保能够解决这些问题将对你大有帮助!

\begin{enumerate}[label=(\arabic*)]
    \item 设 $\ell,k,n \in \mathbb{N}$,用双法计数证明
        \[\begin{pmatrix}n\\k\end{pmatrix}\begin{pmatrix}k\\\ell\end{pmatrix}=\begin{pmatrix}n\\\ell\end{pmatrix}\begin{pmatrix}n-\ell\\k-\ell\end{pmatrix}\]
    \item 用双法计数证明
        \[n \cdot 2^{n-1} = \sum_{k=1}^{n}\begin{pmatrix}n\\k\end{pmatrix} \cdot k\]
    \item 用双法计数证明
        \[3^n=\sum_{k=1}^{n}\begin{pmatrix}n\\k\end{pmatrix}2^{n-k} = \sum_{k=0}^{n}\begin{pmatrix}n\\k\end{pmatrix}2^k\]
        (\textbf{提示}:考虑使用三进制字符串集。)\\
        然后,解释它是如何从二项式定理推导出来的。
    \item 用双法计数证明 $k^2=\big({k \atop 1}\big)+2\big({k \atop 2}\big)$。\\
        应用\textbf{求和恒等式}推导出
        \[\sum_{k=1}^{n} k^2 = \frac{n(n+1)(2n+1)}{6}\]
    \item 用双法计数证明以下\textbf{几何级数公式}:
    \[\forall q \in \mathbb{N}-\{1\} \centerdot \forall n \in \mathbb{N} \centerdot 1+q+q^2+q^3+\dots+q^{n-1} = \sum_{k=0}^{n-1}q^k = \frac{q^n-1}{q-1}\]
    (注意:实际上,对于任意\emph{实数} $q \ne 1$,这个公式都是成立的,但我们所讨论的双法计数证明仅适用于\emph{自然数} $q \ne 1$。要证明实数版本,请使用归纳法。)\\
    (\textbf{提示}:考虑由 $q$ 个元素组成的所有 $n$-元组的集合,但不包括某一个特定元素……)
\end{enumerate}
