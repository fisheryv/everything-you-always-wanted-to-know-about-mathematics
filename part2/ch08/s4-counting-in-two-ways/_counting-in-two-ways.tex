% !TeX root = ../../../book.tex
\section{双法计数}\label{sec:section8.4}

如果你直接阅读本节内容,我们建议你先阅读一下前一节的最后一个例子,因为它为``双法计数''提供了很好的介绍和示例。在那个例子中,我们采用\emph{两种}不同方法计算了到达某个特定点的格路径数目,从而得出两个表达式必然相等的结论。具体而言,我们得到 ${x+y \choose x}={x+y \choose y}$。基于这一例子,我们将在本节概述一种通用策略,并将其应用于若干实例。通过这一过程,我们不仅能熟练掌握这一技巧,还将证明一些有用的组合结论,这些结论可用于解决其他问题。

首先,我们展示前一节例子的\emph{另一种}证明方法。该方法完全不涉及格路径,且更易于记忆和理解。

\begin{proposition}
    设 $n,k \in \mathbb{N} \cup \{0\}$。则 ${n \choose k} = {n \choose n-k}$。
\end{proposition}

\begin{proof}
    设 $S$ 为 $[n]$ 的子集,且大小为 $k$,即
    \[S = \{T \subseteq [n] \mid |T| = k\}\]
    根据 $k$-选择的定义,$|S|={n \choose k}$,因为构建一个集合 $T \subseteq [n]$ 且 $|T| = k$,实际上就是从 $n$ 个元素中选择 $k$ 个元素。

    另一种等价方式是,通过选择\emph{不包含}在 $T$ 中的 $n - k$ 个元素来构造集合 $T \subseteq [n]$,其中 $|T| = k$;这意味着剩下的 $n - (n - k) = k$ 个元素属于 $T$。这样的方法有 ${n \choose n-k}$ 种。由于每个这样的集合 $T$ 都可以通过这种方式构造,我们得出 $|S| = {n \choose n-k}$。

    因此,${n \choose k} = {n \choose n-k}$。
\end{proof}

我们认为这一证明更易于记忆,因为整个证明可以用一句话概括:

\begin{quotation}
    ``通过确定要包含的元素或要排除的元素来计算 $[n]$ 的 $k$-元素子集。''
\end{quotation}

这是我们需要掌握的核心思想;基于这一思想,我们可以重新构建整个证明。逐句``死记硬背''证明过程毫无意义;相反,牢记证明的\emph{核心思路},再在此基础上补充细节,将更为有效。

\clearpage

% !TeX root = ../../../book.tex

\subsection{方法总结}




% !TeX root = ../../../book.tex

\subsection{示例}

让我们仔细看几个例子。这将帮助你理解如何应用``双法计数''技术,并为你提供一些典型示例供你参考和复习,此外还介绍了一些可用于未来问题的基本组合数学结论。在每个例子中,我们不仅证明所讨论的结论,还解释如何推导证明、构建论证时的思考过程,以及你如何自己尝试解决类似问题。双法计数证明的一个优点是,在证明结束时,我们通常能简洁地总结主要思想。我们将为每个证明总结核心思想,并鼓励你在完成类似证明后也进行同样的总结。这样,证明的思想更容易记忆,且只需一句话就能重构整个证明。\\

\begin{proposition}{帕斯卡恒等式 (Pascal's Identity)}
    对于任意 $n,k \in \mathbb{N}$,
    \[{n \choose k}={n-1 \choose k}+{n-1 \choose k-1}\]
\end{proposition}

\begin{questions}{证明策略:}
    看到像 ${n \choose k}$ 这样的二项式系数,表明我们可能想要计算 $[n]$ 中特定大小的子集。公式左边很容易理解(计算所有大小为 $k$ 的子集),但右边该如何解释?看到两项相加,表明这里存在某种划分。我们需要找到 $[n]$ 中大小为 $k$ 的子集的某个特性,使得一部分子集具有该特性,而另一部分没有。注意到两项的唯一区别在于``底部系数'',我们可以据此找到一个有效的划分……在继续阅读之前,尝试自己思考一下!
\end{questions}

\begin{proof}
    设 $S = \{T \subseteq [n] \mid |T| = k\}$。根据 $k$-选择的定义,我们知道 $|T| = {n \choose k}$。

    定义以下两个集合:
    \begin{align*}
        A & = \{T \subseteq [n] \mid |T| = k ∧ 1 \in T\}    \\
        B & = \{T \subseteq [n] \mid |T| = k ∧ 1 \notin T\}
    \end{align*}

    显然,$A \cap B = \varnothing$,因为对于任意集合 $T$,$1 \in T$ 和 $1 \notin T$ 不可能同时成立。并且 $S = A \cup B$,因为对于任意集合 $T \in S$,要么 $1 \in T$ 要么 $1 \notin T$。因此 $\{A,B\}$ 构成 $S$ 的一个划分,由此可知 $|S| = |A| + |B|$。

    为了计算 $|A|$,我们可以通过一个两步过程来构造任意元素 $T \in A$:
    \begin{enumerate}[label=(\arabic*)]
        \item 将元素 $1$ 加入 $T$;
        \item 从剩余的 $n-1$ 个元素中选择 $k-1$ 个元素加入 $T$。
    \end{enumerate}

    由乘法原理可得:
    \[|A| = 1 \cdot {n-1 \choose k-1} = {n-1 \choose k-1}\]

    同理,为了计算 $|B|$,我们可以通过一个两步过程来构造任意元素 $T \in B$:
    \begin{enumerate}[label=(\arabic*)]
        \item 从 $T$ 中删除元素 $1$;
        \item 从剩余的 $n-1$ 个元素中选择 $k$ 个元素构成 $T$。
    \end{enumerate}

    由乘法原理可得:
    \[|B| = {n-1 \choose k}\]

    根据加法原理,我们有:
    \[|S|=|A|+|B| = {n-1 \choose k-1} + {n-1 \choose k}\]

    将 $|S|$ 的两个表达式用等号连结,我们得出
    \[{n \choose k}={n-1 \choose k}+{n-1 \choose k-1}\]
\end{proof}

\begin{questions}{证明总结:}
    通过将 $[n]$ 中所有 $k$ 元子集划分为包含或不包含某一特定元素(例如 $1$),从而完成对子集的计数。
\end{questions}

\begin{questions}{问题:}
    若改用元素 $n$ 而非元素 $1$ 来构造划分,证明过程是否会有所不同?答案是否定的。关键在于选取任意一个特定元素,并基于该元素定义划分集合 $A$ 和 $B$,证明结构保持不变。
\end{questions}

\begin{questions}{历史注记:}
    此命题以法国数学家布莱兹·帕斯卡 (Blaise Pascal) 命名。你可能听说过帕斯卡三角 (Pascal's Triangle),该三角由自然数排列而成,构造方法如下:首行和第二行均为 $1$,三角的左右边界均为 $1$,内部每个数等于其上方两数之和。根据刚才证明的公式,你能猜出三角中的数字吗?正是二项式系数!第 $n$ 行(从 $0$ 开始计)包含形如 $\binom{n}{k}$ 的数值,其中 $k$ 从左至右递增。帕斯卡三角具有许多有趣的性质,我们将在后续示例与练习中进一步探索。
    \begin{center}
        \begin{tabular}{ccccccccccc}
              &     &     &     &      & $1$ &      &     &     &     &     \\
              &     &     &     & $1$  &     & $1$  &     &     &     &     \\
              &     &     & $1$ &      & $2$ &      & $1$ &     &     &     \\
              &     & $1$ &     & $3$  &     & $3$  &     & $1$ &     &     \\
              & $1$ &     & $4$ &      & $6$ &      & $4$ &     & $1$ &     \\
          $1$ &     & $5$ &     & $10$ &     & $10$ &     & $5$ &     & $1$ \\
        \end{tabular}
    \end{center}
\end{questions}

\begin{proposition}{主席恒等式 (Chairperson Identity)}
    对于任意 $n,k \in \mathbb{N}$,
    \[k{n \choose k}=n{n-1 \choose k-1}\]
\end{proposition}

\begin{questions}{证明策略:}
    这个等式的两边都是两项的\emph{乘积},因此我们需要寻找两个两步过程来构造相同的元素集合。左边的项可以表示为 ${n \choose k} \cdot k$,由于乘法满足交换律,这等价于从 $[n]$ 中选择 $k$ 个元素,然后进行下一步操作。如果将 $k$ 写成 ${k \choose 1}$,第二步就变得清晰了:从已选出的 $k$ 个元素中再选择一个元素。

    这引入了一种描述子集及其特定元素的新策略:\emph{委员会 (Committee)} 和\emph{领导者 (Leader)}。这种策略在组合证明中非常流行,因为它减少了技术性数学语言和符号的使用,使关键思想更易于理解。我们将在下面的证明中展示如何运用这种策略,并与使用更多数学语言的证明进行对比。在继续阅读之前,试试看你能不能预见我们如何描述等式右边的过程……
\end{questions}

\begin{proof}
    给定 $n,k \in \mathbb{N}$。设 $S$ 为从 $n$ 个人中选出 $k$ 人组成委员会,并指定其中一人为主席的集合。一种构造集合 $S$ 中元素的方法是:先选出 $k$ 人组成委员会,然后再从中选出一人作为主席。根据乘法原理,我们可以得出
    \[|S| = {n \choose k} \cdot {k \choose 1} = k{n \choose k}\]

    另一种构造集合 $S$ 中元素的方法是:先从所有 $n$ 个人中选出主席,然后从剩余的 $n-1$ 人中选出 $k-1$ 人组成委员会。根据乘法原理,我们可以得出
    \[|S| = {n \choose 1} \cdot {n-1 \choose k-1} = n{n-1 \choose k-1}\]

    将 $|S|$ 的两个表达式用等号连结,我们得到
    \[k{n \choose k}=n{n-1 \choose k-1}\]
\end{proof}

\begin{questions}{证明总结:}
    通过先选择委员会成员再选择主席,或者先选择主席再选择其余委员会成员,计算从 $n$ 个人中选出 $k$ 人委员会的方法数。
\end{questions}

\begin{questions}{注释:}
    如果我们尝试用纯粹的数学语言(即集合论)来描述这个证明,会怎样呢?在集合论中,``主席''具体对应什么?从 $n$ 个人中选出一个大小为 $k$ 的委员会可以表示为集合 $T \subseteq [n]$,其中 $|T| = k$。但如何区分该集合中 $k$ 个成员分别担任主席的情况?一个合理的方法是定义一个有序对,其中第一个元素是委员会成员的集合,第二个元素指定主席。基于这一策略,我们定义集合:
    \[\hat{S} = \{(T, x) \mid T \subseteq [n] \land |T| = k \land x \in T\}\]
    这个集合 $\hat{S}$ 与上述证明中定义的 $S$ 相同,因为它包含了所有包含一个主席的 $k$ 人委员会。然而,在计数 $\hat{S}$ 的元素时,我们可能仍然会使用委员会和主席的口语描述!(你可以尝试不使用这些描述来计数 $\hat{S}$ 的元素。)这种描述方式更自然,也更容易理解。总之,没有必要严格地使用集合论语言描述这些委员会;但指出这种可能性是重要的。这表明我们在上述证明中的描述是足够严格的,它们基于数学概念,但用其他术语表达时更易于理解和遵循。

    本节的练习中介绍了几个涉及委员会和子委员会的双法计数证明的例子。我们在这里再举一个例子作为练习。\\
\end{questions}

\begin{proposition}{所有规模的委员会}
    \[\sum_{k=0}^{n} {n \choose k}=2^n\]
\end{proposition}

\begin{questions}{证明策略:}
    右侧的表达式可能有多种含义,但它似乎涉及一个 $n$ 步过程,每一步都有两种选择。稍后我们将详细探讨这个表达式。左侧表示一个划分,因为它是多个项之和。求和中的每一项 ${n \choose k}$ 表示从 $n$ 个人中选出 $k$ 个人组成委员会的方法数。当 $k$ 从 $0$ 变化到 $n$ 时,我们涵盖了所有可能的委员会规模。这表明我们在计算从 $n$ 个人中选出所有可能的委员会。既然我们知道右侧在计算什么,我们可以为此构建一个证明……在阅读我们的证明之前,试着自己推导一下!
\end{questions}

\begin{proof}
    设 $n \in \mathbb{N}$。并设 $S$ 为从 $n$ 个人中选出的所有规模的委员会的集合。$S$ 中的元素是规模介于 $0$ 到 $n$(含端点)人的委员会。对于每个 $k \in [n] \cup \{0\}$,设 $S_k$ 为 $k$ 人委员会的集合。则集合 $\{S_k \mid k \in [n] \cup \{0\}\}$ 构成 $S$ 的一个划分。因此,根据加法原理,我们得出
    \[|S| = \sum_{k=0}^{n} |S_k| = \sum_{k=0}^{n} {n \choose k} \]
    其中 $|S_k| = {n \choose k}$,因为 $S_k$ 是 $[n]$ 中所有 $k$ 元子集的集合。

    我们还可以通过另一种方式计算集合 $S$ 的元素数量:考虑一个包含 $n$ 个人的集合,并为他们从 $1$ 到 $n$ 编号(例如,可以给每个人分发一件印有唯一编号的 T 恤)。为了构建一个委员会,我们按数字顺序排列所有人,并沿着队伍前进,对每个人说``Yes''或``No'',以表示他们是否属于正在构建的委员会。每一个由 $n$ 个``Yes''和``No''组成的序列都会唯一对应一个委员会。由于这是一个 $n$ 步过程,每一步有两种选择,根据乘法原理,我们有 $2^n$ 种方式完成这个过程,因此 $|S| = 2^n$。通过将这两个 $|S|$ 的表达式等同起来,我们得出结论
    \[\sum_{k=0}^{n} {n \choose k}=2^n\]
\end{proof}

\begin{questions}{证明总结:}
    通过基于规模的划分来计算 $[n]$ 的所有子集。(注意:虽然这个总结使用集合术语表述,但我们认为用委员会术语来书写和理解证明会更直观。)
\end{questions}

你可能会觉得这个证明有些冗长,特别是因为我们已经通过归纳法证明了 $|\mathcal{P}([n])| = 2^n$。既然我们在考虑所有规模的委员会,实际上就是说``令 $S = \mathcal{P}([n])$'',然后用两种方法计算 $|S|$。然而,当我们使用委员会术语书写证明时,如果不解释这些表述为何等价,就不能直接切换到讨论 $[n]$ 的子集。作为练习,尝试不使用委员会术语,完全用集合符号重写这个证明。你更喜欢哪种方法?

\subsubsection*{求和恒等式}

下面的组合恒等式非常有用,将在本章后续的证明和练习中反复出现,因此我们提前在此给出结论。此外,我们将展示\emph{两种不同的}双法计数证明。该问题甚至还有第三种证明方法,留作练习由你来完成。这两种证明方法属于标准的组合证明技巧,我们鼓励你阅读并理解它们之间的联系。你可能会问,为什么要对同一结论给出两种证明?(``一个证明难道不够吗?'')通过理解这些证明的结构及其等价性,你将对这些证明技术有更深入的认识,并能更灵活地运用它们。请相信我们!此外,我们还会将这些技术与前面问题中使用的委员会方法进行比较,探讨这三种方法之间的关联。\\

\begin{theorem}{求和恒等式}\label{theorem8.4.5}
    设 $n,k \in \mathbb{N}$,则
    \[\sum_{i=0}^{n} {i \choose k} = {n+1 \choose k+1}\]
\end{theorem}

\begin{questions}{证明策略 1:}
    右侧是一个二项式系数,表明我们在讨论 $[n+1]$ 中大小为 $k+1$ 的子集。左侧的求和提示我们根据某种性质将这些子集划分成若干类。由于求和中的二项式系数底部都是 $k$ 而非 $k+1$,这意味着索引 $i$ 可能表示子集中某个特定元素(例如最大元素)的取值。在继续阅读之前,请尝试自行写出这一划分的具体细节……
\end{questions}

\begin{proofs}{证明 1.}
    设 $n,k \in \mathbb{N}$。定义
    \[S = \{T \in [n + 1] \mid |T| = k + 1\}\]
    根据 $[n+1]$ 的 $k+1$-选择定义,我们知道 $|S| = {n+1 \choose k+1}$。

    接着,对于每一个 $i \in [n] \cup \{0\}$,定义集合
    \[S_i = \{T \in S \mid i + 1 \in T \land (\forall j \in T \centerdot j \le i + 1)\}\]
    即 $S_i$ 是由 $[n + 1]$ 中所有大小为 $k + 1$ 且\emph{最大索引}元素为 $i + 1$ 的子集构成的集合。我们断言 $\{S_i \mid i \in [n] \cup \{0\}\}$ 构成 $S$ 的一个划分。

    首先,我们注意到 $S_i \cap S_j = \varnothing$ ,当且仅当 $i \ne j$。这是因为 $T \in S_i$ 意味着 $i + 1 \in T$;具体来说,若 $i > j$,则任意 $U \in S_j$ 的最大索引元素是 $j + 1$,而 $j + 1$ 小于 $i + 1$;若 $i < j$,则任意 $U ∈ S_j$ 包含 $j + 1$,但 $j + 1 \notin T$。

    其次,我们注意到每个 $T \in S$ 在 $1$ 到 $n + 1$ 之间必有一个最大索引元素,因此属于某个 $S_i$ 集合。为了更好地理解这一部分的证明,我们在下面提供了一个示例,展示了 $n = 4, k = 2$ 的情况。请注意,有几个集合是空集。通常,对于每个 $i \in [k - 1] \cup \{0\}, S_i=\varnothing$,这很合理,因为对于所有这些 $i$ 值,${i \choose k}=0$。

    接下来,我们必须找出对于每个 $i \in [n] \cup {0}, |S_i|$ 的值。为了构建元素 $T \in S_i$,我们可以定义一个两步过程:
    \begin{enumerate}[label=(\arabic*)]
        \item 将元素 $i + 1$ 加入 $T$;
        \item 从 $i$ 个较小的索引元素中,选择 $k$ 个加入 $T$。
    \end{enumerate}
    根据乘法原理和选择的定义,共有 ${i \choose k}$ 种方式。

    因此,根据加法原理,我们得出
    \[|S| = \sum_{i=0}^{n} |S_i| = \sum_{i=0}^{n} {i \choose k}\]

    将两个关于 $|S|$ 的表达式建立相等关系,我们得到
    \[\sum_{i=0}^{n} {i \choose k} = {n+1 \choose k+1}\]
\end{proofs}

\begin{tcolorbox}[colback=gray!10,
    colframe=black,
    width=\textwidth,
    arc=2mm, auto outer arc,
    title={$n = 4, k = 2$ 的示例},breakable,enhanced jigsaw,
    before upper={\parindent15pt\noindent},	]
    \begin{align*}
        S =  \big\{&\{1, 2, 3\}, \{1, 2, 4\}, \{1, 2, 5\}, \{1, 3, 4\}, \{1, 3, 5\},       \\
                   &\{1, 4, 5\}, \{2, 3, 4\}, \{2, 3, 5\}, \{2, 4, 5\}, \{3, 4, 5\}\big\}  \\
        S_1 = \enspace\; & \varnothing                                                     \\
        S_2 = \enspace\; & \varnothing                                                     \\
        S_3 = \big\{&\{1, 2, 3\} \big\}                                                    \\
        S_4 = \big\{&\{1, 2, 4\}, \{1, 3, 4\}, \{2, 3, 4\} \big\}                          \\
        S_5 = \big\{&\{1, 2, 5\}, \{1, 3, 5\}, \{1, 4, 5\}, \{2, 3, 5\}, \{2, 4, 5\}, \{3, 4, 5\} \big\}
    \end{align*}
\end{tcolorbox}

\begin{questions}{证明 1 总结:}
    通过根据子集中最大索引元素进行划分,计算 $[n+1]$ 中具有 $(k+1)$ 个元素的子集数量。
\end{questions}

这一证明策略源于我们最初的观察:二项式系数 $\binom{n+1}{k+1}$ 表示从 $[n+1]$ 中选取子集的数量。然而,我们还可以通过另一种常见的计数方法——二进制元组,来解释该系数。这将引导我们从不同角度思考左侧的求和式。现在,让我们深入探讨这一证明!

\begin{proofs}{证明 2.}
    设 $n,k \in \mathbb{N}$,并设 $S$ 为所有包含 $k+1$ 个 \verb|1| 的二进制 $(n+1)$-元组的集合。即 $S \subset \{0,1\}^{n+1}$,且每个 $T \in S$ 包含 $k+1$ 个 \verb|1| 和 $(n + 1) - (k + 1) = n - k$ 个 \verb|0|。

    我们可以通过观察直接确定 $|S|$ 的值。构造 $S$ 的一个元素相当于从 $n+1$ 个位置中选择 $k+1$ 个填入 \verb|1|(其余位置填入 \verb|0|)。因此,$|S| = \binom{n+1}{k+1}$。

    接下来,我们根据\emph{最右侧} \verb|1| 出现的位置对 $S$ 进行划分。具体来说,对于每个 $i \in [n+1]$,令 $S_i$ 为 $S$ 中最右侧  \verb|1| 出现在第 $i$ 个位置(从左到右计数)的所有元组构成的子集。(参见证明下方的示例,其中 $n$ 和 $k$ 已赋予具体值。)为计算 $S_i$ 的元素数量,我们在第 $i$ 个位置固定一个 \verb|1|,然后从前 $i-1$ 个位置中选择 $k$ 个位置填入 \verb|1|,其余位置填入 \verb|0|。根据乘法原理,$|S_i| = \binom{i-1}{k}$。

    现在验证 $\{S_i \mid i \in [n+1]\}$ 构成 $S$ 的一个划分。首先,当 $i \ne j$ 时,$S_i \cap S_j = \varnothing$。若 $i < j$,则 $S_i$ 中任何元组的第 $j$ 个位置为 \verb|0|,而 $S_j$ 中任何元组的第 $j$ 个位置为 \verb|1|,因此没有元组同时属于两者;类似地,若 $j < i$,则第 $i$ 个位置在 $S_i$ 中为 \verb|1|,在 $S_j$ 中为 \verb|0|,同样无交集。其次,$S$ 中每个元组均有一个最右侧的 \verb|1|,其位置在 $1$ 至 $n+1$ 之间,因此每个元组均属于某个 $S_i$。

    因此,根据加法原理可得
    \[|S| = \sum_{j=1}^{n+1} |S_i| = \sum_{j=1}^{n+1} {j-1 \choose k}\]

    通过重新定义求和索引为 $i = j - 1$,我们可以将上面表达式写成
    \[|S| = \sum_{i=0}^{n} {i \choose k}\]

    将两个关于 $|S|$ 的表达式建立相等关系,即可证明要证明的结论。
\end{proofs}

\begin{tcolorbox}[colback=gray!10,
    colframe=black,
    width=\textwidth,
    arc=2mm, auto outer arc,
    title={$n = 4, k = 2$ 的示例},breakable,enhanced jigsaw,
    before upper={\parindent15pt\noindent},	]
    \begin{align*}
        S =   \big\{&\{11100\}, \{11010\}, \{11001\}, \{10110\}, \{10101\},                  \\
                    &\{10011\}, \{01110\}, \{01101\}, \{01011\}, \{00111\}\big\}             \\
        S_1 = \enspace\; & \varnothing                                                       \\
        S_2 = \enspace\; & \varnothing                                                       \\
        S_3 = \big\{&\{11100\} \big\}                                                        \\
        S_4 = \big\{&\{11010\}, \{10110\}, \{01110\} \big\}                                  \\
        S_5 = \big\{&\{11001\}, \{10101\}, \{10011\}, \{01101\}, \{01011\}, \{00111\} \big\} 
    \end{align*}
\end{tcolorbox}
\clearpage
\begin{questions}{证明 2 总结:}
    基于最右侧的 \verb|1| 出现的位置进行划分,计算恰好有 $k + 1$ 个 \verb|1| 的二进制 $(n+1)$-元组的数量。
\end{questions}

通过这段内容,我们希望你能更好地理解双法计数论证,并学会如何通过观察等式的形式来构思这样的论证。这一主题需要多加练习,建议你尝试完成本节末尾的练习题。若需进一步帮助,我们推荐阅读下一节。下一节将介绍一些启发式方法,用于分析双计数问题,并帮助你选择合适的集合 $S$ 进行证明。这些方法基于本章前面介绍的标准计数对象及其对应公式。

在继续之前,我们想展示最后一个双法计数证明,因为我们认为它极具启发性,既聪明又优雅。我们并不期望你能独立提出这样的论证,特别是因为它并不完全符合我们迄今为止所描述的``双法计数''证明,但我们认为它值得仔细阅读和欣赏,因此请务必细心阅读。\\

\begin{proposition}{高斯配对求和}
    对于任意 $n \in \mathbb{N}$,
    \[\sum_{k=1}^{n} k = \frac{n(n+1)}{2}\]
\end{proposition}

\begin{proof}
    首先,通过观察易得 $\frac{n(n+1)}{2} = {n+1 \choose 2}$。

    现在,考虑一个由 $n + 1$ 行组成的规则三角形点阵,其中第 $k$ 行有 $k$ 个点。左边的求和表示点阵前 $n$ 行的``面积'',即前 $n$ 行的总点数。

    接下来,我们建立这些点与 $(n+1)$ 行中点对之间的双射关系。对于任意一对点,通过向上绘制指向内的对角线,可以在上方行中找到唯一一个点;反之,对于点阵中的任意一点,通过向下绘制指向外的对角线,可以在底行中找到唯一一对点。因此,$(n+1)$ 行中点对的数量等于 $\sum_{k=1}^{n} k$。
\end{proof}

% !TeX root = ../../../book.tex

\subsection{标准计数对象}

在上一节中,我们已经介绍了几种常见的组合对象。然而,在``双法计数''证明中,真正的挑战在于确定所要计数的具体对象!这类练习通常以这样的形式出现:``这是一个恒等式,请用双法计数来证明它。''这样的表述并没有指明具体的计数对象,而只是提示我们需要通过计数来验证等式。在本节中,我们将提供一个实用指南,帮助大家``解开''组合恒等式,并构建相应的双法计数证明。这些思路基于我们的经验以及组合数学中一些常用的标准论证方法。

\subsubsection*{二项式系数的多种解释}

这些对象及其对应的计数公式已在上一节中讨论过。如果有不熟悉的地方,建议回顾前一节的内容。这里我们重点强调如何\emph{识别}一个计数对象是否与给定的计数问题相关。以``主席恒等式''为例:
\[k{n \choose k} = n{n-1 \choose k-1}\]

假设我们尚未证明该等式。该恒等式仅包含二项式系数的乘积(注意我们总可以将 $k$ 写作 ${k \choose 1}$),这意味着我们可以尝试计数那些能够用简单二项式系数描述的对象。最自然的选择是 $[n]$ 的子集;或者,我们也可以考虑从一群人中选出特定规模的委员会,或是含有 $k$ 个 \verb|1| 的二进制 $n$-元组。这三种选择中的任意一种都有助于我们轻松描述等式中的各项,并将它们相互关联。接下来,我们需要选择一种最易于理解且能清晰解释各项含义的表述方式。

如果选择委员会的解释方式,我们可以沿用之前的证明思路。如果选择 $[n]$ 的子集,则需要设计一个合理的两步过程来描述等式两边的乘积项。在选出大小为 $k$ 的子集后,右边的 ${n \choose 1}$ 项可能表示我们先选出一个``特殊''元素,再填充子集的其余部分。然而,当我们讨论 $[n]$ 的子集时,就不能再使用``委员会主席''这样的术语(这正是我们认为委员会解释更为直观的原因:只需为成员编号,便可自然地使用这些术语)。一种常见的替代说法是``圈出''一个元素以标记其特殊性。也就是说,等式两边都在计数大小为 $k$ 的 $[n]$ 子集,且其中有一个元素被特别标记。左边表示先选子集再标记特殊元素;右边表示先标记特殊元素并确保其属于子集,再补充子集的其余部分。还有其他直观的方式可以解释这一论证,但需要注意的是,除非从一开始就采用委员会框架,否则不宜使用相关术语。(思考题:如何在二进制 $n$-元组的背景下完成这一证明?提示:考虑允许某个``特殊''位置由固定符号填充,而非仅限于 \verb|0| 或 \verb|1|。)

在数学中,我们经常遇到二项式系数相乘的情形,且往往``顶部系数''相同。例如,在双法计数证明中,考虑如何描述如下项(假设这只是等式的一边,另一边在此处不重要):
\[{n \choose k}{n \choose \ell}\]

这类乘积有两种合理的解释方式,具体选择取决于等式的另一边或其他相关项。下面我们将介绍这两种解释,并通过上下文帮助你判断选用哪一种。

假设在委员会的解释框架下,每一项代表从 $n$ 个人中选出一个特定规模的委员会(大小为 $k$ 或 $\ell$)。一种解释是从同一组 $n$ 个人中选出两个委员会。例如,某系有 $n$ 名教授,需要选出 $k$ 名教授监督预算,同时选出 $\ell$ 名教授监督课程,且允许同一教授兼任两个委员会的职务。另一种解释是从不同人群但规模均为 $n$ 的群体中选出两个委员会。例如,一个班级有 $n$ 名男生和 $n$ 名女生,从中选出 $k$ 名男生和 $\ell$ 名女生组成一个俱乐部。两种解释都是合理的,但具体选择需要根据问题的上下文决定。

在委员会类型的论证中,子委员会 (subcommittee) 是一个有用的概念。由于从 $n$ 个人中选出 $k$ 人组成的委员会本身就是一个子集,所以子委员会实际上表示该子集的一个子集。因此,如果在等式中出现如下表达式:
\[{a \choose b}{b \choose c}\]

我们可能会将其解释为:从 $a$ 个人中选出 $b$ 人组成一个委员会,再从这个委员会中选出 $c$ 人组成一个子委员会。这可以类比为成立一个俱乐部并选举其官员,或组建一支运动队并确定首发阵容等场景。

\subsubsection*{指数与过程}

除了二项式系数,组合恒等式中还经常出现指数项,例如 $n^3$, $2^n$, $n^{k-1}$ 等。通常,这些项的解释取决于恒等式中其他项的上下文。我们在此介绍几种标准、常见且易于理解的解释方法。有趣的是,解释有时会取决于底数和指数的大小关系!

考虑诸如这样的项
\[{n \choose k}2^k\]

假设我们根据恒等式的其余部分,将问题解释为``委员会''问题,并声明二项式系数 ${n \choose k}$ 代表从 $n$ 名学生中选出 $k$ 人组成一个委员会。那么 $2^k$ 项代表什么呢?记住,这个项可能来源于一个 $k$-步过程,每一步都有两种选择。由于 $k$ 是委员会的大小,我们可以将其描述为对每个成员进行一个二选一的决策过程。例如,可以为每个成员分配一顶红帽子或蓝帽子;或者选择是否授予每个成员一颗金星;或者让每个成员选择成为共和党人或民主党人。你可以自由发挥创造力!当然,所选择的解释必须与恒等式的其余部分相符,因此有时一种解释比另一种更易理解。记住这一点,如果发现难以表达想法,要主动回头调整解释。

再来考虑诸如这样的项
\[{n \choose k}2^n\]

假设我们仍将问题解释为``委员会''问题。这种情况与之前有何不同?在这里,指数与二项式系数的``顶项''相同。因此,委员会的选择不一定与后续的 $n$-步过程相关。这可能意味着从一个有 $n$ 个学生的班级中选出 $k$ 名班干部,然后将每个学生(包括班干部)分配到 $A$ 组或 $B$ 组,且分配过程不区分班干部身份。如果我们不采用``委员会''解释,这个项可能描述一个二进制 $n$-元组,其中恰好有 $k$ 个 \verb|1|,且某些 \verb|0| 和 \verb|1| 被圈出。具体采用哪种解释取决于问题的背景以及你对这些解释的熟悉程度。

考虑当数字稍作修改时,如何解释这些项。例如
\[{n \choose k}4^n\]

可以解释为:选出一个 $k$ 人委员会,然后每个成员戴一顶红色、蓝色、绿色或黄色的帽子。又如
\[{n \choose k}5^n\]

可以解释为:一个二进制 $n$-元组,其中恰好有 $k$ 个 \verb|1|,且每个 \verb|0| 和 \verb|1| 周围有 $1$ 到 $5$ 个圈。

接下来,让我们探讨一些底数为变量而指数为定值的表达式。例如,考虑如下的项
\[{n \choose k}2^k\]

在这里,我们从 $n$ 个对象中选择 $k$ 个对象,并进行一个两步过程,每步有 $k$ 种选择。也就是说,选择这 $k$ 个对象会影响后续两步过程的结果。如果我们使用``委员会''解释,可以将其视为先选出一个 $k$ 人委员会,然后从中选出两名官员(例如发言人和财务官),委员会中的任何成员都可以担任这两个职位,甚至一人兼任两职。

如果我们使用``元组''解释,可以将其描述为一个二进制 $n$-元组,其中恰好有 $k$ 个 \verb|1|,且一个 \verb|1| 被圈出,另一个 \verb|1| 被框出(允许同一个 \verb|1| 既被圈也被框)。我们还可以使用``字母表''进行解释,从 $n$ 个字母中选择 $k$ 个字母,然后用这 $k$ 个字母组成两个字母的单词。思考这三种解释为何有效及其相互关系。尝试用这些解释重写一个证明,并考虑当项为 $k^3$ 或 $k^4$ 时,解释会有何不同。

最后,考虑诸如这样的项
\[{n \choose k}n^3\]

在``委员会''的语境下。由于指数项的底数与二项式系数的顶项相同,因此 ${n \choose k}$ 项所代表的委员会与随后的三步过程之间不一定存在联系。因此,我们可以解释为选出一个 $k$ 人委员会,然后分配红色、蓝色和绿色丝带各一条,其中一个人可能获得多条丝带,且任何人(无论是否在委员会中)都可能获得一条或多条丝带。用``二进制 $n$-元组''解释该项的任务留给你。试试看吧!

\subsubsection*{求和即划分}

组合恒等式中常常会出现\emph{求和}。在双法计数证明中,处理这一点可能稍显复杂,因为求和表示多个项的总和。不过,最重要的规则是:求和总是代表一个\emph{划分}。特别是,这个划分揭示了所有子集的大小。为了在双法计数证明中阐明这一点,我们需要描述以下三个性质:
\begin{itemize}
    \item 划分的子集是什么。
    \item 为什么求和的索引\emph{限制}在上下文中是合理的。
    \item 对于任意索引,为什么对应集合的大小是求和中的项。
\end{itemize}
我们将通过一个例子来说明这些。

\begin{example}[支持/反对委员会恒等式]
    \[{n \choose k}2^{n-k} = \sum_{i=k}^{n}{n \choose i}{k \choose i}\]
    \begin{questions}{直觉:}
        考虑从 $n$ 个人中选出一个由 $k$ 人组成的委员会,并确定非委员会成员是否支持委员会的决定。另一种方式是,先选择至少 $k$ 个可能进入委员会或支持委员会的人,其余人则反对委员会。然后,从这些支持者中选出 $k$ 人实际进入委员会,其余支持者则仅支持但不进入委员会。(注意:在这些步骤中,详细说明每一步至关重要,不要假设读者能自动理解。)
    \end{questions}

    \begin{proof}
        假设有 $n$ 个人。定义集合 $S$ 为从这 $n$ 个人中选择 $k$ 个人组成委员会的所有可能方式,而每个非委员会成员明确表示\textbf{支持}或\textbf{反对}委员会。

        首先,我们可以通过多个步骤计算 $|S|$:
        \begin{itemize}
            \item 从 $n$ 个人中选择 $k$ 个人组成委员会:有 ${n \choose k}$ 种方法。
            \item 对于其余 $n-k$ 人,每个人决定\textbf{支持}或\textbf{反对}委员会。这个过程有 $n-k$ 步,每步两个选择。所以根据乘法原理:有 $2^{n-k}$ 种方法。
        \end{itemize}

        根据乘法原理,可得 $|S| = {n \choose k} \cdot 2^{n-k}$。

        其次,我们可以根据支持者的人数划分 $S$。根据定义,\textbf{支持}委员会的非委员会成员人数可以从 $0$ 到 $n-k$。因此,委员会成员和支持者的总人数从 $k$ 到 $n$(含端点)。

        对于每个满足 $k \le i \le n$ 的 $i$,令 $S_i \subseteq S$ 表示那些有 $k$ 个委员会成员和 $i - k$ 个支持者的集合。(注意 $0 \le i - k \le n - k$,这与之前的范围一致。)

        请注意,$\{S_i \mid k \le i \le n\}$ 是 $S$ 的一个划分。因为 $S$ 中的每个元素可以由支持委员会的总人数(包括委员会成员和支持者)唯一确定,这个人数是 $i$,其中 $k \le i \le n$。

        现在,对于每个 $i$,我们可以通过多步骤过程计算 $|S_i|$:
        \begin{itemize}
            \item 从所有 $n$ 个人中选择 $i$ 个人作为潜在的委员会候选人:有 ${n \choose i}$ 种方法。
            \item 指定其余 $n - i$ 个人明确反对委员会。\\
                  (这一步是确定的,只有一种方法,但为了完整描述 $S$ 中的元素,我们需要明确指出。)
            \item 从选出的 $i$ 个人中选出 $k$ 人作为委员会成员:有 ${i \choose k}$ 种方法。
            \item 将剩余的 $i - k$ 个人指定为委员会的支持者(非委员会成员)。\\
                  (再次强调,这一步只有一种方法,但需要明确指出这些成员的态度和状态,以便完整描述结果。)
        \end{itemize}

        根据乘法原理,可得 $|S_i| = {n \choose i}{i \choose k}$。

        根据加法原理,可得 $|S| = \sum_{i=k}^{n} |S_i| = \sum_{i=k}^{n}{n \choose i}{i \choose k}$。

        由于我们通过两种方法计算了 $|S|$,因此它们相等,这就证明了恒等式。
    \end{proof}
\end{example}

请注意,在证明中,明确划分之后,我们完成了以下几件事:首先,解释了为什么这是一个划分;其次,说明了划分如何与求和索引关联;然后,解释了求和上下限如何覆盖所有可能的情况;最后,对于每个 $i$,解释了为什么 $|S_i|$ 对应求和中的项。

% !TeX root = ../../../book.tex

\subsection{二项式定理}\label{sec:section8.4.4}

我们可以使用``双法计数''证明技术来证明一个非常重要且强大的定理。这不仅展示了该技术的一个有趣应用,而且正如我们即将看到的,这个定理本身也非常有用!

\begin{theorem}\label{theorem8.4.8}
    设 $x, y \in \mathbb{R}$ 且 $n \in \mathbb{N}$,则
    \[(x+y)^n = \sum_{k=0}^{n} {n \choose k}x^ky^{n-k}\]
\end{theorem}
我们将使用几种不同的方法来证明这个定理。

\begin{proofs}{证明 1.}
    假设 $x, y \in \mathbb{N}$,我们来证明定理成立。

    首先,考虑一个包含 $x$ 个小写字母和 $y$ 个大写字母的集合,那么 $(x + y)^n$ 表示由这些字母组成的长度为 $n$ 的字符串的数量。

    在等式右边,我们根据字符串中被小写字母占据的位置数量来划分所有长度为 $n$ 的字符串。被小写字母占据的位置数量可能从 $0$ 到 $n$(含端点)。对于每个 $0 \le k \le n$,恰好包含 $k$ 个小写字母的长度为 $n$ 的字符串的数量为 ${n \choose k} \cdot x^k \cdot y^{n-k}$。这是因为我们首先选择 $k$ 个位置放置小写字母,然后用小写字母填充这些位置,最后用大写字母填充剩余的 $n-k$ 个位置。
\end{proofs}

\begin{proofs}{证明 2.}
    我们将证明推广到 $x, y \in \mathbb{R}$ 的情形。通过计算展开式中对应 $x$ 的 $k$ 次选择(也就是从乘积的因子中选择 $y$ 共 $n-k$ 次)的项数来证明。考虑乘积
    \[(x+y)^n = \underbrace{(x+y) \cdot (x+y) \dots (x+y)}_{n \;\text{项}}\]
    设想通过反复应用分配律展开这 $n$ 项。例如,当 $n=2$ 时:
    \begin{align*}
        (x+y)^2 & = (x+y)(x+y) = x(x+y)+y(x+y)                    \\
                & = x \cdot x + x \cdot y + x \cdot y + y \cdot y \\
                & = x^2+2xy+y^2
    \end{align*}
    当 $n=3$ 时:
    \begin{align*}
        (x+y)^3 & = (x+y)(x+y)(x+y) = x(x+y)(x+y) + y(x+y)(x+y) \\
                & = x(x^2+2xy+y^2) + y(x^2+2xy+y^2)             \\
                & = x^3 + 2x^2y + xy^2 + x^2y + 2xy^2 + y^3     \\
                & = x^3 + 3x^2y + 3xy^2 +y^3
    \end{align*}
    总体思路是这样的:为了得到展开式中的某一项,我们从每个因子 $(x + y)$ 中选择 $x$ 或 $y$。每个这样的项的形式为 $x^k \cdot y^{n-k}$,其中 $k$ 取值从 $0$ 到 $n$。我们只需确定有多少种方法可以构建形如 $x^k \cdot y^{n-k}$ 的项。这等价于从 $n$ 个因子中选取 $k$ 个并从中选择了``$x$'',同时从剩余 $n-k$ 个因子中选择了``$y$''。根据选择的定义,恰好有 ${n \choose k}$ 种方法!
\end{proofs}

\begin{proofs}{证明 3.}
    我们也可以通过归纳法来证明这一点!\textbf{帕斯卡恒等式 (Pascal's Identity)} 在归纳步骤中起到关键作用,具体见练习 \ref{exc:exercises8.9.14}。
\end{proofs}

\begin{example}
    让我们来看一下二项式定理的具体应用。

    \begin{itemize}
        \item 利用二项式定理证明
              \[2^n = \sum_{k=0}^{n}{n \choose k}\]
              \begin{proof}
                  令 $x=1, y=1$ 代入二项式定理即可。
              \end{proof}
              此法极为简洁!我们先前已经用归纳法和双法计数证明过该结论,现在可以直接应用二项式定理\emph{秒杀}。
        \item 证明 $[n]$ 的奇数大小子集的数量等于其偶数大小子集的数量,即:
              \[\sum_{k=0}^{\lceil n/2 \rceil}-1 {n \choose 2k+1} = \sum_{k=0}^{\lfloor n/2 \rfloor} {n \choose 2k}\]
              可以通过构造偶数大小子集与奇数大小子集之间的双射来证明。甚至可以尝试用计数的方法来解释这个问题。

              此外,我们还可以在等式两边同时减去偶数大小子集的数量,并将等式改写为
              \[\sum_{k=0}^{n}(-1)^k{n \choose k}=0\]
              注意,这正是二项式定理中令 $x = -1, y = 1$ 的情形。太神奇了!
    \end{itemize}
\end{example}

% !TeX root = ../../../book.tex

\subsection{习题}

\subsubsection*{温故知新}

以口头或书面的形式简要回答以下问题。这些问题全都基于你刚刚阅读的内容,所以如果忘记了具体的定义、概念或示例,可以回去重读相关部分。确保在继续学习之前能够自信地回答这些问题,这将有助于你的理解和记忆!

\begin{enumerate}[label=(\arabic*)]
    \item 什么是\textbf{双法计数}论证的基本方法?
    \item 为本节中的每个示例证明写一个简短的\emph{证明摘要}。
    \item 当给出的恒等式中存在一个求和时,我们在后续的双法计数论证中必须讨论什么?
    \item 我们用哪些不同的方法证明了求和恒等式?它们在本质上有何相同之处?
\end{enumerate}

\subsubsection*{小试牛刀}

尝试回答以下问题。这些题目要求你实际动笔写下答案,或(对朋友/同学)口头陈述答案。目的是帮助你练习使用新的概念、定义和符号。题目都比较简单,确保能够解决这些问题将对你大有帮助!

\begin{enumerate}[label=(\arabic*)]
    \item 设 $\ell,k,n \in \mathbb{N}$,用双法计数证明
        \[\begin{pmatrix}n\\k\end{pmatrix}\begin{pmatrix}k\\\ell\end{pmatrix}=\begin{pmatrix}n\\\ell\end{pmatrix}\begin{pmatrix}n-\ell\\k-\ell\end{pmatrix}\]
    \item 用双法计数证明
        \[n \cdot 2^{n-1} = \sum_{k=1}^{n}\begin{pmatrix}n\\k\end{pmatrix} \cdot k\]
    \item 用双法计数证明
        \[3^n=\sum_{k=1}^{n}\begin{pmatrix}n\\k\end{pmatrix}2^{n-k} = \sum_{k=0}^{n}\begin{pmatrix}n\\k\end{pmatrix}2^k\]
        (\textbf{提示}:考虑使用三进制字符串集。)\\
        然后,解释它是如何从二项式定理推导出来的。
    \item 用双法计数证明 $k^2=\big({k \atop 1}\big)+2\big({k \atop 2}\big)$。\\
        应用\textbf{求和恒等式}推导出
        \[\sum_{k=1}^{n} k^2 = \frac{n(n+1)(2n+1)}{6}\]
    \item 用双法计数证明以下\textbf{几何级数公式}:
    \[\forall q \in \mathbb{N}-\{1\} \centerdot \forall n \in \mathbb{N} \centerdot 1+q+q^2+q^3+\dots+q^{n-1} = \sum_{k=0}^{n-1}q^k = \frac{q^n-1}{q-1}\]
    (注意:实际上,对于任意\emph{实数} $q \ne 1$,这个公式都是成立的,但我们所讨论的双法计数证明仅适用于\emph{自然数} $q \ne 1$。要证明实数版本,请使用归纳法。)\\
    (\textbf{提示}:考虑由 $q$ 个元素组成的所有 $n$-元组的集合,但不包括某一个特定元素……)
\end{enumerate}
