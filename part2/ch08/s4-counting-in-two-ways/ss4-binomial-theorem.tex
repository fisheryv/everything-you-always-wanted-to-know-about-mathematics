% !TeX root = ../../../book.tex

\subsection{二项式定理}\label{sec:section8.4.4}

我们可以使用``双法计数''证明技术来证明一个非常重要且强大的定理。这不仅展示了该技术的一个有趣应用,而且正如我们即将看到的,这个定理本身也非常有用!

\begin{theorem}\label{theorem8.4.8}
    设 $x, y \in \mathbb{R}$ 且 $n \in \mathbb{N}$,则
    \[(x+y)^n = \sum_{k=0}^{n} {n \choose k}x^ky^{n-k}\]
\end{theorem}
我们将使用几种不同的方法来证明这个定理。

\begin{proofs}{证明 1.}
    假设 $x, y \in \mathbb{N}$,我们来证明定理成立。

    首先,考虑一个包含 $x$ 个小写字母和 $y$ 个大写字母的集合,那么 $(x + y)^n$ 表示由这些字母组成的长度为 $n$ 的字符串的数量。

    在等式右边,我们根据字符串中被小写字母占据的位置数量来划分所有长度为 $n$ 的字符串。被小写字母占据的位置数量可能从 $0$ 到 $n$(含端点)。对于每个 $0 \le k \le n$,恰好包含 $k$ 个小写字母的长度为 $n$ 的字符串的数量为 ${n \choose k} \cdot x^k \cdot y^{n-k}$。这是因为我们首先选择 $k$ 个位置放置小写字母,然后用小写字母填充这些位置,最后用大写字母填充剩余的 $n-k$ 个位置。
\end{proofs}

\begin{proofs}{证明 2.}
    我们将证明推广到 $x, y \in \mathbb{R}$ 的情形。通过计算展开式中对应 $x$ 的 $k$ 次选择(也就是从乘积的因子中选择 $y$ 共 $n-k$ 次)的项数来证明。考虑乘积
    \[(x+y)^n = \underbrace{(x+y) \cdot (x+y) \dots (x+y)}_{n \;\text{项}}\]
    设想通过反复应用分配律展开这 $n$ 项。例如,当 $n=2$ 时:
    \begin{align*}
        (x+y)^2 & = (x+y)(x+y) = x(x+y)+y(x+y)                    \\
                & = x \cdot x + x \cdot y + x \cdot y + y \cdot y \\
                & = x^2+2xy+y^2
    \end{align*}
    当 $n=3$ 时:
    \begin{align*}
        (x+y)^3 & = (x+y)(x+y)(x+y) = x(x+y)(x+y) + y(x+y)(x+y) \\
                & = x(x^2+2xy+y^2) + y(x^2+2xy+y^2)             \\
                & = x^3 + 2x^2y + xy^2 + x^2y + 2xy^2 + y^3     \\
                & = x^3 + 3x^2y + 3xy^2 +y^3
    \end{align*}
    总体思路是这样的:为了得到展开式中的某一项,我们从每个因子 $(x + y)$ 中选择 $x$ 或 $y$。每个这样的项的形式为 $x^k \cdot y^{n-k}$,其中 $k$ 取值从 $0$ 到 $n$。我们只需确定有多少种方法可以构建形如 $x^k \cdot y^{n-k}$ 的项。这等价于从 $n$ 个因子中选取 $k$ 个并从中选择了``$x$'',同时从剩余 $n-k$ 个因子中选择了``$y$''。根据选择的定义,恰好有 ${n \choose k}$ 种方法!
\end{proofs}

\begin{proofs}{证明 3.}
    我们也可以通过归纳法来证明这一点!\textbf{帕斯卡恒等式 (Pascal's Identity)} 在归纳步骤中起到关键作用,具体见练习 \ref{exc:exercises8.9.14}。
\end{proofs}

\begin{example}
    让我们来看一下二项式定理的具体应用。

    \begin{itemize}
        \item 利用二项式定理证明
              \[2^n = \sum_{k=0}^{n}{n \choose k}\]
              \begin{proof}
                  令 $x=1, y=1$ 代入二项式定理即可。
              \end{proof}
              此法极为简洁!我们先前已经用归纳法和双法计数证明过该结论,现在可以直接应用二项式定理\emph{秒杀}。
        \item 证明 $[n]$ 的奇数大小子集的数量等于其偶数大小子集的数量,即:
              \[\sum_{k=0}^{\lceil n/2 \rceil}-1 {n \choose 2k+1} = \sum_{k=0}^{\lfloor n/2 \rfloor} {n \choose 2k}\]
              可以通过构造偶数大小子集与奇数大小子集之间的双射来证明。甚至可以尝试用计数的方法来解释这个问题。

              此外,我们还可以在等式两边同时减去偶数大小子集的数量,并将等式改写为
              \[\sum_{k=0}^{n}(-1)^k{n \choose k}=0\]
              注意,这正是二项式定理中令 $x = -1, y = 1$ 的情形。太神奇了!
    \end{itemize}
\end{example}