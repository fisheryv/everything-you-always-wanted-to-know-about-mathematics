% !TeX root = ../../../book.tex

\subsection{二项式定理}\label{sec:section8.4.4}

我们可以用``双法计数''证明技术,证明一个非常重要且强大的定理。这不仅是该技术的一个有趣应用,正如我们将看到的,这个定理本身也非常有用!

\begin{theorem}
    设 $x, y \in \mathbb{R}$ 且 $n \in \mathbb{N}$,则
    \[(x+y)^n = \sum_{k=0}^{n} \begin{pmatrix}n\\k\end{pmatrix}x^ky^{n-k}\]
\end{theorem}
我们将用几种不同的方法来证明该定理。

\begin{proofs}{证明 1.}
    假设 $x, y \in \mathbb{N}$,我们来证明以下结论。

    首先,设想有一个包含 $x$ 个小写字母和 $y$ 个大写字母的集合,那么 $(x + y)^n$ 表示由这些字母组成的长度为 $n$ 的字符串的数量。

    在公式右边,我们根据字符串中有多少位置被小写字母填充来划分所有长度为 $n$ 的字符串的集合。可能有从 $0$(含)到 $n$(含)个位置被小写字母填充。对于每个 $0 \le k \le n$,恰好有 $k$ 个小写字母的长度为 $n$ 的字符串的数量为 $\big({n \atop k}\big) \cdot x^k \cdot y^{n-k}$,因为我们首先选择 $k$ 个位置来放置小写字母,然后用这些字母填充这 $k$ 个位置,最后用大写字母填充剩下的 $n - k$ 个位置。
\end{proofs}

\begin{proofs}{证明 2.}
    让我们将证明推广到 $x, y \in \mathbb{R}$。通过计算在 ``FOILed'' 展开中对应于 $x$ 的 $k$ 次选择(也就是从乘积的因子中选择 $y$ 的 $n - k$ 次)的项的数量,来证明这一点。

    考虑乘积
    \[(x+y)^n = \underbrace{(x+y) \cdot (x+y) \dots (x+y)}_{n \;\text{项}}\]
    想象一下通过反复应用分配律来展开这 $n$ 项。比如,如果 $n=2$,我们有
    \begin{align*}
        (x+y)^2 & = (x+y)(x+y) = x(x+y)+y(x+y)                    \\
                & = x \cdot x + x \cdot y + x \cdot y + y \cdot y \\
                & = x^2+2xy+y^2
    \end{align*}
    如果 $n=3$,我们有
    \begin{align*}
        (x+y)^3 & = (x+y)(x+y)(x+y) = x(x+y)(x+y) + y(x+y)(x+y) \\
                & = x(x^2+2xy+y^2) + y(x^2+2xy+y^2)             \\
                & = x^3 + 2x^2y + xy^2 + x^2y + 2xy^2 + y^3     \\
                & = x^3 + 3x^2y + 3xy^2 +y^3
    \end{align*}
    总体思路是这样的:为了在最终乘积中找到某一项,我们从每个因子 $(x + y)$ 中选择 $x$ 或 $y$。每个这样的项的形式为 $x^k \cdot y^{n-k}$,其中 $k$ 在 $0$ 到 $n$ 之间。我们只需要确定有多少种方法可以创建类似 $x^k \cdot y^{n-k}$ 的项。这相当于找到从 $n$ 个因子中选择 $k$ 个的方法,并假设我们从这些因子中选择了 ``$x$'',从剩下的 $n - k$ 个因子中选择了 ``$y$''。根据选择的定义,这样做的方法恰好有 $\big({n \atop k}\big)$ 种!
\end{proofs}

\begin{proofs}{证明 3.}
    我们也可以通过归纳法来证明这一点!\textbf{帕斯卡恒等式 (Pascal's Identity)} 在归纳法步骤中至关重要。这在练习 \ref{exc:exercises8.9.14} 中有所展示。
\end{proofs}

\begin{example}
    让我们来看看这个定理的实际应用。

    \begin{itemize}
        \item 应用二项式定理证明
              \[2^n = \sum_{k=0}^{n}\begin{pmatrix}n\\k\end{pmatrix}\]
              \begin{proof}
                  令 $x=1, y=1$ 代入二项式定理即可。
              \end{proof}
              就是这么简单!我们之前已经通过归纳法证明了这个结论,之后又用双法计数证明了一次,现在可以直接应用二项式定理\emph{秒杀}。
        \item 证明 $[n]$ 的奇数大小子集的个数等于其偶数大小子集的个数;也就是说
              \[\sum_{k=0}^{\lceil n/2 \rceil}-1 \begin{pmatrix}n\\2k+1\end{pmatrix} = \sum_{k=0}^{\lfloor n/2 \rfloor} \begin{pmatrix}n\\2k\end{pmatrix}\]
              可以通过找到偶数大小子集和奇数大小子集之间的双射来证明这一点。我们甚至可以尝试用计数的方法来解释这个问题。

              此外,我们还可以在等式两边同时减去偶数大小子集的数量,并将等式改写为
              \[\sum_{k=0}^{n}(-1)^k\begin{pmatrix}n\\k\end{pmatrix}=0\]
              注意,这正是二项式定理的内容,其中 $x = -1, y = 1$。太神奇了!
    \end{itemize}
\end{example}