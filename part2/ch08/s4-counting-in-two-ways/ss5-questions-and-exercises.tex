% !TeX root = ../../../book.tex

\subsection{习题}

\subsubsection*{温故知新}

以口头或书面的形式简要回答以下问题。这些问题全都基于你刚刚阅读的内容,所以如果忘记了具体的定义、概念或示例,可以回去重读相关部分。确保在继续学习之前能够自信地回答这些问题,这将有助于你的理解和记忆!

\begin{enumerate}[label=(\arabic*)]
    \item 什么是\textbf{双法计数}论证的基本方法?
    \item 为本节中的每个示例证明撰写一个简短的\emph{证明摘要}。
    \item 当恒等式中包含求和时,在双法计数论证中我们需要讨论什么?
    \item 我们使用了哪些不同的方法来证明求和恒等式?这些方法在本质上有何共同点?
\end{enumerate}

\subsubsection*{小试牛刀}

尝试回答以下问题。这些题目要求你实际动笔写下答案,或(对朋友/同学)口头陈述答案。目的是帮助你练习使用新的概念、定义和符号。题目都比较简单,确保能够解决这些问题将对你大有帮助!

\begin{enumerate}[label=(\arabic*)]
    \item 设 $\ell,k,n \in \mathbb{N}$,用双法计数证明
        \[{n \choose k}{k \choose \ell}={n \choose \ell}{n-\ell \choose k-\ell}\]
    \item 用双法计数证明
        \[n \cdot 2^{n-1} = \sum_{k=1}^{n}{n \choose k} \cdot k\]
    \item 用双法计数证明
        \[3^n=\sum_{k=1}^{n}{n \choose k}2^{n-k} = \sum_{k=0}^{n}{n \choose k}2^k\]
        (\textbf{提示}:考虑三进制字符串的集合。)\\
        并解释如何从二项式定理推导出该等式。
    \item 用双法计数证明 $k^2={k \choose 1}+2{k \choose 2}$。\\
        应用\textbf{求和恒等式}推导出
        \[\sum_{k=1}^{n} k^2 = \frac{n(n+1)(2n+1)}{6}\]
    \item 用双法计数证明以下\textbf{几何级数公式}:
    \[\forall q \in \mathbb{N}-\{1\} \centerdot \forall n \in \mathbb{N} \centerdot 1+q+q^2+q^3+\dots+q^{n-1} = \sum_{k=0}^{n-1}q^k = \frac{q^n-1}{q-1}\]
    (注意:实际上,对于任意\emph{实数} $q \ne 1$,这个公式都成立,但我们所讨论的双法计数证明仅适用于\emph{自然数} $q \ne 1$。要证明实数版本,请使用数学归纳法。)\\
    (\textbf{提示}:考虑由 $q$ 个元素组成的所有 $n$-元组的集合,但不包括某个特定元素……)
\end{enumerate}