% !TeX root = ../../../book.tex

\subsection{承上}

在第 \ref{ch:chapter07} 章中,我们讨论了集合的基数,包括有限集和无限集。虽然关于无限集的许多结果都非常有趣且富有数学意义,但这一领域可能会涉及一些令人费解的内容,超出了我们当前的研究范围。因此,本章我们将重点讨论有限集。特别是,我们将探讨如何利用有限集的基数来解决数学对象的``计数''问题。也就是说,我们将回答``具有某种属性的对象有多少?''这类问题。这个数学分支被称为\emph{组合数学},你可以把它看作是``研究对象组合''的学问。在研究这一领域时,我们将引入一些新的符号和定义,证明并应用一些关于有限集的结论,并描述和研究一些组合数学和计算机科学领域中的特定对象。重要的是,我们将学习一种全新的基于计数对象的证明技术。