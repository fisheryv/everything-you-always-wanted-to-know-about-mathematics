% !TeX root = ../../../book.tex

\subsection{启下}

想象一下你正在玩扑克。如果你不熟悉这个游戏,可以把它看作一个简单的系统:两个玩家各自随机抽得 $5$ 张牌,然后比较谁赢。手牌从最好到最差的排序如下:
\begin{itemize}
    \item 同花顺 (同花色的五张连续牌),例如:$T\clubsuit\; J\clubsuit\; Q\clubsuit\; K\clubsuit\; A\clubsuit$
    \item 四条 (同点数的四张牌),例如:$3\clubsuit\; 3\spadesuit\; 3\heartsuit\; 3\diamondsuit\; 7\heartsuit$
    \item 葫芦 (三张同点数的牌加一对),例如:$4\clubsuit\; 4\spadesuit\; 4\diamondsuit\; 6\clubsuit\; 6\heartsuit$
    \item 同花 (同花色的五张牌),例如:$2\heartsuit\; 5\heartsuit\; 8\heartsuit\; Q\heartsuit\; K\heartsuit$
    \item 顺子 (不同花色的五张连续牌),例如:$8\diamondsuit\; 9\spadesuit\; T\diamondsuit\; J\heartsuit\; Q\heartsuit$
    \item 三条 (同点数的三张牌),例如:$K\spadesuit\; K\heartsuit\; K\diamondsuit\; Q\heartsuit\; 9\clubsuit$
    \item 两对 (两对牌),例如:$A\spadesuit\; A\heartsuit\; J\spadesuit\; J\diamondsuit\; 2\clubsuit$
    \item 一对 (一对牌),例如:$8\heartsuit\; 8\diamondsuit\; 2\spadesuit\; 5\clubsuit\; K\heartsuit$
    \item 高牌 (无组合牌),例如:$Q\spadesuit\; J\clubsuit\; 9\diamondsuit\; 7\diamondsuit\; 2\diamondsuit$
\end{itemize}
这是一个公平的游戏吗?如果你玩过扑克,尤其是牌桌老手,你不仅会习惯这种排序系统,还会学会如何利用它来制定决策。在美国梭哈\footnote{美国梭哈 (Five Card Draw),又称``五张抽''。以五张牌的排列组合比大小,每名玩家按顺时针方向各发五张牌,牌面向下,拿到牌后可以决定是否换牌且换多少张(最多五张),也可选择不换或弃牌,要换牌的玩家换完牌后和没换牌的玩家(不包括弃牌的玩家)在之后就翻开各自的的五张牌来比较,根据牌型大小决定胜负。--- 译者注}中,如果你拿到 $2\; 2\; 3\; 4\; 5$,你会选择保留对子还是追求顺子?哪种情况更有可能发生?哪种选择会带来更高的回报?

我们的问题是,``这是一个公平的游戏吗?''我们真正想知道的是,为什么牌型的排序是这样的!同花的概率真的比顺子更低吗?葫芦输给四条合理吗?为什么?我们该如何\emph{证明}这些结论呢?为了回答这些问题,我们将用\emph{计数}的方法来重新表述这些问题,而不是用概率。我们会询问有多少种不同的五张牌型是同花,有多少种是顺子,等等。这样我们就可以直接比较它们。你是否看出这与我们前一章的工作有什么联系?我们实际上将确定所有同花牌集合的\emph{基数},并将其与其他牌型集合的基数进行比较。