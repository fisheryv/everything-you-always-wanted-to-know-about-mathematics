% !TeX root = ../../../book.tex

\subsection{启下}

想象一下你正在玩扑克。如果你不熟悉这个游戏,可以把它想象成一个简单的系统:两名玩家各自随机抽取 $5$ 张牌,然后比较谁的牌型更优。手牌从最好到最差的排序如下:
\begin{center}
    \begin{tabular}{lllllll}
        同花顺 (同花色的五张连续牌) & 例如:& \tenc & \Jc & \Qc & \Kc & \Ac \\
        四条 (同点数的四张牌) & 例如:& \trec & \tres & \treh & \tred & \sevh \\
        葫芦 (三张同点数的牌加一对) & 例如:& \fourc & \fours & \fourd & \sixc & \sixh \\
        同花 (同花色的五张牌) & 例如:& \twoh & \fiveh & \eigh & \Qh & \Kh \\
        顺子 (不同花色的五张连续牌) & 例如:& \eigd & \nines & \tend & \Jh & \Qh \\
        三条 (同点数的三张牌) & 例如:& \Ks & \Kh & \Kd & \Qh & \ninec \\
        两对 (两对牌) & 例如:& \As & \Ah & \Js & \Jd & \twoc \\
        一对 (一对牌) & 例如:& \eigh & \eigd & \twos & \fivec & \Kh \\
        高牌 (无组合牌) & 例如:& \Qs & \Jc & \nined & \sevd & \twod
    \end{tabular}
\end{center}

这是一个公平的游戏吗?如果你玩过扑克,尤其是经验丰富的玩家,你不仅会熟悉这种排序系统,还会学会如何利用它来制定策略。在美国梭哈\footnote{美国梭哈 (Five Card Draw),以五张牌的牌型组合比大小,每名玩家按顺时针方向各发五张牌,牌面向下。拿到牌后,玩家可以决定是否换牌以及换多少张(最多五张),也可以选择不换或弃牌。要换牌的玩家换完牌后,与没换牌的玩家(不包括弃牌的玩家)一起翻开各自的五张牌进行比较,根据牌型大小决定胜负。—— 译者注}中,如果你拿到 $2\ 2\ 3\ 4\ 5$,你会选择保留对子还是尝试组成顺子?哪种情况更可能发生?哪种选择会带来更高的回报?

我们的问题是:``这是一个公平的游戏吗?''但我们真正想知道的是,为什么牌型的排序是这样的!同花的概率真的比顺子更低吗?葫芦输给四条合理吗?为什么?我们该如何\emph{证明}这些结论呢?为了回答这些问题,我们将用\emph{计数}的方法来重新表述这些问题,而不是用概率。我们会计算有多少种不同的五张牌组合是同花,有多少种是顺子,等等。这样我们就可以直接比较它们。你是否看出这与我们前一章的工作有什么联系?我们实际上将确定所有同花牌集合的\emph{基数},并将其与其他牌型集合的基数进行比较。
