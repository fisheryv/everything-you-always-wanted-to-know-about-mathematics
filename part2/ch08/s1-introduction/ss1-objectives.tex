% !TeX root = ../../../book.tex

\subsection{目标}

以下简短内容将向你展示本章如何融入本书的体系。我们会解释之前的工作对本章研究的帮助,说明我们为什么要探讨本章的主题,并告诉你我们的目标以及在阅读时需要注意的事项。现在,我们将通过几条陈述总结本章的主要目标。这些陈述概括了你在完成本章后应掌握的技能和知识。接下来的章节会更详细地解释这些思想,这里仅提供一个简要列表供你参考。完成本章后,请返回这个列表,检查你是否理解所有目标。你能看出我们为什么认为这些目标重要吗?你能解释我们使用的术语并应用我们描述的技术吗?

\textbf{学完本章后,你应该能够……}

\begin{itemize}
    \item 阐述加法原理和乘法原理,并结合它们构建简单的计数论证。
    \item 对几种标准计数对象进行分类,阐述相应的计数公式,并理解如何证明这些公式。
    \item 阐述二项式系数的含义,评估其数值公式,了解如何在计数论证中使用它们,并理解数值公式的推导过程。
    \item 通过正确展示是否低估或高估来评判某个计数论证。
    \item 通过构建``计数双法''证明来证明组合恒等式。
    \item 理解重复选择的各种形式,并使用它们来解决问题。
    \item 阐述鸽笼原理,并在计数论证中使用它。
    \item 阐述包含/排除原理,并在计数论证中使用它。
\end{itemize}