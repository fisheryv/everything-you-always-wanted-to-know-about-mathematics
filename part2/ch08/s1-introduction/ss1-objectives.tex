% !TeX root = ../../../book.tex

\subsection{目标}

以下内容简要说明本章在本书中的定位。我们将解释前期工作如何为本章研究奠定基础,阐明探讨本章主题的动机,并概述学习目标及注意事项。我们会先总结本章的主要目标,概括你在学完本章后应掌握的技能与知识。后续章节将详细展开这些思想,此处仅提供一个简要列表作为学习指引。完成本章后,请你返回此列表,确认自己是否达成了所有目标。你是否能理解这些目标的重要性?能否清晰地解释相关术语并熟练地应用相关技术?

\textbf{学完本章后,你应该能够……}

\begin{itemize}
    \item 阐述加法原理与乘法原理,并运用它们构建简单的计数论证。
    \item 区分几种标准计数对象,说明相应的计数公式,并理解这些公式的证明方法。
    \item 解释二项式系数的含义,计算其数值公式,掌握在计数论证中的应用,并理解公式的推导过程。
    \item 通过准确判断是否低估或高估来评估计数论证的有效性。
    \item 通过构建``双法计数''来证明组合恒等式。
    \item 理解重复选择的不同形式,并运用它们解决实际问题。
    \item 阐述抽屉原理,并在计数论证中应用它。
    \item 阐述包含/排除原理,并在计数论证中应用它。
\end{itemize}