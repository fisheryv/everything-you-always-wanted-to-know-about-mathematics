% !TeX root = ../../../book.tex

\subsection{忠告}

我们需要先引入一些符号和定义,来制定计算特定有限集中元素数量的方法。\emph{组合数学}实际上就是关于如何使用特定方法(我们将在本章中介绍)来计算有限集中的元素数量。我们希望以抽象的方式研究这些计数技术,以便能够\emph{高效}应用。例如,回答之前提出的扑克问题时,我们可以查看所有可能的五张牌组合,并在每次看到同花时做个记号,但这显然太费时了!肯定有更好的方法!当然,我们将在本章的第一节中介绍这种方法。

我们还要强调,本章将引入一种新的证明风格。与之前研究的问题和技术相比,组合数学中的证明非常依赖语言的清晰和具体。在本章的章节练习中,有些证明可能完全由自然语句组成,几乎不使用任何数学符号!刚开始时这可能看起来很奇怪,甚至与我们之前强调的精确性、清晰性和数学严谨性的理念相矛盾。但实际上并非如此;组合数学有严格的有限集理论作为基础,我们会在需要时指出这种关系。组合数学的这种性质要求你在撰写证明时格外小心,确保用词准确清晰。写完证明后,一定要重新阅读一遍,把自己当作他人,确保你想表达的观点得到清晰地传达。

最后,我的一位朋友在我们讨论如何教授组合数学时曾说过一句话,我觉得很好地总结了从正式证明到组合数学证明(相对非正式)的这种转变。
\begin{quotation}
    有限基数的研究相对枯燥,但这并不意味着组合数学就容易。
\end{quotation}
你现在可能还不明白这段话是什么意思,但读完本章后再回头看这段话,你就会明白其中的含意。从抽象和理论的角度来看,有限基数相对简单,结果都是预期中的,比如当 $A \cap B = \varnothing$ 时,$|A \cup B| = |A|+|B|$,而且方法也很单一,就是找到一个适当集合的双射即可。但\emph{无限基数}就复杂得多,结果可能出乎意料。比如,即便 $A \cap B = \varnothing$,$|A \cup B| = |A|+|B|$ 也可能是错误的,甚至连 $|A|+|B|$ 的加法在数学上都难以理解。

那么,组合数学有什么特别之处呢?在所有的组合数学研究中,我们总是处理有限集;不同的是,这些元素只是通过某种方式描述给我们,而不是直接列出让我们去数(那样会很简单:``一、二、三、……'')。我们需要想出有效的策略来确定具有某个特性的对象数量。这就是组合数学的难点所在。例如,当我们说``考虑从一副标准扑克牌中抽取 $5$ 张牌组成的所有组合'',你可能很快理解这是一个集合的概念,但肯定无法想象出所有可能的组合,更不用说一个个地数出它们。从这个角度看,组合数学是复杂的;这也正是它如此有趣和受欢迎的原因!