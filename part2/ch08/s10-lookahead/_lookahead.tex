% !TeX root = ../../../book.tex
\section{展望}

我们的旅程已接近终点!至少,这本书的全部内容已经完整呈现在你面前。希望它能点燃你对数学知识和解决问题的热情。回想我们最初的起点:只是从一些有趣的谜题出发,尝试用已有的知识和逻辑技巧去解答它们。其实,我们现在所做的并无不同——只不过我们已经掌握了更多的数学语言、定理推论和解题方法,能够应对更具挑战性的问题。翻开这本书的那一刻,你是否预料到自己能够解开这些难题?是否对数学家的工作方式以及他们观察世界的视角有了更深的理解?但愿如此!$\smiley{}$

除了数学本身,我们还学会了许多其他有用的技能。虽然在日常生活中你未必会用到形式化的\emph{符号}逻辑,但你一定会面对复杂的陈述,其中包含合取、析取与条件判断。作为每天都在交流和传达复杂思想的人类,我们其实早已习惯处理这些内容。通过学习形式逻辑的基础知识,我们更加擅长分析复杂陈述并判断其真伪,也能更加清晰地表达和分享自己的想法。

同样,你未必会在生活中遇到组合恒等式的正式表述,但双法计数的训练无疑提升了你的分析思维能力。有时候,我们需要构建一个富有创意的``故事'',从两个角度去计数同一组对象。这既需要创造力,也需要智慧,而锻炼这类思维对我们大有裨益。此外,在阅读和分析证明时,判断其中是否存在重复或遗漏的过程,也让我们更善于理解和评价他人的论证。尽管日常生活中不会使用数学术语,我们其实每天都在进行类似的思考与分析。

总而言之,我们已经初步建立起数学思维能力。我们学会了如何阅读和理解问题;如何从多个角度解决问题,并勇于探索可能的错误路径以加深理解;如何识别不同问题背后的共同结构,并利用这些相似性选取合适的技巧;最终,我们学会了如何将理解与思路整理成书面形式,形成可供他人阅读的论证。这个过程不仅让我们成为更好的问题解决者,也让我们成为更加有效的沟通者。在这个瞬息万变的世界里,沟通正变得日益重要(因为与他人建立联系越来越容易),而高效、准确、清晰地表达自己的想法,已经成为一项不可或缺的生活技能。

但请不要让这成为数学之旅的终点!继续前进吧,将你的数学知识与乐趣传递出去。和他人一起探讨书中的问题,一同解决更多新问题。寻找那些真正能激发你热情的数学领域。尝试用这些概念去解决现实中的难题。最重要的是,走出去,\textbf{做数学吧}!
