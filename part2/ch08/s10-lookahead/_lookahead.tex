% !TeX root = ../../../book.tex
\section{展望}

我们已经接近尾声了!至少,这本书的内容已经全部呈现完毕。我们希望这本书能激发你对数学知识和解决问题的兴趣。回想一下我们是如何开始的:我们只是提出一些有趣的谜题,并尝试用现有的知识和逻辑技巧来解决它们。实际上,我们现在做的事情也是一样的!只是我们已经掌握了更多的数学术语、定理推论和解决问题的技能,因此能够处理更高级的问题。当你开始阅读这本书时,你是否想到自己会解决这些难题?你是否对数学家的工作及其看待世界的方式有了更深刻的理解和认识?希望如此!$\smiley{}$

我们还在数学之外学到了许多有用的技能。虽然你在日常生活中可能不会遇到形式\emph{符号}逻辑,但你一定会处理复杂的陈述,包括合取、析取和条件陈述。作为一个每天交流并传达复杂思维的人类,我们已经习惯如此。通过研究形式逻辑的一些基础知识,我们现在更擅长分析复杂陈述并判断其真伪,同时也能更好地表达和分享自己的想法。

同样,你可能不会在日常生活中遇到组合恒等式的正式陈述,但通过双法计数的练习可以提升你的分析思维能力。有时我们需要创造性地构建一个``故事''来描述需要以两种方式计数的元素。这需要一些创造力和智慧,锻炼这些思维能力对我们大有裨益。此外,阅读和分析给出的``证明''是否多算或少算的过程,使我们更擅长理解和评价他人的论证。虽然这些过程在日常生活中未必用到数学术语,但我们每天都在进行类似的思考和分析。

总体来说,我们已经培养起了数学思维能力。我们学会了如何阅读和理解问题;如何从多个角度解决问题,并愿意探索潜在的死胡同以加深理解;如何识别不同问题中的共同结构,并利用这些相似性采用特定技巧;最终,如何将我们对问题的理解和想法整理成书面的、可供他人阅读的论证。这个过程不仅让我们成为更好的问题解决者,也让我们成为更好的沟通者。在这个快速变化的世界里,沟通变得越来越重要(因为与他人快速联系变得越来越容易),有效、正确和清晰地分享我们的想法是一项重要的生活技能。

但不要让这成为我们数学旅程的终点!继续前行,传播你的数学知识和乐趣。与他人一起解决本书中的问题和更多问题。寻找那些激励你的数学领域。看看你是否能用这些概念解决现实生活中的问题。最重要的是,走出去,\textbf{做数学吧}!

