% !TeX root = ../../../book.tex

\subsection{陈述与证明}

\begin{theorem}{容斥原理}\label{theorem8.7.1}
    设 $U$ 为全集,$A_1, A_2, \dots, A_n$ 为 $U$ 的子集。则
    \[|U - (A_1 \cup A_2 \cup \dots \cup A_n)| = \sum_{S \subseteq [n]} (-1)^{|S|} \Bigg|\bigcap_{i \in S} A_i\Bigg| \enspace \text{其中} \enspace \bigcap_{i \in \varnothing} A_i = U\]
\end{theorem}
(请尝试写出 $n = 1$、$n = 2$ 和 $n = 3$ 时的表达式,你会发现它们与上一小节中的表达式是一致的。)

为了证明这个定理,我们将采用双法计数论证。具体来说,考虑任意元素 $x \in U$,并论证它在等式两边被计数的次数是相同的。

\begin{proof}
    设 $x \in U$ 为任意固定元素。我们将分两种情况讨论。

    第一种情况,假设对于每个 $i \in [n]$,都有 $x \notin A_i$。此时,$x$ 在左边被恰好计数一次,因为它不属于任何 $A_i$ 的并集。在右边,仅当 $S = \varnothing$ 时 $x$ 被计数,因为 $x$ 不属于任何 $A_i$,故也不属于任何这些集合的交集。因此,$x$ 在右边也被恰好计数一次。

    第二种情况,假设 $x$ 属于至少一个 $A_i$。此时,$x$ 在左边不被计数。为了说明右边也不计数 $x$,定义 $B \subseteq [n]$ 为满足 $x \in A_i$ 的所有索引 $i$ 的集合,即 $\forall i \in B, x \in A_i$。接下来进行如下观察:

    考虑那些在求和中不计入 $x$ 的项。对于任意 $S \subseteq [n]$,若 $S \nsubseteq B$,则 $x \notin \bigcap_{i \in S} A_i$(因为存在 $j \in S$ 使得 $j \notin B$,而 $B$ 包含了所有满足 $x \in A_i$ 的索引)。这意味着对满足 $S \nsubseteq B$ 的项,$x$ 被计数了 $0$ 次。
    
    接下来,根据组合数学的基本结论,$B$ 的大小为奇数和偶数的子集数目相等。对于任意 $T \subseteq B$,有 $x \in \bigcap_{i \in T} A_i$。若 $|T|$ 为偶数,则该项为正,$x$ 被计数一次;若 $|T|$ 为奇数,则该项为负,$x$ 被减去一次。由于奇偶子集数目相等,$x$ 在这些项中被净计数 $0$ 次。

    综上所述,在任何情况下,任意元素 $x$ 在等式两边被计数的次数相同(且正确)。
\end{proof}

有时,多个 $A_i$ 集合的交集大小均\emph{相同}。下一节将给出这类例子。此时,原表达式中的许多项可以合并,因为它们取值相同。具体来说,与其通过对子集 $S \subseteq [n]$ 求和来计数 $A_i$ 所有可能的交集,不如对相交集合的数量求和,不必考虑具体哪些集合相交。

\begin{corollary}
    设 $U$ 为全集,$A_1, A_2, \dots, A_n$ 为 $U$ 的子集。若任意 $k$ 个 $A_i$ 集合的交集大小均为固定值 $S(k)$(与具体选择的集合无关),则
    \[|U - (A_1 \cup A_2 \cup \dots \cup A_n)| = \sum_{k=0}^{n}(-1)^k {n \choose k}S(k)\]
\end{corollary}

\begin{proof}
    此推论可由定理 \ref{theorem8.7.1} 通过合并相同项得到。具体来说,存在 ${n \choose k}$ 个子集 $S \subseteq [n]$ 满足 $|S| = k$,根据推论假设,对于所有这样的 $S$,有
    \[\left|\bigcap_{i \in S} A_i\right| = S(k)\]
    将这些项合并,并对 $S$ 的所有可能大小求和,即可得到上述结论。
\end{proof}

此推论在接下来的示例中非常有用!