% !TeX root = ../../../book.tex

\subsection{陈述与证明}

\begin{theorem}{容斥原理}\label{theorem8.7.1}
    假设我们有一个全集 $U$ 和若干子集 $A_1, A_2, \dots, A_n \subseteq U$。则
    \[|U - (A_1 \cup A_2 \cup \dots \cup A_n)| = \sum_{S \subseteq [n]} (-1)^{|S|} \Bigg|\bigcap_{i \in S} A_i\Bigg| \enspace \text{其中}\; \bigcap_{i \in [n]} A_i = U\]
\end{theorem}
(试着写出 $n = 1$、$n = 2$ 和 $n = 3$ 时的表达式,此时你会发现它们与我们在上一小节中写出的表达式是相同的。)

为了证明这个定理,我们将使用双法计数论证法。具体来说,我们会考虑一个元素 $x \in U$,并论证它在上述方程的两边都被正确地计数。

\begin{proof}
    设 $x \in U$ 为任意固定元素。我们将考虑两种情况。

    首先,假设对于每个 $i \in [n], x \notin A_i$。那么左边正好计数了 $x$ 一次,因为 $x$ 不是 $A_i$ 集合并集的元素。右边只有在 $S = \varnothing$ 的时候才计数 $x$ ,因为 $x$ 不是任何 $A_i$ 集合的元素,所以它也不是这些集合任意交集的元素。因此,$x$ 在右边也正好被计数了一次。

    其次,假设 $x$ 属于一个或多个 $A_i$ 集合。这意味着 $x$ 不会在左边被计数。为了证明 $x$ 在右边也不被计数,我们定义 $B \subseteq [n]$ 为所有使得 $x \in A_i$ 的索引集,即 $\forall i \in B, x \in A_i$。这里我们需要做一些观察:
    
    考虑求和中不会计数 $x$ 的那些项。对于任意 $S \subseteq \mathbb{N}$,如果 $S \nsubseteq B$ ,那么 $x \notin \bigcap_{i \in S} A_i$。(这是因为存在某个 $j \in S$ 使得 $j \notin B$,而 $B$ 包含了所有使得 $x \in A_i$ 的索引。) 这意味着 $x$ 在求和中对于 $S \nsubseteq B$ 的项被计数了 $0$ 次。

    接下来,根据先前证明的结果,我们知道 $B$ 的子集的奇数大小和偶数大小是相等的。对于任意这样的子集 $T \subseteq B$,我们知道 $x \in T \bigcap_{i \in T} A_i$。现在,如果 $|T|$ 为偶数,则该项为正,所以 $x$ 被计数了一次;如果 $|T|$ 为奇数,则该项为负,所以 $x$ 从计数中被移除一次。由于奇偶项的数量是相等的,我们得到 $x$ 在这些项中被计数了 $0$ 次。

    综上,我们已经证明了在任何情况下,一个任意元素在方程两边被计数的次数是相同的(并且是正确的)。
\end{proof}

有时候,多个 $A_i$ 集合的所有交集具有相同的大小。在下一节中,我们会看到一些这样的例子。在这种情况下,上述表达式中的许多项可以合并,因为它们是相同的。具体来说,与其对 $S \subseteq [n]$ 的子集求和来考虑 $A_i$ 集合的所有可能交集,不如对相交的集合数量求和,而不是具体哪些集合相交。

\newpage

\begin{corollary}
    假设 $U$ 为全集,$A_1, A_2, \dots , A_n$ 为 $U$ 的子集。并进一步假设任意 $k$ 个 $A_i$ 集合的交集的大小都是固定值,我们称之为 $S(k)$,与具体交集的集合无关。则
    \[|U - (A_1 \cup A_2 \cup \dots \cup A_n)| = \sum_{k=0}^{n}(-1)^k \begin{pmatrix}n\\k\end{pmatrix}S(k)\]
\end{corollary}

\begin{proof}
    此推论可以通过合并相同项,并结合定理 \ref{theorem8.7.1} 得出。具体来说,我们知道存在 $\big({n \atop k}\big)$ 个集合 $S \subseteq [n]$ 满足 $|S| = k$。在此推论的假设下,所有满足 $|S| = k$ 的集合将得到
    \[\bigg|\bigcap_{i \in S} A_i\bigg|=S(k)\]
    将这些项合并,并对 $S$ 的所有可能大小求和,我们就得到了上述结论。
\end{proof}