% !TeX root = ../../../book.tex

\subsection{引言}

容斥原理是一种有效的方法,能够帮助我们从一个大集合中\emph{移除}子集并计算剩余的元素数量。我们已经在简单的情况下见识过它的应用。如果我们有 $A \subseteq U$,并且我们想知道 $|U - A|$ 的值,只需计算 $|U|$ 和 $|A|$ 的大小,然后相减即可。这是加法原理在集合 $U$ 的划分 $\{A, U - A\}$ 上的应用。

那么,如果我们从一个更大的集合中移除两个集合会怎样?如果它们有重叠部分呢?我们需要考虑这些重叠吗?如果我们要移除三个集合呢?或者四个集合呢?甚至 $n$ 个集合呢?我们能否写出一个通用的表达式来表示剩余元素的数量?这种表达式能否用于解决计数问题?下面是一些描述``小规模情况''的表达式。

假设我们有一个全集 $U$ 和若干子集 $A_1, A_2, \dots, A_n \subseteq U$。我们希望计算 $U$ 中不属于任何 $A_i$ 集合的元素数量。我们可以这样表示:
\begin{align*}
    |U - A_1| =                     & |U| - |A_1|                                                                  \\
    |U - (A_1 \cup A_2)| =          & |U| - |A_1| - |A_2| + |A_1 \cap A_2|                                         \\
    |U - (A_1 \cup A_2 \cup A_3)| = & |U| - |A_1| - |A_2| - |A_3|                                                  \\
                                    & + |A_1 \cap A_2| + |A_1 \cap A_3| + |A_2 \cap A_3| - |A_1 \cap A_2 \cap A_3|
\end{align*}
你能理解这些公式为什么成立吗?试着考虑一个元素 $x \in U$,并思考它属于多少个 $A_i$ 集合。这个元素在左边和右边的表达式中会被计算多少次?它在两边是否被正确计算了?你能想到如何推广这一思想吗?

将这些表达式看作是对正确计数的一种``猜测'',然后不断``修正''以调整多算或少算的计数,可能会有所帮助。例如,我们可以这样推导出上面最后一个表达式:
\begin{quote}
    我们来计算 $|U - (A_1 \cup A_2 \cup A_3)|$ 的值。首先,取出 $U$ 中的元素总数,然后减去集合 $A_1$、$A_2$ 和 $A_3$ 中的元素数量。

    等等!属于\emph{两个}集合的元素怎么办?我们已经多次重复地从计数中删除了这些元素,所以需要把属于两个集合交集的元素数量加回来。

    再等等!那些同时属于\emph{三个}集合的元素怎么办?我们已经多次重复地加回了这些元素,因此需要再次把它们减去。
\end{quote}
你现在可能已经知道如何泛化这些表达式了,并证明它们适用于任意数量的集合。在下一节中,我们将详细探讨这一过程。