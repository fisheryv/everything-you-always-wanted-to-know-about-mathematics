% !TeX root = ../../../book.tex

\subsection{引言}

容斥原理是一种有效的方法,用于从一个大集合中\emph{移除}若干子集并计算剩余元素的数量。我们已经在简单的情况下见过它的应用。例如,若 $A \subseteq U$,且我们希望计算 $|U - A|$,只需将 $|U|$ 与 $|A|$ 相减即可。这实际上是加法原理在集合 $U$ 的划分 $\{A, U - A\}$ 上的应用。

那么,当我们从一个大集合中移除两个集合时,情况会如何?如果这两个集合存在重叠,我们需要考虑重叠部分吗?进一步地,如果移除三个、四个,甚至 $n$ 个集合呢?我们能否写出一个通用表达式来表示剩余元素的数量?这种表达式能否用于解决计数问题?下面是一些描述``小规模情况''的表达式。

假设我们有一个全集 $U$ 和若干子集 $A_1, A_2, \dots, A_n \subseteq U$。我们希望计算 $U$ 中不属于任何 $A_i$ 的元素数量。具体表达式如下:
\begin{align*}
    |U - A_1| =                     |U| & - |A_1|                                                                  \\
    |U - (A_1 \cup A_2)| =          |U| & - |A_1| - |A_2| + |A_1 \cap A_2|                                         \\
    |U - (A_1 \cup A_2 \cup A_3)| = |U| & - |A_1| - |A_2| - |A_3|                                                  \\
                                    & + |A_1 \cap A_2| + |A_1 \cap A_3| + |A_2 \cap A_3| - |A_1 \cap A_2 \cap A_3|
\end{align*}

你能理解这些公式为什么成立吗?试着考虑一个元素 $x \in U$,并分析它属于哪些 $A_i$ 集合。这个元素在左边和右边的表达式中分别被计算了多少次?它在两边是否被正确计数?你能想到如何推广这一思想吗?

将这些表达式视为对正确计数的一种``猜测'',然后通过不断``修正''来调整多算或少算的部分,可能会有所帮助。例如,我们可以这样推导上面的最后一个表达式:
\begin{quote}
    我们来计算 $|U - (A_1 \cup A_2 \cup A_3)|$ 的值。首先,从 $U$ 的元素总数中减去 $A_1$、$A_2$ 和 $A_3$ 的元素数量。

    但是,等等!那些属于\emph{两个}集合的元素呢?由于这些元素被重复减去了多次,我们需要将属于两个集合交集的元素数量加回来。

    再等等!那些同时属于\emph{三个}集合的元素呢?由于这些元素在加回时被重复计算了,我们需要再次减去属于三个集合交集的元素数量。
\end{quote}
你现在可能已经知道如何推广这些表达式了,并证明它们适用于任意数量的集合。在下一节中,我们将详细探讨这一过程。